\documentclass{amsbook}

\title{Hosker's Almanack (First Proof)}

\usepackage[
    a5paper,
    hmargin=15mm,
    vmargin=15mm,
    marginparwidth=15mm,
    marginparsep=3mm
]{geometry}

\usepackage{
    footmisc,
    %bigfoot,
    caption,
    color,
    fourier-orns,
    marginnote,
    microtype,
    pifont,
    verse,
    xfrac
}

% Sort out fonts.
\usepackage{fontspec}
\usepackage[T1]{fontenc}
\newfontfamily\hoskeroe{English Towne}
\newfontfamily\greekish{CMU Serif}
\newfontfamily\arabicish{KacstQurn}
\setmainfont[
    ItalicFont=cmunti.otf,
    BoldFont=cmunbx.otf,
    BoldItalicFont=cmunbi.otf,
    Numbers=OldStyle
]{cmunrm.otf}

% Sort out BibTex.
\usepackage[backend=bibtex, style=authortitle, citestyle=authortitle]{biblatex}
\addbibresource{sources.bib}

% For Arabic text inside left-to-right text.
\newcommand{\textarabic}[1]{\bgroup\textdir TRT\arabicish #1\egroup}

% Fix fractions.
\usepackage{xfrac}
\DeclareInstance{xfrac}{cmunrm.otf(0)}{text}{
    slash-symbol-font = ptm,
    scale-factor=0.8,
    numerator-top-sep = 0pt,
    denominator-bot-sep = 0pt,
    slash-right-kern=-.25em,
    slash-left-kern=-.3em
}

% Page break stuff.
\makeatletter
\@beginparpenalty=10000
\makeatother

% Standardise the manner in which books, poems, etc are referenced.
\newcommand{\refbook}[1]{\textit{#1}}
\newcommand{\refpoem}[1]{``#1''}

% Always call the hyperref package last.
\usepackage[hidelinks]{hyperref}


\setcounter{secnumdepth}{4}
\setcounter{tocdepth}{0}

\begin{document}

\frontmatter

\maketitle

\tableofcontents

%FRONTMATTER  % Replace % with # to insert introductory prose, and vice versa.

\mainmatter

\renewcommand\thesection{{\Roman{section}}}
\renewcommand\thefootnote{{\thesubsection.}}
\makeatletter
    \def\blfootnote{\xdef\@thefnmark{}\@footnotetext}
    \renewcommand{\@makefnmark}{\hbox{{{{\@thefnmark}}}}\hbox{{{{ }}}}}
\makeatother

\part{The \textit{Almanack} Proper}

\chapter{Primilis}

\section{}

\subsection{}

\blfootnote{`Ode on a Grecian Urn', John Keats (1795 -- 1821), \cite{norton}.}\settowidth{\versewidth}{        What pipes and timbrels? What wild ecstasy?}
\begin{verse}[\versewidth]
Thou still unravished bride of quietness,\\*
\vin Thou foster-child of silence and slow time,\\
Sylvan historian, who canst thus express\\
\vin A flowery tale more sweetly than our rhyme:\\
What leaf-fringed legend haunts about thy shape\\
\vin Of deities or mortals, or of both,\\
\vin \vin In \textsc{Tempe} or the dales of Arcady?\\
\vin What men or gods are these? What maidens loth?\\
What mad pursuit? What struggle to escape?\\*
\vin \vin What pipes and timbrels? What wild ecstasy?\\!

Heard melodies are sweet, but those unheard\\*
\vin Are sweeter; therefore, ye soft pipes, play on;\\
Not to the sensual ear, but, more endeared,\\
\vin Pipe to the spirit ditties of no tone:\\
Fair youth, beneath the trees, thou canst not leave\\
\vin Thy song, nor ever can those trees be bare;\\
\vin \vin Bold lover, never, never canst thou kiss,\\
Though winning near the goal yet, do not grieve;\\
\vin She cannot fade, though thou hast not thy bliss,\\*
\vin \vin For ever wilt thou love, and she be fair!\\!

Ah, happy, happy boughs, that cannot shed\\*
\vin Your leaves, nor ever bid the spring adieu;\\
And, happy melodist, unwearied,\\
\vin For ever piping songs for ever new;\\
More happy love! more happy, happy love!\\
\vin For ever warm and still to be enjoyed,\\
\vin \vin For ever panting, and for ever young;\\
All breathing human passion far above,\\
\vin That leaves a heart high-sorrowful and cloyed,\\*
\vin \vin A burning forehead, and a parching tongue.\\!

Who are these coming to the sacrifice?\\*
\vin To what green altar, O mysterious priest,\\
Lead'st thou that heifer lowing at the skies,\\
\vin And all her silken flanks with garlands drest?\\
What little town by river or sea shore,\\
\vin Or mountain-built with peaceful citadel,\\
\vin \vin Is emptied of this folk, this pious morn?\\
And, little town, thy streets for evermore\\
\vin Will silent be; and not a soul to tell\\*
\vin \vin Why thou art desolate, can e'er return.\\!

O attic shape! Fair attitude! with brede\\*
\vin Of marble men and maidens overwrought,\\
With forest branches and the trodden weed;\\
\vin Thou, silent form, dost tease us out of thought\\
As doth eternity: cold pastoral!\\
\vin When old age shall this generation waste,\\
\vin \vin Thou shalt remain, in midst of other woe\\
Than ours, a friend to man, to whom thou say'st,\\
\vin Beauty is truth, truth beauty -- that is all\\*
\vin \vin Ye know on earth, and all ye need to know.
\end{verse}

\subsection{}

\blfootnote{`On First Looking into Chapman's Homer', John Keats (1795 -- 1821), \cite{treasury}. Dr Gogarty wrote a rather good parody of this, entitled \refpoem{On First Looking through Krafft-Ebing's \refbook{Psychopathia Sexualis}}.}\settowidth{\versewidth}{    He stared at the Pacific -- and all his men}
\begin{verse}[\versewidth]
Much have I travelled in the realms of gold,\\*
\vin And many goodly states \& kingdoms seen;\\
\vin Round many western islands have I been\\
Which bards in fealty to \textit{Apollo} hold.\\
Oft of one wide expanse had I been told\\
\vin That deep-browed \textit{Homer} ruled as his demesne,\\
\vin Yet did I never breathe its pure serene\\
Till I heard \textit{Chapman} speak out loud \& bold.\\
Then felt I like some watcher of the skies\\
\vin When a new planet swims into his ken;\\
Or like stout \textit{Cortez}, when with eagle eyes\\
\vin He stared at the Pacific -- and all his men\\
Looked at each other with a wild surmise --\\*
\vin Silent, upon a peak in Darien.
\end{verse}

\subsection{}

\blfootnote{Matthew 8.22, \cite{kjv}.}{\color{red} Let the dead bury their dead.}

\section{}

\subsection{}

\blfootnote{`Ulysses', Alfred Tennyson, 1st Baron Tennyson, Poet Laureate (1809 -- 1892), \cite{norton}. The last line of this poem is inscribed on the cross on Observation Hill in the Antarctic, which serves as a memorial to Captain Robert Scott.}\settowidth{\versewidth}{Souls that have toiled wrought thought with me --}
\begin{verse}[\versewidth]
It little profits that an idle king,\\*
By this still hearth, among these barren crags,\\
Matched with an aged wife, I mete \& dole\\
Unequal laws unto a savage race,\\
That hoard \& sleep \& feed, and know not me.\\
I cannot rest from travel: I will drink\\
Life to the lees. All times I have enjoyed\\
Greatly, have suffered greatly, both with those\\
That loved me, and alone, on shore, and when\\
Through scudding drifts the rainy Hyades\\
Vexed the dim sea: I am become a name;\\
For always roaming with a hungry heart\\
Much have I seen and known; cities of men\\
And manners, climates, councils, governments,\\
Myself not least, but honoured of them all;\\
And drunk delight of battle with my peers,\\
Far on the ringing plains of windy \textsc{Troy}.\\
I am a part of all that I have met;\\
Yet all experience is an arch wherethrough\\
Gleams that untravelled world whose margin fades\\
For ever \& forever when I move.\\
How dull it is to pause, to make an end,\\
To rust unburnished, not to shine in use!\\
As though to breathe were life! Life piled on life\\
Were all too little, and of one to me\\
Little remains: but every hour is saved\\
From that eternal silence, something more,\\
A bringer of new things; and vile it were\\
For some three suns to store \& hoard myself,\\
And this grey spirit yearning in desire\\
To follow knowledge like a sinking star,\\*
Beyond the utmost bound of human thought.\\!

This is my son, mine own \textit{Telemachus},\\*
To whom I leave the sceptre and the isle --\\
Well-loved of me, discerning to fulfil\\
This labour, by slow prudence to make mild\\
A rugged people, and through soft degrees\\
Subdue them to the useful \& the good.\\
Most blameless is he, centred in the sphere\\
Of common duties, decent not to fail\\
In offices of tenderness, and pay\\
Meet adoration to my household gods,\\*
When I am gone. He works his work, I mine.\\!

There lies the port; the vessel puffs her sail:\\*
There gloom the dark, broad seas. My mariners,\\
Souls that have toiled \& wrought \& thought with me --\\
That ever with a frolic welcome took\\
The thunder \& the sunshine, and opposed\\
Free hearts, free foreheads -- you \& I are old;\\
Old age hath yet his honour and his toil;\\
Death closes all: but something ere the end,\\
Some work of noble note, may yet be done,\\
Not unbecoming men that strove with gods.\\
The lights begin to twinkle from the rocks:\\
The long day wanes: the slow moon climbs: the deep\\
Moans round with many voices. Come, my friends,\\
'Tis not too late to seek a newer world.\\
Push off, and sitting well in order smite\\
The sounding furrows; for my purpose holds\\
To sail beyond the sunset, and the baths\\
Of all the western stars, until I die.\\
It may be that the gulfs will wash us down:\\
It may be we shall touch the Happy Isles,\\
And see the great \textit{Achilles}, whom we knew.\\
Though much is taken, much abides; and though\\
We are not now that strength which in old days\\
Moved earth \& heaven, that which we are, we are;\\
One equal temper of heroic hearts,\\
Made weak by time and fate, but strong in will\\*
To strive, to seek, to find, and not to yield.
\end{verse}

\subsection{}

\blfootnote{`The Eagle', Alfred Tennyson, 1st Baron Tennyson, Poet Laureate (1809 -- 1892), \cite{norton}.}\settowidth{\versewidth}{He clasps the crag with crooked hands;}
\begin{verse}[\versewidth]
He clasps the crag with crooked hands;\\*
Close to the sun in lonely lands,\\*
Ringed with the azure world, he stands.\\!

The wrinkled sea beneath him crawls;\\*
He watches from his mountain walls,\\*
And like a thunderbolt he falls.
\end{verse}

\subsection{}

\blfootnote{Matthew 7.6, \cite{kjv}.}{\color{red} Neither cast ye your pearls before swine.}

\section{}

\subsection{}

\blfootnote{`Elizabeth of Bohemia', Sir Henry Wotton (1568 -- 1639), \cite{treasury}. The poem would seem to be dedicated to Elizabeth, Queen (consort) of Bohemia, wife of Frederick, King of Bohemia, and daughter of James I \& VI. \P 5. Where Palgrave reads `Moon', the best texts consulted read `Sun'; but the Almanacker finds `Moon' more pleasing. \P 15. Philomel or Philomela is a poetical term for a nightingale. According to Greek and Roman mythology (see Ὀδύσσεια XIX.518-23 and many others) Philomela was transformed into a nightingale.}\settowidth{\versewidth}{    As if the spring were all your own --}
\begin{verse}[\versewidth]
You meaner beauties of the night,\\*
\vin Which poorly satisfy our eyes\\
More by your number than your light,\\
\vin You common people of the skies --\\*
What are you, when the moon shall rise?\\!

Ye violets that first appear,\\*
\vin By your pure purple mantles known\\
Like the proud virgins of the year,\\
\vin As if the spring were all your own --\\*
What are you, when the rose is blown?\\!

Ye curious chanters of the wood\\*
\vin That warble forth dame nature's lays,\\
Thinking your passions understood\\
\vin By your weak accents -- what's your praise\\*
When \textit{Philomel} her voice doth raise?\\!

So when my mistress shall be seen\\*
\vin In sweetness of her looks \& mind,\\
By virtue first, then choice, a queen,\\
\vin Tell me, if she were not designed\\*
Th' eclipse \& glory of her kind?
\end{verse}

\subsection{}

\blfootnote{`Speech for Psyche in the Golden Book of Apuleius', Ezra Pound (1885 -- 1972), \cite{gmp}.}\settowidth{\versewidth}{Close, and as the petals of flowers in falling}
\begin{verse}[\versewidth]
All night, and as the wind lieth among\\*
The cypress trees, he lay,\\
Nor held me save as air that brusheth by one\\
Close, and as the petals of flowers in falling\\
Waver and seem not drawn to earth, so he\\
Seemed over me to hover light as leaves\\
And closer me than air,\\
And music flowing through me seemed to open\\
Mine eyes upon new colours.\\*
O winds, what wind can match the weight of him?
\end{verse}

\subsection{}

\blfootnote{Matthew 12.33, \cite{kjv}.}{\color{red} The tree is known by his fruit.}

\section{}

\subsection{}

\blfootnote{`She Walks in Beauty', George Noel, 6th Baron Byron (1788 -- 1824), \cite{treasury}. This poem was inspired by a woman who was born Anne Horton, but whose name was Anne Wilmot -- following her marriage to one Robert Wilmot, a cousin of the poet -- when Lord Byron met her at a party in June 1814. Later, having inherited an estate from his father-in-law, Robert would take Horton as an additional surname, and thus, having climbed through the ranks of the civil service and having inherited his father's baronetcy, he died in 1841 with the much more impressive style of The Rt Hon Sir Robert Wilmot-Horton, 3rd Baronet; and thus Mrs Wilmot became Lady Wilmot-Horton.}\settowidth{\versewidth}{The smiles that win, the tints that glow,}
\begin{verse}[\versewidth]
She walks in beauty, like the night\\*
\vin Of cloudless climes \& starry skies,\\
And all that's best of dark \& bright\\
\vin Meets in her aspect \& her eyes;\\
Thus mellowed to that tender light\\*
\vin Which heaven to gaudy day denies.\\!

One shade the more, one ray the less,\\*
\vin Had \sfrac{$1$}{$2$} impaired the nameless grace\\
Which waves in every raven tress\\
\vin Or softly lightens o'er her face,\\
Where thoughts serenely sweet express\\*
\vin How pure, how dear their dwelling-place.\\!

And on that cheek \& o'er that brow\\*
\vin So soft, so calm, yet eloquent,\\
The smiles that win, the tints that glow,\\
\vin But tell of days in goodness spent --\\
A mind at peace with all below,\\*
\vin A heart whose love is innocent.
\end{verse}

\subsection{}

\blfootnote{John Milton (1608 -- 1674), \cite{obev}. This song concludes Milton's masque \refbook{Arcades}.}\settowidth{\versewidth}{    A better soil shall give ye thanks.}
\begin{verse}[\versewidth]
Nymphs \& shepherds, dance no more\\*
\vin By sandy \textsc{Ladon}'s lilied banks.\\
On old \textsc{Lycaeus} or \textsc{Cyllene} hoar,\\
\vin Trip no more in twilight ranks;\\
Though \textsc{Erymanth} your loss deplore,\\
\vin A better soil shall give ye thanks.\\
From the stony \textsc{Maenalus},\\
Bring your flocks, and live with us;\\
Here ye shall have greater grace\\
To serve the lady of this place.\\
\vin Though syrinx your \textit{Pan}'s mistress were,\\
\vin Yet syrinx well might wait on her.\\
\vin \vin Such a rural queen\\*
\vin All Arcadia hath not seen.
\end{verse}

\subsection{}

\blfootnote{Matthew 6.21, \cite{kjv}.}{\color{red} Where your treasure is, there will your heart be also.}

\section{}

\subsection{}

\blfootnote{`The Passionate Shepherd to His Love', Christopher Marlowe (1564 -- 1593), \cite{treasury}. Sir Walter Raleigh wrote a poem in which the beloved replies.}\settowidth{\versewidth}{Embroidered all with leaves of myrtle.}
\begin{verse}[\versewidth]
Come live with me and be my love,\\*
And we will all the pleasures prove\\
That hills \& valleys, dale \& field,\\*
And all the craggy mountains yield.\\!

There will we sit upon the rocks\\*
And see the shepherds feed their flocks,\\
By shallow rivers, to whose falls\\*
Melodious birds sing madrigals.\\!

There will I make thee beds of roses\\*
And a thousand fragrant posies,\\
A cap of flowers, and a kirtle\\*
Embroidered all with leaves of myrtle.\\!

A gown made of the finest wool\\*
Which from our pretty lambs we pull,\\
Fair lin\`{e}d slippers for the cold,\\*
With buckles of the purest gold.\\!

A belt of straw and ivy buds\\*
With coral clasps \& amber studs:\\
And if these pleasures may thee move,\\*
Come live with me and be my love.\\!

Thy silver dishes for thy meat\\*
As precious as the gods do eat,\\
Shall on an ivory table be\\*
Prepared each day for thee \& me.\\!

The shepherd swains shall dance \& sing\\*
For thy delight each may-morning:\\
If these delights thy mind may move,\\*
Then live with me and be my love.
\end{verse}

\subsection{}

\blfootnote{`To Celia', Ben Jonson (1572 -- 1637), \cite{treasury}. Much of this poem is said to be drawn from antiquity, specifically a love letter by the sophist Philostratus. There is a second verse, but the Almanacker finds it much inferior to the first.}\settowidth{\versewidth}{Drink to me only with thine eyes,}
\begin{verse}[\versewidth]
Drink to me only with thine eyes,\\*
\vin And I will pledge with mine;\\
Or leave a kiss but in the cup\\
\vin And I'll not look for wine.\\
The thirst that from the soul doth rise\\
\vin Doth ask a drink divine;\\
But might I of \textit{Jove}'s nectar sup,\\*
\vin I would not change for thine.
\end{verse}

\subsection{}

\blfootnote{Matthew 5.41, \cite{kjv}.}{\color{red} Whosoever shall compel thee to go a mile, go with him twain.}

\section{}

\subsection{}

\blfootnote{`The Salutation', The Rev Thomas Traherne (1636 -- 1674), \cite{norton}. This poem, as with all the Rev Traherne's verses, was first published more than two centuries after the poet's death.}\settowidth{\versewidth}{    Strange things doth meet, strange glories see;}
\begin{verse}[\versewidth]
\vin \vin \vin These little limbs,\\*
\vin These eyes \& hands which here I find,\\
These rosy cheeks wherewith my life begins,\\
\vin Where have ye been? Behind\\
What curtain were ye from me hid so long?\\*
Where was, in what abyss, my speaking tongue?\\!

\vin \vin \vin When silent I\\*
\vin So many 1000 1000 years\\
Beneath the dust did in a chaos lie,\\
\vin How could I smiles or tears,\\
Or lips or hands or eyes or ears perceive?\\*
Welcome ye treasures which I now receive.\\!

\vin \vin \vin I that so long\\*
\vin Was nothing from eternity,\\
Did little think such joys as ear or tongue\\
\vin To celebrate or see:\\
Such sounds to hear, such hands to feel, such feet,\\*
Beneath the skies on such a ground to meet.\\!

\vin \vin \vin New burnished joys,\\*
\vin Which yellow gold \& pearls excel!\\
Such sacred treasures are the limbs in boys,\\
\vin In such a soul doth dwell;\\
Their organis\`{e}d joints \& azure veins\\*
More wealth include than all the world contains.\\!

\vin \vin \vin From dust I rise,\\*
\vin And out of nothing now awake;\\
These brighter regions which salute mine eyes,\\
\vin A gift from God I take.\\
The earth, the seas, the light, the day, the skies,\\*
The sun \& stars are mine if those I prize.\\!

\vin \vin \vin Long time before\\*
\vin I in my mother's womb was born,\\
A God, preparing, did this glorious store,\\
\vin The world, for me adorn.\\
Into this Eden so divine \& fair,\\*
So wide and bright, I come his son \& heir.\\!

\vin \vin \vin A stranger here\\*
\vin Strange things doth meet, strange glories see;\\
Strange treasures lodged in this fair world appear,\\
\vin Strange all and new to me;\\
But that they mine should be, who nothing was,\\*
That strangest is of all, yet brought to pass.
\end{verse}

\subsection{}

\blfootnote{John Lyly (1553 -- 1606), \cite{treasury}.}\settowidth{\versewidth}{    O love! has she done this to thee?}
\begin{verse}[\versewidth]
\textit{Cupid} \& my \textit{Campaspe} played\\*
At cards for kisses; \textit{Cupid} paid:\\
He stakes his quiver, bow \& arrows,\\
His mother's doves \& team of sparrows;\\
Loses them too; then down he throws\\
The coral of his lip, the rose\\
Growing on 's cheek (but none knows how);\\
With these, the crystal of his brow,\\
And then the dimple on his chin;\\
All these did my \textit{Campaspe} win:\\
And last he set her both his eyes --\\
She won, and \textit{Cupid} blind did rise.\\
\vin O love! has she done this to thee?\\*
\vin What shall, alas! become of me?
\end{verse}

\subsection{}

\blfootnote{Proverbs 1.17, \cite{kjv}.}In vain the net is spread in the sight of any bird.

\section{}

\subsection{}

\blfootnote{`To an Athlete Dying Young', Prof Alfred Housman (1859 -- 1936), \cite{norton}.}\settowidth{\versewidth}{We chaired you through the market-place;}
\begin{verse}[\versewidth]
The time you won your town the race\\*
We chaired you through the market-place;\\
Man \& boy stood cheering by,\\*
And home we brought you shoulder-high.\\!

Today, the road all runners come,\\*
Shoulder-high we bring you home,\\
And set you at your threshold down,\\*
Townsman of a stiller town.\\!

Smart lad, to slip betimes away\\*
From fields where glory does not stay,\\
And early though the laurel grows\\*
It withers quicker than the rose.\\!

Eyes the shady night has shut\\*
Cannot see the record cut,\\
And silence sounds no worse than cheers\\*
After earth has stopped the ears.\\!

Now you will not swell the rout\\*
Of lads that wore their honours out,\\
Runners whom renown outran\\*
And the name died before the man.\\!

So set, before its echoes fade,\\*
The fleet foot on the sill of shade,\\
And hold to the low lintel up\\*
The still-defended challenge-cup.\\!

And round that early-laurelled head\\*
Will flock to gaze the strengthless dead,\\
And find unwithered on its curls\\*
The garland briefer than a girl's.
\end{verse}

\subsection{}

\blfootnote{`Counsel to Girls', Robert Herrick (1591 -- 1674), \cite{treasury}. \refpoem{Counsel to Girls} seems to be Palgrave's bowdlerisation; the original title was \refpoem{To the Virgins, to make much of Time} -- the premise of a joke in \refbook{Dead Poets Society} (1989).}\settowidth{\versewidth}{The glorious lamp of heav'n, the sun,}
\begin{verse}[\versewidth]
Gather ye rosebuds while ye may,\\*
\vin Old time is still a-flying;\\
And this same flower that smiles today,\\*
\vin Tomorrow will be dying.\\!

The glorious lamp of heav'n, the sun,\\*
\vin The higher he's a-getting\\
The sooner will his race be run,\\*
\vin And nearer he's to setting.\\!

That age is best which is the first,\\*
\vin When youth \& blood are warmer;\\
But being spent, the worse, \& worst\\*
\vin Times, still succeed the former.\\!

Then be not coy, but use your time;\\*
\vin And while ye may, go marry:\\
For having lost but once your prime,\\*
\vin You may for ever tarry.
\end{verse}

\subsection{}

\blfootnote{Isaiah 22.13, \cite{kjv}.}Let us eat and drink; for to morrow we shall die.

\section{}

\subsection{}

\blfootnote{`My Last Duchess', Robert Browning (1828 -- 1889), \cite{obev}. The narrator of the poem is Alfonso II of Ferrara. It is likely that the duke was homosexual; he fathered no children despite three marriages, nor was he known ever to have kept a mistress. At the time of her death, it was widely believed that he had had his first wife, the sixteen-year-old Lucrezia de' Medici, of whom Bronzino painted an exquisite portrait, poisoned; although later writers have suggested that she more likely succumbed to tuberculosis.}\settowidth{\versewidth}{Half-flush that dies along her throat.' Such stuff}
\begin{verse}[\versewidth]
That's my last duchess painted on the wall,\\*
Looking as if she were alive. I call\\
That piece a wonder, now; Fr{\`{e}} \textit{Pandolf}'s hands\\
Worked busily a day, and there she stands.\\
Will't please you sit and look at her? I said,\\
`Fra Pandolf' by design, for never read\\
Strangers like you that pictured countenance,\\
The depth \& passion of its earnest glance,\\
But to myself they turned (since none puts by\\
The curtain I have drawn for you, but I)\\
And seemed as they would ask me, if they durst,\\
How such a glance came there; so, not the first\\
Are you to turn and ask thus. Sir, 'twas not\\
Her husband's presence only, called that spot\\
Of joy into the duchess' cheek; perhaps\\
Fr{\`{e}} \textit{Pandolf} chanced to say, `Her mantle laps\\
Over my lady's wrist too much,' or `Paint\\
Must never hope to reproduce the faint\\
Half-flush that dies along her throat.' Such stuff\\
Was courtesy, she thought, and cause enough\\
For calling up that spot of joy. She had\\
A heart -- how shall I say? -- too soon made glad,\\
Too easily impressed; she liked whate'er\\
She looked on, and her looks went everywhere.\\
Sir, 'twas all one! My favour at her breast,\\
The dropping of the daylight in the west,\\
The bough of cherries some officious fool\\
Broke in the orchard for her, the white mule\\
She rode with round the terrace -- all \& each\\
Would draw from her alike the approving speech,\\
Or blush, at least. She thanked men -- good! But thanked\\
Somehow -- I know not how -- as if she ranked\\
My gift of a 900-years-old name\\
With anybody's gift. Who'd stoop to blame\\
This sort of trifling? Even had you skill\\
In speech -- which I have not -- to make your will\\
Quite clear to such an one, and say, `Just this\\
Or that in you disgusts me; here you miss,\\
Or there exceed the mark'' -- and if she let\\
Herself be lessoned so, nor plainly set\\
Her wits to yours, forsooth, and made excuse --\\
E'en then would be some stooping; and I choose\\
Never to stoop. O sir, she smiled, no doubt,\\
Whene'er I passed her; but who passed without\\
Much the same smile? This grew; I gave commands;\\
Then all smiles stopped together. There she stands\\
As if alive. Will't please you rise? We'll meet\\
The company below, then. I repeat,\\
The Count your master's known munificence\\
Is ample warrant that no just pretense\\
Of mine for dowry will be disallowed;\\
Though his fair daughter's self, as I avowed\\
At starting, is my object. Nay, we'll go\\
Together down, sir. Notice \textit{Neptune}, though,\\
Taming a sea-horse, thought a rarity,\\*
Which \textit{Claus of Innsbruck} cast in bronze for me!
\end{verse}

\subsection{}

\blfootnote{Edmund Waller (1606 -- 1687), \cite{treasury}.}\settowidth{\versewidth}{The common fate of all things rare}
\begin{verse}[\versewidth]
\vin Go, lovely rose!\\*
Tell her that wastes her time \& me\\
\vin That now she knows,\\
When I resemble her to thee,\\*
How sweet \& fair she seems to be.\\!

\vin Tell her that's young\\*
And shuns to have her graces spied,\\
\vin That hadst thou sprung\\
In deserts, where no men abide,\\*
Thou must have uncommended died.\\!

\vin Small is the worth\\*
Of beauty from the light retired:\\
\vin Bid her come forth,\\
Suffer herself to be desired,\\*
And not blush so to be admired.\\!

\vin Then die! that she\\*
The common fate of all things rare\\
\vin May read in thee:\\
How small a part of time they share\\*
That are so wondrous sweet \& fair!
\end{verse}

\subsection{}

\blfootnote{1 Kings 12.10, \cite{kjv}.}My little finger shall be thicker than my father's loins.

\section{}

\subsection{}

\blfootnote{`The Tyger', William Blake (1757 -- 1827), \cite{norton}.}\settowidth{\versewidth}{Could twist the sinews of thy heart?}
\begin{verse}[\versewidth]
Tyger, tyger, burning bright,\\*
In the forests of the night;\\
What immortal hand or eye,\\*
Could frame thy fearful symmetry?\\!

In what distant deeps or skies\\*
Burnt the fire of thine eyes?\\
On what wings dare he aspire?\\*
What the hand, dare seize the fire?\\!

And what shoulder \& what art,\\*
Could twist the sinews of thy heart?\\
And when thy heart began to beat,\\*
What dread hand \& what dread feet?\\!

What the hammer? What the chain?\\*
In what furnace was thy brain?\\
What the anvil? What dread grasp,\\*
Dare its deadly terrors clasp!\\!

When the stars threw down their spears\\*
And watered heaven with their tears:\\
Did he smile his work to see?\\*
Did he who made the lamb make thee?\\!

Tyger! tyger! burning bright,\\*
In the forests of the night:\\
What immortal hand or eye,\\*
Dare frame thy fearful symmetry?
\end{verse}

\subsection{}

\blfootnote{$\mathbb{R}$ `An Ode', Matthew Prior (1664 -- 1721), \cite{norton}. \P 6. Where the Almanacker has put `dresser', the orginal reads `toilet', a word which, these days, unfortunately has less pleasant associations. \P 15. The `Loves' in this context are the three Graces of Graeco-Roman mythology.}\settowidth{\versewidth}{    I sung gazed; I played and trembled;}
\begin{verse}[\versewidth]
The merchant, to secure his treasure,\\*
\vin Conveys it in a borrowed name;\\
\textit{Euphalia} serves to grace my measure,\\*
\vin But \textit{Cloe} is my real flame.\\!

My softest verse, my darling lyre,\\*
\vin Upon \textit{Euphalia}'s dresser lay;\\
When \textit{Cloe} noted her desire\\*
\vin That I should sing, that I should play.\\!

My lyre I tune, my voice I raise,\\*
\vin But with my numbers mix my sighs;\\
And whilst I sing \textit{Euphalia}'s praise,\\*
\vin I fix my soul on \textit{Cloe}'s eyes.\\!

Fair \textit{Cloe} blushed; \textit{Euphalia} frowned;\\*
\vin I sung \& gazed; I played and trembled;\\
And \textit{Venus} to the Loves around\\*
\vin Remarked how ill we all dissembled.
\end{verse}

\subsection{}

\blfootnote{1 Corinthians 6.2, \cite{kjv}.}Now is the accepted time.

\section{}

\subsection{}

\blfootnote{Anonymous, \cite{newlove}.}\settowidth{\versewidth}{On her smooth limbs my hands did stray;}
\begin{verse}[\versewidth]
I gently touched her hand: she gave\\*
A look that did my soul enslave;\\
I press\`{e}d to her rebel lips in vain:\\
They rose up to be pressed again.\\
\vin Thus happy, I no further meant\\*
\vin Than to be pleased \& innocent.\\!

On her soft breasts my hand I laid,\\*
And a quick light impression made;\\
They with a kindly warmth did glow,\\
And swelled, \& seemed to overflow.\\
\vin Yet -- trust me -- I no farther meant\\*
\vin Than to be pleased \& innocent.\\!

On her eyes my eyes did stay:\\*
On her smooth limbs my hands did stray;\\
Each sense was ravished with delight,\\
And my soul stood prepared for flight.\\
\vin Blame me not if at last I meant\\*
\vin More to be pleased than innocent.
\end{verse}

\subsection{}

\blfootnote{`The Windhover', Fr Gerard Hopkins (1844 -- 1889), \cite{norton}.}\settowidth{\versewidth}{    Buckle! And the fire that breaks from thee then, a 1,000,000,000}
\begin{verse}[\versewidth]
I caught this morning morning's minion, king-\\*
\vin dom of daylight's dauphin, dapple-dawn-drawn falcon, in his riding\\
\vin Of the rolling level underneath him steady air, and striding\\
High there, how he rung upon the rein of a wimpling wing\\
In his ecstasy! Then off, off forth on swing,\\
\vin As a skate's heel sweeps smooth on a bow-bend: the hurl \& gliding\\
\vin Rebuffed the big wind. My heart in hiding\\*
Stirred for a bird -- the achieve of, the mastery of the thing!\\!

Brute beauty \& valour \& act, oh, air, pride, plume, here\\*
\vin Buckle! \'{A}nd the fire that breaks from thee then, a 1,000,000,000\\
Times told lovelier, more dangerous, O my chevalier!\\
\vin No wonder of it: sh\'{e}er pl\'{o}d makes plough down sillion\\
Shine, and blue-bleak embers, ah my dear,\\*
\vin Fall, gall themselves, and gash gold-vermilion.
\end{verse}

\subsection{}

\blfootnote{Dr William Wordsworth, Poet Laureate (1770 -- 1850), \cite{norton}. This is the seventh of a nine-line poem which begins, `My heart leaps up when I behold'.}The child is father of the man.

\section{}

\subsection{}

\blfootnote{Fulke Greville, 1st Baron Brooke (1554 -- 1628), \cite{obev}. \P 3. There was an ancient practice of divining the entrance of an unexpected guest by the patterns made by burnt material fluttering up a chimney (which Coleridge also alludes to in `Frost at Midnight'). \P 19. Argus is a giant from Greek mythology who, having dozens of eyes, is proverbially wakeful and vigilant; however, he was lulled asleep and murdered by Mercury in order to facilitate Jupiter's illicit liaisons with Io. Vulcan, to the best of the Almanacker's knowledge, was said to have had only one full brother, Mars; the allusion here is perhaps to the trap sprung by Vulcan to catch Mars and Venus \textit{in flagrante delicto}.}\settowidth{\versewidth}{Must I lose ring, flowers, blush, theft, and go naked,}
\begin{verse}[\versewidth]
I, with whose colours \textit{Myra} dressed her head,\\*
\vin I, that ware posies of her own hand-making,\\
I, that mine own name in the chimneys read\\
\vin By \textit{Myra} finely wrought ere I was waking:\\
Must I look on, in hope time coming may\\*
With change bring back my turn again to play?\\!

I, that on sunday at the church-stile found\\*
\vin A garland sweet, with true-love knots in flowers,\\
Which I to wear about mine arm was bound,\\
\vin That each of us might know that all was ours:\\
Must I now lead an idle life in wishes,\\*
And follow \textit{Cupid} for his loaves \& fishes?\\!

I, that did wear the ring her mother left,\\*
\vin I, for whose love she gloried to be blamed,\\
I, with whose eyes her eyes committed theft,\\
\vin I, who did make her blush when I was named:\\
Must I lose ring, flowers, blush, theft, and go naked,\\*
Watching with sighs till dead love be awak\`{e}d?\\!

I, that, when drowsy \textit{Argus} fell asleep,\\*
\vin Like jealousy o'erwatch\`{e}d with desire,\\
Was even warn\`{e}d modesty to keep,\\
\vin While her breath, speaking, kindled nature's fire:\\
Must I look on a-cold, while others warm them?\\*
Do \textit{Vulcan}'s brothers in such fine nets arm them?\\!

Was it for this that I might \textit{Myra} see\\*
\vin Washing the water with her beauties white?\\
Yet would she never write her love to me.\\
\vin Thinks wit of change, while thoughts are in delight?\\
Mad girls must safely love as they may leave;\\*
No man can print a kiss: lines may deceive.
\end{verse}

\subsection{}

\blfootnote{`Sudden Light', Gabriel Rossetti (1828 -- 1882), \cite{norton}.}\settowidth{\versewidth}{The sighing sound, the lights around the shore.}
\begin{verse}[\versewidth]
\vin \vin I have been here before,\\*
\vin \vin \vin But when or how I cannot tell:\\
\vin \vin I know the grass beyond the door,\\
\vin \vin \vin The sweet keen smell,\\*
The sighing sound, the lights around the shore.\\!

\vin \vin You have been mine before.\\*
\vin \vin \vin How long ago I may not know:\\
\vin \vin But just when at that swallow's soar\\
\vin \vin \vin Your neck turned so,\\*
Some veil did fall. I knew it all of yore.\\!

\vin \vin Has this been thus before?\\*
\vin \vin \vin And shall not thus time's eddying flight\\
\vin \vin Still with our lives our love restore\\
\vin \vin \vin In death's despite,\\*
And day \& night yield one delight once more?
\end{verse}

\subsection{}

\blfootnote{Romans 4.15, \cite{kjv}.}Where no law is, there is no transgression.

\section{}

\subsection{}

\blfootnote{Dr Thomas Campion (1567 -- 1620), \cite{norton}. Prof Auden's `O lurcher-loving collier, black as night' was clearly written in response to this poem. One can imagine P W Botha approving of these lines, but that -- it hardly needs saying -- is not what Dr Campion is getting at.}\settowidth{\versewidth}{The sun still proved, the shadow still disdained.}
\begin{verse}[\versewidth]
Follow thy fair sun, unhappy shadow;\\*
Though thou be black as night,\\
And she made all of light,\\*
Yet follow thy fair sun, unhappy shadow.\\!

Follow her whose light thy light depriveth;\\*
Though here thou liv'st disgraced,\\
And she in heaven is placed,\\*
Yet follow her who light the world reviveth.\\!

Follow those pure beams whose beauty burneth,\\*
That so have scorch\`{e}d thee,\\
That thou still black must be,\\*
Till her kind beams thy black to brightness turneth.\\!

Follow her while yet her glory shineth;\\*
There comes a luckless night,\\
That will dim all her light;\\*
And this the black unhappy shade divineth.\\!

Follow still since so thy fates ordain\`{e}d;\\*
The sun must have his shade,\\
Till both at once do fade:\\*
The sun still proved, the shadow still disdain\`{e}d.
\end{verse}

\subsection{}

\blfootnote{`Rooms', Miss Charlotte Mew (1869 -- 1928), \cite{norton}.}\settowidth{\versewidth}{    As we shall somewhere in the other quieter, dustier bed}
\begin{verse}[\versewidth]
I remember rooms that have had their part\\*
In the steady slowing down of the heart.\\
The room in \textsc{Paris}, the room at \textsc{Geneva},\\
The little damp room with the seaweed smell,\\
And that ceaseless maddening sound of the tide --\\
\vin Rooms where for good or for ill -- things died.\\
But there is the room where we two lie dead,\\
Though every morning we seem to wake and might just as well seem to sleep again\\
\vin As we shall somewhere in the other quieter, dustier bed\\*
\vin Out there in the sun -- in the rain.
\end{verse}

\subsection{}

\blfootnote{Proverbs 39.18, \cite{kjv}.}Where there is no vision, the people perish.

\section{}

\subsection{}

\blfootnote{`Kubla Khan', Samuel Coleridge (1772 -- 1834), \cite{treasury}. Coleridge wrote a lengthy prose introduction to this poem, wherein he describes how he was inspired by laudanum and \refbook{Purchas's Pilgrimes}, and how he was prevented from perfecting it by `a person on business from Purlock'. Xanadu = Shangdu, summer capital of the Yuan dynasty. Kubla Khan = Kublai Khan, fifth Khagan of the Mongol Empire and first Yuan Emperor of China.}\settowidth{\versewidth}{As if this earth in fast thick pants were breathing,}
\begin{verse}[\versewidth]
In \textsc{Xanadu} did \textit{Kubla Khan}\\*
A stately pleasure-dome decree:\\
Where \textsc{Alph}, the sacred river, ran\\
Through caverns measureless to man\\
\vin Down to a sunless sea.\\
So twice five miles of fertile ground\\
With walls \& towers were girdled round;\\
And there were gardens bright with sinuous rills,\\
Where blossomed many an incense-bearing tree;\\
And here were forests ancient as the hills,\\*
Enfolding sunny spots of greenery.\\!

But O that deep romantic chasm which slanted\\*
Down the green hill athwart a cedarn cover!\\
A savage place, as holy \& enchanted\\
As e'er beneath a waning moon was haunted\\
By woman wailing for her demon-lover!\\
And from this chasm, with ceaseless turmoil seething,\\
As if this earth in fast thick pants were breathing,\\
A mighty fountain momently was forced:\\
Amid whose swift \sfrac{$1$}{$2$} intermitted burst\\
Huge fragments vaulted like rebounding hail,\\
Or chaffy grain beneath the thresher's flail:\\
And mid these dancing rocks at once \& ever\\
It flung up momently the sacred river.\\
Five miles meandering with a mazy motion\\
Through wood \& dale the sacred river ran,\\
Then reached the caverns measureless to man,\\
And sank in tumult to a lifeless ocean;\\
And mid this tumult \textit{Kubla} heard from far\\
Ancestral voices prophesying war!\\
\vin The shadow of the dome of pleasure\\
\vin Floated midway on the waves;\\
\vin Where was heard the mingled measure\\
\vin From the fountain and the caves.\\
It was a miracle of rare device,\\*
A sunny pleasure-dome with caves of ice!\\!

\vin A damsel with a dulcimer\\*
\vin In a vision once I saw:\\
\vin It was an abyssinian maid\\
\vin And on her dulcimer she played,\\
\vin Singing of Mount \textsc{Abora}.\\
\vin Could I revive within me\\
\vin Her symphony \& song,\\
\vin To such a deep delight 'twould win me,\\
That with music loud \& long,\\
I would build that dome in air,\\
That sunny dome! Those caves of ice!\\
And all who heard should see them there,\\
And all should cry, `Beware! Beware!\\
His flashing eyes, his floating hair!'\\
Weave a circle round him thrice,\\
And close your eyes with holy dread\\
For he on honey-dew hath fed,\\*
And drunk the milk of paradise.
\end{verse}

\subsection{}

\blfootnote{Michael Drayton (1563 -- 1631), \cite{norton}.}\settowidth{\versewidth}{If he from heaven that filched the living fire}
\begin{verse}[\versewidth]
If he from heaven that filched the living fire\\*
Condemned by \textit{Jove} to endless torment be,\\
I greatly marvel how you still go free,\\
That far beyond \textit{Prometheus} did aspire.\\
The fire he stole, although of heavenly kind,\\
Which from above he craftily did take,\\
Of lifeless clods, us living men to make,\\
He did bestow in temper of the mind.\\
But you broke into heaven's immortal store,\\
Where virtue, honour, wit, and beauty lay;\\
Which taking thence you have escaped away,\\
Yet stand as free as ere you did before;\\
\vin Yet old \textit{Prometheus} punished for his rape.\\*
\vin Thus poor thieves suffer while the greater 'scape.
\end{verse}

\subsection{}

\blfootnote{Anonymous, \cite{odq}.}A bird in the hand is worth two in the bush.

\section{}

\subsection{}

\blfootnote{`Dialogue: After Enjoyment', Abraham Cowley (1618 -- 1667), \cite{newlove}.}\settowidth{\versewidth}{When long t'as gnawed within will break the skin at last.}
\begin{verse}[\versewidth]
\vin What have we done? What cruel passion moved thee\\*
\vin \vin Thus to ruin her that loved thee?\\
\vin Me thou'st robbed, but what art thou\\
\vin Thyself the richer now?\\
\vin \vin Shame succeeds the short-lived pleasure;\\*
So soon is spent \& gone, this thy ill-gotten treasure.\\!

\vin We've done no harm; nor was it theft in me,\\*
\vin \vin But noblest charity in thee.\\
\vin I'll the well-gotten pleasure\\
\vin Safe in my mem'ry treasure;\\
\vin \vin What though the flower itself do waste,\\*
The essence from it drawn does long \& sweeter last.\\!

\vin No: I'm undone; my honour thou hast slain,\\*
\vin \vin And nothing can restore't again.\\
\vin Art \& labour to bestow\\
\vin Upon the carcass of it now\\
\vin \vin Is but t'embalm a body dead;\\*
The figure may remain; the life \& beauty's fled.\\!

\vin Never, my dear, was honour yet undone\\*
\vin \vin By love, but indiscretion.\\
\vin To th'wise it all things does allow;\\
\vin And cares not what we do, but how.\\
\vin \vin Like tapers shut in ancient urns,\\*
Unless it let in air for ever shines \& burns.\\!

\vin Thou first perhaps, who didst the fault commit,\\*
\vin \vin Wilt make thy wicked boast of it.\\
\vin For men, with roman pride, above\\
\vin The conquest, do the triumph love:\\
\vin \vin Nor think a perfect vict'ry gained\\*
Unless they through the streets their captive lead enchained.\\!

\vin Whoe'er his secret joys has open laid,\\*
\vin \vin The bawd to his own wife is made.\\
\vin Beside what boast is left for me,\\
\vin Whose whole wealth's a gift from thee?\\
\vin \vin 'Tis you the conqu'ror are; 'tis you\\*
Who have not only ta'en, but bound \& gagged me too.\\!

\vin Though publique pun'shment we escape, the sin\\*
\vin \vin Will rack \& torture us within:\\
\vin Guilt \& sin our bosom bears;\\
\vin And though fair, yet the fruit appears,\\
\vin \vin That worm which now the core does waste,\\*
When long t'as gnawed within will break the skin at last.\\!

\vin That thirsty drink, that hungry food I sought,\\*
\vin \vin That wounded balm, is all my fault.\\
\vin And thou in pity didst apply,\\
\vin The kind \& only remedy:\\
\vin \vin The cause absolves the crime; since me\\*
So mighty force did move, so mighty goodness thee.\\!

\vin Curse on thine arts. Methinks I hate thee now;\\*
\vin \vin And yet I'm sure I love thee too!\\
\vin I'm angry, but my wrath will prove,\\
\vin More innocent than did thy love.\\
\vin \vin Thou hast this day undone me quite;\\*
Yet wilt undo me more should'st thou not come at night.
\end{verse}

\subsection{}

\blfootnote{`A Fable', Matthew Prior (1664 -- 1721), \cite{norton}. Prior supplies a `Moral' to this poem, wherein he explains that the `honest wretch' stands for William III, and his two wives for the Tories and Whigs.}\settowidth{\versewidth}{For different age they had, and different wills;}
\begin{verse}[\versewidth]
In \textit{Aesop}'s tales an honest wretch we find,\\*
Whose years \& comforts equally declined;\\
He in two wives had two domestic ills,\\
For different age they had, and different wills;\\
One plucked his black hairs out, and one his grey;\\
The man for quietness did both obey,\\
Till all his parish saw his head quite bare,\\*
And thought he wanted brains as well as hair.
\end{verse}

\subsection{}

\blfootnote{Anonymous, \cite{odq}.}A blind man's wife needs no paint.

\section{}

\subsection{}

\blfootnote{`To His Mistress Going to Bed', Elegy XIX, The Very Rev Dr John Donne (1572 -- 1631), \cite{norton}. \P 17. Other sources put `softly' instead of `safely'. \P 20. Other sources put `revealed to' instead of `received by'. \P 24. Be sure not to miss the rather crude, though rather good, joke for which, one presumes, this poem was censored from the 1633 \textit{Poems}. \P 38. Other sources put `court' instead of covet. \P 41. Other sources put `bodies' instead of `books'. \P 46. Other sources put `Here is no penance much less innocence' instead of `There is no penance due to innocence'.}\settowidth{\versewidth}{As when from flowery meads th'hill's shadow steals.}
\begin{verse}[\versewidth]
Come, madam, come; all rest my powers defy.\\*
Until I labour, I in labour lie.\\
The foe oft-times having the foe in sight,\\
Is tired with standing though he never fight.\\
Off with that girdle, like heaven's zone glistering,\\
But a far fairer world encompassing.\\
Unpin that spangled breastplate that you wear,\\
That th'eyes of busy fools may be stopped there.\\
Unlace yourself, for that harmonious chime\\
Tells me from you that now it is bed-time.\\
Off with that happy busk, which I envy,\\
That still can be, and still can stand so nigh.\\
Your gown going off, such beauteous state reveals,\\
As when from flowery meads th'hill's shadow steals.\\
Off with that wiry coronet and show\\
The hairy diadem which on you doth grow:\\
Now off with those shoes, and then safely tread\\
In this love's hallowed temple, this soft bed.\\
In such white robes, heaven's angels used to be\\
Received by men: thou, angel, bring'st with thee\\
A heaven like \textit{Mahomet}'s paradise; and though\\
Ill spirits walk in white, we easily know\\
By this these angels from an evil sprite:\\*
Those set our hairs, but these our flesh upright.\\!

License my roving hands, and let them go\\*
Before, behind, between, above, below.\\
O my America, my new-found land,\\
My kingdom, safeliest when with one man manned,\\
My mine of precious stones, my empery,\\
How blessed am I in this discovering thee!\\
To enter in these bonds is to be free;\\*
Then where my hand is set, my seal shall be.\\!

Full nakedness, all joys are due to thee;\\*
As souls unbodied, bodies unclothed must be\\
To taste whole joys. Gems which you women use\\
Are like \textit{Atlanta}'s balls, cast in men's views,\\
That when a fool's eye lighteth on a gem,\\
His earthly soul may covet theirs, not them:\\
Like pictures, or like books' gay coverings made\\
For laymen, are all women thus arrayed.\\
Themselves are mystic books, which only we\\
(Whom their imputed grace will dignify)\\
Must see revealed. Then, since that I may know,\\
As liberally as to a midwife show\\
Thyself. Cast all, yea, this white linen hence;\\*
There is no penance due to innocence.\\!

To teach thee, I am naked first. Why than\\*
What needst thou have more covering than a man?
\end{verse}

\subsection{}

\blfootnote{Sir Thomas Wyatt (1503 -- 1542), \cite{norton}. This poem is a translation of Petrarch's \refbook{Rime} 190. The `hind' is often said to stand for Anne Boleyn and `Caesar' for Henry VIII. `Noli me tangere', meaning `Don't touch me', a phrase from the Vulgate (John 20.17).}\settowidth{\versewidth}{    And, graven with diamonds, in letters plain}
\begin{verse}[\versewidth]
Whoso list to hunt, I know where is an hind,\\*
\vin But as for me, alas, I may no more:\\
\vin The vain travail hath wearied me so sore.\\
\vin I am of them that farthest cometh behind.\\
Yet may I by no means my wearied mind\\
\vin Draw from the deer: but as she fleeth afore,\\
\vin Fainting I follow. I leave off therefore,\\
\vin Since in a net I seek to hold the wind.\\
Who list her hunt, I put him out of doubt,\\
\vin As well as I may spend his time in vain:\\
\vin And, graven with diamonds, in letters plain\\
There is written her fair neck round about:\\
\vin {\hoskeroe Noli me tangere}, for \textit{Caesar}'s I am;\\*
\vin And wild for to hold, though I seem tame.
\end{verse}

\subsection{}

\blfootnote{Anonymous, \cite{odq}.}A change is as good as a rest.

\section{}

\subsection{}

\blfootnote{John Dryden, Poet Laureate (1631 -- 1700), \cite{newlove}. These lines are sung in \refbook{Marriage \`a la Mode} IV.2.}\settowidth{\versewidth}{Till at length she cried, Now, my dear, now let us go;}
\begin{verse}[\versewidth]
\vin Whilst \textit{Alexis} lay pressed\\*
\vin In her arms he loved best,\\
With his hands round her neck, and his head on her breast,\\
He found the fierce pleasure too hasty to stay,\\*
And his soul in the tempest just flying away.\\!

\vin When \textit{Celia} saw this,\\*
\vin With a sigh, and a kiss,\\
She cried, O my dear, I am robbed of my bliss;\\
’Tis unkind to your love, and unfaithfully done,\\*
To leave me behind you, and die all alone.\\!

\vin The youth, though in haste,\\*
\vin And breathing his last,\\
In pity died slowly, while she died more fast;\\
Till at length she cried, Now, my dear, now let us go;\\*
Now die, my \textit{Alexis}, and I will die too.\\!

\vin Thus entranced they did lie,\\*
\vin Till \textit{Alexis} did try\\
To recover new breath, that again he might die:\\
Then often they died; but the more they did so,\\*
The nymph died more quick, \& the shepherd more slow.
\end{verse}

\subsection{}

\blfootnote{$\mathbb{R}$ Edmund Spenser (1552 -- 1599), \cite{norton}. This poem would seem to be a reply or epilogue to Petrarch's \refbook{Rime} 190, which was translated by Sir Thomas Wyatt into a sonnet beginning `Whoso list to hunt'.}\settowidth{\versewidth}{Sought not to fly, but fearless still did bide:}
\begin{verse}[\versewidth]
Like as a huntsman after weary chase,\\*
Seeing the game from him escaped away,\\
Sits down to rest him in some shady place,\\
With panting hounds beguil\`{e}d of their prey:\\
So after long pursuit \& vain assay,\\
When I all weary had the chase forsook,\\
The gentle deer returned the selfsame way,\\
Thinking to quench her thirst at the next brook.\\
There she beholding me with milder look,\\
Sought not to fly, but fearless still did bide:\\
Till I in hand her yet \sfrac{$1$}{$2$} trembling took,\\
And with her own goodwill her firmly tied.\\
Strange thing me seemed to see a beast so wild,\\*
So goodly won with her own will beguiled.
\end{verse}

\subsection{}

\blfootnote{Anonymous, \cite{odq}.}Actions speak louder than words.

\section{}

\subsection{}

\blfootnote{The Rev Richard Duke (1658 -- 1711), \cite{newlove}. Dying was frequently used in seventeenth century poetry as a euphemism for reaching orgasm.}\settowidth{\versewidth}{    Close hugs the charmer, and ashamed to yield,}
\begin{verse}[\versewidth]
After the fiercest pangs of hot desire,\\*
\vin Between \textit{Panthea}'s rising breasts,\\
\vin His bending breast \textit{Philander} rests:\\
Though vanquished, yet unknowing to retire,\\
\vin Close hugs the charmer, and ashamed to yield,\\*
\vin Though he has lost the day, yet keeps the field.\\!

When with a sigh the fair \textit{Panthea} said,\\*
\vin What pity 'tis, ye gods, that all\\
\vin The noblest warriors soonest fall!\\
Then with a kiss he gently reared his head,\\
\vin Armed him again to fight, for nobly she\\*
\vin More loved the combat than the victory.\\!

But more enraged, for being beat before,\\*
\vin With all his strength he does prepare\\
\vin More fiercely to renew the war;\\
Nor ceased he till the noble prize he bore:\\
\vin Ev'n her much wondrous courage did surprise;\\*
\vin She hugs the dart that wounded her, \& dies.
\end{verse}

\subsection{}

\blfootnote{`Carpe Diem', William Shakespeare (1564 -- 1616), \cite{treasury}. This song is sung by Feste in \refbook{Twelfth Night} II.3.}\settowidth{\versewidth}{O mistress mine, where are you roaming?}
\begin{verse}[\versewidth]
O mistress mine, where are you roaming?\\*
O stay \& hear! your truelove's coming\\
\vin That can sing both high \& low;\\
Trip no further, pretty sweeting,\\
Journeys end in lovers meeting --\\*
\vin Every wise man's son doth know.\\!

What is love? 'Tis not hereafter;\\*
Present mirth hath present laughter;\\
\vin What's to come is still unsure:\\
In delay there lies no plenty --\\
Then come kiss me, sweet \& 20,\\*
\vin Youth's a stuff will not endure.
\end{verse}

\subsection{}

\blfootnote{Anonymous, \cite{odq}.}All's fair in love and war.

\section{}

\subsection{}

\blfootnote{`Jordan (I)', The Rev George Herbert (1593 -- 1633), \cite{norton}.}\settowidth{\versewidth}{May no lines pass, except they do their duty,}
\begin{verse}[\versewidth]
Who says that fictions only \& false hair\\*
Become a verse? Is there in truth no beauty?\\
Is all good structure in a winding stair?\\
May no lines pass, except they do their duty,\\*
\vin Not to a true, but painted chair?\\!

Is it no verse, except enchanted groves\\*
And sudden arbours shadow coarse-spun lines?\\
Must purling streams refresh a lover's loves?\\
Must all be veiled while he that reads, divines,\\*
\vin Catching the sense at two removes?\\!

Shepherds are honest people; let them sing:\\*
Riddle who list, for me, and pull the prime:\\
I envy no man's nightingale or spring;\\
Nor let them punish me with loss of rhyme,\\*
\vin Who plainly say, My God, my King.
\end{verse}

\subsection{}

\blfootnote{`In an Artist's Studio', Miss Christina Rossetti (1830 -- 1894), \cite{norton}.}\settowidth{\versewidth}{    One selfsame figure sits or walks or leans:}
\begin{verse}[\versewidth]
One face looks out from all his canvases,\\*
\vin One selfsame figure sits or walks or leans:\\
\vin We found her hidden just behind those screens,\\
That mirror gave back all her loveliness.\\
A queen in opal or in ruby dress,\\
\vin A nameless girl in freshest summer-greens,\\
\vin A saint, an angel -- every canvas means\\
The same one meaning, neither more or less.\\
He feeds upon her face by day and night,\\
\vin And she with true kind eyes looks back on him,\\
Fair as the moon and joyful as the light:\\
\vin Not wan with waiting, not with sorrow dim;\\
Not as she is, but was when hope shone bright;\\*
\vin Not as she is, but as she fills his dream.
\end{verse}

\subsection{}

\blfootnote{Anonymous, \cite{odq}.}Appetite comes with eating.

\section{}

\subsection{}

\blfootnote{`A Renunciation', Edward de Vere, 17th Earl of Oxford (1550 -- 1604), \cite{treasury}. The attribution of this poem to Lord Oxford is uncertain.}\settowidth{\versewidth}{And let them fly, fair fools, which way they list?}
\begin{verse}[\versewidth]
If women could be fair, and yet not fond,\\*
\vin Or that their love were firm, not fickle still,\\
I would not marvel that they make men bond\\
\vin By service long to purchase their good will;\\
But when I see how frail those creatures are,\\*
I muse that men forget themselves so far.\\!

To mark the choice they make, \& how they change,\\*
\vin How oft from \textit{Phoebus} they do flee to \textit{Pan};\\
Unsettled still, like haggards wild they range,\\
\vin These gentle birds that fly from man to man;\\
Who would not scorn \& shake them from the fist,\\*
And let them fly, fair fools, which way they list?\\!

Yet for disport we fawn \& flatter both,\\*
\vin To pass the time when nothing else can please,\\
And train them to our lure with subtle oath,\\
\vin Till, weary of their wiles, ourselves we ease;\\
And then we say when we their fancy try,\\*
To play with fools, `O what a fool was I!'
\end{verse}

\subsection{}

\blfootnote{`To Homer', John Keats (1795 -- 1821), \cite{norton}.}\settowidth{\versewidth}{So thou wast blind; but then the veil was rent,}
\begin{verse}[\versewidth]
Standing aloof in giant ignorance,\\*
\vin Of thee I hear and of the Cyclades,\\
As one who sits ashore and longs perchance\\
\vin To visit dolphin-coral in deep seas.\\
So thou wast blind; but then the veil was rent,\\
\vin For \textit{Jove} uncurtained heaven to let thee live,\\
And \textit{Neptune} made for thee a spumy tent,\\
\vin And \textit{Pan} made sing for thee his forest-hive;\\
Aye on the shores of darkness there is light,\\
\vin And precipices show untrodden green,\\
There is a budding morrow in midnight,\\
\vin There is a triple sight in blindness keen;\\
Such seeing hadst thou, as it once befell\\*
To \textit{Dian}, queen of earth, and heaven, and hell.
\end{verse}

\subsection{}

\blfootnote{Anonymous, \cite{odq}.}As you sow, so you reap.

\section{}

\subsection{}

\blfootnote{`To a Skylark', Percy Shelley (1792 -- 1822), \cite{treasury}.}\settowidth{\versewidth}{Our sweetest songs are those that tell of saddest thought.}
\begin{verse}[\versewidth]
\vin \vin Hail to thee, blithe spirit!\\*
\vin \vin \vin Bird thou never wert,\\
\vin \vin That from heaven, or near it,\\
\vin \vin \vin Pourest thy full heart\\*
In profuse strains of unpremeditated art.\\!

\vin \vin Higher still \& higher\\*
\vin \vin \vin From the earth thou springest\\
\vin \vin Like a cloud of fire;\\
\vin \vin \vin The blue deep thou wingest,\\*
And singing still dost soar, and soaring ever singest.\\!

\vin \vin In the golden lightning\\*
\vin \vin \vin Of the sunken sun,\\
\vin \vin O'er which clouds are bright'ning,\\
\vin \vin \vin Thou dost float \& run;\\*
Like an unbodied joy whose race is just begun.\\!

\vin \vin The pale purple even\\*
\vin \vin \vin Melts around thy flight;\\
\vin \vin Like a star of heaven,\\
\vin \vin \vin In the broad daylight\\*
Thou art unseen, but yet I hear thy shrill delight,\\!

\vin \vin Keen as are the arrows\\*
\vin \vin \vin Of that silver sphere,\\
\vin \vin Whose intense lamp narrows\\
\vin \vin \vin In the white dawn clear\\*
Until we hardly see, we feel that it is there.\\!

\vin \vin All the earth \& air\\*
\vin \vin \vin With thy voice is loud,\\
\vin \vin As, when night is bare,\\
\vin \vin \vin From one lonely cloud\\*
The moon rains out her beams, and heaven is overflowed.\\!

\vin \vin What thou art we know not;\\*
\vin \vin \vin What is most like thee?\\
\vin \vin From rainbow clouds there flow not\\
\vin \vin \vin Drops so bright to see\\*
As from thy presence showers a rain of melody.\\!

\vin \vin Like a poet hidden\\*
\vin \vin \vin In the light of thought,\\
\vin \vin Singing hymns unbidden,\\
\vin \vin \vin Till the world is wrought\\*
To sympathy with hopes \& fears it heeded not:\\!

\vin \vin Like a high-born maiden\\*
\vin \vin \vin In a palace tower,\\
\vin \vin Soothing her love-laden\\
\vin \vin \vin Soul in secret hour\\*
With music sweet as love, which overflows her bower:\\!

\vin \vin Like a glow-worm golden\\*
\vin \vin \vin In a dell of dew,\\
\vin \vin Scattering unbeholden\\
\vin \vin \vin Its aereal hue\\*
Among the flowers \& grass, which screen it from the view:\\!

\vin \vin Like a rose embowered\\*
\vin \vin \vin In its own green leaves,\\
\vin \vin By warm winds deflowered,\\
\vin \vin \vin Till the scent it gives\\*
Makes faint with too much sweet those heavy-wing\`{e}d thieves:\\!

\vin \vin Sound of vernal showers\\*
\vin \vin \vin On the twinkling grass,\\
\vin \vin Rain-awakened flowers,\\
\vin \vin \vin All that ever was\\*
Joyous, \& clear, \& fresh, thy music doth surpass.\\!

\vin \vin Teach us, sprite or bird,\\*
\vin \vin \vin What sweet thoughts are thine:\\
\vin \vin I have never heard\\
\vin \vin \vin Praise of love or wine\\*
That panted forth a flood of rapture so divine.\\!

\vin \vin Chorus hymeneal,\\*
\vin \vin \vin Or triumphal chant,\\
\vin \vin Matched with thine would be all\\
\vin \vin \vin But an empty vaunt,\\*
A thing wherein we feel there is some hidden want.\\!

\vin \vin What objects are the fountains\\*
\vin \vin \vin Of thy happy strain?\\
\vin \vin What fields, or waves, or mountains?\\
\vin \vin \vin What shapes of sky or plain?\\*
What love of thine own kind? what ignorance of pain?\\!

\vin \vin With thy clear keen joyance\\*
\vin \vin \vin Languor cannot be:\\
\vin \vin Shadow of annoyance\\
\vin \vin \vin Never came near thee:\\*
Thou lovest: but ne'er knew love's sad satiety.\\!

\vin \vin Waking or asleep,\\*
\vin \vin \vin Thou of death must deem\\
\vin \vin Things more true \& deep\\
\vin \vin \vin Than we mortals dream,\\*
Or how could thy notes flow in such a crystal stream?\\!

\vin \vin We look before \& after,\\*
\vin \vin \vin And pine for what is not:\\
\vin \vin Our sincerest laughter\\
\vin \vin \vin With some pain is fraught;\\*
Our sweetest songs are those that tell of saddest thought.\\!

\vin \vin Yet if we could scorn\\*
\vin \vin \vin Hate, \& pride, \& fear;\\
\vin \vin If we were things born\\
\vin \vin \vin Not to shed a tear,\\*
I know not how thy joy we ever should come near.\\!

\vin \vin Better than all measures\\*
\vin \vin \vin Of delightful sound,\\
\vin \vin Better than all treasures\\
\vin \vin \vin That in books are found,\\*
Thy skill to poet were, thou scorner of the ground!\\!

\vin \vin Teach me \sfrac{$1$}{$2$} the gladness\\*
\vin \vin \vin That thy brain must know,\\
\vin \vin Such harmonious madness\\
\vin \vin \vin From my lips would flow\\*
The world should listen then, as I am listening now.
\end{verse}

\subsection{}

\blfootnote{`An Irish Airman Foresees His Death', William Yeats (1865 -- 1939), \cite{norton}.}\settowidth{\versewidth}{In balance with this life, this death.}
\begin{verse}[\versewidth]
I know that I shall meet my fate\\*
Somewhere among the clouds above;\\
Those that I fight I do not hate;\\
Those that I guard I do not love;\\
My country is \textsc{Kiltartan Cross},\\
My countrymen \textsc{Kiltartan}'s poor,\\
No likely end could bring them loss\\
Or leave them happier than before.\\
Nor law, nor duty bade me fight,\\
Nor public man, nor cheering crowds,\\
A lonely impulse of delight\\
Drove to this tumult in the clouds;\\
I balanced all, brought all to mind,\\
The years to come seemed waste of breath,\\
A waste of breath the years behind\\*
In balance with this life, this death.
\end{verse}

\subsection{}

\blfootnote{Anonymous, \cite{odq}.}Ask no questions and hear no lies.

\section{}

\subsection{}

\blfootnote{Thomas Carew (1595 -- 1640), \cite{newlove}.}\settowidth{\versewidth}{Ask me no more where those stars' light,}
\begin{verse}[\versewidth]
Ask me no more where \textit{Jove} bestows,\\*
When june is past, the fading rose;\\
For in your beauty's orient deep\\*
These flowers, as in their causes, sleep.\\!

Ask me no more whither do stray\\*
The golden atoms of the day;\\
For in pure love heaven did prepare\\*
Those powders to enrich your hair.\\!

Ask me no more whither doth haste\\*
The nightingale, when may is past;\\
For in your sweet dividing throat\\*
She winters, and keeps warm her note.\\!

Ask me no more where those stars' light,\\*
That downwards fall in dead of night;\\
For in your eyes they sit, and there\\*
Fixed become, as in their sphere.\\!

Ask me no more if east or west\\*
The \textit{Phoenix} builds her spicy nest;\\
For unto you at last she flies,\\*
And in your fragrant bosom dies.
\end{verse}

\subsection{}

\blfootnote{`To Lucasta, on Going to the Wars', Col Richard Lovelace (1617 -- 1657), \cite{treasury}.}\settowidth{\versewidth}{True, a new mistress now I chase,}
\begin{verse}[\versewidth]
Tell me not, sweet, I am unkind\\*
\vin That from the nunnery\\
Of thy chaste breast \& quiet mind,\\*
\vin To war \& arms I fly.\\!

True, a new mistress now I chase,\\*
\vin The first foe in the field;\\
And with a stronger faith embrace\\*
\vin A sword, a horse, a shield.\\!

Yet this inconstancy is such\\*
\vin As you too shall adore;\\
I could not love thee, dear, so much,\\*
\vin Loved I not honour more.
\end{verse}

\subsection{}

\blfootnote{Anonymous, \cite{odq}.}Beauty is in the eye of the beholder.

\section{}

\subsection{}

\blfootnote{$\mathbb{R}$ George Chapman (1559 -- 1634), \cite{obev}. This is a translation of Homer's Ἰλιάς XVIII.541-592. Prof Auden wrote his own poem (\refpoem{The Shield of Achilles}) concerning the same portion of Book XVIII.}\settowidth{\versewidth}{Store of white cakes, and mixed the labourers' feast}
\begin{verse}[\versewidth]
He carved besides a soft and fruitful field,\\*
Broad \& thrice new-tilled in that heavenly shield,\\
Where many ploughmen turned up here \& there\\
The earth in furrows, and their sovereign near\\
They strived to work; and every furrow ended\\
A bowl of sweetest wine he still extended\\
To him that first had done, then turned they hand,\\
Desirous to dispatch that piece of land,\\
Deep \& new-eared; black grew the plough with mould\\
Which looked like blackish earth though forged of gold.\\
And this he did with miracle adorn.\\
Then made he grow a field of high-sprung corn,\\
In which did reapers sharpened sickles ply;\\
Others, their handles fall'n confusedly,\\
Laid on the ridge together; others bound\\
Their gathered handfuls to sheaves hard \& round.\\
Their binders were appointed for the place,\\
And at their heels did children glean apace,\\
Whole armfuls to the binders ministering.\\
Amongst all these all silent stood their king.\\
Upon a balk, his sceptre in his hand,\\
Glad at his heart to see his yieldy land.\\
The heralds then the harvest feast prepare,\\
Beneath an oak far off, and for their fare,\\
A mighty ox was slain, and women dressed\\
Store of white cakes, and mixed the labourers' feast\\
In it besides a vine ye might behold\\
Loaded with grapes, the leaves were all of gold.\\
The bunches black \& thick did through it grow\\
And silver props sustained them from below:\\
About the vine an azure dyke was wrought\\
And about it a hedge of tin he brought.\\
One path went through it, through the which did pass\\
The vintagers, when ripe their vintage was.\\
The virgins then, \& youths, childishly wise,\\
For the sweet fruit did painted cups devise,\\
And in a circle bore them dancing round,\\
In midst whereof a boy did sweetly sound\\
His silver harp, and with a piercing voice,\\
Sung a sweet song; when each youth with his choice\\
Triumphing over earth, quick dances treads.\\
A herd of oxen thrusting out their heads\\
And bellowing, from their stalls rushing to feed\\
Near a swift flood, raging and crowned with reed,\\
In gold and tin he carv\`{e}d next the vine\\
Four golden herdsmen following: herd-dogs nine\\
Waiting on them; in head of all the herd,\\
Two lions shook a bull, that bellowing, reared\\
In desperate horror, and was dragged away:\\
The dogs \& youths pursued; but their slain prey,\\
The lions rent out of his spacious hide,\\
And in their entrails did his flesh divide,\\
Lapping his sable blood; the men to fight\\
Set on their dogs in vain that durst not bite,\\
But barked \& backwards flew: he forged beside\\
In a fair vale, a pasture sweet \& wide\\
Of white-fleeced sheep, in which he did impress\\
Sheepcots, sheepfolds \& covered cottages.\\
In this rare shield the famous \textit{Vulcan} cast\\
A dancing mace; like that in ages past,\\
Which in broad \textsc{Knossos} \textit{Daedalus} did dress\\*
For \textit{Ariadne} with the golden tress.
\end{verse}

\subsection{}

\blfootnote{`Upon Westminster Bridge', Dr William Wordsworth, Poet Laureate (1770 -- 1850), \cite{treasury}.}\settowidth{\versewidth}{    Ships, towers, domes, theatres, temples lie}
\begin{verse}[\versewidth]
Earth has not anything to show more fair;\\*
\vin Dull would he be of soul who could pass by\\
\vin A sight so touching in its majesty.\\
This city now doth like a garment wear\\
The beauty of the morning: silent, bare,\\
\vin Ships, towers, domes, theatres, \& temples lie\\
\vin Open unto the fields, \& to the sky --\\
All bright \& glittering in the smokeless air.\\
Never did sun more beautifully steep\\
\vin In his first splendour valley, rock, or hill;\\
Ne'er saw I, never felt, a calm so deep!\\
\vin The river glideth at his own sweet will:\\
Dear God, the very houses seem asleep;\\*
\vin And all that mighty heart is lying still!
\end{verse}

\subsection{}

\blfootnote{Anonymous, \cite{odq}.}Better be envied than pitied.

\section{}

\subsection{}

\blfootnote{`Two Rural Sisters', Charles Cotton (1630 -- 1687), \cite{newlove}.}\settowidth{\versewidth}{    White as blanched almonds, or the falling snow,}
\begin{verse}[\versewidth]
\textit{Alice} is tall \& upright as a pine,\\*
\vin White as blanched almonds, or the falling snow,\\
\vin Sweet as the damask roses when they blow,\\
And doubtless fruitful as the swelling vine.\\
Ripe to be cut, \& ready to be pressed,\\
\vin Her full cheeked beauties very well appear,\\
\vin And a year's fruit she loses every year,\\*
Wanting a man to improve her to the best.\\!

Full fain she would be husbanded, and yet,\\*
Alas, she cannot a fit labourer get\\
\vin To cultivate her own content:\\
Fain she would be (God wot) about her task,\\
And yet (forsooth) she is too proud to ask,\\*
\vin And (which is worse) too modest to consent.\\!

\textit{Margaret} is of humbler stature by the head\\*
\vin Is (as oft falls out with yellow hair)\\
\vin Than her fair sister, yet so much more fair,\\
As her pure white is better mixed with red.\\
This, hotter than the other 10 to one,\\
\vin Longs to be put into her mother's trade,\\
\vin And loud proclaims she lives too long a maid,\\*
Wishing for one t'untie her virgin zone.\\!

She finds virginity a kind of ware,\\*
That's very very troublesome to bear,\\
\vin And being gone, she thinks will ne'er be missed:\\
And yet withal, the girl has so much grace,\\
To call for help I know she wants the face,\\*
\vin Though asked, I know not how she would resist.
\end{verse}

\subsection{}

\blfootnote{Sir William Davenant (1606 -- 1668), \cite{obev}.}\settowidth{\versewidth}{    Who look for day before his mistress wakes.}
\begin{verse}[\versewidth]
The lark now leaves his watery nest\\*
\vin And climbing, shakes his dewy wings;\\
He takes this window for the east;\\
\vin And to implore your light, he sings;\\
Awake, awake. The morn will never rise,\\*
Till she can dress her beauty at your eyes.\\!

The merchant bows unto the seaman's star,\\*
\vin The ploughman from the sun his season takes;\\
But still the lover wonders what they are,\\
\vin Who look for day before his mistress wakes.\\
Awake, awake. Break through your veils of lawn!\\*
Then draw your curtains, and begin the dawn.
\end{verse}

\subsection{}

\blfootnote{Anonymous, \cite{odq}. That is, it's better for two objectionable people to marry each other than for each to take a pleasant spouse.}Better one house spoiled than two.

\section{}

\subsection{}

\blfootnote{`The Indifferent', The Very Rev Dr John Donne (1572 -- 1631), \cite{newlove}.}\settowidth{\versewidth}{She heard not this till now; and that it should be so no more.}
\begin{verse}[\versewidth]
\vin \vin I can love both fair \& brown,\\*
Her whom abundance melts, \& her whom want betrays,\\
Her who loves loneness best, \& her who masks \& plays,\\
\vin Her whom the country formed, \& whom the town,\\
\vin \vin Her who believes, \& her who tries,\\
\vin \vin Her who still weeps with spongy eyes,\\
\vin And her who is dry cork, \& never cries;\\
\vin I can love her, \& her, and you, \& you;\\*
\vin I can love any, so she be not true.\\!

\vin \vin Will no other vice content you?\\*
Will it not serve your turn to do as did your mothers?\\
Or have you all old vices spent, and now would find out others?\\
\vin Or doth a fear that men are true torment you?\\
\vin \vin O we are not; be not you so;\\
\vin \vin Let me, and do you, 20 know.\\
\vin Rob me, but bind me not, and let me go.\\
\vin Must I, who came to travail thorough you,\\*
\vin Grow your fixed subject, because you are true?\\!

\vin \vin \textit{Venus} heard me sigh this song,\\*
And by love's sweetest part, variety, she swore,\\
She heard not this till now; and that it should be so no more.\\
\vin She went, examined, and returned ere long,\\
\vin \vin And said, Alas, some two or three\\
\vin \vin Poor heretics in love there be\\
\vin Which think to 'stablish dangerous constancy.\\
\vin But I have told them, Since you will be true,\\*
\vin You shall be true to them who are false to you.
\end{verse}

\subsection{}

\blfootnote{Sir Thomas Wyatt (1503 -- 1542), \cite{norton}. This poem is a translation of Petrarch's \refbook{Rime} 140. Another translation of the same sonnet was made by the Earl of Surrey.}\settowidth{\versewidth}{The long love, that in my thought doth harbour,}
\begin{verse}[\versewidth]
The long love, that in my thought doth harbour,\\*
\vin And in my heart doth keep his residence,\\
\vin Into my face presseth with bold pretense,\\
\vin And therein campeth, spreading his banner.\\
She that me learneth to love \& suffer,\\
\vin And wills that my trust \& lust's negligence\\
\vin Be reined by reason, shame and reverence,\\
\vin With his hardiness taketh displeasure.\\
Wherewithal, unto the heart's forest he fleeth,\\
\vin Leaving his enterprise with pain \& cry:\\
\vin And there him hideth, and not appeareth.\\
What may I do when my master feareth\\
\vin But in the field with him to live \& die?\\*
\vin For good is the life, ending faithfully.
\end{verse}

\subsection{}

\blfootnote{Anonymous, \cite{odq}.}Better the devil you know.

\section{}

\subsection{}

\blfootnote{Prof Thomas Eliot (1888 -- 1965), \cite{norton}. These are the opening lines of \refbook{The Waste Land}.}\settowidth{\versewidth}{Bin gar keine Russin, stamm' aus Litauen, echt deutsch.}
\begin{verse}[\versewidth]
April is the cruellest month, breeding\\*
Lilacs out of the dead land, mixing\\
Memory and desire, stirring\\
Dull roots with spring rain.\\
Winter kept us warm, covering\\
Earth in forgetful snow, feeding\\
A little life with dried tubers.\\
Summer surprised us, coming over the \textsc{Starnbergersee}\\
With a shower of rain; we stopped in the colonnade,\\
And went on in sunlight, into the \textsc{Hofgarten},\\
And drank coffee, and talked for an hour.\\
{\hoskeroe Bin gar keine Russin, stamm' aus Litauen, echt deutsch.}\\
And when we were children, staying at the Archduke's,\\
My cousin's, he took me out on a sled,\\
And I was frightened. He said, \textit{Marie},\\
\textit{Marie}, hold on tight. And down we went.\\
In the mountains, there you feel free.\\*
I read, much of the night, and go south in the winter.\\!

What are the roots that clutch, what branches grow\\*
Out of this stony rubbish? Son of man,\\
You cannot say, or guess, for you know only\\
A heap of broken images, where the sun beats,\\
And the dead tree gives no shelter, the cricket no relief,\\
And the dry stone no sound of water. Only\\
There is shadow under this red rock,\\
(Come in under the shadow of this red rock),\\
And I will show you something different from either\\
Your shadow at morning striding behind you\\
Or your shadow at evening rising to meet you;\\*
I will show you fear in a handful of dust.
\end{verse}

\subsection{}

\blfootnote{William Blake (1757 -- 1827), \cite{norton}.}\settowidth{\versewidth}{I asked a thief to steal me a peach:}
\begin{verse}[\versewidth]
I ask\`{e}d a thief to steal me a peach:\\*
\vin He turned up his eyes.\\
I asked a lithe lady to lie her down:\\*
\vin Holy \& meek, she cries.\\!

As soon as I went\\*
\vin An angel came:\\
He winked at the thief,\\*
\vin And smiled at the dame;\\!

And without one word said\\*
\vin Had a peach from the tree,\\
And still as a maid\\*
\vin Enjoyed the lady.
\end{verse}

\subsection{}

\blfootnote{Anonymous, \cite{odq}.}Better to wear out than rust out.

\section{}

\subsection{}

\blfootnote{`Arabia', John Falkner (1858 -- 1932), \cite{oxfordlarkin}. The poem has the subtitle: `[David George] Hogarth's \refbook{Penetration of Arabia}'. \P 9. This poem contains a number of intriguingly obscure references, beginning with a roll-call of significant -- though now largely forgotten -- European explorers. Jean Louis Burckhardt (1784 -- 1817) was a Swiss explorer and the first European to set eyes on the city of Petra in over a thousand years. Joseph Hal\'evy (1827 -- 1917) was an Ottoman-French-Jewish orientalist who was most notable for his exploration of the Yemen. Karsten Niebuhr (1733 -- 1815) was the cartographer of the Royal Danish Arabian Expedition, and the only member of that group to return to Europe alive. Ulrich Jasper Seetzen (1767 -- 1811) was murdered as an infidel by his fellow Muslims -- he had undertaken an apparently sincere conversion two years before -- while in search of the lost city that Burckhardt would finally rediscover. George Sadleir (1789 -- 1859) was a captain in the British Army who, in endeavouring (successfully) to deliver a ceremonial sword to an Egyptian commander on behalf of Queen Victoria, inadvertently became the first European to cross the Arabia Peninsula. Jan Jansz Struys (1630 -- 1694) was a Dutch sailor more famous for exploring Russia, but who, as prisoner of war in the Ottoman Empire, must have seen more of the Middle East than most Europeans of his day. The exact Slater being referred to, however, remains unclear. \P 17. The location of Samna is likewise unclear. Is this perhaps an archaic name for -- or a garbled version of -- Sana'a? \P 24. `Zob\"eide' is an archaic romanisation of {\arabicish زبيدة}, now more commonly transliterated as Zubaidah, the granddaughter, niece and wife of three distinct Abbasid caliphs, famous for constructing a series of aqueducts for Mecca and Medina.}\settowidth{\versewidth}{    From the noonday furnace to the purple night.}
\begin{verse}[\versewidth]
Who are these from the strange ineffable places,\\*
\vin From the topaz mountain to the desert of doubt,\\
With the glow of the Yemen full on their faces,\\*
\vin And a breath from the spices of Hadramaut?\\!

Travel-apprentices, travel-indenturers,\\*
\vin Young men, old men, black hair, white,\\
Names to conjure with, wild adventurers,\\*
\vin From the noonday furnace to the purple night.\\!

\textit{Burckhardt}, \textit{Hal\'{e}vy}, \textit{Niebuhr}, \textit{Slater},\\*
\vin Seventeenth, 18th century beys,\\
\textit{Seetzen}, \textit{Sadleir}, \textit{Struys} and later\\*
\vin Down to the long victorian days.\\!

A 1000 miles at the back of \textsc{Aden},\\*
\vin There they had time to think of things;\\
In the outer silence and burnt air laden\\*
\vin With the shadow of death \& a vulture's wings.\\!

There they remembered the last house in \textsc{Samna},\\*
\vin Last of the plane-trees, last shepherd \& flock,\\
Prayed for the heavens to rain down manna,\\*
\vin Prayed for a \textit{Moses} to strike down the rock.\\!

Famine \& fever flagged their forces\\*
\vin Till they died in a dream of ice \& fruit\\
In the long-forgotten watercourses\\*
\vin By the edge of Queen \textit{Zob\"{e}ide}'s route.\\!

They have left the hope of the green oases,\\*
\vin The fear of the bleaching bones \& the pest,\\
They have found the more ineffable places --\\*
\vin {\hoskeroe Allah} has given them rest.
\end{verse}

\subsection{}

\blfootnote{Prof Alfred Housman (1859 -- 1936), \cite{norton}. Prof Housman's `threescore years and ten' is a direct quotation from the King James Version of Psalm 90.10; although, happily, he died at the age of seventy-seven.}\settowidth{\versewidth}{Is hung with bloom along the bough,}
\begin{verse}[\versewidth]
Loveliest of trees, the cherry now\\*
Is hung with bloom along the bough,\\
And stands about the woodland ride\\*
Wearing white for eastertide.\\!

Now, of my threescore years \& 10,\\*
Twenty will not come again,\\
And take from 70 springs a score,\\*
It only leaves me 50 more.\\!

And since to look at things in bloom\\*
Fifty springs are little room,\\
About the woodlands I will go\\*
To see the cherry hung with snow.
\end{verse}

\subsection{}

\blfootnote{Anonymous, \cite{odq}.}Catching's before hanging.

\section{}

\subsection{}

\blfootnote{`The Long White Seam', Miss Jean Ingelow (1820 -- 1897), \cite{obev}.}\settowidth{\versewidth}{Like a shaft of light her voice breaks forth,}
\begin{verse}[\versewidth]
As I came round the harbour buoy,\\*
\vin The lights began to gleam,\\
No wave the land-locked water stirred,\\
\vin The crags were white as cream;\\
And I marked my love by candlelight\\
\vin Sewing her long white seam.\\
\vin \vin It's aye sewing ashore, my dear,\\
\vin \vin \vin Watch and steer at sea,\\
\vin \vin It's reef and furl, and haul the line,\\*
\vin \vin \vin Set sail and think of thee.\\!

I climbed to reach her cottage door;\\*
\vin O sweetly my love sings!\\
Like a shaft of light her voice breaks forth,\\
\vin My soul to meet it springs\\
As the shining water leaped of old,\\
\vin When stirred by angel wings.\\
\vin \vin Aye longing to list anew,\\
\vin \vin \vin Awake and in my dream.\\
\vin \vin But never a song she sang like this,\\*
\vin \vin \vin Sewing her long white seam.\\!

Fair fall the lights, the harbour lights.\\*
\vin That brought me in to thee.\\
And peace drop down on that low roof\\
\vin For the sight that I did see,\\
And the voice, my dear, that rang so clear,\\
\vin All for the love of me.\\
\vin \vin For O, for O with brows bent low\\
\vin \vin \vin By the candle's flickering gleam,\\
\vin \vin Her wedding gown it was she wrought,\\*
\vin \vin \vin Sewing the long white seam.
\end{verse}

\subsection{}

\blfootnote{Fulke Greville, 1st Baron Brooke (1554 -- 1628), \cite{norton}.}\settowidth{\versewidth}{The nurse-life wheat within his green husk growing,}
\begin{verse}[\versewidth]
The nurse-life wheat within his green husk growing,\\*
Flatters our hope, and tickles our desire,\\
Nature's true riches in sweet beauties showing,\\*
Which set all hearts, with labour's love, on fire.\\!

No less fair is the wheat when golden ear\\*
Shows unto hope the joys of near enjoying:\\
Fair \& sweet is the bud, more sweet \& fair\\*
The rose, which proves that time is not destroying.\\!

\textit{Caelica}, your youth, the morning of delight,\\*
Enamelled o'er with beauties white \& red,\\
All sense and thoughts did to belief invite,\\
That love \& glory there are brought to bed:\\
\vin And your ripe year's love-noon; he goes no higher,\\*
\vin Turns all the spirits of man into desire.
\end{verse}

\subsection{}

\blfootnote{Anonymous, \cite{odq}.}Councils of war never fight.

\section{}

\subsection{}

\blfootnote{$\mathbb{R}$ William Shakespeare (1564 -- 1616), \cite{obev}. These lines are spoken by Mark Antony in \refbook{Julius Caesar} III.2.}\settowidth{\versewidth}{Friends, romans, countrymen, lend me your ears.}
\begin{verse}[\versewidth]
Friends, romans, countrymen, lend me your ears.\\*
I come to bury \textit{Caesar}, not to praise him.\\
The evil that men do lives after them;\\
The good is oft interr\`{e}d with their bones;\\
So let it be with \textit{Caesar}. The noble \textit{Brutus}\\
Hath told you \textit{Caesar} was ambitious:\\
If it were so, it was a grievous fault,\\
And grievously hath \textit{Caesar} answered it.\\
Here, under leave of \textit{Brutus} \& the rest --\\
For \textit{Brutus} is an honourable man,\\
So are they all, all honourable men --\\
Come I to speak in \textit{Caesar}'s funeral.\\
He was my friend, faithful \& just to me:\\
But \textit{Brutus} says he was ambitious;\\
And \textit{Brutus} is an honourable man.\\
He hath brought many captives home to \textsc{Rome}\\
Whose ransoms did the general coffers fill:\\
Did this in \textit{Caesar} seem ambitious?\\
When that the poor have cried, \textit{Caesar} hath wept:\\
Ambition should be made of sterner stuff:\\
Yet \textit{Brutus} says he was ambitious;\\
And \textit{Brutus} is an honourable man.\\
You all did see that on the Lupercal\\
I thrice presented him a kingly crown,\\
Which he did thrice refuse: was this ambition?\\
Yet \textit{Brutus} says he was ambitious;\\
And, sure, he is an honourable man.\\
I speak not to disprove what \textit{Brutus} spoke,\\
But here I am to speak what I do know.\\
You all did love him once, not without cause:\\
What cause withholds you then, to mourn for him?\\
O judgment! Thou art fled to brutish beasts,\\
And men have lost their reason. Bear with me;\\
My heart is in the coffin there with \textit{Caesar},\\*
And I must pause till it come back to me.
\end{verse}

\subsection{}

\blfootnote{Henry Howard, Earl of Surrey (1517 -- 1547), \cite{treasury}. \P 1. Where the Almanacker has put `sweet', the original reads `soote', which means the same thing. \P 4. The word `turtle' in this context means ``turtledove''. Where the Almanacker has put `mate', the original reads `make', which means the same thing.}\settowidth{\versewidth}{The sweet season, that bud bloom forth brings,}
\begin{verse}[\versewidth]
The sweet season, that bud \& bloom forth brings,\\*
With green hath clad the hill and eke the vale;\\
The nightingale with feathers new she sings;\\
The turtle to her mate hath told her tale.\\
Summer is come, for every spray now springs;\\
The hart hath hung his old head on the pale;\\
The buck in brake his winter coat he flings;\\
The fishes float with new-repair\`{e}d scale;\\
The adder all her slough away she slings;\\
The swift swallows pursueth the flies small;\\
The busy bee her honey now she mings.\\
Winter is worn, that was the flowers' bale.\\
And thus I see among these pleasant things,\\*
Each care decays, and yet my sorrow springs.
\end{verse}

\subsection{}

\blfootnote{Anonymous, \cite{odq}.}Dead men tell no tales.

\section{}

\subsection{}

\blfootnote{$\mathbb{R}$ William Shakespeare (1564 -- 1616), \cite{obev}. It is with these lines that Orsino opens \refbook{Twelfth Night}.}\settowidth{\versewidth}{Even in a minute: so full of shapes is fancy}
\begin{verse}[\versewidth]
If music be the food of love, play on;\\*
Give me excess of it, that, surfeiting,\\
The appetite may sicken, and so die.\\
That strain again! It had a dying fall:\\
O it came o'er my ear like the sweet south,\\
That breathes upon a bank of violets,\\
Stealing \& giving odour! Enough; no more:\\
'Tis not so sweet now as it was before.\\
O spirit of love! How quick and fresh art thou,\\
That, notwithstanding thy capacity\\
Receiveth as the sea; nought enters there,\\
Of what validity \& pitch soe'er,\\
But falls into abatement \& low price,\\
Even in a minute: so full of shapes is fancy\\*
That it alone is high fantastical.
\end{verse}

\subsection{}

\blfootnote{William Shakespeare (1564 -- 1616), \cite{treasury}. This song is sung by Amiens in \refbook{As You Like It} II.5.}\settowidth{\versewidth}{And pleased with what he gets,}
\begin{verse}[\versewidth]
Under the greenwood tree,\\*
Who loves to lie with me,\\
And turn his merry note\\
Unto the sweet bird's throat,\\
Come hither, come hither, come hither:\\
\vin \vin Here shall he see\\
\vin \vin No enemy\\*
But winter \& rough weather.\\!

Who doth ambition shun,\\*
And loves to live i' the sun,\\
Seeking the food he eats,\\
And pleased with what he gets,\\
Come hither, come hither, come hither:\\
\vin \vin Here shall he see\\
\vin \vin No enemy\\*
But winter \& rough weather.
\end{verse}

\subsection{}

\blfootnote{Anonymous, \cite{odq}.}Don't cry before you're hurt.

\section{}

\subsection{}

\blfootnote{$\mathbb{R}$ William Shakespeare (1564 -- 1616), \cite{obev}. These are the opening lines of \refbook{Richard III}, spoken by the eponymous villain, who at this time is Duke of Gloucester. \P 2. The sun in splendour was, with the white rose, one of the symbols of the House of York. There is also a pun here, since Edward IV, the man being praised, was the eldest surviving son of Richard Plantagenet, 3rd Duke of York.}\settowidth{\versewidth}{Grim-visaged war hath smoothed his wrinkled front;}
\begin{verse}[\versewidth]
Now is the winter of our discontent\\*
Made glorious summer by this sun of \textit{York};\\
And all the clouds that loured upon our house\\
In the deep bosom of the ocean buried.\\
Now are our brows bound with victorious wreaths;\\
Our bruised arms hung up for monuments;\\
Our stern alarums changed to merry meetings,\\
Our dreadful marches to delightful measures.\\
Grim-visaged war hath smoothed his wrinkled front;\\
And now, instead of mounting barded steeds\\
To fright the souls of fearful adversaries,\\
He capers nimbly in a lady's chamber\\
To the lascivious pleasing of a lute.\\
But I, that am not shaped for sportive tricks,\\
Nor made to court an amorous looking-glass;\\
I, that am rudely stamped, and want love's majesty\\
To strut before a wanton ambling nymph;\\
I, that am curtailed of this fair proportion,\\
Cheated of feature by dissembling nature,\\
Deformed, unfinished, sent before my time\\
Into this breathing world, scarce \sfrac{$1$}{$2$} made up,\\
And that so lamely \& unfashionable\\
That dogs bark at me as I halt by them;\\
Why, I, in this weak piping time of peace,\\
Have no delight to pass away the time,\\
Unless to spy my shadow in the sun\\
And descant on mine own deformity:\\
And therefore, since I cannot prove a lover,\\
To entertain these fair well-spoken days,\\
I am determined to prove a villain\\
And hate the idle pleasures of these days.\\
Plots have I laid, inductions dangerous,\\
By drunken prophecies, libels \& dreams,\\
To set my brother \textit{Clarence} and the King\\
In deadly hate the one against the other:\\
And if King \textit{Edward} be as true \& just\\
As I am subtle, false \& treacherous,\\
This day should \textit{Clarence} closely be mewed up,\\
About a prophecy, which says that \textit{G}\\
Of \textit{Edward}'s heirs the murderer shall be.\\*
Dive, thoughts, down to my soul: here \textit{Clarence} comes.
\end{verse}

\subsection{}

\blfootnote{`Echo', Miss Christina Rossetti (1830 -- 1894), \cite{norton}. There are two more verses to this poem, but the Almanacker finds them much inferior to the first.}\settowidth{\versewidth}{    Come in the speaking silence of a dream;}
\begin{verse}[\versewidth]
Come to me in the silence of the night;\\*
\vin Come in the speaking silence of a dream;\\
Come with soft rounded cheeks and eyes as bright\\
\vin As sunlight on a stream;\\
\vin \vin Come back in tears,\\*
O memory, hope, love of finished years.
\end{verse}

\subsection{}

\blfootnote{Robert Frost, Poet Laureate of Vermont (1874 -- 1963), \cite{norton}. This quotation comes at the very end of Frost's \refpoem{Mending Wall}.}Good fences make good neighbours.

\chapter{Sectilis}

\section{}

\subsection{}

\blfootnote{William Blake (1757 -- 1827), \cite{blakea}.}\settowidth{\versewidth}{    But, Kitty, I better love thee;}
\begin{verse}[\versewidth]
I love the jocund dance,\\*
\vin The softly-breathing song,\\
Where innocent eyes do glance,\\*
\vin And where lisps the maiden's tongue.\\!

I love the laughing vale;\\*
\vin I love the echoing hill,\\
Where mirth does never fail,\\*
\vin And the jolly swain laughs his fill.\\!

I love the pleasant cot,\\*
\vin I love the innocent bower,\\
Where white \& brown is our lot,\\*
\vin Or fruit in the midday hour.\\!

I love the oaken seat\\*
\vin Beneath the oaken tree,\\
Where all the old villagers meet,\\*
\vin And laugh our sports to see.\\!

I love our neighbours all,\\*
\vin But, \textit{Kitty}, I better love thee;\\
And love them I ever shall;\\*
\vin But thou art all to me.
\end{verse}

\subsection{}

\blfootnote{$\mathbb{R}$ `Ego Sum Vitis', The Rev Dr William Alabaster (1568 -- 1640), \cite{obev}. The title is taken from the Vulgate of John 14.6.}\settowidth{\versewidth}{    Whose leaves are intertwist with love entire,}
\begin{verse}[\versewidth]
Now that the midday heat doth scorch my shame\\*
\vin With lightning of fond lust, I will retire\\
\vin Under this vine whose arms with wandering spire\\
Do climb upon the cross, and on the same\\
Devise a cool repose from lawless flame,\\
\vin Whose leaves are intertwist with love entire,\\
\vin That envy's eye cannot transfuse her fire,\\
But is rebated on the shady frame;\\
\vin And youthful vigour from the leav\`{e}d tier,\\
\vin Doth stream upon my soul a new desire.\\
List, list, the ditties of sublim\`{e}d fame,\\
\vin Which in the closet of those leaves the choir\\
Of heavenly birds do warble to his name.\\*
Or where was I that was not where I am?
\end{verse}

\subsection{}

\blfootnote{Anonymous, \cite{odq}. The \refbook{Oxford Dictionary of Quotations} lists this saying under `Proverbs'. Dryden provides an interesting variation in the dedication to his translation of the \refbook{Aeneid}: `We must beat the iron while it is hot, but we may polish it at leisure.'}Strike while the iron is hot.

\section{}

\subsection{}

\blfootnote{`The Camel Rider', Wilfrid Blunt (1840 -- 1922), \cite{faber20th}.}\settowidth{\versewidth}{And the white broom which bees had found, the wild bees' brood.}
\begin{verse}[\versewidth]
There is no thing in all the world but love,\\*
No jubilant thing of sun or shade worth one sad tear.\\
Why dost thou ask my lips to fashion songs\\*
Other than this, my song of love to thee?\\!

See where I lie and pluck the thorns of grief,\\*
Dust on my head and fire, as one who mourns his slain.\\
Are they not slain, my treasures of dear peace?\\*
This their red burial is, sand heaped on sand.\\!

Here came I in the morning of my joys.\\*
Before the dawn was born, through the dark downs I rode.\\
The low stars led me on as with a voice,\\*
Stars of the scorpion's tail in the deep Ssouth.\\!

Sighing I came, and scattering wide the sand.\\*
No need had I to urge her speed with hand or heel,\\
The creature I bestrode. She knew my haste,\\*
And knew the road I sought, the road to thee.\\!

Jangling her bells aloud in wantonness,\\*
And sighing soft, she too, her sighs to my soul's sighs.\\
Behind us the wind followed thick with scents\\*
Of incense blossoms \& the dews of night.\\!

The thorn trees caught at us with their crook'd hands;\\*
The hills in blackness hemmed us in and hid the road;\\
The spectres of the desert howled and warned;\\*
I heeded nothing of their words of woe.\\!

Thus till the dawn I sped in my desire,\\*
Breasting the ridges, slope on slope, till morning broke;\\
And lo, the sun revealed to me no sign,\\*
And lo, the day was widowed of my hope.\\!

Where are the tents of pleasure \& dear love,\\*
Set in the vale of thyme, where winds in spring are fain?\\
The highways of the valley, where they stood\\*
Strong in their flocks, are there. But where are they?\\!

The plain was dumb, as emptied of all voice;\\*
No bleat of herds, no camels roaring far below\\
Told of their presence in the pastures void,\\*
Of the waste places which had been their homes.\\!

I climbed down from my watch-tower of the rocks,\\*
To where the tamarisks grow, \& the dwarf palms, alarmed.\\
I called them with my voice, as the deer calls,\\*
Whose young the wolves have hunted from their place.\\!

I sought them in the foldings of the hill,\\*
In the deep hollows shut with rocks, where no winds blow.\\
I sought their footstep under the tall cliffs,\\*
Shut from the storms, where the first lambs are born.\\!

The tamarisk boughs had blossomed in the night,\\*
And the white broom which bees had found, the wild bees' brood.\\
But no dear signal told me of their life,\\*
No spray was torn in all that world of flowers.\\!

Where are the tents of pleasure \& dear love,\\*
For which my soul took ease for its delight in spring,\\
The black tents of her people beautiful\\*
Beyond the beauty of the sons of kings?\\!

The wind of war has swept them from their place,\\*
Scattering them wide as quails, whom the hawk's hate pursues;\\
The terror of the sword importunate\\*
Was at their backs, nor spared them as they flew.\\!

The summer wind has passed upon their fields;\\*
The rain has purged their hearth-stones, and made smooth their floors;\\
Low in the valley lie their broken spears,\\*
And the white bones which are their tale forlorn.\\!

Where are the sons of Saba in the south,\\*
The men of mirth \& pride to whom my songs were sung,\\
The kinsmen of her soul who is my soul,\\*
The brethren of her beauty whom I love?\\!

She mounted her tall camel in the waste,\\*
Loading it high for flight with her most precious things;\\
She went forth weeping in the wilderness,\\*
Alone with fear on that far night of ill.\\!

She fled mistrusting, as the wild roe flees,\\*
Turning her eyes behind her, while fear fled before;\\
No other refuge knew she than her speed,\\*
And the black land that lies where night is born.\\!

Under what canopy of sulphurous heaven,\\*
Dark with the thunderclouds unloosing their mad tongues,\\
Didst thou lie down aweary of thy burden,\\*
In that dread place of silence thou hadst won?\\!

Close to what shelter of what naked rocks,\\*
Carved with what names of terror of what kings of old,\\
Near to what monstrous shapes unmerciful,\\*
Watching thy death, didst thou give up thy soul?\\!

Or dost thou live by some forgotten well,\\*
Waiting thy day of ransom to return and smile,\\
As the birds come when spring is in the heaven,\\*
And dost thou watch me near while I am blind?\\!

Blind in my tears, because I only weep,\\*
Kindling my soul to fire because I mourn my slain,\\
My kindred slain, and thee, \& my dear peace,\\*
Making their burial thus, sand heaped on sand.\\!

For see, there nothing is in all the world\\*
But only love worth any strife or song or tear.\\
Ask me not then to sing or fashion songs\\*
Other than this, my song of love to thee.
\end{verse}

\subsection{}

\blfootnote{William Blake (1757 -- 1827), \cite{obev}. This poem constitutes one of the middle sections of \refbook{Visions of the Daughters of Albion}.}\settowidth{\versewidth}{Where she gets poison; and the winged eagle why he loves the sun;}
\begin{verse}[\versewidth]
With what sense is it that the chicken shuns the ravenous hawk?\\*
With what sense does the tame pigeon measure out the expanse?\\
With what sense does the bee form cells? Have not the mouse \& frog\\
Eyes \& ears \& sense of touch? Yet are their habitations\\
And their pursuits as different as their forms \& as their joy.\\
Ask the wild ass why he refuses burdens, and the meek camel\\
Why he loves man: is it because of eye, ear, mouth or skin,\\
Or breathing nostrils? No: for these the wolf \& tyger have.\\
Ask the blind worm the secrets of the grave and why her spires\\
Love to curl around the bones of death: and ask the ravenous snake\\
Where she gets poison; and the winged eagle why he loves the sun;\\*
And then tell me the thoughts of man, that have been hid of old.
\end{verse}

\subsection{}

\blfootnote{Francis Bacon, Viscount St Alban (1561 -- 1626), \cite{odq}.}He that will not apply new remedies must expect new evils.

\section{}

\subsection{}

\blfootnote{`Home Thoughts, from Abroad', Robert Browning (1828 -- 1889), \cite{norton}.}\settowidth{\versewidth}{Blossoms and dewdrops -- at the bent spray's edge --}
\begin{verse}[\versewidth]
O to be in England\\*
Now that april's there,\\
And whoever wakes in England\\
Sees, some morning, unaware,\\
That the lowest boughs \& the brushwood sheaf\\
Round the elm-tree bole are in tiny leaf,\\
While the chaffinch sings on the orchard bough\\*
In England -- now!\\!

And after april, when may follows,\\*
And the whitethroat builds, and all the swallows!\\
Hark, where my blossomed pear-tree in the hedge\\
Leans to the field and scatters on the clover\\
Blossoms and dewdrops -- at the bent spray's edge --\\
That's the wise thrush; he sings each song twice over,\\
Lest you should think he never could recapture\\
The first fine careless rapture!\\
And though the fields look rough with hoary dew,\\
All will be gay when noontide wakes anew\\
The buttercups, the little children's dower --\\*
Far brighter than this gaudy melon-flower!
\end{verse}

\subsection{}

\blfootnote{$\mathbb{R}$ George Chapman (1559 -- 1634), \cite{pbev}. These four lines are taken from Chapman's poem \refpoem{Ovid's Banquet of Sense}, in which the English poet imagines the Roman slipping into one of Augustus' palace gardens and watching Julia the Elder take a bath. Chapman identifies this Julia with Corinna, the heroine of Ovid's \refbook{Amores}.}\settowidth{\versewidth}{And thus she sung, all naked as she sat,}
\begin{verse}[\versewidth]
And thus she sung, all naked as she sat,\\*
Laying the happy lute upon her thigh,\\
Not thinking any near to wonder at\\*
The bliss of her sweet breasts' divinity.
\end{verse}

\subsection{}

\blfootnote{Francis Bacon, Viscount St Alban (1561 -- 1626), \cite{odq}.}Hope is a good breakfast, but it is a bad supper.

\section{}

\subsection{}

\blfootnote{`My Last Duchess', Robert Browning (1828 -- 1889), \cite{obev}.}\settowidth{\versewidth}{Heart, fear nothing, for, heart, thou shalt find her --}
\begin{verse}[\versewidth]
Room after room,\\*
I hunt the house through\\
We inhabit together.\\
Heart, fear nothing, for, heart, thou shalt find her --\\
Next time, herself! -- not the trouble behind her\\
Left in the curtain, the couch's perfume!\\
As she brushed it, the cornice-wreath blossomed anew:\\*
Yon looking-glass gleamed at the wave of her feather.\\!

Yet the day wears,\\*
And door succeeds door;\\
I try the fresh fortune --\\
Range the wide house from the wing to the centre.\\
Still the same chance! She goes out as I enter.\\
Spend my whole day in the quest -- who cares?\\
But 'tis twilight, you see -- with such suites to explore,\\*
Such closets to search, such alcoves to importune!
\end{verse}

\subsection{}

\blfootnote{`Peggy Browne', Austin Clarke (1896 -- 1974), \cite{obev}. These lines are a translation of an Irish song by the eighteenth century harper Turlough O'Carolan \P 6. The foundations of Westport House were laid by one Col John Browne, whose descendants, eleven of whom held the title Marquess of Sligo, continue to possess the house into the twenty-first century.}\settowidth{\versewidth}{The dark-haired girl, who holds my thought entirely}
\begin{verse}[\versewidth]
The dark-haired girl, who holds my thought entirely\\*
Yet keeps me from her arms and what I desire,\\
Will never take my word for he is proud\\*
And none may have his way with \textit{Peggy Browne}.\\!

Often I dream that I am in the woods\\*
At \textsc{Westport House}. She strays alone, blue-hooded,\\
Then lifts her flounces, hurries from a shower,\\*
But sunlight stays all day with \textit{Peggy Browne}.\\!

Her voice is music, every little echo\\*
My pleasure and O her shapely breasts, I know,\\
Are white as her own milk, when taffeta gown\\*
Is let out, inch by inch, for \textit{Peggy Browne}.\\!

A lawless dream comes to me in the night time,\\*
That we are stretching together side by side;\\
Nothing I want to do can make her frown.\\*
I wake alone, sighing for \textit{Peggy Browne}.
\end{verse}

\subsection{}

\blfootnote{`On His Books', Hilaire Belloc (1870 -- 1953), \cite{odq}.}\settowidth{\versewidth}{When I am dead, I hope it may be said:}
\begin{verse}[\versewidth]
When I am dead, I hope it may be said:\\*
His sins were scarlet, but his books were read.
\end{verse}

\section{}

\subsection{}

\blfootnote{$\mathbb{R}$ Dr Thomas Campion (1567 -- 1620), \cite{obev}. \P 24. Dione's oracle at Dodona was said to have been founded at the command of a black dove. \P 35. Avernus is a volcanic crater in Campania, believed by the Romans to be an entrance into Hades; an obscure proverb states that women who die unwed will have to lead some sort of procession of primates through this netherworld; for instance, one reads in \textit{The London Prodigal} (a Jacobean play of uncertain authorship) that, `'Tis an old proverb, and you know it well,/ That women dying maids lead apes in hell.'}\settowidth{\versewidth}{Bids you increase that loving humour more:}
\begin{verse}[\versewidth]
Hark, all you ladies that do sleep;\\*
The fairy queen \textit{Proserpina}\\
Bids you awake and pity them that weep.\\
\vin You may do in the dark\\
\vin \vin What the day doth forbid;\\
\vin Fear not the dogs that bark;\\*
\vin \vin Night will have all hid.\\!

But if you let your lovers moan,\\*
The fairy queen \textit{Proserpina}\\
Will send abroad her fairies ev'ry one,\\
\vin That shall pinch black \& blue\\
\vin \vin Your white hands \& fair arms\\
\vin That did not kindly rue\\*
\vin \vin Your paramour's harms.\\!

In myrtle arbours on the downs\\*
The fairy queen \textit{Proserpina},\\
This night by moonshine leading merry rounds\\
\vin Holds a watch with sweet love,\\
\vin \vin Down the dale, up the hill;\\
\vin No plaints or groans may move\\*
\vin \vin Their holy vigil.\\!

All you that will hold watch with love,\\*
The fairy queen \textit{Proserpina}\\
Will make you fairer than \textit{Dione}'s dove;\\
\vin Roses red, lilies white,\\
\vin \vin And the clear damask hue,\\
\vin Shall on your cheeks alight:\\*
\vin \vin Love will adorn you.\\!

All you that love, or loved before,\\*
The fairy queen \textit{Proserpina}\\
Bids you increase that loving humour more:\\
\vin They that yet have not fed\\
\vin \vin On delight amorous,\\
\vin She vows that they shall lead\\*
\vin \vin Apes in \textsc{Avernus}.
\end{verse}

\subsection{}

\blfootnote{`The Planter's Daughter', Austin Clarke (1896 -- 1974), \cite{oxfordlarkin}.}\settowidth{\versewidth}{For the house of the planter}
\begin{verse}[\versewidth]
When night stirred at sea\\*
And the fire brought a crowd in,\\
They say that her beauty\\
Was music in mouth\\
And few in the candlelight\\
Thought her too proud,\\
For the house of the planter\\*
Is known by the trees.\\!

Men that had seen her\\*
Drank deep and were silent;\\
The women were speaking\\
Wherever she went --\\
As a bell that is rung\\
Or a wonder told shyly,\\
And O she was the sunday\\*
In every week.
\end{verse}

\subsection{}

\blfootnote{William Blake (1757 -- 1827), \cite{blakea}. This is one of Blake's `Proverbs of Hell' from \refbook{The Marriage of Heaven and Hell}.}A fool sees not the same tree that a wise man sees.

\section{}

\subsection{}

\blfootnote{`First Love', John Clare (1793 -- 1864), \cite{norton}. This poem describes Clare's first meeting with Mary Joyce, the local beauty he could never possess.}\settowidth{\versewidth}{Her face it bloomed like a sweet flower}
\begin{verse}[\versewidth]
I ne'er was struck before that hour\\*
\vin With love so sudden \& so sweet,\\
Her face it bloomed like a sweet flower\\
\vin And stole my heart away complete.\\
My face turned pale as deadly pale,\\
\vin My legs refused to walk away,\\
And when she looked, what could I ail?\\*
\vin My life \& all seemed turned to clay.\\!

And then my blood rushed to my face\\*
\vin And took my eyesight quite away,\\
The trees \& bushes round the place\\
\vin Seemed midnight at noonday.\\
I could not see a single thing,\\
\vin Words from my eyes did start --\\
They spoke as chords do from the string,\\*
\vin And blood burnt round my heart.\\!

Are flowers the winter's choice?\\*
\vin Is love's bed always snow?\\
She seemed to hear my silent voice,\\
\vin Not love's appeals to know.\\
I never saw so sweet a face\\
\vin As that I stood before.\\
My heart has left its dwelling-place\\*
\vin And can return no more.
\end{verse}

\subsection{}

\blfootnote{William Congreve (1670 -- 1729), \cite{newlove}.}\settowidth{\versewidth}{    When she believes I'll leave her.}
\begin{verse}[\versewidth]
Pious \textit{Selinda} goes to prayers\\*
\vin If I but ask the favour;\\
And yet the tender fool's in tears\\*
\vin When she believes I'll leave her.\\!

Would I were free from this restraint,\\*
\vin Or else had hopes to win her;\\
Would she could make of me a saint,\\*
\vin Or I of her a sinner.
\end{verse}

\subsection{}

\blfootnote{William Blake (1757 -- 1827), \cite{blakea}. This is one of Blake's `Proverbs of Hell' from \refbook{The Marriage of Heaven and Hell}.}Dip him in the river who loves water.

\section{}

\subsection{}

\blfootnote{The Rev Charles Dodgson (1832 -- 1898), \cite{obev}. These lines have been parodied many times, and yet they themselves were originally intended as a pardody of Robert Southey's \refpoem{The Old Man's Comforts and How He Gained Them}.}\settowidth{\versewidth}{By the use of this ointment -- one shilling the box --}
\begin{verse}[\versewidth]
You are old, Father \textit{William}, the young man said,\\*
\vin And your hair has become very white;\\
And yet you incessantly stand on your head --\\*
\vin Do you think, at your age, it is right?\\!

In my youth, Father \textit{William} replied to his son,\\*
\vin I feared it might injure the brain;\\
But, now that I'm perfectly sure I have none,\\*
\vin Why, I do it again \& again.\\!

You are old, said the youth, as I mentioned before,\\*
\vin And have grown most uncommonly fat;\\
Yet you turned a back-somersault in at the door --\\*
\vin Pray, what is the reason of that?\\!

In my youth, said the sage, as he shook his grey locks,\\*
\vin I kept all my limbs very supple\\
By the use of this ointment -- one shilling the box --\\*
\vin Allow me to sell you a couple?\\!

You are old, said the youth, and your jaws are too weak\\*
\vin For anything tougher than suet;\\
Yet you finished the goose, with the bones \& the beak --\\*
\vin Pray, how did you manage to do it?\\!

In my youth, said his father, I took to the law,\\*
\vin And argued each case with my wife;\\
And the muscular strength, which it gave to my jaw,\\*
\vin Has lasted the rest of my life.\\!

You are old, said the youth. One would hardly suppose\\*
\vin That your eye was as steady as ever;\\
Yet you balanced an eel on the end of your nose --\\*
\vin What made you so awfully clever?\\!

I have answered three questions, and that is enough,\\*
\vin Said his father; don't give yourself airs!\\
Do you think I can listen all day to such stuff?\\*
\vin Be off, or I'll kick you downstairs!'
\end{verse}

\subsection{}

\blfootnote{$\mathbb{R}$ Abraham Cowley (1618 -- 1667), \cite{pbev}. This is the first verse of a poem called \refpoem{The Change}.}\settowidth{\versewidth}{In all her outward parts love's always seen;}
\begin{verse}[\versewidth]
Love in her sunny eyes does basking play;\\*
\vin Love walks the pleasant mazes of her hair;\\
Love does on both her lips for ever stray\\
\vin And sows and reaps a 1000 kisses there.\\
In all her outward parts love's always seen;\\*
\vin \vin But O he never went within.
\end{verse}

\subsection{}

\blfootnote{William Blake (1757 -- 1827), \cite{blakea}. This is one of Blake's `Proverbs of Hell' from \refbook{The Marriage of Heaven and Hell}.}He whose face gives no light shall never become a star.

\section{}

\subsection{}

\blfootnote{`The Sun Rising', The Very Rev Dr John Donne (1572 -- 1631), \cite{norton}.}\settowidth{\versewidth}{    Go tell court huntsmen that the king will ride;}
\begin{verse}[\versewidth]
\vin \vin Busy old fool, unruly sun,\\*
\vin \vin Why dost thou thus,\\
Through windows, and through curtains, call on us?\\
Must to thy motions lovers' seasons run?\\
\vin \vin Saucy pedantic wretch, go chide\\
\vin \vin Late school boys \& sour prentices;\\
\vin Go tell court huntsmen that the king will ride;\\
\vin Call country ants to harvest offices.\\
Love, all alike, no season knows nor clime,\\*
Nor hours, days, months, which are the rags of time.\\!

\vin \vin Thy beams, so reverend \& strong\\*
\vin \vin Why shouldst thou think?\\
I could eclipse \& cloud them with a wink,\\
But that I would not lose her sight so long;\\
\vin \vin If her eyes have not blinded thine,\\
\vin \vin Look, and tomorrow late, tell me,\\
\vin Whether both th' Indias of spice \& mine\\
\vin Be where thou leftst them, or lie here with me.\\
Ask for those kings whom thou saw'st yesterday,\\*
And thou shalt hear all here in one bed lay.\\!

\vin \vin She's all states, and all princes, I;\\*
\vin \vin Nothing else is.\\
Princes do but play us; compared to this,\\
All honour's mimic, all wealth alchemy.\\
\vin \vin Thou, sun, art \sfrac{$1$}{$2$} as happy as we,\\
\vin \vin In that the world's contracted thus.\\
\vin Thine age asks ease, and since thy duties be\\
\vin To warm the world, that's done in warming us.\\
Shine here to us, and thou art everywhere;\\*
This bed thy centre is, these walls thy sphere.
\end{verse}

\subsection{}

\blfootnote{`Still-Life', Mrs Elizabeth Daryush (1887 -- 1977), \cite{obev}.}\settowidth{\versewidth}{    Her delicate desires with all that's good,}
\begin{verse}[\versewidth]
Through the open french window the warm sun\\*
\vin Lights up the polished breakfast table, laid\\
Round a bowl of crimson roses, for one --\\
\vin A service of worcester porcelain, arrayed\\
Near it a melon, peaches, figs, small hot\\
\vin Rolls in a napkin, fairy rack of toast,\\
Butter in ice, high silver coffee pot,\\*
\vin And, heaped on a salver, the morning's post.\\!

She comes over the lawn, the young heiress,\\*
\vin From her early walk in her garden wood,\\
Feeling that life's a table set to bless\\*
\vin Her delicate desires with all that's good,\\!

That even the unopened future lies\\*
Like a love letter, full of sweet surprise.
\end{verse}

\subsection{}

\blfootnote{William Blake (1757 -- 1827), \cite{blakea}. This is one of Blake's `Proverbs of Hell' from \refbook{The Marriage of Heaven and Hell}.}If the fool would persist in his folly he would be wise.

\section{}

\subsection{}

\blfootnote{$\mathbb{R}$ Sir John Harington (1560 -- 1612), \cite{obev}. A good chunk of explanation is perhaps helpful with respect to this poem. Firstly, these lines are from Sir John's translation of Ludovico Ariosto's \refbook{Orlando Furioso}, being XXVIII v.55-65. Before our story begins, two friends, Giocondo and Astolfo (who also happens to be King of Lombardy), go on a kind of lads' holiday, seducing numerous women -- they aim for a thousand each -- in order to console themselves over their wives' infidelity. Believing no single man of being able to satisfy a woman's lust, they form a kind of polyandrous marriage with an innkeeper's daughter, Fiametta, who sleeps each night between the two men. In the lines here printed, Fiametta happens to bump into her Greek childhood sweetheart. The two wish to start a life together; however, Fiametta, being taken, comes up with a consolation prize. She tells her Greek paramour to sneak into her room and make love to her, and tricks both Giacondo and Astolfo into believing that the other is enjoying her instead. After these lines end, the two friends discover Fiametta's infidelity after the fact, but take being re-cuckolded with a surprisingly robust sense of humour, and allow her and the Greek to get married.}\settowidth{\versewidth}{With these two lords wilt thou from Spain be banished.}
\begin{verse}[\versewidth]
The masters go abroad to view the town,\\*
\vin And first the churches for devotions' sake;\\
And then the monuments of most renown,\\
\vin As travellers a common custom take:\\
The girl within the chamber sate her down;\\
\vin The men are busied; some the beds do make;\\
Some care to dress their wearied horse, and some\\*
Make ready meat against their masters come.\\!

In this same house the girl a greek had spied,\\*
\vin That in her father's house a boy had been,\\
And slept full often sweetly by her side,\\
\vin And much good sport had pass\`{e}d them between;\\
Yet fearing lest their love should be descried,\\
\vin In open talk they durst not to be seen,\\
But when by hap the pages down were gone,\\*
Old love renewed and thus they talk thereon.\\!

The greek demands her whither she was going,\\*
\vin And which of these two great estates her keeps.\\
She told them all; she needs no further wooing,\\
\vin And how a-night between them both she sleeps:\\
`Ah!' quoth the greek. `Thou tellest my undoing,\\
\vin My dear \textit{Fiametta}, and with that he weeps;\\
With these two lords wilt thou from Spain be banished.\\*
Are all my hopes thus into nothing vanished?\\!

`My sweet designments turn\`{e}d are to sour;\\*
\vin My service long finds little recompense;\\
I made a stock according to my power,\\
\vin By hoarding up my wages, and the pence\\
That guests did give, that came in lucky hour;\\
\vin I meant ere long to have departed hence,\\
And to have asked thy sires good will to marry thee,\\*
And that obtained, unto a house to carry thee.'\\!

The wench of her hard fortune doth complain,\\*
\vin And saith that now she doubts he sues too late;\\
The greek doth sigh \& sob, and part doth fain.\\
\vin `And shall I die,' quoth he, `in this estate?\\
Let me enjoy thy sweetness once again,\\
\vin Before my days draw to their doleful date;\\
One small refreshing ere we quite depart\\*
Will make me die with more contented heart.'\\!

The girl with pity mov`ed, thus replies,\\*
\vin `Think not,' quoth she, `but I desire the same;\\
But hard it is among so many eyes,\\
\vin Without incurring punishment \& shame.'\\
`Ah!' quoth the greek, `some means thou wouldst devise,\\
\vin If thou but felt a \sfrac{$1$}{$4$} of my flame,\\
To meet this night in some convenient place,\\*
And be together but a little space.\\!

`Tush!' answered she. `You sue now out of season,\\*
\vin For every night I lie betwixt them two\\
And they will quickly fear and find the treason,\\
\vin Sith still with one of them I have to do.'\\
`Well,' quoth the greek, `I could refute that reason,\\
\vin If you would put your helping hand thereto;\\
You must,' said he, `some pretty 'scuse devise,\\*
And find occasion from them both to rise.'\\!

She first bethinks herself, and after bad\\*
\vin He should return when all were sound asleep,\\
And learn\`{e}d him, who was thereof right glad,\\
\vin To go \& come, what order he should keep.\\
Now came the greek, as he his lesson had,\\
\vin When all was hushed, as soft as he could creep,\\
First to the door, which opened when he pushed,\\*
Then to the chamber, which was softly rushed.\\!

He takes a long \& leisureable stride,\\*
\vin And longest on the hinder foot he stayed,\\
So soft he treads, although his steps were wide,\\
\vin As though to tread on eggs he were afraid;\\
And as he goes, he gropes on either side\\
\vin To find the bed, with hands abroad displayed,\\
And having found the bottom of the bed,\\*
He creepeth in, and forward go'th his head.\\!

Between \textit{Fiametta}'s tender thighs he came,\\*
\vin That lay upright, as ready to receive;\\
At last they fell unto their merry game,\\
\vin Embracing sweetly now to take their leave;\\
He rode in post, nor can he bait for shame;\\
\vin The beast was good, and would not him deceive;\\
He thinks her pace so easy \& so sure,\\*
That all the night to ride he could endure.\\!

\textit{Giocundo} and the king do both perceive\\*
\vin The bed to rock, as oft it comes to pass,\\
And both of them one error did deceive,\\
\vin For either thought it his companion was:\\
Now hath the greek taken his latter leave,\\
\vin And as he came, he back again doth pass,\\
And \textit{Phoebus}' beams did now to shine begin;\\*
\textit{Fiametta} rose and let the pages in.
\end{verse}

\subsection{}

\blfootnote{`Of Beauty', Sir Richard Fanshaw, 1st Baronet (1608 -- 1666), \cite{newlove}.}\settowidth{\versewidth}{Snatch those joys that haste away.}
\begin{verse}[\versewidth]
Let us use it while we may,\\*
Snatch those joys that haste away.\\
Earth her winter coat may cast,\\
And renew her beauty past:\\
But, our winter come, in vain\\
We solicit spring again;\\
And when our furrows snow shall cover,\\*
Love may return but never lover.
\end{verse}

\subsection{}

\blfootnote{William Blake (1757 -- 1827), \cite{blakea}. This is one of Blake's `Proverbs of Hell' from \refbook{The Marriage of Heaven and Hell}.}No bird soars too high if he soars with his own wings.

\section{}

\subsection{}

\blfootnote{`Man', The Rev George Herbert (1593 -- 1633), \cite{norton}. \P 8. Other sources put `no fruit' instead of fruit.}\settowidth{\versewidth}{Night draws the curtain, which the sun withdraws;}
\begin{verse}[\versewidth]
\vin \vin My God, I heard this day\\*
That none doth build a stately habitation,\\
\vin But he that means to dwell therein.\\
\vin What house more stately hath there been,\\
Or can be, than is man, to whose creation\\*
\vin \vin All things are in decay?\\!

\vin \vin For man is every thing,\\*
And more: he is a tree, yet bears more fruit;\\
\vin A beast, yet is or should be more:\\
\vin Reason \& speech we only bring.\\
Parrots may thank us, if they are not mute,\\*
\vin \vin They go upon the score.\\!

\vin \vin Man is all symmetry,\\*
Full of proportions, one limb to another,\\
\vin And all to all the world besides:\\
\vin Each part may call the furthest brother;\\
For head with foot hath private amity,\\*
\vin \vin And both with moons \& tides.\\!

\vin \vin Nothing hath got so far,\\*
But man hath caught \& kept it, as his prey.\\
\vin His eyes dismount the highest star:\\
\vin He is in little all the sphere.\\
Herbs gladly cure our flesh, because that they\\*
\vin \vin Find their acquaintance there.\\!

\vin \vin For us the winds do blow,\\*
The earth doth rest, heaven move, and fountains flow.\\
\vin Nothing we see but means our good,\\
\vin As our delight or as our treasure:\\
The whole is either our cupboard of food,\\*
\vin \vin Or cabinet of pleasure.\\!

\vin \vin The stars have us to bed:\\*
Night draws the curtain, which the sun withdraws;\\
\vin Music \& light attend our head.\\
\vin All things unto our flesh are kind\\
In their descent \& being; to our mind\\*
\vin \vin In their ascent \& cause.\\!

\vin \vin Each thing is full of duty.\\*
Waters united are our navigation;\\
\vin Distinguish\`{e}d, our habitation;\\
\vin Below, our drink; above, our meat;\\
Both are our cleanliness. Hath one such beauty?\\*
\vin \vin Then how are all things neat?\\!

\vin \vin More servants wait on man\\*
Than he'll take notice of: in every path\\
\vin He treads down that which doth befriend him\\
\vin When sickness makes him pale \& wan.\\
O might love! Man is one world, and hath\\*
\vin \vin Another to attend him.\\!

\vin \vin Since then, my God, thou hast\\*
So brave a palace built, O dwell in it,\\
\vin That it may dwell with thee at last.\\
\vin Till then, afford us so much wit,\\
That, as the world serves us, we may serve thee,\\*
\vin \vin And both thy servants be.
\end{verse}

\subsection{}

\blfootnote{$\mathbb{R}$ `Lovers Rejoyce', John Fletcher (1579 -- 1625), \cite{pbev}. This song is appears in \refbook{Cupid's Revenge} I.2. \P 3. The word `regarded' in this context means something closer to ``congealed'' in modern English.}\settowidth{\versewidth}{    No more faint-hearted girls shall dream of harms,}
\begin{verse}[\versewidth]
Lovers, rejoice. Your pains shall be rewarded,\\*
\vin The god of love himself grieves at your crying:\\
No more shall frozen honour be regarded,\\
\vin Nor the coy faces of a maid denying.\\
No more shall virgins sigh, and say, We dare not,\\
For men are false, and what they do they care not.\\
All shall be well again; then do not grieve;\\*
Men shall be true, and women shall believe.\\!

Lovers, rejoice. What you shall say henceforth,\\*
\vin When you have caught your sweethearts in your arms,\\
It shall be accounted oracle \& worth:\\
\vin No more faint-hearted girls shall dream of harms,\\
And cry they are too young. The god hath said\\
Fifteen shall make a mother of a maid.\\
Then, wise men, pull your roses yet unblown;\\*
Love hates the too ripe fruit that falls alone.
\end{verse}

\subsection{}

\blfootnote{William Blake (1757 -- 1827), \cite{blakea}. This is one of Blake's `Proverbs of Hell' from \refbook{The Marriage of Heaven and Hell}.}Prisons are built with stones of law, brothels with bricks of religion.

\section{}

\subsection{}

\blfootnote{`The Vine', Robert Herrick (1591 -- 1674), \cite{norton}.}\settowidth{\versewidth}{And found (ah me!) this flesh of mine}
\begin{verse}[\versewidth]
I dreamed this mortal part of mine\\*
Was metamorphosed to a vine,\\
Which crawling one \& every way\\
Enthralled my dainty \textit{Lucia}.\\
Methought her long small legs \& thighs\\
I with my tendrils did surprise;\\
Her belly, buttocks and her waist\\
By my soft nervelets were embraced.\\
About her head I writhing hung,\\
And with rich clusters (hid among\\
The leaves) her temples I behung,\\
So that my \textit{Lucia} seemed to me\\
Young \textit{Bacchus} ravished by his tree.\\
My curls about her neck did crawl,\\
And arms \& hands they did enthrall,\\
So that she could not freely stir\\
(All parts there made one prisoner).\\
But when I crept with leaves to hide\\
Those parts which maids keep unespied,\\
Such fleeting pleasures there I took\\
That with the fancy I awoke;\\
And found (ah me!) this flesh of mine\\*
More like a stock than like a vine.
\end{verse}

\subsection{}

\blfootnote{John Gay (1685 -- 1732), \cite{pbev}. Handel set these words to music in his operetta \refbook{Acis and Galatea}.}\settowidth{\versewidth}{    And warbling in her breath.}
\begin{verse}[\versewidth]
Love in her eyes sits playing,\\*
\vin And sheds delicious death;\\
Love on her lips is straying,\\
\vin And warbling in her breath.\\
Love on her breast sits panting\\
\vin And swells with soft desire;\\
No grace, no charm is wanting,\\*
\vin To set the heart on fire.
\end{verse}

\subsection{}

\blfootnote{William Blake (1757 -- 1827), \cite{blakea}. This is one of Blake's `Proverbs of Hell' from \refbook{The Marriage of Heaven and Hell}.}The weak in courage is strong in cunning.

\section{}

\subsection{}

\blfootnote{`If --', Rudyard Kipling (1865 -- 1936), \cite{oxfordlarkin}.}\settowidth{\versewidth}{If you can think -- and not make thoughts your aim;}
\begin{verse}[\versewidth]
If you can keep your head when all about you\\*
Are losing theirs and blaming it on you,\\
If you can trust yourself when all men doubt you,\\
But make allowance for their doubting too;\\
If you can wait and not be tired by waiting,\\
Or being lied about, don't deal in lies,\\
Or being hated, don't give way to hating,\\*
And yet don't look too good, nor talk too wise:\\!

If you can dream -- and not make dreams your master;\\*
If you can think -- and not make thoughts your aim;\\
If you can meet with triumph \& disaster\\
And treat those two impostors just the same;\\
If you can bear to hear the truth you've spoken\\
Twisted by knaves to make a trap for fools,\\
Or watch the things you gave your life to broken,\\*
And stoop and build 'em up with worn-out tools:\\!

If you can make one heap of all your winnings\\*
And risk it on one turn of pitch-and-toss,\\
And lose, and start again at your beginnings\\
And never breathe a word about your loss;\\
If you can force your heart \& nerve \& sinew\\
To serve your turn long after they are gone,\\
And so hold on when there is nothing in you\\*
Except the will which says to them, Hold on!\\!

If you can talk with crowds and keep your virtue,\\*
Or walk with kings -- nor lose the common touch,\\
If neither foes nor loving friends can hurt you,\\
If all men count with you, but none too much;\\
If you can fill the unforgiving minute\\
With 60 seconds' worth of distance run,\\
Yours is the earth and everything that's in it,\\*
And -- which is more -- you'll be a man, my son!
\end{verse}

\subsection{}

\blfootnote{`A Church Romance', Thomas Hardy (1840 -- 1928), \cite{oxfordlarkin}. This poem is subtitled `Mellstock: circa 1835', Mellstock being a name Hardy coined himself for a village in his semi-fictional Wessex which corresponded to his native Stinsford. The poem describes Hardy's own parents' courtship.}\settowidth{\versewidth}{Thus their hearts' bond began, in due time signed.}
\begin{verse}[\versewidth]
She turned in the high pew, until her sight\\*
\vin Swept the west gallery, and caught its row\\
\vin Of music-men with viol, book, \& bow\\
Against the sinking sad tower-window light.\\
She turned again; and in her pride's despite\\
\vin One strenuous viol's inspirer seemed to throw\\
\vin A message from his string to her below,\\*
Which said: `I claim thee as my own forthright!'\\!

Thus their hearts' bond began, in due time signed.\\*
\vin And long years thence, when age had scared romance,\\
\vin At some old attitude of his or glance\\
That gallery-scene would break upon her mind,\\
\vin With him as minstrel, ardent, young, \& trim,\\*
\vin Bowing ``New Sabbath'' or ``Mount Ephraim''.
\end{verse}

\subsection{}

\blfootnote{William Blake (1757 -- 1827), \cite{blakea}. This is one of Blake's `Proverbs of Hell' from \refbook{The Marriage of Heaven and Hell}.}What is now proved was once only imagined.

\section{}

\subsection{}

\blfootnote{`The Owl and the Pussy-Cat', Edward Lear (1812 -- 1888), \cite{norton}. The Almanacker's (maternal) grandfather had a picture of `The owl and the pussy-cat', and, while pointing out the said image, used to sing the poem to his grandson.}\settowidth{\versewidth}{So they took it away, and were married next day}
\begin{verse}[\versewidth]
The owl \& the pussy-cat went to sea\\*
\vin In a beautiful pea-green boat;\\
They took some honey, and plenty of money,\\
\vin Wrapped up in a {\pounds}5 note.\\
The owl looked up to the stars above,\\
\vin And sang to a small guitar,\\
O lovely pussy! O pussy, my love,\\
\vin What a beautiful pussy you are,\\
\vin \vin You are,\\
\vin \vin You are!\\*
\vin What a beautiful pussy you are!\\!

Pussy said to the owl, You elegant fowl!\\*
\vin How charmingly sweet you sing!\\
O let us be married! Too long we have tarried:\\
\vin But what shall we do for a ring?\\
They sailed away, for a year \& a day,\\
\vin To the land where the bong tree grows\\
And there in a wood a piggy-wig stood\\
\vin With a ring at the end of his nose,\\
\vin \vin His nose,\\
\vin \vin His nose,\\*
\vin With a ring at the end of his nose.\\!

Dear pig, are you willing to sell for one shilling\\*
\vin Your ring? Said the piggy: I will.\\
So they took it away, and were married next day\\
\vin By the turkey who lives on the hill.\\
They dined on mince, \& slices of quince,\\
\vin Which they ate with a runcible spoon;\\
And hand in hand, on the edge of the sand,\\
\vin They danced by the light of the moon,\\
\vin \vin The moon,\\
\vin \vin The moon,\\*
\vin They danced by the light of the moon.
\end{verse}

\subsection{}

\blfootnote{`A Thunderstorm in Town', Thomas Hardy (1840 -- 1928), \cite{newlove}. Hardy's subtitle indicates that the poem is based on a memory of his from 1893.}\settowidth{\versewidth}{Then the downpour ceased, to my sharp sad pain}
\begin{verse}[\versewidth]
She wore a new terracotta dress,\\*
And we stayed, because of the pelting storm,\\
Within the hansom's dry recess,\\
Though the horse had stopped; yea, motionless\\*
\vin We sat on, snug \& warm.\\!

Then the downpour ceased, to my sharp sad pain\\*
And the glass that had screened our forms before\\
Flew up, and out she sprang to her door:\\
I should have kissed her if the rain\\*
\vin Had lasted a minute moor.
\end{verse}

\subsection{}

\blfootnote{William Blake (1757 -- 1827), \cite{blakea}. This is one of Blake's `Proverbs of Hell' from \refbook{The Marriage of Heaven and Hell}.}You never know what is enough unless you know what is more than enough.

\section{}

\subsection{}

\blfootnote{`The Scrutiny', Col Richard Lovelace (1617 -- 1657), \cite{newlove}.}\settowidth{\versewidth}{    And 'twas last night I swore to thee}
\begin{verse}[\versewidth]
Why should you swear I am forsworn,\\*
\vin Since thine I vowed to be?\\
Lady, it is already morn,\\
\vin And 'twas last night I swore to thee\\*
\vin That fond impossibility.\\!

Have I not loved thee much \& long,\\*
\vin A tedious 12 hours' space?\\
I must all other beauties wrong,\\
\vin And rob thee of a new embrace,\\*
\vin Could I still dote upon thy face.\\!

Not but all joy in thy brown hair\\*
\vin By others may be found;\\
But I must search the black \& fair,\\
\vin Like skilful mineralists that sound\\*
\vin For treasure in unploughed-up ground.\\!

Then if, when I have loved my round,\\*
\vin Thou prov'st the pleasant she,\\
With spoils of meaner beauties crowned\\
\vin I laden will return to thee,\\*
\vin Ev'n sated with variety.
\end{verse}

\subsection{}

\blfootnote{`The Fired Pot', Mrs Edith Hepburn (1883 -- 1947), \cite{oxfordlarkin}. Despite these lines, the act of \emph{turning down} a prospective sexual partner doesn't appear to be something the poetess ever practised herself.}\settowidth{\versewidth}{Till the heart within my wedded breast grew cold,}
\begin{verse}[\versewidth]
In our town, people live in rows.\\*
\vin The only irregular thing in a street is the steeple;\\
And where that points to, God only knows,\\*
\vin And not the poor disciplined people!\\!

And I have watched the women growing old,\\*
\vin Passionate about pins, \& pence, \& soap,\\
Till the heart within my wedded breast grew cold,\\*
\vin And I lost hope.\\!

But a young soldier came to our town;\\*
\vin He spoke his mind most candidly.\\
He asked me quickly to lie down,\\*
\vin And that was very good for me.\\!

For though I gave him no embrace\\*
\vin -- Remembering my duty --\\
He altered the expression of my face,\\*
\vin And gave me back my beauty.
\end{verse}

\subsection{}

\blfootnote{John Bunyan (1628 -- 1688), \cite{bunyan}. This is a line from the shepherd boy's song in the second part of \refbook{Pilgrim's Progress}.}He that is down needs fear no fall.

\section{}

\subsection{}

\blfootnote{`To my excellent Lucasia, on Our Friendship', Mrs Katherine Philips (1632 -- 1664), \cite{norton}. The Lucasia in question was a certain Anne Owens. Orinda seems to have been the poetess's name for herself.}\settowidth{\versewidth}{No bridegroom's nor crown-conqueror's mirth}
\begin{verse}[\versewidth]
I did not live until this time\\*
\vin Crowned my felicity,\\
When I could say without a crime,\\*
\vin I was not thine, but thee.\\!

This carcass breathed, and walked, and slept,\\*
\vin So that the world believed\\
There was a soul the motions kept;\\*
\vin But they were all deceived.\\!

For as a watch by art is wound\\*
\vin To motion, such was mine:\\
But never had \textit{Orinda} found\\*
\vin A soul till she found thine;\\!

Which now inspires, cures and supplies,\\*
\vin And guides my darkened breast:\\
For thou art all that I can prize,\\*
\vin My joy, my life, my rest.\\!

No bridegroom's nor crown-conqueror's mirth\\*
\vin To mine compared can be:\\
They have but pieces of the earth;\\*
\vin I've all the world in thee.\\!

Then let our flames still light \& shine,\\*
\vin And no false fear control,\\
As innocent as our design,\\*
\vin Immortal as our soul.
\end{verse}

\subsection{}

\blfootnote{`Spelt from Sibyl's Leaves', Fr Gerard Hopkins (1844 -- 1889), \cite{obev}. The title is probably an allusion to the Sibylline Books of ancient Rome.}\settowidth{\versewidth}{Her fond yellow hornlight wound to the west, | her wild hollow hoarlight hung to the height}
\begin{verse}[\versewidth]
Earnest, earthless, equal, attuneable, $\wp$ vaulty, voluminous... stupendous\\*
Evening strains to be time's v\'{a}st, $\wp$ womb-of-all, home-of-all, hearse-of-all night.\\
Her fond yellow hornlight wound to the west, $\wp$ her wild hollow hoarlight hung to the height\\
Waste; her earliest stars, earl-stars, $\wp$ st\'{a}rs principal, overbend us,\\
F\'{i}re-f\'{e}aturing heaven. For earth $\wp$ her being as unbound, her dapple is at an end, as-\\
tray or aswarm, all throughther, in throngs; $\wp$ self \'{i}n self steep{\'{e}}d and p\'{a}shed -- quite\\
Disremembering, d\'{i}sm\'{e}mbering, $\wp$ \'{a}ll now. Heart, you round me right\\
With: \'{o}ur \'{e}vening is over us; \'{o}ur night $\wp$ wh\'{e}lms, wh\'{e}lms, \'{a}nd will end us.\\
Only the beak-leaved boughs dragonish $\wp$ damask the tool-smooth bleak light; black,\\
Ever so black on it. \'{O}ur tale, O \'{o}ur oracle! $\wp$ L\'{e}t life, w\'{a}ned, ah l\'{e}t life wind\\
Off h\'{e}r once sk\'{e}ined stained v\'{e}ined var\'{i}ety $\wp$ upon \'{a}ll on tw\'{o} spools; p\'{a}rt, pen, p\'{a}ck\\
Now her \'{a}ll in tw\'{o} flocks, tw\'{o} folds -- black, white; $\wp$ right, wrong; reckon but, reck but, mind\\
But th\'{e}se two; w\'{a}re of a w\'{o}rld where b\'{u}t these $\wp$ tw\'{o} tell, each off the \'{o}ther; of a rack\\*
Where, selfwrung, selfstrung, sheathe- \& shelterless, $\wp$ th\'{o}ughts aga\'{i}nst thoughts \'{i}n groans gr\'{i}nd.
\end{verse}

\subsection{}

\blfootnote{John Bunyan (1628 -- 1688), \cite{bunyan}. These words are taken from Bunyan's introduction to \emph{The Holy City}, his commentary on the closing chapters of Revelation.}Words easy to be understood do often hit the mark; where high and learned ones do only pierce the air.

\section{}

\subsection{}

\blfootnote{`Commission', Ezra Pound (1885 -- 1972), \cite{newlove}. \P 35. `Mortmain' (literally ``dead hand'') is an obscure piece of Anglo-Norman legalese which refers to the manner in which a legal person -- as opposed to a literal or ``natural'' person -- owns land. Said legal person was, more often than not, some sort of religious organisation.}\settowidth{\versewidth}{Bring confidence upon the algae the tentacles of the soul.}
\begin{verse}[\versewidth]
Go, my songs, to the lonely and the unsatisfied;\\*
Go also to the nerve-racked; go to the enslaved-by-convention.\\
Bear to them my contempt for their oppressors.\\
Go as a great wave of cool water;\\*
Bear my contempt of oppressors.\\!

Speak against unconscious oppression;\\*
Speak against the tyranny of the unimaginative;\\
Speak against bonds.\\
Go to the {\hoskeroe bourgeoise} who is dying of her ennuis;\\
Go to the women in suburbs.\\
Go to the hideously wedded;\\
Go to them whose failure is concealed;\\
Go to the unluckily mated;\\
Go to the bought wife;\\*
Go to the woman entailed.\\!

Go to those who have delicate lust;\\*
Go to those whose delicate desires are thwarted;\\
Go like a blight upon the dulness of the world;\\
Go with your edge against this;\\
Strengthen the subtle cords;\\
Bring confidence upon the algae \& the tentacles of the soul.\\
Go in a friendly manner;\\
Go with an open speech.\\
Be eager to find new evils \& new good;\\
Be against all forms of oppression.\\
Go to those who are thickened with middle age,\\*
To those who have lost their interest.\\!

Go to the adolescent who are smothered in family --\\*
O how hideous it is\\
To see three generations of one house gathered together!\\
It is like an old tree with shoots,\\*
And with some branches rotted \& falling.\\!

Go out and defy opinion;\\*
Go against this vegetable bondage of the blood.\\*
Be against all sorts of mortmain.
\end{verse}

\subsection{}

\blfootnote{`Inversnaid', Fr Gerard Hopkins (1844 -- 1889), \cite{londonbook}. Inversnaid is a hamlet on the southern edge of the Scottish Highlands, famous for a nearby cave associated with the folk hero Rob Roy, and in more recent times for having a primary school with only two pupils, which subsequently closed for that reason.}\settowidth{\versewidth}{And the beadbonny ash that sits over the burn.}
\begin{verse}[\versewidth]
This darksome burn, horseback brown,\\*
His rollrock highroad roaring down,\\
In coop \& in comb the fleece of his foam\\*
Flutes and low to the lake falls home.\\!

A windpuff-bonnet of f\'{a}wn-fr\'{o}th\\*
Turns and twindles over the broth\\
Of a pool so pitchblack, f\'{e}ll-fr\'{o}wning,\\*
It rounds and rounds despair to drowning.\\!

Degged with dew, dappled with dew\\*
Are the groins of the braes that the brook treads through,\\
Wiry heathpacks, flitches of fern,\\*
And the beadbonny ash that sits over the burn.\\!

What would the world be, once bereft\\*
Of wet \& of wildness? Let them be left;\\
O let them be left, wildness \& wet;\\*
Long live the weeds \& the wilderness yet.
\end{verse}

\subsection{}

\blfootnote{Samuel Butler (1612 -- 1680), \cite{odq}.}Oaths are but words.

\section{}

\subsection{}

\blfootnote{`The Cloud', Percy Shelley (1792 -- 1822), \cite{norton}. This poem may well have been influenced by, and in any case bears a striking likeness to, riddles from the \refbook{Exeter Book}.}\settowidth{\versewidth}{And when sunset may breathe, from the lit sea beneath,}
\begin{verse}[\versewidth]
I bring fresh showers for the thirsting flowers,\\*
\vin From the seas and the streams;\\
I bear light shade for the leaves when laid\\
\vin In their noonday dreams.\\
From my wings are shaken the dews that waken\\
\vin The sweet buds every one,\\
When rocked to rest on their mother's breast,\\
\vin As she dances about the sun.\\
I wield the flail of the lashing hail,\\
\vin And whiten the green plains under,\\
And then again I dissolve it in rain,\\*
\vin And laugh as I pass in thunder.\\!

I sift the snow on the mountains below,\\*
\vin And their great pines groan aghast;\\
And all the night 'tis my pillow white,\\
\vin While I sleep in the arms of the blast.\\
Sublime on the towers of my skiey bowers,\\
\vin Lightning my pilot sits;\\
In a cavern under is fettered the thunder,\\
\vin It struggles and howls at fits;\\
Over earth and ocean, with gentle motion,\\
\vin This pilot is guiding me,\\
Lured by the love of the genii that move\\
\vin In the depths of the purple sea;\\
Over the rills, and the crags, and the hills,\\
\vin Over the lakes \& the plains,\\
Wherever he dream, under mountain or stream,\\
\vin The spirit he loves remains;\\
And I all the while bask in heaven's blue smile,\\*
\vin Whilst he is dissolving in rains.\\!

The sanguine sunrise, with his meteor eyes,\\*
\vin And his burning plumes outspread,\\
Leaps on the back of my sailing rack,\\
\vin When the morning star shines dead;\\
As on the jag of a mountain crag,\\
\vin Which an earthquake rocks \& swings,\\
An eagle alit one moment may sit\\
\vin In the light of its golden wings.\\
And when sunset may breathe, from the lit sea beneath,\\
\vin Its ardours of rest \& of love,\\
And the crimson pall of eve may fall\\
\vin From the depth of Heaven above,\\
With wings folded I rest, on mine aEDDOTry nest,\\*
\vin As still as a brooding dove.\\!

That orb\`{e}d maiden with white fire laden,\\*
\vin Whom mortals call the moon,\\
Glides glimmering o'er my fleece-like floor,\\
\vin By the midnight breezes strewn;\\
And wherever the beat of her unseen feet,\\
\vin Which only the angels hear,\\
May have broken the woof of my tent's thin roof,\\
\vin The stars peep behind her and peer;\\
And I laugh to see them whirl \& flee,\\
\vin Like a swarm of golden bees,\\
When I widen the rent in my wind-built tent,\\
\vin Till calm the rivers, lakes, and seas,\\
Like strips of the sky fallen through me on high,\\*
\vin Are each paved with the moon and these.\\!

I bind the sun's throne with a burning zone,\\*
\vin And the moon's with a girdle of pearl;\\
The volcanoes are dim, and the stars reel \& swim,\\
\vin When the whirlwinds my banner unfurl.\\
From cape to cape, with a bridge-like shape,\\
\vin Over a torrent sea,\\
Sunbeam-proof, I hang like a roof,\\
\vin The mountains its columns be.\\
The triumphal arch through which I march\\
\vin With hurricane, fire, and snow,\\
When the powers of the air are chained to my chair,\\
\vin Is the million-coloured bow;\\
The sphere-fire above its soft colours wove,\\*
\vin While the moist earth was laughing below.\\!

I am the daughter of earth \& water,\\*
\vin And the nursling of the sky;\\
I pass through the pores of the ocean \& shores;\\
\vin I change, but I cannot die.\\
For after the rain when with never a stain\\
\vin The pavilion of heaven is bare,\\
And the winds \& sunbeams with their convex gleams\\
\vin Build up the blue dome of air,\\
I silently laugh at my own cenotaph,\\
\vin And out of the caverns of rain,\\
Like a child from the womb, like a ghost from the tomb,\\*
\vin I arise and unbuild it again.
\end{verse}

\subsection{}

\blfootnote{$\mathbb{R}$ Ben Jonson (1572 -- 1637), \cite{pbev}. These lines are taken from Jonson's verse letter to Elizabeth, the wife of the 5th Earl of Rutland (of the third creation) and daughter of Sir Philip Sidney.}\settowidth{\versewidth}{Riches thought most; but, madam, think what store}
\begin{verse}[\versewidth]
Beauty, I know, is good, and blood is more;\\*
Riches thought most; but, madam, think what store\\
The world hath seen, which all these had in trust\\
And now lie in their forgotten dust.\\
It is the muse alone, can raise to heaven,\\
And at her strong arm's end, hold up, and even\\
The souls she loves. Those other glorious notes,\\
Inscribed in touch or marble, or the coats\\
Painted or carved upon our great men's tombs,\\
Or in their windows, do but prove the wombs\\
That bred them, graves: when they were born they died\\
That had no muse to make their fame abide\\
How many equal with the argive queen,\\*
Have beauty known, yet none so famous seen?
\end{verse}

\subsection{}

\blfootnote{George Noel, 6th Baron Byron (1788 -- 1824), \cite{odq}. This is a line from the first canto of \refbook{Don Juan}.}Good workmen never quarrel with their tools.

\section{}

\subsection{}

\blfootnote{`The Author Apologizes to a Lady for His Being a Little Man', Christopher Smart (1722 -- 1771), \cite{newlove}. Smart affixed two quotations to this poem: one from the Ἰλιάς (I.167) -- `ὀλίγον τε φίλον τε', which means, `A small but dear thing' -- and one from Pliny's \refbook{Naturalis Historia} -- `Natura nusquam magis, quam in minimis tota est', which means, `Nature is nowhere greater than in the smallest of things'.}\settowidth{\versewidth}{        And to some youth gigantic yield your charms,}
\begin{verse}[\versewidth]
\vin \vin \vin Yes, contumelious fair, you scorn\\*
\vin The amorous dwarf, that courts you to his arms,\\
\vin \vin \vin But ere you leave him quite forlorn,\\
\vin \vin And to some youth gigantic yield your charms,\\
\vin Hear him, O hear him, if you will not try,\\*
And let your judgment check th'ambition of your eye.\\!

\vin \vin \vin Say, is it carnage makes the man?\\*
\vin Is to be monstrous really to be great?\\
\vin \vin \vin Say, is it wise or just to scan\\
\vin \vin Your lover's worth by quantity, or weight?\\
\vin Ask your mamma \& nurse, if it be so;\\*
Nurse \& mamma, I ween, shall jointly answer, no.\\!

\vin \vin \vin The less the body to the view,\\*
\vin The soul (like springs in closer durance pent)\\
\vin \vin \vin Is all exertion, ever new,\\
\vin \vin Unceasing, unextinguished, and unspent;\\
\vin Still pouring forth executive desire,\\*
As bright, as brisk, \& lasting, as the vestal fire.\\!

\vin \vin \vin Does thy young bosom pant for fame?\\*
\vin Would'st thou be of posterity the toast?\\
\vin \vin \vin The poets shall ensure thy name,\\
\vin \vin Who magnitude of mind not body boast.\\
\vin Laurels on bulky bards as rarely grow,\\*
As on the sturdy oak the virtuous misletoe.\\!

\vin \vin \vin Look in the glass, survey that cheek\\*
\vin Where \textit{Flora} has with all her roses blushed;\\
\vin \vin \vin The shape so tender, looks so meek,\\
\vin \vin The breasts made to be pressed, not to be crushed --\\
\vin Then turn to me -- turn with obliging eyes,\\*
Nor longer nature's works, in miniature, despise.\\!

\vin \vin \vin Young \textit{Ammon} did the world subdue,\\*
\vin Yet had not more external man than I;\\
\vin \vin \vin Ah charmer, should I conquer you,\\
\vin \vin With him in fame, as well as size, I'll vie.\\
\vin Then, scornful nymph, come forth to yonder grove,\\*
Where I defy, and challenge, all thy utmost love.
\end{verse}

\subsection{}

\blfootnote{John Keats (1795 -- 1821), \cite{newlove}.}\settowidth{\versewidth}{So haunt thy days and chill thy dreaming nights}
\begin{verse}[\versewidth]
This living hand, now warm \& capable\\*
Of earnest grasping, would, if it were cold\\
And in the icy silence of the tomb,\\
So haunt thy days and chill thy dreaming nights\\
That thou would wish thine own heart dry of blood\\
So in my veins red life might stream again,\\
And thou be conscience-calmed; see here it is;\\*
I hold it towards you.
\end{verse}

\subsection{}

\blfootnote{Mrs Charlotte Nicholls (1816 -- 1855), \cite{odq}. These words are taken from the author's preface to \refbook{Jane Eyre}.}Conventionality is not morality.

\section{}

\subsection{}

\blfootnote{$\mathbb{R}$ `Tymes Goe by Turnes', Saint Robert Southwell (1561 -- 1595), \cite{pbev}.}\settowidth{\versewidth}{Times go by turns, and chances change by course:}
\begin{verse}[\versewidth]
The lopped tree in time may grow again;\\*
\vin Most naked plants renew both fruit \& flower;\\
The sorest wight may find release of pain;\\
\vin The driest soil suck in some moist'ning shower.\\
Times go by turns, and chances change by course:\\*
From foul to fair, from better hap to worse.\\!

The sea of fortune doth not ever flow;\\*
\vin She draws her favours to the lowest ebb;\\
Her tide hath equal times to come \& go;\\
\vin Her loom doth weave the fine \& coarsest web.\\
No joy so great, but runneth to an end;\\*
No hap so hard, but may in fine amend.\\!

Not always fall of leaf, nor ever spring;\\*
\vin No endless night, yet not eternal day;\\
The saddest birds a season find to sing;\\
\vin The roughest storm a calm may soon allay.\\
Thus with succeeding turns God tempereth all,\\*
That man may hope to rise, yet fear to fall.\\!

A chance may win that by mischance was lost;\\*
\vin The net that holds no great, takes little fish;\\
In some things all, in all things none are crossed:\\
\vin Few all they need, but none have all they wish.\\
Unmeddled joys here to no man befall;\\*
Who least, hath some, who most, hath never all.
\end{verse}

\subsection{}

\blfootnote{`Sea-Fever', Dr John Masefield, Poet Laureate (1878 -- 1967), \cite{oxfordlarkin}.}\settowidth{\versewidth}{I must go down to the seas again, for the call of the running tide}
\begin{verse}[\versewidth]
I must go down to the seas again, to the lonely sea \& the sky,\\*
And all I ask is a tall ship \& a star to steer her by;\\
And the wheel's kick \& the wind's song \& the white sail's shaking,\\*
And a grey mist on the sea's face, \& a grey dawn breaking.\\!

I must go down to the seas again, for the call of the running tide\\*
Is a wild call \& a clear call that may not be denied;\\
And all I ask is a windy day with the white clouds flying,\\*
And the flung spray \& the blown spume, \& the sea-gulls crying.\\!

I must go down to the seas again, to the vagrant gypsy life,\\*
To the gull's way \& the whale's way where the wind's like a whetted knife;\\
And all I ask is a merry yarn from a laughing fellow-rover,\\*
And quiet sleep \& a sweet dream when the long trick's over.
\end{verse}

\subsection{}

\blfootnote{William Congreve (1670 -- 1729), \cite{odq}.}To go naked is the best disguise.

\section{}

\subsection{}

\blfootnote{`Chorus', Algernon Swinburne (1837 -- 1909), \cite{obev}. This is a chorus from Swinburne's tragedy \refbook{Atalanta in Calydon}. \P 6. According to the Ὀδύσσεια XIX.519-24, Aedon killed her own son, Itylus, during a psychotic episode, for which Zeus transformed her into a nightingale -- hence the bird's mournful song. \P 44. The terms `maenad' and `bassarid' are synonyms.}\settowidth{\versewidth}{    Fire, or the strength of the streams that spring!}
\begin{verse}[\versewidth]
When the hounds of spring are on winter's traces,\\*
\vin The mother of months in meadow or plain\\
Fills the shadows and windy places\\
\vin With lisp of leaves \& ripple of rain;\\
And the brown bright nightingale amorous\\
Is \sfrac{$1$}{$2$} assuaged for \textit{Itylus},\\
For the thracian ships \& the foreign faces,\\*
\vin The tongueless vigil, \& all the pain.\\!

Come with bows bent and with emptying of quivers,\\*
\vin Maiden most perfect, lady of light,\\
With a noise of winds \& many rivers,\\
\vin With a clamour of waters, \& with might;\\
Bind on thy sandals, O thou most fleet,\\
Over the splendour \& speed of thy feet;\\
For the faint east quickens, the wan west shivers,\\*
\vin Round the feet of the day and the feet of the night.\\!

Where shall we find her? How shall we sing to her,\\*
\vin Fold our hands round her knees, and cling?\\
O that man's heart were as fire and could spring to her,\\
\vin Fire, or the strength of the streams that spring!\\
For the stars \& the winds are unto her\\
As raiment, as songs of the harp-player;\\
For the risen stars \& the fallen cling to her,\\*
\vin And the southwest wind \& the west wind sing.\\!

For winter's rains \& ruins are over,\\*
\vin And all the season of snows \& sins;\\
The days dividing lover \& lover,\\
\vin The light that loses, the night that wins;\\
And time remembered is grief forgotten,\\
And frosts are slain and flowers begotten,\\
And in green underwood \& cover\\*
\vin Blossom by blossom the spring begins.\\!

The full streams feed on flower of rushes,\\*
\vin Ripe grasses trammel a traveling foot,\\
The faint fresh flame of the young year flushes\\
\vin From leaf to flower and flower to fruit;\\
And fruit \& leaf are as gold \& fire,\\
And the oat is heard above the lyre,\\
And the hoof{\`{e}}d heel of a satyr crushes\\*
\vin The chestnut husk at the chestnut root.\\!

And \textit{Pan} by noon and \textit{Bacchus} by night,\\*
\vin Fleeter of foot than the fleet-foot kid,\\
Follows with dancing and fills with delight\\
\vin The maenad \& the bassarid;\\
And soft as lips that laugh and hide\\
The laughing leaves of the trees divide,\\
And screen from seeing and leave in sight\\*
\vin The god pursuing, the maiden hid.\\!

The ivy falls with the bacchanal's hair\\*
\vin Over her eyebrows hiding her eyes;\\
The wild vine slipping down leaves bare\\
\vin Her bright breast shortening into sighs;\\
The wild vine slips with the weight of its leaves,\\
But the berried ivy catches and cleaves\\
To the limbs that glitter, the feet that scare\\*
\vin The wolf that follows, the fawn that flies.
\end{verse}

\subsection{}

\blfootnote{Miss Charlotte Mew (1869 -- 1928), \cite{oxfordlarkin}.}\settowidth{\versewidth}{Because it was these you so liked to hear --}
\begin{verse}[\versewidth]
I so liked spring last year\\*
\vin Because you were here --\\
\vin \vin The thrushes too --\\
Because it was these you so liked to hear --\\*
\vin \vin I so liked you.\\!

\vin This year's a different thing;\\*
\vin \vin I'll not think of you.\\
But I'll like the spring because it is simply spring\\*
\vin \vin As the thrushes do.
\end{verse}

\subsection{}

\blfootnote{The Rev Charles Dodgson (1832 -- 1898), \cite{odq}.}It's a poor sort of memory that only works backwards.

\section{}

\subsection{}

\blfootnote{Alfred Tennyson, 1st Baron Tennyson, Poet Laureate (1809 -- 1892), \cite{obev}. These are the closing lines of Part I of Lord Tennyson's long poem \refbook{Maud}.}\settowidth{\versewidth}{From the lake to the meadow and on to the wood,}
\begin{verse}[\versewidth]
Come into the garden, \textit{Maud},\\*
\vin For the black bat, night, has flown;\\
Come into the garden, \textit{Maud};\\
\vin I am here at the gate alone;\\
And the woodbine spices are wafted abroad,\\*
\vin And the musk of the roses blown.\\!

For a breeze of morning moves,\\*
\vin And the planet of love is on high,\\
Beginning to faint in the light that she loves\\
\vin On a bed of daffodil sky,\\
To faint in the light of the sun she loves,\\*
\vin To faint in his light, and to die.\\!

All night have the roses heard\\*
\vin The flute, violin, bassoon;\\
All night has the casement jessamine stirred\\
\vin To the dangers dancing in tune;\\
Till a silence fell with the waking bird,\\*
\vin And a hush with the setting moon.\\!

I said to the lily, `There is but one\\*
\vin With whom she has heart to be gay.\\
When will the dancers leave her alone?\\
\vin She is weary of dance \& play.'\\
Now \sfrac{$1$}{$2$} to the setting moon are gone,\\
\vin And \sfrac{$1$}{$2$} to the rising day;\\
Low on the sand \& loud on the stone\\*
\vin The last wheel echoes away.\\!

I said to the rose, `The brief night goes\\*
\vin In babble \& revel \& wine.\\
Young lord-lover, what sighs are those,\\
\vin For one that will never be thine?\\
But mine, but mine,' so I sware to the rose,\\*
\vin 'For ever \& ever, mine.'\\!

And the soul of the rose went into my blood,\\*
\vin As the music clashed in the hall;\\
And long by the garden lake I stood,\\
\vin For I heard your rivulet fall\\
From the lake to the meadow and on to the wood,\\*
\vin Our wood, that is dearer than all;\\!

From the meadow your walks have left so sweet\\*
\vin That whenever a march wind sighs\\
He sets the jewel-print of your feet\\
\vin In violets blue as your eyes,\\
To the woody hollows in which we meet\\*
\vin And the valleys of paradise.\\!

The slender acacia would not shake\\*
\vin One long milk bloom on the tree;\\
The white lake blossom fell into the lake,\\
\vin As the pimpernel dozed on the lea;\\
But the rose was awake all night for your sake,\\
\vin Knowing your promise to me;\\
The lilies \& roses were all awake.\\*
\vin They sighed for the dawn \& thee.\\!

Queen rose of the rosebud garden of girls,\\*
\vin Come hither; the dances are done,\\
In gloss of satin \& glimmer of pearls,\\
\vin Queen lily \& rose in one;\\
Shine out, little head, sunning over with curls,\\*
\vin To the flowers, and be their sun.\\!

There has fallen a splendid tear\\*
\vin From the passion-flower at the gate.\\
She is coming, my dove, my dear;\\
\vin She is coming, my life, my fate;\\
The red rose cries, She is near, she is near;\\
\vin And the white rose weeps, She is late;\\
The larkspur listens, I hear, I hear;\\*
\vin And the lily whispers, I wait.\\!

She is coming, my own, my sweet;\\*
\vin Were it ever so airy a tread.\\
My heart would hear her and beat,\\
\vin Were it earth in an earthy bed;\\
My dust would hear her and beat,\\
\vin Had I lain for a century dead,\\
Would start and tremble under her feet,\\*
\vin And blossom in purple \& red.
\end{verse}

\subsection{}

\blfootnote{`A Quoi Bon Dire', Miss Charlotte Mew (1869 -- 1928), \cite{oxfordlarkin}. The French title appears to be somewhat untranslatable -- the Almanacker only speaks a very broken form of French -- but means something like, `What good is there to say?' or, `What's the point of saying?' Miss Mew's title, however, is without a question mark; it's unclear whether this was deliberate or an oversight.}\settowidth{\versewidth}{        Something that sounded like good-by:}
\begin{verse}[\versewidth]
Seventeen years ago you said\\*
\vin \vin Something that sounded like good-by:\\
\vin \vin And everybody thinks you are dead\\*
\vin \vin \vin \vin But I.\\!

\vin \vin So I as I grow stiff \& cold\\*
\vin To this \& that say good-by too;\\
\vin \vin And everybody sees that I am old\\*
\vin \vin \vin \vin But you.\\!

\vin \vin And one fine morning in a sunny lane\\*
\vin Some boy \& girl will meet \& kiss \& swear\\
\vin \vin That nobody can love their way again\\
\vin \vin \vin \vin While over there\\*
\vin You will have smiled; I shall have tossed your hair.
\end{verse}

\subsection{}

\blfootnote{Sir Francis Drake (1540 -- 1596), \cite{odq}.}There is plenty of time to win this game, and to thrash the Spaniards too.

\section{}

\subsection{}

\blfootnote{`The Charge of the Light Brigade', Alfred Tennyson, 1st Baron Tennyson, Poet Laureate (1809 -- 1892), \cite{norton}. The poem relates the famous and, as Lord Tennyson does his best to gloss over, clearly idiotic British cavalry charge at the Battle of Balaclava in 1854.}\settowidth{\versewidth}{Charge for the guns!' he said:}
\begin{verse}[\versewidth]
Half a league, \sfrac{$1$}{$2$} a league,\\*
\vin Half a league onward,\\
All in the valley of death\\
\vin Rode the 600.\\
`Forward, the Light Brigade!\\
Charge for the guns!' he said:\\
Into the valley of death\\*
\vin Rode the 600.\\!

`Forward, the Light Brigade!'\\*
Was there a man dismayed?\\
Not though the soldier knew\\
\vin Someone had blundered.\\
Theirs not to make reply;\\
Theirs not to reason why;\\
Theirs but to do \& die:\\
Into the valley of death\\*
\vin Rode the 600.\\!

Cannon to right of them,\\*
Cannon to left of them,\\
Cannon in front of them\\
\vin Volleyed \& thundered;\\
Stormed at with shot \& shell,\\
Boldly they rode and well;\\
Into the jaws of death,\\
Into the mouth of hell\\*
\vin Rode the 600.\\!

Flashed all their sabres bare;\\*
Flashed as they turned in air,\\
Sabring the gunners there,\\
Charging an army, while\\
\vin All the world wondered:\\
Plunged in the battery-smoke\\
Right through the line they broke;\\
Cossack \& russian\\
Reeled from the sabre stroke\\
\vin Shattered \& sundered.\\
Then they rode back, but not\\*
\vin Not the 600.\\!

Cannon to right of them,\\*
Cannon to left of them,\\
Cannon behind them\\
\vin Volleyed \& thundered;\\
Stormed at with shot \& shell,\\
While horse \& hero fell,\\
They that had fought so well\\
Came through the jaws of death\\
Back from the mouth of hell,\\
All that was left of them,\\*
\vin Left of 600.\\!

When can their glory fade?\\*
O the wild charge they made!\\
\vin All the world wondered.\\
Honour the charge they made,\\
Honour the Light Brigade,\\*
\vin Noble 600.
\end{verse}

\subsection{}

\blfootnote{$\mathbb{R}$ George Peele (1556 -- 1596), \cite{pbev}. \P 2. Chop-cherry was a traditional English children's game in which the player attempts to catch a cherry, perhaps suspended from a thread, between his teeth.}\settowidth{\versewidth}{And chop-cherry, chop-cherry ripe within,}
\begin{verse}[\versewidth]
Whenas the rye reach to the chin,\\*
And chop-cherry, chop-cherry ripe within,\\
Strawberries swimming in the cream,\\
And schoolboys playing in the stream;\\
Then O, then O, then O, my true love said,\\
Till that time come again\\*
She could not live a maid.
\end{verse}

\subsection{}

\blfootnote{Sir Francis Drake (1540 -- 1596), \cite{odq}.}There must be a beginning of any great matter.

\section{}

\subsection{}

\blfootnote{`To the Same Purpose', The Rev Thomas Traherne (1636 -- 1674), \cite{norton}. Lugwardine (which the \refbook{Norton Anthology} spells without an \textit{e}) is a village in Herefordshire.}\settowidth{\versewidth}{    'Tis want of sense which makes us poor.}
\begin{verse}[\versewidth]
To the same purpose: he, not long before\\*
\vin Brought home from nurse, going to the door\\
\vin \vin To do some little thing\\
\vin \vin He must not do within,\\
\vin \vin \vin With wonder cries,\\
\vin \vin \vin As in the skies\\
He saw the moon, `O yonder is the moon,\\
\vin Newly come after me to town,\\
That shined at \textsc{Lugwardine} but yesternight,\\*
\vin Where I enjoyed the self-same sight.'\\!

As if it had ev'n 20,000 faces,\\*
\vin It shines at once in many places;\\
\vin \vin To all the earth so wide\\
\vin \vin God doth the stars divide,\\
\vin \vin \vin With so much art\\
\vin \vin \vin The moon impart,\\
They serve us all; serve wholly every one\\
\vin As if they serv\`{e}d him alone.\\
While every single person hath such store,\\*
\vin 'Tis want of sense which makes us poor.
\end{verse}

\subsection{}

\blfootnote{$\mathbb{R}$ `Two or Three: A Recipe to Make a Cuckold', Alexander Pope (1688 -- 1744), \cite{newlove}.}\settowidth{\versewidth}{Two or three civil things, two or three vows,}
\begin{verse}[\versewidth]
Two or three visits, \& two or three bows,\\*
Two or three civil things, two or three vows,\\
Two or three kisses, with two or three sighs,\\
Two or three \textit{Jesuses} -- \& Let me dies --\\
Two or three squeezes, \& two or three touses,\\
With two or three {\pounds}1000 lost at their houses,\\*
Can never fail cuckolding two or three spouses.
\end{verse}

\subsection{}

\blfootnote{`Of Treason', Sir John Harington (1560 -- 1612), \cite{obev}.}\settowidth{\versewidth}{Treason doth never prosper. What's the reason?}
\begin{verse}[\versewidth]
Treason doth never prosper. What's the reason?\\*
For if it prosper none dare call it treason.
\end{verse}

\section{}

\subsection{}

\blfootnote{Walt Whitman (1819 -- 1892), \cite{norton}. These lines constitute \refpoem{Song of Myself} \S 5.}\settowidth{\versewidth}{And parted the shirt from my bosom-bone, and plunged your tongue to my bare-stripped heart,}
\begin{verse}[\versewidth]
I believe in you, my soul. The other I am must not abase itself to you,\\*
And you must not be abased to the other.\\!

Loaf with me on the grass; loose the stop from your throat.\\*
Not words, not music or rhyme I want, not custom or lecture, not even the best.\\*
Only the lull I like, the hum of your valv\`{e}d voice.\\!

I mind how once we lay such a transparent summer morning,\\*
How you settled your head athwart my hips and gently turned over upon me,\\
And parted the shirt from my bosom-bone, and plunged your tongue to my bare-stripped heart,\\*
And reached till you felt my beard, and reached till you held my feet.\\!

Swiftly arose and spread around me the peace \& knowledge that pass all the argument of the earth,\\*
And I know that the hand of God is the promise of my own,\\
And I know that the spirit of God is the brother of my own,\\
And that all the men ever born are also my brothers, and the women my sisters \& lovers,\\
And that a kelson of the creation is love,\\
And limitless are leaves stiff or drooping in the fields,\\
And brown ants in the little wells beneath them,\\*
And mossy scabs of the worm fence, heaped stones, elder, mullein \& poke-weed.
\end{verse}

\subsection{}

\blfootnote{$\mathbb{R}$ William Shakespeare (1564 -- 1616), \cite{obev}.}\settowidth{\versewidth}{    Than unswept stone, besmeared with sluttish time.}
\begin{verse}[\versewidth]
Not marble, nor the gilded monuments,\\*
\vin Of princes shall outlive this powerful rhyme;\\
But you shall shine more bright in these contents\\
\vin Than unswept stone, besmeared with sluttish time.\\
When wasteful war shall statues overturn,\\
\vin And broils root out the work of masonry,\\
Nor \textit{Mars} his sword nor war's quick fire shall burn\\
\vin The living record of your memory.\\
'Gainst death \& all-oblivious enmity\\
\vin Shall you pace forth; your praise shall still find room,\\
Even in the eyes of all posterity\\
\vin That wear this world out to the ending doom.\\
So, till the judgment that yourself arise,\\*
You live in this, and dwell in lovers' eyes.
\end{verse}

\subsection{}

\blfootnote{George Meredith (1828 -- 1909), \cite{pbev}. This is the first line of \refbook{Modern Love} XIII.}I play for seasons; not eternities.

\section{}

\subsection{}

\blfootnote{$\mathbb{R}$ `The Unequal Fetters', Anne Finch, Countess of Winchilsea (1661 -- 1720), \cite{obev}.}\settowidth{\versewidth}{    To love would then be worth our cost.}
\begin{verse}[\versewidth]
Could we stop the time that's flying\\*
\vin Or recall it when 'tis past,\\
Put far off the day of dying\\
\vin Or make youth for ever last,\\*
\vin To love would then be worth our cost.\\!

But since we must lose those graces\\*
\vin Which at first your hearts have won,\\
And you seek for in new faces\\
\vin When our spring of life is done,\\*
\vin It would but urge our ruin on.\\!

Free as nature's first intention\\*
\vin Was to make us, I'll be found,\\
Nor by subtle man's invention\\
\vin Yield to be in fetters bound\\*
\vin But one that walks a freer round.\\!

Marriage does but slightly tie men\\*
\vin Whilst close prisoners we remain;\\
They the larger slaves of \textit{Hymen}\\
\vin Still are begging love again\\*
\vin At the full length of all their chain.
\end{verse}

\subsection{}

\blfootnote{`To ---', Percy Shelley (1792 -- 1822), \cite{newlove}. Shelley, ever the prophet of the new secular post-Christian morality, wrote this poem as a means of propositioning his best friend's wife.}\settowidth{\versewidth}{I can give not what men call love;}
\begin{verse}[\versewidth]
One word is too often profaned\\*
\vin For me to profane it,\\
One feeling too falsely disdained\\
\vin For thee to disdain it.\\
One hope is too like despair\\
\vin For prudence to smother,\\
And pity from thee more dear\\*
\vin Than that from another.\\!

I can give not what men call love;\\*
\vin But wilt thou accept not\\
The worship the heart lifts above\\
\vin And the heavens reject not:\\
The desire of the moth for the star,\\
\vin Of the night for the morrow,\\
The devotion to something afar\\*
\vin From the sphere of our sorrow?
\end{verse}

\subsection{}

\blfootnote{William Shakespeare (1564 -- 1616), \cite{obev}. This is a dialogue between Romeo and Mercutio from \refbook{Romeo and Juliet} I.4.}\settowidth{\versewidth}{I dreamt a dream tonight. `And so did I.'}
\begin{verse}[\versewidth]
I dreamt a dream tonight. `And so did I.'\\*
Well what was yours? `That dreamers often lie.'
\end{verse}

\section{}

\subsection{}

\blfootnote{`The Green Linnet', Dr William Wordsworth, Poet Laureate (1770 -- 1850), \cite{treasury}.}\settowidth{\versewidth}{Beneath these fruit-tree boughs that shed}
\begin{verse}[\versewidth]
Beneath these fruit-tree boughs that shed\\*
Their snow-white blossoms on my head,\\
With brightest sunshine round me spread\\
\vin Of spring's unclouded weather,\\
In this sequestered nook how sweet\\
To sit upon my orchard-seat!\\
And birds \& flowers once more to greet,\\*
\vin My last year's friends together.\\!

One have I marked, the happiest guest\\*
In all this covert of the blest:\\
Hail to thee, far above the rest\\
\vin In joy of voice \& pinion!\\
Thou, linnet! in thy green array,\\
Presiding spirit here today,\\
Dost lead the revels of the may;\\*
\vin And this is thy dominion.\\!

While birds, \& butterflies, \& flowers,\\*
Make all one band of paramours,\\
Thou, ranging up \& down the bowers,\\
\vin Art sole in thy employment:\\
A life, a presence like the air,\\
Scattering thy gladness without care,\\
Too blest with any one to pair;\\*
\vin Thyself thy own enjoyment.\\!

Amid yon tuft of hazel trees,\\*
That twinkle to the gusty breeze,\\
Behold him perched in ecstasies,\\
\vin Yet seeming still to hover;\\
There! where the flutter of his wings\\
Upon his back \& body flings\\
Shadows \& sunny glimmerings,\\*
\vin That cover him all over.\\!

My dazzled sight he oft deceives,\\*
A brother of the dancing leaves;\\
Then flits, and from the cottage-eaves\\
\vin Pours forth his song in gushes;\\
As if by that exulting strain\\
He mocked \& treated with disdain\\
The voiceless form he chose to feign,\\*
\vin While fluttering in the bushes.
\end{verse}

\subsection{}

\blfootnote{`Love's Philosophy', Percy Shelley (1792 -- 1822), \cite{newlove}.}\settowidth{\versewidth}{The fountains mingle with the river}
\begin{verse}[\versewidth]
The fountains mingle with the river\\*
\vin And the rivers with the ocean,\\
The winds of heaven mix for ever\\
\vin With a sweet emotion;\\
Nothing in the world is single;\\
\vin All things by a law divine\\
In one spirit meet \& mingle.\\*
\vin Why not I with thine?\\!

See the mountains kiss high heaven\\*
\vin And the waves clasp one another;\\
No sister-flower would be forgiven\\
\vin If it disdained its brother;\\
And the sunlight clasps the earth\\
\vin And the moonbeams kiss the sea:\\
What is all this sweet work worth\\*
\vin If thou kiss not me?
\end{verse}

\subsection{}

\blfootnote{William Shakespeare (1564 -- 1616), \cite{shakespeare}. This famous line is uttered by Miranda in \refbook{The Tempest} V.1. It provided the title for Huxley's dystopia \refbook{Brave New World}.}O brave new world, that has such people in 't!

\section{}

\subsection{}

\blfootnote{Dr William Wordsworth, Poet Laureate (1770 -- 1850), \cite{treasury}.}\settowidth{\versewidth}{Endurance, foresight, strength, and skill;}
\begin{verse}[\versewidth]
She was a phantom of delight\\*
When first she gleamed upon my sight;\\
A lovely apparition, sent\\
To be a moment's ornament;\\
Her eyes as stars of twilight fair;\\
Like twilight's, too, her dusky hair;\\
But all things else about her drawn\\
From may-time and the cheerful dawn;\\
A dancing shape, an image gay,\\*
To haunt, to startle, and way-lay.\\!

I saw her upon nearer view,\\*
A spirit, yet a woman too!\\
Her household motions light \& free,\\
And steps of virgin-liberty;\\
A countenance in which did meet\\
Sweet records, promises as sweet;\\
A creature not too bright or good\\
For human nature's daily food;\\
For transient sorrows, simple wiles,\\*
Praise, blame, love, kisses, tears, and smiles.\\!

And now I see with eye serene\\*
The very pulse of the machine;\\
A being breathing thoughtful breath,\\
A traveller between life and death;\\
The reason firm, the temperate will,\\
Endurance, foresight, strength, and skill;\\
A perfect woman, nobly planned,\\
To warn, to comfort, and command;\\
And yet a spirit still, and bright\\*
With something of angelic light.
\end{verse}

\subsection{}

\blfootnote{$\mathbb{R}$ Aurelian Townshend (1583 -- 1651), \cite{obev}. This poem is sometimes printed under the title \refpoem{To the Lady Mary}.}\settowidth{\versewidth}{    Soft winds their breath, green trees their shade,}
\begin{verse}[\versewidth]
Your smiles are not, as other women's be,\\*
\vin Only the drawing of the mouth awry;\\
For breasts \& cheeks \& forehead we may see,\\
\vin Parts wanting motion, all stand smiling by:\\
Heaven hath no mouth, and yet is said to smile\\
\vin \vin \vin After your style:\\
No more hath earth, yet that smiles too,\\*
\vin \vin \vin Just as you do.\\!

No simpering lips nor looks can breed\\*
Such smiles as from your face proceed:\\
The sun must lend his golden beams,\\
\vin Soft winds their breath, green trees their shade,\\
Sweet fields their flowers, clear springs their streams,\\
\vin Ere such another smile be made:\\
But these concurring, we may say,\\*
So smiles the spring and so smiles lovely may.
\end{verse}

\subsection{}

\blfootnote{William Shakespeare (1564 -- 1616), \cite{shakespeare}. This line is uttered by Prospero in \refbook{The Tempest} IV.1.}The strongest oaths are straw to the fire in the blood.

\section{}

\subsection{}

\blfootnote{`Sailing to Byzantium', William Yeats (1865 -- 1939), \cite{norton}.}\settowidth{\versewidth}{-- Those dying generations -- at their song,}
\begin{verse}[\versewidth]
That is no country for old men. The young\\*
In one another's arms, birds in the trees,\\
-- Those dying generations -- at their song,\\
The salmon-falls, the mackerel-crowded seas,\\
Fish, flesh, or fowl, commend all summer long\\
Whatever is begotten, born, and dies.\\
Caught in that sensual music all neglect\\*
Monuments of unageing intellect.\\!

An aged man is but a paltry thing,\\*
A tattered coat upon a stick, unless\\
Soul clap its hands \& sing, and louder sing\\
For every tatter in its mortal dress,\\
Nor is there singing school but studying\\
Monuments of its own magnificence;\\
And therefore I have sailed the seas and come\\*
To the holy city of \textsc{Byzantium}.\\!

O sages standing in God's holy fire\\*
As in the gold mosaic of a wall,\\
Come from the holy fire, perne in a gyre,\\
And be the singing-masters of my soul.\\
Consume my heart away; sick with desire\\
And fastened to a dying animal\\
It knows not what it is; and gather me\\*
Into the artifice of eternity.\\!

Once out of nature I shall never take\\*
My bodily form from any natural thing,\\
But such a form as grecian goldsmiths make\\
Of hammered gold \& gold enamelling\\
To keep a drowsy emperor awake;\\
Or set upon a golden bough to sing\\
To lords \& ladies of \textsc{Byzantium}\\*
Of what is passed, or passing, or to come.
\end{verse}

\subsection{}

\blfootnote{$\mathbb{R}$ `On His Returne from Spaine', Sir Thomas Wyatt (1503 -- 1542), \cite{pbev}. This poem appears in other sources with most of the lines changed subtlely; but the Almanacker finds this alternative version much inferior. \P 5. According to Geoffrey of Monmouth's \textit{Historia Regum Britanniae}, Brutus of Troy was inspired to found the city of London in a dream.}\settowidth{\versewidth}{Tagus, farewell, that westward, with thy streams,}
\begin{verse}[\versewidth]
\textsc{Tagus}, farewell, that westward, with thy streams,\\*
\vin Turns up the grains of gold already tried,\\
For I, with spur \& sail, go seek the \textsc{Thames},\\
\vin Gainward the sun that show'th her wealthy pride,\\
And to the town which \textit{Brutus} sought by dreams,\\
Like bended moon that leans her lusty side.\\
My king, my country, I seek for whom I live;\\*
O mighty \textit{Jove}, the winds for this me give.
\end{verse}

\subsection{}

\blfootnote{William Shakespeare (1564 -- 1616), \cite{shakespeare}. This line is uttered by Caliban in \refbook{The Tempest} I.2.}\settowidth{\versewidth}{You taught me language; and my profit on 't}
\begin{verse}[\versewidth]
You taught me language; and my profit on 't\\*
Is, I know how to curse.
\end{verse}

\section{}

\subsection{}

\blfootnote{`Byzantium', William Yeats (1865 -- 1939), \cite{norton}.}\settowidth{\versewidth}{Flames that no faggot feeds, nor steel has lit,}
\begin{verse}[\versewidth]
The unpurged images of day recede;\\*
The emperor's drunken soldiery are abed;\\
Night resonance recedes, night-walkers' song\\
After great cathedral gong;\\
A starlit or a moonlit dome disdains\\
All that man is,\\
All mere complexities,\\*
The fury and the mire of human veins.\\!

Before me floats an image, man or shade,\\*
Shade more than man, more image than a shade;\\
For Hades' bobbin bound in mummy-cloth\\
May unwind the winding path;\\
A mouth that has no moisture \& no breath\\
Breathless mouths may summon;\\
I hail the superhuman;\\*
I call it death-in-life and life-in-death.\\!

Miracle, bird or golden handiwork,\\*
More miracle than bird or handiwork,\\
Planted on the starlit golden bough,\\
Can like the cocks of Hades crow,\\
Or, by the moon embittered, scorn aloud\\
In glory of changeless metal\\
Common bird or petal\\*
And all complexities of mire or blood.\\!

At midnight on the Emperor's pavement flit\\*
Flames that no faggot feeds, nor steel has lit,\\
Nor storm disturbs, flames begotten of flame,\\
Where blood-begotten spirits come\\
And all complexities of fury leave,\\
Dying into a dance,\\
An agony of trance,\\*
An agony of flame that cannot singe a sleeve.\\!

Astraddle on the dolphin's mire and blood,\\*
Spirit after spirit! The smithies break the flood,\\
The golden smithies of the Emperor!\\
Marbles of the dancing floor\\
Break bitter furies of complexity,\\
Those images that yet\\
Fresh images beget,\\*
That dolphin-torn, that gong-tormented sea.
\end{verse}

\subsection{}

\blfootnote{$\mathbb{R}$ `A Renouncing of Love', Sir Thomas Wyatt (1503 -- 1542), \cite{pbev}. \P 14. Other sources give: `Me lusteth no lenger rotten boughs to climb.'}\settowidth{\versewidth}{Therefore, farewell; go trouble younger hearts}
\begin{verse}[\versewidth]
Farewell, love, and all thy laws forever.\\*
\vin Thy baited hooks shall tangle me no more.\\
\vin \textit{Senec} and \textit{Plato} call me from thy lore\\
To perfect wealth, my wit for to endeavour.\\
In blind error when I did persever,\\
\vin Thy sharp repulse, that pricketh ay so sore,\\
\vin Hath taught me to set in trifles no store\\
And scape forth, since liberty is lever.\\
Therefore, farewell; go trouble younger hearts\\
\vin And in me claim no more authority.\\
\vin With idle youth go use thy property\\
And thereon spend thy many brittle darts,\\
For hitherto though I have lost all my time,\\*
Me list no longer rotten boughs to climb.
\end{verse}

\subsection{}

\blfootnote{William Shakespeare (1564 -- 1616), \cite{shakespeare}. This line is uttered by Miranda in \refbook{The Tempest} I.2.}Your tale, sir, would cure deafness.

\chapter{Tertilis}

\section{}

\subsection{}

\blfootnote{`To Charlotte Pulteney', Ambrose Philips (1674 -- 1749), \cite{treasury}.}\settowidth{\versewidth}{    Sleeping, waking, still at ease,}
\begin{verse}[\versewidth]
\vin Timely blossom, infant fair,\\*
\vin Fondling of a happy pair,\\
\vin Every morn \& every night\\
\vin Their solicitous delight,\\
\vin Sleeping, waking, still at ease,\\
\vin Pleasing, without skill to please;\\
\vin Little gossip, blithe \& hale,\\
\vin Tattling many a broken tale,\\
\vin Singing many a tuneless song,\\
\vin Lavish of a heedless tongue;\\
\vin Simple maiden, void of art,\\
\vin Babbling out the very heart,\\
\vin Yet abandoned to thy will,\\
\vin Yet imagining no ill,\\
\vin Yet too innocent to blush;\\
\vin Like the linnet in the bush\\
\vin To the mother-linnet's note\\
\vin Moduling her slender throat;\\
\vin Chirping forth thy petty joys,\\
\vin Wanton in the change of toys,\\
\vin Like the linnet green, in may\\
\vin Flitting to each bloomy spray;\\
\vin Wearied then \& glad of rest,\\
\vin Like the linnet in the nest:--\\
\vin This thy present happy lot,\\
\vin This in time will be forgot:\\
\vin Other pleasures, other cares,\\
\vin Ever-busy time prepares;\\
And thou shalt in thy daughter see,\\*
This picture, once, resembled thee.
\end{verse}

\subsection{}

\blfootnote{`The Clod \& the Pebble', William Blake (1757 -- 1827), \cite{blakea}.}\settowidth{\versewidth}{    Trodden with the cattle's feet,}
\begin{verse}[\versewidth]
Love seeketh not itself to please,\\*
\vin Nor for itself hath any care,\\
But for another gives its ease,\\*
\vin And builds a heaven in hell's despair.\\!

So sung a little clod of clay\\*
\vin Trodden with the cattle's feet,\\
But a pebble of the brook\\*
\vin Warbled out these metres meet:\\!

Love seeketh only self to please,\\*
\vin To bind another to its delight,\\
Joys in another's loss of ease,\\*
\vin And builds a hell in heaven's despite.
\end{verse}

\subsection{}

\blfootnote{Miss Mary Astell (1668 -- 1731), \cite{odq}.}Fetters of gold are still fetters.

\section{}

\subsection{}

\blfootnote{`A Mother to Her Waking Infant', Miss Joanna Baillie (1762 -- 1851), \cite{norton}.}\settowidth{\versewidth}{Thy mouth is worn with old wives' kissing;}
\begin{verse}[\versewidth]
Now in thy dazzling \sfrac{$1$}{$2$} oped eye,\\*
Thy curl\`{e}d nose \& lip awry,\\
Uphoisted arms \& noddling head,\\
And little chin with crystal spread,\\
Poor helpless thing, what do I see,\\*
\vin That I should sing of thee?\\!

From thy poor tongue no accents come,\\*
Which can but rub thy toothless gum:\\
Small understanding boasts thy face,\\
Thy shapeless limbs nor step nor grace:\\
A few short words thy feats may tell,\\*
\vin And yet I love thee well.\\!

When wakes the sudden bitter shriek,\\*
And redder swells thy little cheek\\
When rattled keys thy woes beguile,\\
And through thine eyelids gleams the smile,\\
Still for thy weakly self is spent\\*
\vin Thy little silly plaint.\\!

But when thy friends are in distress,\\*
Thou’lt laugh and chuckle ne'er the less,\\
Nor with kind sympathy be smitten,\\
Though all are sad but thee \& kitten;\\
Yet puny varlet that thou art,\\*
\vin Thou twitchest at the heart.\\!

Thy smooth round cheek so soft \& warm;\\*
Thy pinky hand \& dimpled arm;\\
Thy silken locks that scantly peep,\\
With gold tipped ends, where circle deep,\\
Around thy neck in harmless grace,\\
So soft and sleekly hold their place,\\
Might harder hearts with kindness fill,\\*
\vin And gain our right goodwill.\\!

Each passing clown bestows his blessing,\\*
Thy mouth is worn with old wives' kissing;\\
E'en lighter looks the gloomy eye\\
Of surly sense when thou art by;\\
And yet, I think, whoe'er they be,\\*
\vin They love thee not like me.\\!

Perhaps when time shall add a few\\*
Short years to thee, thou'lt love me too;\\
And after that, through life’s long way,\\
Become my sure and cheering stay;\\
Wilt care for me and be my hold,\\*
\vin When I am weak and old.\\!

Thou'lt listen to my lengthened tale,\\*
And pity me when I am frail --\\
But see, the sweepy spinning fly\\
Upon the window takes thine eye.\\
Go to thy little senseless play;\\*
\vin Thou dost not heed my lay.
\end{verse}

\subsection{}

\blfootnote{William Blake (1757 -- 1827), \cite{blakea}. These lines are taken from \refpoem{Several Questions Answered}.}\settowidth{\versewidth}{Because 'tis filled with fire,}
\begin{verse}[\versewidth]
The look of love alarms\\*
Because 'tis filled with fire,\\
But the look of soft deceit\\*
Shall win the lover's hire.\\!

Soft deceit \& idleness,\\*
These are beauty's sweetest dress.
\end{verse}

\subsection{}

\blfootnote{Mrs Aphra Behn (1640 -- 1689), \cite{odq}.}Money speaks sense in a language all nations understand.

\section{}

\subsection{}

\blfootnote{$\mathbb{R}$ Dr Thomas Beddoes (1803 -- 1849), \cite{pbev}. The first verse is to be sung `By female voices', and the second by male.}\settowidth{\versewidth}{            And he, who shall embrace thee,}
\begin{verse}[\versewidth]
We have bathed, where none have seen us,\\*
\vin In the lake \& in the fountain,\\
\vin \vin Underneath the charm{\`{e}}d statue\\
Of the timid, bending \textit{Venus},\\
\vin When the water nymphs were counting\\
In the waves the stars of night,\\
\vin \vin And those maidens started at you,\\
Your limbs shone through so soft \& bright.\\
\vin \vin But no secrets dare we tell,\\
\vin \vin \vin For thy slaves unlace thee,\\
\vin \vin \vin And he, who shall embrace thee,\\*
\vin \vin Waits to try thy beauty's spell.\\!

`We have crowned thee queen of women,\\*
\vin Since love's love, the rose, hath kept her\\
\vin \vin Court within thy lips \& blushes,\\
And thine eye, in beauty swimming,\\
\vin Kissing, we rendered up the sceptre,\\
At whose touch the startled soul\\
\vin \vin Like an ocean bounds \& gushes,\\
And spirits bend at thy control.\\
\vin \vin But no secrets dare we tell,\\
\vin \vin \vin For thy slaves unlace thee,\\
\vin \vin \vin And he, who shall embrace thee,\\*
\vin \vin Is at hand, and so farewell.'
\end{verse}

\subsection{}

\blfootnote{William Blake (1757 -- 1827), \cite{pbev}. These are lines 28-44 of \S 31 (in Book the Second) of Blake's long poem \refbook{Milton}.}\settowidth{\versewidth}{Then loud from their green covert all the birds begin their song:}
\begin{verse}[\versewidth]
Thou hearest the nightingale begin the song of spring.\\*
The lark sitting upon his earthy bed, just as the morn\\
Appears, listens silent; then springing from the waving cornfield, loud\\
He leads the choir of day: trill, trill, trill, trill,\\
Mounting upon the wings of light into the great expanse,\\
Re-echoing against the lovely blue \& shining heavenly shell,\\
His little throat labours with inspiration; every feather\\
On throat \& breast \& wings vibrates with the effluence divine\\
All nature listens silent to him, \& the awful sun\\
Stands still upon the mountain looking on this little bird\\
With eyes of soft humility \& wonder, love \& awe,\\
Then loud from their green covert all the birds begin their song:\\
The thrush, the linnet \& the goldfinch, robin \& the wren\\
Awake the sun from his sweet reverie upon the mountain.\\
The nightingale again assays his song, \& thro' the day\\
And thro' the night warbles luxuriant, every bird of song\\*
Attending his loud harmony with admiration \& love.
\end{verse}

\subsection{}

\blfootnote{`Fatigue', Hilaire Belloc (1870 -- 1953), \cite{oxfordlarkin}.}\settowidth{\versewidth}{I'm tired of love: I'm still more tired of rhyme.}
\begin{verse}[\versewidth]
I'm tired of love: I'm still more tired of rhyme.\\*
But money gives me pleasure all the time.
\end{verse}

\section{}

\subsection{}

\blfootnote{`To the Fair Clarinda, Who Made Love to Me, Imagined More than Woman', Mrs Aphra Behn (1640 -- 1689), \cite{norton}. The last line is perhaps a reference to the deity Hermaphroditus.}\settowidth{\versewidth}{For who, that gathers fairest flowers believes}
\begin{verse}[\versewidth]
Fair lovely maid, or if that title be\\*
Too weak, too feminine for nobler thee,\\
Permit a name that more approaches truth:\\
And let me call thee lovely charming youth.\\
This last will justify my soft complaint,\\
While that may serve to lessen my constraint;\\
And without blushes I the youth pursue,\\
When so much beauteous woman is in view.\\
Against thy charms we struggle but in vain\\
With thy deluding form thou giv'st us pain,\\
While the bright nymph betrays us to the swain.\\
In pity to our sex sure thou wert sent,\\
That we might love, and yet be innocent:\\
For sure no crime with thee we can commit;\\
Or if we should -- thy form excuses it.\\
For who, that gathers fairest flowers believes\\*
A snake lies hid beneath the fragrant leaves.\\!

Thou beauteous wonder of a different kind,\\*
Soft \textit{Chloris} with the dear \textit{Alexis} joined;\\
When e'er the manly part of thee, would plead\\
Thou tempts us with the image of the maid,\\
While we the noblest passions do extend\\*
The love to \textit{Hermes}, \textit{Aphrodite} the friend.
\end{verse}

\subsection{}

\blfootnote{William Blake (1757 -- 1827), \cite{pbev}. These lines are one couplet away from closing Blake's \refpoem{The Book of Thel}.}\settowidth{\versewidth}{Or an eye of gifts graces, showering fruits coined gold?}
\begin{verse}[\versewidth]
Why cannot the ear be closed to its own destruction?\\*
Or the glistening eye to the poison of a smile?\\
Why are eyelids stored with arrows ready drawn,\\
Where a 1000 fighting men in ambush lie?\\
Or an eye of gifts \& graces, showering fruits \& coin{\`{e}}d gold?\\
Why a tongue impressed with honey from every wind?\\
Why an ear, a whirlpool fierce to draw creations in?\\
Why a nostril wide inhaling terror, trembling, and affright?\\
Why a tender curb upon the youthful burning boy?\\*
Why a little curtain of flesh on the bed of our desire?
\end{verse}

\subsection{}

\blfootnote{`The Pacifist', Hilaire Belloc (1870 -- 1953), \cite{odq}.}\settowidth{\versewidth}{Pale Ebenezer thought it wrong to fight,}
\begin{verse}[\versewidth]
Pale \textit{Ebenezer} thought it wrong to fight,\\*
But \textit{Roaring Bill} (who killed him) thought it right.
\end{verse}

\section{}

\subsection{}

\blfootnote{`A Little Girl Lost', William Blake (1757 -- 1827), \cite{blakea}. Be careful not to confuse this poem with \refpoem{\emph{The} Little Girl Lost}.}\settowidth{\versewidth}{That shakes the blossoms of my hoary hair!}
\begin{verse}[\versewidth]
\vin In the age of gold,\\*
\vin Free from winter's cold,\\
\vin Youth \& maiden bright,\\
\vin To the holy light,\\*
Naked in the sunny beams' delight.\\!

\vin Once a youthful pair,\\*
\vin Filled with softest care,\\
\vin Met in garden bright\\
\vin Where the holy light\\*
Had just removed the curtains of the night.\\!

\vin There, in rising day,\\*
\vin On the grass they play;\\
\vin Parents were afar;\\
\vin Strangers came not near,\\*
And the maiden soon forgot her fear.\\!

\vin Tired with kisses sweet,\\*
\vin They agree to meet\\
\vin When the silent sleep\\
\vin Waves o'er heaven's deep,\\*
And the weary tired wanderers weep.\\!

\vin To her father white\\*
\vin Came the maiden bright;\\
\vin But his loving look,\\
\vin Like the holy book,\\*
All her tender limbs with terror shook.\\!

\vin \textit{Ona}, pale and weak,\\*
\vin To thy father speak.\\
\vin O the trembling fear!\\
\vin O the dismal care\\*
That shakes the blossoms of my hoary hair!
\end{verse}

\subsection{}

\blfootnote{$\mathbb{R}$ Dr Thomas Campion (1567 -- 1620), \cite{pbev}.}\settowidth{\versewidth}{Then burst with sighing in her sight and ne'er return again.}
\begin{verse}[\versewidth]
Follow your saint, follow with accents sweet;\\*
Haste you, sad notes, fall at her flying feet.\\
There, wrapped in cloud of sorrow, pity move,\\
And tell the ravisher of my soul I perish for her love:\\
But if she scorns my never-ceasing pain,\\*
Then burst with sighing in her sight and ne'er return again.\\!

All that I sung still to her praise did tend,\\*
Still she was first; still she my songs did end;\\
Yet she my love \& music both doth fly,\\
The music that her echo is and beauty's sympathy.\\
Then let my notes pursue her scornful flight:\\*
It shall suffice that they were breathed and died for her delight.
\end{verse}

\subsection{}

\blfootnote{Valentine Blacker (1778 -- 1826), \cite{odq}. Sometimes attributed to Oliver Cromwell.}Put your trust in God, my boys, and keep your powder dry.

\section{}

\subsection{}

\blfootnote{`The Little Girl Lost', William Blake (1757 -- 1827), \cite{blakea}. Blake also wrote a poem called \refpoem{The Little Girl Found}.}\settowidth{\versewidth}{`Frowning, frowning night,}
\begin{verse}[\versewidth]
In futurity\\*
I prophetic see\\
That the earth from sleep\\*
(Grave the sentence deep)\\!

Shall arise and seek\\*
For her maker meek;\\
And in the desert wild\\*
Become a garden mild.\\!

In the southern clime,\\*
Where the summer's prime\\
Never fades away,\\*
Lovely \textit{Lyca} lay.\\!

Seven summers old\\*
Lovely \textit{Lyca} told;\\
She had wandered long\\*
Hearing wild birds' song.\\!

`Sweet sleep, come to me\\*
Underneath this tree.\\
Do father, mother weep,\\*
Where can \textit{Lyca} sleep?\\!

`Lost in desert wild\\*
Is your little child.\\
How can \textit{Lyca} sleep\\*
If her mother weep?\\!

`If her heart does ache\\*
Then let \textit{Lyca} wake;\\
If my mother sleep,\\*
\textit{Lyca} shall not weep.\\!

`Frowning, frowning night,\\*
O'er this desert bright,\\
Let thy moon arise\\*
While I close my eyes.'\\!

Sleeping \textit{Lyca} lay,\\*
While the beasts of prey\\
Come from caverns deep,\\*
Viewed the maid asleep.\\!

The kingly lion stood\\*
And the virgin viewed,\\
Then he gambolled round\\*
O'er the hollowed ground.\\!

Leopards, tygers, play\\*
Round her as she lay,\\
While the lion old\\*
Bowed his mane of gold;\\!

And her bosom lick,\\*
And upon her neck;\\
From his eyes of flame\\*
Ruby tears there came;\\!

While the lioness\\*
Loosed her slender dress,\\
And naked they conveyed\\*
To caves the sleeping maid.
\end{verse}

\subsection{}

\blfootnote{Dr Thomas Campion (1567 -- 1620), \cite{londonbook}.}\settowidth{\versewidth}{These screech-owl's feathers this prickling briar,}
\begin{verse}[\versewidth]
Thrice toss these oaken ashes in the air,\\*
Thrice sit thou mute in this enchanted chair,\\
Then thrice three times tie up this true love's knot,\\*
And murmur soft, She will, or she will not.\\!

Go burn these pois'nous weeds in yon blue fire,\\*
These screech-owl's feathers \& this prickling briar,\\
This cypress gathered at a dead man's grave,\\*
That all my fears \& cares an end may have.\\!

Then come, you fairies, dance with me a round;\\*
Melt her hard heart with your melodious sound.\\
In vain are all the charms I can devise:\\*
She hath an art to break them with her eyes.
\end{verse}

\subsection{}

\blfootnote{William Blake (1757 -- 1827), \cite{blakea}. This is one of Blake's `Proverbs of Hell' from \refbook{The Marriage of Heaven and Hell}.}A dead body revenges not injuries.

\section{}

\subsection{}

\blfootnote{`A Dream', William Blake (1757 -- 1827), \cite{blakea}.}\settowidth{\versewidth}{Troubled, wildered, and forlorn,}
\begin{verse}[\versewidth]
Once a dream did weave a shade\\*
O'er my angel-guarded bed\\
That an emmet lost its way\\*
Where on grass methought I lay.\\!

Troubled, wildered, and forlorn,\\*
Dark, benighted, travel-worn,\\
Over many a tangle spray,\\*
All heart-broke, I heard her say:\\!

`Oh my children! Do they cry?\\*
Do they hear their father sigh?\\
Now they look abroad to see,\\*
Now return and weep for me.'\\!

Pitying, I dropped a tear:\\*
But I saw a glow-worm near\\
Who replied, `What wailing wight\\*
Calls the watchman of the night?\\!

`I am set to light the ground\\*
While the beetle goes his round:\\
Follow now the beetle's hum;\\*
Little wanderer, hie thee home.
\end{verse}

\subsection{}

\blfootnote{$\mathbb{R}$ Dr Thomas Campion (1567 -- 1620), \cite{pbev}.}\settowidth{\versewidth}{Thus will I mourne, thus will I sing:}
\begin{verse}[\versewidth]
What then is love but mourning?\\*
What desire, but a self-burning?\\
Till she that hates doth love returne,\\
Thus will I mourne, thus will I sing:\\*
Come away, come away, my darling.\\!

Beautie is but a blooming,\\*
Youth in his glory entombing;\\
Time hath a wheel which none can stay:\\
Then come away, while thus I sing:\\*
Come away, come away, my darling.\\!

Summer in winter fadeth;\\*
Gloomy night heavenly light shadeth;\\
Like to the morn are \textit{Venus} flowers;\\
Such are her howers: then will I sing:\\*
Come away, come away, my darling.
\end{verse}

\subsection{}

\blfootnote{William Blake (1757 -- 1827), \cite{blakea}. This is one of Blake's `Proverbs of Hell' from \refbook{The Marriage of Heaven and Hell}.}He who desires but acts not, breeds pestilence.

\section{}

\subsection{}

\blfootnote{`A Riddle', Miss Catherine Fanshawe (1765 -- 1834), \cite{obev}. The solution to the riddle is the letter H.}\settowidth{\versewidth}{'Twas in heaven pronounced, and 'twas muttered in hell,}
\begin{verse}[\versewidth]
'Twas in heaven pronounced, and 'twas muttered in hell,\\*
And echo caught faintly the sound as it fell:\\
On the confines of earth 'twas permitted to rest,\\
And the depths of the ocean its presence confessed;\\
'Twill be found in the sphere when 'tis riven asunder,\\
Be seen in the lightning, and heard in the thunder.\\
'Twas allotted to man with his earliest breath,\\
Attends at his birth, and awaits him in death,\\
Presides o'er his happiness, honor \& health,\\
Is the prop of his house, and the end of his wealth.\\
In the heaps of the miser 'tis hoarded with care,\\
But is sure to be lost on his prodigal heir.\\
It begins every hope, every wish it must bound,\\
With the husbandman toils, and with monarchs is crowned.\\
Without it the soldier, the seaman may roam,\\
But woe to the wretch who expels it from home.\\
In the whispers of conscience its voice will be found,\\
Nor e'en in the whirlwind of passion be drowned.\\
'Twill not soften the heart; but though deaf be the ear,\\
It will make it acutely \& instantly hear.\\
Yet in shade let it rest like a delicate flower;\\*
Ah breathe on it softly -- it dies in an hour.
\end{verse}

\subsection{}

\blfootnote{$\mathbb{R}$ Dr Thomas Campion (1567 -- 1620), \cite{pbev}.}\settowidth{\versewidth}{Smoke can never burn they say,}
\begin{verse}[\versewidth]
Young \& simple though I am,\\*
I have heard of \textit{Cupid}'s name;\\
Guess I can what thing it is\\
Men desire when they do kiss.\\
Smoke can never burn they say,\\*
But the flames that follow may.
\end{verse}

\subsection{}

\blfootnote{$\mathbb{R}$ William Blake (1757 -- 1827), \cite{blakea}. This is one of Blake's `Proverbs of Hell' from \refbook{The Marriage of Heaven and Hell}.}Shame is pride's cloak.

\section{}

\subsection{}

\blfootnote{$\mathbb{R}$ George Chapman (1559 -- 1634), \cite{obev}. This is a translation of Homer's Ἰλιάς XVIII.468-95. Prof Auden wrote his own poem (\refpoem{The Shield of Achilles}) concerning the same portion of Book XVIII.}\settowidth{\versewidth}{All sorts of blasts t'enflame his tempered coals;}
\begin{verse}[\versewidth]
This said, the smith did to his bellows go,\\*
Set them to fire, and made his cyclops blow:\\
Full 20 pair breathed through his furnace holes\\
All sorts of blasts t'enflame his tempered coals;\\
Now blusterd hard, and now did contrarise,\\
As \textit{Vulcan} would, and as his exercise\\
Might with perfection serve the dame's desire.\\
Hard brass \& tin he cast into the fire,\\
High-priz\`{e}d gold \& silver, and did set\\
Within the stock an anvil bright \& great:\\
His massy hammer then his right hand held;\\
His other hand his gasping tongs compelled.\\
And first he forged a huge \& solid shield,\\
Which every way did variant artship yield,\\
Through which he three ambitious circles cast,\\
Round \& refulgent; and without he placed\\
A silver handle; fivefold proof it was,\\
And in it many things with special grace,\\
And passing artificial pomp were graven;\\
In it was earth's green globe, the sea \& heaven,\\
Th'unwearied sun; the moon exactly round,\\
And all the stars with which the sky is crowned,\\
The \textit{Pleiades}, the \textit{Hyads} and the force\\
Of great \textit{Orion}; and the \textit{Bear}, whose course\\
Turns her about his sphere observing him\\
Surnamed the \textit{Chariot}, and doth never swim\\
Upon the unmeasured oceans' marble face,\\
Of all the flames that heaven's blue veil enchase.\\
In it two beautious cities he did build\\
Of divers-languaged men; the one was filled\\
With sacred nuptials \& with solemn feasts,\\
And through the streets the fair officious guests\\
Lead from their bridal chambers their fair brides\\
With golden torches burning by their sides.\\
\textit{Hymen}'s sweet triumphs were abundant there,\\
Of youths \& damsels dancing in a sphere;\\*
Amongst whom masking flutes \& harps were heard...
\end{verse}

\subsection{}

\blfootnote{George Chapman (1559 -- 1634), \cite{londonbook}. This is the first part of \refpoem{A Coronet for His Mistress Philosophy}.}\settowidth{\versewidth}{    Blown with the empty breath of vain desires;}
\begin{verse}[\versewidth]
Muses that sing love's sensual empery,\\*
\vin And lovers kindling your enrag{\`{e}}d fires\\
At \textit{Cupid}'s bonfires burning in the eye,\\
\vin Blown with the empty breath of vain desires;\\
You that prefer the painted cabinet\\
\vin Before the wealthy jewels it doth store ye,\\
That all your joys in dying figures set,\\
\vin And stain the living substance of your glory;\\
Abjure those joys, abhor their memory,\\
\vin And let my love the honoured subject be\\
Of love, and honour's complete history.\\
\vin Your eyes were never yet let in to see\\
The majesty \& riches of the mind,\\*
But dwell in darkness; for your god is blind.
\end{verse}

\subsection{}

\blfootnote{William Blake (1757 -- 1827), \cite{blakea}. This is one of Blake's `Proverbs of Hell' from \refbook{The Marriage of Heaven and Hell}.}The cut worm forgives the plough.

\section{}

\subsection{}

\blfootnote{Arthur Clough (1819 -- 1861), \cite{obev}. These lines are from Scene VI of Clough's \refbook{Dipsychus}.}\settowidth{\versewidth}{    The tradesman thinks, `'twere funny}
\begin{verse}[\versewidth]
`There is no God,' the wicked saith,\\*
\vin `And truly it's a blessing,\\
For what he might have done with us\\*
\vin It's better only guessing.'\\!

`There is no God,' a youngster thinks,\\*
\vin `Or really, if there may be,\\
He surely did not mean a man\\*
\vin Always to be a baby.'\\!

`There is no God, or if there is,'\\*
\vin The tradesman thinks, `'twere funny\\
If he should take it ill in me\\*
\vin To make a little money.'\\!

`Whether there be,' the rich man says,\\*
\vin `It matters very little,\\
For I \& mine, thank somebody,\\*
\vin Are not in want of victual.'\\!

Some others, also, to themselves,\\*
\vin Who scarce so much as doubt it,\\
Think there is none, when they are well,\\*
\vin And do not think about it.\\!

But country folks who live beneath\\*
\vin The shadow of the steeple;\\
The parson \& the parson's wife,\\*
\vin And mostly married people;\\!

Youths green \& happy in first love,\\*
\vin So thankful for illusion;\\
And men caught out in what the world\\*
\vin Calls guilt, in first confusion;\\!

And almost everyone when age,\\*
\vin Disease, or sorrows strike him,\\
Inclines to think there is a God,\\*
\vin Or something very like him.
\end{verse}

\subsection{}

\blfootnote{`The Latest Decalogue', Arthur Clough (1819 -- 1861), \cite{obev}. Clough attached a code -- mocking Christ's summary of the law -- to these lines.}\settowidth{\versewidth}{Will serve to keep the world thy friend.}
\begin{verse}[\versewidth]
Thou shalt have one God only; who\\*
Would be at the expense of two?\\
No graven images may be\\
Worshipped, except the currency.\\
Swear not at all; for for thy curse\\
Thine enemy is none the worse.\\
At church on sunday to attend\\
Will serve to keep the world thy friend.\\
Honor thy parents; that is, all\\
From whom advancement may befall.\\
Thou shalt not kill; but need'st not strive\\
Officiously to keep alive.\\
Do not adultery commit;\\
Advantage rarely comes of it.\\
Thou shalt not steal; an empty feat,\\
When it's so lucrative to cheat.\\
Bear not false witness; let the lie\\
Have time on its own wings to fly.\\
Thou shalt not covet, but tradition\\*
Approves all forms of competition.
\end{verse}

\subsection{}

\blfootnote{Philip Bliss (1838 -- 1876), \cite{odq}. This is the chorus to a hymn, inspired by a flag message from General Sherman.}Hold the fort, for I am coming.

\section{}

\subsection{}

\blfootnote{$\mathbb{R}$ `Laura Sleeping', Charles Cotton (1630 -- 1687), \cite{pbev}.}\settowidth{\versewidth}{Winds, whisper gently whilst she sleeps,}
\begin{verse}[\versewidth]
Winds, whisper gently whilst she sleeps,\\*
\vin And fan her with your cooling wings;\\
While she her drops of beauty weeps,\\*
\vin From pure, and yet unrivalled springs.\\!

Glide over beauty's field, her face,\\*
\vin To kiss her lip \& cheek be bold;\\
But with a calm \& stealing pace;\\*
\vin Neither too rude, nor yet too cold.\\!

Play in her beams, and crisp her hair\\*
\vin With such a gale as wings soft love,\\
And with so sweet, so rich an air,\\*
\vin As breathes from the arabian grove.\\!

A breath as hushed as lover's sigh;\\*
\vin Or that unfolds the morning's door:\\
Sweet as the winds that gently fly\\*
\vin To sweep the spring's enamelled floor.\\!

Murmur soft music to her dreams,\\*
\vin That pure \& unpolluted run\\
Like to the new-born crystal streams,\\*
\vin Under the bright enamoured sun.\\!

But when she walking shall display,\\*
\vin Her light, retire within your bar;\\
Her breath is life, her eyes are day,\\*
\vin And all mankind her creatures are.
\end{verse}

\subsection{}

\blfootnote{$\mathbb{R}$ `To the Queen, Entertain'd at Night by the Countess of Anglesey', Sir William Davenant (1606 -- 1668), \cite{pbev}.}\settowidth{\versewidth}{Dares praise, with such full art, what make you here?}
\begin{verse}[\versewidth]
Fair as unshaded light, or as the day\\*
In its first birth, when all the year was may;\\
Sweet as the altar's smoke, or as the new\\
Unfolded bud, swelled by the early dew;\\
Smooth as the face of waters first appeared,\\
Ere tides began to strive or winds were heard;\\
Kind as the willing saints, and calmer far\\
Than in their sleeps forgiven hermits are.\\
You that are more than our discreeter fear\\
Dares praise, with such full art, what make you here?\\
Here, where the summer is so little seen,\\
That leaves, her cheapest wealth, scarce reach at green;\\
You come, as if the silver planet were\\
Misled awhile from her much injured sphere;\\
And t'ease the travels of her beams tonight,\\*
In this small lantern would contract her light.
\end{verse}

\subsection{}

\blfootnote{The Rev Robert Burton (1577 -- 1640), \cite{odq}. The ultimate source of this idiom seems to be a passage from Plutarch's Ἠθικά, specifically the section containing the `Sayings of the Spartans'.}I call a spade a spade.

\section{}

\subsection{}

\blfootnote{Allan Cunningham (1784 -- 1842), \cite{treasury}.}\settowidth{\versewidth}{And bends the gallant mast, my boys,}
\begin{verse}[\versewidth]
A wet sheet \& a flowing sea,\\*
\vin A wind that follows fast\\
And fills the white \& rustling sail\\
\vin And bends the gallant mast;\\
And bends the gallant mast, my boys,\\
\vin While like the eagle free\\
Away the good ship flies, and leaves\\*
\vin Old England on the lee.\\!

`O for a soft \& gentle wind!'\\*
\vin I heard a fair one cry:\\
But give to me the snoring breeze\\
\vin And white waves heaving high;\\
And white waves heaving high, my lads,\\
\vin The good ship tight \& free --\\
The world of waters is our home,\\*
\vin And merry men are we.\\!

There's tempest in yon horn\`{e}d moon,\\*
\vin And lightning in yon cloud:\\
But hark the music, mariners!\\
\vin The wind is piping loud;\\
The wind is piping loud, my boys,\\
\vin The lightning flashes free --\\
While the hollow oak our palace is,\\*
\vin Our heritage the sea.
\end{verse}

\subsection{}

\blfootnote{$\mathbb{R}$ `A Song of the River Thames', John Dryden, Poet Laureate (1631 -- 1700), \cite{pbev}.}\settowidth{\versewidth}{All the calmer gales befriend thee}
\begin{verse}[\versewidth]
Old Father \textit{Ocean} calls my tide:\\*
\vin Come away; come away.\\
The barks upon the billows ride;\\
\vin The master will not stay:\\
The merry boatswain from his side\\
His whistle takes to check \& chide\\
\vin The lingering lad's delay,\\
And all the crew aloud has cried,\\*
\vin Come away; come away.\\!

See the god of seas attends thee,\\*
\vin Nymphs divine, a beauteous train;\\
All the calmer gales befriend thee\\
\vin In thy passage o'er the main:\\
Every maid her locks is binding;\\
Every \textit{Triton}'s horn is winding;\\*
\vin Welcome to the watery plain.
\end{verse}

\subsection{}

\blfootnote{George Noel, 6th Baron Byron (1788 -- 1824), \cite{odq}. This is a line from the third canto of \refbook{Childe Harold's Pilgrimage}.}Quiet to quick bosoms is a hell.

\section{}

\subsection{}

\blfootnote{$\mathbb{R}$ `The Good Morrow', The Very Rev Dr John Donne (1572 -- 1631), \cite{pbev}.}\settowidth{\versewidth}{Which I desired, and got, 'twas but a dream of thee.}
\begin{verse}[\versewidth]
I wonder, by my troth, what thou \& I\\*
\vin Did, till we loved? Were we not weaned till then?\\
But sucked on country pleasures, childishly?\\
\vin Or snorted we in the Seven Sleepers' den?\\
'Twas so; but this, all pleasures fancies be.\\
If ever any beauty I did see,\\*
Which I desired, and got, 'twas but a dream of thee.\\!

And now good-morrow to our waking souls,\\*
\vin Which watch not one another out of fear;\\
For love, all love of other sights controls,\\
\vin And makes one little room an everywhere.\\
Let sea-discoverers to new worlds have gone,\\
Let maps to other, worlds on worlds have shown,\\*
Let us possess one world; each hath one, and is one.\\!

My face in thine eye, thine in mine appears,\\*
\vin And true plain hearts do in the faces rest;\\
Where can we find two better hemispheres,\\
\vin Without sharp north, without declining west?\\
Whatever dies, was not mixed equally;\\
If our two loves be one, or, thou and I\\*
Love so alike, that none do slacken, none can die.
\end{verse}

\subsection{}

\blfootnote{`Hymn: Sung at the Completion of the Concord Monument, April 19, 1838', The Rev Prof Ralph Emerson (1803 -- 1882), \cite{pbev}. The Battles of Lexington and Concord were the first engagements of the American Revolutionary War.}\settowidth{\versewidth}{    And fired the shot heard round the world.}
\begin{verse}[\versewidth]
By the rude bridge that arched the flood,\\*
\vin Their flag to april's breeze unfurled,\\
Here once the embattled farmers stood,\\*
\vin And fired the shot heard round the world.\\!

The foe long since in silence slept;\\*
\vin Alike the conqueror silent sleeps;\\
And time the ruined bridge has swept\\*
\vin Down the dark stream which seaward creeps.\\!

On this green bank, by this soft stream,\\*
\vin We set today a votive stone;\\
That memory may their deed redeem,\\*
\vin When, like our sires, our sons are gone.\\!

Spirit, that made those heroes dare,\\*
\vin To die, and leave their children free,\\
Bid time \& nature gently spare\\*
\vin The shaft we raise to them \& thee.
\end{verse}

\subsection{}

\blfootnote{Mrs Carrie Chapman-Catt (1859 -- 1947), \cite{odq}.}No written law has been more binding than unwritten custom supported by popular opinion.

\section{}

\subsection{}

\blfootnote{`Love's Emblems', John Fletcher (1579 -- 1625), \cite{newlove}.}\settowidth{\versewidth}{        Ladies, if not plucked, we die.}
\begin{verse}[\versewidth]
Now the lusty spring is seen;\\*
\vin Golden yellow, gaudy blue,\\
\vin Daintily invite the view:\\
Everywhere on every green\\
Roses blushing as they blow\\
\vin And enticing men to pull,\\
Lilies whiter than the snow,\\
\vin Woodbines of sweet honey full:\\
\vin \vin All love's emblems, and all cry,\\*
\vin \vin Ladies, if not plucked, we die.\\!

Yet the lusty spring hath stayed;\\*
\vin Blushing red \& purest white\\
\vin Daintily to love invite\\
Every woman, every maid:\\
Cherries kissing as they grow,\\
\vin And inviting men to taste,\\
Apples even ripe below,\\
\vin Winding gently to the waist:\\
\vin \vin All love's emblems, and all cry,\\*
\vin \vin Ladies, if not plucked, we die.
\end{verse}

\subsection{}

\blfootnote{$\mathbb{R}$ John Fletcher (1579 -- 1625), \cite{pbev}. This song appears in the play \refbook{Bloody Brother}.}\settowidth{\versewidth}{    Lights that do mislead the morn;}
\begin{verse}[\versewidth]
Take O take those lips away\\*
\vin That so sweetly were forsworn,\\
And those eyes, like break of day,\\
\vin Lights that do mislead the morn;\\
But my kisses bring again,\\*
Seals of love, though sealed in vain.\\!

Hide O hide those hills of snow\\*
\vin Which thy frozen bosom bears,\\
On whose tops the pinks that grow\\
\vin Are of those that april wears;\\
But first set my poor heart free,\\*
Bound in those icy chains by thee.
\end{verse}

\subsection{}

\blfootnote{Charles, by the Grace of God, King of England, Scotland, France and Ireland, Defender of the Faith (1600 -- 1649), \cite{odq}.}A subject and a sovereign are clean different things.

\section{}

\subsection{}

\blfootnote{`The Quiet Glades of Eden', Prof Robert Graves (1895 -- 1985), \cite{newlove}.}\settowidth{\versewidth}{Enough, I will not claim a heart unfluttered}
\begin{verse}[\versewidth]
All such proclivities are tabulated\\*
By trained pathologists -- in detail too --\\
The obscener parts of speech compulsively\\*
\vin Shrouded in classic latin.\\!

But though my pleasure in your feet \& hair\\*
Is ungainsayable, let me protest\\
(Dear love) I am no trichomaniac\\*
\vin And no foot-fetichist.\\!

If it should please you, for your own best reasons,\\*
To take \& flog me with a rawhide whip,\\
I might (who knows?) suprisedly accept\\*
\vin This earnest of affection.\\!

Nothing, agreed, is alien to love\\*
When pure desire has overflowed its baulks;\\
But why must private sportiveness be viewed\\*
\vin Through public spectacles?\\!

Enough, I will not claim a heart unfluttered\\*
By these case-histories of aberrancy;\\
Nevertheless a long cool draught of water,\\*
\vin Or a long swim in the bay,\\!

Serves to restore my wholesome appetite\\*
For you \& what we do at night together:\\
Which is no more than \textit{Adam} did with \textit{Eve}\\*
\vin In the quiet glades of Eden.
\end{verse}

\subsection{}

\blfootnote{`The Soaking', Ivor Gurney (1890 -- 1937), \cite{obev}.}\settowidth{\versewidth}{Of its moisture, and the made roads, all dust clad;}
\begin{verse}[\versewidth]
The rain has come, and the earth must be very glad\\*
Of its moisture, and the made roads, all dust clad;\\
It lets a veil down on the lucent dark,\\*
And not of any bright ground thing shows its spark.\\!

Tomorrow's gray morning will show cow parsley,\\*
Hung all with shining drops, and the river will be\\
Duller because of the all soddenness of things,\\*
Till the skylark breaks his reluctance, hangs shaking, and sings.
\end{verse}

\subsection{}

\blfootnote{Philip Stanhope, 4th Earl of Chesterfield (1694 -- 1773), \cite{odq}.}An injury is much sooner forgotten than an insult.

\section{}

\subsection{}

\blfootnote{$\mathbb{R}$ `To Anthea, who may command him any thing', Robert Herrick (1591 -- 1674), \cite{obev}.}\settowidth{\versewidth}{Thou art my life, my love, my heart,}
\begin{verse}[\versewidth]
Bid me to live, and I will live\\*
\vin Thy protestant to be;\\
Or bid me love, and I will give\\*
\vin A loving heart to thee.\\!

A heart as soft, a heart as kind,\\*
\vin A heart as sound \& free,\\
As in the whole world thou canst find,\\*
\vin That heart I'll give to thee.\\!

Bid that heart stay, and it will stay,\\*
\vin To honour thy decree;\\
Or bid it languish quite away,\\*
\vin And 't shall do so for thee.\\!

Bid me to weep, and I will weep,\\*
\vin While I have eyes to see;\\
And having none, yet I will keep\\*
\vin A heart to weep for thee.\\!

Bid me despair, and I'll despair,\\*
\vin Under that cypress tree;\\
Or bid me die, and I will dare\\*
\vin E'en death, to die for thee.\\!

Thou art my life, my love, my heart,\\*
\vin The very eyes of me;\\
And hast command of every part,\\*
\vin To live and die for thee.
\end{verse}

\subsection{}

\blfootnote{$\mathbb{R}$ William Drummond of Hawthornden (1585 -- 1649), \cite{pbev}.}\settowidth{\versewidth}{    And while her pleasant rays abroad are rolled,}
\begin{verse}[\versewidth]
As, in a dusky \& tempestuous night,\\*
\vin A star is wont to spread her locks of gold,\\
\vin And while her pleasant rays abroad are rolled,\\
Some spiteful cloud doth rob us of her sight;\\
Fair soul, in this black age so shined thou bright,\\
\vin And made all eyes with wonder thee behold,\\
Till ugly death, depriving us of light,\\
\vin In his grim misty arms thee did enfold.\\
Who more shall vaunt true beauty here to see?\\
\vin What hope doth more in any heart remain,\\
\vin That such perfections shall his reason rein,\\
If beauty, with thee born, too died with thee?\\
World, plain no more of love, nor count his harms;\\*
With his pale trophies death hath hung his arms.
\end{verse}

\subsection{}

\blfootnote{Gilbert Chesterton, Knight (1874 -- 1936), \cite{odq}.}An inconvenience is only an adventure wrongly considered.

\section{}

\subsection{}

\blfootnote{Thomas Heywood (1572 -- 1641), \cite{treasury}.}\settowidth{\versewidth}{Wings from the wind to please her mind,}
\begin{verse}[\versewidth]
Pack, clouds, away, and welcome day,\\*
\vin With night we banish sorrow;\\
Sweet air blow soft, mount larks aloft\\
\vin To give my love good-morrow!\\
Wings from the wind to please her mind,\\
\vin Notes from the lark I'll borrow;\\
Bird, prune thy wing, nightingale sing,\\
\vin To give my love good-morrow;\\
\vin \vin To give my love good-morrow\\*
\vin \vin Notes from them both I'll borrow.\\!

Wake from thy nest, robin-red-breast,\\*
\vin Sing, birds, in every furrow;\\
And from each hill, let music shrill\\
\vin Give my fair love good-morrow!\\
Blackbird \& thrush in every bush,\\
\vin Stare, linnet, \& cock-sparrow!\\
You pretty elves, amongst yourselves\\
\vin Sing my fair love good-morrow;\\
\vin \vin To give my love good-morrow\\*
\vin \vin Sing, birds, in every furrow!
\end{verse}

\subsection{}

\blfootnote{$\mathbb{R}$ William Drummond of Hawthornden (1585 -- 1649), \cite{pbev}. \P 1. The word `idalian' refers to the city of Idalium, located near the more famous city of Nicosia, the former of which was the site a major shrine to Aphrodite. \P 6. In one telling of the ancient and intriguing myth of Venus and Adonis, Venus creates the red rose by shedding her blood on the petals of a white one. The earliest written source for this version is the Προγυμνάσματα of Aphthonius.}\settowidth{\versewidth}{In Cyprus' gardens gathering those fair flowers}
\begin{verse}[\versewidth]
Like the idalian queen,\\*
Her hair about her eyne,\\
With neck \& breasts ripe apples to be seen,\\
At first glance of the morn\\
In Cyprus' gardens gathering those fair flowers\\
Which of her blood were born,\\
I saw, but fainting saw, my paramours.\\
The Graces naked danced about the place;\\
The winds \& trees amazed\\
With silence on her gazed;\\
The flowers did smile, like those upon her face;\\
And as their aspen stalks those fingers band,\\
That she might read my case,\\*
A hyacinth I wished me in her hand.
\end{verse}

\subsection{}

\blfootnote{Edward Hyde, 1st Earl of Clarendon (1609 -- 1674), \cite{odq}. The `he' in question John Hampden, the Parliamentarian general.}When he first drew the sword, he threw away the scabbard.

\section{}

\subsection{}

\blfootnote{`Snake', David Lawrence (1885 -- 1930), \cite{norton}. Lawrence's own note indicates that he wrote this poem in Taormina in 1923.}\settowidth{\versewidth}{A sort of horror, a sort of protest against his withdrawing into that horrid black hole,}
\begin{verse}[\versewidth]
A snake came to my water-trough\\*
On a hot, hot day, and I in pyjamas for the heat,\\
To drink there.\\
In the deep, strange-scented shade of the great dark carob-tree\\
I came down the steps with my pitcher\\*
And must wait, must stand \& wait, for there he was at the trough before me.\\!

He reached down from a fissure in the earth-wall in the gloom\\*
And trailed his yellow-brown slackness soft-bellied down, over the edge of the stone trough\\
And rested his throat upon the stone bottom,\\
And where the water had dripped from the tap, in a small clearness,\\
He sipped with his straight mouth,\\
Softly drank through his straight gums, into his slack long body,\\*
Silently.\\!

Someone was before me at my water-trough,\\*
And I, like a second comer, waiting.\\!

He lifted his head from his drinking, as cattle do,\\*
And looked at me vaguely, as drinking cattle do,\\
And flickered his two-forked tongue from his lips, and mused a moment,\\
And stooped and drank a little more,\\
Being earth-brown, earth-golden from the burning bowels of the earth\\
On the day of sicilian july, with \textsc{Etna} smoking.\\
The voice of my education said to me,\\
He must be killed,\\*
For in Sicily the black, black snakes are innocent, the gold are venomous.\\!

And voices in me said, If you were a man\\*
You would take a stick and break him now, and finish him off.\\!

But must I confess how I liked him,\\*
How glad I was he had come like a guest in quiet, to drink at my water-trough\\
And depart peaceful, pacified, \& thankless,\\*
Into the burning bowels of this earth?\\!

Was it cowardice, that I dared not kill him?\\*
Was it perversity, that I longed to talk to him?\\
Was it humility, to feel so honoured?\\*
I felt so honoured.\\!

And yet those voices:\\*
If you were not afraid, you would kill him!\\!

And truly I was afraid, I was most afraid,\\*
But even so, honoured still more\\
That he should seek my hospitality\\*
From out the dark door of the secret earth.\\!

He drank enough\\*
And lifted his head, dreamily, as one who has drunken,\\
And flickered his tongue like a forked night on the air, so black,\\
Seeming to lick his lips,\\
And looked around like a god, unseeing, into the air,\\
And slowly turned his head,\\
And slowly, very slowly, as if thrice adream,\\
Proceeded to draw his slow length curving round\\*
And climb again the broken bank of my wall-face.\\!

And as he put his head into that dreadful hole,\\*
And as he slowly drew up, snake-easing his shoulders, and entered farther,\\
A sort of horror, a sort of protest against his withdrawing into that horrid black hole,\\
Deliberately going into the blackness, and slowly drawing himself after,\\*
Overcame me now his back was turned.\\!

I looked round, I put down my pitcher,\\*
I picked up a clumsy log\\*
And threw it at the water-trough with a clatter.\\!

I think it did not hit him,\\*
But suddenly that part of him that was left behind convulsed in undignified haste.\\
Writhed like lightning, and was gone\\
Into the black hole, the earth-lipped fissure in the wall-front,\\*
At which, in the intense still noon, I stared with fascination.\\!

And immediately I regretted it.\\*
I thought how paltry, how vulgar, what a mean act!\\*
I despised myself \& the voices of my accursed human education.\\!

And I thought of the albatross\\*
And I wished he would come back, my snake.\\!

For he seemed to me again like a king,\\*
Like a king in exile, uncrowned in the underworld,\\*
Now due to be crowned again.\\!

And so, I missed my chance with one of the lords\\*
Of life.\\
And I have something to expiate:\\*
A pettiness.
\end{verse}

\subsection{}

\blfootnote{`Echoes', Dr William Henley (1849 -- 1903), \cite{londonbook}. The \refbook{London Book of Verse} gives the title of this poem as \refpoem{Echoes}, although it is perhaps better known under the title \refpoem{Invictus}.}\settowidth{\versewidth}{    Black as the pit from pole to pole,}
\begin{verse}[\versewidth]
Out of the night that covers me\\*
\vin Black as the pit from pole to pole,\\
I thank whatever gods may be\\*
\vin For my unconquerable soul.\\!

In the fell clutch of circumstance,\\*
\vin I have not winced nor cried aloud.\\
Under the bludgeonings of chance\\*
\vin My head is bloody, but unbowed.\\!

Beyond this place of wrath \& tears\\*
\vin Looms but the horror of the shade,\\
And yet the menace of the years\\*
\vin Finds, and shall find me, unafraid.\\!

It matters not how strait the gate,\\*
\vin How charged with punishments the scroll,\\
I am the master of my fate:\\*
\vin I am the captain of my soul.
\end{verse}

\subsection{}

\blfootnote{Robert Clive, 1st Baron Clive (1725 -- 1774), \cite{odq}. Uttered during a parliamentary inquiry into his dealings in India.}I stand astonished at my own moderation.

\section{}

\subsection{}

\blfootnote{`Self-Protection', David Lawrence (1885 -- 1930), \cite{norton}.}\settowidth{\versewidth}{And I don't suppose the ichthyosaurus sparkled like the humming-bird.}
\begin{verse}[\versewidth]
When science starts to be interprative\\*
It is more unscientific even than mysticism.\\!

To make self-preservation \& self-protection the first law of existence\\*
Is about as scientific as making suicide the first law of existence,\\*
And amounts to very much the same thing.\\!

A nightingale singing at the top of his voice\\*
Is neither hiding himself nor preserving himself nor propagating his species;\\
He is giving himself away in every sense of the word;\\*
And obviously, it is the culminating point of his existence.\\!

A tiger is striped \& golden for his own glory.\\*
He would certainly be much more invisible if he were grey-green.\\
And I don't suppose the ichthyosaurus sparkled like the humming-bird.\\
No doubt, he was khaki-colored with muddy protective colouration,\\*
So why didn't he survive?\\!

As a matter of fact, the only creatures that seem to survive\\*
Are those that give themselves away in flash \& sparkle\\
And gay flicker of joyful life;\\
Those that go glittering abroad\\*
With a bit of splendor.\\!

Even mice play quite beautifully at shadows,\\*
And some of them are brilliantly piebald.\\!

I expect the dodo looked like a clod,\\*
A drab \& dingy bird.
\end{verse}

\subsection{}

\blfootnote{Edward Lear (1812 -- 1888), \cite{obev}. The Almanacker has omitted some weaker verses.}\settowidth{\versewidth}{    But a few think him pleasant enough.}
\begin{verse}[\versewidth]
How pleasant to know Mr \textit{Lear}!\\*
\vin Who has written such volumes of stuff!\\
Some think him ill-tempered \& queer,\\*
\vin But a few think him pleasant enough.\\!

His mind is concrete \& fastidious;\\*
\vin His nose is remarkably big;\\
His visage is more or less hideous;\\*
\vin His beard it resembles a wig.\\!

He sits in a beautiful parlour,\\*
\vin With 100s of books on the wall;\\
He drinks a great deal of marsala,\\*
\vin But never gets tipsy at all.\\!

He has many friends, lay men \& clerical;\\*
\vin \textit{Old Foss} is the name of his cat;\\
His body is perfectly spherical;\\*
\vin He weareth a runcible hat.\\!

He reads, but he cannot speak, spanish,\\*
\vin He cannot abide ginger beer:\\
Ere the days of his pilgrimage vanish,\\*
\vin How pleasant to know Mr \textit{Lear}!
\end{verse}

\subsection{}

\blfootnote{Cuthbert Collingwood, Baron Collingwood (1748 -- 1810), \cite{odq}. Uttered on the morning of the Battle of Trafalgar.}Let us do something today which the world may talk of hereafter.

\section{}

\subsection{}

\blfootnote{`Gloire de Dijon', David Lawrence (1885 -- 1930), \cite{newlove}.}\settowidth{\versewidth}{She spreads the bath-cloth underneath the window}
\begin{verse}[\versewidth]
When she rises in the morning\\*
I linger to watch her;\\
She spreads the bath-cloth underneath the window\\
And the sunbeams catch her\\
Glistening white on the shoulders,\\
While down her sides the mellow\\
Golden shadow glows as\\
She stoops to the sponge, and her swung breasts\\
Sway like full-blown yellow\\*
Gloire de Dijon roses.\\!

She drips herself with water, and her shoulders\\*
Glisten as silver, they crumple up\\
Like wet \& falling roses, and I listen\\
For the sluicing of their rain-dishevelled petals.\\
In the window full of sunlight\\
Concentrates her golden shadow\\
Fold on fold, until it glows as\\*
Mellow as the glory roses.
\end{verse}

\subsection{}

\blfootnote{$\mathbb{R}$ Andrew Marvell (1621 -- 1678), \cite{pbev}. These are verses 44 and 45 of Marvell's longer poem \refpoem{Upon Appleton House}. These two verses are said to describe a kingfisher. \P 4. For `eben shuts' read ``ebony shutters''. \P 7. The word `horror' in this context refers to awe, rather than fear and revulsion.}\settowidth{\versewidth}{From underneath these banks do creep,}
\begin{verse}[\versewidth]
So when the shadows laid asleep\\*
From underneath these banks do creep,\\
And on the river as it flows\\
With eben shuts begin to close;\\
The modest halcyon comes in sight,\\
Flying betwixt the day \& night;\\
And such an horror calm \& dumb,\\*
Admiring nature does benumb;\\!

The viscous air, wheresoe'er she fly,\\*
Follows and sucks her azure dye;\\
The gellying stream compacts below,\\
If it might fix her shadow so;\\
The stupid fishes hang, as plain\\
As flies in crystal overta'en,\\
And men the silent scene assist,\\*
Charmed with the sapphire-wing{\`{e}}d mist.
\end{verse}

\subsection{}

\blfootnote{William Congreve (1670 -- 1729), \cite{odq}.}I know that's a secret, for it's whispered everywhere.

\section{}

\subsection{}

\blfootnote{$\mathbb{R}$ `Rosalindes Madrigall', Thomas Lodge (1558 -- 1625), \cite{pbev}. This lines are taken from Lodge's \refbook{Rosalind, Euphues' Golden Legacy}.}\settowidth{\versewidth}{Now with his wings he plays with me,}
\begin{verse}[\versewidth]
Love in my bosom like a bee\\*
\vin Doth suck his sweet;\\
Now with his wings he plays with me,\\
\vin Now with his feet.\\
Within mine eyes he makes his nest,\\
His bed amidst my tender breast;\\
My kisses are his daily feast,\\
And yet he robs me of my rest.\\*
\vin Ah, wanton, will ye?\\!

And if I sleep, then percheth he\\*
\vin With pretty flight,\\
And makes his pillow of my knee\\
\vin The livelong night.\\
Strike I my lute, he tunes the string;\\
He music plays if so I sing;\\
He lends me every lovely thing;\\
Yet cruel he my heart doth sting.\\*
\vin Whist, wanton, still ye.\\!

Else I with roses every day\\*
\vin Will whip you hence,\\
And bind you, when you long to play,\\
\vin For your offense.\\
I'll shut mine eyes to keep you in,\\
I'll make you fast it for your sin,\\
I'll count your power not worth a pin.\\
Alas! what hereby shall I win\\*
\vin If he gainsay me?\\!

What if I beat the wanton boy\\*
\vin With many a rod?\\
He will repay me with annoy,\\
\vin Because a god.\\
Then sit thou safely on my knee,\\
And let thy bower my bosom be;\\
Lurk in mine eyes, I like of thee.\\
O \textit{Cupid}, so thou pity me,\\*
\vin Spare not, but play thee.
\end{verse}

\subsection{}

\blfootnote{$\mathbb{R}$ John Milton (1608 -- 1674), \cite{pbev}.}\settowidth{\versewidth}{All seasons, and their change, all please alike.}
\begin{verse}[\versewidth]
With thee conversing I forget all time;\\*
All seasons, and their change, all please alike.\\
Sweet is the breath of morn, her rising sweet,\\
With charm of earliest birds: pleasant the sun,\\
When first on this delightful land he spreads\\
His orient beams, on herb, tree, fruit, \& flower,\\
Glistering with dew; fragrant the fertile earth\\
After soft showers; and sweet the coming on\\
Of grateful evening mild; then silent night\\
With this her solemn bird \& this fair moon,\\
And these the gems of heaven, her starry train:\\
But neither breath of morn when she ascends\\
With charm of earliest birds; nor rising sun\\
On this delightful land, nor herb, fruit, flower,\\
Glistering with dew; nor fragrance after showers;\\
Nor grateful evening mild; nor silent night\\
With this her solemn bird; nor walk by moon,\\*
Or glittering star-light without thee is sweet.
\end{verse}

\subsection{}

\blfootnote{Dr Charles Darwin (1809 -- 1882), \cite{odq}.}I feel like an old warhorse at the sound of a trumpet.

\section{}

\subsection{}

\blfootnote{$\mathbb{R}$ Christopher Marlowe (1564 -- 1593), \cite{pbev}. These lines are taken from the First Sestiad of Marlowe's \refbook{Hero and Leander}.}\settowidth{\versewidth}{Her wide sleeves green, and bordered with a grove,}
\begin{verse}[\versewidth]
At \textsc{Sestos} \textit{Hero} dwelt; \textit{Hero} the fair,\\*
Whom young \textit{Apollo} courted for her hair,\\
And offered as a dower his burning throne,\\
Where she could sit for men to gaze upon.\\
The outside of her garments were of lawn,\\
The lining purple silk, with gilt stars drawn;\\
Her wide sleeves green, and bordered with a grove,\\
Where \textit{Venus} in her naked glory strove\\
To please the careless \& disdainful eyes\\
Of proud \textit{Adonis}, that before her lies;\\
Her kirtle blue, whereon was many a stain,\\
Made with the blood of wretched lovers slain.\\
Upon her head she ware a myrtle wreath,\\
From whence her veil reached to the ground beneath;\\
Her veil was artificial flowers \& leaves,\\
Whose workmanship both man \& beast deceives;\\
Many would praise the sweet smell as she passed,\\
When 'twas the odour which her breath forth cast;\\
And there for honey bees have sought in vain,\\
And beat from thence, have lighted there again.\\
About her neck hung chains of pebble-stone,\\
Which lightened by her neck, like diamonds shone.\\
She ware no gloves; for neither sun nor wind\\
Would burn or parch her hands, but, to her mind,\\
Or warm or cool them, for they took delight\\
To play upon those hands, they were so white.\\
Buskins of shells, all silvered, us{\`{e}}d she,\\
And branched with blushing coral to the knee;\\
Where sparrows perched, of hollow pearl \& gold,\\
Such as the world would wonder to behold:\\
Those with sweet water oft her handmaid fills,\\
Which as she went, would chirrup through the bills.\\
Some say, for her the fairest \textit{Cupid} pined,\\
And looking in her face, was strooken blind.\\
But this is true; so like was one the other,\\
As he imagined \textit{Hero} was his mother;\\
And oftentimes into her bosom flew,\\
About her naked neck his bare arms threw,\\
And laid his childish head upon her breast,\\*
And with still panting rocked there took his rest.
\end{verse}

\subsection{}

\blfootnote{`An Argument', Thomas Moore (1779 -- 1852), \cite{newlove}.}\settowidth{\versewidth}{    That wishing and the crime are one,}
\begin{verse}[\versewidth]
I've oft been told by learn\`{e}d friars\\*
\vin That wishing and the crime are one,\\
And heaven punishes desires\\*
\vin As much as if the deed were done.\\!

If wishing damns us, you \& I\\*
\vin Are damned to all our heart's content;\\
Come, then, at least we may enjoy\\*
\vin Some pleasure for our punishment.
\end{verse}

\subsection{}

\blfootnote{Sir Arthur Doyle (1859 -- 1930), \cite{odq}. This famous maxim ends with a question mark in \refbook{The Sign of Four}, the text in which it first appears.}When you have eliminated the impossible, whatever remains, however improbable, must be the truth.

\section{}

\subsection{}

\blfootnote{$\mathbb{R}$ Christopher Marlowe (1564 -- 1593), \cite{pbev}. These lines are taken from the First Sestiad of Marlowe's \refbook{Hero and Leander}.}\settowidth{\versewidth}{When two are stripped, long ere the course begin,}
\begin{verse}[\versewidth]
So fair a church as this had \textit{Venus} none:\\*
The walls were of discoloured jasper stone,\\
Wherein was \textit{Proteus} carved; and overhead\\
A lively vine of green sea-agate spread,\\
Where by one hand light-headed \textit{Bacchus} hung,\\
And with the other wine from grapes out-wrung.\\
Of crystal shining fair the pavement was;\\
The town of \textsc{Sestos} called it \textit{Venus}' glass:\\
There might you see the gods in sundry shapes,\\
Committing heady riots, incest, rapes:\\
For know, that underneath this radiant flower\\
Was \textit{Danae}'s statue in a brazen tower,\\
\textit{Jove} slyly stealing from his sister's bed,\\
To dally with idalian \textit{Ganimed},\\
And for his love \textit{Europa} bellowing loud,\\
And tumbling with the rainbow in a cloud;\\
Blood-quaffing \textit{Mars} heaving the iron net,\\
Which limping \textit{Vulcan} and his cyclops set;\\
Love kindling fire, to burn such towns as \textsc{Troy},\\
\textit{Sylvanus} weeping for the lovely boy\\
That now is turned into a cypress tree,\\
Under whose shade the wood-gods love to be.\\
And in the midst a silver altar stood:\\
There \textit{Hero}, sacrificing turtles' blood,\\
Veiled to the ground, veiling her eyelids close;\\
And modestly they opened as she rose.\\
Thence flew love's arrow with the golden head;\\
And thus \textit{Leander} was enamour{\`{e}}d.\\
Stone-still he stood, and evermore he gazed,\\
Till with the fire that from his countenance blazed\\
Relenting \textit{Hero}'s gentle heart was strook:\\*
Such force \& virtue hath an amorous look.\\!

It lies not in our power to love or hate,\\*
For will in us is overruled by fate.\\
When two are stripped, long ere the course begin,\\
We wish that one should lose, the other win;\\
And one especially do we affect\\
Of two gold ingots, like in each respect:\\
The reason no man knows; let it suffice,\\
What we behold is censured by our eyes.\\
Where both deliberate, the love is slight:\\*
Who ever loved, that loved not at first sight?
\end{verse}

\subsection{}

\blfootnote{$\mathbb{R}$ William Shakespeare (1564 -- 1616), \cite{pbev}.}\settowidth{\versewidth}{    And, tender churl, mak'st waste in niggarding.}
\begin{verse}[\versewidth]
From fairest creatures we desire increase,\\*
\vin That thereby beauty's rose might never die,\\
But as the riper should by time decease\\
\vin His tender heir might bear his memory:\\
But thou, contracted to thine own bright eyes,\\
\vin Feed'st thy light's flame with self-substantial fuel,\\
Making a famine where abundance lies,\\
\vin Thyself thy foe, to thy sweet self too cruel.\\
Thou that art now the world's fresh ornament,\\
\vin And only herald to the gaudy spring,\\
Within thine own bud buriest thy content,\\
\vin And, tender churl, mak'st waste in niggarding.\\
Pity the world, or else this glutton be,\\*
To eat the world's due, by the grave \& thee.
\end{verse}

\subsection{}

\blfootnote{John Dryden, Poet Laureate (1631 -- 1700), \cite{odq}. This is a line from the prologue to \refbook{All for Love}.}He who would search for pearls must dive below.

\section{}

\subsection{}

\blfootnote{`The Sea Similised to Meadows and Pastures: the Mariners, to Shepherds: the Mast, to a May-Pole: the Fish, to Beasts', Margaret Cavendish, Duchess of Newcastle-upon-Tyne (1623 -- 1673), \cite{norton}.}\settowidth{\versewidth}{The waves, like ridges of ploughed land, are high;}
\begin{verse}[\versewidth]
The waves, like ridges of ploughed land, are high;\\*
Whereat the ship oft stumbling, down doth lie.\\
But, in a calm, the sea's like meadows seen\\
Level; its saltness makes it look as green.\\
When ships thereon a slow soft pace do walk;\\
Then mariners, as shepherds, sing \& talk:\\
Some whistle, and some on their pipes do play;\\
And thus, with mirth, they pass their time away.\\
And every mast is like a may-pole high,\\
Round which they dance, though not so merrily\\
As shepherds do, when they their lasses bring\\
Garlands, to may-poles tied with a silk string.\\
Instead of garlands, they hang on their mast\\
Huge sails \& ropes, to tie these garlands fast.\\
Instead of lasses, they do dance with death;\\
And for their music, they have \textit{Boreas}' breath.\\
Instead of wine \& wassails, drink salt tears;\\
And for their meat, they feed on nought but fears.\\
For flocks of sheep, great schools of herrings swim;\\
The whales, as ravenous wolves, do feed on them.\\
As sportful kids skip over hillocks green,\\
So dancing dolphins, on the waves are seen.\\
The porpoise, like their watchful dogs espies,\\
And gives them warning when great winds will rise.\\
Instead of barking, he his head doth show\\
Above the waters, when they roughly flow:\\
And, like as men, in time of showering rain\\
And wind, do not in open fields remain;\\
But quickly run for shelter to a tree:\\*
So ships at anchor lie upon the sea.
\end{verse}

\subsection{}

\blfootnote{William Shakespeare (1564 -- 1616), \cite{shakespeare}. These lines are uttered by Caliban in \refbook{The Tempest} II.2.}\settowidth{\versewidth}{Young scamels from the rock. Wilt thou go with me?}
\begin{verse}[\versewidth]
I prithee, let me bring thee where crabs grow;\\*
And I with my long nails will dig thee pignuts,\\
Show thee a jay's nest, and instruct thee how\\
To snare the nimble marmoset. I'll bring thee\\
To clustering filberts, and sometimes I'll get thee\\*
Young scamels from the rock. Wilt thou go with me?
\end{verse}

\subsection{}

\blfootnote{John Dryden, Poet Laureate (1631 -- 1700), \cite{odq}. This is a line from Dryden's translation of Horace's \refbook{Odes} III.29.}Tomorrow do thy worst, for I have lived today.

\section{}

\subsection{}

\blfootnote{`The Haystack in the Flood', William Morris (1834 -- 1896), \cite{norton}.}\settowidth{\versewidth}{When the roads crossed; and sometimes, when}
\begin{verse}[\versewidth]
Had she come all the way for this,\\*
To part at last without a kiss?\\
Yea, had she borne the dirt \& rain\\
That her own eyes might see him slain\\
Beside the haystack in the floods?\\
Along the dripping leafless woods,\\
The stirrup touching either shoe,\\
She rode astride as troopers do;\\
With kirtle kilted to her knee,\\
To which the mud splashed wretchedly;\\
And the wet dripped from every tree\\
Upon her head \& heavy hair,\\
And on her eyelids broad \& fair;\\
The tears \& rain ran down her face.\\
By fits \& starts they rode apace,\\
And very often was his place\\
Far off from her; he had to ride\\
Ahead, to see what might betide\\
When the roads crossed; and sometimes, when\\
There rose a murmuring from his men,\\
Had to turn back with promises;\\
Ah me! she had but little ease;\\
And often for pure doubt \& dread\\
She sobbed, made giddy in the head\\
By the swift riding; while, for cold,\\
Her slender fingers scarce could hold\\
The wet reins; yea, and scarcely, too,\\
She felt the foot within her shoe\\
Against the stirrup : all for this,\\
To part at last without a kiss\\
Beside the haystack in the floods.\\
For when they neared that old soaked hay,\\
They saw across the only way\\
That \textit{Judas}, \textit{Godmar}, and the three.\\
Red running lions dismally\\
Grinned from his pennon, under which,\\
In one straight line along the ditch,\\*
They counted thirty heads.\\!

\textcolor{white}{They counted thirty heads.} So then,\\*
While \textit{Robert} turned round to his men,\\
She saw at once the wretched end,\\
And, stooping down, tried hard to rend\\
Her coif the wrong way from her head,\\
And hid her eyes; while \textit{Robert} said:\\
`Nay, love, 'tis scarcely two to one,\\
At \textsc{Poitiers} where we made them run\\
So fast -- why, sweet my love, good cheer,\\
The gascon frontier is so near,\\*
Nought after this.'\\!

\textcolor{white}{Nought after this.'} But, `O,' she said,\\*
`My God! my God! I have to tread\\
The long way back without you; then\\
The court at \textsc{Paris}; those six men;\\
The gratings of the \textsc{Ch\^{a}telet};\\
The swift \textsc{Seine} on some rainy day\\
Like this, and people standing by,\\
And laughing, while my weak hands try\\
To recollect how strong men swim.\\
All this, or else a life with him,\\
For which I should be damned at last,\\
Would God that this next hour were past!'\\
He answered not, but cried his cry,\\
`St \textit{George} for \textit{Marny}!' cheerily;\\
And laid his hand upon her rein.\\
Alas! no man of all his train\\
Gave back that cheery cry again;\\
And, while for rage his thumb beat fast\\
Upon his sword-hilts, some one cast\\
About his neck a kerchief long,\\*
And bound him.\\!

\textcolor{white}{And bound him.} Then they went along\\*
To \textit{Godmar}; who said: `Now, \textit{Jehane},\\
Your lover's life is on the wane\\
So fast, that, if this very hour\\
You yield not as my paramour,\\
He will not see the rain leave off --\\
Nay, keep your tongue from gibe \& scoff,\\
Sir \textit{Robert}, or I slay you now.'\\
She laid her hand upon her brow,\\
Then gazed upon the palm, as though\\
She thought her forehead bled, and -- `No.'\\
She said, and turned her head away,\\
As there were nothing else to say,\\
And everything were settled: red\\
Grew \textit{Godmar}'s face from chin to head:\\
`\textit{Jehane}, on yonder hill there stands\\
My castle, guarding well my lands:\\
What hinders me from taking you,\\
And doing that I list to do\\
To your fair wilful body, while\\*
Your knight lies dead?'\\!

\textcolor{white}{Your knight lies dead?'} A wicked smile\\*
Wrinkled her face, her lips grew thin,\\
A long way out she thrust her chin: go\\
`You know that I should strangle you\\
While you were sleeping; or bite through\\
Your throat, by God's help -- ah!' she said,\\
`Lord \textit{Jesus}, pity your poor maid!\\
For in such wise they hem me in,\\
I cannot choose but sin \& sin,\\
Whatever happens : yet I think\\
They could not make me eat or drink,\\
And so should I just reach my rest.'\\
`Nay, if you do not my behest,\\
O \textit{Jehane}! though I love you well,'\\
Said \textit{Godmar}, 'would I fail to tell\\
All that I know.' `Foul lies,' she said.\\
`Eh? lies my \textit{Jehane}? by God's head,\\
At \textsc{Paris} folks would deem them true!\\
Do you know, \textit{Jehane}, they cry for you,\\
``\textit{Jehane} the brown! \textit{Jehane} the brown!\\
Give us \textit{Jehane} to bum or drown!'' --\\
Eh -- gag me \textit{Robert}! -- sweet my friend,\\
This were indeed a piteous end no\\
For those long fingers, and long feet,\\
And long neck, and smooth shoulders sweet;\\
An end that few men would forget\\
That saw it. So, an hour yet:\\
Consider, \textit{Jehane}, which to take\\*
Of life or death!'\\!

\textcolor{white}{Of life or death!'} So, scarce awake,\\*
Dismounting, did she leave that place,\\
And totter some yards : with her face\\
Turned upward to the sky she lay,\\
Her head on a wet heap of hay,\\
And fell asleep: and while she slept,\\
And did not dream, the minutes crept\\
Round to the 12 again; but she,\\
Being waked at last, sighed quietly,\\
And strangely childlike came, and said:\\
'I will not.' Straightway \textit{Godmar}'s head,\\
As though it hung on strong wires, turned\\*
Most sharply round, and his face burned.\\!

For \textit{Robert} -- both his eyes were dry,\\*
He could not weep, but gloomily\\
He seemed to watch the rain; yea, too,\\
His lips were firm; he tried once more\\
To touch her lips; she reached out, sore\\
And vain desire so tortured them,\\
The poor grey lips, and now the hem\\*
Of his sleeve brush'd them.\\!

\textcolor{white}{Of his sleeve brush'd them.} With a start\\*
Up \textit{Godmar} rose, thrust them apart;\\
From \textit{Robert}'s throat he loosed the bands\\
Of silk \& mail; with empty hands\\
Held out, she stood \& gazed, and saw,\\
The long bright blade without a flaw\\
Glide out from \textit{Godmar}'s sheath, his hand\\
In \textit{Robert}'s hair; she saw him bend\\
Back \textit{Robert}'s head; she saw him send\\
The thin steel down; the blow told well,\\
Right backward the knight \textit{Robert} fell,\\
And moaned as dogs do, being \sfrac{$1$}{$2$} dead,\\
Unwitting, as I deem : so then\\
\textit{Godmar} turned grinning to his men,\\
Who ran, some five or six, and beat\\*
His head to pieces at their feet.\\!

Then \textit{Godmar} turned again and said:\\*
`So, \textit{Jehane}, the first fitte is read!\\
Take note, my lady, that your way\\
Lies backward to the \textsc{Ch\^{a}telet}!'\\
She shook her head and gazed awhile\\
At her cold hands with a rueful smile,\\*
As though this thing had made her mad.\\!

This was the parting that they had\\*
Beside the haystack in the floods.
\end{verse}

\subsection{}

\blfootnote{$\mathbb{R}$ `A Complaint by Night of the Lover not Beloved', Henry Howard, Earl of Surrey (1517 -- 1547), \cite{pbev}.}\settowidth{\versewidth}{The beasts, the air, the birds their song do cease;}
\begin{verse}[\versewidth]
Alas, so all things now do hold their peace.\\*
\vin Heaven \& earth disturb{\`{e}}d in no thing;\\
The beasts, the air, the birds their song do cease;\\
\vin The night's car the stars about doth bring;\\
Calm is the sea; the waves work less \& less:\\
\vin So am not I, whom love, alas, doth wring,\\
Bringing before my face the great increase\\
\vin Of my desires, whereat I weep and sing,\\
In joy \& woe, as in a doubtful case.\\
\vin For my sweet thoughts sometime do pleasure bring:\\
But by \& by, the cause of my disease\\
\vin Gives me a pang that inwardly doth sting,\\
When that I think what grief it is again\\*
To live and lack the thing should rid my pain.
\end{verse}

\subsection{}

\blfootnote{The Rt Hon Edmund Burke (1729 -- 1797), \cite{odq}.}A little alarm now and then keeps life from stagnation.

\section{}

\subsection{}

\blfootnote{`The Butterfly's Ball and the Grasshopper's Feast', William Roscoe (1753 -- 1831), \cite{obev}.}\settowidth{\versewidth}{But they all laughed so loud that he pulled in his head,}
\begin{verse}[\versewidth]
Come, take up your hats, and away let us haste\\*
To the butterfly's ball \& the grasshopper's feast:\\
The trumpeter gad-fly has summoned the crew,\\*
And the revels are now only waiting for you.\\!

On the smooth-shaven grass by the side of a wood\\*
Beneath a broad oak which for ages has stood,\\
See the children of earth \& the tenants of air\\*
For an evening's amusement together repair.\\!

And there came the beetle so blind \& so black,\\*
Who carried the emmet his friend on his back;\\
And there came the gnat \& the dragonfly too,\\*
And all their relations, green, orange \& blue.\\!

And there came the moth in his plumage of down,\\*
And the hornet in jacket of yellow \& brown,\\
Who with him the wasp his companion did bring;\\*
But they promised that evening to lay by their sting.\\!

And the sly little dormouse crept out of his hole,\\*
And led to the feast his blind brother the mole;\\
And the snail, with his horns peeping out from his shell,\\*
Came from a great distance -- the length of an ell.\\!

A mushroom their table, and on it was laid\\*
A water-dock leaf, which a tablecloth made;\\
The viands were various, to each of their taste,\\*
And the bee brought his honey to crown the repast.\\!

There close on his haunches, so solemn \& wise,\\*
The frog from a corner look'd up to the skies;\\
And the squirrel, well-pleased such diversion to see,\\*
Sat cracking his nuts overhead in a tree.\\!

Then out came a spider, with fingers so fine,\\*
To show his dexterity on the tight-line;\\
From one branch to another his cobweb he slung,\\*
Then as quick as an arrow he darted along.\\!

But just in the middle -- oh, shocking to tell! --\\*
From his rope in an instant poor \textit{Harlequin} fell;\\
Yet he touched not the ground, but with talons outspread,\\*
Hung suspended in air at the end of a thread.\\!

Then the grasshopper came, with a jerk \& a spring,\\*
Very long was his leg, though but short was his wing\\
He took but three leaps, and was soon out of sight,\\*
Then chirped his own praises the rest of the night\\!

With steps quite majestic the snail did advance,\\*
And promis'd the gazers a minuet to dance;\\
But they all laughed so loud that he pulled in his head,\\*
And went in his own little chamber to bed.\\!

Then as evening gave way to the shadows of night,\\*
Their watchman, the glow-worm, came out with his light;\\
Then home let us hasten while yet we can see,\\*
For no watchman is waiting for you \& for me.
\end{verse}

\subsection{}

\blfootnote{$\mathbb{R}$ `A Description of the Morning', The Very Rev Dr Jonathan Swift (1667 -- 1745), \cite{pbev}.}\settowidth{\versewidth}{The kennel's edge, where wheels had worn the place.}
\begin{verse}[\versewidth]
Now hardly here and there an hackney coach\\*
Appearing, showed the ruddy morn's approach.\\
Now \textit{Betty} from her master's bed had flown,\\
And softly stole to discompose her own;\\
The slipshod 'prentice from his master's door\\
Had pared the dirt, and sprinkled round the floor.\\
Now \textit{Moll} had whirled her mop with dextrous airs,\\
Prepared to scrub the entry \& the stairs.\\
The youth with broomy stumps began to trace\\
The kennel's edge, where wheels had worn the place.\\
The small-coal man was heard with cadence deep,\\
Till drowned in shriller notes of chimney sweep:\\
Duns at His Lordship's gate began to meet;\\
And brickdust \textit{Moll} had screamed through \sfrac{$1$}{$2$} the street.\\
The turn-key now his flock returning sees,\\
Duly let out a-nights to steal for fees:\\
The watchful bailiffs take their silent stands,\\*
And schoolboys lag with satchels in their hands.
\end{verse}

\subsection{}

\blfootnote{Robert Frost, Poet Laureate of Vermont (1874 -- 1963), \cite{pbev}. This is the first line of Frost's poem \refpoem{The Gift Outright}. As subsequent lines make clear, the primary sense which the poet is trying to convey is that Frost's ancestors first possessed the land, as subjects of the Crown, but were only later possessed by the land after the Revolutionary War.}The land was ours before we were the land's.

\section{}

\subsection{}

\blfootnote{Sir Charles Sedley, 5th Baronet (1639 -- 1701), \cite{newlove}.}\settowidth{\versewidth}{No drowning man can know which drop}
\begin{verse}[\versewidth]
\textit{Cloris}, I cannot say your eyes\\*
Did my unwary heart surprise;\\
Nor will I swear it was your face,\\
Your shape, or any nameless grace:\\
For you are so entirely fair,\\
To love a part, injustice were;\\
No drowning man can know which drop\\
Of water his last breath did stop;\\
So when the stars in heaven appear,\\
And join to make the night look clear;\\
The light we no one's bounty call,\\
But the obliging gift of all.\\
He that does lips or hands adore,\\
Deserves them only, \& no more;\\
But I love all \& every part,\\
And nothing less can ease my heart.\\
\textit{Cupid}, that lover, weakly strikes,\\*
Who can express what 'tis he likes.
\end{verse}

\subsection{}

\blfootnote{John Synge (1871 -- 1909), \cite{newlove}.}\settowidth{\versewidth}{Till I turned jealous of the Lord next door...}
\begin{verse}[\versewidth]
Beside a chapel I'd a room looked down,\\*
Where all the women from the farm \& town\\
On holy days \& sundays used to pass\\*
To marriages \& christenings \& to mass.\\!

Then I sat lonely, watching score \& score,\\*
Till I turned jealous of the Lord next door...\\
Now by this window, where there's none can see,\\*
The Lord God's jealous of yourself \& me.
\end{verse}

\subsection{}

\blfootnote{Gerald Gould (1885 -- 1936), \cite{oxfordlarkin}. This is a line from Gould's sonnet which begins `This is the horror that, night after night'.}For God's sake, if you sin, take pleasure in it.

\section{}

\subsection{}

\blfootnote{`Iambicum Trimetrum', Edmund Spenser (1552 -- 1599), \cite{obev}. `Iambicum Trimetrum' means iambic trimeter, but these lines follow a very different prosody. \P 21. The word `immerito' is Italian for `undeserved', although in this poem it seems to be used primarily as a man's name.}\settowidth{\versewidth}{    Thought, and fly forth unto my love, wheresoever she be:}
\begin{verse}[\versewidth]
Unhappy verse, the witness of my unhappy state,\\*
\vin Make thy self flutt'ring wings of thy fast flying\\
\vin Thought, and fly forth unto my love, wheresoever she be:\\
Whether lying restless in heavy bed, or else\\
\vin Sitting so cheerless at the cheerful board, or else\\
\vin Playing alone careless on her heavenly virginals.\\
If in bed, tell her, that my eyes can take no rest:\\
\vin If at board, tell her, that my mouth can eat no meat:\\
\vin If at her virginals, tell her, I can hear no mirth.\\
Asked why say: waking love suffereth no sleep:\\
\vin Say that raging love doth appal the weak stomach:\\
\vin Say that lamenting love marreth the musical.\\
Tell her, that her pleasures were wont to lull me asleep:\\
\vin Tell her, that her beauty was wont to feed mine eyes:\\
\vin Tell her, that her sweet tongue was wont to make me mirth.\\
Now do I nightly waste, wanting my kindly rest:\\
\vin Now do I daily starve, wanting my lively food:\\
\vin Now do I always die, wanting thy timely mirth.\\
And if I waste, who will bewail my heavy chance?\\
\vin And if I starve, who will record my curs\`{e}d end?\\*
\vin And if I die, who will say, `This was \textit{Immerito}'?
\end{verse}

\subsection{}

\blfootnote{$\mathbb{R}$ `The Dalliance of the Eagles', Walt Whitman (1819 -- 1892), \cite{pbev}.}\settowidth{\versewidth}{A motionless still balance in the air, then parting, talons loosing,}
\begin{verse}[\versewidth]
Skirting the river road (my forenoon walk, my rest),\\*
Skyward in air a sudden muffled sound, the dalliance of the eagles,\\
The rushing amorous contact high in space together,\\
The clinching interlocking claws, a living, fierce, gyrating wheel,\\
Four beating wings, two beaks, a swirling mass tight grappling,\\
In tumbling turning clustering loops, straight downward falling,\\
Till o'er the river poised, the twain yet one, a moment's lull,\\
A motionless still balance in the air, then parting, talons loosing,\\
Upward again on slow-firm pinions slanting, their separate diverse flight,\\*
She hers, he his, pursuing.
\end{verse}

\subsection{}

\blfootnote{Wentworth Dillon, 4th Earl of Roscommon (1637 -- 1885), \cite{odq}.}The multitude is always in the wrong.

\section{}

\subsection{}

\blfootnote{Sir John Suckling (1609 -- 1642), \cite{norton}.}\settowidth{\versewidth}{There's no such thing as what we beauty call;}
\begin{verse}[\versewidth]
Of thee, kind boy, I ask no red \& white,\\*
\vin To make up my delight;\\
\vin No odd becoming graces,\\
Black eyes, or little know-not-whats in faces;\\
Make me but made enough, give me good store\\
Of love for her I count:\\
\vin I ask no more;\\*
'Tis love in love that makes the sport.\\!

There's no such thing as what we beauty call;\\*
\vin It is mere cozenage all:\\
\vin For though some, long ago,\\
Liked certain colours mingled so \& so,\\
That doth not tie me now from choosing new;\\
If I a fancy take\\
\vin To black \& blue,\\*
That fancy doth it beauty make.\\!

'Tis not the meat, but 'tis the appetite\\*
\vin Makes eating a delight;\\
\vin And if I like one dish\\
More than another, that a pheasant is;\\
What in our watches, that in us is found:\\
So to the height \& nick\\
\vin We up be wound,\\*
No matter by what hand or trick.
\end{verse}

\subsection{}

\blfootnote{$\mathbb{R}$ `Comparison of Love to a Streame Falling from the Alpes', Sir Thomas Wyatt (1503 -- 1542), \cite{pbev}.}\settowidth{\versewidth}{From these high hills as when a spring doth fall}
\begin{verse}[\versewidth]
From these high hills as when a spring doth fall\\*
\vin It trilleth down with still \& subtle course,\\
Of this \& that it gathers aye and shall\\
\vin Till it have just off flowed the stream \& force,\\
Then at the foot it rageth over all.\\
\vin So fareth love when he hath ta'en a source:\\
His rein is rage; resistance vaileth none;\\*
The first eschew is remedy alone.
\end{verse}

\subsection{}

\blfootnote{Samuel Butler (1835 -- 1902), \cite{odq}.}An apology for the devil: it must be remembered that we have only heard one side of the case. God has written all the books.

\section{}

\subsection{}

\blfootnote{`The Manly Heart', George Wither (1588 -- 1667), \cite{treasury}.}\settowidth{\versewidth}{'Cause her fortune seems too high}
\begin{verse}[\versewidth]
Shall I, wasting in despair,\\*
Die because a woman's fair?\\
Or make pale my cheeks with care\\
'Cause another's rosy are?\\
Be she fairer than the day,\\
Or the flowery meads in may --\\
\vin If she be not so to me,\\*
\vin What care I how fair she be?\\!

Shall my foolish heart be pined\\*
'Cause I see a woman kind?\\
Or a well-disposed nature\\
Join\`{e}d with a lovely feature?\\
Be she meeker, kinder, than\\
Turtle dove or pelican,\\
\vin If she be not so to me,\\*
\vin What care I how kind she be?\\!

Shall a woman's virtues move\\*
Me to perish for her love?\\
Or her merits' value known\\
Make me quite forget mine own?\\
Be she with that goodness blest\\
Which may gain her name of Best;\\
\vin If she seem not such to me,\\*
\vin What care I how good she be?\\!

'Cause her fortune seems too high\\*
Shall I play the fool and die?\\
Those that bear a noble mind\\
Where they want of riches find,\\
Think what with them they would do\\
That without them dare to woo;\\
\vin And unless that mind I see,\\*
\vin What care I how great she be?\\!

Great or good, or kind or fair,\\*
I will ne'er the more despair:\\
If she love me, this believe,\\
I will die ere she shall grieve;\\
If she slight me when I woo,\\
I can scorn and let her go;\\
\vin For if she be not for me,\\*
\vin What care I for whom she be?
\end{verse}

\subsection{}

\blfootnote{`No Second Troy', William Yeats (1865 -- 1939), \cite{oxfordlarkin}.}\settowidth{\versewidth}{    Or hurled the little streets upon the great,}
\begin{verse}[\versewidth]
Why should I blame her that she filled my days\\*
\vin With misery, or that she would of late\\
Have taught to ignorant men most violent ways,\\
\vin Or hurled the little streets upon the great,\\
Had they but courage equal to desire?\\
\vin What could have made her peaceful with a mind\\
That nobleness made simple as a fire,\\
\vin With beauty like a tightened bow, a kind\\
That is not natural in an age like this,\\
\vin Being high \& solitary \& most stern?\\
Why, what could she have done, being what she is?\\*
\vin Was there another \textsc{Troy} for her to burn?
\end{verse}

\subsection{}

\blfootnote{Samuel Butler (1835 -- 1902), \cite{odq}.}To live is like to love: all reason is against it, and all healthy instinct for it.

\chapter{Quartilis}

\section{}

\subsection{}

\blfootnote{`Pratty Flowers', Anonymous, \cite{rusby_mortals}.}\begin{center}
\textit{Tune: The Holmfirth Anthem}
\end{center}

\settowidth{\versewidth}{No more to yon green banks will I take thee,}
\begin{verse}[\versewidth]
Abroad for pleasure as I was a-walking --\\*
It was one summer summer's evening clear --\\
There I beheld a most beautiful damsel,\\*
Lamenting for her shepherd swain.\\!

The fairest evening that e'er I beheld\\*
Was ever evermore with the lad I adore.\\
Wilt thou fight yon french \& spaniards?\\*
Wilt thou leave me thus, my dear?\\!

No more to yon green banks will I take thee,\\*
With pleasure for to rest yourself and view the land.\\
But I will take you to yon green gardens,\\*
Where the pretty pretty flowers grow.
\end{verse}

\subsection{}

\blfootnote{`All Last Night', Prof Lascelles Abercrombie (1881 -- 1938), \cite{faber20th}. \P 10. Woodruffe seems to be a farm near Epsom in the County of Surrey.}\settowidth{\versewidth}{    Woodruffe 'twas, when spring}
\begin{verse}[\versewidth]
All last night I had quiet\\*
\vin In a fragrant dream \& warm:\\
She had become my sabbath,\\*
\vin And round my neck, her arm.\\!

I knew the warmth in my dreaming;\\*
\vin The fragrance, I suppose,\\
Was her hair about me,\\*
\vin Or else she wore a rose.\\!

Her hair, I think; for likest\\*
\vin \textsc{Woodruffe} 'twas, when spring\\
Loitering down wet woodways\\*
\vin Treads it sauntering.\\!

No light, nor any speaking;\\*
\vin Fragrant only \& warm.\\
Enough to know my lodging,\\*
\vin The white sabbath of her arm.
\end{verse}

\subsection{}

\blfootnote{Deuteronomy 32.27, \cite{kjv}.}The eternal God is thy refuge, and underneath are the everlasting arms.

\section{}

\subsection{}

\blfootnote{Anonymous, \cite{spanning}.}\begin{center}
\textit{Tune: Dark Eyes Sailor}
\end{center}

\settowidth{\versewidth}{    For my dark-eyed sailor has proved his honour long.'}
\begin{verse}[\versewidth]
As I roved out one evening fair,\\*
\vin It being the summertime to take the air,\\
I spied a sailor \& a lady gay,\\*
\vin And I stood to listen to hear what they would say.\\!

He said, `Fair lady, why do you roam,\\*
\vin For the day is spent and the night is on?'\\
She heaved a sigh while the tears did roll:\\*
\vin `For my dark-eyed sailor, so young \& stout \& bold.\\!

`'Tis seven long years since he left this land.\\*
\vin A ring he took from off his lily-white hand.\\
One \sfrac{$1$}{$2$} of the ring is still here with me,\\*
\vin But the other's rolling at the bottom of the sea.'\\!

He said, `You may drive him out of your mind.\\*
\vin Some other young man you will surely find.\\
Love turns aside and soon cold has grown.\\*
\vin Like the winter's morning, the hills are white with snow.'\\!

She said, `I'll never forsake my dear,\\*
\vin Although we're parted this many a year.\\
Genteel he was and a rake like you\\*
\vin To induce a maiden to slight the jacket blue.'\\!

One \sfrac{$1$}{$2$} of the ring did young \textit{William} show;\\*
\vin She ran distracted in grief \& woe,\\
Saying, `\textit{William}, \textit{William}, I have gold in store\\*
\vin For my dark-eyed sailor has proved his honour long.'\\!

And there is a cottage by yonder lea.\\*
\vin This couple's married and does agree.\\
So maids be loyal when your love's at sea,\\*
\vin For a cloudy morning brings in a sunny day.
\end{verse}

\subsection{}

\blfootnote{$\mathbb{R}$ `The Lilly', William Blake (1757 -- 1827), \cite{blakea}.}\settowidth{\versewidth}{While the lily white shall in love delight,}
\begin{verse}[\versewidth]
The modest rose puts forth a thorn:\\*
The humble sheep a threatening horn:\\
While the lily white shall in love delight,\\*
Nor a thorn nor a threat stain her beauty bright.
\end{verse}

\subsection{}

\blfootnote{George Borrow (1803 -- 1881), \cite{odq}.}Life is very sweet, brother. Who would wish to die?

\section{}

\subsection{}

\blfootnote{Anonymous, \cite{spanning}.}\begin{center}
\textit{Tune: New York Girls}
\end{center}

\settowidth{\versewidth}{As I walked down through Chatham Street,}
\begin{verse}[\versewidth]
As I walked down through \textsc{Chatham Street},\\*
\vin A fair maid I did meet.\\
She asked me to see her home;\\*
\vin She lived in \textsc{Bleeker Street}.\\!

{\itshape
And away, you santy,\\
\vin My dear honey!\\
O you \textsc{New York} girls,\\*
\vin Can't you dance the polka?}\\!

And when we got to \textsc{Bleeker Street}\\*
\vin We stopped at 44:\\
Her mother \& her sister there\\*
\vin To meet her at the door.\\!

And when I got inside the house,\\*
\vin The drinks were passed around.\\
The liquor was so awful strong,\\*
\vin My head went round and round.\\!

And then we had another drink\\*
\vin Before we sat to eat.\\
The liquor was so awful strong,\\*
\vin I quickly fell asleep.\\!

When I awoke next morning,\\*
\vin I had an aching head.\\
There was I, \textit{Jack}, all alone,\\*
\vin Stark naked in my bed.\\!

My gold watch \& my pocketbook\\*
\vin And lady friend were gone.\\
And there was I \textit{Jack}, all alone,\\*
\vin Stark naked in my room.\\!

On looking round this little room,\\*
\vin There's nothing I could see\\
But a woman's shift \& apron\\*
\vin That were no use to me.\\!

With a flour barrel for a suit of clothes\\*
\vin Down \textsc{Cherry Street} forlorn,\\
There \textit{Martin Churchill} took me in\\*
\vin And sent me round \textsc{Cape Horn}.\\!

{\itshape
And away, you santy,\\
\vin My dear honey!\\
O you \textsc{New York} girls,\\*
\vin Can't you dance the polka?}
\end{verse}

\subsection{}

\blfootnote{`Laughing Song', William Blake (1757 -- 1827), \cite{blakea}.}\settowidth{\versewidth}{When the green woods laugh, with the voice of joy}
\begin{verse}[\versewidth]
When the green woods laugh, with the voice of joy\\*
And the dimpling stream runs laughing by,\\
When the air does laugh with our merry wit,\\*
And the green hill laughs with the noise of it...\\!

When the meadows laugh with lively green,\\*
And the grasshopper laughs in the merry scene;\\
When \textit{Mary} and \textit{Susan} and \textit{Emily}\\*
With their sweet round mouths sing, 'Ha ha he!'\\!

When the painted birds laugh in the shade\\*
Where our table with cherries \& nuts is spread:\\
Come live \& be merry and join with me\\*
To sing the sweet chorus of 'Ha ha he!'
\end{verse}

\subsection{}

\blfootnote{Nicholas Breton (1545 -- 1626), \cite{odq}.}We rise with the lark and go to bed with the lamb.

\section{}

\subsection{}

\blfootnote{William Graham (1830 -- 1891), \cite{oldengland1}.}\begin{center}
\textit{Tune: The Lish Young Buy-a-Broom}
\end{center}

\settowidth{\versewidth}{And she was right; I was tight; everybody has their way.}
\begin{verse}[\versewidth]
As I was a-travelling in the north country,\\*
Down by \textsc{Kirkby Stephen} I happened for to be,\\
As I was a-walking up \& down the street,\\*
A pretty little buy-a-broom I chanc\`{e}d for to meet.\\!

{\itshape
And she was right; I was tight; everybody has their way.\\*
It was the lish young buy-a-broom that led me astray.}\\!

She kindly then invited me to go along the way.\\*
Yes was the answer to her that I did say.\\
There was me with my music walking down the street,\\*
And her with her tambourine was beating hand \& feet.\\!

Straightway for \textsc{Kendal} we steer\`{e}d, her \& I;\\*
Over yon white mountain, the weather it was dry.\\
We each had a bottle filled up to the top,\\*
And whenever we were feeling dry, we took a little drop.\\!

The night's coming on, good lodgings we did find,\\*
Eatables of all sorts \& plenty of good wine,\\
Good bed \& blankets just for we two.\\*
And I rolled her in my arms, my boys, and wouldn't you do too?\\!

Well early the next morning we arose to go away.\\*
I called to the landlord to see what was to pay:\\
Fourteen and sixpence, just for you two.\\*
And a fiver on the table O my darling then she threw.\\!

Now the reason that we parted, I now will let you hear.\\*
She started off for Germany right early the next year;\\
And me not being willing to cross the raging sea --\\*
Here's a health to my buy-a-broom, wherever she may be.\\!

{\itshape
And she was right; I was tight; everybody has their way.\\*
It was the lish young buy-a-broom that led me astray.}
\end{verse}

\subsection{}

\blfootnote{Charles Bowen, Baron Bowen (1835 -- 1894), \cite{odq}.}\settowidth{\versewidth}{    And also on the unjust fella,}
\begin{verse}[\versewidth]
The rain it raineth on the just\\*
\vin And also on the unjust fella,\\
But mainly on the just because\\*
\vin The unjust steals the just's umbrella.
\end{verse}

\subsection{}

\blfootnote{Louis Brownstein (1893 -- 1958), \cite{odq}.}Life is just a bowl of cherries.

\section{}

\subsection{}

\blfootnote{Anonymous, \cite{rusby_couldnt}.}\begin{center}
\textit{Tune: The Game of All Fours}
\end{center}

\settowidth{\versewidth}{    To hear the birds whistle and the nightingales play.}
\begin{verse}[\versewidth]
As I was a-walking one midsummer's morning\\*
\vin To hear the birds whistle and the nightingales play.\\
'Twas there that I met a beautiful maiden\\*
\vin As I was a-walking all on the highway.\\!

`Where are you going, my fair pretty lady?\\*
\vin Where are you going so early this morn?'\\
She answered, `Kind sir, to visit my neighbours;\\*
\vin I'm going down to \textsc{Lincoln}, the place I was born.'\\!

`May I go with you, my fair pretty lady?\\*
\vin May I go along in your sweet company?'\\
She turned her head round and, smiling all at me,\\*
\vin Said, `You may come with me, kind sir, if you please.'\\!

We hadn't been walking a few miles together\\*
\vin Before this young damsel began to show free.\\
She sat herself down, saying, `Sit down beside me.\\*
\vin The games we shall play will be one, two \& three.'\\!

I said, `My dear lady, if you're fond of the gaming,\\*
\vin There's one game I know I would like you to learn.\\
The game it is called ``The Game of All Fours''.'\\*
\vin So I took out my pack and began the first turn.\\!

She cut the cards and I fell a-dealing.\\*
\vin I dealt her a trump and myself the poor jack.\\
She led off her ace and stole my jack from me,\\*
\vin Saying, `Jack is the card I like best in your pack.'\\!

`I dealt them last time: it's your turn to shuffle,\\*
\vin My turn to show the best card in the pack.'\\
Once more she'd the ace \& deuce for to beat me;\\*
\vin Once again I had lost when I laid down poor jack.\\!

So I took up my hat and I bid her good morning.\\*
\vin I said, `You're the best that I know at this game.'\\
She answered, `Young man, if you'll come back tomorrow,\\*
\vin We'll play the game over \& over again.'
\end{verse}

\subsection{}

\blfootnote{Miss Emily Brontë (1818 -- 1848), \cite{obev}.}\settowidth{\versewidth}{Forgive me if I've shunned so long}
\begin{verse}[\versewidth]
I know not how it falls on me,\\*
This summer evening, hushed \& lone;\\
Yet the faint wind comes soothingly\\*
With something of an olden tone.\\!

Forgive me if I've shunned so long\\*
Your gentle greeting, earth \& air!\\
But sorrow withers e'en the strong,\\*
And who can fight against despair?
\end{verse}

\subsection{}

\blfootnote{$\mathbb{R}$ George Noel, 6th Baron Byron (1788 -- 1824), \cite{londonbook}. This couplet is from the opening `Dedication' to Lord Byron's \refbook{Don Juan}.}\settowidth{\versewidth}{And he who understands it would be able}
\begin{verse}[\versewidth]
And he who understands it would be able\\*
To add a storey to the Tower of Babel.
\end{verse}

\section{}

\subsection{}

\blfootnote{Anonymous, \cite{rusby_littlelights}.}\begin{center}
\textit{Tune: Merry Green Broom}
\end{center}

\settowidth{\versewidth}{He saw the five rings laid there on his chest;}
\begin{verse}[\versewidth]
A wager with you, my pretty fair maid,\\*
\vin Five hundred pounds to your 10:\\
A maid you will go to the merry green broom,\\
\vin And a maid you'll no longer return O.\\
`A wager, a wager with you, kind sir,\\
\vin Five hundred pounds to my 10:\\
A maid I will go to the merry green broom,\\*
\vin And a maid I will boldly return.\\!

The maiden she sat in her bower alone:\\*
\vin She is in torment \& strife:\\
If I don't go to the broom this night,\\
\vin My love he won't make me his wife O.\\
So up and she's gone on her good white steed,\\
\vin Away for her young man to meet.\\
She found him there and all fast asleep,\\*
\vin With a blood red rose at his feet.\\!

She's kissed him twice on cheek \& on chin,\\*
\vin Then over his body did lean.\\
There she did place five rings on his chest\\
\vin Just so he would know she had been O.\\
Then off through the woods the young maid did go,\\
\vin Just when her love did arise.\\
He saw the five rings laid there on his chest;\\*
\vin On his face was nought but surprise.
\end{verse}

\subsection{}

\blfootnote{John Clare (1793 -- 1864), \cite{obev}.}\settowidth{\versewidth}{And when I looked I fancied something stirred,}
\begin{verse}[\versewidth]
I found a ball of grass among the hay\\*
And progged it as I passed and went away;\\
And when I looked I fancied something stirred,\\
And turned again and hoped to catch the bird --\\
When out an old mouse bolted in the wheats\\
With all her young ones hanging at her teats;\\
She looked so odd and so grotesque to me,\\
I ran and wondered what the thing could be,\\
And pushed the knapweed bunches where I stood;\\
Then the mouse hurried from the craking brood.\\
The young ones squeaked, and as I went away\\
She found her nest again among the hay.\\
The water o'er the pebbles scarce could run\\*
And broad old cesspools glittered in the sun.
\end{verse}

\subsection{}

\blfootnote{George Noel, 6th Baron Byron (1788 -- 1824), \cite{odq}. This is a line from the second canto of \refbook{Don Juan}.}The best of life is but intoxication.

\section{}

\subsection{}

\blfootnote{Anonymous, \cite{aamhat}.}\begin{center}
\textit{Tune: Hard Times of Old England}
\end{center}

\settowidth{\versewidth}{Come home to be starved: better stayed where they were.}
\begin{verse}[\versewidth]
Come all brother tradesmen that travel along:\\*
O pray come and tell me where the trade is all gone.\\*
Long time have I travelled and I cannot find none.\\!

{\itshape
And sing O the hard times of old England!\\*
In old England very hard times!}\\!

Provisions you buy at the shop it is true\\*
But if you've no money there's none there for you.\\*
So what's a poor man and his family to do?\\!

You must go to the shop and you'll ask for a job.\\*
They'll answer you there with a shake and a nod.\\*
Well that's enough to make a man turn out and rob.\\!

You will see the poor tradesmen a-walking the street\\*
From morning till night for employment to seek,\\*
And scarce have they got any shoes on their feet\\!

Our soldiers \& sailors have just come from war,\\*
Been fighting for Queen \& country this year,\\*
Come home to be starved: better stayed where they were.\\!

And now to conclude and to finish my song:\\*
Let us hope that these hard times they will not last long.\\*
I hope soon to have occasion to alter my song.\\!

{\itshape
And sing O the good times of old England!\\*
In old England jolly good times!}
\end{verse}

\subsection{}

\blfootnote{`Love's Farewell', Michael Drayton (1563 -- 1631), \cite{treasury}.}\settowidth{\versewidth}{When his pulse failing, passion speechless lies,}
\begin{verse}[\versewidth]
Since there's no help, come let us kiss \& part --\\*
Nay I have done, you get no more of me;\\
And I am glad, yea, glad with all my heart,\\
That thus so cleanly I myself can free.\\
Shake hands for ever, cancel all our vows;\\
And when we meet at any time again,\\
Be it not seen in either of our brows\\
That we one jot of former love retain.\\
Now at the last gasp of love's latest breath,\\
When his pulse failing, passion speechless lies,\\
When faith is kneeling by his bed of death,\\
And innocence is closing up his eyes --\\
Now if thou wouldst, when all have given him over,\\*
From death to life thou mightst him yet recover!
\end{verse}

\subsection{}

\blfootnote{Raymond Chandler (1888 -- 1959), \cite{odq}.}If my books had been any worse, I should not have been invited to Hollywood, and if they had been any better, I should not have come.

\section{}

\subsection{}

\blfootnote{Anonymous, \cite{herringbone}. George Goodenough wrote in \refbook{The Handy Man Afloat and Ashore} (1901): `The following song always struck me as having one of the finest airs ever sung on a foc's'le. To hear the chorus pealing forth from some hundred or more throats was a thing to be remembered. The only pity is that the words are not more sensible. Such as they are they were very difficult to obtain. A bluejacket once wrote down all he could remember of them for me, but the copy got mixed up with other papers and I thought I had lost it. No one else could I find that could repair the supposed loss. Inquiries at second-hand music shops in London were fruitless. Many men could tell me that they knew the song but could not give me the words. Quite recently I came across my copy and here is the song.'}\begin{center}
\textit{Tune: Cadgwith Anthem}
\end{center}

\settowidth{\versewidth}{We shall laugh at your agony, and scorn at your threats.}
\begin{verse}[\versewidth]
Come fill up your glasses, and let us be merry,\\*
For to rob and to plunder it is our intent.\\!

{\itshape
As we roam through the valleys\\
Where the lilies and the roses,\\
And the beautiful cashmere lies drooping its head:\\
Then away, then away, then away, away,\\
To the caves in yonder mountains,\\*
To the robbers' retreat!}\\!

We come from yonder mountains. Our pistols are loaded,\\*
For to rob and to plunder it is our intent.\\!

Hark, hark! In the distance there's footsteps approaching.\\*
Stand, stand and deliver, shall be our watchword.\\!

Your gold and your jewels -- your life if resisted!\\*
We shall laugh at your agony, and scorn at your threats.\\!

Come fill up your glasses and let be a-drinking,\\*
For the moonbeams are shining all over our heads.\\!

{\itshape
As we roam through the valleys\\
Where the lilies and the roses,\\
And the beautiful cashmere lies drooping its head:\\
Then away, then away, then away, away,\\
To the caves in yonder mountains,\\*
To the robbers' retreat!}
\end{verse}

\subsection{}

\blfootnote{`A Silent Love', Sir Edward Dyer (1543 -- 1607), \cite{newlove}.}\settowidth{\versewidth}{    True hearts have ears eyes, no tongues to speak;}
\begin{verse}[\versewidth]
The lowest trees have tops, the ant her gall,\\*
The fly his spleen, the little spark his heat;\\
The slender hairs cast shadows, though but small,\\
And bees have stings, although they be not great;\\
\vin Seas have their source, and so have shallow springs;\\*
\vin And love is love, in beggars \& in kings.\\!

Where waters smoothest run, there deepest are the fords;\\*
The dial stirs, yet none perceives it move;\\
The firmest faith is found in fewest words;\\
The turtles do not sing, and yet they love;\\
\vin True hearts have ears \& eyes, no tongues to speak;\\*
\vin They hear \& see, and sigh, and then they break.
\end{verse}

\subsection{}

\blfootnote{$\mathbb{R}$ Gilbert Chesterton, Knight (1874 -- 1936), \cite{odq}.}Happiness is a mystery like religion, and should never be rationalised.

\section{}

\subsection{}

\blfootnote{Anonymous, \cite{rusby_underneath}.}\begin{center}
\textit{Tune: The Blind Harper}
\end{center}

\settowidth{\versewidth}{    And three times over for the good grey mare.}
\begin{verse}[\versewidth]
Have you heard of the blind harper,\\*
\vin How he lived in \textsc{Lochmaven} town,\\
How he went down to fair England,\\*
\vin To steal King \textit{Henry}'s wanton brown?\\!

First he went unto his wife,\\*
\vin With all the haste that go could he.\\
This work, he said, it will never go well,\\*
\vin Without the help of our good grey mare.\\!

Says she, You take the good grey mare,\\*
\vin She'll run o'er hills both low \& high.\\
Go take the halter in your hose,\\*
\vin And leave the foal at home with me.\\!

So he's up and went to England gone.\\*
\vin He went as fast as go could he.\\
And when he got to \textsc{Carlisle} gates\\*
\vin Who should be there but King \textit{Henry}?\\!

Come in, come in, you blind harper,\\*
\vin And of your music let me hear.\\
But up and says the blind harper,\\*
\vin I'd rather have a stable for my mare.\\!

The king he looks over his left shoulder\\*
\vin And he says unto his stable groom,\\
Go take the poor blind harper's mare,\\*
\vin And put her beside my wanton brown.\\!

Then he's harped and then he sang,\\*
\vin Till he played them all so sound asleep,\\
And quietly he took off his shoes,\\*
\vin And down the stairs he did creep.\\!

Straight to the stable door he goes,\\*
\vin With a tread so light as light could be,\\
And when he opened and went in\\*
\vin There he found 30 steeds \& three.\\!

And he took the halter from his horse\\*
\vin And from his purse he did not fail.\\
He slipped it over the wanton's nose\\*
\vin And he's tied it to the grey mare's tail.\\!

Then he let her loose at the castle gates\\*
\vin And the mare didn't fail to find her way.\\
She's went back to her own colt foal,\\*
\vin Three long hours before the day.\\!

So then in the morning, at fair daylight\\*
\vin When they had ended all their cheer,\\
Behold the wanton brown has gone,\\*
\vin And so has the poor blind harper's mare.\\!

And O \& alas, says the blind harper,\\*
\vin However alas that I came here!\\
In Scotland I've got me a little colt foal;\\*
\vin In England they stole my good grey mare.\\!

Hold your tongue, says King \textit{Henry},\\*
\vin And all your mournings let them be,\\
For you shall get a far better mare\\*
\vin And well paid shall your colt foal be.\\!

Again he harped and again he sang;\\*
\vin The sweetest music he let them hear.\\
And he was paid for a foal that he never had lost\\*
\vin And three times over for the good grey mare.
\end{verse}

\subsection{}

\blfootnote{`His Own Epitaph', John Gay (1685 -- 1732), \cite{obev}. This couplet is on Gay's tombstone in Westminster Abbey, below Pope's epitaph on the poet.}\settowidth{\versewidth}{Life is a jest; and all things show it,}
\begin{verse}[\versewidth]
Life is a jest; and all things show it,\\*
I thought so once; but now I know it.
\end{verse}

\subsection{}

\blfootnote{Gilbert Chesterton, Knight (1874 -- 1936), \cite{odq}. This is a line from Chesterton's \refpoem{Wine and Water}.}I don't care where the water goes if it doesn't get into the wine.

\section{}

\subsection{}

\blfootnote{Anonymous, \cite{spinners}.}\begin{center}
\textit{Tune: The Waters of Tyne}
\end{center}

\settowidth{\versewidth}{    Or scull her across the rough river to me.}
\begin{verse}[\versewidth]
I cannot get to my love if I would dee,\\*
\vin The waters of \textsc{Tyne} run between her \& me.\\
And here I must stand with a tear in my e'e,\\*
\vin Both sighing \& sickly my sweetheart to see.\\!

O where is the boatman, my bonnie hinny?\\*
\vin Where is the boatman? Bring him to me,\\
To ferry me over the \textsc{Tyne} to my hinny,\\*
\vin And I will remember the boatman \& thee.\\!

O bring me the boatman. I'll gi'e all my money,\\*
\vin And you for your trouble rewarded shall be.\\
To ferry me o'er the \textsc{Tyne} to my honey\\*
\vin Or scull her across the rough river to me.
\end{verse}

\subsection{}

\blfootnote{`On a Midsummer's Eve', Thomas Hardy (1840 -- 1928), \cite{faber20th}.}\settowidth{\versewidth}{    I thought not what my words might be;}
\begin{verse}[\versewidth]
I idly cut a parsley stalk,\\*
\vin And blew therein towards the moon;\\
I had not thought what ghosts would walk\\*
\vin With shivering footsteps to my tune.\\!

I went, and knelt, and scooped my hand\\*
\vin As if to drink, into the brook,\\
And a faint figure seemed to stand\\*
\vin Above me, with the bygone look.\\!

I lipped rough rhymes of chance, not choice,\\*
\vin I thought not what my words might be;\\
There came into my ear a voice\\*
\vin That turned a tenderer verse for me.
\end{verse}

\subsection{}

\blfootnote{William Congreve (1670 -- 1729), \cite{odq}.}If this be not love, it is madness, and then it is pardonable.

\section{}

\subsection{}

\blfootnote{Rudyard Kipling (1865 -- 1936), \cite{kipling}.}\begin{center}
\textit{Tune: A Pilgrim's Way}
\end{center}

\settowidth{\versewidth}{I will not cherish hate too long (my hands are none too clean),}
\begin{verse}[\versewidth]
I do not look for holy saints to guide me on my way\\*
Or male \& female devilkins to lead my feet astray.\\
If these are added I rejoice -- if not, I shall not mind\\
So long as I have leave \& choice to meet my fellow-kind.\\
For as we come and as we go (and deadly soon go we!)\\*
The people, Lord, thy people, are good enough for me.\\!

Thus I will honour pious men whose virtue shines so bright\\*
(Though none are more amazed than I when I by chance do right)\\
And I will pity foolish men for woe their sins have bred\\
(Though 99\% of mine I brought on my own head)\\
And amorite or eremite or general averagee,\\*
The people, Lord, thy people, are good enough for me.\\!

And when they bore me overmuch, I will not shake mine ears,\\*
Recalling many 1000 such whom I have bored to tears,\\
And when they labour to impress I will not doubt nor scoff,\\
Since I myself have done no less and sometimes pulled it off.\\
Yea as we are and we are not and we pretend to be,\\*
The people, Lord, thy people, are good enough for me.\\!

And when they work me random wrong, as oftentimes hath been,\\*
I will not cherish hate too long (my hands are none too clean),\\
And when they do me random good, I will not feign surprise,\\
No more than those whom I have cheered with wayside courtesies,\\
But as we give and as we take (whate'er our takings be)\\*
The people, Lord, thy people, are good enough for me.\\!

But when I meet with frantic folk who sinfully declare\\*
There is no pardon for their sin, the same I will not spare\\
Till I have proved that heaven and hell, which in our hearts we have,\\
Show nothing irredeemable on either side the grave,\\
For as we live and as we die -- if utter death there be --\\*
The people, Lord, thy people, are good enough for me.\\!

Deliver me from every pride -- the middle, high \& low --\\*
That bars me from a brother's side, whatever pride he show,\\
And purge me from all heresies of thought \& speech \& pen\\
That bid me judge him otherwise than I am judged. Amen.\\
That I might sing of crowd or king or road-borne company,\\
That I may labour in my day, vocation and degree,\\
To prove the same by deed \& name, and hold unshakenly\\
(Where'er I go, whate'er I know, whoe'er my neighbour be)\\
This single faith in life \& death and to eternity:\\*
The people, Lord, thy people, are good enough for me.
\end{verse}

\subsection{}

\blfootnote{`Vanity', Mrs Edith Hepburn (1883 -- 1947), \cite{faber20th}.}\settowidth{\versewidth}{For I remembered I was young as you, dear lad.}
\begin{verse}[\versewidth]
I saw old duchesses with their young loves,\\*
I, in a pair of very shabby gloves;\\
Even my shapeless garments could not make me sad,\\
For I remembered I was young as you, dear lad.\\
That I am lovelier without my dress\\*
Gave me sweet wanton happiness.
\end{verse}

\subsection{}

\blfootnote{Thomas de Quincey (1785 -- 1859), \cite{odq}.}If once a man indulges himself in murder, very soon he comes to think little of robbing; and from robbing he comes next to drinking and sabbath-breaking, and from that to incivility and procrastination.

\section{}

\subsection{}

\blfootnote{`The Dalesman's Litany', Prof Frederic Moorman (1872 -- 1918), \cite{moorman}. Tim Hart wrote in the sleeve notes to \refbook{Folk Songs of Old England Vol 1} (1968):`The words of this song were collected by F W Moorman who was president of the Yorkshire Dialect Society during the latter part of the 19th century. The beautiful haunting melody was written only a few years ago by Dave Keddie of Bradford to whom we are indebted for allowing its inclusion on this record. Although the lyrics were originally in broad dialect Tim translated them where necessary to enable more people to understand them.' The penultimate line of Moorman's original has been excised in order to fit the tune.}\begin{center}
\textit{Tune: The Dalesman's Litany}
\end{center}

\settowidth{\versewidth}{It's hard when folks can't find their work}
\begin{verse}[\versewidth]
It's hard when folks can't find their work\\*
\vin Where they've been bred an' born;\\
I were young I always thought\\
\vin I'd bide 'mong t'roots an' corn.\\
I've bin forced to work i' towns,\\
\vin So here's my litany:\\
From \textsc{Hull}, an' \textsc{Halifax}, an' hell,\\*
\vin Good Lord, deliver me!\\!

When I were courtin' Mary Ann,\\*
\vin T'owd squire, he says one day:\\
I've got no bield for wedded folks;\\
\vin Choose, wilt ta wed or stay?\\
I couldn't gi'e up t'lass I loved;\\
\vin To t'town we had to flee:\\
From \textsc{Hull}, an' \textsc{Halifax}, an' hell,\\*
\vin Good Lord, deliver me!\\!

I've wrought i'\textsc{Leeds} an' \textsc{Huddersfield},\\*
\vin An' addled honest brass;\\
I'\textsc{Bradford}, \textsc{Keighley}, \textsc{Rotherham},\\
\vin I've kept my barns an' lass.\\
I've travelled all three Ridin's round,\\
\vin And once I went to sea:\\
From forges, mills, an' coalin' boats,\\*
\vin Good Lord, deliver me!\\!

I've walked at night through \textsc{Sheffield} lanes,\\*
\vin 'Twere same as bein' i' hell:\\
Furnaces thrust out tongues o' fire,\\
\vin An' roared like t'wind on t'fell.\\
I've sammed up coals i'\textsc{Barnsley} pits,\\
\vin Wi' muck up to my knee:\\
From \textsc{Sheffield}, \textsc{Barnsley}, \textsc{Rotherham},\\*
\vin Good Lord, deliver me!\\!

I've seen grey fog creep o'er \textsc{Leeds Bridge}\\*
\vin As thick as bastile soup;\\
I've lived where folks were stowed away\\
\vin Like rabbits in a coop.\\
I've watched snow float down \textsc{Bradford Beck}\\
\vin As black as ebony:\\
From \textsc{Hunslet}, \textsc{Holbeck}, \textsc{Wibsey Slack},\\*
\vin Good Lord, deliver me!\\!

But now, when all wer children's fledged,\\*
\vin To t'country we've come back.\\
There's 40 mile o' heathery moor\\
\vin Twix' us an' t'coal-pit slack.\\
And when I sit o'er t'fire at night,\\
\vin I laugh an' shout wi' glee:\\
From \textsc{Hull}, an' \textsc{Halifax}, an' hell,\\*
\vin T'good Lord's delivered me!
\end{verse}

\subsection{}

\blfootnote{Robert Herrick (1591 -- 1674), \cite{treasury}.}\settowidth{\versewidth}{That brave vibration each way free;}
\begin{verse}[\versewidth]
Whenas in silks my \textit{Julia} goes\\*
Then, then (methinks) how sweetly flows\\*
That liquefaction of her clothes.\\!

Next, when I cast mine eyes \& see\\*
That brave vibration each way free;\\*
O how that glittering taketh me!
\end{verse}

\subsection{}

\blfootnote{Bernard de Voto (1897 -- 1955), \cite{odq}.}The proper union of gin and vermouth is a great and sudden glory; it is one of the happiest marriages on earth, and one of the shortest lived.

\section{}

\subsection{}

\blfootnote{Anonymous, \cite{rusby_hourglass}. This song seems to have originated from a broadside printed around 1820.}\begin{center}
\textit{Tune: We're All Jolly Fellows that Follow the Plough}
\end{center}

\settowidth{\versewidth}{You've not ploughed your acre. I'll swear and I'll vow:}
\begin{verse}[\versewidth]
It was early one morning at the break of the day;\\*
The farmer came to us, and this he did say:\\
`Come rise up, young fellows, with the best of good will.\\*
Your horses need something, their bellies to fill.'\\!

When four o'clock comes, my boys, it's up we do rise,\\*
And off to the stable we merrily flies.\\
With a-rubbing and a-scrubbing, our horses will go,\\*
For we're all jolly fellows that follows the plough.\\!

When six o'clock comes, at breakfast we'll meet,\\*
And with cold beef and pork we'll heartily eat.\\
With a piece in our pocket, to the fields we do go,\\*
For we're all jolly fellows that follows the plough.\\!

Then up spoke the farmer, and this he did say:\\*
`What have you been doing this long summer's day?\\
You've not ploughed your acre. I'll swear and I'll vow:\\*
You are all lazy fellows that follows the plough.'\\!

Then up spoke our carter, and this he did cry:\\*
`We've all ploughed our acre. You tell us a lie.\\
We've all ploughed our acre. I'll swear and I'll vow:\\*
We are all jolly fellows that follows the plough.'\\!

Then up spoke the farmer, and laughed at the joke:\\*
`O it's gone \sfrac{$1$}{$2$} past two boys. It's time to unyoke.\\
Unharness your horses, and rub them down well,\\*
And I'll give you a jug of the very best ale.'\\!

So all you young ploughboys, where'er you may be,\\*
Come take this advice and be ruled by me:\\
Never fear any master, where'er you may go,\\*
Fror we're all jolly fellows that follows the plough.
\end{verse}

\subsection{}

\blfootnote{`Rondeau', Leigh Hunt (1784 -- 1859), \cite{obev}. The Jenny in question was Mrs Jane Carlyle, wife of the polymath Thomas Carlyle. She kissed the poet -- she was married to Carlyle at the time -- during an influenza epidemic, when the newly-recovered Hunt made an unexpected visit to the couple's home.}\settowidth{\versewidth}{    Sweets into your list, put that in!}
\begin{verse}[\versewidth]
\textit{Jenny} kissed me when we met,\\*
\vin Jumping from the chair she sat in;\\
Time, you thief, who love to get\\
\vin Sweets into your list, put that in!\\
Say I'm weary, say I'm sad,\\
\vin Say that health \& wealth have missed me,\\
Say I'm growing old, but add\\*
\vin \textit{Jenny} kissed me.
\end{verse}

\subsection{}

\blfootnote{Charles Dibdin (1745 -- 1814), \cite{odq}. \refbook{The Oxford Dictionary of Quotations} gives `finds', but at least one broadsheet ballad gives `find'.}In every mess I find a friend, in every port a wife

\section{}

\subsection{}

\blfootnote{`The Blaydon Races', George Ridley (1835 -- 1864), \cite{spinners}. The words here are from The Spinners' version; they seem to have modifed Ridley's lyrics to improve the meter.}\begin{center}
\textit{Tune: The Blaydon Races}
\end{center}

\settowidth{\versewidth}{To gan an' see Geordie Ridley's show at the Mechanics' Hall in Blaydon.}
\begin{verse}[\versewidth]
I went to \textsc{Blaydon} races. 'Twas on the ninth o' june,\\*
Eighteen hundred \& 62 on a summer's afternoon.\\
I took the bus from \textsc{Balmbras}, an' she was heavy laden.\\*
Away we went along \textsc{Collingwood Street} that's on the road to \textsc{Blaydon}.\\!

{\itshape
O lads! You should've seen us gannin'!\\
Passin' the folks along the road, just as they were stan'in':\\
Al' the lads \& lasses there, an' al' wi' smilin' faces,\\*
Gannin' along the \textsc{Scotswood Road} to see the \textsc{Blaydon} races.}\\!

We flew past Armstrong's factory an' up to the \textsc{Robin Adair}.\\*
Gannin' down the railway bridge, the bus wheel flew off there.\\
The lasses lost their crinolines \& the veils that hide their faces.\\*
I got two black eyes \& a broken nose a-gannin to \textsc{Blaydon} races.\\!

An' when we got the wheel put on, away we went again.\\*
Them that had their noses broke had to go back ower hyem.\\
Some went to the dispensary, an' some to Dr \textit{Gibbs},\\*
An' some went to the infirmary to mend their broken ribs.\\!

An' when we got to \textsc{Paradise}, there was bonnie gam's begun.\\*
There was four \& 20 on the bus, man. How they danced \& sung!\\
They called on me to sing a song, and I sang them ``Paddy Fagan''.\\*
An' I danced a jig an' I swung me twig on the day we went to \textsc{Blaydon}.\\!

We flew across the chain bridge right into \textsc{Blaydon} toon.\\*
The bellman he was callin' there. They call him \textit{Jacky Broon}.\\
I saw 'im talkin' to some chaps, an' he was them persuadin'\\*
To gan an' see \textit{Geordie Ridley}'s show at the \textsc{Mechanics' Hall} in \textsc{Blaydon}.\\!

The rain it poured al' the day an' made the ground quite muddy.\\*
\textit{Coffee John} 'ad a white hat on, an' he yelled, We stole a cuddy!\\
There was spice stalls \& monkey stalls an' old wives sellin' ciders.\\*
An' a chap wi' a ha'penny round about shoutin, Now, me lads, for riders!\\!

{\itshape
O lads! You should've seen us gannin!\\
Passin' the folks along the road, just as they were stan'in':\\
Al' the lads \& lasses there, an' al' wi' smilin' faces,\\*
Gannin along the \textsc{Scotswood Road} to see the \textsc{Blaydon} races.}
\end{verse}

\subsection{}

\blfootnote{Ben Jonson (1572 -- 1637), \cite{londonbook}. This song is sung by Patrico in Jonson's masque \refbook{The Gipsies Metamorphosed}.}\settowidth{\versewidth}{And the foul to be loved at leisure.}
\begin{verse}[\versewidth]
The fairy beam upon you,\\*
The stars to glister on you;\\
\vin A moon of light\\
\vin In the noon of night,\\*
Till the fire-drake hath o'ergone you.\\!

The wheel of fortune guide you\\*
The boy with the bow beside you;\\
\vin Run aye in the way\\
\vin Till the bird of day,\\*
And the luckier lot betide you.\\!

To the old, long life and treasure,\\*
To the young, all health and pleasure;\\
\vin To the fair, their face\\
\vin With eternal grace,\\*
And the foul to be loved at leisure.\\!

To the witty, all clear mirrors,\\*
To the foolish, their dark errors;\\
\vin To the loving sprite,\\
\vin A secure delight;\\*
To the jealous, his own false terrors.
\end{verse}

\subsection{}

\blfootnote{Miss Emily Dickinson (1830 -- 1886), \cite{odq}.}We turn not older with years, but newer every day.

\section{}

\subsection{}

\blfootnote{`Who's the Fool Now?', Anonymous, \cite{oldengland1}. This song is truly ancient; a version of it apppears in Ravenscroft's \refbook{Deuteromelia} (1609). The narrative behind it is a rich man and his servant drinking together. The rich man is drinking considerably larger quantities, and is amusing his servant with improbable tales. \P 9. The `man in the moon' is said to refer here to Henry VIII, and his troubles with Clement VII, although this seems more likely to be just nonsense verse, at least at first glance.}\begin{center}
\textit{Tune: Who's the Fool Now?}
\end{center}

\settowidth{\versewidth}{Twenty miles above the ground.}
\begin{verse}[\versewidth]
O \textit{Martin} said to his man,\\*
\vin \textit{Fie, man! Fie!}\\
\textit{Martin} said to his man,\\
\vin \textit{Who's the fool now?}\\
\textit{Martin} said to his man,\\
Fill thou the cup and I the can.\\
\vin \textit{Thou hast well drunken, man.}\\*
\vin \textit{Who's the fool now?}\\!

I saw the man in the moon\\*
Clouting of St \textit{Peter}'s shoon.\\!

I saw a hare chase a hound\\*
Twenty miles above the ground.\\!

I saw a mouse chase a cat,\\*
Saw a cheese eat a rat.\\!

O \textit{Martin} said to his man,\\*
\vin \textit{Fie, man! Fie!}\\
\textit{Martin} said to his man,\\
\vin \textit{Who's the fool now?}\\
\textit{Martin} said to his man,\\
Fill thou the cup and I the can.\\
\vin \textit{Thou hast well drunken, man.}\\*
\vin \textit{Who's the fool now?}
\end{verse}

\subsection{}

\blfootnote{`Cloe', George Granville, Baron Lansdowne (1666 -- 1735), \cite{obev}.}\settowidth{\versewidth}{    'Tis well her heart is tender.}
\begin{verse}[\versewidth]
\textit{Cloe}'s the wonder of her sex;\\*
\vin 'Tis well her heart is tender.\\
How might such killing eyes perplex\\*
\vin With virtue to defend her?\\!

But nature, graciously inclined,\\*
\vin With liberal hand to please us,\\
Has to her boundless beauty joined\\*
\vin A boundless bent to ease us.
\end{verse}

\subsection{}

\blfootnote{The Rev Charles Dodgson (1832 -- 1898), \cite{odq}.}Everybody has won, and all must have prizes

\section{}

\subsection{}

\blfootnote{`The Devil and the Ploughman', Anonymous, \cite{herringbone}.}\begin{center}
\textit{Tune: Lily Bulero}
\end{center}

\settowidth{\versewidth}{Saying, Take her back, father; she'll murder us all.}
\begin{verse}[\versewidth]
O there was an old farmer in Sussex did dwell,\\*
And he had a wife who he didn't love well.\\!

Well the devil he came to the farmer at plough,\\*
Saying one of your family I've got to have now.\\!

Well you see, mister farmer, I've come for your wife.\\*
I hear she's the bane and torment of your life,\\*
So now I'll take her without more strife.\\!

O take her, O take her, with all of my heart,\\*
And I'm hoping that you \& she never will part.\\!

So the devil he hoisted her up on his hump,\\*
And off down to hell he has gone with a jump.\\!

And when they've travelled to purgatory's gate,\\*
The night it was dark and the hour it was late.\\*
Says he, Take in an old Sussex chap's mate.\\!

There were two little devil a-dancing in chains;\\*
She took out a stick and clattered their brains.\\!

There were two little devils a-playing at ball,\\*
Saying, Take her back, father; she'll murder us all.\\!

So the devil he hoisted her up on his back,\\*
Like a bunch of potatoes bound up in a sack,\\*
And back to her house she's gone with a crack.\\!

Well I've been a tormentor for most of my life,\\*
But I never knew torment till I met your wife.\\!

This proves that the women is better than men;\\*
They go down to hell and get sent back again.
\end{verse}

\subsection{}

\blfootnote{`Sea Love', Miss Charlotte Mew (1869 -- 1928), \cite{obev}.}\settowidth{\versewidth}{Here's the same little fishes that sputter and swim,}
\begin{verse}[\versewidth]
Tide be runnin' the great world over:\\*
\vin 'Twas only last june month I mind that we\\
Was thinking the toss \& the call in the breast of the lover\\*
\vin So everlastin' as the sea.\\!

Here's the same little fishes that sputter and swim,\\*
\vin Wi' the moon's old glim on the grey, wet sand;\\
An' him no more to me nor me to him\\*
\vin Than the wind goin' over my hand.
\end{verse}

\subsection{}

\blfootnote{The Rt Hon Sir Winston Spencer-Churchill (1874 -- 1965), \cite{odq}.}Mr Gladstone read Homer for fun, which I thought served him right.

\section{}

\subsection{}

\blfootnote{`The Streams of Lovely Nancy', Anonymous, \cite{rusby_awkward}. The origin and meaning of this beautiful song are as fascinating as they are obscure, but an educated guess seems to be that it was cobbled together in the West Country in the eighteenth century from a number of sailors' songs.}\begin{center}
\textit{Tune: The Streams of Lovely Nancy}
\end{center}

\settowidth{\versewidth}{And the noise in the valley makes the rocks for to ring.}
\begin{verse}[\versewidth]
O the streams of lovely \textit{Nancy} divide in three parts,\\*
Oh, the streams of lovely Nancy are divided in three parts\\
It's the drinking of good liquor that makes my heart sing,\\*
And the noise in the valley makes the rocks for to ring.\\!

At the top of this mountain, my love's castle stands,\\*
It's all o'erbuilt with ivory on yonder black sands.\\
Fine arches, fine porches, and diamonds so bright:\\*
It's a beacon for a sailor on a dark winter's night.\\!

On yonder high mountain where the wild fowl do fly.\\*
There is one in amongst them that flies very high.\\
If I had her in my arms, near the diamond's black land\\*
How soon I would secure her by the sleight of my hand.\\!

At the base of this mountain a river runs clear.\\*
A ship from the Indies did once anchor there,\\
With her red flags a-flying, the beating of her drum,\\*
Sweet instruments of music and the firing of her gun.
\end{verse}

\subsection{}

\blfootnote{`The Raining Summer', Mrs Alice Meynell (1847 -- 1922), \cite{oxfordlarkin}.}\settowidth{\versewidth}{    The winds hunt up the sun, hunt up the moon,}
\begin{verse}[\versewidth]
There's much afoot in heaven \& earth this year;\\*
\vin The winds hunt up the sun, hunt up the moon,\\
Trouble the dubious dawn, hasten the drear\\*
\vin Height of a threatening noon.\\!

No breath of boughs, no breath of leaves, of fronds,\\*
\vin May linger or grow warm; the trees are loud;\\
The forest, rooted, tosses in her bonds,\\*
\vin And strains against the cloud.\\!

No scents may pause within the garden-fold;\\*
\vin The rifled flowers are cold as ocean-shells;\\
Bees, humming in the storm, carry their cold\\*
\vin Wild honey to cold cells.
\end{verse}

\subsection{}

\blfootnote{The Rt Hon Sir Winston Spencer-Churchill (1874 -- 1965), \cite{odq}.}This is the sort of English up with which I will not put.

\section{}

\subsection{}

\blfootnote{`Smith of Bristol', Anonymous, \cite{dublinersa}. The origins of this song are obscure, but it may have been written in Ireland, and perhaps was originally intended as a parody, similar to \refpoem{The Auld Orange Flute}.}\begin{center}
\textit{Tune: Smith of Bristol}
\end{center}

\settowidth{\versewidth}{He'd watch his beaten foemen jump out into the tide,}
\begin{verse}[\versewidth]
\textit{Smith} was a \textsc{Bristol} man and a rare old sort was he.\\*
\vin \textit{With his cutlass \& his pistol, heave your ho!}\\
With a noble crew of cutthroats, he used to scour the sea,\\
A-plundering and a-robbing high \& low.\\
He swore 'twas no concernin',\\
He didn't give a herrin',\\
About right or wrong or any holy show.\\
He swore that grabbin' booty\\
Was Britain's foremost duty,\\*
Wherever she could get it! Heave your ho!\\!

{\itshape
Heave your ho! Heave your ho! Heave your ho!\\
He swore that grabbin' booty\\
Was Britain's foremost duty,\\*
Wherever she could get it! Heave your ho!}\\!

\textit{Smith} had a noble soul and lofty was his pride.\\*
He'd watch his beaten foemen jump out into the tide,\\
All ye beggers who had nowhere else to go.\\
And hanging from his lanyards\\
Was portuguese and spaniards;\\
And beatmen frenchmen jumping to \& fro,\\
Right along the blazin' story,\\
Shall illumine England's glory:\\*
Pirate \textit{Smith} of \textsc{Bristol}, heave your ho!\\!

But accidents can happen even to heroes such as he.\\*
\vin \textit{With his cutlass \& his pistol, heave your ho!}\\
He was standing on his capstan as happy as could be,\\
Hoping soon to have another prize in tow,\\
When a whistling spanish bullet\\
Came and caught him in his gullet,\\
And very sad to say it laid him low.\\
He was only 97\\
When his soul had gone to heaven\\*
To rest on \textit{Nelson}'s bosom. Heave your ho!\\!

{\itshape
Heave your ho! Heave your ho! Heave your ho!\\
He swore that grabbin' booty\\
Was Britain's foremost duty,\\*
Wherever she could get it! Heave your ho!}
\end{verse}

\subsection{}

\blfootnote{`Celanta at the Well of Life', George Peele (1556 -- 1596), \cite{londonbook}. These lines are from Peele's play \refbook{The Old Wives' Tale}.}\settowidth{\versewidth}{For fear you make the golden beard to weep.}
\begin{verse}[\versewidth]
Gently dip, but not too deep,\\*
For fear you make the golden beard to weep.\\
Fair maiden, white \& red,\\
Comb me smooth, and stroke my head,\\*
And thou shalt have some cockell bread.\\!

Gently dip, but not too deep,\\*
For fear thou make the golden beard to weep.\\
Fair maid, white \& red,\\
Comb me smooth, and stroke my head,\\
And every hair a sheaf shall be,\\*
And every sheaf a golden tree.
\end{verse}

\subsection{}

\blfootnote{Oscar Wilde (1854 -- 1900), \cite{aaat}.}A poet can survive everything but a misprint.

\section{}

\subsection{}

\blfootnote{Anonymous, \cite{rusby_underneath}. This song seems to have its roots in an old Scottish song, Child Ballad 274, but from there it has grown into thousands of forms, in a number of European languages.}\begin{center}
\textit{Tune: The Goodman}
\end{center}

\settowidth{\versewidth}{It's sheets, just sheets! cried the goodman's wife,}
\begin{verse}[\versewidth]
The goodman he came home one night.\\*
The goodman home came he.\\
There he spied an old saddle horse\\
Where no horse should there be.\\
It's a cow! It's a cow! cried the goodman's wife,\\
A cow, just a cow, can't you see?\\
Far have I ridden, and much I've seen,\\*
But a saddle on a cow has never been.\\!

The goodman he came home one night.\\*
The goodman home came he.\\
There he spied a powdered wig\\
Where no wig should there be.\\
It's a hen! It's a hen! cried the goodman's wife,\\
A hen, just a hen, can't you see?\\
Far have I ridden, and much I've seen,\\*
But powder on a hen has never been.\\!

The goodman he came home one night.\\*
The goodman home came he.\\
There he spied a riding coat\\
Where no coat should there be.\\
It's sheets, just sheets! cried the goodman's wife,\\
Sheets, just sheets, can't you see?\\
Far have I ridden, and much I've seen,\\*
But buttons on a sheet has never been.\\!

The goodman he came home one night.\\*
The good man home came he.\\
There he spied a handsome man\\
Where no man should there be.\\
It's the maid! It's the maid! cried the goodman's wife,\\
The milking maid, can't you see?\\
Far have I ridden, and much I've seen,\\*
But a beard on a maid has never been.
\end{verse}

\subsection{}

\blfootnote{`A Half-Way Pause', Gabriel Rossetti (1828 -- 1882), \cite{obev}.}\settowidth{\versewidth}{    And will not be at once gone through.}
\begin{verse}[\versewidth]
The turn of noontide has begun.\\*
\vin In the weak breeze the sunshine yields.\\
\vin There is a bell upon the fields.\\
On the long hedgerow's tangled run\\
\vin A low white cottage intervenes:\\
\vin Against the wall a blind man leans,\\*
And sways his face to have the sun.\\!

Our horses' hoofs stir in the road,\\*
\vin Quiet \& sharp. Light hath a song\\
\vin Whose silence, being heard, seems long.\\
The point of noon maketh abode,\\
\vin And will not be at once gone through.\\
\vin The sky's deep colour saddens you,\\*
And the heat weighs a dreamy load.
\end{verse}

\subsection{}

\blfootnote{Oscar Wilde (1854 -- 1900), \cite{odq}.}A thing is not necessarily true because a man dies for it.

\section{}

\subsection{}

\blfootnote{Anonymous, \cite{rusby_underneath}. This song is essentially the same as the seventy-seventh Child Ballad 77.}\begin{center}
\textit{Tune: Sweet William's Ghost}
\end{center}

\settowidth{\versewidth}{`Tis' time, tis' time, my dear Margaret}
\begin{verse}[\versewidth]
There came a ghost to \textit{Margaret}'s door\\*
With many a grievious groan,\\
And, ay, he's turled long at the pin,\\
But answer she gave none.\\
`Is it my father \textit{Philip}?\\
Or yet my brother \textit{John}?\\
Or yet my own dear \textit{William}\\*
From Scotland now came home?'\\!

`My faith \& troth you'll never get,\\*
Of me you'll never win,\\
Till you take me to yon churchyard\\
And wed me with a ring.'\\
`O I do dwell in a churchyard\\
But far beyond the sea,\\
And it is but my ghost, \textit{Margaret},\\*
That speaks now unto thee.'\\!

So she's put on her robes of green,\\*
With a piece below the knee,\\
And all the live long winter's night\\
The sweet ghost followed she.\\
`O is there room at your head, \textit{Willie},\\
Or room here at your feet,\\
Or room here at your side, \textit{Willie},\\*
Wherein that I may sleep?'\\!

`There's no room at my head, \textit{Margaret}.\\*
There's no room at my feet.\\
There's no room at my side, \textit{Margaret}.\\
My coffin is so neat.'\\
Then up and spoke the red robin,\\
And up and spoke the grey.\\
`Tis' time, tis' time, my dear \textit{Margaret}\\*
That I were gone away.'\\!

No more the ghost to \textit{Margaret} came\\*
With many a grievious groan.\\
He's vanished out into the mist\\
And left her there alone\\
`O stay my own true love, stay!\\
My heart you do divide.'\\
Pale grew her cheeks. She closed her eyes,\\*
Stretched out her limbs and cried.
\end{verse}

\subsection{}

\blfootnote{`The Silent Noon', Gabriel Rossetti (1828 -- 1882), \cite{newlove}. This is \refbook{The House of Life} \S 19.}\settowidth{\versewidth}{    The finger-points look through like rosy blooms:}
\begin{verse}[\versewidth]
Your hands lie open in the long fresh grass --\\*
\vin The finger-points look through like rosy blooms:\\
\vin Your eyes smile peace. The pasture gleams \& glooms\\
'Neath billowing skies that scatter \& amass.\\
All round our nest, far as the eye can pass,\\
\vin Are golden kingcup fields with silver edge\\
\vin Where the cow parsley skirts the hawthorn hedge.\\*
'Tis visible silence, still as the hour-glass.\\!

Deep in the sun-searched growths the dragon-fly\\*
Hangs like a blue thread loosened from the sky:\\
\vin So this winged hour is dropped to us from above.\\
\vin \vin O clasp we to our hearts, for deathless dower,\\
\vin \vin This close-companioned inarticulate hour\\*
\vin When twofold silence was the song of love.
\end{verse}

\subsection{}

\blfootnote{Oscar Wilde (1854 -- 1900), \cite{happyprince}.}Hard work is simply the refuge of people who have nothing whatever to do.

\section{}

\subsection{}

\blfootnote{`An Acre of Land', Anonymous, \cite{albionband}. This song is closely related to Child Ballad 2.}\begin{center}
\textit{Tune: Y'Acre of Land}
\end{center}

\settowidth{\versewidth}{And scattered it over with one peppercorn.}
\begin{verse}[\versewidth]
My father he left me an acre of land --\\*
\vin \textit{Sing hey! Sing ho! Sing hivy!}\\
My father he left me an acre of land --\\*
\vin \textit{And a bunch of holly \& ivy!}\\!

And we ploughed it up with a ram's horn,\\*
And scattered it over with one peppercorn.\\!

And we reaped it with a sickle of leather,\\*
And tied it all up with a tomtit's feather.\\!

And we threshed it with a peddler's pack,\\*
\vin \textit{Sing hey! Sing ho! Sing hivy!}\\
And carried it home on a butterfly's back.\\*
\vin \textit{And a bunch of holly \& ivy!}
\end{verse}

\subsection{}

\blfootnote{`Everyone Sang', Siegfried Sassoon (1886 -- 1967), \cite{norton}. This short poem commemorates the 1918 armistice.}\settowidth{\versewidth}{Orchards and dark-green fields; on -- on -- and out of sight.}
\begin{verse}[\versewidth]
Everyone suddenly burst out singing;\\*
And I was filled with such delight\\
As prisoned birds must find in freedom,\\
Winging wildly across the white\\*
Orchards and dark-green fields; on -- on -- and out of sight.\\!

Everyone's voice was suddenly lifted;\\*
And beauty came like the setting sun:\\
My heart was shaken with tears; and horror\\
Drifted away... O but everyone\\*
Was a bird; and the song was wordless; the singing will never be done.
\end{verse}

\subsection{}

\blfootnote{Oscar Wilde (1854 -- 1900), \cite{odq}. Wilde was describing Dr William Henley.}He has fought the good fight and has had to face every difficulty except popularity.

\section{}

\subsection{}

\blfootnote{Anonymous, \cite{herringbone}.}\begin{center}
\textit{Tune: John Blount}
\end{center}

\settowidth{\versewidth}{But you spoke the first word, John Blunt, she said,}
\begin{verse}[\versewidth]
There was an old couple lived under a hill,\\*
\vin And \textit{Blunt} it was their name O.\\
And they had a good beer and ale for to sell\\*
\vin And it bore a wonderful fame O.\\!

\textit{John Blunt} \& his wife they drank of the drink\\*
\vin Till they could drink no more O.\\
They both got tired and went up to bed\\*
\vin But forgot to bar the door O.\\!

So they made a bargain; a bargain they made.\\*
\vin They made it strong \& sure O.\\
The first of them to speak the first word\\*
\vin Gets up to bar the door O.\\!

So there came ramblers, ramblers three,\\*
\vin Travelling in the night O.\\
And so in need of lodgings were they,\\*
\vin They crept in by candlelight O.\\!

They went to his larder and ate up his food\\*
\vin Till they could eat no more O;\\
But never a word did the old couple speak\\*
\vin For fear of who stood at the door O.\\!

And down to his cellar they drank of his drink\\*
\vin Till they could drink no more O;\\
But never a word did the old couple speak\\*
\vin For fear of who stood at the door O.\\!

They went upstairs; they went to his room;\\*
\vin They broke it down the door O;\\
But never a word did the old couple speak\\*
\vin For fear of who stood at the door O.\\!

They hauled his wife all out of the bed,\\*
\vin Laid her out on the floor O.\\
Then up got \textit{John}, said, It's time to be gone,\\*
\vin For he could stand no more O.\\!

He said, You've eaten my food and drunk of my drink,\\*
\vin And my wife you hurt full sore O.\\
But you spoke the first word, \textit{John Blunt}, she said,\\*
\vin So go down and bar the door O.
\end{verse}

\subsection{}

\blfootnote{William Shakespeare (1564 -- 1616), \cite{treasury}.}\settowidth{\versewidth}{Though absence seemed my flame to qualify;}
\begin{verse}[\versewidth]
O never say that I was false of heart\\*
Though absence seemed my flame to qualify;\\
As easy might I from myself depart\\*
As from my soul, which in thy breast doth lie.\\!

That is my home of love; if I have ranged,\\*
Like him that travels, I return again,\\
Just to the time, not with the time exchanged,\\*
So that myself bring water for my stain.\\!

Never believe, though in my nature reigned\\*
All frailties that besiege all kinds of blood,\\
That it could so preposterously be stained\\*
To leave for nothing all thy sum of good;\\!

For nothing this wide universe I call,\\*
Save thou, my rose: in it thou art my all.
\end{verse}

\subsection{}

\blfootnote{Oscar Wilde (1854 -- 1900), \cite{pomegranates}.}He is really not so ugly after all, provided, of course, that one shuts one's eyes.

\section{}

\subsection{}

\blfootnote{Anonymous, \cite{oldengland2}.}\begin{center}
\textit{Tune: Paddy Stole the Rope}
\end{center}

\settowidth{\versewidth}{If that's the way you steal a rope, it's no wonder that you're groaning.}
\begin{verse}[\versewidth]
There were once two irish labouring men; to England they came over;\\*
And they tramped about in search of work from \textsc{Liverpool} to \textsc{Dover}.\\
Says \textit{Mike} to \textit{Pat}, I'm tired of this. We're both left in the lurch,\\
And if I don't get work, bedad, I'll go and rob a church.\\
What, rob a church! says \textit{Pat} to \textit{Mike}; How could you be so vile?\\
For something sure would happen while we're going down the aisle.\\
But if you do, I'll go with you, and we'll get safe out, I hope.\\*
So, listen and I'll tell you now how \textit{Paddy} stole the rope.\\!

Well off they went with theft intent to the place they wanted finding,\\*
And they broke inside of a country church where nobody was minding.\\
And they scraped together all they could, they were prepared to slope,\\
When \textit{Paddy} says, Now hold on, \textit{Mike}. What have we got for rope?\\
For we have no bag to hold our swag, and before we get outside,\\
With something strong \& stout, my lad, this bundle must be tied.\\
And just then he spied the church bell rope, and before you could turn about,\\*
He did ride up the belfry high to climb that rope so stout.\\!

And holding on by one hand \& leg, he pulled his clasp knife out,\\*
And right above his hand \& head he cut that rope so stout.\\
Well he quite forgot it held him up, and, by the holy Pope,\\
Down to the bottom of the church fell Paddy and the rope.\\
Come out of that, says \textit{Mike} to \textit{Pat}, and will you stop your moaning,\\
If that's the way you steal a rope, it's no wonder that you're groaning.\\
I'll show you how to steal the rope, if you lend to me your knife.\\*
O \textit{Mike}, go easy, says old \textit{Pat}, or else you'll lose your life.\\!

So \textit{Mike} climbed up the other rope, just like an antelope,\\*
But instead of cutting it off above, he cut it off below.\\
Down fell the other piece of rope, and \textit{Mike} was left on high.\\
Come down, says \textit{Pat}. I can't, says \textit{Mick}, for if I drop, I'll die.\\
Well their noise it brought the beagles out, the sexton \& police,\\
And though they got poor \textit{Micky} down, they spared them no release.\\
And for their ingenuity they have now a wider scope\\*
Than when they broke inside a church to go and steal the rope.
\end{verse}

\subsection{}

\blfootnote{`Piping Peace', James Shirley (1596 -- 1666), \cite{londonbook}. This song is sung in Act I of Shirley's \refbook{The Imposture}.}\settowidth{\versewidth}{With choice of sweetest flowers make}
\begin{verse}[\versewidth]
You virgins, that did late despair\\*
To keep your wealth from cruel men,\\
Tie up in silk your careless hair:\\*
Soft peace is come again.\\!

Now lovers' eyes may gently shoot\\*
A flame that will not kill;\\
The drum was angry, but the lute\\*
Shall whisper what you will.\\!

Sing, I{\"{o}}, I{\"{o}}! for his sake\\*
That hath restored your drooping heads;\\
With choice of sweetest flowers make\\*
A garden where he treads;\\!

Whilst we whole groves of laurel bring,\\*
A petty triumph for his brow,\\
Who is the master of our spring\\*
And all the bloom we owe.
\end{verse}

\subsection{}

\blfootnote{Oscar Wilde (1854 -- 1900), \cite{idlerwilde}.}I have the simplest tastes; I am always satisfied with the best.

\section{}

\subsection{}

\blfootnote{Anonymous, \cite{albionband}. The words to this song are said to have been written by one L M Thornton, flourishing in 1860.}\begin{center}
\textit{Tune: The Postman's Knock}
\end{center}

\settowidth{\versewidth}{When you open the door to his loud rat-tat}
\begin{verse}[\versewidth]
What a wonderful man the postman is\\*
As he hastens from door to door!\\
What medley of news his hands contain\\
For high, low, rich, \& poor!\\
In many's the face the joy he can trace,\\
In many's the grief he can see,\\
When you open the door to his loud rat-tat\\*
And his quick delivery.\\!

{\itshape
Every morning as true as the clock\\*
Somebody hears the postman's knock!}\\!

Number 1 he presents with news of a birth;\\*
With tidings of death, \textnumero 4.\\
At 13 a bill of terrible length\\
He drops through the hole in the door.\\
Now a cheque or an order for 15 he leaves\\
In 16 his presence to prove,\\
While 17 doth an acknowledgement get,\\*
And 18 a letter of love.\\!

And the mail must get through\\*
Whatever the hazards or odds.\\
This low man of letters just peddles on through\\
Pursued by a pack of wild dogs.\\
But ease \& complaining whatever the trial\\
Or beating he never retreats,\\
For you get a free bag \& a hat with a badge\\*
And it's better than walking the streets.\\!

{\itshape
Every morning as true as the clock\\*
Somebody hears the postman's knock!}
\end{verse}

\subsection{}

\blfootnote{`Tall Nettles', Edward Thomas (1878 -- 1917), \cite{obev}.}\settowidth{\versewidth}{    Except to prove the sweetness of a shower.}
\begin{verse}[\versewidth]
Tall nettles cover up, as they have done\\*
\vin These many springs, the rusty harrow, the plough\\
Long worn out, and the roller made of stone:\\*
\vin Only the elm butt tops the nettles now.\\!

This corner of the farmyard I like most:\\*
\vin As well as any bloom upon a flower\\
I like the dust on the nettles, never lost\\*
\vin Except to prove the sweetness of a shower.
\end{verse}

\subsection{}

\blfootnote{Oscar Wilde (1854 -- 1900), \cite{decayoflying}.}If nature had been comfortable, mankind would never have invented architecture.

\section{}

\subsection{}

\blfootnote{Anonymous, \cite{carthy_rigs}.}\begin{center}
\textit{Tune: The Fox Hunt}
\end{center}

\settowidth{\versewidth}{Old Reynold beat and out of breath and dreading of these hounds,}
\begin{verse}[\versewidth]
You gentlemen of high renown, come listen unto me\\*
That take delight in fox hunting by every degree.\\
A story now I'll tell to you concerning of a fox,\\*
O'er \textsc{Royston} hills \& mountains high \& over stony rocks.\\!

\textit{Old Reynold} being in his den and hearing of these hounds,\\*
Which made him for to prick his ears and tread upon the ground.\\
`Methink me hear some jubal hounds pressing upon my life;\\*
Before that they do come to me I'll tread upon the ground.'\\!

We hunted full four hours or more by parishes 16.\\*
We hunted full four hours or more and came by \textsc{Barkworth Green}.\\
`O if you'll only spare my life I'll promise and fulfil:\\*
I'll touch no more your feathered fowl nor lambs in yonder fold.'\\!

\textit{Old Reynold} beat and out of breath and dreading of these hounds,\\*
Thinking that he might lose his life before these jubal hounds:\\
`O here's adieu to duck \& geese, likewise young lamb also.'\\*
They've got \textit{Old Reynold} by the brush and will not let him go.
\end{verse}

\subsection{}

\blfootnote{`Adlestrop', Edward Thomas (1878 -- 1917), \cite{norton}.}\settowidth{\versewidth}{Of heat the express-train drew up there}
\begin{verse}[\versewidth]
Yes, I remember \textsc{Adlestrop} --\\*
\vin The name, because one afternoon\\
Of heat the express-train drew up there\\*
\vin Unwontedly. It was late june.\\!

The steam hissed. Someone cleared his throat.\\*
\vin No one left \& no one came\\
On the bare platform. What I saw\\*
\vin Was \textsc{Adlestrop} -- only the name\\!

And willows, willow-herb, \& grass,\\*
\vin And meadowsweet, \& haycocks dry,\\
No whit less still \& lonely fair\\*
\vin Than the high cloudlets in the sky.\\!

And for that minute a blackbird sang\\*
\vin Close by, and round him, mistier,\\
Farther \& farther, all the birds\\*
\vin Of Oxfordshire \& Gloucestershire.
\end{verse}

\subsection{}

\blfootnote{Oscar Wilde (1854 -- 1900), \cite{odq}. This a line from \refbook{The Importance of Being Earnest}.}If the lower orders don't set us a good example, what on earth is the use of them?

\section{}

\subsection{}

\blfootnote{`The White Cockade', Anonymous, \cite{rusby_underneath}. This love song is not to be confused with the stirring Jacobite marching song of the same name. \P 3. A `flowing bowl' seems to have been some sort of party involving a bowl of punch.}\begin{center}
\textit{Tune: The White Cockade}
\end{center}

\settowidth{\versewidth}{I wish the man that's listed him might prosper night nor day,}
\begin{verse}[\versewidth]
`One day as I was walking all o'er yon fields of moss,\\*
I had no thoughts of enlisting till some soldiers did me cross.\\
They kindly did invite me to a flowing bowl, and down\\*
They advanc{\`{e}}d me some money, a shilling from the crown.'\\!

My true love he is handsome and he wears a white cockade.\\*
He is a handsome young man, likewise a roving blade.\\
He is a handsome young man. He's gone to serve the king.\\*
O my very heart is aching all for the love of him.\\!

My true love he is handsome \& comely for to see,\\*
And by a sad misfortune a soldier now is he.\\
I wish the man that's listed him might prosper night nor day,\\*
And I wish that the hollanders might sink him in the sea.\\!

Then he took out his hankerchief to wipe my flowing eye.\\*
`Leave off your lamentations, likewise your mournful sighs.\\
Leave off your grief \& sorrow until I march o'er yon plain.\\*
We'll be married in the springtime when I return again.'\\!

My true love he is listed, and it's all for him I'll rove.\\*
I'll write his name on every tree that grows in yonder grove.\\
My poor heart it does hallow; how my poor heart it does cry,\\*
To remind me of my ploughboy, until the day I die.
\end{verse}

\subsection{}

\blfootnote{`A Country Dance', The Rev Charles Turner (1808 -- 1879), \cite{obev}.}\settowidth{\versewidth}{With jealous heed her lessening voice he hears}
\begin{verse}[\versewidth]
He has not wooed, but he has lost his heart.\\*
That country dance is a sore test for him;\\
He thinks her cold; his hopes are faint \& dim;\\
But though with seeming mirth she takes her part\\
In all the dances \& the laughter there,\\
And though to many a youth, on brief demand,\\
She gives a kind assent \& courteous hand,\\
She loves but him, for him is all her care.\\
With jealous heed her lessening voice he hears\\
Down that long vista, where she seems to move\\
Among fond faces \& relays of love,\\
And sweet occasion, full of tender fears:\\
Down those long lines he watches from above,\\*
Till with the refluent dance she reappears.s
\end{verse}

\subsection{}

\blfootnote{Oscar Wilde (1854 -- 1900), \cite{happyprince}. `It', in the context of the story, is a red rose.}It is so beautiful that I am sure it has a long Latin name.

\section{}

\subsection{}

\blfootnote{$\mathbb{R}$ `The Lancashire Witches', Anonymous, \cite{bod17687}. There is an association between witchcraft and the County Palatine of Lancashire due to the infamous Pendle witch trials. The Almanacker is only familiar with the tune sung by the Oldham Tinkers in \refbook{For Old Time's Sake}.}\begin{center}
\textit{Tune: The Lancashire Witches}
\end{center}

\settowidth{\versewidth}{    With the pride of the Lancashire witches.}
\begin{verse}[\versewidth]
In vain I attempt to describe\\*
\vin The charms of my favourite fair;\\
She's the sweetest of mother \textit{Eve}'s tribe;\\
\vin With her there is none to compare.\\
She's a pride of beauty so bright,\\
\vin Her image my fancy enriches;\\
My charmer's the village delight,\\*
\vin And the pride of the Lancashire witches.\\!

{\itshape
Then hurrah for the Lancashire witches,\\
Whose smile every bosom enriches;\\
\vin O dearly I prize\\
\vin The pretty blue eyes\\*
Of the pride of the Lancashire witches.}\\!

They may talk of the dark eyes of Spain:\\*
\vin 'Tis useless to talk as they do:\\
They attempt to compare them in vain\\
\vin With the Lancashire ladies' of blue.\\
Only view the dear heavenly belles,\\
\vin You're soon seized with love's sudden twitches,\\
Which nought could create but the spells\\*
\vin From the eyes of the Lancashire witches.\\!

The Lancashire witches, believe me,\\*
\vin Are beautiful every one;\\
But mine, or my fancy deceives me,\\
\vin Is the prettiest under the sun.\\
If the wealth of the Indies, I swear,\\
\vin Were mine, and I wallowed in riches,\\
How gladly my fortune I'd share\\*
\vin With the pride of the Lancashire witches.\\!

{\itshape
Then hurrah for the Lancashire witches,\\
Whose smile every bosom enriches;\\
\vin O dearly I prize\\
\vin The pretty blue eyes\\*
Of the pride of the Lancashire witches.}
\end{verse}

\subsection{}

\blfootnote{`Letty's Globe', The Rev Charles Turner (1808 -- 1879), \cite{obev}.}\settowidth{\versewidth}{When Letty had scarce passed her third glad year,}
\begin{verse}[\versewidth]
When \textit{Letty} had scarce passed her third glad year,\\*
And her young artless words began to flow,\\
One day we gave the child a colored sphere\\
Of the wide earth, that she might mark \& know,\\
By tint \& outline, all its sea \& land.\\
She patted all the world; old empires peeped\\
Between her baby fingers; her soft hand\\
Was welcome at all frontiers. How she leaped,\\
And laughed and prattled in her world-wide bliss!\\
But when we turned her sweet unlearn{\`{e}}d eye\\
On our own isle, she raised a joyous cry:\\
O yes! I see it, \textit{Letty}'s home is there!\\
And while she hid all England with a kiss,\\*
Bright over Europe fell her golden hair.
\end{verse}

\subsection{}

\blfootnote{Oscar Wilde (1854 -- 1900), \cite{vera}.}Life is much too important a thing ever to talk seriously about it.

\section{}

\subsection{}

\blfootnote{$\mathbb{R}$ Anonymous, \cite{rusby_hourglass}. Sir Eglamore is an ill-defined knight floating around late medieval literature: see the Middle English long poem \refbook{Sir Eglamour of Artois}.}\begin{center}
\textit{Tune: Sir Eglamore}
\end{center}

\settowidth{\versewidth}{Which vexed the knight and made her grin,}
\begin{verse}[\versewidth]
Sir \textit{Eglamore} was a valiant knight --\\*
\vin \textit{Fa la lanky down dilly!}\\
He took up his sword and he went to fight --\\
\vin \textit{Fa la lanky down dilly!}\\
As he rode o'er hill \& dale,\\
All armoured in a coat of mail.\\
\vin \textit{Fa la la! Fa la la!}\\*
\vin \textit{Fa la lanky down dilly!}\\!

Out came a dragon from her den,\\*
That killed God knows how many men.\\
When she saw Sir \textit{Eglamore},\\*
You should have heard that dragon roar.\\!

Well then the trees began to shake.\\*
Horse did tremble and man did quake.\\
The birds betook them all to peep.\\*
it would have made a grown man weep.\\!

But all in vain it was to fear,\\*
For now they fall to fight like bears.\\
To it they go and soundly fight,\\*
The livelong day from morn till night.\\!

This dragon had a plaguey hide,\\*
That could the sharpest steel abide.\\
No sword could enter through her skin,\\*
Which vexed the knight and made her grin,\\!

But as in choler she did burn;\\*
He fetched the dragon a great good turn.\\
As a-yawning she did fall,\\*
He thrust his sword up, hilt \& all.\\!

Like a coward she did fly --\\*
\vin \textit{Fa la lanky down dilly!}\\
To her den which was hard by.\\
\vin \textit{Fa la lanky down dilly!}\\
There she lay all night and roared.\\
The knight was sorry for his sword.\\
\vin \textit{Fa la la! Fa la la!}\\*
\vin \textit{Fa la lanky down dilly!}
\end{verse}

\subsection{}

\blfootnote{`Seduced Girl', Louis Untermeyer (1885 -- 1977), \cite{newlove}. This is a translation of a poem attributed to Hedylus, a Greek poet of the Alexandrian school. The original Greek text can be found in the \refbook{Greek Anthology} V.199.}\settowidth{\versewidth}{    He lulled me into bed and closed my eyes,}
\begin{verse}[\versewidth]
With wine \& words of love \& every vow\\*
\vin He lulled me into bed and closed my eyes,\\
A sleepy, stupid innocent... So now\\
\vin I dedicate the spoils of my surprise:\\
The silk that bound my breasts, my virgin zone,\\
\vin The cherished purity I could not keep.\\
Goddess, remember we were all alone,\\*
\vin And he was strong, and I was \sfrac{$1$}{$2$} asleep.
\end{verse}

\subsection{}

\blfootnote{Oscar Wilde (1854 -- 1900), \cite{epigramswilde}.}Prayer must never be answered: if it is, it ceases to be prayer and becomes correspondence.

\section{}

\subsection{}

\blfootnote{$\mathbb{R}$ `An Elegy on the Glory of Her Sex, Mrs Mary Blaize', Oliver Goldsmith (1728 -- 1774), \cite{goldsmith}. \P 3. The Almanacker has amended this line to follow the version sung by Mrs McKusker, as a cherished example of the so-called folk process. \P 21. The Almanacker has amended this verse to follow the version sung by Mrs McKusker, as a cherished example of the so-called folk process.}\begin{center}
\textit{Tune: Mary Blaize}
\end{center}

\settowidth{\versewidth}{The doctors found, when she was dead,}
\begin{verse}[\versewidth]
Good people all, with one accord,\\*
\vin Lament for Madam \textit{Blaize},\\
Who never wanted one good word --\\*
\vin From those who spoke her praise.\\!

The needy seldom passed her door,\\*
\vin And always found her kind;\\
She freely lent to all the poor --\\*
\vin Who left a pledge behind.\\!

She strove the neighbourhood to please\\*
\vin With manners wondrous winnings\\
And never followed wicked ways --\\*
\vin Unless when she was sinning.\\!

At church, in silks \& satins new,\\*
\vin With hoop of monstrous size,\\
She never slumbered in her pew --\\*
\vin But when she shut her eyes.\\!

Her love was sought, I do aver,\\*
\vin By twenty beaux \& more;\\
The king himself has followed her --\\*
\vin When she has walked before.\\!

But wealth \& finery all fled\\*
\vin And hangers-on all gone,\\
The doctors found, when she was dead,\\*
\vin The life within her none.\\!

Let us lament in sorrow sore,\\*
\vin For \textsc{Kent Street} well may say\\
That had she lived a twelvemonth more\\*
\vin She had not died today.
\end{verse}

\subsection{}

\blfootnote{`On a Girdle', Edmund Waller (1606 -- 1687), \cite{treasury}.}\settowidth{\versewidth}{That which her slender waist confined}
\begin{verse}[\versewidth]
That which her slender waist confined\\*
Shall now my joyful temples bind;\\
No monarch but would give his crown\\*
His arms might do what this has done.\\!

It was my heaven's extremest sphere,\\*
The pale which held that lovely deer\\
My joy, my grief, my hope, my love\\*
Did all within this circle move.\\!

A narrow compass! and yet there\\*
Dwelt all that's good, \& all that's fair:\\
Give me but what this ribband bound,\\*
Take all the rest the sun goes round.
\end{verse}

\subsection{}

\blfootnote{Dr William Wordsworth, Poet Laureate (1770 -- 1850), \cite{norton}. This is a line from Dr Wordsworth's \refpoem{The Tables Turned}.}Let nature be your teacher.

\chapter{Quintilis}

\section{}

\subsection{}

\blfootnote{Anonymous, \cite{dublinersa}.}\begin{center}
\textit{Tune: Whisky in the Jar}
\end{center}

\settowidth{\versewidth}{    That she never would deceive me.}
\begin{verse}[\versewidth]
As I was going over\\*
\vin The far-famed Kerry mountains,\\
I met with Captain \textit{Farrell}\\
\vin And his money he was counting.\\
I first produced my pistol\\
\vin And then produced my rapier\\
Said, Stand and deliver\\*
\vin For you are a bold deceiver!\\!

{\itshape
Musha-ring dumma-do dumma-da!\\
Wack-fol the daddy-oh!\\
Wack-fol the daddy-oh!\\*
There's whisky in the jar!}\\!

I counted out his money\\*
\vin And it made a pretty penny.\\
I put it in my pocket\\
\vin And brought it home to \textit{Jenny}.\\
She sighed \& she swore\\
\vin That she never would deceive me.\\
But the devil take the women\\*
\vin For they never can be easy.\\!

I went up to my chamber\\*
\vin All for to take a slumber.\\
I dreamt of gold \& jewels,\\
\vin And for sure it was no wonder.\\
But \textit{Jenny} drew my charges\\
\vin And she filled them up with water,\\
And sent for Captain \textit{Farrell}\\*
\vin To ready for the slaughter.\\!

'Twas early in the morning\\*
\vin Just before I rose to travel.\\
Up comes a band of footmen\\
\vin And likewise Captain \textit{Farrell}.\\
I first produced my pistol\\
\vin For she'd stolen away my rapier.\\
I couldn't shoot the water,\\*
\vin So a prisoner I was taken.\\!

If anyone can aid me,\\*
\vin 'Tis my brother in the army.\\
If I could find his station\\
\vin In Cork or in Killarney.\\
And if he'll come with me\\
\vin We'll go roamin' in Kilkenny.\\
And I'm sure he'll treat me better\\*
\vin Than my own misportin' Jenny.\\!

{\itshape
Musha-ring dumma-do dumma-da!\\
Wack-fol the daddy-oh!\\
Wack-fol the daddy-oh!\\*
There's whisky in the jar!}
\end{verse}

\subsection{}

\blfootnote{Anonymous, \cite{sss}. \refbook{SSS} is only the primary source: there are very many versions of this shanty: the Almanacker has done his best to extract the best of each of them.}\begin{center}
\textit{Tune: Outward Bound}
\end{center}

\settowidth{\versewidth}{Saying, `Drink, my boys. It's worth your while.'}
\begin{verse}[\versewidth]
To the \textsc{Liverpool} docks we bid adieu,\\*
To \textit{Suke} \& \textit{Sall} \& \textit{Kitty} too.\\
Our anchor's wieghed and our sails unfurled;\\*
We're off to plough the watery world.\\!

{\itshape
Hurrah! We're outward bound!\\*
Hurrah! We're outward bound!}\\!

When the wind it blows from the east-nor'-east,\\*
Our ship will sail 10\textsc{kts} at least.\\
The purser will our wants supply,\\*
So while we've rum we'll never say die.\\!

And should we touch at \textsc{Malabar}\\*
Or any other port as far,\\
Our purser he will tip the chink\\*
And just like fishes we will drink.\\!

One day the man on the look-out\\*
Proclaims a sail with a joyful shout:\\
`Can you make her out?' `I think I can.\\*
She's a pilot standing out from the land.'\\!

{\itshape
Hurrah! We're homeward bound!\\*
Hurrah! We're homeward bound!}\\!

Now when we get to the \textsc{Blackwall} docks,\\*
The pretty young girls come down in flocks;\\
One to the other you'll hear them say,\\*
`O here comes \textit{Jack} with his 10 months' pay.\\!

And when we get to the \textsc{Dog and Bell},\\*
It's there they've got good liquor to sell.\\
In comes old \textit{Grouse} with a smile,\\*
Saying, `Drink, my boys. It's worth your while.'\\!

But when the money's all gone \& spent,\\*
And there's none to be borrowed and none to be lent,\\
In comes old \textit{Grouse} with a frown,\\*
Saying, `Get up, \textit{Jack}. Let \textit{John} sit down.'\\!

Then poor old \textit{Jack} must understand\\*
There's ships in docks all wanting hands;\\
So he goes onboard as he did before,\\*
And bids adieu to his native shore.\\!

{\itshape
Hurrah! We're outward bound!\\*
Hurrah! We're outward bound!}
\end{verse}

\subsection{}

\blfootnote{Proverbs 9.17, \cite{kjv}.}Stolen waters are sweet.

\section{}

\subsection{}

\blfootnote{Anonymous, \cite{dublinersb}.}\begin{center}
\textit{Tune: The Auld Orange Flute}
\end{center}

\settowidth{\versewidth}{There was many the ruction that myself had a han' in.}
\begin{verse}[\versewidth]
In the County Tyrone, near the town of \textsc{Dungannon},\\*
There was many the ruction that myself had a han' in.\\
\textit{Bob Williamson} lived there, a weaver by trade,\\
And all of us thought him a stout orange blade.\\
On the 12th of july as it yearly did come,\\
\textit{Bob} played on his old flute to the sound of the drum.\\
You can talk of your harp, your piano or lute,\\*
But nothing compared with the old orange flute.\\!

{\itshape
Toora-loo! Toora-loo!\\*
Sure it's six miles from \textsc{Bangor} to \textsc{Donaghadee}!}\\!

Ah but \textit{Bob} the deceiver, sure he took us all in,\\*
And he married a papist called \textit{Bridget McGinn},\\
Turned papist himself, and forsook the old cause\\
That gave us our freedom, religion \& laws.\\
Now the boys of the place made some comment upon it,\\
And \textit{Bob} had to flee to the province of Connaught.\\
Well he fled with his wife \& his fixings to boot;\\*
And, along with the latter, his old orange flute.\\!

At the chapel on sundays, to atone for past deeds,\\*
He said paters \& aves and he counted his beads;\\
Till after some time, at the priest's own desire,\\
\textit{Bob} went with his old flute to play in the choir.\\
Well he went with his old flute to play in the mass,\\
But the instrument shivered \& sighed -- O alas --\\
And blow as he would, though it made a great noise,\\*
The flute would play only ``The Protestant Boys''.\\!

At the council of priests that was held the next day,\\*
They decided to banish the old flute away.\\
They couldn't knock heresy out of its head,\\
So they bought \textit{Bob} a new one to play in its stead.\\
Now the old flute it was doomed and its fate was pathetic.\\
'Twas fastened and burned at the stake as heretic.\\
As the flames roared around sure they heard a queer noise;\\*
'Twas the old flute still playing ``The Protestant Boys''.\\!

{\itshape
Toora-loo! Toora-loo!\\*
Sure it's six miles from \textsc{Bangor} to \textsc{Donaghadee}!}
\end{verse}

\subsection{}

\blfootnote{Anonymous, \cite{sailors}.}\begin{center}
\textit{Tune: Haul on the Bowline}
\end{center}

\settowidth{\versewidth}{Haul on the bowline; the bully ship's a-rollin'.}
\begin{verse}[\versewidth]
Haul on the bowline; the bully ship's a-rollin'.\\!

{\itshape
Haul on the bowline! The bowline haul!}\\!

Haul on the bowline; \textit{Kitty} is me darlin'.\\!

Haul on the bowline; \textit{Kitty} comes from \textsc{Liverpool}.\\!

Haul on the bowline; it's a far cry to pay day.\\!

{\itshape
Haul on the bowline! The bowline haul!}
\end{verse}

\subsection{}

\blfootnote{Ambrose Bierce (1842 -- 1914), \cite{odq}.}Alliance, n: In international politics, the union of two thieves who have their hands so deeply inserted in each other's pocket that they cannot separately plunder a third.

\section{}

\subsection{}

\blfootnote{Anonymous, \cite{dublinersa}.}\begin{center}
\textit{Tune: Home, Boys, Home}
\end{center}

\settowidth{\versewidth}{Till she wished the short night had been seven years long.'}
\begin{verse}[\versewidth]
O well who wouldn't be a sailor lad, sailing on the main?\\*
To gain the good will of his captain's good name\\
He came ashore one evening for to see,\\*
And that was the beginning of my own true love \& me.\\!

{\itshape
And it's home, boys, home!\\
Home I'd like to be,\\
Home for a while in my own count-ry,\\
Where the oak \& the ash \& the bonny rowan tree\\*
Are all a-growing green in the north count-ry!}\\!

`Well I asked her for a candle for to light my way to bed,\\*
Likewise for a handkerchief to tie around my head.\\
She tended to my needs like a young maid ought to do,\\*
So then I says to her, ``Now won't you jump in with me too?''\\!

`Well she jumped into bed a-making no alarm,\\*
Thinking a young sailor lad could do to her no harm.\\
I hugged her, I kissed her the whole night long,\\*
Till she wished the short night had been seven years long.'\\!

Well early next morning the sailor lad arose\\*
And into \textit{Mary}'s apron threw a handful of gold,\\
Saying, `Take this, my dear, for the damage that I've done,\\*
For tonight I fear I've left you with a daughter or a son.\\!

`And if it be a girl-child, send her out to nurse\\*
With gold in her pocket \& with silver in her purse;\\
If it be a boy-child, he'll wear the jacket blue\\*
And go climbing up the rigging like his daddy used to do.'\\!

And so come all of you fair maidens, a warning take by me;\\*
Never let a sailor lad an inch above your knee.\\
I trusted one, and he beguiled me;\\*
He left me with a pair of twins to dangle on my knee.\\!

{\itshape
And it's home, boys, home!\\
Home I'd like to be,\\
Home for a while in my own count-ry,\\
Where the oak \& the ash \& the bonny rowan tree\\*
Are all a-growing green in the north count-ry!}
\end{verse}

\subsection{}

\blfootnote{Anonymous, \cite{sailors}. Paddy Doyle seems to have been a Liverpudlian boarding master.}\begin{center}
\textit{Tune: Paddy Doyle's Boots}
\end{center}

\settowidth{\versewidth}{We'll all shave under the chin.}
\begin{verse}[\versewidth]
{\itshape
To me, way-ay ay-ay-ay yah!}\\!

We'll pay \textit{Paddy Doyle} for 'is boots.\\!

{\itshape
To me, way-ay ay-ay-ay yah!}\\!

We'll all drink whisky \& gin.\\!

{\itshape
To me, way-ay ay-ay-ay yah!}\\!

We'll all shave under the chin.
\end{verse}

\subsection{}

\blfootnote{Prof Morris Morgan (1859 -- 1910), \cite{vitruvius}. Prof Morgan is here translating a passage by the ancient Roman architect Vitruvius, who in turn was relating a famous (and seemingly apocryphal) anecdote about Archimedes.}Eureka!

\section{}

\subsection{}

\blfootnote{Anonymous, \cite{dublinersb}.}\begin{center}
\textit{Tune: Leaving of Liverpool}
\end{center}

\settowidth{\versewidth}{    And they say that she's a floating hell.}
\begin{verse}[\versewidth]
Farewell, the \textsc{Princes Landing Stage};\\*
\vin \textsc{River Mersey}, fare thee well.\\
I am bound for California,\\*
\vin A place I know right well.\\!

{\itshape
So fare thee well, my own true love.\\
\vin When I return united we will be.\\
It's not the leaving of \textsc{Liverpool} that grieves me,\\*
\vin But my darling when I think of thee.}\\!

I've shipped on a yankee clipper ship;\\*
\vin {\hoskeroe Davy Crockett} is her name.\\
\textit{Dan Burgess} is the captain of her,\\*
\vin And they say that she's a floating hell.\\!

I have sailed with \textit{Burgess} once before;\\*
\vin I think I know him well.\\
If a man's a sailor he will get along;\\*
\vin If not then he's sure in hell.\\!

Farewell to \textsc{Lower Frederick Street},\\*
\vin \textsc{Anson Terrace} and \textsc{Park Lane}.\\
I am bound away for to leave you,\\*
\vin And I'll never see you again.\\!

I am bound for California\\*
\vin By way of stormy \textsc{Cape Horn},\\
And I will write to thee a letter, love,\\*
\vin When I am homeward bound.\\!

{\itshape
So fare thee well, my own true love.\\
\vin When I return united we will be.\\
It's not the leaving of \textsc{Liverpool} that grieves me,\\*
\vin But my darling when I think of thee.}
\end{verse}

\subsection{}

\blfootnote{Anonymous, \cite{ac4}.}\begin{center}
\textit{Tune: Way Me Susiana}
\end{center}

\settowidth{\versewidth}{If ye growl too hard your head they'll bust.}
\begin{verse}[\versewidth]
We'll heave him up and away we'll go!\\*
\vin \textit{Way, my \textit{Susiana}!}\\
That is where the cocks do crow --\\*
\vin \textit{We're all bound over the mountain!}\\!

And when the cocks begin to crow,\\*
'Tis time for me to roll \& go.\\!

And if we drown while we are young,\\*
It's better to drown than to wait to be hung.\\!

O growl ye may but go ye must.\\*
If ye growl too hard your head they'll bust.\\!

Up socks, you cocks; hand her two blocks,\\*
And go below to your old ditty box.\\!

Oh rock \& shake her one more drag.\\*
\vin \textit{Way, my \textit{Susiana}!}\\
O bend your duds and pack your bag.\\*
\vin \textit{We're all bound over the mountain!}
\end{verse}

\subsection{}

\blfootnote{Thomas Appleton (1812 -- 1884), \cite{odq}.}Good Americans, when they die, go to Paris.

\section{}

\subsection{}

\blfootnote{Prof John Wilson (1785 -- 1854), \cite{corriesa}. `Horo Mhairi dhu' would seem to be a (slightly archaic) Gallic phrase, meaning `O black Mary'.}\begin{center}
\textit{Tune: Turn Ye To Me}
\end{center}

\settowidth{\versewidth}{Hushed be thy moaning, lone bird of the sea;}
\begin{verse}[\versewidth]
The stars are burning\\*
\vin \vin Cheerily, cheerily.\\
\vin {\hoskeroe Horo} \textit{Mhairi} {\hoskeroe dhu}, turn ye to me.\\
The sea mew is moaning\\
\vin \vin Drearily, drearily.\\
\vin {\hoskeroe Horo} \textit{Mhairi} {\hoskeroe dhu}, turn ye to me.\\
Cold is the stormwind that ruffles his breast,\\
But warm are the downy plumes lining his nest.\\
\vin \vin Cold blows the storm there;\\
\vin \vin Soft falls the snow there.\\*
\vin {\hoskeroe Horo} \textit{Mhairi} {\hoskeroe dhu}, turn ye to me.\\!

The waves are dancing\\*
\vin \vin Merrily, merrily.\\
\vin {\hoskeroe Horo} \textit{Mhairi} {\hoskeroe dhu}, turn ye to me.\\
The seabirds are wailing\\
\vin \vin Wearily, wearily.\\
\vin {\hoskeroe Horo} \textit{Mhairi} {\hoskeroe dhu}, turn ye to me.\\
Hushed be thy moaning, lone bird of the sea;\\
Thy home on the rocks is a shelter to thee;\\
\vin \vin Thy home is the angry wave,\\
\vin \vin Mine but the lonely grave.\\*
\vin {\hoskeroe Horo} \textit{Mhairi} {\hoskeroe dhu}, turn ye to me.
\end{verse}

\subsection{}

\blfootnote{Anonymous, \cite{youngtradition}.}\begin{center}
\textit{Tune: Hanging Johnny}
\end{center}

\settowidth{\versewidth}{They call me Hanging Johnny --}
\begin{verse}[\versewidth]
They call me \textit{Hanging Johnny} --\\*
\vin \textit{Away, boys! Away!}\\
But I never hanged nobody --\\*
\vin \textit{So hang, boys! Hang!}\\!

They says I hanged my graddy,\\*
And then I hanged my family.\\!

They says I hanged my mother.\\*
It is they and my brother.\\!

I hanged a rotten liar,\\*
But I hanged a bloody friar.\\!

They tells I hang for money,\\*
But hanging's so bloody funny.\\!

We all will hang together --\\*
\vin \textit{Away, boys! Away!}\\
It's all for better weather --\\*
\vin \textit{So hang, boys! Hang!}
\end{verse}

\subsection{}

\blfootnote{William Beaverbrook, 1st Baron Beaverbrook (1879 -- 1964), \cite{odq}.}He did not seem to care which way he travelled providing he was in the driver's seat.

\section{}

\subsection{}

\blfootnote{$\mathbb{R}$ Anonymous, \cite{child}. A handful of lines have been removed to make this song singable to the tune with which the Almanacker is familiar.}\begin{center}
\textit{Tune: The Mother's Malison}
\end{center}

\settowidth{\versewidth}{    Your streams seem wondrous strang;}
\begin{verse}[\versewidth]
\textit{Willie} stands in his stable door\\*
\vin And clapping at his steed,\\
And looking o'er his white fingers\\
\vin His nose began to bleed.\\
`Gi'e corn to my horse, mother,\\
\vin And meat to my young man,\\
And I'll awa' to \textit{Maggie}'s bower;\\*
\vin I'll win ere she lie down.'\\!

`O 'bide this night wi' me, Willie,\\*
\vin O 'bide this night wi' me;\\
The best an cock o' a' the roost\\
\vin At your supper shall be.'\\
‘A' your cocks, and a' your roosts,\\
\vin I value not a prin,\\
For I'll awa' to \textit{Maggie}'s bower;\\*
\vin I'll win ere she lie down.'\\!

`Stay this night wi' me, \textit{Willie},\\*
\vin O stay this night wi' me;\\
The best an sheep in a' the flock\\
\vin At your supper shall be.'\\
`A' your sheep, and a' your flocks,\\
\vin I value not a prin,\\
For I'll awa' to \textit{Maggie}'s bower;\\*
\vin I'll win ere she lie down.'\\!

`O on ye gang to \textit{Maggie}'s bower,\\*
\vin So sore against my will,\\
The deepest pot in \textsc{Clyde}'s water,\\
\vin My malison ye's feel.'\\
`The good steed that I ride upon\\
\vin Cost me thrice {\pounds}30;\\
And I'll put trust in his swift feet\\*
\vin To ha'e me safe to land.'\\!

As he rode o'er yon high, high hill,\\*
\vin And down yon dowie den,\\
The noise that was in \textsc{Clyde}'s water\\
\vin Would feared 500 men.\\
`O roaring \textsc{Clyde}, ye roar o'er loud,\\
\vin Your streams seem wondrous strang;\\
Make me your wreck as I come back,\\*
\vin But spare me as I gang!'\\!

Then he is on to \textit{Maggie}'s bower,\\*
\vin And tirl\`{e}d at the pin.\\
`O sleep ye, wake ye, \textit{Maggie},' he said;\\
\vin `Ye'll open, let me come in.'\\
`O who is this at my bower door,\\
\vin That calls me by my name?'\\
`It is your first love, sweet \textit{Willie},\\*
\vin This night newly come hame.'\\!

`I ha'e few lovers thereout, thereout,\\*
\vin As few ha'e I therein;\\
The best an love that ever I had\\
\vin Was here jus' late yestreen.'\\
`The worst an bower in a' your bowers,\\
\vin For me to lie therin!\\
My boots are fu' o' \textsc{Clyde}'s water;\\*
\vin I'm shivering at the chin.'\\!

`My barns are fu' o' corn, \textit{Willie};\\*
\vin My stables are fu' o' hay.\\
My bowers are fu' o' gentlemen;\\
\vin They'll not remove till day.'\\
`O fare ye well, my false \textit{Maggie}!\\
\vin O farewell, and adieu!\\
I've got my mother's malison\\*
\vin This night coming to you.’\\!

As he rode o'er yon high, high hill\\*
\vin And down yon dowie den,\\
The rushing that was in \textsc{Clyde}'s water\\
\vin Took \textit{Willie}'s hat from him.\\
He leaned him o'er his saddle-bow,\\
\vin To catch his hat through force;\\
The rushing that was in \textsc{Clyde}'s water\\*
\vin Took \textit{Willie} from his horse.\\!

His brither stood upo' the bank,\\*
\vin Says, `Fye, man, will ye drown?\\
Ye’ll turn ye to your high horse head\\
\vin And learn how to sowm.'\\
`How can I turn to my horse head\\
\vin And learn how to sowm?\\
I've got my mother's malison,\\*
\vin It’s here that I must drown.'\\!

The very hour this young man sank\\*
\vin Into the pot so deep,\\
Up it waked his love \textit{Maggie}\\
\vin Out o' her drowsy sleep.\\
`Come here, come here, my mother dear,\\
\vin And read this dreary dream;\\
I dreamed my love was at our gates,\\*
\vin And none would let him in.'\\!

`Lie still, lie still now, my \textit{Maggie},\\*
\vin Lie still \& tak' your rest;\\
Sin' your truelove was at your gates,\\
\vin It's but two \sfrac{$1$}{$4$}s past.'\\
Nimbly, nimbly raise she up,\\
\vin And nimbly pat she on,\\
And the higher that the lady cried,\\*
\vin The louder blew the win'.\\!

The first an step that she stepped in,\\*
\vin She stepped to the queet;\\
`Ohon! Alas!’ said that lady,\\
\vin `This water's wondrous deep.'\\
The next an step that she wade in,\\
\vin She waded to the knee;\\
Says she, `I coud wade farther in,\\*
\vin If I my love coud see.'\\!

The next an step that she wade in,\\*
\vin She waded to the chin;\\
The deepest pot in \textsc{Clyde}'s water\\
\vin She got sweet \textit{Willie} in.\\
`You've had a cruel mother, \textit{Willie},\\
\vin And I have had another;\\
But we shall sleep in \textsc{Clyde}'s water\\*
\vin Like sister an' like brother.'
\end{verse}

\subsection{}

\blfootnote{Anonymous, \cite{ac4}. This shanty seems to have roots that are quite ancient, going back as far as the sixteenth century, but its most significant source of inspiration was surely the Barbary Wars of the early nineteenth century.}\begin{center}
\textit{Tune: Coast of High Barbaree}
\end{center}

\settowidth{\versewidth}{For broadside, for broadside they fought all on the main;}
\begin{verse}[\versewidth]
Look ahead; look astern; look the weather in the lee.\\*
\vin \textit{Blow high! Blow low! And so sail\`{e}d we!}\\
I see a wreck to windward and a lofty ship to lee --\\*
\vin \textit{A-sailing down all on the coasts of high Barbary!}\\!

O are you a pirate or a man-o'-war? cried we.\\*
O no! I'm not a pirate but a man-o'-war, cried he.\\!

We'll back up our topsails and heave our vessel to;\\*
For we have got some letters to be carried home by you.\\!

For broadside, for broadside they fought all on the main;\\*
Until at last the frigate shot the pirate's mast away.\\!

With cutlass \& gun, O we fought for hours three;\\*
\vin \textit{Blow high! Blow low! And so sail\`{e}d we!}\\
The ship it was their coffin and their grave it was the sea.\\*
\vin \textit{A-sailing down all on the coasts of high Barbary!}
\end{verse}

\subsection{}

\blfootnote{Anonymous, \cite{odq}.}You should make a point of trying every experience once, except incest and folk-dancing.

\section{}

\subsection{}

\blfootnote{$\mathbb{R}$ Robert Burns (1759 -- 1796), \cite{burns}. Although Palgrave includes this song, he prefers the first version; whereas the Almanacker prefers the second. Cunningham reports, `An Ayrshire legend says the heroine of this affecting song was Miss Kennedy, of Dalgarrock, a young creature, beautiful and accomplished, who fell a victim to her love for her kinsman, McDoual, of Logan.'}\begin{center}
\textit{Tune: Caledonian Hunt's Delight}
\end{center}

\settowidth{\versewidth}{Thou'lt break my heart, thou warbling bird,}
\begin{verse}[\versewidth]
Ye banks \& braes o' bonnie \textsc{Doon},\\*
\vin How can ye bloom sae fresh \& fair?\\
How can ye chant, ye little birds,\\
\vin And I sae weary fu' o' care?\\
Thou'lt break my heart, thou warbling bird,\\
\vin That wantons through the flowering thorn:\\
Thou 'minds me o' departed joys,\\*
\vin Departed -- never to return.\\!

Aft hae I roved by bonnie \textsc{Doon},\\*
\vin To see the rose \& woodbine twine:\\
And ilka bird sang o' its love,\\
\vin And fondly sae did I o' mine;\\
Wi' lightsome heart I pu'd a rose,\\
\vin Fu' sweet upon its thorny tree!\\
And my false lover sto' my rose,\\*
\vin But ah! he left the thorn wi' me.
\end{verse}

\subsection{}

\blfootnote{Anonymous, \cite{ac4}. One commentator writes of this shanty: `According to Hugill, this shanty probably originated as a Scottish fisherman's song. It was also popular with Gloucester fishermen in the American Northeast. Hugill also collected a version in Devonshire, and it was known in Canada... This was a capstan shanty, and sailors would take turns with verses, giving a new fish each time for as long as was necessary.'}\begin{center}
\textit{Tune: Fish of the Sea}
\end{center}

\settowidth{\versewidth}{Then up jumps the shark with his nine rows of teeth,}
\begin{verse}[\versewidth]
Come all you young sailor-men, listen to me;\\*
I'll sing you a song of the fish in the sea.\\!

{\itshape
And it's windy weather boys!\\
\vin Stormy weather, boys!\\
When the wind blows,\\
\vin We're all together, boys!\\
Blow ye winds westerly!\\
\vin Blow ye winds, blow!\\
Jolly sou'wester, boys!\\*
\vin Steady she goes!}\\!

Up jumps the eel with his slippery tail,\\*
Climbs up aloft and reefs the topsail.\\!

Then up jumps the shark with his nine rows of teeth,\\*
Saying, You eat the dough boys, and I'll eat the beef!\\!

Up jumps the whale, the largest of all.\\*
If you want any wind, well, I'll blow ye a squall!\\!

{\itshape
And it's windy weather boys!\\
\vin Stormy weather, boys!\\
When the wind blows,\\
\vin We're all together, boys!\\
Blow ye winds westerly!\\
\vin Blow ye winds, blow!\\
Jolly sou'wester, boys!\\*
\vin Steady she goes!}
\end{verse}

\subsection{}

\blfootnote{William Beckford (1759 -- 1844), \cite{odq}.}I am not over-fond of resisting temptation.

\section{}

\subsection{}

\blfootnote{$\mathbb{R}$ `MacPherson's Farewell', Robert Burns (1759 -- 1796), \cite{burns}. A version of this song is said to have been composed by James MacPherson, on the eve of his execution, who (rightly or wrongly) was hanged in the autumn of 1700 for banditry. The second verse is from another version; there are many, although the best known one comes from Burns.}\begin{center}
\textit{Tune: McPherson's Rant}
\end{center}

\settowidth{\versewidth}{I've dared his face, and in this place}
\begin{verse}[\versewidth]
Farewell, ye dungeons dark \& strang,\\*
\vin The wretch's destiny.\\
\textit{Macpherson}'s time will no' be lang\\*
\vin On yonder gallows-tree.\\!

It was by a woman's treacherous hand\\*
\vin That I was condemned to dee.\\
She stood above a window ledge\\*
\vin And a blanket threw over me.\\!

{\itshape
Sae rantingly, sae wantonly,\\
\vin Sae dauntingly gaed he.\\
He played a spring, and danced it round,\\*
\vin Below the gallows-tree.}\\!

O what is death but parting breath?\\*
\vin On many a bloody plain\\
I've dared his face, and in this place\\*
\vin I scorn him yet again.\\!

Untie these bands from off my hands,\\*
\vin And bring to me my sword;\\
And there's no' a man in all Scotland,\\*
\vin But I'll brave him at a word.\\!

I've lived a life of sturt \& strife;\\*
\vin I die by treachery.\\
It burns my heart I must depart\\*
\vin And not aveng\`{e}d be.\\!

Now farewell light, thou sunshine bright\\*
\vin And all beneath the sky.\\
May coward shame distain his name,\\*
The wretch that dares not die.\\!

{\itshape
Sae rantingly, sae wantonly,\\
\vin Sae dauntingly gaed he.\\
He played a spring, and danced it round,\\*
\vin Below the gallows-tree.}
\end{verse}

\subsection{}

\blfootnote{Anonymous, \cite{sailors}.}\begin{center}
\textit{Tune: A Hundred Years Ago}
\end{center}

\settowidth{\versewidth}{She promised me that little thing.}
\begin{verse}[\versewidth]
A 100 years on the eastern shore,\\*
\vin \textit{O! Yes! O!}\\
A 100 years on the eastern shore.\\*
\vin \textit{A 100 years ago!}\\!

When I sailed across the sea,\\*
My gal said she'd be true to me.\\!

I promised her a golden ring.\\*
She promised me that little thing.\\!

O pulley \textit{John} was the boy for me:\\*
A buck a-land, and a bully at sea.\\!

It's up aloft this yard must go,\\*
For Mr Mate has told me so.\\!

I thought I heard the skipper say,\\*
\vin \textit{O! Yes! O!}\\
Just one more pull, and then belay.\\*
\vin \textit{A 100 years ago!}
\end{verse}

\subsection{}

\blfootnote{The Rev Prof William Shedd (1820 -- 1894), \cite{confessions}. Rev Prof Shedd is here translating the opening of St Augustine's \refbook{Confessions}. The original translation gives `repose' instead of `rest', reflecting the fact that Augustine uses quite different words for `restless' (`inquietum') and `rest'/`repose' (`requiescat'); but `rest' is better.}Thou madest us for thyself, and our heart is restless, until it rest in thee.

\section{}

\subsection{}

\blfootnote{$\mathbb{R}$ Anonymous, \cite{dublinersb}. This nineteenth-century folk song was originally accompanied by a children's game. For largely political reasons, the Dubliners chose to change `Belfast' to `Dublin' in their rendition.}\begin{center}
\textit{Tune: I'll Tell Me Ma}
\end{center}

\settowidth{\versewidth}{They knock at the door and they ring at the bell,}
\begin{verse}[\versewidth]
I'll tell my ma when I get home;\\*
The boys won't leave the girls alone.\\
They pulled my hair; they stole my comb,\\*
But that's all right till I go home.\\!

She is handsome; she is pretty;\\*
She is the belle of \textsc{Belfast} city.\\
She is a-courting. One, two, three:\\*
Pray, won't you tell me, who is she?\\!

\textit{Albert Mooney} says he loves her.\\*
All the boys are fighting for her.\\
They knock at the door and they ring at the bell,\\*
Saying, O my true love, are you well?\\!

Out she comes, as white as snow,\\*
Rings on her fingers, bells on her toes.\\
Old \textit{Jenny Morrissey} says she'll die\\*
If she doesn't get the feller with the roving eye.\\!

Let the wind \& the rain \& the hail blow high.\\*
Let the snow come travelling through the sky.\\
She's as sweet as apple pie,\\*
And she'll get her own lad by \& by.\\!

When she gets a lad of her own,\\*
She won't tell her ma when she gets home.\\
Let them all come as they will;\\*
For it's \textit{Albert Mooney} she loves still.
\end{verse}

\subsection{}

\blfootnote{Anonymous, \cite{sailors}.}\begin{center}
\textit{Tune: Stormer Longjohn}
\end{center}

\settowidth{\versewidth}{They dug 'is grave with a silver spade.}
\begin{verse}[\versewidth]
\textit{Stormy}'s gone, that good ol' man.\\*
\vin \textit{Way! \textit{Stormer Longjohn}!}\\
\textit{Stormy}'s gone, that good ol' man.\\*
\vin \textit{Way-hey! Mr \textit{Storm-Along}!}\\!

They dug 'is grave with a silver spade.\\*
A shroud of finest silk was made.\\!

An able sailor, bold \& true,\\*
A good ol' boatswain to 'is crew.\\!

I wish I was ol' \textit{Stormy}'s son.\\*
I'd build a ship of a 1000 tonne.\\!

I'd fill 'er with New England rope.\\*
My shell-backs they would all 'ave some.\\!

Ol' \textit{Stormy}'s dead an' gone to rest.\\*
\vin \textit{Way! \textit{Stormer Longjohn}!}\\
Of all the sailors, 'e was best.\\*
\vin \textit{Way-hey! Mr \textit{Storm-Along}!}
\end{verse}

\subsection{}

\blfootnote{William Murray, 1st Earl of Mansfield (1705 -- 1793), \cite{justices}. Scholars have debated whether or not Lord Mansfield actually uttered these words; but, whatever the truth of the matter, it's a handsome summary of the legal principle which he confirmed in Somersett's famous case.}The air of England has long been too pure for a slave, and free is any man who breathes it.

\section{}

\subsection{}

\blfootnote{Anonymous, \cite{dublinersc}. James Joyce named one of his infamous emperor's-new-clothes novels after this song. A few Irish words and phrases ought to be explained. The word `cr\'eat\'ur' is pronounced like the English word `crater', and means liquor. The phrase `mo mhuirn\'in' is pronounced `mavourneen' as in the Irish folk song `Kathleen Mavourneen', and it means `my darling'; whereas `sail \'eille' (sometimes semi-anglicised as `shillelagh') is pronounced to rhyme with `ukulele', and refers to a kind of blunt weapon typically made from blackthorn wood. `D'anam don diabhal' is a curse, literally, `Your soul to the devil', and is pronounced something like `Denim done dowel'.}\begin{center}
\textit{Tune: Finnegan's Wake}
\end{center}

\settowidth{\versewidth}{    And they carried him home his corpse to wake.}
\begin{verse}[\versewidth]
\textit{Tim Finnegan} lived in \textsc{Walking Street},\\*
\vin A gentleman irish, mighty odd.\\
He had a brogue both rich \& sweet,\\
\vin And to rise in the world he carried a hod.\\
Now \textit{Tim} had a bit of a tippling way:\\
\vin With a love of the liquor poor \textit{Tim} was born,\\
And to help him on with his work each day,\\*
\vin He'd a drop of the {\hoskeroe creatur} every morn.\\!

{\itshape
Whack! Fol-the-da! Will you dance to your partner?\\
\vin Round the floor your trotters shake!\\
Wasn't it the truth I told you?\\*
\vin Lots of fun at \textit{Finnegan}'s wake!}\\!

One morning \textit{Tim} was feeling full:\\*
\vin His head was heavy, and it made him shake.\\
He fell off the ladder and broke his skull,\\
\vin And they carried him home his corpse to wake.\\
They rolled him up in a nice clean sheet,\\
\vin And they laid him out upon the bed,\\
With a bucket of whisky at his feet\\*
\vin And a barrel of porter at his head.\\!

\textit{Tim}'s friends assembled at the wake,\\*
\vin And the widow \textit{Finnegan} called for lunch:\\
First she brought in tea \& cake;\\
\vin Then pipes, tobacco and whisky punch.\\
\textit{Biddy O'Brien} began to cry,\\
\vin `Such a nice, clean corpse, did you ever see?\\
O \textit{Tim}, {\hoskeroe mo mhuirnin}, why did you die?'\\*
\vin `Ara, hold your gob!' said \textit{Paddy McGee}.\\!

Then \textit{Maggie O'Connor} took up the job:\\*
\vin `O \textit{Biddy},' says she, `you're wrong, I'm sure!'\\
\textit{Biddy} fetched her a belt in the gob,\\
\vin And she left her sprawling on the floor.\\
Then war did soon engage:\\
\vin 'Twas woman to woman and man to man;\\
{\hoskeroe Sail eille} law was all the rage,\\*
\vin And a row and a ruction soon began.\\!

Then \textit{Mickey Maloney} ducked his head\\*
\vin When a noggin of whisky flew at him;\\
It missed and landed on the bed,\\
\vin And the liquor scattered over \textit{Tim}!\\
By God, he revives! See how he rises!\\
\vin \textit{Tim Finnegan} rising from the bed,\\
Saying, `Whirl your whisky around like blazes!\\*
\vin {\hoskeroe D'anam don diabhal}! Do you think I'm dead?!'\\!

{\itshape
Whack! Fol-the-da! Will you dance to your partner?\\
\vin Round the floor your trotters shake!\\
Wasn't it the truth I told you?\\*
\vin Lots of fun at \textit{Finnegan}'s wake!}
\end{verse}

\subsection{}

\blfootnote{Anonymous, \cite{sailors}.}\begin{center}
\textit{Tune: Boston Harbour}
\end{center}

\settowidth{\versewidth}{For it's better weather here than it is on top.}
\begin{verse}[\versewidth]
From \textsc{Boston} harbour we set sail,\\*
When it was blowin' a devil of a gale,\\
With a ring-tail set all abaft the mizzen peak\\*
An' the Rule Britannia ploughin' up the deep.\\!

{\itshape
With a big boe-woe! Toe-roe-roe!\\*
Fol-dee-rol dee-rye doe-day!}\\!

Then up comes the skipper from down below.\\*
It's look aloft, lads; look a-low.\\
Then it's look a-low, and it's look aloft,\\*
And coil up your ropes, lads, fore \& aft.\\!

Then down to 'is cabin well he quickly crawls,\\*
An' to 'is poor old steward balls,\\
Go an' mix me a glass that'll make me cough,\\*
For it's better weather here than it is on top.\\!

Now there's one thing that we 'ave to crave:\\*
That the captain meets with a watery grave.\\
So we'll throw 'im down into some dark hole\\*
Where the sharks'll 'ave 'is body an' the devil 'ave 'is soul.\\!

{\itshape
With a big boe-woe! Toe-roe-roe!\\*
Fol-dee-rol dee-rye doe-day!}
\end{verse}

\subsection{}

\blfootnote{Francis Bacon, Viscount St Alban (1561 -- 1626), \cite{odq}.}Riches are for spending.

\section{}

\subsection{}

\blfootnote{Anonymous, \cite{dublinersb}. IRA = Irish Republican Army.}\begin{center}
\textit{Tune: Off to Dublin in the Green}
\end{center}

\settowidth{\versewidth}{Where the bayonets flash and the rifles crash}
\begin{verse}[\versewidth]
O I am a merry plough-boy,\\*
\vin And I plough the fields all day,\\
Till a sudden thought came to my head\\
\vin That I should a-roam away.\\
For I'm sick \& tired of slavery\\
\vin Since the day that I was born,\\
And I'm off to join the IRA\\*
\vin And I'm off tomorrow morn.\\!

{\itshape
And we're all off to \textsc{Dublin} in the green, in the green,\\
\vin Where the helmets glisten in the sun,\\
Where the bayonets flash and the rifles crash\\*
\vin To the rattle of the thompson gun.}\\!

I'll leave aside my pick \& spade;\\*
\vin I'll leave aside my plough.\\
I'll leave aside my horse \& yoke;\\
\vin I no longer need them now.\\
I'll leave aside my \textit{Mary} --\\
\vin She's the girl that I adore --\\
And I wonder if she'll think of me\\*
\vin When she hears the rifles roar.\\!

And when the war is over,\\*
\vin And dear old Ireland is free,\\
I'll take her to the church to wed\\
\vin And a rebel's wife she'll be.\\
Well, some men fight for silver,\\
\vin And some men fight for gold;\\
But the IRA are fighting for\\*
\vin The land that the saxons stole.\\!

{\itshape
And we're all off to \textsc{Dublin} in the green, in the green,\\
\vin Where the helmets glisten in the sun,\\
Where the bayonets flash and the rifles crash\\*
\vin To the rattle of the thompson gun.}
\end{verse}

\subsection{}

\blfootnote{Anonymous, \cite{fiddlers}. This shanty clearly shares a common ancestor with \refpoem{Sally Racket}.}\begin{center}
\textit{Tune: Ring Down Shanty}
\end{center}

\settowidth{\versewidth}{She ran off with a Quaker.}
\begin{verse}[\versewidth]
No beef in the market,\\*
\vin \textit{Ring down!}\\
No mutton in the market,\\
\vin \textit{Ring down!}\\
\vin \textit{To me way-hey hey-hey hey O!}\\*
\vin \textit{We're the boys to ring down!}\\!

Little \textit{Sally Racket},\\*
She shipped in a packet.\\!

Little \textit{Betty Baker},\\*
She ran off with a Quaker.\\!

Little \textit{Kitty Carson},\\*
She ran off with a parson.\\!

No beef in the market,\\*
\vin \textit{Ring down!}\\
No mutton in the market,\\
\vin \textit{Ring down!}\\
\vin \textit{To me way-hey hey-hey hey O!}\\*
\vin \textit{We're the boys to ring down!}
\end{verse}

\subsection{}

\blfootnote{The Rt Hon Joseph Addison (1672 -- 1719), \cite{odq}.}One Englishman could beat three Frenchmen.

\section{}

\subsection{}

\blfootnote{`Lachin y Gair', George Noel, 6th Baron Byron (1788 -- 1824), \cite{byron}. Byron notes: `Lachin y Gair, or, as it is pronounced in the Erse, Loch na Garr, towers proudly pre-eminent in the Northern Highlands, near Invercauld. One of our modern tourists mentions it as the highest mountain, perhaps, in Great Britain. Be this as it may, it is certainly one of the most sublime and picturesque amongst our ``Caledonian Alps''. Its appearance is of a dusky hue, but the summit is the seat of eternal snows. Near Lachin y Gair I spent some of the early part of my life, the recollection of which has given birth to these stanzas.'}\begin{center}
\textit{Tune: Dark Lochnagar}
\end{center}

\settowidth{\versewidth}{    And rides on the wind, o'er his own highland vale.}
\begin{verse}[\versewidth]
Away, ye gay landscapes, ye gardens of roses!\\*
\vin In you let the minions of luxury rove;\\
Restore me the rocks where the snow-flake reposes,\\
\vin Though still they are sacred to freedom \& love.\\
Yet, Caledonia, beloved are thy mountains,\\
\vin Round their white summits though elements war;\\
Though cataracts form 'stead of smooth-flowing fountains,\\*
\vin I sigh for the valley of dark \textsc{Loch na Garr}.\\!

Ah there my young footsteps in infancy wandered;\\*
\vin My cap was the bonnet; my cloak was the plaid.\\
On chieftains long perished my memory pondered,\\
\vin As daily I strode through the pine-covered glade.\\
I sought not my home till the day's dying glory\\
\vin Gave place to the rays of the bright polar star;\\
For fancy was cheered by traditional story,\\*
\vin Disclosed by the natives of dark \textsc{Loch na Garr}.\\!

Shades of the dead! have I not heard your voices\\*
\vin Rise on the night-rolling breath of the gale?\\
Surely the soul of the hero rejoices,\\
\vin And rides on the wind, o'er his own highland vale.\\
Round \textsc{Loch na Garr} while the stormy mist gathers,\\
\vin Winter presides in his cold icy car:\\
Clouds there encircle the forms of my fathers;\\*
\vin They dwell in the tempests of dark \textsc{Loch na Garr}.\\!

Ill-starred, though brave, did no visions foreboding\\*
\vin Tell you that fate had forsaken your cause?\\
Ah! were you destined to die at \textsc{Culloden},\\
\vin Victory crowned not your fall with applause:\\
Still were you happy in death's earthly slumber,\\
\vin You rest with your clan in the caves of \textsc{Braemar};\\
The pibroch resounds to the piper's loud number,\\*
\vin Your deeds on the echoes of dark \textsc{Loch na Garr}.\\!

Years have rolled on, \textsc{Loch na Garr}, since I left you;\\*
\vin Years must elapse ere I tread you again:\\
Nature of verdure \& flowers has bereft you,\\
\vin Yet still are you dearer than Albion's plain.\\
England! thy beauties are tame \& domestic,\\
\vin To one who has roved o'er the mountains afar;\\
Oh for the crags that are wild \& majestic!\\*
\vin The steep frowning glories of dark \textsc{Loch na Garr}!
\end{verse}

\subsection{}

\blfootnote{Anonymous, \cite{sailors}.}\begin{center}
\textit{Tune: Sally Racket}
\end{center}

\settowidth{\versewidth}{Now she's got a little barson --}
\begin{verse}[\versewidth]
Little \textit{Sally Racket},\\*
\vin \textit{Haul 'im away!}\\
She pawned my best jacket,\\
\vin \textit{Haul 'im away!}\\
An' she lost the ticket --\\
\vin \textit{Haul 'im away!}\\
An' a haul-ee high-O!\\*
\vin \textit{Haul 'im away!}\\!

Little \textit{Kitty Carson}\\*
Got off with a parson;\\
Now she's got a little barson --\\*
An' a haul-ee high-O!\\!

Little \textit{Nancy Dawson},\\*
She got a notion\\
For a poor old boatswain --\\*
An' a haul-ee high-O!\\!

Little \textit{Susie Skinner}\\*
She said she's a beginner,\\
And she prefers it to 'er dinner,\\
So up, lads, an' win 'er --\\*
An' a haul-ee high-O!\\!

Well, me fighting cocks now,\\*
\vin \textit{Haul 'im away!}\\
Haul an' split 'er blocks now,\\
\vin \textit{Haul 'im away!}\\
An' we'll stretch a luff, boys,\\
\vin \textit{Haul 'im away!}\\
An' that'll be enough, boys.\\*
\vin \textit{Haul 'im away!}
\end{verse}

\subsection{}

\blfootnote{The Rt Hon Joseph Addison (1672 -- 1719), \cite{odq}.}Man is distinguished from all other creatures by the faculty of laughter.

\section{}

\subsection{}

\blfootnote{$\mathbb{R}$ Robert Burns (1759 -- 1796), \cite{burns}. Cunningham remarks: `In one of the variations of this song the name of the heroine is Jeanie: the song itself owes some of the sentiments as well as words to an old favourite Nithsdale chant of the same name. ``Is `Whistle, and I'll come to you, my lad','' Burns inquires of Thomson, ``one of your airs? I admire it much, and yesterday I set the following verses to it.'' The poet, two years afterwards, altered the fourth line thus: ``Thy Jeany will venture wi' ye, my lad,'' and assigned this reason: ``In fact, a fair dame at whose shrine I, the priest of the Nine, offer up the incense of Parnassus; a dame whom the Graces have attired in witchcraft, and whom the Loves have armed with lightning; a fair one, herself the heroine of the song, insists on the amendment, and dispute her commands if you dare.{''}'}\begin{center}
\textit{Tune: Whistle, and I'll Come to You}
\end{center}

\settowidth{\versewidth}{Tho' father and mother and a' should go mad,}
\begin{verse}[\versewidth]
{\itshape
O whistle, and I'll come to you, my lad!\\
O whistle, and I'll come to you, my lad!\\
Tho' father and mother and a' should go mad,\\*
Whistle, and I'll come to you, my lad!}\\!

But warily tent, when you come to court me.\\*
And come na unless the back-yett be ajee;\\
Syne up the back-stile and let nobody see,\\
And come as you were na comin' to me,\\*
And come as you were na comin' to me.\\!

At kirk, or at market, whene'er you meet me,\\*
Gang by me as tho' that ye cared na a flie;\\
But steal me a blink o' your bonnie black e'e;\\
Yet look as you were na lookin' at me,\\*
Yet look as you were na lookin' at me.\\!

Ay vow and protest that you care na for me,\\*
And whiles you may lightly my beauty a wee;\\
But court na another, though jokin' you be,\\
For fear that she wile your fancy from me,\\*
For fear that she wile your fancy from me.\\!

{\itshape
O whistle, and I'll come to you, my lad!\\
O whistle, and I'll come to you, my lad!\\
Tho' father and mother and a' should go mad,\\*
Whistle, and I'll come to you, my lad!}
\end{verse}

\subsection{}

\blfootnote{Anonymous, \cite{planxty}.}\begin{center}
\textit{Tune: Sally Brown}
\end{center}

\settowidth{\versewidth}{And we rolled all night and we rolled till the day,}
\begin{verse}[\versewidth]
I shipped onboard of a \textsc{Liverpool} liner.\\!

{\itshape
Way! Hey! Roll \& go!\\
And we rolled all night and we rolled till the day,\\*
To spend my money along with \textit{Sally Brown}!}\\!

\textit{Sally Brown} is a nice young lady.\\!

She's tall and she's dark but she's not too shady.\\!

Her mother doesn't like no tarry sailor.\\!

She wants her to marry a one-legg\`{e}d captain.\\!

\textit{Sally} wouldn't wed me, so I shipped across the water.\\!

And now I am courting \textit{Sally}'s daughter.\\!

{\itshape
Way! Hey! Roll \& go!\\
And we rolled all night and we rolled till the day,\\*
To spend my money along with \textit{Sally Brown}!}
\end{verse}

\subsection{}

\blfootnote{Anonymous, \cite{homilies}. It's unclear whether the translator wished to remain anonymous, or if his work was done by a committee; either way, he was translating a passage from St Augustine's Seventh Homily on the First Letter of John.}Love, and do what thou wilt.

\section{}

\subsection{}

\blfootnote{$\mathbb{R}$ Robert Burns (1759 -- 1796), \cite{burns}. Cunningham comments: `{``}Conjugal love,'' says the poet, ``is a passion which I deeply feel and highly venerate: but somehow it does not make such a figure in poesie as that other species of the passion, where love is liberty and nature law. Musically speaking, the first is an instrument of which the gamut is scanty and confined, but the tones inexpressibly sweet, while the last has powers equal to all the intellectual modulations of the human soul.'' It must be owned that the bard could render very pretty reasons for his rapture about Jean Lorimer.' Cunningham states that this song ought to be sung to a tune called \refpoem{Rothemurche's Rant}, but this seems quite a different one from that which the Almanacker is used to singing.}\begin{center}
\textit{Tune: Lassie wi' the Lint-White Locks}
\end{center}

\settowidth{\versewidth}{Has cheered ilk drooping little flower.}
\begin{verse}[\versewidth]
{\itshape
Lassie, wi' the lint-white locks,\\
\vin Bonnie lassie, artless lassie,\\
Wilt thou wi' me tend the flocks?\\*
\vin Wilt thou be my dearie, O?}\\!

Now nature cleeds the flowery lea,\\*
And a' is young \& sweet like thee;\\
Wilt thou share its joy wi' me,\\*
\vin And say thou'lt be my dearie, O?\\!

And when the welcome summer shower\\*
Has cheered ilk drooping little flower.\\
We'll to the breathing woodbine lower\\*
\vin At sultry noon, my dearie, O.\\!

When \textit{Cynthia} lights wi' silver ray,\\*
The weary shearer's hameward way;\\
Through yellow waving fields we'll stray,\\*
\vin And talk o' love, my dearie, O.\\!

And when the howling wintry blast\\*
Disturbs my lassie's midnight rest;\\
Enclasp\`{e}d to my faithfu' breast,\\*
\vin I'll comfort thee, my dearie, O.\\!

{\itshape
Lassie, wi' the lint-white locks,\\
\vin Bonnie lassie, artless lassie,\\
Wilt thou wi' me tend the flocks?\\*
\vin Wilt thou be my dearie, O?}
\end{verse}

\subsection{}

\blfootnote{Anonymous, \cite{sailors}. `Booble Alley' seems to have been a slang term for one of the roughest parts of town, the slang in question being possibly local to Liverpool (which tells you just how rough it must have been). The term `old man' refers to the captain. There seem to be many more verses to this song, but these are the only ones that are printable.}\begin{center}
\textit{Tune: Haul Away for Rosie}
\end{center}

\settowidth{\versewidth}{I thought I heard the old man say, It's time for us to roll go.}
\begin{verse}[\versewidth]
Talk about your harbour girls around the corner, \textit{Sally},\\*
\vin \textit{Away! Haul away! Haul away, me \textit{Rosie}!}\\
\vin \textit{Away! Haul away! Haul away, me \textit{Johnny}-O!}\\
But they wouldn't go to tea with the girls from Booble Alley.\\
\vin \textit{Away! Haul away! Haul away, me \textit{Rosie}!}\\*
\vin \textit{Away! Haul away! Haul away, me \textit{Johnny}-O!}\\!

King \textit{Louis} was the king of France before the revolution,\\*
But the people cut 'is 'ead off and it spoiled 'is constitution.\\!

Well now we're leaving \textsc{Liverpool} bound for the bay of Mexico,\\*
\vin \textit{Away! Haul away! Haul away, me \textit{Rosie}!}\\
\vin \textit{Away! Haul away! Haul away, me \textit{Johnny}-O!}\\
I thought I heard the old man say, It's time for us to roll \& go.\\
\vin \textit{Away! Haul away! Haul away, me \textit{Rosie}!}\\*
\vin \textit{Away! Haul away! Haul away, me \textit{Johnny}-O!}
\end{verse}

\subsection{}

\blfootnote{`Judges 5.25', Judges 5.25, \cite{kjv}.}He asked for water and she gave him milk.

\section{}

\subsection{}

\blfootnote{$\mathbb{R}$ Robert Burns (1759 -- 1796), \cite{treasury}. James Fenton, in his \refbook{Faber Book of Love Poems}, calls the tune \refpoem{Major Graham}, and this would seem to be the commonly-received name; but Cunningham calls it \refpoem{Graham's Strathspey}.}\begin{center}
\textit{Tune: Major Graham}
\end{center}

\settowidth{\versewidth}{And I will love thee still, my dear,}
\begin{verse}[\versewidth]
O my love's like a red red rose\\*
\vin That's newly sprung in june!\\
O my love's like the melody\\
\vin That's sweetly played in tune!\\
As fair art thou, my bonnie lass,\\
\vin So deep in love am I;\\
And I will love thee still, my dear,\\*
\vin Till a' the seas gang dry --\\!

Till a' the seas gang dry, my dear,\\*
\vin And the rocks melt wi' the sun;\\
I will love thee still, my dear,\\
\vin While the sands o' life shall run.\\
And fare thee well, my only love!\\
\vin And fare thee well awhile!\\
And I will come again, my love,\\*
\vin Though it were 10,000 mile.
\end{verse}

\subsection{}

\blfootnote{Anonymous, \cite{lloyd}. The Almanacker read a piece of folklore somewhere -- far too poetical to be true -- stating that dying sailors used to request this shanty be sung over their deathbeds, so as to pass into the next world feeling happy.}\begin{center}
\textit{Tune: South Australia}
\end{center}

\settowidth{\versewidth}{You'll wish to God you'd never been born.}
\begin{verse}[\versewidth]
In South Australia I was born,\\*
\vin \textit{Heave away! Haul away!}\\
In South Australia round \textsc{Cape Horn}.\\*
\vin \textit{We're bound for South Australia!}\\!

{\itshape
Haul away, you rolling king!\\
Heave away! Haul away!\\
Haul away! O hear me sing:\\*
We're bound for South Australia!}\\!

As I walked out one morning fair,\\*
There I met Miss \textit{Nancy Blair}.\\!

There ain't but one thing grieves my mind:\\*
To leave Miss \textit{Nancy Blair} behind.\\!

I ran her all night; I ran her all day,\\*
Ran her before we sailed away.\\!

I shook her up; I shook her down;\\*
I shook her round \& round \& round.\\!

O when we lollop around \textsc{Cape Horn},\\*
\vin \textit{Heave away! Haul away!}\\
You'll wish to God you'd never been born.\\*
\vin \textit{We're bound for South Australia!}\\!

{\itshape
Haul away, you rolling king!\\
Heave away! Haul away!\\
Haul away! O hear me sing:\\*
We're bound for South Australia!}
\end{verse}

\subsection{}

\blfootnote{1 John 2.8, \cite{kjv}.}The darkness is past, and the true light now shineth.

\section{}

\subsection{}

\blfootnote{$\mathbb{R}$ Robert Burns (1759 -- 1796), \cite{burns}. Cunningham comments: `An old strain, called ``The Birks of Aberfeldie'', was the forerunner of this sweet song: it was written, the poet says, standing under the Falls of Aberfeldy, near Moness, in Perthshire, during one of the tours which he made to the north, in the year 1757.'}\begin{center}
\textit{Tune: The Birks of Aberfeldy}
\end{center}

\settowidth{\versewidth}{While o'er their heads the hazels hing,}
\begin{verse}[\versewidth]
Now summer blinks on flowery braes,\\*
And o'er the crystal streamlet plays;\\
Come let us spend the lightsome days\\*
\vin In the birks of \textsc{Aberfeldy}.\\!

{\itshape
Bonnie lassie, will ye go?\\
Will ye go? Will ye go?\\
Bonnie lassie, will ye go\\*
To the birks of \textsc{Aberfeldy}?}\\!

The little birdies blithely sing,\\*
While o'er their heads the hazels hing,\\
Or lightly flit on wanton wing\\*
\vin In the birks of \textsc{Aberfeldy}.\\!

The braes ascend, like lofty wa's;\\*
The foamy stream deep-roaring fa's,\\
O'erhung wi' fragrant spreading shaws,\\*
\vin The birks of \textsc{Aberfeldy}.\\!

The hoary cliffs are crowned wi' flowers,\\*
White o'er the linns the burnie pours,\\
And rising, wets wi' misty showers\\*
\vin The birks of \textsc{Aberfeldy}.\\!

Let fortune's gifts at random flee;\\*
They ne'er shall draw a wish frae me,\\
Supremely blest wi' love \& thee,\\*
\vin In the birks of \textsc{Aberfeldy}.\\!

{\itshape
Bonnie lassie, will ye go?\\
Will ye go? Will ye go?\\
Bonnie lassie, will ye go\\*
To the birks of \textsc{Aberfeldy}?}
\end{verse}

\subsection{}

\blfootnote{Anonymous, \cite{ac4}. Albert Lloyd wrote in the sleeve notes to \refbook{Blow Boys Blow}: `One of the great halyard shanties, seemingly better-known in English ships than American ones, though some versions of it have become crossed with the American song called ``Huckleberry Hunting''. From the graceful movement of its melody it is possible that this is an older shanty than most. Perhaps it evolved out of some long-lost lyrical song.'}\begin{center}
\textit{Tune: Wild Goose Shanty}
\end{center}

\settowidth{\versewidth}{They're just like them pretty girls when they gets the notion.}
\begin{verse}[\versewidth]
Did you ever see a wild goose sailing on the ocean?\\*
\vin \textit{Ranzo! Ranzo! Way-hey!}\\
They're just like them pretty girls when they gets the notion.\\*
\vin \textit{Ranzo! Ranzo! Way-hey!}\\!

The other morning I was walkin' by the river,\\*
When I saw a young girl walkin' with her top-sails all aquiver.\\!

I said, Pretty fair maid, then how are you this mornin'?\\*
She said, None the better for the seein' of you.\\!

Did you ever see a wild goose sailin' o'er the ocean?\\*
\vin \textit{Ranzo! Ranzo! Way-hey!}\\
They're just like them pretty girls when they gets the notion.\\*
\vin \textit{Ranzo! Ranzo! Way-hey!}
\end{verse}

\subsection{}

\blfootnote{`Judges 14.14', Judges 14.14, \cite{kjv}.}Out of the strong came forth sweetness.

\section{}

\subsection{}

\blfootnote{$\mathbb{R}$ `Jessie, the Flower o' Dunblane', Robert Tannahill (1774 -- 1810), \cite{tannahill}. The Almanacker has changed the tense in the first sentence from the present to the past.}\begin{center}
\textit{Tune: Jessie the Flower of Dunblane}
\end{center}

\settowidth{\versewidth}{    Till charmed wi' sweet Jessie, the flower o' Dunblane.}
\begin{verse}[\versewidth]
The sun had gone down o'er the lofty \textsc{Ben Lomond},\\*
\vin And left the red clouds to preside o'er the scene,\\
While lonely I strayed in the calm summer gloamin'\\
\vin To muse on sweet \textit{Jessie}, the flower o' \textsc{Dunblane}.\\
How sweet is the brier, wi' its soft folding blossom,\\
\vin And sweet is the birch, wi' its mantle o' green;\\
Yet sweeter \& fairer, \& dear to this bosom,\\*
\vin Is lovely young \textit{Jessie}, the flower o' \textsc{Dunblane}:\\!

{\itshape
Is lovely young \textit{Jessie},\\
Lovely young \textit{Jessie},\\
Lovely young \textit{Jessie},\\*
The flower o' \textsc{Dunblane}.}\\!

She's modest as any, and blithe as she's bonny,\\*
\vin For guileless simplicity marks her its ain;\\
And far be the villain, divested o' feeling,\\
\vin Who'd blight, in its bloom, the sweet flower o' \textsc{Dunblane}.\\
Sing on, thou sweet mavis, thy hymn to the evening;\\
\vin Thou'rt dear to the echoes of \textsc{Calderwood Glen};\\
So dear to this bosom, so artless \& winning,\\*
\vin Is charming young \textit{Jessie}, the flower o' \textsc{Dunblane}:\\!

{\itshape
Is charming young \textit{Jessie}, \&c.}\\!

How lost were my days till I met wi' my \textit{Jessie},\\*
\vin The sports o' the city seemed foolish \& vain;\\
I ne'er saw a nymph I would ca' my dear lassie,\\
\vin Till charmed wi' sweet \textit{Jessie}, the flower o' \textsc{Dunblane}.\\
Though mine were the station o' loftiest grandeur,\\
\vin Amidst its profusion I'd languish in pain;\\
And reckon as nothing the height o' its splendour,\\*
\vin If wanting young \textit{Jessie}, the flower o' \textsc{Dunblane}:\\!

{\itshape
If wanting young \textit{Jessie},\\
Lovely young \textit{Jessie},\\
Lovely young \textit{Jessie},\\*
The flower o' \textsc{Dunblane}.}
\end{verse}

\subsection{}

\blfootnote{`Whip Jamboree', Anonymous, \cite{lloyd}. Lizard Point and Start Point are lighthouses on the south coast of England, in Cornwall and Devon respectively; they remain important landmarks for a mariner making his way up the English Channel to this day. The Blackwall docks were an important London dockyard, now, sadly, out of use.}\begin{center}
\textit{Tune: Whup Jamboree}
\end{center}

\settowidth{\versewidth}{And the Start, my boys, we'll heave in sight;}
\begin{verse}[\versewidth]
The pilot he looks out ahead,\\*
With a hand on the chains a-heaving on the lead,\\
And the old man roars to wake the dead.\\*
Come and get your oats, my son!\\!

{\itshape
Whip jamboree! Whip jamboree!\\
You long-tailed black man, come up behind!\\
Whip jamboree! Whip jamboree!\\*
\textit{Johnny}, get your oats, my son!}\\!

O now we're past the \textsc{Lizard} light;\\*
And the \textsc{Start}, my boys, we'll heave in sight;\\
We'll soon be abreast of the Isle of Wight.\\*
Come and get your oats, my son!\\!

O when we get to the \textsc{Blackwall} docks,\\*
Those pretty young girls come down in flocks\\
With short-legged drawers \& long-tailed frocks.\\*
Come and get your oats, my son!\\!

{\itshape
Whip jamboree! Whip jamboree!\\
You long-tailed black man, come up behind!\\
Whip jamboree! Whip jamboree!\\*
\textit{Johnny}, get your oats, my son!}
\end{verse}

\subsection{}

\blfootnote{The Rev Prof William Shedd (1820 -- 1894), \cite{confessions}. Rev Prof Shedd is here translating a passage from St Augustine's \refbook{Confessions}.}Give me chastity and continency, only not yet.

\section{}

\subsection{}

\blfootnote{Sir Hugh Roberton (1874 -- 1952), \cite{isla}. The poet may have been inspired by the Irish folk song \refpoem{Trasna na dTonnta}. `Cathay' is an archaic name for China, while `Islay' is pronounced to rhyme with `tiler'.}\begin{center}
\textit{Tune: Westering Home}
\end{center}

\settowidth{\versewidth}{There I would hie me and there I would rest}
\begin{verse}[\versewidth]
{\itshape
Westering home, and a song in the air,\\
Light in the eye \& it's goodbye to care;\\
Laughter o' love, and a welcoming there,\\*
\vin Isle of my heart, my own one.}\\!

Tell me o' lands o' the orient gay,\\*
Speak o' the riches \& joys o' Cathay;\\
Eh, but it's grand to be wakin' ilk day\\*
\vin To find yourself nearer to \textsc{Islay}.\\!

Where are the folk like the folk o' the west,\\*
Canty \& couthy \& kindly, the best?\\
There I would hie me and there I would rest\\*
\vin At home wi' my own folk in \textsc{Islay}.\\!

Now I'm at home and at home I do lay,\\*
Dreaming of riches that come from Cathay,\\
I'll hop a good ship and be on my way,\\*
\vin And bring back my fortune to \textsc{Islay}.\\!

{\itshape
Westering home, and a song in the air,\\
Light in the eye \& it's goodbye to care;\\
Laughter o' love, and a welcoming there,\\*
\vin Isle of my heart, my own one.}
\end{verse}

\subsection{}

\blfootnote{Anonymous, \cite{ccrown}. It's unclear whether or not this song is a true sea shanty; although it's certainly old enough to be one, and, in any case, it displays a marked influence from more bona fide shanties in terms of structure and subject matter.}\begin{center}
\textit{Tune: New York Girls}
\end{center}

\settowidth{\versewidth}{As I walked down through Chatham Street a fair maid I did meet.}
\begin{verse}[\versewidth]
As I walked down through \textsc{Chatham Street} a fair maid I did meet.\\*
She asked me to see her home; she lived in \textsc{Bleeker Street}.\\!

{\itshape
And away, you \textit{Santy}! My dear honey!\\*
O you \textsc{New York} girls, can't you dance the polka?}\\!

And when we got to \textsc{Bleeker Street} we stopped at 44.\\*
Her mother \& her sister were to meet her at the door.\\!

And when I got inside the house the drinks were passed around.\\*
The liquor was so awful strong my head went round \& round.\\!

And then we had another drink before we sat to eat.\\*
The liquor was so awful strong I quickly fell asleep.\\!

When I awoke next morning I had an aching head.\\*
There was I, \textit{Jack} all alone, stark naked in my bed.\\!

My gold watch \& my pocket book \& lady friend were gone.\\*
And there was I, \textit{Jack} all alone, stark naked in my room.\\!

On looking round this little room there's nothing I could see\\*
But a woman's shift \& apron that were no use to me.\\!

With a flour barrel for a suit of clothes down \textsc{Cherry Street} forlorn,\\*
Where \textit{Martin Churchill} took me in and sent me round \textsc{Cape Horn}.\\!

{\itshape
And away, you \textit{Santy}! My dear honey!\\*
O you \textsc{New York} girls, can't you dance the polka?}
\end{verse}

\subsection{}

\blfootnote{Henry Shaw (1818 -- 1885), \cite{odq}.}Love is like the measles; we can't have it bad but once, and the later in life we have it the tougher it goes with us.

\section{}

\subsection{}

\blfootnote{Anonymous, \cite{corriesa}.}\begin{center}
\textit{Tune: The Ettrick Lady}
\end{center}

\settowidth{\versewidth}{Falla talla-roo! Dumma-roo! Dumma-roo-dum!}
\begin{verse}[\versewidth]
As I gaed down the \textsc{Ettrick} valley\\*
\vin At the hour of 12 at night,\\
Who did I see but a handsome lassie\\
\vin Combing her hair by candlelight?\\
`Lassie, I have come a-courting,\\
\vin Your fine favours for to win;\\
And, if you'll but smile upon me,\\*
\vin Next sunday night I'll call again.'\\!

{\itshape
Falla talla-roo! Dumma-roo! Dumma-roo-dum!\\*
Falla talla-roo! Dumma-roo-dum-day!}\\!

`So to me you've to come your courting,\\*
\vin My fine favours for to win,\\
But it would give me the greatest pleasure\\
\vin If you never did call again.\\
What would I do when I go to walking,\\
\vin Walking out for the \textsc{Ettrick} view?\\
What would I do when I go to walking,\\*
\vin Walking out with a laddie like you?'\\!

`Lassie, I have gold \& silver.\\*
\vin Lassie, I have houses \& land.\\
Lassie, I have ships in the ocean;\\
\vin They'll be all at your command.'\\
`What do I care for your ships on the ocean?\\
\vin What do I care fpr your houses \& land?\\
What do I care for your gold \& silver\\*
\vin When all I want is a handsome man?\\!

`Did you ever see the grass in the morning\\*
\vin All bedecked with jewels rare?\\
Did you ever see a handsome lassie,\\
\vin Diamonds sparkling in her hair?\\
Did you ever see a copper kettle\\
\vin Mended with an old tin can?\\
Did you ever see a handsome lassie\\*
\vin Married off to an ugly man?'\\!

{\itshape
Falla talla-roo! Dumma-roo! Dumma-roo-dum!\\*
Falla talla-roo! Dumma-roo-dum-day!}
\end{verse}

\subsection{}

\blfootnote{Anonymous, \cite{sailors}.}\begin{center}
\textit{Tune: Reuben Ranzo}
\end{center}

\settowidth{\versewidth}{Well she begged 'er dad for mercy.}
\begin{verse}[\versewidth]
O poor old \textit{Reuben Ranzo},\\*
\vin \textit{\textit{Ranzo}, me boys! \textit{Ranzo}!}\\
O poor old \textit{Reuben Ranzo},\\*
\vin \textit{\textit{Ranzo}, me boys! \textit{Ranzo}!}\\!

O \textit{Ranzo} was no sailor,\\*
So 'e shipped aboard a whaler.\\!

O \textit{Ranzo} was no beauty,\\*
So 'e couldn't do his duty.\\!

O because 'e was so dirty,\\*
We gave 'im five \& 30.\\!

O the skipper's daughter \textit{Susie},\\*
Well she begged 'er dad for mercy.\\!

O she gave 'im wine \& water,\\*
And a bit more than she ought t'.\\!

Well 'e got 'is first-mate papers.\\*
'E's a terror to the whalers.\\!

Now 'e's known wherever them whale-fish blow\\*
\vin \textit{\textit{Ranzo}, me boys! \textit{Ranzo}!}\\
As the hardest bastard on the go.\\*
\vin \textit{\textit{Ranzo}, me boys! \textit{Ranzo}!}
\end{verse}

\subsection{}

\blfootnote{Prof Benjamin Jowett (1817 -- 1893), \cite{politics}. Prof Jowett is here translating a passage from Aristotle's \refbook{Politics}.}Nature makes nothing incomplete, and nothing in vain.

\section{}

\subsection{}

\blfootnote{$\mathbb{R}$ Robert Burns (1759 -- 1796), \cite{burns}. Cunningham remarks that Burns removed `some of the coarse chaff' from the old chant in adapting it to this song, but enough remains for the listener to fill in the blanks. Holden Caulfield's innocent misunderstanding of the true meaning of this song is the explanation behind the odd title of Salinger's \refbook{Catcher in the Rye}.}\begin{center}
\textit{Tune: Common Frae the Town}
\end{center}

\settowidth{\versewidth}{Coming through the rye, poor body,}
\begin{verse}[\versewidth]
Coming through the rye, poor body,\\*
\vin Coming through the rye,\\
She draiglet a' her petticoatie,\\*
\vin Coming through the rye.\\!

{\itshape
\textit{Jenny}'s a' wet, poor body;\\
\textit{Jenny}'s seldom dry.\\
She draiglet a' her petticoatie.\\*
Coming through the rye.}\\!

Gin a body meet a body\\*
\vin Coming through the rye,\\
Gin a body kiss a body,\\*
\vin Need a body cry?\\!

Gin a body meet a body,\\*
\vin Coming through the glen,\\
Gin a body kiss a body\\*
\vin Need the world ken?\\!

{\itshape
\textit{Jenny}'s a' wet, poor body;\\
\textit{Jenny}'s seldom dry.\\
She draiglet a' her petticoatie.\\*
Coming through the rye.}
\end{verse}

\subsection{}

\blfootnote{Anonymous, \cite{lloyd}. Lloyd comments in the sleeve-notes: `This topsail halyard shanty, ``Blow Boys Blow'', originated on the West African run, during the days of the slave trade. Later, with the Congo River stanzas dropped, it passed into use aboard Atlantic packets. The skipper's name is given variously as Bully Hayes, Bully Sims, and One-Eyed Kelly. The stanza about the packet-ship firing its gun may date from the Civil War, or may refer to an anti-slavery patrol.'}\begin{center}
\textit{Tune: Blow Boys Blow}
\end{center}

\settowidth{\versewidth}{And who do you think was the skipper of her?}
\begin{verse}[\versewidth]
O was you ever on the \textsc{Congo} river --\\*
\vin \textit{Blow, boys! Blow!}\\
Where fever makes the white man shiver? --\\*
\vin \textit{Blow, my bully boys! Blow!}\\!

A yankee ship come down the river.\\*
Her mast \& yards they shone like silver.\\!

And who do you think was the skipper of her?\\*
Why, \textit{Bully Hayes}, the nigger lover.\\!

Who do you think was first mate of her?\\*
Why, \textit{Shanghai Brown}, the sailor robber.\\!

What do you think she's got for cargo?\\*
Why, black sheep that have run the embargo.\\!

What do you think they've got for dinner?\\*
O monkey hearts and donkey's liver.\\!

Yonder comes the {\hoskeroe Arrow} packet.\\*
She fires the gun. Can't you hear the racket?\\!

O blow, my boys, and blow forever.\\*
\vin \textit{Blow, boys! Blow!}\\
O blow me down that \textsc{Congo} river.\\*
\vin \textit{Blow, my bully boys! Blow!}
\end{verse}

\subsection{}

\blfootnote{Ambrose Bierce (1842 -- 1914), \cite{odq}.}Peace, n. In international affairs, a period of cheating between two periods of fighting.

\section{}

\subsection{}

\blfootnote{$\mathbb{R}$ Robert Burns (1759 -- 1796), \cite{burns}. `Hallowmass' is an archaic word for All Saints' Day, i.e. the first day of November.}\begin{center}
\textit{Tune: I'm O'er Young to Marry Yet}
\end{center}

\settowidth{\versewidth}{    I'm feared ye'd spoil the lacing o't.}
\begin{verse}[\versewidth]
I am my mammy's ae bairn;\\*
\vin Wi' uncou' folk I weary, sir;\\
And lying in a man's bed,\\*
\vin I'm fleyed it mak' me eerie, sir.\\!

{\itshape
I'm o'er young! I'm o'er young!\\
\vin I'm o'er young to marry yet!\\
I'm o'er young! 'Twould be a sin\\*
\vin To tak' me frae my mammy yet!}\\!

My mammy coft me a new gown;\\*
\vin The kirk maun ha'e the gracing o't;\\
Were I to lie wi' you, kind sir,\\*
\vin I'm feared ye'd spoil the lacing o't.\\!

Hallowmass is come \& gone;\\*
\vin The nights are long in winter, sir,\\
And you an' I in ae bed,\\*
\vin In truth, I dare na venture, sir.\\!

Fu' loud an' sh'ill the frosty wind\\*
\vin Blows through the leafless tim'er, sir;\\
But if ye come this gate again\\*
\vin I'll older be gin simmer, sir.\\!

{\itshape
I'm o'er young! I'm o'er young!\\
\vin I'm o'er young to marry yet!\\
I'm o'er young! 'Twould be a sin\\*
\vin To tak' me frae my mammy yet!}
\end{verse}

\subsection{}

\blfootnote{Anonymous, \cite{sailors}. Santy Anna = Gen Antonio L\'opez de Santa Anna.}\begin{center}
\textit{Tune: Santy Anna}
\end{center}

\settowidth{\versewidth}{I knocked them little girls two at a time.}
\begin{verse}[\versewidth]
O \textit{Santy Anna} gained a day.\\*
\vin \textit{Hurray! \textit{Santy Ann}-O!}\\
O \textit{Santy Anna} gained a day.\\*
\vin \textit{All on the plains of Mexico!}\\!

O Mexico! Mexico!\\*
O Mexico, where I must go!\\!

Them little girls I do adore,\\*
Their shining eyes \& long black hair.\\!

Why do them yellow girls love me so?\\*
Because I don't tell 'em all I know.\\!

When I was a young man in my prime,\\*
I knocked them little girls two at a time.\\!

Them \textsc{Liverpool} girls ain't got no coal.\\*
They comb their hair with a kipper backbone.\\!

Times is hard and the wages low.\\*
\vin \textit{Hurray! \textit{Santy Ann}-O!}\\
It's time for us to roll \& go.\\*
\vin \textit{All on the plains of Mexico!}
\end{verse}

\subsection{}

\blfootnote{The Rt Hon Joseph Addison (1672 -- 1719), \cite{odq}.}We are always doing... something for posterity, but I would fain see posterity do something for us.

\section{}

\subsection{}

\blfootnote{$\mathbb{R}$ Robert Burns (1759 -- 1796), \cite{ivscots}. In making this short song, Burns has compressed (beautifully) a far longer and older ballad about a young nobleman who is poisoned and killed by his lover.}\begin{center}
\textit{Tune: O Where hae You been, Lord Ronald, My Son?}
\end{center}

\settowidth{\versewidth}{For I'm weary wi' the hunting, and fain would lie doon.'}
\begin{verse}[\versewidth]
`O where ha'e you been, Lord \textit{Ronald}, my son?\\*
O where ha'e you been, Lord \textit{Ronald}, my son?'\\
`I ha'e been wi' my sweetheart. Mother, mak' my bed soon,\\*
For I'm weary wi' the hunting, and fain would lie doon.'\\!

`What got ye frae your sweetheart, Lord \textit{Ronald}, my son?\\*
What got ye frae your sweetheart, Lord \textit{Ronald}, my son?'\\
`I ha'e got deadly poison. Mother, mak' my bed soon,\\*
For life is a burden that soon I'll lay doon.'
\end{verse}

\subsection{}

\blfootnote{Anonymous, \cite{ac4}.}\begin{center}
\textit{Tune: Whisky Johnny}
\end{center}

\settowidth{\versewidth}{I drink it out from an old tin can.}
\begin{verse}[\versewidth]
Whisky is the life of man.\\*
\vin \textit{Whisky! \textit{Johnny}!}\\
O whisky is the life of man.\\*
\vin \textit{Whisky for my \textit{Johnny}-O!}\\!

O I drink whisky when I can.\\*
I drink it out from an old tin can.\\!

Whisky gave me a broken nose.\\*
Whisky made me pawn my clothes.\\!

Whisky drove me around \textsc{Cape Horn}.\\*
It was many a month when I was gone.\\!

I thought I heard the old man say,\\*
I'll treat my crew in a decent way.\\!

A glass of grog for every man --\\*
\vin \textit{Whisky! \textit{Johnny}!}\\
And a bottle for the shantyman --\\*
\vin \textit{Whisky for my \textit{Johnny}-O!}
\end{verse}

\subsection{}

\blfootnote{Anonymous, \cite{odq}. This is an ancient English paraphrase of a sumblime truth which Aristotle articulated more awkwardly.}The whole is more than the sum of the parts.

\section{}

\subsection{}

\blfootnote{Anonymous, \cite{isla}. This song relates a raid made by `Argyll' (i.e. the Covenanter Archibald Campbell, 8th Earl of Argyll and Chief of Clan Campbell) on Airlie Castle (seat of the Royalist James Ogilvie, 1st Earl of Airlie and Chief of Clan Ogilvie) in 1640 during the Wars of the Three Kingdoms. In spite of the unhappy ending, the listener may take some comfort in the fact that Lord Argyll died on the scaffold, whereas Lord Airlie died in his bed.}\begin{center}
\textit{Tune: The Bonnie Hoose o' Airlie}
\end{center}

\settowidth{\versewidth}{    That I wi' na leave a stan'in' stone in Airlie.'}
\begin{verse}[\versewidth]
It fell on a day, and a bonny summer's day,\\*
\vin When the sun shone bright \& clearly,\\
That there fell out a great dispute\\*
\vin Atween \textit{Argyll} and \textit{Airlie}.\\!

\textit{Argyll}, he has mustered a 1000 o' his men;\\*
\vin He has marched them out right early;\\
He has marched them in by the back o' \textsc{Dunkeld},\\*
\vin To plunder the bonny house o' \textit{Airlie}.\\!

Lady \textit{Ogilvie}, she looked from her window so high,\\*
\vin And O but she grat sairly\\
To see \textit{Argyll} and a' his men\\*
\vin Come to plunder the bonny house o' \textit{Airlie}.\\!

`Come down, come down, Lady \textit{Ogilvie},' he cried.\\*
\vin `Come down and kiss me fairly,\\
Or I swear by the hilt o' my good broadsword\\*
\vin That I wi' na leave a stan'in' stone in \textit{Airlie}.'\\!

`I wi' na come down, ye cruel \textit{Argyll};\\*
\vin I wi' na kiss ye fairly;\\
I would na kiss ye, false \textit{Argyll},\\*
\vin Though ye should na leave a stan'in' stone in \textit{Airlie}.'\\!

`Come tell me where your dowry is hid;\\*
\vin Come down and tell me fairly.'\\
`I wi' na tell ye where my dowry is hid,\\*
\vin Though ye should na leave a stan'in' stone in \textit{Airlie}.'\\!

They sought it up \& they sought it down;\\*
\vin I wat they sought it early;\\
And it was below yon bowling green\\*
\vin They found the dowry o' \textit{Airlie}.\\!

`Eleven bairns I ha'e born\\*
\vin And the 12th ne'er saw his daddy,\\
But though I had gotten as many again,\\*
\vin They should 'a' gang to fetch for \textit{Charlie}.\\!

`Gin my good lord had been at home,\\*
\vin As he's awa' for \textit{Charlie},\\
There dares na a \textit{Campbell} o' a' \textit{Argyll}\\*
\vin Set a foot on the bonny house o' \textit{Airlie}.'\\!

He's ta'en her by the milk-white hand,\\*
\vin But he did na lead her fairly;\\
He led her up to the top o' the hill,\\*
\vin Where she saw the burnin' o' \textit{Airlie}.\\!

The smoke \& flame they rose so high;\\*
\vin The walls they were blackened fairly;\\
And the lady laid her down on the green to die\\*
\vin When she saw the burnin' o' \textit{Airlie}.
\end{verse}

\subsection{}

\blfootnote{Anonymous, \cite{johnnycollins}.}\begin{center}
\textit{Tune: Eliza Lee}
\end{center}

\settowidth{\versewidth}{    Clear away the track and let the bulgine run!}
\begin{verse}[\versewidth]
The smartest clipper you can find --\\*
\vin \textit{Ho-ay ho! Are you most done?!}\\
Is the {\hoskeroe Margaret Evans} on the Blue Star Line --\\*
\vin \textit{Clear away the track and let the bulgine run!}\\!

{\itshape
To me aye! Rig a jig in a jolting car!\\
\vin Ho-ay ho! Are you most done?!\\
With \textit{Liza Lee} all on my knee,\\*
\vin Clear away the track and let the bulgine run!}\\!

O the {\hoskeroe Margaret Evans} on the Blue Star Line,\\*
She's never a day behind the time.\\!

O we're outward bound for \textsc{New York} town.\\*
We'll dance them \textsc{Bowery} girls around.\\!

Well we stowed our freight on the \textsc{West Creek} pier.\\*
We'll head right back for some \textsc{Liverpool} beer.\\!

O I thought I heard the old man say,\\*
We'll leave that brig three points away.\\!

And when we're back in \textsc{Liverpool} town --\\*
\vin \textit{Ho-ay ho! Are you most done?!}\\
I'll stand youse whiskys all around --\\*
\vin \textit{Clear away the track and let the bulgine run!}\\!

{\itshape
To me aye! Rig a jig in a jolting car!\\
\vin Ho-ay ho! Are you most done?!\\
With \textit{Liza Lee} all on my knee,\\*
\vin Clear away the track and let the bulgine run!}
\end{verse}

\subsection{}

\blfootnote{Mrs Aphra Behn (1640 -- 1689), \cite{odq}. Mrs Behn, cunning linguist that she was, seems to have lived up to this maxim in her personal life.}Variety is the soul of pleasure.

\section{}

\subsection{}

\blfootnote{$\mathbb{R}$ Robert Burns (1759 -- 1796), \cite{burns}. This song describes the Battle of Sheriffmuir (also called Sherramuir), at which a Government force of six thousand under Archibald Campbell, 2nd Duke of Argyll stood their ground against twelve thousand Jacobites under John Erskine, 23rd Earl of Mar.}\begin{center}
\textit{Tune: Cameronian Rant}
\end{center}

\settowidth{\versewidth}{And through they dashed, and hewed, and smashed,}
\begin{verse}[\versewidth]
O cam' ye here the fight to shun,\\*
\vin Or herd the sheep wi' me, man?\\
Or were ye at the \textsc{Sherramuir},\\
\vin And did the battle see, man?'\\
I saw the battle, sair \& tough;\\
And reekin' red ran many a sheugh;\\
My heart, for fear, gaed sough for sough,\\
To hear the thuds, and see the cluds,\\
O' clans frae woods, in tartan duds,\\*
\vin Who glaumed at kingdoms three, man.\\!

The red-coat lads, wi' black cockades,\\*
\vin To meet them were na slaw, man;\\
They rushed and pushed, and blood out-gushed.\\
\vin And many a bouk did fa', man:\\
The great \textit{Argyll} led on his files,\\
I wat they glanced for 20 miles:\\
They houghed the clans like nine-pin kyles,\\
They hacked and hashed, while broadsword clashed.\\
And through they dashed, and hewed, and smashed,\\*
\vin Till fey men died awa', man.\\!

But had you seen the philibegs\\*
\vin And skyrin tartan trews, man;\\
When in the teeth they dared our Whigs\\
\vin And covenant true blues, man;\\
In lines exten'ed lang \& large,\\
When bayonets o'erpowered the targe,\\
And thousands hastened to the charge,\\
Wi' highland wrath, they frae the sheath\\
Drew blades o' death, till, out o' breath,\\*
\vin They fled like frighted doos, man.\\!

They've lost some gallant gentlemen\\*
\vin Among the highland clans, man;\\
I fear my Lord \textit{Panmure} is slain\\
\vin Or fallen in Whiggish hands, man:\\
Now would ye sing this double fight;\\
Some fell for wrong, and some for right;\\
And many bade the world goodnight;\\
Then ye may tell, how pell \& mell,\\
By red claymores, \& muskets' knell,\\
Wi' dying yell, the Tories fell.\\*
\vin And Whigs to hell did flee, man.
\end{verse}

\subsection{}

\blfootnote{Anonymous, \cite{ac4}.}\begin{center}
\textit{Tune: Homeward Bound}
\end{center}

\settowidth{\versewidth}{And when we gets home, boys, O won't we fly round?}
\begin{verse}[\versewidth]
O don't youse hear the old man say --\\*
\vin \textit{Goodbye, fare ye well! Goodbye, fare ye well!}\\
O don't youse hear the old man say --\\*
\vin \textit{Hurrah, my boys! We're homeward bound!}\\!

We're homeward bound to \textsc{Liverpool} town,\\*
Where all them \textit{Judies}, they will come down.\\!

And when we gets to the \textsc{Wallasey} gates,\\*
\textit{Sally} \& \textit{Oily} for their flash-men do wait.\\!

And one to the other ye'll hear them say,\\*
Here comes \textit{Johnny} with his 14 months' pay!\\!

We meet these fly gals and we'll ring the old bell.\\*
With them \textit{Judies}, we'll raise merry hell.\\!

We're homeward bound to the gals of the town,\\*
And stamp up my bullies and heave it around.\\!

And when we gets home, boys, O won't we fly round?\\*
We'll heave up the anchor to this bully sound.\\!

We're all homeward bound for the old backyard.\\*
Then heave, my bullies. We're all bound homeward.\\!

O heave with a will boys. O heave long and strong.\\*
And sing a good chorus for 'tis a good song.\\!

We're homeward bound, we'll have youse to know --\\*
\vin \textit{Goodbye, fare ye well! Goodbye, fare ye well!}\\
And over the water to England must go --\\*
\vin \textit{Hurrah, my boys! We're homeward bound!}
\end{verse}

\subsection{}

\blfootnote{Francis Bacon, Viscount St Alban (1561 -- 1626), \cite{odq}.}There is no excellent beauty that hath not some strangeness in the proportion.

\section{}

\subsection{}

\blfootnote{$\mathbb{R}$ Robert Burns (1759 -- 1796), \cite{burns}. In Scots poetry the word `o'er' is pronounced like the English word `hour'. Old Nick is a euphemism for Satan, and is probably used here figuratively to refer to the Hanoverian kings of Great Britain and Ireland. A `bawbee' was a kind of coin, peculiar to the Kingdom of Scotland, worth about the same as an English halfpenny; though who John Ross was remains a mystery.}\begin{center}
\textit{Tune: O'er the Water to Charlie}
\end{center}

\settowidth{\versewidth}{Come weal, come woe, we'll gather go,}
\begin{verse}[\versewidth]
Come boat me o'er; come row me o'er;\\*
\vin Come boat me o'er to Charlie;\\
I'll gi'e \textit{John Ross} another bawbee,\\*
\vin To boat me o'er to \textit{Charlie}.\\!

{\itshape
We'll o'er the water and o'er the sea;\\
\vin We'll o'er the water to \textit{Charlie};\\
Come weal, come woe, we'll gather \& go,\\*
\vin And live or die wi' \textit{Charlie}.}\\!

I lo'e well my \textit{Charlie}'s name,\\*
\vin Though some there be abhor him:\\
But O, to see \textit{Old Nick} gone hame,\\*
\vin And \textit{Charlie}'s foes before him!\\!

I swear \& vow by moon \& stars,\\*
\vin And sun that shines so early,\\
If I had 20,000 lives,\\*
\vin I'd die as oft for \textit{Charlie}.\\!

{\itshape
We'll o'er the water and o'er the sea;\\
\vin We'll o'er the water to \textit{Charlie};\\
Come weal, come woe, we'll gather \& go,\\*
\vin And live or die wi' \textit{Charlie}.}
\end{verse}

\subsection{}

\blfootnote{Anonymous, \cite{travelin}. Paradise Street is in Liverpool, where this sort of thing is not uncommon.}\begin{center}
\textit{Tune: Blow the Man Down}
\end{center}

\settowidth{\versewidth}{Don't never take heed of what pretty girls say --}
\begin{verse}[\versewidth]
As I was out walking down \textsc{Paradise Street} --\\*
\vin \textit{To me! Way! Hey! Blow the man down!}\\
A pretty young damsel I chanced for to meet --\\*
\vin \textit{Give me some time to blow the man down!}\\!

She was round in the counter and bluff in the bow,\\*
So I took in all sail and cried, `Way enough now!'\\!

I hailed her in english; she answered me clear,\\*
I'm from the \textsc{Black Arrow} bound to the \textsc{Shakespeare}.\\!

So I tailed her my flipper and took her in tow,\\*
And yard-arm to yard-arm, away we did go.\\!

But as we were a-going she said unto me,\\*
There's a spankin' full rigger just ready for sea.\\!

That spankin' full rigger to \textsc{New York} was bound.\\*
She was very well mannered and very well found.\\!

But as soon as that packet was clear of the bar,\\*
The mate knocked me down with the end of a spar.\\!

As soon as that packet was out on the sea,\\*
'Twas devilish hard treatment of every degree.\\!

So I give you fair warning before we belay --\\*
\vin \textit{To me! Way! Hey! Blow the man down!}\\
Don't never take heed of what pretty girls say --\\*
\vin \textit{Give me some time to blow the man down!}
\end{verse}

\subsection{}

\blfootnote{Walter Bagehot (1826 -- 1877), \cite{odq}.}Throughout the greater part of his life George III was a kind of consecrated obstacle.

\section{}

\subsection{}

\blfootnote{Anonymous, \cite{corriesa}.}\begin{center}
\textit{Tune: Come O'er the Stream, Charlie}
\end{center}

\settowidth{\versewidth}{    And his friend the Maclean.}
\begin{verse}[\versewidth]
{\itshape
Come o'er the stream, \textit{Charlie},\\
Dear \textit{Charlie}, brave \textit{Charlie};\\
Come o'er the stream \textit{Charlie},\\
\vin And dine wi' \textit{Maclean}.\\
And, though ye be weary,\\
We'll mak' your heart cheery,\\
And welcome our \textit{Charlie}\\*
\vin And his loyal train.}\\!

We'll bring down the red deer;\\*
We'll bring down the black steer,\\
The lamb frae the bracken\\
\vin And the doe frae the glen.\\
The salt sea we'll harry\\
And bring to our \textit{Charlie}\\
The cream frae the bothy,\\*
\vin The curd frae the pen.\\!

And you shall drink freely\\*
The dews of \textsc{Glensheerly}\\
That stream in the starlight,\\
\vin Where kings di' na ken.\\
And deep be your meed\\
Of the wine that is red,\\
To drink to your sire\\*
\vin And his friend the \textit{Maclean}.\\!

It ought to invite you,\\*
Or more will delight you:\\
'Tis ready; a troop\\
\vin Of our bold highland men\\
Shall range on the heather,\\
With bayonet \& feather,\\
Strong arms \& broad claymores,\\*
\vin Three hundred and 10.\\!

{\itshape
Come o'er the stream, \textit{Charlie},\\
Dear \textit{Charlie}, brave \textit{Charlie};\\
Come o'er the stream \textit{Charlie},\\
\vin And dine wi' \textit{Maclean}.\\
And, though ye be weary,\\
We'll mak' your heart cheery,\\
And welcome our \textit{Charlie}\\*
\vin And his loyal train.}
\end{verse}

\subsection{}

\blfootnote{Anonymous, \cite{sailors}. As with all the best short lyrics, a larger narrative is hinted at in this song -- a narrative which, sadly, is now lost.}\begin{center}
\textit{Tune: Lowlands Away}
\end{center}

\settowidth{\versewidth}{I dreamed a dream the other night.}
\begin{verse}[\versewidth]
I dreamed a dream the other night.\\*
\vin \textit{Lowlands, lowlands away, my \textit{John}.}\\
I dreamed a dream the other night.\\*
\vin \textit{Lowlands away.}\\!

I dreamed I saw my own true love.\\*
I dreamed I saw my own true love.\\!

I dreamed my love was drowned \& dead.\\*
\vin \textit{Lowlands, lowlands away, my \textit{John}.}\\
I dreamed my love was drowned \& dead.\\*
\vin \textit{Lowlands away.}
\end{verse}

\subsection{}

\blfootnote{Dr Bernard Baruch (1870 -- 1965), \cite{odq}.}To me old age is always fifteen years older than I am.

\section{}

\subsection{}

\blfootnote{$\mathbb{R}$ Robert Burns (1759 -- 1796), \cite{burns}. This song seems to relate the story of a Scotttish Jacobite soldier who fought in the Jacobite-Williamite War in Ireland. The final verse of Burns's original has been omitted.}\begin{center}
\textit{Tune: It was A' for Our Rightfu' King}
\end{center}

\settowidth{\versewidth}{    With adieu for evermore, my dear;}
\begin{verse}[\versewidth]
It was a' for our rightfu' king\\*
\vin We left fair Scotland's strand;\\
It was a' for our rightfu' king\\
\vin We e'er saw irish land, my dear;\\*
\vin We e'er saw irish land.\\!

Now a' is done that men can do,\\*
\vin And a' is done in vain;\\
My love and native land farewell,\\
\vin For I maun cross the main, my dear;\\*
\vin For I maun cross the main.\\!

He turned him right, and round about\\*
\vin Upon the irish shore;\\
And ga'e his bridle-reins a shake,\\
\vin With adieu for evermore, my dear;\\*
\vin With adieu for evermore.\\!

The soldier from the wars returns,\\*
\vin The sailor frae the main;\\
But I ha'e parted frae my love.\\
\vin Never to meet again, my dear;\\*
\vin Never to meet again.
\end{verse}

\subsection{}

\blfootnote{Anonymous, \cite{ac4}.}\begin{center}
\textit{Tune: Rollicking Randy Dandy}
\end{center}

\settowidth{\versewidth}{Our boots our clothes, boys, are all in the pawn --}
\begin{verse}[\versewidth]
Now we are ready to sail for the \textsc{Horn} --\\*
\vin \textit{Way hey! Roll \& go!}\\
Our boots \& our clothes, boys, are all in the pawn --\\*
\vin \textit{To me rollicking randy dandy O!}\\!

{\itshape
Heave a pawl! O heave away!\\
\vin Way hey, roll and go!\\
The anchor's onboard and the cable's all stored!\\*
\vin To me rollicking randy dandy O!}\\!

Soon we'll be warping her out through the locks,\\*
Where the pretty young girls all come down in their frocks.\\!

Come breast the bars, bullies. Heave her away.\\*
\vin \textit{Way hey! Roll \& go!}\\
Soon we'll be rolling her way down the bay.\\*
\vin \textit{To me rollicking randy dandy O!}\\!

{\itshape
Heave a pawl! O heave away!\\
\vin Way hey, roll and go!\\
The anchor's onboard and the cable's all stored!\\*
\vin To me rollicking randy dandy O!}
\end{verse}

\subsection{}

\blfootnote{Ambrose Bierce (1842 -- 1914), \cite{odq}.}Conservative, n. A statesman who is enamoured of existing evils, as distinguished from the liberal, who wishes to replace them with others.

\section{}

\subsection{}

\blfootnote{`Bonnie Charlie', Carolina Nairne, Lady Nairne (1766 -- 1845), \cite{jacobite}. Although this song was written several decades after the Jacobite risings, Lady Nairne herself was from a prominent Jacobite family, members of which did actually fight for the House of Stuart.}\begin{center}
\textit{Tune: Will Ye No' Come Back?}
\end{center}

\settowidth{\versewidth}{    Brak' the band o' nature's law;}
\begin{verse}[\versewidth]
Bonny \textit{Charlie}'s now awa',\\*
\vin Safely o'er the friendly main;\\
Many a heart will break i' twa,\\*
\vin Should he no' come back again.\\!

{\itshape
Will ye no come back again?\\
\vin Will ye no come back again?\\
Better lo'ed ye can na be.\\*
\vin Will ye no come back again?}\\!

Many a traitor 'mong the isles\\*
\vin Brak' the band o' nature's law;\\
Many a traitor wi' his wiles\\*
\vin Sought to wear his life awa'.\\!

Many a gallant soldier fought;\\*
\vin Many a gallant chief did fa';\\
Death itself were dearly bought,\\*
\vin A' for Scotland's king and law.\\!

Whene'er I hear the blackbird sing\\*
\vin Unto the evening sinking down,\\
Or meryl that makes the wood to ring,\\*
\vin To me they ha'e no other soun'.\\!

Sweet the laverock's note \& long,\\*
\vin Lilting wildly up the glen;\\
And aye the o'erworld o' the song:\\*
\vin Will he no' come back again?\\!

{\itshape
Will ye no come back again?\\
\vin Will ye no come back again?\\
Better lo'ed ye can na be.\\*
\vin Will ye no come back again?}
\end{verse}

\subsection{}

\blfootnote{Anonymous, \cite{hydestreet}. There's a story, of dubious authority, that, following his death at Trafalgar, Lord Nelson's body was preserved in brandy (which, in itself is almost certainly true). The sailors onboard the \textit{Victory}, not wanting to waste any intoxicating fluid, gradually siphoned off and drank most of said brandy on the journey back to England. Hence the term `Nelson's blood' was and is used amongst seamen to refer to any kind of hard liquor.}\begin{center}
\textit{Tune: Roll the Old Chariot}
\end{center}

\settowidth{\versewidth}{O a drop of Nelson's blood wouldn't do us any harm!}
\begin{verse}[\versewidth]
O a drop of \textit{Nelson}'s blood wouldn't do us any harm!\\*
\vin \textit{O a drop of \textit{Nelson}'s blood wouldn't do us any harm!}\\
\vin \textit{O a drop of \textit{Nelson}'s blood wouldn't do us any harm!}\\*
\vin \textit{And we'll all hang on behind!}\\!

{\itshape
And we'll roll the old chariot along!\\
We'll roll the old chariot along!\\
We'll roll the old chariot along!\\*
And we'll all hang on behind!}\\!

O a plate of irish stew wouldn't do us any harm! \&c.\\!

O a nice fat cook wouldn't do us any harm! \&c.\\!

O a nice watch below wouldn't do us any harm! \&c.\\!

O a good night ashore wouldn't do us any harm!\\*
\vin \textit{A good night ashore wouldn't do us any harm!}\\
\vin \textit{A good night ashore wouldn't do us any harm!}\\*
\vin \textit{And we'll all hang on behind!}\\!

{\itshape
And we'll roll the old chariot along!\\
We'll roll the old chariot along!\\
We'll roll the old chariot along!\\*
And we'll all hang on behind!}
\end{verse}

\subsection{}

\blfootnote{Prof George Long (1800 -- 1879), \cite{aurelius}. Prof Long is here translating a passage from Book VI of Marcus Aurelius's \refbook{Meditations}.}He who has seen present things has seen all, both everything which has taken place from all eternity and everything which will be for time without end; for all things are of one kin and of one form.

\section{}

\subsection{}

\blfootnote{$\mathbb{R}$ `Auld Lang Syne', Robert Burns (1759 -- 1796), \cite{fenton}.}\begin{center}
\textit{Tune: For Old Long Sine, My Jo}
\end{center}

\settowidth{\versewidth}{But seas between us braid ha'e roared}
\begin{verse}[\versewidth]
Should old acquaintance be forgot\\*
\vin And never brought to min'?\\
Should old acquaintance be forgot,\\*
\vin And auld lang syne?\\!

{\itshape
For auld lang syne, my jo,\\
\vin For auld lang syne,\\
We'll tak' a cup o' kindness yet\\*
\vin For auld lang syne.}\\!

And surely ye'll be your pint stowp!\\*
\vin And surely I'll be mine!\\
And we'll tak' a cup o' kindness yet\\*
\vin For auld lang syne.\\!

We twa ha'e run about the braes,\\*
\vin And pu'ed the gowans fine;\\
But we've wandered many a weary foot\\*
\vin Sin' auld lang syne.\\!

We twa ha'e paidled i' the burn,\\*
\vin Frae mornin' sun till dine;\\
But seas between us braid ha'e roared\\*
\vin Sin' auld lang syne.\\!

And here's a hand, my trusty fiere,\\*
\vin And gi'e 's a hand o' thine;\\
And we'll tak' a right guid-willie-waught\\*
\vin For auld lang syne.
\end{verse}

\subsection{}

\blfootnote{Anonymous, \cite{ac4}. Stan Hugill, in his \refbook{Shanties from the Seven Seas}, comments: `And now we come to the ``Johnny'' song that usually ended the voyage -- `Leave Her, Johnny, Leave Her!' Collectors give pumps and halyards alike as the job it was used for. Terry and Whall call it a hauling song; Miss Colcord and Doerflinger give it for pumps. I think they are all right. It was probably sung at halyards with two solos and refrains, and when a full chorus was added then it was used at the pumps and even capstan. I learnt it partly from my mother's father, and he always sang the full chorus, and partly from an old Irish sailor, who also used the final chorus. It probably came to life about the time of the Irish potato famine, in the forties, and was originally sung in the Western Ocean Packets in this fashion... The later version ``Leave Her, Johnnies'' or as some sang it ``Leave Her, Bullies'' was sometimes sung during the voyage -- at the pumps -- but its better-known function was that of airing grievances just prior to the completion of the voyage either when warping the vessel in through the locks or at the final spell of the pumps (in wooden ships) after the vessel had docked. Many unprintable stanzas were sung, directed at the afterguard, the grub, and the owners. Bullen writes that ``to sing it before the last day or so was almost tantamount to mutiny.{''}'}\begin{center}
\textit{Tune: Leave Her, Johnny, Leave Her}
\end{center}

\settowidth{\versewidth}{But now we're through so we'll go on shore.}
\begin{verse}[\versewidth]
I thought I heard the old man say --\\*
\vin \textit{Leave her, \textit{Johnny}! Leave her!}\\
Tomorrow ye will get your pay --\\*
\vin \textit{And it's time for us to leave her.}\\!

{\itshape
Leave her, \textit{Johnny}! Leave her!\\
\vin O leave her, \textit{Johnny}! Leave her!\\
For the voyage is long and the winds don't blow,\\*
\vin And it's time for us to leave her!}\\!

O the wind was foul and the sea ran high;\\*
She shipped it green and none went by.\\!

I hate to sail on this rotten tub,\\*
No grog allowed and rotten grub.\\!

We swear by rote for want of more,\\*
\vin \textit{Leave her, \textit{Johnny}! Leave her!}\\
But now we're through so we'll go on shore.\\*
\vin \textit{And it's time for us to leave her.}\\!

{\itshape
Leave her, \textit{Johnny}! Leave her!\\
\vin O leave her, \textit{Johnny}! Leave her!\\
For the voyage is long and the winds don't blow,\\*
\vin And it's time for us to leave her!}
\end{verse}

\subsection{}

\blfootnote{$\mathbb{R}$ Anonymous, \cite{odq}.}Here's to us. Who's like us? Gey few, and they're a' dead.

\section{}

\subsection{}

\blfootnote{Francis McPeake (1885 -- 1971), \cite{corriesb}. McPeake seems to have been inspired to compose this song by a poem of Robert Tannahill's. Some interpreters of this song (Kate Rusby et al.) have been known to render the first line as, `O the summer time is coming', which, as a certain learned gentleman pointed out to the Almanacker, shows their ignorance of the natural world. Heather blooms in late summer, and not in the spring.}\begin{center}
\textit{Tune: Wild Mountain Thyme}
\end{center}

\settowidth{\versewidth}{    Grows around the purple heather.}
\begin{verse}[\versewidth]
O the summer time has come,\\*
\vin And the trees are sweetly blooming,\\
And wild mountain thyme\\
\vin Grows around the purple heather.\\*
\vin \vin Will you go, lassie, go?\\!

{\itshape
And we'll all go together\\
\vin To pull wild mountain thyme\\
All around the purple heather.\\*
\vin Will you go, lassie, go?}\\!

I will build my love a bower\\*
\vin By yon clear crystal fountain,\\
And on it I will pile\\
\vin All the flowers of the mountain.\\*
\vin \vin Will you go, lassie, go?\\!

I will range through the wilds\\*
\vin And the deep land so dreary\\
And return with the spoils\\
\vin To the bower o' my dearie.\\*
\vin \vin Will ye go, lassie, go?\\!

If my truelove she'll not come,\\*
\vin Then I'll surely find another\\
To pull wild mountain thyme\\
\vin All around the purple heather.\\*
\vin \vin Will you go, lassie, go?\\!

{\itshape
And we'll all go together\\
\vin To pull wild mountain thyme\\
All around the purple heather.\\*
\vin Will you go, lassie, go?}
\end{verse}

\subsection{}

\blfootnote{Anonymous, \cite{ac4}. Padstow is a fishing village on the north coast of Cornwall.}\begin{center}
\textit{Tune: Padstow Farwell}
\end{center}

\settowidth{\versewidth}{Haul away your halyards! etc}
\begin{verse}[\versewidth]
It is time to go now.\\*
\vin \textit{Haul away your anchor!}\\
\vin \textit{Haul away your anchor!}\\*
\vin \textit{'Tis our sailing time!}\\!

Get some sail upon her.\\*
\vin \textit{Haul away your halyards! \&c.}\\!

Get her on her course now.\\*
\vin \textit{Haul away your foresheets! \&c.}\\!

Waves are surging under.\\*
\vin \textit{Haul away down-channel!}\\
\vin \textit{Haul away down-channel}\\*
\vin \textit{On the evening tide!}\\!

When your sailing's over:\\*
\vin \textit{Haul away for heaven!}\\
\vin \textit{Haul away for heaven!}\\*
\vin \textit{God be by your side!}
\end{verse}

\subsection{}

\blfootnote{Anonymous, \cite{odq}. The `fellow traveller' is Sir Francis Drake, who was one of the first men to circumnavigate the globe, thus, in a certain sense, under the geocentric view of things, made the same journey which the sun makes every day.}The sun himself cannot forget his fellow traveller.

\chapter{Sextilis}

\section{}

\subsection{}

\blfootnote{\cite{anglican}. Although this song was written in the 1970s, the words are -- it hardly needs pointing out -- very much older, being drawn more or less as-is from the King James Version. \P 1. This first verse is drawn verbatim from the KJV Matthew 6.33. \P 7. This second verse is drawn from the KJV 4.4. \P 9. The only difference from the KJV in this verse is in this line, with the awkward `that proceedeth out of the mouth of God' being massaged into the form given here. \P 10. This third verse is drawn from the KJV Matthew 7.7. The only difference is that an `unto' has been inserted after `Ask, and it shall be given'.}\begin{center}
\textit{Tune: Lafferty}
\end{center}

{\color{red} \settowidth{\versewidth}{Knock, and it shall be opened unto you:}
\begin{verse}[\versewidth]
Seek ye first the kingdom of God\\*
And his righteousness\\
And all these things shall be added unto you.\\*
\vin \textit{Allelu alleluia!}\\!

{\itshape
Alleluia! Alleluia!\\*
Allelui! Allelu alleluia!}\\!

Man shall not live by bread alone,\\*
But by every word\\*
That proceeds from the mouth of God.\\!

Ask, and it shall be given unto you;\\*
Seek, and ye shall find;\\
Knock, and it shall be opened unto you:\\*
\vin \textit{Allelu alleluia!}\\!

{\itshape
Alleluia! Alleluia!\\*
Allelui! Allelu alleluia!}
\end{verse}}

\subsection{}

\blfootnote{Anonymous, \cite{obev}.}\settowidth{\versewidth}{The cow jumped over the moon.}
\begin{verse}[\versewidth]
Hey diddle diddle,\\*
The cat \& the fiddle.\\
The cow jumped over the moon.\\
The little dog laughed\\
To see such sport\\*
And the dish ran away with the spoon.
\end{verse}

\subsection{}

\blfootnote{William Blake (1757 -- 1827), \cite{blakea}. This is one of Blake's `Proverbs of Hell' from \refbook{The Marriage of Heaven and Hell}.}The nakedness of woman is the work of God.

\section{}

\subsection{}

\blfootnote{Anthony Showalter (1858 -- 1924), \cite{truegrit}. The `everlasting arms' in question are those referred to in Deuteronomy 33.27. This song features prominently in the 2010 version of the film \refbook{True Grit}.}\begin{center}
\textit{Tune: Leaning on the Everlasting Arms}
\end{center}

\settowidth{\versewidth}{Leaning, leaning, safe secure from all alarms;}
\begin{verse}[\versewidth]
What a fellowship, what a joy divine --\\*
\vin \textit{Leaning on the everlasting arms.}\\
What a blessedness, what a peace is mine --\\*
\vin \textit{Leaning on the everlasting arms.}\\!

{\itshape
Leaning, leaning, safe \& secure from all alarms;\\*
Leaning, leaning, leaning on the everlasting arms.}\\!

O how sweet to walk in this pilgrim way.\\*
O how bright the path grows from day to day.\\!

What have I to dread? What have I to fear?\\*
\vin \textit{Leaning on the everlasting arms.}\\
I have blessed peace with my Lord so near --\\*
\vin \textit{Leaning on the everlasting arms.}\\!

{\itshape
Leaning, leaning, safe \& secure from all alarms;\\*
Leaning, leaning, leaning on the everlasting arms.}
\end{verse}

\subsection{}

\blfootnote{Anonymous, \cite{treasury}.}\settowidth{\versewidth}{My love in her attire doth show her wit,}
\begin{verse}[\versewidth]
My love in her attire doth show her wit,\\*
\vin It doth so well become her:\\
For every season she hath dressings fit,\\
\vin For winter, spring \& summer.\\
No beauty she doth miss\\
When all her robes are on;\\
But beauty's self she is\\*
When all her robes are gone.
\end{verse}

\subsection{}

\blfootnote{Sir Thomas Browne (1605 -- 1682), \cite{odq}.}There is surely a piece of divinity in us, something that was before the elements, and owes no homage unto the sun.

\section{}

\subsection{}

\blfootnote{`Faith's Review and Expectation', The Rev Dr John Newton (1725 -- 1807), \cite{olney}. A great mass of hagiography surrounds this beautiful hymn, most of it worthless. The Rev Dr Newton's own note indicates that it was inspired chiefly by II Chronicles 17.16-17, which concerns God's promises to King David and his descendants. \P 25. This last verse was not written by The Rev Dr Newton, nor is it included in the original \textit{Olney Hymns}. Its first appearance in print was probably in \textit{A Collection of Sacred Ballads} (1790), as one of the verses to ``Jerusalem, My Happy Home''. It passed from there into the black American oral tradition, and its inclusion in \textit{Uncle Tom's Cabin}, now transplanted into ``Amazing Grace'', cemented it in the popular imagination.}\begin{center}
\textit{Tune: New Britain}
\end{center}

\settowidth{\versewidth}{The earth shall soon dissolve like snow,}
\begin{verse}[\versewidth]
Amazing grace (how sweet the sound)\\*
\vin That saved a wretch like me!\\
I once was lost, but now am found,\\*
\vin Was blind, but now I see.\\!

'Twas grace that taught my heart to fear,\\*
\vin And grace my fears relieved;\\
How precious did that grace appear\\*
\vin The hour I first believed!\\!

Through many dangers, toils \& snares,\\*
\vin I have already come;\\
'Tis grace hath brought me safe thus far,\\*
\vin And grace will lead me home.\\!

The Lord has promised good to me;\\*
\vin His word my hope secures;\\
He will my shield \& portion be\\*
\vin As long as life endures.\\!

Yes, when this flesh \& heart shall fail,\\*
\vin And mortal life shall cease,\\
I shall possess, within the veil,\\*
\vin A life of joy \& peace.\\!

The earth shall soon dissolve like snow,\\*
\vin The sun forbear to shine;\\
But God, who called me here below,\\*
\vin Will be forever mine.\\!

When we've been there 10,000 years,\\*
\vin Bright shining as the sun,\\
We've no less days to sing God's praise,\\*
\vin Than when we first begun.
\end{verse}

\subsection{}

\blfootnote{Anonymous, \cite{newlove}. This poem is often attributed to the Very Rev Dr Donne, with some other verses (sadly of inferior quality) affixed. However, Prof Stallworthy contests this, attributing the poem to `Anon' instead. Prof Sir Herbert Grierson suggests the true author may have been a John Dowland.}\settowidth{\versewidth}{The day breaks not: it is my heart,}
\begin{verse}[\versewidth]
Stay, O sweet, and do not rise.\\*
The light that shines comes from thine eyes:\\
The day breaks not: it is my heart,\\
Because that you \& I must part.\\
\vin Stay, or else my joys will die\\*
\vin And perish in their infancy.
\end{verse}

\subsection{}

\blfootnote{The Rev Robert Burton (1577 -- 1640), \cite{odq}.}To enlarge or illustrate this power and effect of love is to set a candle to the sun.

\section{}

\subsection{}

\blfootnote{William Blake (1757 -- 1827), \cite{miltonpoem}. Blake's original poem -- actually a snippet from a larger work -- is a sarcastic retelling of the (eccentric) theory that Jesus and Mary visited England with Joseph of Arimathea. The stirring, patriotic tune to which Parry set it is thus profoundly incongruous with Blake's intentions -- though no less rousing for any English heart. \P 8. Oceans of ink and forests of paper have been expended trying to identify what exactly Blake meant by `these dark satanic mills'. But it seems impossible that Blake could have been referring to anything other than the factories of the nascent industrial revolution, the coming of which he so dreaded.}\begin{center}
\textit{Tune: Jerusalem}
\end{center}

\settowidth{\versewidth}{    Shine forth upon our clouded hills?}
\begin{verse}[\versewidth]
And did those feet in ancient time,\\*
\vin Walk upon England's mountains green?\\
And was the holy Lamb of God\\
\vin On England's pleasant pastures seen?\\
And did the countenance divine\\
\vin Shine forth upon our clouded hills?\\
And was \textsc{Jerusalem} builded here,\\*
\vin Among these dark satanic mills?\\!

Bring me my bow of burning gold:\\*
\vin Bring me my arrows of desire:\\
Bring me my spear: O clouds, unfold:\\
\vin Bring me my chariot of fire!\\
I will not cease from mental fight,\\
\vin Nor shall my sword sleep in my hand:\\
Till we have built \textsc{Jerusalem},\\*
\vin In England's green \& pleasant Land.
\end{verse}

\subsection{}

\blfootnote{Anonymous, \cite{oxfordlocal}. Prof Holloway labels this `A Berkshire rhyme'.}\settowidth{\versewidth}{Whistle, daughter, whistle, and you shall have a sheep.}
\begin{verse}[\versewidth]
Whistle, daughter, whistle, and you shall have a sheep.\\*
Mother, I cannot whistle, neither can I sleep.\\
Whistle, daughter, whistle, and you shall have a cow.\\
Mother, I cannot whistle, neither now I how.\\
Whistle, daughter, whistle, and you shall have a man.\\*
Mother, I cannot whistle, but I'll do the best I can.
\end{verse}

\subsection{}

\blfootnote{William Byrd (1543 -- 1623), \cite{odq}.}The exercise of singing is delightful to nature, and good to preserve the health of man.

\section{}

\subsection{}

\blfootnote{John Bunyan (1628 -- 1688), \cite{bunyan}. These words can be found towards the end of the second part of Bunyan's magnum opus. A Mr Valiant-for-Truth sings this song to the party after they've left the Delectable Mountains.}\begin{center}
\textit{Tune: Monk's Gate}
\end{center}

\settowidth{\versewidth}{Do but themselves confound;}
\begin{verse}[\versewidth]
Who would true valor see,\\*
Let him come hither;\\
One here will constant be,\\
Come wind, come weather.\\
There's no discouragement\\
Shall make him once relent\\
His first avowed intent\\*
To be a pilgrim.\\!

Whoso beset him round\\*
With dismal stories,\\
Do but themselves confound;\\
His strength the more is.\\
No lion can him fright,\\
He'll with a giant fight,\\
But he will have a right\\*
To be a pilgrim.\\!

Hobgoblin nor foul fiend\\*
Can daunt his spirit;\\
He knows he at the end\\
Shall life inherit.\\
Then fancies fly away,\\
He'll not fear what men say;\\
He'll labour night \& day\\*
To be a pilgrim.
\end{verse}

\subsection{}

\blfootnote{`Juliet', Hilaire Belloc (1870 -- 1953), \cite{newlove}.}\settowidth{\versewidth}{How did the party go in Portman Square?}
\begin{verse}[\versewidth]
How did the party go in \textsc{Portman Square}?\\*
I cannot tell you; \textit{Juliet} was not there.\\
And how did Lady \textit{Gaster}'s party go?\\*
\textit{Juliet} was next me and I do not know.
\end{verse}

\subsection{}

\blfootnote{George Noel, 6th Baron Byron (1788 -- 1824), \cite{odq}. This is a couplet from the second canto of \refbook{Don Juan}.}\settowidth{\versewidth}{Let us have wine women, mirth laughter,}
\begin{verse}[\versewidth]
Let us have wine \& women, mirth \& laughter,\\*
Sermons \& soda water the day after.
\end{verse}

\section{}

\subsection{}

\blfootnote{Reginald Heber, Bishop of Calcutta (1783 -- 1826), \cite{tmwwbk}. Sean Connery and Michael Caine sing this song during the tragic climax of the 1975 adaptation of Kipling's \refbook{The Man Who Would Be King}. The original text also features the hymn, albeit with reworked lyrics and in a different context.}\begin{center}
\textit{Tune: The Minstrel Boy}
\end{center}

\settowidth{\versewidth}{They bowed their heads the death to feel:}
\begin{verse}[\versewidth]
The Son of God goes forth to war,\\*
\vin A kingly crown to gain;\\
His blood-red banner streams afar:\\*
\vin Who follows in his train?\\!

A glorious band, the chosen few\\*
\vin On whom the Spirit came;\\
Twelve valiant saints, their hope they knew,\\*
\vin And mocked the cross \& flame.\\!

They met the tyrant's brandished steel,\\*
\vin The lion's gory mane;\\
They bowed their heads the death to feel:\\*
\vin Who follows in their train?
\end{verse}

\subsection{}

\blfootnote{`The Shepherd', William Blake (1757 -- 1827), \cite{blakea}.}\settowidth{\versewidth}{    For they know when their shepherd is nigh.}
\begin{verse}[\versewidth]
How sweet is the shepherd's sweet lot;\\*
\vin From the morn to the evening he strays;\\
He shall follow his sheep all the day,\\*
\vin And his tongue shall be filled with praise.\\!

For he hears the lamb's innocent call,\\*
\vin And he hears the ewe's tender reply;\\
He is watchful while they are in peace,\\*
\vin For they know when their shepherd is nigh.
\end{verse}

\subsection{}

\blfootnote{George Noel, 6th Baron Byron (1788 -- 1824), \cite{odq}. These words are from a letter to John Murray. The `part of the world' Lord Byron had in mind seems to have been Venice.}Love in this part of the world is no sinecure.

\section{}

\subsection{}

\blfootnote{The Rev Prof John Keble (1792 -- 1866), \cite{lyra}. Known as Φῶς ἱλαρὸν in the original Greek, this is probably the earliest surviving Christian hymn. Indeed, St Basil the Great (d 379) wrote in his Περί του Αγίου Πνεύματος 29.73: `We cannot say to who was the father of those expressions in the thanksgiving at the lighting of the lamps; but it is an ancient formula which people repeat, and no one has ever yet been accused of impiety for saying, ``We hymn the Father and the Son and the Holy Spirit of God.{''}' Anyone with a functioning pair of ears ought to sing these word's to Sir John Stainer's tune `Sebaste', and not to Charles Wood's polyphonic atrocity. \P 10. The `they' in the last line -- apparently this is much clearer in the original Greek -- refers to all the individual component of God's creation.}\begin{center}
\textit{Tune: Sebaste}
\end{center}

\settowidth{\versewidth}{Hail, gladdening light, of his pure glory poured,}
\begin{verse}[\versewidth]
Hail, gladdening light, of his pure glory poured,\\*
Who is the immortal Father, heavenly, blest,\\*
Holiest of holies, \textit{Jesus Christ} our Lord.\\!

Now we are come to the sun's hour of rest,\\*
The lights of evening round us shine.\\*
We hymn the Father, Son and Holy Spirit divine.\\!

Worthiest art thou at all times to be sung\\*
With undefil{\`{e}}d tongue,\\
Son of our God, giver of life alone!\\*
Therefore in all the world, thy glories, Lord, they own.
\end{verse}

\subsection{}

\blfootnote{`Farewell to Juliet', Wilfrid Blunt (1840 -- 1922), \cite{oxfordlarkin}. This is from Blunt's somewhat disappointing (in the light of the brilliance of this poem) series, \refbook{Love Sonnets of Proteus}.}\settowidth{\versewidth}{    Loaded with vines, and with your dear pale face,}
\begin{verse}[\versewidth]
I see you, \textit{Juliet}, still, with your straw hat\\*
\vin Loaded with vines, and with your dear pale face,\\
On which those 30 years so lightly sat,\\
\vin And the white outline of your muslin dress.\\
You wore a little {\hoskeroe fichu} trimmed with lace\\
\vin And crossed in the front, as was the fashion then,\\
Bound at your waist with a broad band or sash,\\
\vin All white \& fresh \& virginally plain.\\
There was a sound of shouting far away\\
\vin Down in the valley, as they called to us,\\
And you, with hands clasped seeming still to pray\\
\vin Patience of fate, stood listening to me thus\\
With heaving bosom. There a rose lay curled.\\*
It was the reddest rose in all the world.
\end{verse}

\subsection{}

\blfootnote{George Noel, 6th Baron Byron (1788 -- 1824), \cite{odq}. This is a couplet from the third canto of \refbook{Don Juan}.}\settowidth{\versewidth}{Think you, if Laura had been Petrarch's wife,}
\begin{verse}[\versewidth]
Think you, if \textit{Laura} had been \textit{Petrarch}'s wife,\\*
He would have written sonnets all his life?
\end{verse}

\section{}

\subsection{}

\blfootnote{William Golden (1878 -- 1934), \cite{rusby_littlelights}.}\begin{center}
\textit{Tune: To Canaan's Land}
\end{center}

\settowidth{\versewidth}{It shines to light the shores of home.}
\begin{verse}[\versewidth]
To Canaan's land I'm on my way.\\*
\vin \textit{Where the soul of man never dies.}\\
My darkest night will turn to day.\\*
\vin \textit{Where the soul of man never dies.}\\!

{\itshape
Dear friends, there'll be no sad farewells;\\
There'll be no tear-dimmed eyes,\\
Where all is peace \& joy \& love,\\*
And the soul of man never dies.}\\!

A rose is blooming there for me.\\*
It blooms for all eternity.\\!

A love light beams across the foam.\\*
It shines to light the shores of home.\\!

My life will end in deathless sleep,\\*
And everlasting joys I'll reap.\\!

I'm on my way to that fair land,\\*
\vin \textit{Where the soul of man never dies.}\\
Where there will be no parting hand.\\*
\vin \textit{Where the soul of man never dies.}\\!

{\itshape
Dear friends, there'll be no sad farewells;\\
There'll be no tear-dimmed eyes,\\
Where all is peace \& joy \& love,\\*
And the soul of man never dies.}
\end{verse}

\subsection{}

\blfootnote{`To My Dear and Loving Husband', Mrs Anne Bradstreet (1612 -- 1672), \cite{norton}.}\settowidth{\versewidth}{Then while we live, in love let's so persever,}
\begin{verse}[\versewidth]
If ever two were one, then surely we.\\*
If ever man were loved by wife, then thee.\\
If ever wife was happy in a man,\\
Compare with me, ye women, if you can.\\
I prize thy love more than whole mines of gold,\\
Or all the riches that the east doth hold.\\
My love is such that rivers cannot quench,\\
Nor ought but love from thee give recompense.\\
Thy love is such I can no way repay;\\
The heavens reward thee manifold, I pray.\\
Then while we live, in love let's so persever,\\*
That when we live no more, we may live ever.
\end{verse}

\subsection{}

\blfootnote{George Noel, 6th Baron Byron (1788 -- 1824), \cite{odq}. This is a couplet from the first canto of \refbook{Don Juan}.}\settowidth{\versewidth}{What men call gallantry, and gods adultery,}
\begin{verse}[\versewidth]
What men call gallantry, and gods adultery,\\*
Is much more common when the cllimate's sultry.
\end{verse}

\section{}

\subsection{}

\blfootnote{Wallace Willis (1820 -- 1880), \cite{cabinfever}. Willis's version has two further verses.}\begin{center}
\textit{Tune: Swing Low}
\end{center}

\settowidth{\versewidth}{A band of angels coming after me.}
\begin{verse}[\versewidth]
{\itshape
Swing low, sweet chariot,\\
Coming for to carry me home!\\
Swing low, sweet chariot,\\*
Coming for to carry me home!}\\!

I looked over \textsc{Jordan}, and what did I see?\\*
\vin \textit{Coming for to carry me home!}\\
A band of angels coming after me.\\*
\vin \textit{Coming for to carry me home!}\\!

If you get there before I do,\\*
\vin \textit{Coming for to carry me home!}\\
Tell all my friends I coming too.\\*
\vin \textit{Coming for to carry me home!}\\!

{\itshape
Swing low, sweet chariot,\\
Coming for to carry me home!\\
Swing low, sweet chariot,\\*
Coming for to carry me home!}
\end{verse}

\subsection{}

\blfootnote{`Meeting at Night', Robert Browning (1828 -- 1889), \cite{newlove}.}\settowidth{\versewidth}{Three fields to cross till a farm appears;}
\begin{verse}[\versewidth]
The grey sea \& the long black land;\\*
And the yellow \sfrac{$1$}{$2$} moon large \& low;\\
And the startled little waves that leap\\
In fiery ringlets from their sleep,\\
As I gain the cove with pushing prow,\\*
And quench its speed i' the slushy sand.\\!

Then a mile of warm sea-scented beach;\\*
Three fields to cross till a farm appears;\\
A tap at the pane, the quick sharp scratch\\
And blue spurt of a lighted match,\\
And a voice less loud, through its joys \& fears,\\*
Than the two hearts beating each to each.
\end{verse}

\subsection{}

\blfootnote{George Cadbury (1839 -- 1922), \cite{odq}.}No man ought to be compelled to live where a rose cannot grow.

\section{}

\subsection{}

\blfootnote{Josiah Alwood (1828 -- 1909), \cite{myallandall}. Alwood wrote of the circumstances leading to his composition of this song: `It was a balmy night in August 1879, when returning from a debate in Spring Hill, Ohio, to my home in Morenci, Michigan, about 1:00 a.m. I saw a beautiful rainbow north by northwest against a dense black nimbus cloud. The sky was all perfectly clear except this dark cloud which covered about forty degrees of the horizon and extended about halfway to the zenith. The phenomenon was entirely new to me and my nerves refreshed by the balmy air and the lovely sight. Old Morpheus was playing his sweetest lullaby. Another mile of travel, a few moments of time, a fellow of my size was ensconced in sweet home and wrapped in sweet sleep. A first class know-nothing till rosy-sweet morning was wide over the fields.'}\begin{center}
\textit{Tune: The Unclouded Day}
\end{center}

\settowidth{\versewidth}{And they tell me that no tears will ever come again}
\begin{verse}[\versewidth]
O they tell me of a home far beyond the sky.\\*
O they tell me of a home far away.\\
They tell me of a home where no storm clouds rise.\\*
O they tell me of an uncloudy day.\\!

{\itshape
O the land of cloudless day!\\
O the land of an uncloudy sky!\\
O they tell me of a home where no storm clouds rise!\\*
O they tell me of an uncloudy day!}\\!

O they tell me of a home where my friends have gone.\\*
O they tell me of that land far away,\\
Where the tree of life in eternal bloom\\*
Sheds its fragrance through the uncloudy day.\\!

O they tell me that he smiles on his children there,\\*
And his smile drives their sorrows away.\\
And they tell me that no tears will ever come again\\*
In that lovely land of uncloudy day.\\!

{\itshape
O the land of cloudless day!\\
O the land of an uncloudy sky!\\
O they tell me of a home where no storm clouds rise!\\*
O they tell me of an uncloudy day!}
\end{verse}

\subsection{}

\blfootnote{George Noel, 6th Baron Byron (1788 -- 1824), \cite{treasury}.}\settowidth{\versewidth}{There be none of beauty's daughters}
\begin{verse}[\versewidth]
There be none of beauty's daughters\\*
\vin With a magic like thee;\\
And like music on the waters\\
\vin Is thy sweet voice to me:\\
When, as if its sound were causing\\
The charm\`{e}d ocean's pausing,\\
The waves lie still \& gleaming,\\*
And the lulled winds seem dreaming:\\!

And the midnight moon is weaving\\*
\vin Her bright chain o'er the deep,\\
Whose breast is gently heaving\\
\vin As an infant's asleep:\\
So the spirit bows before thee\\
To listen \& adore thee;\\
With a full but soft emotion,\\*
Like the swell of summer's ocean.
\end{verse}

\subsection{}

\blfootnote{Philip Stanhope, 4th Earl of Chesterfield (1694 -- 1773), \cite{odq}. In the original text, `it' is a pocket watch -- used as an analogy for being modest about one's education -- although Lord Chesterfield may have had something else in mind.}Do not merely pull it out and strike it, merely to show that you have one.

\section{}

\subsection{}

\blfootnote{`The Battle Cry of Freedom', Dr George Root (1820 -- 1895), \cite{carolina2nd_hard}.}\begin{center}
\textit{Tune: The Battle Cry of Freedom}
\end{center}

\settowidth{\versewidth}{And we'll fill their vacant ranks with a 1,000,000 freemen more.}
\begin{verse}[\versewidth]
O we'll rally round the flag, boys, we'll rally once again --\\*
\vin \textit{Shouting the battle cry of freedom!}\\
And we'll rally from the hillside; we'll gather from the plain --\\*
\vin \textit{Shouting the battle cry of freedom!}\\!

{\itshape
The Union forever! Hurrah, boys! Hurrah!\\
Down with the traitor, and up with the stars!\\
And we'll rally round the flag, boys, we'll rally once again --\\*
Shouting the battle cry of freedom!}\\!

O we're springing to the ranks of our brothers gone before,\\*
And we'll fill their vacant ranks with a 1,000,000 freemen more.\\!

We will welcome to our numbers the loyal, true \& brave;\\*
And although he may be poor, he shall never be a slave.\\!

We are springing to the call from the east \& from the west --\\*
\vin \textit{Shouting the battle cry of freedom!}\\
And we'll hurl the rebel crew from the land we love the best --\\*
\vin \textit{Shouting the battle cry of freedom!}\\!

{\itshape
The Union forever! Hurrah, boys! Hurrah!\\
Down with the traitor, and up with the stars!\\
And we'll rally round the flag, boys, we'll rally once again --\\*
Shouting the battle cry of freedom!}
\end{verse}

\subsection{}

\blfootnote{William Corkine (? -- ?), \cite{londonbook}. This is number VII in the first of Corkine's books of \refbook{Ayres}.}\settowidth{\versewidth}{    Which once too ripe will never rise,}
\begin{verse}[\versewidth]
Sweet \textit{Cupid}, ripen her desire\\*
\vin Thy joyful harvest may begin;\\
If age approach a little higher,\\*
\vin 'Twill be too late to get it in.\\!

Cold winter storms lay standing corn,\\*
\vin Which once too ripe will never rise,\\
And lovers wish themselves unborn,\\*
\vin When all their joys lie in their eyes.\\!

Then, sweet, let us embrace and kisse.\\*
\vin Shall beauty shale upon the ground?\\
If age bereave us of this bliss,\\*
\vin Then will no more such sport be found.
\end{verse}

\subsection{}

\blfootnote{Philip Stanhope, 4th Earl of Chesterfield (1694 -- 1773), \cite{odq}.}Ridicule is the best test of truth.

\section{}

\subsection{}

\blfootnote{Stephen Foster (1826 -- 1864), \cite{oldtime}. The folk process is said to have had its way with Foster's original version.}\begin{center}
\textit{Tune: Angeline the Baker}
\end{center}

\settowidth{\versewidth}{Angelina Baker! Angelina Baker's gone!}
\begin{verse}[\versewidth]
Way down on the old plantation,\\*
\vin That's where I was born;\\
I used to beat the whole creation\\
\vin Hoeing in the corn.\\
O then I work, and then I sing,\\
\vin So happy all the day,\\
Till \textit{Angelina Baker} come\\*
\vin And stole my heart away.\\!

{\itshape
\textit{Angelina Baker}! \textit{Angelina Baker}'s gone!\\
\vin She left me here\\
\vin To weep a tear\\*
And beat on the old jawbone.}\\!

Early in the morning\\*
\vin Of a lovely summer's day,\\
I ax for \textit{Angelina}\\
\vin And they say she's gone away.\\
I don't know where to find her\\
\vin 'Cause I don't know where she's gone.\\
She left me here to weep a tear\\*
\vin And beat on the old jawbone.\\!

I've seen her in the springtime,\\*
\vin And I've seen her in the fall.\\
I've seen her in the cornfield,\\
\vin And I've seen her at the ball;\\
And every time I seen her\\
\vin She was smiling like the sun,\\
But now I'm left to weep a tear\\*
\vin 'Cause \textit{Angelina}'s gone.\\!

{\itshape
\textit{Angelina Baker}! \textit{Angelina Baker}'s gone!\\
\vin She left me here\\
\vin To weep a tear\\*
And beat on the old jawbone.}
\end{verse}

\subsection{}

\blfootnote{$\mathbb{R}$ Sidney Godolphin (1610 -- 1643), \cite{obev}. These lines drawn from a longer poem, called \refpoem{A Ballet}. These lines in particular are the shepherd's speech to the nymph Amarillis.}\settowidth{\versewidth}{Thou safest of pleasures}
\begin{verse}[\versewidth]
Thou joy of my life,\\*
First love of my youth,\\
Thou safest of pleasures\\
And fullest of truth,\\
Thou purest of nymphs\\
And never more fair,\\
Breathe this way and cool me,\\
Thou pitying air;\\
Come hither and hover\\
On every part,\\
Thou life of my sense\\*
And joy of my heart.
\end{verse}

\subsection{}

\blfootnote{John Clare (1793 -- 1864), \cite{odq}. These are two lines from Clare's poem \refpoem{The Dying Child}.}\settowidth{\versewidth}{He could not die when the trees were green,}
\begin{verse}[\versewidth]
He could not die when the trees were green,\\*
For he loved the time too well.
\end{verse}

\section{}

\subsection{}

\blfootnote{Anonymous, \cite{animals}. Although strongly associated with the Animals' rendition thereof, this song was probably composed in the late nineteenth century, and has roots going back to the seventeenth. The house in question was likely a bordello of some description, though no brothel of that name has been shown to have existed in nineteenth century New Orleans.}\begin{center}
\textit{Tune: The House of the Rising Sun}
\end{center}

\settowidth{\versewidth}{And the only time that he is satisfied}
\begin{verse}[\versewidth]
There is a house in \textsc{New Orleans}\\*
They call \textsc{The Rising Sun},\\
And it's been the ruin of many a poor boy,\\*
And, God, I know I'm one.\\!

My mother was a tailor.\\*
She sewed my new blue jeans.\\
My father was a gambling man\\*
Down in \textsc{New Orleans}.\\!

Now the only thing a gambler needs\\*
Is a suitcase \& a trunk,\\
And the only time that he is satisfied\\*
Is when he's on a drunk.\\!

O mother, tell your children\\*
Not to do what I have done,\\
Spend your lives in sin \& misery\\*
In \textsc{The House of the Rising Sun}.\\!

I got one foot on the platform,\\*
The other foot on the train.\\
I'm going back to \textsc{New Orleans}\\*
To wear that ball \& chain.\\!

There is a house in \textsc{New Orleans}\\*
They call \textsc{The Rising Sun},\\
And it's been the ruin of many a poor boy,\\*
And, God, I know I'm one.
\end{verse}

\subsection{}

\blfootnote{`The Poetry of Dress', Robert Herrick (1591 -- 1674), \cite{treasury}.}\settowidth{\versewidth}{Kindles in clothes a wantonness:--}
\begin{verse}[\versewidth]
A sweet disorder in the dress\\*
Kindles in clothes a wantonness:--\\
A lawn about the shoulders thrown\\
Into a fine distraction;\\
An erring lace, which here \& there\\
Enthrals the crimson stomacher;\\
A cuff neglectful, and thereby\\
Ribbands to flow confusedly;\\
A winning wave, deserving note,\\
In the tempestuous petticoat;\\
A careless shoestring, in whose tie\\
I see a wild civility --\\
Do more bewitch me, than when art\\*
Is too precise in every part.
\end{verse}

\subsection{}

\blfootnote{John Cleveland (1613 -- 1658), \cite{odq}.}\settowidth{\versewidth}{Had Cain been scot, God would have changed his doom:}
\begin{verse}[\versewidth]
Had \textit{Cain} been scot, God would have changed his doom:\\*
Nor forced him wander, but confined him home.
\end{verse}

\section{}

\subsection{}

\blfootnote{Stephen Foster (1826 -- 1864), \cite{carolina2nd_lightning}.}\begin{center}
\textit{Tune: Ring Ring de Banjo}
\end{center}

\settowidth{\versewidth}{    That he'd like to hear me play.}
\begin{verse}[\versewidth]
O never count the bubbles\\*
\vin When there's water in the spring.\\
A darkie has no troubles\\
\vin When he's got a song to sing.\\
The beauties of creation\\
\vin Will never lose their charm\\
While a roam the old plantation\\*
\vin With my truelove on my arm.\\!

{\itshape
Ring, ring the banjo!\\
\vin I like that good old song!\\
Come again, my truelove!\\*
\vin O where you been so long?!}\\!

Well the time is never dreary\\*
\vin If a darkie never groans.\\
The ladies never weary\\
\vin With a rattle of the bones.\\
Then come again, \textit{Susannah},\\
\vin By the gaslight of the moon.\\
I'll tum that old piano\\*
\vin When the banjo's out of tune.\\!

O once I was so lucky\\*
\vin My mas'er set me free.\\
So I went to old Kentucky\\
\vin For to see what I could see.\\
I could not go no farder,\\
\vin And I turned to mas'er's door.\\
I'll love him all the harder,\\*
\vin And I'll go away no more.\\!

Well, early in the morning\\*
\vin Of a lovely summer's day,\\
My mas'er gave me warning\\
\vin That he'd like to hear me play.\\
On the banjo I was tapping,\\
\vin And I come with dulcet string;\\
My mas'er falled a-napping\\*
\vin And he'll never wake again.\\!

My love, I'll have to leave you\\*
\vin While the river's running high,\\
But I never can deceive you,\\
\vin So don't you wipe your eye.\\
I'se gonna make some money,\\
\vin But I'll come another day.\\
I'll come again, my honey,\\*
\vin If I have to work my way.\\!

{\itshape
Ring, ring the banjo!\\
\vin I like that good old song!\\
Come again, my truelove!\\*
\vin O where you been so long?!}
\end{verse}

\subsection{}

\blfootnote{`The Argument of His Book', Robert Herrick (1591 -- 1674), \cite{norton}.}\settowidth{\versewidth}{Of bridegrooms, brides, of their bridal-cakes.}
\begin{verse}[\versewidth]
I sing of brooks, of blossoms, birds, \& bowers,\\*
Of april, may, of june, \& july flowers.\\
I sing of may-poles, hock-carts, wassails, wakes,\\
Of bridegrooms, brides, \& of their bridal-cakes.\\
I write of youth, of love, and have access\\
By these to sing of cleanly wantonness.\\
I sing of dews, of rains, and piece by piece\\
Of balm, of oil, of spice, \& ambergris.\\
I sing of time's trans-shifting; and I write\\
How roses first came red, \& lilies white.\\
I write of groves, of twilights, and I sing\\
The court of \textit{Mab}, \& of the fairy king.\\
I write of hell; I sing (and ever shall)\\*
Of heaven, and hope to have it after all.
\end{verse}

\subsection{}

\blfootnote{John Constable (1776 -- 1837), \cite{odq}.}I never saw an ugly thing in my life.

\section{}

\subsection{}

\blfootnote{`Dixie', Dan Emmett (1815 -- 1904), \cite{carolina2nd_southern}. This song served as one of several de facto national anthems for the Confederacy, which provoked one Union songwriter to compose a rather good parody: `Away down South in the land of traitors,/ Rattlesnakes and alligators,/ Right away, come away, right away, come away./ Where cotton's king and men are chattels,/ Union boys will win the battles,/ Right away, come away, right away, come away.' \P 18. The Rifled Breech-Loading 40 pounder Armstrong gun was first made in 1860, the year this song was first performed.}\begin{center}
\textit{Tune: Dixie}
\end{center}

\settowidth{\versewidth}{His face was sharp as a butcher's cleaver,}
\begin{verse}[\versewidth]
I wish I was in the land of cotton --\\*
Old times there are not forgotten --\\
\vin \textit{Look away! Look away! Look away, Dixie-land!}\\
In Dixie's land where I was born in,\\
Early on one frosty mornin'\\*
\vin \textit{Look away! Look away! Look away, Dixie-land!}\\!

{\itshape
I wish I was in Dixie!\\
Hooray! Hooray!\\
In Dixie's land I'll take my stand\\
To live and die in Dixie.\\
Away! Away!\\
Away down south in Dixie!\\
Away! Away!\\*
Away down south in Dixie!}\\!

\textit{Old Missus} marry \textit{Will} the weaver.\\*
\textit{William} was a gay deceiver,\\
And when he put his arm around her\\*
He smiled as fierce as a 40 pounder.\\!

His face was sharp as a butcher's cleaver,\\*
But that did not seem to grieve her.\\
\textit{Old Missus} acted the foolish part.\\*
She died for a man that broke her heart.\\!

Now here's a health to the next \textit{Old Missus}\\*
And all the girls that want to kiss us!\\
\vin \textit{Look away! Look away! Look away, Dixie-land!}\\
And if you want to drive away sorrow,\\
Come back hear our song tomorrow.\\*
\vin \textit{Look away! Look away! Look away, Dixie-land!}\\!

{\itshape
I wish I was in Dixie!\\
Hooray! Hooray!\\
In Dixie's land I'll take my stand\\
To live and die in Dixie.\\
Away! Away!\\
Away down south in Dixie!\\
Away! Away!\\*
Away down south in Dixie!}
\end{verse}

\subsection{}

\blfootnote{$\mathbb{R}$ `The comming of good luck', Robert Herrick (1591 -- 1674), \cite{obev}.}\settowidth{\versewidth}{So good luck came, and on my roof did light,}
\begin{verse}[\versewidth]
So good luck came, and on my roof did light,\\*
Like noiseless snow, or as the dew of night:\\
Not all at once, but gently, as the trees\\*
Are by the sunbeams tickled by degrees.
\end{verse}

\subsection{}

\blfootnote{Mrs Beatrice Cornwallis-West (1865 -- 1940), \cite{odq}.}It doesn't matter what you do in the bedroom as long as you don't do it in the street and frighten the horses.

\section{}

\subsection{}

\blfootnote{Anonymous, \cite{spinners}. \P 15. Ackee and salt fish (the former being a peculiar kind of fruit, and the latter being salted cod) is a classic Jamaican recipe.}\begin{center}
\textit{Tune: Jamaica Farewell}
\end{center}

\settowidth{\versewidth}{    And the sun shines daily on the mountain top,}
\begin{verse}[\versewidth]
Far away where the nights are gay\\*
\vin And the sun shines daily on the mountain top,\\
I took a trip on a sailing ship\\*
\vin And when I reached Jamaica I made a stop.\\!

{\itshape
But I'm sad to say, I'm on my way;\\
\vin Won't be back for many a day.\\
My heart is down; my head is turning around.\\*
\vin I had to leave a little girl in \textsc{Kingston} town.}\\!

Sounds of laughter everywhere,\\*
\vin And the dancing girls sway to \& fro:\\
I must declare my heart is there,\\*
\vin Though I've been from Maine to Mexico.\\!

Down the market you can hear it:\\*
\vin Ladies cry out while on their heads they bear:\\
Ackee, rice, salt fish is nice,\\*
\vin Though the rum is fine any time of year.\\!

{\itshape
But I'm sad to say, I'm on my way;\\
\vin Won't be back for many a day.\\
My heart is down; my head is turning around.\\*
\vin I had to leave a little girl in \textsc{Kingston} town.}
\end{verse}

\subsection{}

\blfootnote{`Hymn to Diana', Ben Jonson (1572 -- 1637), \cite{treasury}.}\settowidth{\versewidth}{    Heaven to clear when day did close:}
\begin{verse}[\versewidth]
Queen \& huntress, chaste \& fair,\\*
\vin Now the sun is laid to sleep,\\
Seated in thy silver chair\\
\vin State in wonted manner keep:\\
\vin \vin \textit{Hesperus} entreats thy light,\\*
\vin \vin Goddess excellently bright.\\!

Earth, let not thy envious shade\\*
\vin Dare itself to interpose;\\
\textit{Cynthia}'s shining orb was made\\
\vin Heaven to clear when day did close:\\
\vin \vin Bless us then with wish\`{e}d sight,\\*
\vin \vin Goddess excellently bright.\\!

Lay thy bow of pearl apart\\*
\vin And thy crystal-shining quiver;\\
Give unto the flying hart\\
\vin Space to breathe, how short soever:\\
\vin \vin Thou that mak'st a day of night,\\*
\vin \vin Goddess excellently bright!
\end{verse}

\subsection{}

\blfootnote{John Dryden, Poet Laureate (1631 -- 1700), \cite{odq}. This is a line from \refbook{The Secular Masque}.}Joy ruled the day, and love the night.

\section{}

\subsection{}

\blfootnote{`The Bonnie Blue Flag', Harry McCarthy (1834 -- 1888), \cite{carolina2nd_cotton}.}\begin{center}
\textit{Tune: The Bonnie Blue Flag}
\end{center}

\settowidth{\versewidth}{We hoist on high the bonnie blue flag that bears a single star.}
\begin{verse}[\versewidth]
We are a band of brothers and native to the soil,\\*
Fighting for the property we gained by honest toil.\\
And when our rights were threatened, the cry rose near \& far:\\*
Hurrah for the bonnie blue flag that bears a single star!\\!

{\itshape
Hurrah! Hurrah! For southern rights, hurrah!\\*
Hurrah for the bonnie blue flag that bears a single star!}\\!

As long as the Union was faithful to her trust,\\*
Like friends and like brethren, kind were we, and just.\\
But now, when northern treachery attempts our rights to mar,\\*
We hoist on high the bonnie blue flag that bears a single star.\\!

First gallant South Carolina nobly made the stand,\\*
Then came Alabama and took her by the hand.\\
Next, quickly Mississippi, Georgia, Florida\\*
All raised on high the bonnie blue flag that bears a single star.\\!

Ye men of valour gather round the banner of the right.\\*
Texas \& fair Louisiana join us in the fight.\\
\textit{Davis}, our lov{\`{e}}d president, and \textit{Stephens} statesmen rare\\*
Now rally round the bonnie blue flag that bears a single star.\\!

Now here's to brave Virginia, the Old Dominion State,\\*
With the young Confederacy at last has sealed her fate;\\
And spurred by her example, now other states prepare\\*
To hoist on high the bonnie blue flag that bears a single star.\\!

Then cheer, boys, cheer. Raise a joyous shout.\\*
For Arkansas and North Carolina now have both gone out;\\
And let another rousing cheer for Tennessee be given,\\*
The single star of the bonnie blue flag has grown to be 11.\\!

Then here's to our Confederacy. Strong we are \& brave.\\*
Like patriots of old we'll fight, our heritage to save.\\
And rather than submit to shame, to die we would prefer.\\*
So cheer for the bonnie blue flag that bears a single star.\\!

{\itshape
Hurrah! Hurrah! For southern rights, hurrah!\\*
Hurrah for the bonnie blue flag that bears a single star!}
\end{verse}

\subsection{}

\blfootnote{`The Mower to the Glowworms', Andrew Marvell (1621 -- 1678), \cite{norton}. Prof Philip Larkin wrote a poem called \refpoem{The Mower}, which perhaps is a response to these lines.}\settowidth{\versewidth}{    Her matchless songs does meditate;}
\begin{verse}[\versewidth]
Ye living lamps, by whose dear light\\*
\vin The nightingale does sit so late,\\
And studying all the summer night,\\*
\vin Her matchless songs does meditate;\\!

Ye country comets, that portend\\*
\vin No war nor prince's funeral,\\
Shining unto no higher end\\*
\vin Than to presage the grass's fall;\\!

Ye glow-worms, whose officious flame\\*
\vin To wand'ring mowers shows the way,\\
That in the night have lost their aim,\\*
\vin And after foolish fires do stray;\\!

Your courteous lights in vain you waste,\\*
\vin Since \textit{Juliana} here is come,\\
For she my mind hath so displaced\\*
\vin That I shall never find my home.
\end{verse}

\subsection{}

\blfootnote{Julius Marx (1890 -- 1977), \cite{odq}. This a line from \refbook{The Importance of Being Earnest}.}Remember, you're fighting for this woman's honour... which is probably more than she ever did.

\section{}

\subsection{}

\blfootnote{Anonymous, \cite{llewyn}. John Lomax recorded in his \refbook{American Ballads and Folk Songs}: `{``}Dink knows all the songs,'' said her companion. But I did not find her helpful until I walked a mile to a farm commissary and bought her a pint of gin. As she drank the gin, the sounds from her scrubbing board increased in intensity and in volume. She worked as she talked: ``That little boy there ain't got no daddy an' he ain't got no name. I comes from Mississippi and we never saw these levee niggers, till us got here. I brung along my little boy. My man drives a four-wheel scraper down there where you see the dust risin'. I keeps his tent, cooks his vittles and washes his clothes. Some day Ize goin' to wrap up his wet breeches and shirts, roll 'em up in a knot, put 'em in the middle of the bed, and tuck down the covers right nice. Then I'm going on up the river where I belong.'' She sipped her gin and sang and drank until the bottle was empty.'}\begin{center}
\textit{Tune: Dink's Song}
\end{center}

\settowidth{\versewidth}{Fare thee well, my honey.}
\begin{verse}[\versewidth]
If I had wings\\*
\vin Like \textit{Noah}'s dove,\\
I'd fly the river\\*
\vin To the one I love.\\!

{\itshape
Fare thee well, my honey.\\*
Fare thee well.}\\!

I had a man.\\*
\vin He was long \& tall.\\
He moved his body\\*
\vin like a cannonball.\\!

I remember one evening\\*
\vin In the pouring rain,\\
And in my heart\\*
\vin Was an aching pain.\\!

Muddy river\\*
\vin Runs muddy \& wild.\\
Can't give a bloody\\*
\vin For my unborn child.\\!

Just as sure as a bird\\*
\vin Flying high above,\\
Life ain't worth living\\*
\vin Without the one you love.\\!

{\itshape
Fare thee well, my honey.\\*
Fare thee well.}
\end{verse}

\subsection{}

\blfootnote{`Love and Life', John Wilmot, 2nd Earl of Rochester (1647 -- 1680), \cite{newlove}.}\settowidth{\versewidth}{Like transitory dreams giv'n o'er,}
\begin{verse}[\versewidth]
All my past life is mine no more;\\*
\vin The flying hours are gone\\
Like transitory dreams giv'n o'er,\\
Whose images are kept in store\\*
\vin By memory alone.\\!

The time that is to come is not.\\*
\vin How can it then be mine?\\
The present moment's all my lot;\\
And that, as fast as it is got,\\*
\vin \textit{Phyllis}, is only thine.\\!

Then talk not of inconstancy,\\*
\vin False hearts \& broken vows;\\
If I, by miracle, can be\\
This live-long minute true to thee,\\*
\vin 'Tis all that heav'n allows.
\end{verse}

\subsection{}

\blfootnote{Dr William Butler (1535 -- 1618), \cite{odq}.}Doubtless God could have made a better berry, but doubtless God never did.

\section{}

\subsection{}

\blfootnote{Anonymous, \cite{carolina2nd_southern}. The origins of this song are unclear, although Dan Emmett claimed he wrote it as a boy. Certain sensitive persons have claimed these lines perpetuate stereotypes about the large apetites of black men for food, alcohol and women; but the Almanacker finds something genuinely noble in Dan Tucker's straightforward lust for life. \P 18. There is an English folk song -- well known on the other side of the Atlantic even in George Washington's day -- about a Derby ram which was `ten yards high', amongst other improbably large proportions.}\begin{center}
\textit{Tune: Ole Dan Tucker}
\end{center}

\settowidth{\versewidth}{And he died with a toothache in his heel.}
\begin{verse}[\versewidth]
I came to town the other night.\\*
I heard the noise. I saw the fight.\\
The watchman he was running around,\\*
Crying old \textit{Dan Tucker} had come to town.\\!

{\itshape
Get out the way!\\
Get out the way!\\
Get out the way, old \textit{Dan Tucker}!\\*
You're too late to get your supper!}\\!

Old \textit{Dan Tucker} was a mighty man.\\*
He washed his face in a frying pan.\\
He combed his hair with a wagon wheel,\\*
And he died with a toothache in his heel.\\!

Old \textit{Dan Tucker} is back in town,\\*
Swinging the ladies round \& round:\\
First to the right, then to the left,\\*
Then to the girl he liked the best.\\!

Old \textit{Dan Tucker} was a nice old man.\\*
He used to ride a Derby ram.\\
He sent him whizzing down the hill.\\*
If he hadn't got up, he'd lay there still.\\!

Old \textit{Dan Tucker} \& I got drunk,\\*
Fell in the fire, kicked up a chunk.\\
The charcoal got inside his shoe.\\*
Lord bless me, honey, how the ashes flew!\\!

I went to town to buy some goods.\\*
I lost myself in a piece of woods.\\
The night was dark. I had to suffer.\\*
I froze to the heel of \textit{Daniel Tucker}.\\!

\textit{Tucker} was a hardened sinner.\\*
He never said his grace at dinner.\\
The old sow squealed; the pigs did squall.\\*
He ate whole hog, tail \& all.\\!

{\itshape
Get out the way!\\
Get out the way!\\
Get out the way, old \textit{Dan Tucker}!\\*
You're too late to get your supper!}
\end{verse}

\subsection{}

\blfootnote{William Shakespeare (1564 -- 1616), \cite{treasury}.}\settowidth{\versewidth}{Love's not time's fool, though rosy lips cheeks}
\begin{verse}[\versewidth]
Let me not to the marriage of true minds\\*
Admit impediments. Love is not love\\
Which alters when it alteration finds,\\*
Or bends with the remover to remove:--\\!

O no! it is an ever-fix\`{e}d mark\\*
That looks on tempests, and is never shaken;\\
It is the star to every wandering bark,\\*
Whose worth's unknown, although his height be taken.\\!

Love's not time's fool, though rosy lips \& cheeks\\*
Within his bending sickle's compass come;\\
Love alters not with his brief hours \& weeks,\\*
But bears it out ev'n to the edge of doom:--\\!

If this be error, and upon me proved,\\*
I never writ, nor no man ever loved.
\end{verse}

\subsection{}

\blfootnote{Oscar Wilde (1854 -- 1900), \cite{odq}.}A little sincerity is a dangerous thing, and a great deal of it is absolutely fatal.

\section{}

\subsection{}

\blfootnote{Anonymous, \cite{springsteen}. Jesse James has gone down in folklore as a Confederate guerilla and American Robin Hood. In truth, like all the historical folk heroes -- Richard the Lionheart, Joan of Arc, Stalin -- James was little more than your common-or-garden murderer. But the truth should never get in the way of a good story. \P 11. `Mr Howard' was one of the false names James used whilst on the run.}\begin{center}
\textit{Tune: Jesse James}
\end{center}

\settowidth{\versewidth}{It was on a saturday night, and the moon was shining bright.}
\begin{verse}[\versewidth]
\textit{Jesse James} was a lad that killed many a man.\\*
\vin He robbed the \textsc{Glendale} train.\\
He stole from the rich and he gave to the poor.\\*
\vin He'd a hand \& a heart \& a brain.\\!

Well, it was \textit{Robert Ford}, that dirty little coward,\\*
\vin I wonder how he feels,\\
For he ate of \textit{Jesse}'s bread and he slept in \textit{Jesse}'s bed,\\*
\vin And he laid poor \textit{Jesse} in his grave.\\!

{\itshape
Well \textit{Jesse} had a wife to mourn for his life,\\
\vin Three children -- now they were brave.\\
That dirty little coward that shot Mr \textit{Howard}\\*
\vin Has laid \textit{Jesse James} in his grave.}\\!

\textit{Jesse} was a man, a friend to the poor.\\*
\vin He'd never rob a mother or a child.\\
There never was a man with the law in his hand,\\*
\vin That could take \textit{Jesse James} when alive.\\!

It was on a saturday night, and the moon was shining bright.\\*
\vin They robbed the \textsc{Glendale} train,\\
And people they did say, o'er many miles away,\\*
\vin It was those outlaws; they're \textit{Frank} and \textit{Jesse James}!\\!

Now the people held their breath when they heard of \textit{Jesse}'s death,\\*
\vin And wondered how he ever came to fall.\\
\textit{Robert Ford} -- it was a fact -- he shot \textit{Jesse} in the back\\*
\vin While \textit{Jesse}hung a picture on the wall\\!

Now \textit{Jesse} went to rest with his hand on his breast,\\*
\vin The devil will be upon his knee.\\
He was born one day in the County Clay,\\*
\vin And he came from a solitary race.\\!

{\itshape
Well \textit{Jesse} had a wife to mourn for his life,\\
\vin Three children -- now they were brave.\\
That dirty little coward that shot Mr \textit{Howard}\\*
\vin Has laid \textit{Jesse James} in his grave.}
\end{verse}

\subsection{}

\blfootnote{William Shakespeare (1564 -- 1616), \cite{treasury}.}\settowidth{\versewidth}{Nor shall death brag thou wanderest in his shade,}
\begin{verse}[\versewidth]
Shall I compare thee to a summer's day?\\*
Thou art more lovely \& more temperate:\\
Rough winds do shake the darling buds of may,\\*
And summer's lease hath all too short a date;\\!

Sometime too hot the eye of heaven shines,\\*
And often is his gold complexion dimmed;\\
And every fair from fair sometime declines,\\*
By chance, or nature's changing course, untrimmed.\\!

But thy eternal summer shall not fade,\\*
Nor lose possession of that fair thou owest;\\
Nor shall death brag thou wanderest in his shade,\\*
When in eternal lines to time thou growest:--\\!

So long as men can breathe, or eyes can see,\\*
So long lives this, and this gives life to thee.
\end{verse}

\subsection{}

\blfootnote{Oscar Wilde (1854 -- 1900), \cite{odq}.}A man cannot be too careful in the choice of his enemies.

\section{}

\subsection{}

\blfootnote{Anonymous, \cite{terfel_simple}. In negro spirituals (and elsewhere), crossing the river Jordan is a metaphor for death.}\begin{center}
\textit{Tune: Deep River}
\end{center}

\settowidth{\versewidth}{My home is over Jordan.}
\begin{verse}[\versewidth]
Deep river,\\*
My home is over \textsc{Jordan}.\\
Deep river, Lord,\\*
I want to cross over into campground.\\!

O don't you want to go\\*
To that gospel feast,\\
That promised land\\*
Where all is peace?\\!

Deep river,\\*
My home is over \textsc{Jordan}.\\
Deep river, Lord,\\*
I want to cross over into campground.
\end{verse}

\subsection{}

\blfootnote{William Shakespeare (1564 -- 1616), \cite{norton}. This song is sung by Ariel in \refbook{The Tempest} V.1.}\settowidth{\versewidth}{Where the bee sucks, there suck I:}
\begin{verse}[\versewidth]
Where the bee sucks, there suck I:\\*
In a cowslip's bell I lie;\\
There I couch when owls do cry.\\
On the bat’s back I do fly\\
After summer merrily.\\
Merrily, merrily shall I live now\\*
Under the blossom that hangs on the bough.
\end{verse}

\subsection{}

\blfootnote{Oscar Wilde (1854 -- 1900), \cite{aaat}.}Consistency is the last refuge of the unimaginative.

\section{}

\subsection{}

\blfootnote{Anonymous, \cite{donegan}. \P 9. Comparing with Lomax's \textit{Our Singing Country}, it seems that Donegan bowdlerised the word `nigger' out of his version.}\begin{center}
\textit{Tune: Long Summer Day}
\end{center}

\settowidth{\versewidth}{Well, the mas'er and the mis'ess is a-sittin' in the parlour,}
\begin{verse}[\versewidth]
Well, a long summer day make a white man lazy.\\*
\vin \textit{Long summer day!}\\
Long summer day make a white man lazy.\\*
\vin \textit{Long summer day!}\\!

{\itshape
Long summer! Long summer!\\
\vin Long summer day!\\
Long summer! Long summer!\\*
\vin Long summer day!}\\!

Well, a long summer day make a nigger run away, sir.\\*
Long summer day make a slave run away, sir.\\!

Well, a-pickin' that cotton in the bottom field.\\*
It's a gath'rin' up the cotton in the bottom field.\\!

Well, the mas'er and the mis'ess is a-sittin' in the parlour,\\*
Just a-fixin' and a-studyin' how to work a slave harder.\\!

Run away to see his \textit{Mary}.\\*
He run away to see his baby.\\!

The mas'er killed his jersey bull to give the bull his bellyful.\\*
\vin \textit{Long summer day!}\\
The mas'er killed his jersey bull to give the bull his bellyful.\\*
\vin \textit{Long summer day!}\\!

{\itshape
Long summer! Long summer!\\
\vin Long summer day!\\
Long summer! Long summer!\\*
\vin Long summer day!}
\end{verse}

\subsection{}

\blfootnote{Percy Shelley (1792 -- 1822), \cite{treasury}.}\settowidth{\versewidth}{I fear thy kisses, gentle maiden;}
\begin{verse}[\versewidth]
I fear thy kisses, gentle maiden;\\*
\vin Thou needest not fear mine;\\
My spirit is too deeply laden\\*
\vin Ever to burthen thine.\\!

I fear thy mien, thy tones, thy motion;\\*
\vin Thou needest not fear mine;\\
Innocent is the heart's devotion\\*
\vin With which I worship thine.
\end{verse}

\subsection{}

\blfootnote{Oscar Wilde (1854 -- 1900), \cite{odq}.}I can resist everything except temptation.

\section{}

\subsection{}

\blfootnote{`Home! Sweet Home!', John Payne (1791 -- 1852), \cite{farcrynd}. Payne wrote the words to this song for his 1823 opera \refbook{Clari, or the Maid of Milan}.}\begin{center}
\textit{Tune: Home Sweet Home}
\end{center}

\settowidth{\versewidth}{Which, seek through the world, is ne'er met with elsewhere.}
\begin{verse}[\versewidth]
'Mid pleasures and palaces though we may roam,\\*
Be it ever so humble, there's no place like home;\\
A charm from the sky seems to hallow us there,\\*
Which, seek through the world, is ne'er met with elsewhere.\\!

{\itshape
Home! Home!\\
Sweet, sweet home!\\
There's no place like home!\\*
There's no place like home!}\\!

An exile from home, splendour dazzles in vain;\\*
O give me my lowly thatched cottage again!\\
The birds singing gaily, that come at my call --\\*
Give me them -- and the peace of mind, dearer than all!\\!

I gaze on the moon as I tread the drear wild,\\*
And feel that my mother now thinks of her child,\\
As she looks on that moon from our own cottage door\\*
Through the woodbine, whose fragrance shall cheer me no more.\\!

How sweet 'tis to sit 'neath a fond father's smile,\\*
And the caress of a mother to soothe and beguile!\\
Let others delight 'mid new pleasures to roam,\\*
But give me, O give me, the pleasures of home.\\!

To thee I'll return, overburdened with care;\\*
The heart's dearest solace will smile on me there;\\
No more from that cottage again will I roam;\\*
Be it ever so humble, there's no place like home.\\!

{\itshape
Home! Home!\\
Sweet, sweet home!\\
There's no place like home!\\*
There's no place like home!}
\end{verse}

\subsection{}

\blfootnote{`A Ditty', Sir Philip Sidney (1554 -- 1586), \cite{treasury}.}\settowidth{\versewidth}{My heart in him his thoughts senses guides:}
\begin{verse}[\versewidth]
My truelove hath my heart, and I have his,\\*
By just exchange one to the other given:\\
I hold his dear, and mine he cannot miss,\\
There never was a better bargain driven:\\*
\vin My truelove hath my heart, and I have his.\\!

His heart in me keeps him \& me in one,\\*
My heart in him his thoughts \& senses guides:\\
He loves my heart, for once it was his own,\\
I cherish his because in me it bides:\\*
\vin My truelove hath my heart, and I have his.
\end{verse}

\subsection{}

\blfootnote{Oscar Wilde (1854 -- 1900), \cite{odq}. This a line from \refbook{The Importance of Being Earnest}.}I hope you have not been leading a double life, pretending to be wicked and being really good all the time; that would be hypocrisy.

\section{}

\subsection{}

\blfootnote{Anonymous, \cite{oas1}.}\begin{center}
\textit{Tune: The Boatmen's Dance}
\end{center}

\settowidth{\versewidth}{He spends his cash and works for more.}
\begin{verse}[\versewidth]
The boatmen dance; the boatmen sing;\\*
The boatmen up to ev'rything.\\
And when the boatman gets on shore\\*
He spends his cash and works for more.\\!

{\itshape
High row! The boatmen row!\\
Floatin' down the river the \textsc{Ohio}!\\
Then dance! The boatmen dance!\\
O dance! The boatmen dance!\\
O dance all night till broad daylight!\\*
Go home with the gals in the mornin'!}\\!

I went on board the other day\\*
To see what the boatmen had to say.\\
There I let my passion loose\\*
An' they cram me in the calaboose.\\!

The boatman is a thrifty man.\\*
There's none can do as the boatman can.\\
I never see a pretty gal in my life\\*
But that she was a boatman's wife.\\!

{\itshape
High row! The boatmen row!\\
Floatin' down the river the \textsc{Ohio}!\\
Then dance! The boatmen dance!\\
O dance! The boatmen dance!\\
O dance all night till broad daylight!\\*
Go home with the gals in the mornin'!}
\end{verse}

\subsection{}

\blfootnote{$\mathbb{R}$ Edmund Spenser (1552 -- 1599), \cite{norton}. This is one of Spenser's \refbook{Amoretti}.}\settowidth{\versewidth}{Fair is my love, when her fair golden heares,}
\begin{verse}[\versewidth]
Fair is my love, when her fair golden heares,\\*
With the loose wind ye waving chance to mark:\\
Fair when the rose in her red cheeks appears,\\
Or in her eyes the fire of love does spark.\\
Fair when her breast like a rich-laden bark,\\
With precious merchandise she forth doth lay:\\
Fair when that cloud of pride which oft doth dark\\
Her goodly light with smiles she drives away,\\
But fairest she, when so she doth display\\
The gate with pearls \& rubies richly dight:\\
Through which her words so wise do make their way\\
To bear the message of her gentle sprite.\\
The rest be works of nature's wonderment,\\*
But this the work of heart's astonishment.
\end{verse}

\subsection{}

\blfootnote{Oscar Wilde (1854 -- 1900), \cite{phrasesandphilosophies}.}If one tells the truth, one is sure, sooner or later, to be found out.

\section{}

\subsection{}

\blfootnote{Harry McClintock (1882 -- 1957), \cite{obrother}. \P 2. In the ``Hobo'' slang of the worker-vagrants of the United States, a `jungle' is an improvised camp, often located near the freight yard of a railway. \P 3. In the same slang as above, a `hobo' is distinguished from other kinds of homeless man in that he often gains transitory employment. \P 36. In the same slang as above, `bulls' are policemen.}\begin{center}
\textit{Tune: Big Rock Candy Mountains}
\end{center}

\settowidth{\versewidth}{I'm headed for a land that's far away,}
\begin{verse}[\versewidth]
One evening as the sun went down\\*
\vin And the jungle fire was burning,\\
Up the track came a hobo hiking,\\
\vin And he said, Boys, I'm not turning.\\
I'm headed for a land that's far away,\\
\vin Besides the crystal fountains.\\
So come with me. We'll go and see\\*
\vin The Big Rock Candy Mountains.\\!

In the Big Rock Candy Mountains,\\*
\vin There's a land that's fair \& bright,\\
Where the handouts grow on bushes,\\
\vin And you sleep out every night,\\
Where the boxcars all are empty,\\
And the sun shines every day\\
\vin On the birds \& the bees\\
\vin And the cigarette trees.\\
\vin The lemonade springs\\
\vin Where the bluebird sings,\\*
In the Big Rock Candy Mountains.\\!

In the Big Rock Candy Mountains,\\*
\vin All the cops have wooden legs,\\
And the bulldogs all have rubber teeth,\\
\vin And the hens lay soft-boiled eggs.\\
The farmers' trees are full of fruit,\\
And the barns are full of hay.\\
\vin O I'm bound to go\\
\vin Where there ain't no snow.\\
\vin The rain don't fall,\\
\vin The wind don't blow,\\*
In the Big Rock Candy Mountains.\\!

In the Big Rock Candy Mountains,\\*
\vin You never change your socks,\\
And the little streams of alcohol\\
\vin Come trickling down the rocks.\\
The brakemen have to tip their hats\\
And the railroad bulls are blind.\\
\vin There's a lake of stew\\
\vin And of whiskey too.\\
\vin You can paddle all around 'em\\
\vin In a big canoe.\\*
In the Big Rock Candy Mountains.\\!

In the Big Rock Candy Mountains,\\*
\vin The jails are made of tin,\\
And you can walk right out again\\
\vin As soon as you are in.\\
There ain't no short-handled shovels,\\
No axes, saws or picks.\\
\vin I'm a-going to stay\\
\vin Where you sleep all day,\\
\vin Where they hung the turk\\
\vin That invented work,\\*
In the Big Rock Candy Mountains.\\!

\vin I'll see you all\\*
\vin This coming fall,\\*
In the Big Rock Candy Mountains.
\end{verse}

\subsection{}

\blfootnote{$\mathbb{R}$ Edmund Spenser (1552 -- 1599), \cite{norton}. This is the first of Spenser's \refbook{Amoretti}. \P 10. The Helicon is a river in Greece, part of which runs underground. The women who murdered Orpheus attempted to wash their hands therein after their crime, but the water sank into the earth so as not to be stained with the poet's blood.}\settowidth{\versewidth}{        Those lamping eyes will deign sometimes to look}
\begin{verse}[\versewidth]
Happy ye leaves when as those lily hands,\\*
\vin Which hold my life in their dead doing might\\
Shall handle you and hold in love's soft bands,\\
\vin Like captives trembling at the victor's sight.\\
\vin And happy lines, on which with starry light,\\
\vin \vin Those lamping eyes will deign sometimes to look\\
\vin And read the sorrows of my dying sprite,\\
\vin \vin Written with tears in heart's close bleeding book.\\
And happy rhymes bathed in the sacred brook\\
\vin Of \textsc{Helicon} whence she deriv\`{e}d is,\\
When ye behold that angel's bless\`{e}d look,\\
\vin My soul's long lack\`{e}d food, my heaven's bliss.\\
Leaves, lines \& rhymes, seek her to please alone,\\*
Whom if ye please, I care for other none.
\end{verse}

\subsection{}

\blfootnote{Oscar Wilde (1854 -- 1900), \cite{odq}.}Many a woman has a past, but I am told that she has at least a dozen.

\section{}

\subsection{}

\blfootnote{Anonymous, \cite{carolina2nd_southern}.}\begin{center}
\textit{Tune: Turkey in the Straw}
\end{center}

\settowidth{\versewidth}{And the first man I chanced to meet was old Zip Coon.}
\begin{verse}[\versewidth]
Old \textit{Zip Coon}, he's a learned scholar.\\*
Old \textit{Zip Coon}, he's a learned scholar.\\
Old \textit{Zip Coon}, he's a learned scholar,\\*
Sings possum up the gum tree, coonie in the holler.\\!

{\itshape
Possum up the gum tree! Coonie on the stump!\\
Possum up the gum tree! Coonie on the stump!\\
Possum up the gum tree! Coonie on the stump!\\*
Then over double trouble \textit{Zip Coon} will jump!}\\!

It's old \textit{Sukie Blueskin}, she is in love with me.\\*
I went the other afternoon to take a cup of tea.\\
Now what do you old \textit{Sukie} had for supper?\\*
Why chicken foot \& possum foot without any butter.\\!

I went down \textsc{Sandy Hollow} the other afternoon,\\*
And the first man I chanced to meet was old \textit{Zip Coon}.\\
Old \textit{Zip Coon}, he's a natty scholar,\\*
For he plays upon the banjo ``Coonie in the holler''.\\!

Well you heard about the battle of old \textsc{New 'Leans},\\*
Where old General \textit{Jackson} gave the british beans.\\
The yankees did the redcoats up so slick,\\*
For they catched old \textit{Pakenham} and rowed him up the creek.\\!

{\itshape
Possum up the gum tree! Coonie on the stump!\\
Possum up the gum tree! Coonie on the stump!\\
Possum up the gum tree! Coonie on the stump!\\*
Then over double trouble \textit{Zip Coon} will jump!}
\end{verse}

\subsection{}

\blfootnote{Edmund Spenser (1552 -- 1599), \cite{norton}. This is one of Spenser's \refbook{Amoretti}.}\settowidth{\versewidth}{Through your bright beams doth not the blinded guest}
\begin{verse}[\versewidth]
More than most fair, full of the living fire,\\*
\vin Kindled above unto the maker near:\\
No eyes but joys, in which all powers conspire,\\
\vin That to the world naught else be counted dear.\\
Through your bright beams doth not the blinded guest\\
\vin Shoot out his darts to base affection's wound?\\
But angels come to lead frail minds to rest\\
\vin In chaste desires on heavenly beauty bound.\\
You frame my thoughts \& fashion me within;\\
\vin You stop my tongue, \& teach my heart to speak,\\
You calm the storm that passion did begin,\\
\vin Strong through your cause, but by your virtue weak.\\
Dark is the world, where your light shin\`{e}d never;\\*
Well is he born, that may behold you ever.
\end{verse}

\subsection{}

\blfootnote{Oscar Wilde (1854 -- 1900), \cite{odq}. This a line from \refbook{The Importance of Being Earnest}.}None of us are perfect; I myself am peculiarly susceptible to draughts.

\section{}

\subsection{}

\blfootnote{Dan Emmett (1815 -- 1904), \cite{carolina2nd_southern}.}\begin{center}
\textit{Tune: O Lud Gals}
\end{center}

\settowidth{\versewidth}{Pretty little black gal just like the other.}
\begin{verse}[\versewidth]
It's up the rope and down the cable.\\*
Forty horses in the stable.\\
First an indian then a squaw,\\*
Going away to the \textsc{Arkansas}.\\!

{\itshape
\vin O lud, gals!\\
Give me chaw tobacco!\\
\vin O lud, gals!\\
Fetch along the whisky!\\*
Makes my head swim when I get a little tipsy!}\\!

It's vinegar shoes \& paper stockings,\\*
Says to me Miss \textit{Polly Hopkins}.\\
My wife's dead, and I'm a widder,\\*
All the way from the rolling river.\\!

If I had wife \& a little baby,\\*
I'd support her like a lady.\\
Gods of war \& little fishes,\\*
Earthen plates \& pewter dishes.\\!

Cow hide shoes \& buck skin breeches:\\*
Give the gal that sews the stitches,\\
Prettiest thing in all creation:\\*
Yaller gal in the wild goose nation.\\!

It's all the way from the indian nation:\\*
A little corn crib on a big plantation.\\
My wife's dead; I'll get another,\\*
Pretty little black gal just like the other.\\!

Blow away, ye gentle breezes,\\*
Down among them cinammon treeses.\\
There I sit long with the muses,\\*
Mending my old boots \& shoeses.\\!

{\itshape
\vin O lud, gals!\\
Give me chaw tobacco!\\
\vin O lud, gals!\\
Fetch along the whisky!\\*
Makes my head swim when I get a little tipsy!}
\end{verse}

\subsection{}

\blfootnote{Edmund Spenser (1552 -- 1599), \cite{norton}. This is one of Spenser's \refbook{Amoretti}.}\settowidth{\versewidth}{    Vain man, said she, that dost in vain assay,}
\begin{verse}[\versewidth]
One day I wrote her name upon the strand,\\*
\vin But came the waves \& wash\`{e}d it away:\\
Again I wrote it with a second hand,\\
\vin But came the tide, and made my pains his prey.\\
\vin Vain man, said she, that dost in vain assay,\\
\vin \vin A mortal thing so to immortalise;\\
\vin For I myself shall like to this decay,\\
\vin \vin And eke my name be wip\`{e}d out likewise.\\
Not so, quod I. Let baser things devise\\
\vin To die in dust, but you shall live by fame:\\
My verse your virtues rare shall eternise,\\
\vin And in the heavens write your glorious name:\\
Where whenas death shall all the world subdue,\\*
Our love shall live, and later life renew.
\end{verse}

\subsection{}

\blfootnote{Oscar Wilde (1854 -- 1900), \cite{idealhusband}.}Only dull people are brilliant at breakfast.

\section{}

\subsection{}

\blfootnote{Anonymous, \cite{dumbangel}. According to William Allen's \emph{Slave Songs of the United States}, this song sprang up among newly-freed slaves who had been stranded on Saint Helena Island, South Carolina, due to the vicissitudes of the American Civil War. The river Jordan has long been a metaphor for death, and the Archangel Michael is often said to be the conductor of the souls of the departed into the afterlife.}\begin{center}
\textit{Tune: Michael Row de Boat Ashore}
\end{center}

\settowidth{\versewidth}{The River Jordan is chilly and cold,}
\begin{verse}[\versewidth]
{\itshape
Michael, row the boat ashore.\\
\vin Hallelujah!\\
Michael row the boat ashore.\\*
\vin Hallelujah!}\\!

Sister help to trim the sail.\\*
\vin \textit{Hallelujah!}\\
Sister help to trim the sail.\\*
\vin \textit{Hallelujah!}\\!

The \textsc{River Jordan} is chilly and cold,\\*
\vin \textit{Hallelujah!}\\
Chills the body but not the soul.\\*
\vin \textit{Hallelujah!}\\!

The river is deep and the river is wide.\\*
\vin \textit{Hallelujah!}\\
Milk and honey on the other side.\\*
\vin \textit{Hallelujah!}\\!

{\itshape
Michael, row the boat ashore.\\
\vin Hallelujah!\\
Michael row the boat ashore.\\*
\vin Hallelujah!}
\end{verse}

\subsection{}

\blfootnote{`Love and Sleep', Algernon Swinburne (1837 -- 1909), \cite{newlove}.}\settowidth{\versewidth}{Smooth-skinned dark, with bare throat made to bite,}
\begin{verse}[\versewidth]
Lying asleep between the strokes of night,\\*
\vin I saw my love lean over my sad bed,\\
\vin Pale as the duskiest lily's leaf or head,\\
Smooth-skinned \& dark, with bare throat made to bite,\\
Too wan for blushing \& too warm for white,\\
\vin But perfect-coloured without white or red.\\
\vin And her lips opened amorously, and said\\
I wist not what, saving one word: delight.\\
And all her face was honey to my mouth,\\
\vin And all her body pasture to mine eyes;\\
\vin \vin The long lithe arms \& hotter hands than fire,\\
The quivering flanks, hair smelling of the south,\\
\vin The bright light feet, the splendid supple thighs\\*
\vin \vin And glittering eyelids of my soul's desire.
\end{verse}

\subsection{}

\blfootnote{Oscar Wilde (1854 -- 1900), \cite{odq}.}There is only one thing in the world worse than being talked about, and that is not being talked about.

\section{}

\subsection{}

\blfootnote{Anonymous, \cite{theblackalbum}. While the origins of this folk song are elusive, its first appearance in print seems to have been in \refbook{Immortalia} (1927).}\begin{center}
\textit{Tune: The Bastard King of England}
\end{center}

\settowidth{\versewidth}{She loved to fool with the bastard's tool,}
\begin{verse}[\versewidth]
O the minstrels sing of english king,\\*
\vin Many long years ago,\\
Who ruled his land with an iron hand,\\
\vin Though his morals were weak \& low.\\
He loved to hunt the royal stag\\
\vin That lived in the royal wood,\\
But better than it he loved to sit\\
\vin And pound the royal pud.\\
He was dirty \& lousy \& full of fleas!\\
His terrible tool hung down to his knees!\\*
God bless the bastard King of England!\\!

The Queen of Spain was an amorous jane;\\*
\vin A lascivious wench was she.\\
She loved to fool with the bastard's tool,\\
\vin So far across the sea;\\
So she sent a special letter\\
\vin By a special messenger,\\
To ask the king if he would spend\\
\vin A night or two with her.\\
He was wild \& woolly \& full of fleas!\\
He had his women by twos \& three!\\*
God save the bastard King of England!\\!

When \textit{Philip} of France heard of this,\\*
\vin He summoned the royal court --\\
`Because she loves my rival more\\
\vin Because my tool is short.'\\
He sent the Duke of Zippity-Zap\\
To give the queen a dose of the clap,\\*
To give to the bastard King of England.\\!

When the bastard king he heard of this,\\*
\vin All in fair \textsc{Windsor}'s walls,\\
He took the oath: by his hairy growth,\\
\vin He'd have the frenchman's balls.\\
So he offered \sfrac{$1$}{$2$} his kingdom\\
\vin And a piece of the Queen \textit{Hortense}\\
To any man with a cunning plan\\
\vin Who'd diddle the King of France.\\
A volunteer he soon was found.\\
His cries \& spies were well renowned.\\*
Farwell, the bastard King of England!\\!

Then the royal Duke of Buttock\\*
\vin Betook himself to France.\\
For he was a faggot,\\
\vin So he took off his pants.\\
But at the crucial moment --\\
\vin Now here's the best of all --\\
As \textit{Philip} left, the duke's right cleft\\
\vin Had seized the frenchman's balls.\\
Around his dong he slipped a thong,\\
Upped on his horse and dragged him along\\*
Back to the bastard King of England.\\!

Well, when he reached fair England's shore,\\*
\vin He fainted on the shore,\\
For on the ride King \textit{Philip}'s pride\\
\vin Had stretched out six times more.\\
The maids of all the countryside\\
\vin Who gathered in the town,\\
They took one look at the frenchman's crook\\
\vin And denounced the royal crown.\\
They set King \textit{Philip} upon the throne.\\
His sceptre was his royal bone.\\*
Farewell, the bastard King of England!\\!

He was dirty \& lousy \& full of fleas!\\*
His terrible tool hung down to his knees!\\*
God damn the bastard King of England.
\end{verse}

\subsection{}

\blfootnote{Dr William Wordsworth, Poet Laureate (1770 -- 1850), \cite{norton}.}\settowidth{\versewidth}{My heart leaps up when I behold}
\begin{verse}[\versewidth]
My heart leaps up when I behold\\*
\vin A rainbow in the sky:\\
So was it when my life began;\\
So is it now I am a man;\\
So be it when I shall grow old,\\
\vin Or let me die!\\
The child is father of the man;\\
And I could wish my days to be\\*
Bound each to each by natural piety.
\end{verse}

\subsection{}

\blfootnote{Oscar Wilde (1854 -- 1900), \cite{odq}.}We are all in the gutter, but some of us are looking at the stars.

\chapter{September}

\section{}

\subsection{}

\blfootnote{`When Lilacs Last in the Door-Yard Bloom'd', Walt Whitman (1819 -- 1892), \cite{norton}. This poem is an elegy for Abraham Lincoln, 16th President of the United States.}\settowidth{\versewidth}{In the close of the day with its light and the fields of spring, and the farmers preparing their crops,}
\begin{verse}[\versewidth]
When lilacs last in the door-yard bloomed,\\*
And the great star early drooped in the western sky in the night,\\*
I mourned, and yet shall mourn with ever-returning spring.\\!

Ever-returning spring, trinity sure to me you bring,\\*
Lilac blooming perennial and drooping star in the west,\\*
And thought of him I love. \aldine\\!

O powerful western fallen star!\\*
O shades of night -- O moody, tearful night!\\
O great star disappeared -- O the black murk that hides the star!\\
O cruel hands that hold me powerless -- O helpless soul of me!\\*
O harsh surrounding cloud that will not free my soul.\\!

In the door-yard fronting an old farm-house near the white-washed palings,\\*
Stands the lilac-bush tall-growing with heart-shaped leaves of rich green,\\
With many a pointed blossom rising delicate, with the perfume strong I love,\\
With every leaf a miracle—and from this bush in the door-yard,\\
With delicate-coloured blossoms \& heart-shaped leaves of rich green,\\*
A sprig with its flower I break. \aldine\\!

In the swamp in secluded recesses,\\*
A shy \& hidden bird is warbling a song.\\!

Solitary the thrush,\\*
The hermit withdrawn to himself, avoiding the settlements,\\*
Sings by himself a song.\\!

Song of the bleeding throat,\\*
Death's outlet song of life, (for well, dear brother, I know,\\*
If thou wast not granted to sing thou would'st surely die). \aldine\\!

Over the breast of the spring, the land, amid cities,\\*
Amid lanes \& through old woods, where lately the violets peeped from the ground, spotting the gray debris,\\
Amid the grass in the fields each side of the lanes, passing the endless grass,\\
Passing the yellow-speared wheat, every grain from its shroud in the dark-brown fields uprisen,\\
Passing the apple-tree blows of white \& pink in the orchards,\\
Carrying a corpse to where it shall rest in the grave,\\*
Night \& day journeys a coffin. \aldine\\!

Coffin that passes through lanes \& streets,\\*
Through day \& night with the great cloud darkening the land,\\
With the pomp of the inlooped flags with the cities draped in black,\\
With the show of the States themselves as of crape-veiled women standing,\\
With processions long and winding and the flambeaus of the night,\\
With the countless torches lit, with the silent sea of faces and the unbared heads,\\
With the waiting depot, the arriving coffin, and the sombre faces,\\
With dirges through the night, with the 1000 voices rising strong \& solemn,\\
With all the mournful voices of the dirges poured around the coffin,\\
The dim-lit churches \& the shuddering organs -- where amid these you journey,\\
With the tolling tolling bells' perpetual clang,\\
Here, coffin that slowly passes,\\*
I give you my sprig of lilac. \aldine\\!

(Nor for you, for one alone,\\*
Blossoms \& branches green to coffins all I bring,\\*
For fresh as the morning, thus would I chant a song for you, O sane \& sacred death.\\!

All over bouquets of roses,\\*
O death, I cover you over with roses \& early lilies,\\
But mostly \& now the lilac that blooms the first,\\
Copious I break, I break the sprigs from the bushes,\\
With loaded arms I come, pouring for you,\\*
For you and the coffins all of you, O death.) \aldine\\!

O western orb sailing the heaven,\\*
Now I know what you must have meant as a month since I walked,\\
As I walked in silence the transparent shadowy night,\\
As I saw you had something to tell as you bent to me night after night,\\
As you drooped from the sky low down as if to my side, (while the other stars all looked on,)\\
As we wandered together the solemn night, (for something I know not what kept me from sleep,)\\
As the night advanced, and I saw on the rim of the west how full you were of woe,\\
As I stood on the rising ground in the breeze in the cool transparent night,\\
As I watched where you passed and was lost in the nether-ward black of the night,\\
As my soul in its trouble dissatisfied sank, as where you, sad orb,\\*
Concluded, dropped in the night, and was gone. \aldine\\!

Sing on there in the swamp,\\*
O singer bashful \& tender. I hear your notes; I hear your call,\\
I hear, I come presently, I understand you,\\
But a moment I linger, for the lustrous star has detained me,\\*
The star my departing comrade holds and detains me. \aldine\\!

O how shall I warble myself for the dead one there I loved?\\*
And how shall I deck my song for the large sweet soul that has gone?\\*
And what shall my perfume be for the grave of him I love?\\!

Sea-winds blown from east \& west,\\*
Blown from the eastern sea and blown from the western sea, till there on the prairies meeting,\\
These \& with these \& the breath of my chant,\\*
I'll perfume the grave of him I love. \aldine\\!

O what shall I hang on the chamber walls?\\*
And what shall the pictures be that I hang on the walls,\\*
To adorn the burial-house of him I love?\\!

Pictures of growing spring \& farms \& homes,\\*
With the fourth-month eve at sundown, \& the grey smoke lucid \& bright,\\
With floods of the yellow gold of the gorgeous, indolent, sinking sun, burning, expanding the air,\\
With the fresh sweet herbage under foot, \& the pale green leaves of the trees prolific,\\
In the distance the flowing glaze, the breast of the river, with a wind-dapple here \& there,\\
With ranging hills on the banks, with many a line against the sky, \& shadows,\\
And the city at hand with dwellings so dense, \& stacks of chimneys,\\*
And all the scenes of life \& the workshops, \& the workmen homeward returning. \aldine\\!

Lo, body \& soul -- this land,\\*
My own \textsc{Manhattan} with spires, and the sparkling \& hurrying tides, and the ships,\\
The varied \& ample land, the south and the north in the light, \textsc{Ohio}'s shores and flashing \textsc{Missouri},\\*
And ever the far-spreading prairies covered with grass \& corn.\\!

Lo, the most excellent sun so calm \& haughty,\\*
The violet \& purple morn with just-felt breezes,\\
The gentle soft-born measureless light,\\
The miracle spreading bathing all, the fulfilled noon,\\
The coming eve delicious, the welcome night \& the stars,\\*
Over my cities shining all, enveloping man and land. \aldine\\!

Sing on, sing on, you gray-brown bird.\\*
Sing from the swamps, the recesses, pour your chant from the bushes,\\*
Limitless out of the dusk, out of the cedars and pines.\\!

Sing on dearest brother, warble your reedy song,\\*
Loud human song, with voice of uttermost woe.\\!

O liquid \& free \& tender!\\*
O wild \& loose to my soul -- O wondrous singer!\\
You only I hear -- yet the star holds me, (but will soon depart,)\\*
Yet the lilac with mastering odor holds me. \aldine\\!

Now while I sat in the day and looked forth,\\*
In the close of the day with its light and the fields of spring, and the farmers preparing their crops,\\
In the large unconscious scenery of my land with its lakes \& forests,\\
In the heavenly aerial beauty (after the perturbed winds and the storms)\\
Under the arching heavens of the afternoon swift passing, and the voices of children \& women,\\
The many-moving sea-tides, and I saw the ships how they sailed,\\
And the summer approaching with richness, and the fields all busy with labour,\\
And the infinite separate houses, how they all went on, each with its meals \& minutiae of daily usages,\\
And the streets how their throbbings throbbed, and the cities pent -- lo, then \& there,\\
Falling upon them all \& among them all, enveloping me with the rest,\\
Appeared the cloud, appeared the long black trail,\\*
And I knew death, its thought, and the sacred knowledge of death.\\!

Then with the knowledge of death as walking one side of me,\\*
And the thought of death close-walking the other side of me,\\
And I in the middle as with companions, and as holding the hands of companions,\\
I fled forth to the hiding receiving night that talks not,\\
Down to the shores of the water, the path by the swamp in the dimness,\\*
To the solemn shadowy cedars and ghostly pines so still.\\!

And the singer so shy to the rest received me,\\*
The grey-brown bird I know received us comrades three,\\*
And he sang the carol of death, and a verse for him I love.\\!

From deep secluded recesses,\\*
From the fragrant cedars and the ghostly pines so still,\\*
Came the carol of the bird.\\!

And the charm of the carol rapt me,\\*
As I held as if by their hands my comrades in the night,\\*
And the voice of my spirit tallied the song of the bird.\\!

Come lovely \& soothing death,\\*
Undulate round the world, serenely arriving, arriving,\\
In the day, in the night, to all, to each,\\*
Sooner or later delicate death.\\!

Praised be the fathomless universe,\\*
For life \& joy, and for objects \& knowledge curious,\\
And for love, sweet love—but praise! praise! praise!\\*
For the sure-enwinding arms of cool-enfolding death.\\!

Dark mother always gliding near with soft feet,\\*
Have none chanted for thee a chant of fullest welcome?\\
Then I chant it for thee, I glorify thee above all,\\*
I bring thee a song that when thou must indeed come, come unfalteringly.\\!

Approach strong deliveress,\\*
When it is so, when thou hast taken them I joyously sing the dead,\\
Lost in the loving floating ocean of thee,\\*
Laved in the flood of thy bliss, O death.\\!

From me to thee glad serenades,\\*
Dances for thee I propose saluting thee, adornments \& feastings for thee,\\
And the sights of the open landscape \& the high-spread sky are fitting,\\*
And life and the fields, and the huge and thoughtful night.\\!

The night in silence under many a star,\\*
The ocean shore and the husky whispering wave whose voice I know,\\
And the soul turning to thee, O vast \& well-veiled death,\\*
And the body gratefully nestling close to thee.\\!

Over the tree-tops I float thee a song,\\*
Over the rising and sinking waves, over the myriad fields and the prairies wide,\\
Over the dense-packed cities all and the teeming wharves \& ways,\\*
I float this carol with joy, with joy to thee, O death. \aldine\\!

To the tally of my soul,\\*
Loud \& strong kept up the grey-brown bird,\\*
With pure deliberate notes spreading, filling the night.\\!

Loud in the pines \& cedars dim,\\*
Clear in the freshness moist and the swamp-perfume,\\*
And I with my comrades there in the night.\\!

While my sight that was bound in my eyes unclosed,\\*
As to long panoramas of visions.\\!

And I saw askant the armies;\\*
I saw as in noiseless dreams 100s of battle-flags,\\
Borne through the smoke of the battles and pierced with missiles I saw them,\\
And carried hither \& yon through the smoke, and torn \& bloody,\\
And at last but a few shreds left on the staffs, (and all in silence,)\\*
And the staffs all splintered and broken.\\!

I saw battle-corpses, myriads of them,\\*
And the white skeletons of young men, I saw them,\\
I saw the debris \& debris of all the slain soldiers of the war,\\
But I saw they were not as was thought,\\
They themselves were fully at rest; they suffered not;\\
The living remained and suffered; the mother suffered;\\
And the wife \& the child \& the musing comrade suffered,\\*
And the armies that remained suffered. \aldine\\!

Passing the visions, passing the night,\\*
Passing, unloosing the hold of my comrades' hands,\\
Passing the song of the hermit bird and the tallying song of my soul,\\
Victorious song, death's outlet song, yet varying ever-altering song,\\
As low \& wailing, yet clear the notes, rising \& falling, flooding the night,\\
Sadly sinking \& fainting, as warning \& warning, and yet again bursting with joy,\\
Covering the earth \& filling the spread of the heaven,\\
As that powerful psalm in the night I heard from recesses,\\
Passing, I leave thee lilac with heart-shaped leaves,\\*
I leave thee there in the door-yard, blooming, returning with spring.\\!

I cease from my song for thee,\\*
From my gaze on thee in the west, fronting the west, communing with thee,\\*
O comrade lustrous with silver face in the night.\\!

Yet each to keep and all, retrievements out of the night,\\*
The song, the wondrous chant of the grey-brown bird,\\
And the tallying chant, the echo aroused in my soul,\\
With the lustrous \& drooping star with the countenance full of woe,\\
With the holders holding my hand nearing the call of the bird,\\
Comrades mine \& I in the midst, and their memory ever to keep, for the dead I loved so well,\\
For the sweetest, wisest soul of all my days \& lands -- and this for his dear sake,\\
Lilac \& star \& bird twined with the chant of my soul,\\*
There in the fragrant pines and the cedars dusk \& dim.
\end{verse}

\subsection{}

\blfootnote{Anonymous, \cite{norton}.}\settowidth{\versewidth}{    Rest you then, rest, sad eyes;}
\begin{verse}[\versewidth]
Weep you no more, sad fountains;\\*
\vin What need you flow so fast?\\
Look how the snowy mountains\\
\vin Heaven's sun doth gently waste.\\
\vin But my sun's heav'nly eyes\\
\vin \vin View not your weeping,\\
\vin \vin That now lies sleeping,\\
\vin Softly, softly, now softly lies\\*
\vin \vin \vin Sleeping.\\!

Sleep is a reconciling,\\*
\vin A rest that peace begets.\\
Doth not the sun rise smiling\\
\vin When fair at e'en he sets?\\
\vin Rest you then, rest, sad eyes;\\
\vin \vin Melt not in weeping\\
\vin \vin While she lies sleeping,\\
\vin Softly, softly, now softly lies\\*
\vin \vin \vin Sleeping.
\end{verse}

\subsection{}

\blfootnote{Proverbs 15.17, Anonymous, \cite{kjv}.}Better is a dinner of herbs where love is, than a stalled ox and hatred therewith.

\section{}

\subsection{}

\blfootnote{Dr William Wordsworth, Poet Laureate (1770 -- 1850), \cite{norton}.}\settowidth{\versewidth}{    Out-did the sparkling waves in glee:}
\begin{verse}[\versewidth]
I wandered lonely as a cloud\\*
\vin That floats on high o'er vales \& hills,\\
When all at once I saw a crowd,\\
\vin A host, of golden daffodils;\\
Beside the lake, beneath the trees,\\*
Fluttering and dancing in the breeze.\\!

Continuous as the stars that shine\\*
\vin And twinkle on the milky way,\\
They stretched in never-ending line\\
\vin Along the margin of a bay:\\
Ten thousand saw I at a glance,\\*
Tossing their heads in sprightly dance.\\!

The waves beside them danced; but they\\*
\vin Out-did the sparkling waves in glee:\\
A poet could not but be gay,\\
\vin In such a jocund company:\\
I gazed \& gazed, but little thought\\*
What wealth the show to me had brought:\\!

For oft, when on my couch I lie\\*
\vin In vacant or in pensive mood,\\
They flash upon that inward eye\\
\vin Which is the bliss of solitude;\\
And then my heart with pleasure fills,\\*
And dances with the daffodils.
\end{verse}

\subsection{}

\blfootnote{Dr William Wordsworth, Poet Laureate (1770 -- 1850), \cite{norton}.}\settowidth{\versewidth}{Dear child, dear girl, that walkest with me here,}
\begin{verse}[\versewidth]
It is a beauteous evening, calm \& free,\\*
\vin The holy time is quiet as a nun\\
\vin Breathless with adoration; the broad sun\\
Is sinking down in its tranquillity;\\
The gentleness of heaven broods o'er the sea;\\
\vin Listen! The mighty Being is awake,\\
\vin And doth with his eternal motion make\\
A sound like thunder -- everlastingly.\\
Dear child, dear girl, that walkest with me here,\\
\vin If thou appear untouched by solemn thought,\\
\vin \vin Thy nature is not therefore less divine:\\
Thou liest in \textit{Abraham}'s bosom all the year;\\
\vin \vin And worshipp'st at the Temple's inner shrine,\\*
\vin God being with thee when we know it not.
\end{verse}

\subsection{}

\blfootnote{Herbert Asquith, 1st Earl of Oxford and Asquith (1852 -- 1928), \cite{odq}.}Youth would be an ideal state if it came a little later in life.

\section{}

\subsection{}

\blfootnote{`Resolution and Independence', Dr William Wordsworth, Poet Laureate (1770 -- 1850), \cite{norton}.}\settowidth{\versewidth}{And all the air is filled with pleasant noise of waters.}
\begin{verse}[\versewidth]
There was a roaring in the wind all night;\\*
\vin The rain came heavily and fell in floods;\\
But now the sun is rising calm \& bright;\\
\vin The birds are singing in the distant woods;\\
\vin Over his own sweet voice the stock-dove broods;\\
The jay makes answer as the magpie chatters;\\*
And all the air is filled with pleasant noise of waters.\\!

All things that love the sun are out of doors;\\*
\vin The sky rejoices in the morning's birth;\\
The grass is bright with rain-drops; on the moors\\
\vin The hare is running races in her mirth;\\
\vin And with her feet she from the plashy earth\\
Raises a mist, that, glittering in the sun,\\*
Runs with her all the way, wherever she doth run.\\!

I was a traveller then upon the moor;\\*
\vin I saw the hare that raced about with joy;\\
I heard the woods and distant waters roar;\\
\vin Or heard them not, as happy as a boy:\\
\vin The pleasant season did my heart employ:\\
My old remembrances went from me wholly;\\*
And all the ways of men, so vain and melancholy.\\!

But, as it sometimes chanceth, from the might\\*
\vin Of joys in minds that can no further go,\\
As high as we have mounted in delight\\
\vin In our dejection do we sink as low;\\
\vin To me that morning did it happen so;\\
And fears and fancies thick upon me came;\\*
Dim sadness -- and blind thoughts, I knew not, nor could name.\\!

I heard the sky-lark warbling in the sky;\\*
\vin And I bethought me of the playful hare:\\
Even such a happy child of earth am I;\\
\vin Even as these blissful creatures do I fare;\\
\vin Far from the world I walk, and from all care;\\
But there may come another day to me --\\*
Solitude, pain of heart, distress, and poverty.\\!

My whole life I have lived in pleasant thought,\\*
\vin As if life's business were a summer mood;\\
As if all needful things would come unsought\\
\vin To genial faith, still rich in genial good;\\
\vin But how can He expect that others should\\
Build for him, sow for him, and at his call\\*
Love him, who for himself will take no heed at all?\\!

I thought of \textit{Chatterton}, the marvellous boy,\\*
\vin The sleepless soul that perished in his pride;\\
Of him who walked in glory and in joy\\
\vin Following his plough, along the mountain-side:\\
\vin By our own spirits are we deified:\\
We Poets in our youth begin in gladness;\\*
But thereof come in the end despondency \& madness.\\!

Now, whether it were by peculiar grace,\\*
\vin A leading from above, a something given,\\
Yet it befell that, in this lonely place,\\
\vin When I with these untoward thoughts had striven,\\
\vin Beside a pool bare to the eye of heaven\\
I saw a man before me unawares:\\*
The oldest man he seemed that ever wore grey hairs.\\!

As a huge stone is sometimes seen to lie\\*
\vin Couched on the bald top of an eminence;\\
Wonder to all who do the same espy,\\
\vin By what means it could thither come, and whence;\\
\vin So that it seems a thing endued with sense:\\
Like a sea-beast crawled forth, that on a shelf\\*
Of rock or sand reposeth, there to sun itself;\\!

Such seemed this man, not all alive nor dead,\\*
\vin Nor all asleep -- in his extreme old age:\\
His body was bent double, feet \& head\\
\vin Coming together in life's pilgrimage;\\
\vin As if some dire constraint of pain, or rage\\
Of sickness felt by him in times long past,\\*
A more than human weight upon his frame had cast.\\!

Himself he propped, limbs, body, and pale face,\\*
\vin Upon a long grey staff of shaven wood:\\
And, still as I drew near with gentle pace,\\
\vin Upon the margin of that moorish flood\\
\vin Motionless as a cloud the old Man stood,\\
That heareth not the loud winds when they call,\\*
And moveth all together, if it move at all.\\!

At length, himself unsettling, he the pond\\*
\vin Stirred with his staff, and fixedly did look\\
Upon the muddy water, which he conned,\\
\vin As if he had been reading in a book:\\
\vin And now a stranger's privilege I took;\\
And, drawing to his side, to him did say,\\*
`This morning gives us promise of a glorious day.'\\!

A gentle answer did the old man make,\\*
\vin In courteous speech which forth he slowly drew:\\
And him with further words I thus bespake,\\
\vin `What occupation do you there pursue?\\
\vin This is a lonesome place for one like you.'\\
Ere he replied, a flash of mild surprise\\*
Broke from the sable orbs of his yet-vivid eyes.\\!

His words came feebly, from a feeble chest,\\*
\vin But each in solemn order followed each,\\
With something of a lofty utterance drest --\\
\vin Choice word and measured phrase, above the reach\\
\vin Of ordinary men; a stately speech;\\
Such as grave livers do in Scotland use,\\*
Religious men, who give to God and man their dues.\\!

He told, that to these waters he had come\\*
\vin To gather leeches, being old and poor:\\
Employment hazardous and wearisome!\\
\vin And he had many hardships to endure:\\
\vin From pond to pond he roamed, from moor to moor;\\
Housing, with God's good help, by choice or chance;\\*
And in this way he gained an honest maintenance.\\!

The old man still stood talking by my side;\\*
\vin But now his voice to me was like a stream\\
Scarce heard; nor word from word could I divide;\\
\vin And the whole body of the Man did seem\\
\vin Like one whom I had met with in a dream;\\
Or like a man from some far region sent,\\*
To give me human strength, by apt admonishment.\\!

My former thoughts returned: the fear that kills;\\*
\vin And hope that is unwilling to be fed;\\
Cold, pain, and labour, and all fleshly ills;\\
\vin And mighty poets in their misery dead.\\
\vin Perplexed, and longing to be comforted,\\
My question eagerly did I renew,\\*
`How is it that you live, and what is it you do?'\\!

He with a smile did then his words repeat;\\*
\vin And said that, gathering leeches, far \& wide\\
He travelled; stirring thus about his feet\\
\vin The waters of the pools where they abide.\\
\vin `Once I could meet with them on every side;\\
But they have dwindled long by slow decay;\\*
Yet still I persevere, and find them where I may.'\\!

While he was talking thus, the lonely place,\\*
\vin The old man's shape, and speech -- all troubled me:\\
In my mind's eye I seemed to see him pace\\
\vin About the weary moors continually,\\
\vin Wandering about alone and silently.\\
While I these thoughts within myself pursued,\\*
He, having made a pause, the same discourse renewed.\\!

And soon with this he other matter blended,\\*
\vin Cheerfully uttered, with demeanour kind,\\
But stately in the main; and, when he ended,\\
\vin I could have laughed myself to scorn to find\\
\vin In that decrepit man so firm a mind.\\
`God,' said I, `be my help and stay secure;\\*
I'll think of the leech-gatherer on the lonely moor!'
\end{verse}

\subsection{}

\blfootnote{`Up-Hill', Miss Christina Rossetti (1830 -- 1894), \cite{norton}.}\settowidth{\versewidth}{Will the day's journey take the whole long day?}
\begin{verse}[\versewidth]
Does the road wind up-hill all the way?\\*
\vin Yes, to the very end.\\
Will the day's journey take the whole long day?\\*
\vin From morn to night, my friend.\\!

But is there for the night a resting-place?\\*
\vin A roof for when the slow dark hours begin.\\
May not the darkness hide it from my face?\\*
\vin You cannot miss that inn.\\!

Shall I meet other wayfarers at night?\\*
\vin Those who have gone before.\\
Then must I knock, or call when just in sight?\\*
\vin They will not keep you standing at that door.\\!

Shall I find comfort, travel-sore and weak?\\*
\vin Of labour you shall find the sum.\\
Will there be beds for me and all who seek?\\*
\vin Yea, beds for all who come.
\end{verse}

\subsection{}

\blfootnote{George Borrow (1803 -- 1881), \cite{odq}.}Fear God, and take your own part.

\section{}

\subsection{}

\blfootnote{`Lines Composed a Few Miles above Tintern Abbey, On Revisiting the Banks of the Wye during a Tour, July 13, 1798', Dr William Wordsworth, Poet Laureate (1770 -- 1850), \cite{norton}.}\settowidth{\versewidth}{Five years have passed; five summers, with the length}
\begin{verse}[\versewidth]
Five years have passed; five summers, with the length\\*
Of five long winters, and again I hear\\
These waters, rolling from their mountain springs\\
With a soft inland murmur. Once again\\
Do I behold these steep \& lofty cliffs,\\
That on a wild secluded scene impress\\
Thoughts of more deep seclusion; and connect\\
The landscape with the quiet of the sky.\\
The day is come when I again repose\\
Here, under this dark sycamore, and view\\
These plots of cottage-ground, these orchard-tufts,\\
Which at this season, with their unripe fruits,\\
Are clad in one green hue, and lose themselves\\
'Mid groves and copses. Once again I see\\
These hedge-rows, hardly hedgerows, little lines\\
Of sportive wood run wild: these pastoral farms,\\
Green to the very door; and wreaths of smoke\\
Sent up, in silence, from among the trees!\\
With some uncertain notice, as might seem\\
Of vagrant dwellers in the houseless woods,\\
Or of some hermit's cave, where by his fire\\
The hermit sits alone. These beauteous forms,\\
Through a long absence, have not been to me\\
As is a landscape to a blind man's eye:\\
But oft, in lonely rooms, and 'mid the din\\
Of towns and cities, I have owed to them,\\
In hours of weariness, sensations sweet,\\
Felt in the blood, and felt along the heart;\\
And passing even into my purer mind\\
With tranquil restoration: feelings too\\
Of unremembered pleasure: such, perhaps,\\
As have no slight or trivial influence\\
On that best portion of a good man's life,\\
His little, nameless, unremembered, acts\\
Of kindness and of love. Nor less, I trust,\\
To them I may have owed another gift,\\
Of aspect more sublime; that blessed mood,\\
In which the burthen of the mystery,\\
In which the heavy and the weary weight\\
Of all this unintelligible world,\\
Is lightened: that serene and bless\`{e}d mood,\\
In which the affections gently lead us on,\\
Until, the breath of this corporeal frame\\
And even the motion of our human blood\\
Almost suspended, we are laid asleep\\
In body, and become a living soul:\\
While with an eye made quiet by the power\\
Of harmony, and the deep power of joy,\\
We see into the life of things. If this\\
Be but a vain belief, yet, O! how oft\\
In darkness and amid the many shapes\\
Of joyless daylight; when the fretful stir\\
Unprofitable, and the fever of the world,\\
Have hung upon the beatings of my heart\\
How oft, in spirit, have I turned to thee,\\
O sylvan \textsc{Wye}! thou wanderer through the woods,\\*
How often has my spirit turned to thee!\\!

And now, with gleams of half-extinguished thought,\\*
With many recognitions dim and faint,\\
And somewhat of a sad perplexity,\\
The picture of the mind revives again:\\
While here I stand, not only with the sense\\
Of present pleasure, but with pleasing thoughts\\
That in this moment there is life and food\\
For future years. And so I dare to hope,\\
Though changed, no doubt, from what I was when first\\
I came among these hills; when like a roe\\
I bounded o'er the mountains, by the sides\\
Of the deep rivers, and the lonely streams,\\
Wherever nature led: more like a man\\
Flying from something that he dreads, than one\\
Who sought the thing he loved. For nature then\\
(The coarser pleasures of my boyish days\\
And their glad animal movements all gone by)\\
To me was all in all. I cannot paint\\
What then I was. The sounding cataract\\
Haunted me like a passion: the tall rock,\\
The mountain, and the deep and gloomy wood,\\
Their colours and their forms, were then to me\\
An appetite; a feeling and a love,\\
That had no need of a remoter charm,\\
By thought supplied, not any interest\\
Unborrowed from the eye. That time is past,\\
And all its aching joys are now no more,\\
And all its dizzy raptures. Not for this\\
Faint I, nor mourn nor murmur; other gifts\\
Have followed; for such loss, I would believe,\\
Abundant recompense. For I have learned\\
To look on nature, not as in the hour\\
Of thoughtless youth; but hearing oftentimes\\
The still sad music of humanity,\\
Nor harsh nor grating, though of ample power\\
To chasten and subdue. And I have felt\\
A presence that disturbs me with the joy\\
Of elevated thoughts; a sense sublime\\
Of something far more deeply interfused,\\
Whose dwelling is the light of setting suns,\\
And the round ocean and the living air,\\
And the blue sky, and in the mind of man:\\
A motion and a spirit, that impels\\
All thinking things, all objects of all thought,\\
And rolls through all things. Therefore am I still\\
A lover of the meadows \& the woods\\
And mountains; and of all that we behold\\
From this green earth; of all the mighty world\\
Of eye, and ear, both what they half create,\\
And what perceive; well pleased to recognise\\
In nature and the language of the sense\\
The anchor of my purest thoughts, the nurse,\\
The guide, the guardian of my heart, and soul\\
Of all my moral being. Nor perchance,\\
If I were not thus taught, should I the more\\
Suffer my genial spirits to decay:\\
For thou art with me here upon the banks\\
Of this fair river; thou my dearest friend,\\
My dear, dear friend; and in thy voice I catch\\
The language of my former heart, and read\\
My former pleasures in the shooting lights\\
Of thy wild eyes. O! yet a little while\\
May I behold in thee what I was once,\\
My dear, dear sister! and this prayer I make,\\
Knowing that nature never did betray\\
The heart that loved her; 'tis her privilege,\\
Through all the years of this our life, to lead\\
From joy to joy: for she can so inform\\
The mind that is within us, so impress\\
With quietness and beauty, and so feed\\
With lofty thoughts, that neither evil tongues,\\
Rash judgments, nor the sneers of selfish men,\\
Nor greetings where no kindness is, nor all\\
The dreary intercourse of daily life,\\
Shall e'er prevail against us, or disturb\\
Our cheerful faith, that all which we behold\\
Is full of blessings. Therefore let the moon\\
Shine on thee in thy solitary walk;\\
And let the misty mountain-winds be free\\
To blow against thee: and, in after years,\\
When these wild ecstasies shall be matured\\
Into a sober pleasure; when thy mind\\
Shall be a mansion for all lovely forms,\\
Thy memory be as a dwelling-place\\
For all sweet sounds and harmonies; O! then,\\
If solitude, or fear, or pain, or grief,\\
Should be thy portion, with what healing thoughts\\
Of tender joy wilt thou remember me,\\
And these my exhortations! Nor, perchance\\
If I should be where I no more can hear\\
Thy voice, nor catch from thy wild eyes these gleams\\
Of past existence -- wilt thou then forget\\
That on the banks of this delightful stream\\
We stood together; and that I, so long\\
A worshipper of nature, hither came\\
Unwearied in that service: rather say\\
With warmer love -- O! with far deeper zeal\\
Of holier love. Nor wilt thou then forget,\\
That after many wanderings, many years\\
Of absence, these steep woods and lofty cliffs,\\
And this green pastoral landscape, were to me\\*
More dear, both for themselves and for thy sake!
\end{verse}

\subsection{}

\blfootnote{William Yeats (1865 -- 1939), \cite{norton}. This poem is a translation of a sonnet by the French poet Pierre de Ronsard; the Almanacker is not qualified to judge how faithful it is.}\settowidth{\versewidth}{    And nodding by the fire, take down this book,}
\begin{verse}[\versewidth]
When you are old \& grey \& full of sleep,\\*
\vin And nodding by the fire, take down this book,\\
\vin And slowly read, and dream of the soft look\\*
Your eyes had once, and of their shadows deep;\\!

How many loved your moments of glad grace,\\*
\vin And loved your beauty with love false or true,\\
\vin But one man loved the pilgrim soul in you,\\*
And loved the sorrows of your changing face;\\!

And bending down beside the glowing bars,\\*
\vin Murmur, a little sadly, how love fled\\
\vin And paced upon the mountains overhead\\*
And hid his face amid a crowd of stars.
\end{verse}

\subsection{}

\blfootnote{John Adams, 2nd President of the United States (1735 -- 1826), \cite{odq}. Adams wrote these words in a letter to Thomas Jefferson, 3rd President of the United States.}You and I ought not to die before we have explained ourselves to each other.

\section{}

\subsection{}

\blfootnote{Dr William Wordsworth, Poet Laureate (1770 -- 1850), \cite{norton}. The poet needs someone to interpret the song for him because the girl is singing in Gallic.}\settowidth{\versewidth}{    Perhaps the plaintive numbers flow}
\begin{verse}[\versewidth]
Behold her, single in the field,\\*
\vin Yon solitary highland lass,\\
Reaping and singing by herself.\\
\vin Stop here, or gently pass!\\
Alone she cuts \& binds the grain,\\
And sings a melancholy strain;\\
O listen, for the vale profound\\*
Is overflowing with the sound.\\!

No nightingale did ever chaunt\\*
\vin More welcome notes to weary bands\\
Of travellers in some shady haunt,\\
\vin Among arabian sands:\\
A voice so thrilling ne'er was heard\\
In spring-time from the cuckoo-bird,\\
Breaking the silence of the seas\\*
Among the farthest Hebrides.\\!

Will no one tell me what she sings?\\*
\vin Perhaps the plaintive numbers flow\\
For old, unhappy, far-off things,\\
\vin And battles long ago:\\
Or is it some more humble lay,\\
Familiar matter of today?\\
Some natural sorrow, loss, or pain,\\*
That has been, and may be again?\\!

Whate'er the theme, the maiden sang\\*
\vin As if her song could have no ending;\\
I saw her singing at her work,\\
\vin And o'er the sickle bending;\\
I listened, motionless \& still;\\
And, as I mounted up the hill,\\
The music in my heart I bore,\\*
Long after it was heard no more.
\end{verse}

\subsection{}

\blfootnote{`George Crabbe', Edwin Robinson (1869 -- 1935), \cite{norton}. The Rev George Crabbe was an English poet (and also a surgeon and, later, vicar) of the early nineteenth century.}\settowidth{\versewidth}{With the sure strength that fearless truth endows.}
\begin{verse}[\versewidth]
Give him the darkest inch your shelf allows,\\*
\vin Hide him in lonely garrets, if you will,\\
\vin But his hard, human pulse is throbbing still\\
With the sure strength that fearless truth endows.\\
In spite of all fine science disavows,\\
\vin Of his plain excellence \& stubborn skill\\
\vin There yet remains what fashion cannot kill,\\*
Though years have thinned the laurel from his brows.\\!

Whether or not we read him, we can feel\\*
\vin From time to time the vigour of his name\\
\vin Against us like a finger for the shame\\
And emptiness of what our souls reveal\\
In books that are as altars where we kneel\\*
\vin To consecrate the flicker, not the flame.
\end{verse}

\subsection{}

\blfootnote{Anonymous, \cite{treasury}. This couplet is taken from a longer poem, which Palgrave names \refpoem{Present in Absence}. The poem is sometimes attributed to the Rev Dr Donne; Prof Sir Herbert Grierson attributes it to John Hoskins.}\settowidth{\versewidth}{For hearts of truest mettle}
\begin{verse}[\versewidth]
For hearts of truest mettle\\*
Absence doth join, and time doth settle.
\end{verse}

\section{}

\subsection{}

\blfootnote{`The Garden of Prosperpine', Algernon Swinburne (1837 -- 1909), \cite{norton}. Proserpine was the consort of Pluto, lord of the underworld.}\settowidth{\versewidth}{    The old loves with wearier wings;}
\begin{verse}[\versewidth]
Here, where the world is quiet;\\*
\vin Here, where all trouble seems\\
Dead winds' \& spent waves' riot\\
\vin In doubtful dreams of dreams;\\
I watch the green field growing\\
For reaping folk \& sowing,\\
For harvest-time \& mowing,\\*
\vin A sleepy world of streams.\\!

I am tired of tears and laughter,\\*
\vin And men that laugh and weep;\\
Of what may come hereafter\\
\vin For men that sow to reap:\\
I am weary of days \& hours,\\
Blown buds of barren flowers,\\
Desires \& dreams \& powers\\*
\vin And everything but sleep.\\!

Here life has death for neighbour,\\*
\vin And far from eye or ear\\
Wan waves \& wet winds labour,\\
\vin Weak ships and spirits steer;\\
They drive adrift, and whither\\
They wot not who make thither;\\
But no such winds blow hither,\\*
\vin And no such things grow here.\\!

No growth of moor or coppice,\\*
\vin No heather-flower or vine,\\
But bloomless buds of poppies,\\
\vin Green grapes of \textit{Proserpine},\\
Pale beds of blowing rushes\\
Where no leaf blooms or blushes\\
Save this whereout she crushes\\*
\vin For dead men deadly wine.\\!

Pale, without name or number,\\*
\vin In fruitless fields of corn,\\
They bow themselves \& slumber\\
\vin All night till light is born;\\
And like a soul belated,\\
In hell \& heaven unmated,\\
By cloud \& mist abated\\*
\vin Comes out of darkness morn.\\!

Though one were strong as seven,\\*
\vin He too with death shall dwell,\\
Nor wake with wings in heaven,\\
\vin Nor weep for pains in hell;\\
Though one were fair as roses,\\
His beauty clouds \& closes;\\
And well though love reposes,\\*
\vin In the end it is not well.\\!

Pale, beyond porch \& portal,\\*
\vin Crowned with calm leaves, she stands\\
Who gathers all things mortal\\
\vin With cold immortal hands;\\
Her languid lips are sweeter\\
Than love's who fears to greet her\\
To men that mix and meet her\\*
\vin From many times \& lands.\\!

She waits for each and other,\\*
\vin She waits for all men born;\\
Forgets the earth her mother,\\
\vin The life of fruits \& corn;\\
And spring \& seed \& swallow\\
Take wing for her and follow\\
Where summer song rings hollow\\*
\vin And flowers are put to scorn.\\!

There go the loves that wither,\\*
\vin The old loves with wearier wings;\\
And all dead years draw thither,\\
\vin And all disastrous things;\\
Dead dreams of days forsaken,\\
Blind buds that snows have shaken,\\
Wild leaves that winds have taken,\\*
\vin Red strays of ruined springs.\\!

We are not sure of sorrow,\\*
\vin And joy was never sure;\\
To-day will die to-morrow;\\
\vin Time stoops to no man's lure;\\
And love, grown faint and fretful,\\
With lips but half regretful\\
Sighs, and with eyes forgetful\\*
\vin Weeps that no loves endure.\\!

From too much love of living,\\*
\vin From hope and fear set free,\\
We thank with brief thanksgiving\\
\vin Whatever gods may be\\
That no life lives for ever;\\
That dead men rise up never;\\
That even the weariest river\\*
\vin Winds somewhere safe to sea.\\!

Then star nor sun shall waken,\\*
\vin Nor any change of light:\\
Nor sound of waters shaken,\\
\vin Nor any sound or sight:\\
Nor wintry leaves nor vernal,\\
Nor days nor things diurnal;\\
Only the sleep eternal\\*
\vin In an eternal night.
\end{verse}

\subsection{}

\blfootnote{`Pied Beauty', Fr Gerard Hopkins (1844 -- 1889), \cite{norton}.}\settowidth{\versewidth}{    Landscape plotted pieced -- fold, fallow, and plough;}
\begin{verse}[\versewidth]
Glory be to God for dappled things --\\*
\vin For skies of couple-colour as a brinded cow;\\
\vin \vin For rose-moles all in stipple upon trout that swim;\\
Fresh-firecoal chestnut-falls; finches' wings;\\
\vin Landscape plotted \& pieced -- fold, fallow, and plough;\\*
\vin \vin And \'{a}ll tr\'{a}des, their gear \& tackle \& trim.\\!

All things counter, original, spare, strange;\\*
\vin Whatever is fickle, freckled (who knows how?)\\
\vin \vin With swift, slow; sweet, sour; adazzle, dim;\\
He fathers-forth whose beauty is past change:\\*
\vin \vin \vin \vin Praise him.
\end{verse}

\subsection{}

\blfootnote{Francis Bradley (1846 -- 1924), \cite{odq}.}Where everything is bad it must be good to know the worst.

\section{}

\subsection{}

\blfootnote{`A Forsaken Garden', Algernon Swinburne (1837 -- 1909), \cite{norton}.}\settowidth{\versewidth}{For the foam-flowers endure when the rose-blossoms wither,}
\begin{verse}[\versewidth]
In a coign of the cliff between lowland \& highland,\\*
\vin At the sea-down's edge between windward \& lee,\\
Walled round with rocks as an inland island,\\
\vin The ghost of a garden fronts the sea.\\
A girdle of brushwood \& thorn encloses\\
\vin The steep square slope of the blossomless bed\\
Where the weeds that grew green from the graves of its roses\\*
\vin \vin \vin Now lie dead.\\!

The fields fall southward, abrupt \& broken,\\*
\vin To the low last edge of the long lone land.\\
If a step should sound or a word be spoken,\\
\vin Would a ghost not rise at the strange guest's hand?\\
So long have the grey bare walks lain guestless,\\
\vin Through branches \& briars if a man make way,\\
He shall find no life but the sea-wind's, restless\\*
\vin \vin \vin Night \& day.\\!

The dense hard passage is blind \& stifled\\*
\vin That crawls by a track none turn to climb\\
To the strait waste place that the years have rifled\\
\vin Of all but the thorns that are touched not of time.\\
The thorns he spares when the rose is taken;\\
\vin The rocks are left when he wastes the plain.\\
The wind that wanders, the weeds wind-shaken,\\*
\vin \vin \vin These remain.\\!

Not a flower to be pressed of the foot that falls not;\\*
\vin As the heart of a dead man the seed-plots are dry;\\
From the thicket of thorns whence the nightingale calls not,\\
\vin Could she call, there were never a rose to reply.\\
Over the meadows that blossom \& wither\\
\vin Rings but the note of a sea-bird's song;\\
Only the sun \& the rain come hither\\*
\vin \vin \vin All year long.\\!

The sun burns sere and the rain dishevels\\*
\vin One gaunt bleak blossom of scentless breath.\\
Only the wind here hovers \& revels\\
\vin In a round where life seems barren as death.\\
Here there was laughing of old, there was weeping,\\
\vin Haply, of lovers none ever will know,\\
Whose eyes went seaward a hundred sleeping\\*
\vin \vin \vin Years ago.\\!

Heart handfast in heart as they stood, `Look thither,'\\*
\vin Did he whisper? 'Look forth from the flowers to the sea;\\
For the foam-flowers endure when the rose-blossoms wither,\\
\vin And men that love lightly may die -- but we?'\\
And the same wind sang and the same waves whitened,\\
\vin And or ever the garden's last petals were shed,\\
In the lips that had whispered, the eyes that had lightened,\\*
\vin \vin \vin Love was dead.\\!

Or they loved their life through, and then went whither?\\*
\vin And were one to the end—but what end who knows?\\
Love deep as the sea as a rose must wither,\\
\vin As the rose-red seaweed that mocks the rose.\\
Shall the dead take thought for the dead to love them?\\
\vin What love was ever as deep as a grave?\\
They are loveless now as the grass above them\\*
\vin \vin \vin Or the wave.\\!

All are at one now, roses \& lovers,\\*
\vin Not known of the cliffs \& the fields \& the sea.\\
Not a breath of the time that has been hovers\\
\vin In the air now soft with a summer to be.\\
Not a breath shall there sweeten the seasons hereafter\\
\vin Of the flowers or the lovers that laugh now or weep,\\
When as they that are free now of weeping \& laughter\\*
\vin \vin \vin We shall sleep.\\!

Here death may deal not again for ever;\\*
\vin Here change may come not till all change end.\\
From the graves they have made they shall rise up never,\\
\vin Who have left nought living to ravage \& rend.\\
Earth, stones, \& thorns of the wild ground growing,\\
\vin While the sun \& the rain live, these shall be;\\
Till a last wind's breath upon all these blowing\\*
\vin \vin \vin Roll the sea.\\!

Till the slow sea rise and the sheer cliff crumble,\\*
\vin Till terrace \& meadow the deep gulfs drink,\\
Till the strength of the waves of the high tides humble\\
\vin The fields that lessen, the rocks that shrink,\\
Here now in his triumph where all things falter,\\
\vin Stretched out on the spoils that his own hand spread,\\
As a god self-slain on his own strange altar,\\*
\vin \vin \vin Death lies dead.
\end{verse}

\subsection{}

\blfootnote{`Spring and Fall', Fr Gerard Hopkins (1844 -- 1889), \cite{norton}. There are several places in England and Wales known as Golden Grove (or some variation thereupon), the most famous of which is the Golden Grove estate in Camarthenshire. It's unclear, at least to the Almanacker, which Golden Grove Fr Hopkins had in mind.}\settowidth{\versewidth}{Though worlds of wan-wood leaf-meal lie;}
\begin{verse}[\versewidth]
\textit{M\'{a}rgar\'{e}t}, \'{a}re you gr\'{i}eving\\*
Over \textsc{Goldengrove} unleaving?\\
Le\'{a}ves like the things of man, you\\
With your fresh thoughts care for, can you?\\
Ah \'{a}s the heart grows older\\
It will come to such sights colder\\
By \& by, nor spare a sigh\\
Though worlds of wan-wood leaf-meal lie;\\
And yet you w\'{i}ll weep and know why.\\
Now no matter, child, the name:\\
S\'{o}rrow's spr\'{i}ngs \'{a}re the same.\\
Nor mouth had, no nor mind, expressed\\
What heart heard of, ghost guessed:\\
It \'{i}s the blight man was born for,\\*
It is \textit{Margaret} you mourn for.
\end{verse}

\subsection{}

\blfootnote{Charles Bowen, Baron Bowen (1835 -- 1894), \cite{odq}.}I am reminded of a blind man in a dark room -- looking for a black hat -- which isn't there.

\section{}

\subsection{}

\blfootnote{`Light Shining out of Darkness', William Cowper (1731 -- 1800), \cite{norton}.}\settowidth{\versewidth}{Judge not the Lord by feeble sense,}
\begin{verse}[\versewidth]
God moves in a mysterious way\\*
\vin His wonders to perform;\\
He plants his footsteps in the sea\\*
\vin And rides upon the storm.\\!

Deep in unfathomable mines\\*
\vin Of never-failing skill,\\
He treasures up his bright designs\\*
\vin And works his sovereign will.\\!

Ye fearful saints, fresh courage take;\\*
\vin The clouds ye so much dread\\
Are big with mercy and shall break\\*
\vin In blessings on your head.\\!

Judge not the Lord by feeble sense,\\*
\vin But trust him for his grace;\\
Behind a frowning providence\\*
\vin He hides a smiling face.\\!

His purposes will ripen fast,\\*
\vin Unfolding every hour;\\
The bud may have a bitter taste,\\*
\vin But sweet will be the flower.\\!

Blind unbelief is sure to err\\*
\vin And scan his work in vain;\\
God is his own interpreter,\\*
\vin And he will make it plain.
\end{verse}

\subsection{}

\blfootnote{`Vita summa brevis spem nos vetat incohare longam', Ernest Dowson (1867 -- 1900), \cite{norton}. The title is a quotation from Horace, \refbook{Odes} 1.4 -- `The shortness of life forbids us any long-term hopes' -- the truth of which Dowson, with his short, occasionally brilliant, mostly miserable life, knew all too well.}\settowidth{\versewidth}{They are not long, the weeping the laughter,}
\begin{verse}[\versewidth]
They are not long, the weeping \& the laughter,\\*
\vin Love \& desire \& hate:\\
I think they have no portion in us after\\*
\vin We pass the gate.\\!

They are not long, the days of wine \& roses:\\*
\vin Out of a misty dream\\
Our path emerges for a while, then closes\\*
\vin Within a dream.
\end{verse}

\subsection{}

\blfootnote{The Rev Dr Drummond Chase (1820 -- 1902), \cite{nicomachean}. The Rev Dr Chase is here translating a remark from Aristotle.}We toil that we may rest, and war that we may be at peace.

\section{}

\subsection{}

\blfootnote{`The Universal Prayer', Alexander Pope (1688 -- 1744), \cite{norton}. The Almanacker has taken the liberty of removing some of the weaker verses.}\settowidth{\versewidth}{Thou first great cause, least understood:}
\begin{verse}[\versewidth]
Thou first great cause, least understood:\\*
\vin Who all my sense confined,\\
To know but this -- that thou art good,\\*
\vin And that myself am blind:\\!

What blessings thy free bounty gives,\\*
\vin Let me not cast away;\\
For God is paid when man receives,\\*
\vin To enjoy is to obey.\\!

Let not this weak, unknowing hand\\*
\vin Presume thy bolts to throw,\\
And deal damnation round the land,\\*
\vin On each I judge thy foe.\\!

Teach me to feel another's woe,\\*
\vin To hide the fault I see;\\
That mercy I to others show,\\*
\vin That mercy show to me.\\!

This day, be bread and peace my lot:\\*
\vin All else beneath the sun,\\
Thou know’st if best bestowed or not,\\*
\vin And let thy will be done.\\!

To thee, whose temple is all space,\\*
\vin Whose altar, earth, sea, skies!\\
One chorus let all being raise!\\*
\vin All nature's incense rise!
\end{verse}

\subsection{}

\blfootnote{Rupert Brooke (1887 -- 1915), \cite{norton}. The Almanacker was encouraged to hate this poem at school as an example of the mindless jingoism that led to the Great War in the first place; and, of course, there is something idiotic about it. But there's something noble and beautiful in it too.}\settowidth{\versewidth}{    Gave, once, her flowers to love, her ways to roam,}
\begin{verse}[\versewidth]
If I should die, think only this of me:\\*
\vin That there's some corner of a foreign field\\
That is forever England. There shall be\\
\vin In that rich earth a richer dust concealed;\\
A dust whom England bore, shaped, made aware,\\
\vin Gave, once, her flowers to love, her ways to roam,\\
A body of England's, breathing english air,\\*
\vin Washed by the rivers, blest by the suns of home.\\!

And think, this heart, all evil shed away,\\*
\vin A pulse in the eternal mind, no less\\
\vin \vin Gives somewhere back the thoughts by England given;\\
Her sights \& sounds; dreams happy as her day;\\
\vin And laughter, learnt of friends; and gentleness,\\*
\vin \vin In hearts at peace, under an english heaven.
\end{verse}

\subsection{}

\blfootnote{Rupert Brooke (1887 -- 1915), \cite{odq}.}History repeats itself; historians repeat one another.

\section{}

\subsection{}

\blfootnote{Isaac Watts (1674 -- 1748), \cite{norton}. This poem is a paraphrase of Psalm 90. It is often sung as `O God, our help in ages past', and this may be a slight improvement.}\settowidth{\versewidth}{Thy word commands our flesh to dust;}
\begin{verse}[\versewidth]
Our God, our help in ages past,\\*
\vin Our hope for years to come,\\
Our shelter from the stormy blast,\\*
\vin And our eternal home:\\!

Under the shadow of thy throne\\*
\vin Thy saints have dwelt secure;\\
Sufficient is thine arm alone,\\*
\vin And our defense is sure.\\!

Before the hills in order stood\\*
\vin Or earth received her frame,\\
From everlasting thou art God,\\*
\vin To endless years the same.\\!

A thousand ages in thy sight\\*
\vin Are like an evening gone,\\
Short as the watch that ends the night\\*
\vin Before the rising sun.\\!

Thy word commands our flesh to dust;\\*
\vin Return, ye sons of men.\\
All nations rose from earth at first\\*
\vin And turn to earth again.\\!

Time, like an ever-rolling stream,\\*
\vin Bears all its sons away;\\
They fly forgotten as a dream\\*
\vin Dies at the opening day.\\!

Our God, our help in ages past,\\*
\vin Our hope for years to come,\\
Be thou our guard while troubles last,\\*
\vin And our eternal home.
\end{verse}

\subsection{}

\blfootnote{`The Hill Summit', Gabriel Rossetti (1828 -- 1882), \cite{norton}.}\settowidth{\versewidth}{    In the broad west has blazed for vesper-song;}
\begin{verse}[\versewidth]
This feast-day of the sun, his altar there\\*
\vin In the broad west has blazed for vesper-song;\\
\vin And I have loitered in the vale too long\\
And gaze now a belated worshipper.\\
Yet may I not forget that I was 'ware,\\
\vin So journeying, of his face at intervals\\
\vin Transfigured where the fringed horizon falls\\
A fiery bush with coruscating hair.\\
And now that I have climbed \& won this height,\\
\vin I must tread downward through the sloping shade\\
And travel the bewildered tracks till night.\\
\vin Yet for this hour I still may here be stayed\\
\vin And see the gold air \& the silver fade\\*
And the last bird fly into the last light.
\end{verse}

\subsection{}

\blfootnote{Sir Thomas Browne (1605 -- 1682), \cite{odq}.}We all labour against our own cure, for death is the cure of all diseases.

\section{}

\subsection{}

\blfootnote{`Song: Love Lives Beyond', John Clare (1793 -- 1864), \cite{norton}.}\settowidth{\versewidth}{The tomb, the earth, the flowers, dew.}
\begin{verse}[\versewidth]
\vin Love lives beyond\\*
The tomb, the earth, which fades like dew --\\
\vin I love the fond,\\*
The faithful, and the true.\\!

\vin Love lives in sleep;\\*
'Tis happiness of healthy dreams;\\
\vin Eve's dews may weep,\\*
But love delightful seems.\\!

\vin 'Tis seen in flowers,\\*
And in the even's pearly dew,\\
\vin On earth's green hours,\\*
And in the heaven's eternal blue.\\!

\vin 'Tis heard in spring\\*
When light \& sunbeams, warm \& kind,\\
\vin On angels' wing\\*
Bring love and music to the wind.\\!

\vin And where is voice,\\*
So young, so beautiful \& sweet\\
\vin As nature's choice,\\*
Where spring and lovers meet?\\!

\vin Love lives beyond\\*
The tomb, the earth, the flowers, \& dew.\\
\vin I love the fond,\\*
The faithful, young \& true.
\end{verse}

\subsection{}

\blfootnote{John Clare (1793 -- 1864), \cite{norton}. These are the closing lines of Clare's \refpoem{I Am}.}\settowidth{\versewidth}{I long for scenes, where man hath never trod,}
\begin{verse}[\versewidth]
I long for scenes, where man hath never trod,\\*
\vin A place where woman never smiled or wept,\\
There to abide with my Creator, God,\\
\vin And sleep as I in childhood sweetly slept,\\
Untroubling and untroubled where I lie:\\*
The grass below, above the vaulted sky.
\end{verse}

\subsection{}

\blfootnote{Sir Thomas Browne (1605 -- 1682), \cite{odq}.}The long habit of living indisposeth us for dying.

\section{}

\subsection{}

\blfootnote{`The World', Dr Henry Vaughan (1621 -- 1695), \cite{norton}. Dr Vaughan affixed a quotation from John's Gospel (2.16-17) to the end of this poem.}\settowidth{\versewidth}{And round beneath it, time in hours, days, years,}
\begin{verse}[\versewidth]
I saw eternity the other night,\\*
Like a great ring of pure and endless light,\\
\vin All calm, as it was bright;\\
And round beneath it, time in hours, days, years,\\
\vin Driven by the spheres\\
Like a vast shadow moved; in which the world\\
\vin And all her train were hurled.\\
The doting lover in his quaintest strain\\
\vin Did there complain;\\
Near him, his lute, his fancy, and his flights,\\
\vin Wit's sour delights,\\
With gloves, and knots, the silly snares of pleasure,\\
\vin Yet his dear treasure\\
All scattered lay, while he his eyes did pour\\*
\vin Upon a flower.\\!

The darksome statesman hung with weights \& woe,\\*
Like a thick midnight-fog moved there so slow,\\
\vin He did not stay, nor go;\\
Condemning thoughts (like sad eclipses) scowl\\
\vin Upon his soul,\\
And clouds of crying witnesses without\\
\vin Pursued him with one shout.\\
Yet digged the mole, and lest his ways be found,\\
\vin Worked under ground,\\
Where he did clutch his prey; but one did see\\
\vin That policy;\\
Churches \& altars fed him; perjuries\\
\vin Were gnats \& flies;\\
It rained about him blood and tears, but he\\*
\vin Drank them as free.\\!

The fearful miser on a heap of rust\\*
Sate pining all his life there, did scarce trust\\
\vin His own hands with the dust,\\
Yet would not place one piece above, but lives\\
\vin In fear of thieves;\\
Thousands there were as frantic as himself,\\
\vin And hugged each one his pelf;\\
The downright epicure placed heaven in sense,\\
\vin And scorned pretence,\\
While others, slipped into a wide excess,\\
\vin Said little less;\\
The weaker sort slight, trivial wares enslave,\\
\vin Who think them brave;\\
And poor despised truth sate counting by\\*
\vin Their victory.\\!

Yet some, who all this while did weep and sing,\\*
And sing, and weep, soar'd up into the ring;\\
\vin But most would use no wing.\\
O fools (said I) thus to prefer dark night\\
\vin Before true light,\\
To live in grots \& caves, and hate the day\\
\vin Because it shows the way,\\
The way, which from this dead \& dark abode\\
\vin Leads up to God,\\
A way where you might tread the sun, and be\\
\vin More bright than he.\\
But as I did their madness so discuss\\
\vin One whispered thus,\\
This ring the Bridegroom did for none provide,\\*
\vin But for his bride.
\end{verse}

\subsection{}

\blfootnote{Samuel Daniel (1562 -- 1619), \cite{norton}.}\settowidth{\versewidth}{When yet th' unborn shall say, `Lo where she lies}
\begin{verse}[\versewidth]
Let others sing of knights \& paladins\\*
\vin In ag\`{e}d accents \& untimely words;\\
Paint shadows in imaginary lines\\
\vin Which well the reach of their high wits records:\\
But I must sing of thee, and those fair eyes\\
\vin Authentic shall my verse in time to come,\\
When yet th' unborn shall say, `Lo where she lies\\
\vin Whose beauty made him speak that else was dumb.'\\
These are the arks, the trophies I erect,\\
\vin That fortify thy name against old age;\\
And these thy sacred virtues must protect\\
\vin Against the dark, and time's consuming rage.\\
Though th' error of my youth they shall discover,\\*
Suffice they show I lived and was thy lover.
\end{verse}

\subsection{}

\blfootnote{Sir Thomas Browne (1605 -- 1682), \cite{odq}.}Generations will pass while some trees stand, and old families last not three oaks.

\section{}

\subsection{}

\blfootnote{`After Apple Picking', Robert Frost, Poet Laureate of Vermont (1874 -- 1963), \cite{norton}.}\settowidth{\versewidth}{My long two-pointed ladder's sticking through a tree}
\begin{verse}[\versewidth]
My long two-pointed ladder's sticking through a tree\\*
Toward heaven still,\\
And there's a barrel that I didn't fill\\
Beside it, and there may be two or three\\
Apples I didn't pick upon some bough.\\
But I am done with apple-picking now.\\
Essence of winter sleep is on the night,\\
The scent of apples: I am drowsing off.\\
I cannot rub the strangeness from my sight\\
I got from looking through a pane of glass\\
I skimmed this morning from the drinking trough\\
And held against the world of hoary grass.\\
It melted, and I let it fall \& break.\\
But I was well\\
Upon my way to sleep before it fell,\\
And I could tell\\
What form my dreaming was about to take.\\
Magnified apples appear \& disappear,\\
Stem end \& blossom end,\\
And every fleck of russet showing clear.\\
My instep arch not only keeps the ache,\\
It keeps the pressure of a ladder-round.\\
I feel the ladder sway as the boughs bend.\\
And I keep hearing from the cellar bin\\
The rumbling sound\\
Of load on load of apples coming in.\\
For I have had too much\\
Of apple-picking: I am overtired\\
Of the great harvest I myself desired.\\
There were 10,000 thousand fruit to touch,\\
Cherish in hand, lift down, and not let fall.\\
For all\\
That struck the earth,\\
No matter if not bruised or spiked with stubble,\\
Went surely to the cider-apple heap\\
As of no worth.\\
One can see what will trouble\\
This sleep of mine, whatever sleep it is.\\
Were he not gone,\\
The woodchuck could say whether it's like his\\
Long sleep, as I describe its coming on,\\*
Or just some human sleep.
\end{verse}

\subsection{}

\blfootnote{Edmund Waller (1606 -- 1687), \cite{norton}. This is the final verse of \refpoem{Of the Last Verses in the Book}.}\settowidth{\versewidth}{Leaving the old, both worlds at once they view,}
\begin{verse}[\versewidth]
The soul's dark cottage, battered \& decayed,\\*
Lets in new light through chinks that time has made;\\
Stronger by weakness, wiser men become\\
As they draw near to their eternal home:\\
Leaving the old, both worlds at once they view,\\*
That stand upon the threshold of the new.
\end{verse}

\subsection{}

\blfootnote{Sir Thomas Browne (1605 -- 1682), \cite{odq}.}But how shall we expect charity towards others when we are so uncharitable to ourselves?

\section{}

\subsection{}

\blfootnote{`The Road Not Taken', Robert Frost, Poet Laureate of Vermont (1874 -- 1963), \cite{norton}. Frost wrote this poem as a kind of parody, based on an in-joke between himself and his friend Edward Thomas (or, at least, Frost used to claim as much, but poets often have mixed feelings towards their most famous works).}\settowidth{\versewidth}{    To where it bent in the undergrowth;}
\begin{verse}[\versewidth]
Two roads diverged in a yellow wood,\\*
\vin And sorry I could not travel both\\
And be one traveler, long I stood\\
And looked down one as far as I could\\*
\vin To where it bent in the undergrowth;\\!

Then took the other, as just as fair,\\*
\vin And having perhaps the better claim,\\
Because it was grassy \& wanted wear;\\
Though as for that the passing there\\*
\vin Had worn them really about the same,\\!

And both that morning equally lay\\*
\vin In leaves no step had trodden black.\\
O I kept the first for another day!\\
Yet knowing how way leads on to way,\\*
\vin I doubted if I should ever come back.\\!

I shall be telling this with a sigh\\*
\vin Somewhere ages \& ages hence:\\
Two roads diverged in a wood, and I --\\
I took the one less traveled by,\\*
\vin And that has made all the difference.
\end{verse}

\subsection{}

\blfootnote{Percy Shelley (1792 -- 1822), \cite{norton}. This is \refpoem{Ode to the West Wind} \S 5.}\settowidth{\versewidth}{Like withered leaves to quicken a new birth!}
\begin{verse}[\versewidth]
Make me thy lyre, even as the forest is:\\*
What if my leaves are falling like its own!\\*
The tumult of thy mighty harmonies\\!

Will take from both a deep, autumnal tone,\\*
Sweet though in sadness. Be thou, spirit fierce,\\*
My spirit! Be thou me, impetuous one!\\!

Drive my dead thoughts over the universe\\*
Like withered leaves to quicken a new birth!\\*
And, by the incantation of this verse,\\!

Scatter, as from an unextinguished hearth\\*
Ashes and sparks, my words among mankind!\\*
Be through my lips to unawakened earth\\!

The trumpet of a prophecy! O wind,\\*
If winter comes, can spring be far behind?
\end{verse}

\subsection{}

\blfootnote{Sir Thomas Browne (1605 -- 1682), \cite{odq}.}A man may be in as just possession of truth as of a city, and yet be forced to surrender.

\section{}

\subsection{}

\blfootnote{`To Autumn', John Keats (1795 -- 1821), \cite{norton}.}\settowidth{\versewidth}{    With fruit the vines that round the thatch-eves run;}
\begin{verse}[\versewidth]
Season of mists and mellow fruitfulness,\\*
\vin Close bosom-friend of the maturing sun;\\
Conspiring with him how to load \& bless\\
\vin With fruit the vines that round the thatch-eves run;\\
To bend with apples the mossed cottage-trees,\\
\vin And fill all fruit with ripeness to the core;\\
\vin \vin To swell the gourd, and plump the hazel shells\\
\vin With a sweet kernel; to set budding more,\\
And still more, later flowers for the bees,\\
Until they think warm days will never cease,\\*
\vin \vin For summer has o'er-brimmed their clammy cells.\\!

Who hath not seen thee oft amid thy store?\\*
\vin Sometimes whoever seeks abroad may find\\
Thee sitting careless on a granary floor,\\
\vin Thy hair soft-lifted by the winnowing wind;\\
Or on a half-reaped furrow sound asleep,\\
\vin Drowsed with the fume of poppies, while thy hook\\
\vin \vin Spares the next swath and all its twin\`{e}d flowers:\\
And sometimes like a gleaner thou dost keep\\
\vin Steady thy laden head across a brook;\\
\vin Or by a cyder-press, with patient look,\\*
\vin \vin Thou watchest the last oozings hours by hours.\\!

Where are the songs of spring? Ay, where are they?\\*
\vin Think not of them; thou hast thy music too,\\
While barr\`{e}d clouds bloom the soft-dying day,\\
\vin And touch the stubble-plains with rosy hue;\\
Then in a wailful choir the small gnats mourn\\
\vin Among the river sallows, borne aloft\\
\vin \vin Or sinking as the light wind lives or dies;\\
And full-grown lambs loud bleat from hilly bourn;\\
\vin Hedge-crickets sing; and now with treble soft\\
\vin The red-breast whistles from a garden-croft;\\*
And gathering swallows twitter in the skies.
\end{verse}

\subsection{}

\blfootnote{`Sonnet of Black Beauty', Edward Herbert, 1st Baron Herbert of Cherbury (1582 -- 1648), \cite{norton}.}\settowidth{\versewidth}{Art neither changed with day, nor hid with night;}
\begin{verse}[\versewidth]
Black beauty, which above that common light,\\*
\vin Whose power can no colours here renew\\
\vin But those which darkness can again subdue,\\
Dost still remain unvaryed to the sight;\\
\vin And, like an object equal to the view,\\
Art neither changed with day, nor hid with night;\\
When all these colours which the world call bright,\\
\vin And which old poetry doth so pursue,\\
Are with the night so perish\`{e}d \& gone,\\
\vin That of their being there remains no mark,\\
Thou still abidest so entirely one,\\
\vin That we may know thy blackness is a spark\\
Of light inaccessible, and alone\\*
\vin Our darkness which can make us think it dark.
\end{verse}

\subsection{}

\blfootnote{Francis Bacon, Viscount St Alban (1561 -- 1626), \cite{odq}. Lord St Alban is here quoting an anecdote about Alfonso X \& IV, King of Le\'on and Castile.}Age appears to be best in four things: old wood best to burn, old wine to drink, old friends to trust, and old authors to read.

\section{}

\subsection{}

\blfootnote{`Ode to Psyche', John Keats (1795 -- 1821), \cite{norton}. Psyche was a minor goddess of the Greco-Roman mythological tradition, whose marriage to Cupid (`Love') and subsequent elevation to immortality were most famously related by Apuleius.}\settowidth{\versewidth}{    Of leaves and trembled blossoms, where there ran}
\begin{verse}[\versewidth]
O goddess, hear these tuneless numbers, wrung\\*
\vin By sweet enforcement and remembrance dear,\\
And pardon that thy secrets should be sung\\
\vin Even into thine own soft-conch\`{e}d ear:\\
Surely I dreamt today, or did I see\\
\vin The winged \textit{Psyche} with awakened eyes?\\
I wandered in a forest thoughtlessly,\\
\vin And, on the sudden, fainting with surprise,\\
Saw two fair creatures, couch\`{e}d side by side\\
\vin In deepest grass, beneath the whisp'ring roof\\
\vin Of leaves and trembled blossoms, where there ran\\
\vin \vin A brooklet, scarce espied:\\
Mid hushed, cool-rooted flowers, fragrant-eyed,\\
\vin Blue, silver-white, and budded tyrian,\\
They lay calm-breathing, on the bedded grass;\\
\vin Their arms embraced, and their pinions too;\\
\vin Their lips touched not, but had not bade adieu,\\
As if disjoined by soft-handed slumber,\\
And ready still past kisses to outnumber\\
\vin At tender eye-dawn of aurorean love:\\
\vin \vin The wing\`{e}d boy I knew;\\
\vin But who wast thou, O happy, happy dove?\\*
\vin \vin His \textit{Psyche} true!\\!

O latest born and loveliest vision far\\*
\vin Of all \textsc{Olympus}' faded hierarchy!\\
Fairer than \textit{Phoebe}'s sapphire-regioned star,\\
\vin Or \textit{Vesper}, amorous glow-worm of the sky;\\
Fairer than these, though temple thou hast none,\\
\vin \vin Nor altar heaped with flowers;\\
Nor virgin-choir to make delicious moan\\
\vin \vin Upon the midnight hours;\\
No voice, no lute, no pipe, no incense sweet\\
\vin From chain-swung censer teeming;\\
No shrine, no grove, no oracle, no heat\\*
\vin Of pale-mouthed prophet dreaming.\\!

O brightest! though too late for antique vows,\\*
\vin Too, too late for the fond believing lyre,\\
When holy were the haunted forest boughs,\\
\vin Holy the air, the water, and the fire;\\
Yet even in these days so far retired\\
\vin From happy pieties, thy lucent fans,\\
\vin Fluttering among the faint olympians,\\
I see, and sing, by my own eyes inspired.\\
So let me be thy choir, and make a moan\\
\vin \vin Upon the midnight hours;\\
Thy voice, thy lute, thy pipe, thy incense sweet\\
\vin From swing\`{e}d censer teeming;\\
Thy shrine, thy grove, thy oracle, thy heat\\*
\vin Of pale-mouthed prophet dreaming.\\!

Yes, I will be thy priest, and build a fane\\*
\vin In some untrodden region of my mind,\\
Where branch\`{e}d thoughts, new grown with pleasant pain,\\
\vin Instead of pines shall murmur in the wind:\\
Far, far around shall those dark-clustered trees\\
\vin Fledge the wild-ridg\`{e}d mountains steep by steep;\\
And there by zephyrs, streams, and birds, and bees,\\
\vin The moss-lain dryads shall be lulled to sleep;\\
And in the midst of this wide quietness\\
A rosy sanctuary will I dress\\
With the wreathed trellis of a working brain,\\
\vin With buds, and bells, and stars without a name,\\
With all the gardener fancy e'er could feign,\\
\vin Who breeding flowers, will never breed the same:\\
And there shall be for thee all soft delight\\
\vin That shadowy thought can win,\\
A bright torch, and a casement ope at night,\\*
\vin To let the warm \textit{Love} in!
\end{verse}

\subsection{}

\blfootnote{`The Night Piece, to Julia', Robert Herrick (1591 -- 1674), \cite{norton}. A `will-o'-the-wisp' is a phenomenon, which appears as a pale patch of light, sometimes seen by travellers walking through the countryside at night. A `slow-worm', meanwhile, is an archaic name for an adder, i.e. \textit{Vipera berus}.}\settowidth{\versewidth}{Like the sparks of fire, befriend thee.}
\begin{verse}[\versewidth]
Her eyes the glow-worm lend thee;\\*
The shooting stars attend thee;\\
\vin And the elves also,\\
\vin Whose little eyes glow\\*
Like the sparks of fire, befriend thee.\\!

No will-o'-the-wisp mis-light thee,\\*
Nor snake or slow-worm bite thee;\\
\vin But on, on thy way,\\
\vin Not making a stay,\\*
Since ghost there's none to affright thee.\\!

Let not the dark thee cumber;\\*
What though the moon does slumber?\\
\vin The stars of the night\\
\vin Will lend thee their light,\\*
Like tapers clear without number.\\!

Then \textit{Julia} let me woo thee,\\*
Thus, thus to come unto me;\\
\vin And when I shall meet\\
\vin Thy silv'ry feet,\\*
My soul I'll pour into thee.
\end{verse}

\subsection{}

\blfootnote{Francis Bacon, Viscount St Alban (1561 -- 1626), \cite{odq}.}Houses are built to live in and not to look on.

\section{}

\subsection{}

\blfootnote{`Bavarian Gentians', David Lawrence (1885 -- 1930), \cite{norton}. Lawrence wrote this poem only a few months before his own painfully-anticipated death from tuberculosis.}\settowidth{\versewidth}{Darkening the daytime torch-like with the smoking blueness of Pluto's gloom,}
\begin{verse}[\versewidth]
Not every man has gentians in his house\\*
In soft september, at slow, sad michaelmas.\\!

Bavarian gentians, big \& dark, only dark\\*
Darkening the daytime torch-like with the smoking blueness of \textit{Pluto}'s gloom,\\
Ribbed \& torch-like, with their blaze of darkness spread blue\\
Down flattening into points, flattened under the sweep of white day,\\
Torch-flower of the blue-smoking darkness, \textit{Pluto}'s dark-blue daze,\\
Black lamps from the halls of \textit{Dis}, burning dark-blue,\\
Giving off darkness, blue darkness, as \textit{Demeter}'s pale lamps\\
Give off light,\\*
Lead me then; lead me the way.\\!

Reach me a gentian; give me a torch!\\*
Let me guide myself with the blue, forked torch of a flower\\
Down the darker \& darker stairs, where blue is darkened on blueness,\\
Even where \textit{Persephone} goes, just now, from the frosted september\\
To the sightless realm where darkness is awake upon the dark\\
And \textit{Persephone} herself is but a voice\\
Or a darkness invisible enfolded in the deeper dark\\
Of the arms plutonic, and pierced with the passion of dense gloom,\\*
Among the splendour of torches of darkness, shedding darkness on the lost bride \& her groom.
\end{verse}

\subsection{}

\blfootnote{William Shakespeare (1564 -- 1616), \cite{shakespeare}. These lines are uttered by Iris (1-11) and Ceres (12-13) in \refbook{The Tempest} IV.1.}\settowidth{\versewidth}{Dove-drawn with her. Here thought they to have done}
\begin{verse}[\versewidth]
Be not afraid. I met her deity\\*
Cutting the clouds towards \textsc{Paphos}, and her son\\
Dove-drawn with her. Here thought they to have done\\
Some wanton charm upon this man \& maid,\\
Whose vows are that no bed-right shall be paid\\
Till \textit{Hymen}'s torch be lighted -- but in vain.\\
\textit{Mars}'s hot minion is returned again.\\
Her waspish-headed son has broke his arrows,\\
Swears he will shoot no more, but play with sparrows\\
And be a boy right out.\\
\textcolor{white}{And be a boy right out.} `Highest queen of state,\\*
Great \textit{Juno}, comes. I know her by her gait.'
\end{verse}

\subsection{}

\blfootnote{Robert Bruce (1554 -- 1631), \cite{odq}. Bruce's original gives `Lord Jesus Christ' instead of `Lord' on its own.}I have breakfasted with you and shall sup with my Lord... this night.

\section{}

\subsection{}

\blfootnote{`The River-Merchant's Wife: a Letter', Ezra Pound (1885 -- 1972), \cite{norton}. This poem is a translation of what is sometimes called \refpoem{The Song of Chang'an}, by the Chinese king of poets, Li Bai (also called Li Po, and known to Pound and various Japanese scholars as Rihaku). Many of the place-names, e.g. Chokan, Ku-to-yen, seem to be a melange of archaism, misunderstanding and poor transliteration.}\settowidth{\versewidth}{You went into far Ku-to-yen, by the river of swirling eddies,}
\begin{verse}[\versewidth]
While my hair was still cut straight across my forehead\\*
I played about the front gate, pulling flowers.\\
You came by on bamboo stilts, playing horse;\\
You walked about my seat, playing with blue plums.\\
And we went on living in the village of \textsc{Chokan}:\\
Two small people, without dislike or suspicion.\\
At 14 I married my lord, you.\\
I never laughed, being bashful.\\
Lowering my head, I looked at the wall.\\*
Called to, a 1000 times, I never looked back.\\!

At 15 I stopped scowling;\\*
I desired my dust to be mingled with yours\\
Forever \& forever \& forever.\\*
Why should I climb the look out?\\!

At 16 you departed\\*
You went into far \textsc{Ku-to-yen}, by the river of swirling eddies,\\
And you have been gone five months.\\*
The monkeys make sorrowful noise overhead.\\!

You dragged your feet when you went out.\\*
By the gate now, the moss is grown, the different mosses,\\
Too deep to clear them away!\\
The leaves fall early this autumn, in wind.\\
The paired butterflies are already yellow with august\\
Over the grass in the west garden;\\
They hurt me.\\
I grow older.\\
If you are coming down through the narrows of the river \textsc{Kiang},\\
Please let me know beforehand,\\
And I will come out to meet you\\*
\vin \vin As far as \textsc{Cho-fu-Sa}.
\end{verse}

\subsection{}

\blfootnote{William Shakespeare (1564 -- 1616), \cite{norton}.}\settowidth{\versewidth}{    Now is the time that face should form another;}
\begin{verse}[\versewidth]
Look in thy glass, and tell the face thou viewest\\*
\vin Now is the time that face should form another;\\
Whose fresh repair if now thou not renewest,\\
\vin Thou dost beguile the world, unbless some mother,\\
For where is she so fair whose uneared womb\\
\vin Disdains the tillage of thy husbandry?\\
Or who is he so fond will be the tomb\\
\vin Of his self-love, to stop posterity?\\
Thou art thy mother's glass, and she in thee\\
\vin Calls back the lovely april of her prime:\\
So thou through windows of thine age shall see\\
\vin Despite of wrinkles this thy golden time.\\
But if thou live, remembered not to be,\\*
Die single, and thine image dies with thee.
\end{verse}

\subsection{}

\blfootnote{Dr Samuel Johnson (1709 -- 1784), \cite{odq}.}Most vices may be committed very genteelly.

\section{}

\subsection{}

\blfootnote{William Shakespeare (1564 -- 1616), \cite{shakespeare}. These lines are uttered by Prospero in \refbook{The Tempest} V.1.}\settowidth{\versewidth}{Ye elves of hills, brooks, standing lakes and groves,}
\begin{verse}[\versewidth]
Ye elves of hills, brooks, standing lakes and groves,\\*
And ye that on the sands with printless foot\\
Do chase the ebbing \textit{Neptune} and do fly him\\
When he comes back; you demi-puppets that\\
By moonshine do the green sour ringlets make,\\
Whereof the ewe not bites, and you whose pastime\\
Is to make midnight mushrooms, that rejoice\\
To hear the solemn curfew; by whose aid,\\
Weak masters though ye be, I have bedimmed\\
The noontide sun, called forth the mutinous winds,\\
And 'twixt the green sea \& the azured vault\\
Set roaring war: to the dread rattling thunder\\
Have I given fire and rifted \textit{Jove}'s stout oak\\
With his own bolt; the strong-based promontory\\
Have I made shake and by the spurs plucked up\\
The pine \& cedar: graves at my command\\
Have waked their sleepers, oped, and let 'em forth\\
By my so potent art. But this rough magic\\
I here abjure, and, when I have required\\
Some heavenly music, which even now I do,\\
To work mine end upon their senses that\\
This airy charm is for, I'll break my staff,\\
Bury it certain fathoms in the earth,\\
And deeper than did ever plummet sound\\*
I'll drown my book.
\end{verse}

\subsection{}

\blfootnote{William Shakespeare (1564 -- 1616), \cite{shakespeare}. These famous lines are uttered by Caliban in \refbook{The Tempest} III.2.}\settowidth{\versewidth}{Will make me sleep again: and then, in dreaming,}
\begin{verse}[\versewidth]
Be not afeard; the isle is full of noises,\\*
Sounds \& sweet airs, that give delight and hurt not.\\
Sometimes a 1000 twangling instruments\\
Will hum about mine ears, and sometime voices\\
That, if I then had waked after long sleep,\\
Will make me sleep again: and then, in dreaming,\\
The clouds methought would open and show riches\\
Ready to drop upon me that, when I waked,\\*
I cried to dream again.
\end{verse}

\subsection{}

\blfootnote{William Shakespeare (1564 -- 1616), \cite{shakespeare}. This is uttered by Stephano in \refbook{The Tempest} III.2.}He that dies pays all debts.

\section{}

\subsection{}

\blfootnote{$\mathbb{R}$ `The Love Song of J Alfred Prufrock', Prof Thomas Eliot (1888 -- 1965), \cite{norton}. Prof Eliot begins this poem with a lenghty quotation from Dante, which bears no obvious relation to the text itself -- but that's Eliot.}\settowidth{\versewidth}{Though I have seen my head (grown slightly bald) brought in upon a platter,}
\begin{verse}[\versewidth]
Let us go then, you \& I,\\*
When the evening is spread out against the sky\\
Like a patient etherised upon a table;\\
Let us go, through certain \sfrac{$1$}{$2$}-deserted streets,\\
The muttering retreats\\
Of restless nights in one-night cheap hotels\\
And sawdust restaurants with oyster-shells:\\
Streets that follow like a tedious argument\\
Of insidious intent\\
To lead you to an overwhelming question...\\
O do not ask, What is it?\\*
Let us go and make our visit.\\!

In the room the women come \& go\\*
Talking of \textit{Michelangelo}.\\!

The yellow fog that rubs its back upon the window-panes,\\*
The yellow smoke that rubs its muzzle on the window-panes,\\
Licked its tongue into the corners of the evening,\\
Lingered upon the pools that stand in drains,\\
Let fall upon its back the soot that falls from chimneys,\\
Slipped by the terrace, made a sudden leap,\\
And seeing that it was a soft october night,\\*
Curled once about the house, and fell asleep.\\!

And indeed there will be time\\*
For the yellow smoke that slides along the street,\\
Rubbing its back upon the window-panes;\\
There will be time, there will be time\\
To prepare a face to meet the faces that you meet;\\
There will be time to murder \& create,\\
And time for all the works \& days of hands\\
That lift \& drop a question on your plate;\\
Time for you \& time for me,\\
And time yet for a 100 indecisions,\\
And for a 100 visions \& revisions,\\*
Before the taking of a toast \& tea.\\!

In the room the women come \& go\\*
Talking of \textit{Michelangelo}.\\!

And indeed there will be time\\*
To wonder, Do I dare? and, Do I dare?\\
Time to turn back and descend the stair,\\
With a bald spot in the middle of my hair --\\
(They will say, How his hair is growing thin!)\\
My morning coat, my collar mounting firmly to the chin,\\
My necktie rich \& modest, but asserted by a simple pin\\
(They will say, But how his arms \& legs are thin!)\\
Do I dare\\
Disturb the universe?\\
In a minute there is time\\*
For decisions \& revisions which a minute will reverse.\\!

For I have known them all already, known them all:\\*
Have known the evenings, mornings, afternoons,\\
I have measured out my life with coffee spoons;\\
I know the voices dying with a dying fall\\
Beneath the music from a farther room.\\*
So how should I presume?\\!

And I have known the eyes already, known them all --\\*
The eyes that fix you in a formulated phrase,\\
And when I am formulated, sprawling on a pin,\\
When I am pinned and wriggling on the wall,\\
Then how should I begin\\
To spit out all the butt-ends of my days \& ways?\\*
And how should I presume?\\!

And I have known the arms already, known them all --\\*
Arms that are braceleted \& white \& bare\\
(But in the lamplight, downed with light brown hair!)\\
Is it perfume from a dress\\
That makes me so digress?\\
Arms that lie along a table, or wrap about a shawl.\\
And should I then presume?\\*
And how should I begin?\\!

Shall I say, I have gone at dusk through narrow streets\\*
And watched the smoke that rises from the pipes\\*
Of lonely men in shirt-sleeves, leaning out of windows?...\\!

I should have been a pair of ragged claws\\*
Scuttling across the floors of silent seas.\\!

And the afternoon, the evening, sleeps so peacefully!\\*
Smoothed by long fingers,\\
Asleep... tired... or it malingers,\\
Stretched on the floor, here beside you \& me.\\
Should I, after tea \& cakes \& ices,\\
Have the strength to force the moment to its crisis?\\
But though I have wept \& fasted, wept \& prayed,\\
Though I have seen my head (grown slightly bald) brought in upon a platter,\\
I am no prophet -- and here's no great matter;\\
I have seen the moment of my greatness flicker,\\
And I have seen the eternal Footman hold my coat, and snicker,\\*
And in short, I was afraid.\\!

And would it have been worth it, after all,\\*
After the cups, the marmalade, the tea,\\
Among the porcelain, among some talk of you \& me,\\
Would it have been worth while,\\
To have bitten off the matter with a smile,\\
To have squeezed the universe into a ball\\
To roll it towards some overwhelming question,\\
To say, I am Lazarus, come from the dead,\\
Come back to tell you all, I shall tell you all --\\
If one, settling a pillow by her head\\
Should say, That is not what I meant at all;\\*
That is not it, at all.\\!

And would it have been worth it, after all,\\*
Would it have been worth while,\\
After the sunsets \& the dooryards \& the sprinkled streets,\\
After the novels, after the teacups, after the skirts that trail along the floor --\\
And this, and so much more? --\\
It is impossible to say just what I mean!\\
But as if a magic lantern threw the nerves in patterns on a screen:\\
Would it have been worth while\\
If one, settling a pillow or throwing off a shawl,\\
And turning toward the window, should say,\\
That is not it at all,\\*
That is not what I meant, at all.\\!

No! I am not Prince \textit{Hamlet}, nor was meant to be;\\*
Am an attendant lord, one that will do\\
To swell a progress, start a scene or two,\\
Advise the prince; no doubt, an easy tool,\\
Deferential, glad to be of use,\\
Politic, cautious, and meticulous;\\
Full of high sentence, but a bit obtuse;\\
At times, indeed, almost ridiculous --\\*
Almost, at times, the fool.\\!

I grow old... I grow old...\\*
I shall wear the bottoms of my trousers rolled.\\!

Shall I part my hair behind? Do I dare to eat a peach?\\*
I shall wear white flannel trousers, and walk upon the beach.\\*
I have heard the mermaids singing, each to each.\\!

I do not think that they will sing to me.\\!

I have seen them riding seaward on the waves\\*
Combing the white hair of the waves blown back\\
When the wind blows the water white \& black.\\
We have lingered in the chambers of the sea\\
By sea-girls wreathed with seaweed red \& brown\\*
Till human voices wake us, and we drown.
\end{verse}

\subsection{}

\blfootnote{William Shakespeare (1564 -- 1616), \cite{shakespeare}. These famous lines are uttered by Prospero in \refbook{The Tempest} IV.1. The Almanacker has excised two lines after `Sir, I am vexed...'}\settowidth{\versewidth}{And, like the baseless fabric of this vision,}
\begin{verse}[\versewidth]
Our revels now are ended. These our actors,\\*
As I foretold you, were all spirits and\\
Are melted into air, into thin air:\\
And, like the baseless fabric of this vision,\\
The cloud-capped towers, the gorgeous palaces,\\
The solemn temples, the great globe itself,\\
Yea, all which it inherit, shall dissolve\\
And, like this insubstantial pageant faded,\\
Leave not a rack behind. We are such stuff\\
As dreams are made on, and our little life\\
Is rounded with a sleep. Sir, I am vexed...\\
If you be pleased, retire into my cell\\
And there repose: a turn or two I'll walk,\\*
To still my beating mind.
\end{verse}

\subsection{}

\blfootnote{William Shakespeare (1564 -- 1616), \cite{shakespeare}. This is uttered by Prospero in \refbook{The Tempest} I.2.}There's no harm done.

\section{}

\subsection{}

\blfootnote{`A Letter to Her Husband, Absent upon Public Employment', Mrs Anne Bradstreet (1612 -- 1672), \cite{norton}. The Ipswich in question would seem to be the one in Massachusetts, and not the one in England. The word `sol' is Latin for ``sun''.}\settowidth{\versewidth}{Whom whilst I 'joyed, nor storms, nor frosts I felt,}
\begin{verse}[\versewidth]
My head, my heart, mine eyes, my life, nay more,\\*
My joy, my magazine of earthly store,\\
If two be one, as surely thou and I,\\
How stayest thou there, whilst I at \textsc{Ipswich} lie?\\
So many steps, head from the heart to sever\\
If but a neck, soon should we be together:\\
I like the earth this season, mourn in black,\\
My sun is gone so far in 's zodiack,\\
Whom whilst I 'joyed, nor storms, nor frosts I felt,\\
His warmth such frigid colds did cause to melt.\\
My chill\`{e}d limbs now numb\`{e}d lie forlorn;\\
Return, return sweet {\hoskeroe sol} from Capricorn;\\
In this dead time, alas, what can I more\\
Then view those fruits which through thy heat I bore,\\
Which sweet contentment yield me for a space,\\
True living pictures of their father's face?\\
O strange effect! now thou art southward gone,\\
I weary grow, the tedious day so long;\\
But when thou northward to me shalt return,\\
I wish my sun may never set, but burn\\
Within the Cancer of my glowing breast,\\
The welcome house of him my dearest guest.\\
Where ever, ever stay, and go not thence,\\
Till natures sad decree shall call thee hence;\\
Flesh of thy flesh, bone of thy bone,\\*
I here, thou there, yet both but one.
\end{verse}

\subsection{}

\blfootnote{William Shakespeare (1564 -- 1616), \cite{shakespeare}. This song is sung by Ariel in \refbook{The Tempest}, I.2. Chanticleer is a name given to a cockerel in several English folk-tales.}\settowidth{\versewidth}{Curtsied when you have, and kissed}
\begin{verse}[\versewidth]
Come unto these yellow sands,\\*
\vin And then take hands.\\
Curtsied when you have, and kissed\\
\vin The wild waves whist.\\
Foot it featly here \& there,\\
And, sweet sprites, the burden bear.\\
\vin \vin Hark; hark!\\
\vin \vin The watch-dogs bark:\\
\vin \vin Hark; hark! I hear\\*
The strain of strutting \textit{Chanticleer}.
\end{verse}

\subsection{}

\blfootnote{William Shakespeare (1564 -- 1616), \cite{shakespeare}. This is uttered by Ferdinand in \refbook{The Tempest} III.1.}Some kinds of baseness are nobly undergone.

\section{}

\subsection{}

\blfootnote{Alfred Tennyson, 1st Baron Tennyson, Poet Laureate (1809 -- 1892), \cite{norton}. This is \refbook{In Memoriam A H H} \S 95.}\settowidth{\versewidth}{From knoll to knoll, where, couched at ease,}
\begin{verse}[\versewidth]
By night we lingered on the lawn,\\*
For underfoot the herb was dry;\\
And genial warmth; and o'er the sky\\*
The silvery haze of summer drawn;\\!

And calm that let the tapers burn\\*
Unwavering: not a cricket chirred:\\
The brook alone far-off was heard,\\*
And on the board the fluttering urn:\\!

And bats went round in fragrant skies,\\*
And wheeled or lit the filmy shapes\\
That haunt the dusk, with ermine capes\\*
And woolly breasts and beaded eyes;\\!

While now we sang old songs that pealed\\*
From knoll to knoll, where, couched at ease,\\
The white kine glimmered, and the trees\\*
Laid their dark arms about the field.\\!

But when those others, one by one,\\*
Withdrew themselves from me and night,\\
And in the house light after light\\*
Went out, and I was all alone,\\!

A hunger seized my heart; I read\\*
Of that glad year which once had been,\\
In those fall'n leaves which kept their green,\\*
The noble letters of the dead:\\!

And strangely on the silence broke\\*
The silent-speaking words, and strange\\
Was love's dumb cry defying change\\*
To test his worth; and strangely spoke\\!

The faith, the vigour, bold to dwell\\*
On doubts that drive the coward back,\\
And keen through wordy snares to track\\*
Suggestion to her inmost cell.\\!

So word by word, and line by line,\\*
The dead man touched me from the past,\\
And all at once it seemed at last\\*
The living soul was flashed on mine,\\!

And mine in this was wound, and whirled\\*
About empyreal heights of thought,\\
And came on that which is, and caught\\*
The deep pulsations of the world,\\!

Aeonian music measuring out\\*
The steps of time, the shocks of chance,\\
The blows of death. At length my trance\\*
Was cancelled, stricken through with doubt.\\!

Vague words! but ah, how hard to frame\\*
In matter-moulded forms of speech,\\
Or ev'n for intellect to reach\\*
Through memory that which I became:\\!

Till now the doubtful dusk revealed\\*
The knolls once more where, couched at ease,\\
The white kine glimmered, and the trees\\*
Laid their dark arms about the field:\\!

And sucked from out the distant gloom\\*
A breeze began to tremble o'er\\
The large leaves of the sycamore,\\*
And fluctuate all the still perfume,\\!

And gathering freshlier overhead,\\*
Rocked the full-foliaged elms, and swung\\
The heavy-folded rose, and flung\\*
The lilies to \& fro, and said,\\!

The dawn, the dawn! and died away;\\*
And east \& west, without a breath,\\
Mixt their dim lights, like life and death,\\*
To broaden into boundless day.
\end{verse}

\subsection{}

\blfootnote{Alfred Tennyson, 1st Baron Tennyson, Poet Laureate (1809 -- 1892), \cite{norton}.}\settowidth{\versewidth}{    And in my thoughts with scarce a sigh}
\begin{verse}[\versewidth]
When on my bed the moonlight falls,\\*
\vin So quickly, not as one that weeps\\
\vin I come once more; the city sleeps;\\*
I smell the meadow in the street;\\!

I hear a chirp of birds; I see\\*
\vin Betwixt the black fronts long-withdrawn\\
\vin A light-blue lane of early dawn,\\*
And think of early days \& thee,\\!

And bless thee, for thy lips are bland,\\*
\vin And bright the friendship of thine eye;\\
\vin And in my thoughts with scarce a sigh\\*
I take the pressure of thine hand.
\end{verse}

\subsection{}

\blfootnote{Walter Bagehot (1826 -- 1877), \cite{odq}.}We must not let in daylight upon magic.

\section{}

\subsection{}

\blfootnote{Alfred Tennyson, 1st Baron Tennyson, Poet Laureate (1809 -- 1892), \cite{norton}.}\settowidth{\versewidth}{    I hear thee where the waters run;}
\begin{verse}[\versewidth]
Thy voice is on the rolling air;\\*
\vin I hear thee where the waters run;\\
\vin Thou standest in the rising sun,\\*
And in the setting thou art fair.\\!

What art thou then? I cannot guess;\\*
\vin But though I seem in star \& flower\\
\vin To feel thee some diffusive power,\\*
I do not therefore love thee less.\\!

My love involves the love before;\\*
\vin My love is vaster passion now;\\
\vin Though mixed with God \& nature thou,\\*
I seem to love thee more \& more.\\!

Far off thou art, but ever nigh;\\*
\vin I have thee still, and I rejoice;\\
\vin I prosper, circled with thy voice;\\*
I shall not lose thee though I die.
\end{verse}

\subsection{}

\blfootnote{Alfred Tennyson, 1st Baron Tennyson, Poet Laureate (1809 -- 1892), \cite{norton}. The title (which means, `Greetings, brother, and farewell') is a quotation from Catullus 101. The same poet wrote of his affection for Sirmione (which he called Sirmio) in Catullus 31, `venusta' meaning `beautiful'.}\settowidth{\versewidth}{There to me through all the groves of olive in the summer glow,}
\begin{verse}[\versewidth]
Row us out from \textsc{Desenzano}; to your \textsc{Sirmione} row!\\*
So they rowed, and there we landed -- {\hoskeroe O venusta} \textsc{Sirmio} --\\
There to me through all the groves of olive in the summer glow,\\
There beneath the roman ruin where the purple flowers grow,\\
Came that {\hoskeroe ave atque vale} of the poet's hopeless woe,\\
Tenderest of roman poets 19 hundred years ago, {\hoskeroe Frater, ave atque vale} -- as we wandered to \& fro\\
Gazing at the lydian laughter of the \textsc{Garda Lake} below\\*
Sweet \textit{Catullus}'s all-but-island, olive-silvery \textsc{Sirmio}!
\end{verse}

\subsection{}

\blfootnote{Mrs Elizabeth Browning (1806 -- 1861), \cite{odq}.}The devil's most devilish when respectable.

\section{}

\subsection{}

\blfootnote{Alfred Tennyson, 1st Baron Tennyson, Poet Laureate (1809 -- 1892), \cite{norton}.}\settowidth{\versewidth}{    And closing eaves of wearied eyes}
\begin{verse}[\versewidth]
When on my bed the moonlight falls,\\*
\vin I know that in thy place of rest\\
\vin By that broad water of the west,\\*
There comes a glory on the walls;\\!

Thy marble bright in dark appears,\\*
\vin As slowly steals a silver flame\\
\vin Along the letters of thy name,\\*
And o'er the number of thy years.\\!

The mystic glory swims away;\\*
\vin From off my bed the moonlight dies;\\
\vin And closing eaves of wearied eyes\\*
I sleep till dusk is dipped in grey;\\!

And then I know the mist is drawn\\*
\vin A lucid veil from coast to coast,\\
\vin And in the dark church like a ghost\\*
Thy tablet glimmers to the dawn.
\end{verse}

\subsection{}

\blfootnote{Alfred Tennyson, 1st Baron Tennyson, Poet Laureate (1809 -- 1892), \cite{norton}. This sonnet is taken from a larger work, \refbook{The Princess}. It has been set to music by a number of famous composers.}\settowidth{\versewidth}{Now sleeps the crimson petal, now the white;}
\begin{verse}[\versewidth]
Now sleeps the crimson petal, now the white;\\*
Nor waves the cypress in the palace walk;\\
Nor winks the gold fin in the porphyry font.\\*
The firefly wakens; waken thou with me.\\!

Now droops the milk-white peacock like a ghost,\\*
And like a ghost she glimmers on to me.\\!

Now lies the earth all \textit{Dana\"{e}} to the stars,\\*
And all thy heart lies open unto me.\\!

Now slides the silent meteor on, and leaves\\*
A shining furrow, as thy thoughts in me.\\!

Now folds the lily all her sweetness up,\\*
And slips into the bosom of the lake.\\
So fold thyself, my dearest, thou, and slip\\*
Into my bosom and be lost in me.
\end{verse}

\subsection{}

\blfootnote{George Berkeley, Bishop of Cloyne (1685 -- 1753), \cite{odq}.}The same principles which at first view lead to scepticism, pursued to a certain point bring men back to common sense.

\section{}

\subsection{}

\blfootnote{`Tithonus', Alfred Tennyson, 1st Baron Tennyson, Poet Laureate (1809 -- 1892), \cite{norton}. A figure from Greek mythology, Tithonus was abducted by the goddess of the dawn to be a kind of fancy-man. He was granted everlasting life -- but forgot to ask for everlasting youth -- cursing him with an eternity of senility. The ancients used to say that the goddess' consequent sexual frustration explained why she rose so early in the morning.}\settowidth{\versewidth}{But thy strong hours indignant worked their wills,}
\begin{verse}[\versewidth]
The woods decay, the woods decay and fall;\\*
The vapours weep their burthen to the ground;\\
Man comes and tills the field and lies beneath,\\
And after many a summer dies the swan.\\
Me only cruel immortality\\
Consumes: I wither slowly in thine arms,\\
Here at the quiet limit of the world,\\
A white-haired shadow roaming like a dream\\
The ever-silent spaces of the east,\\*
Far-folded mists, and gleaming halls of morn.\\!

Alas! for this grey shadow, once a man\\*
So glorious in his beauty and thy choice,\\
Who madest him thy chosen, that he seemed\\
To his great heart none other than a god!\\
I asked thee, 'Give me immortality.'\\
Then didst thou grant mine asking with a smile,\\
Like wealthy men, who care not how they give.\\
But thy strong hours indignant worked their wills,\\
And beat me down \& marred \& wasted me,\\
And though they could not end me, left me maimed\\
To dwell in presence of immortal youth,\\
Immortal age beside immortal youth,\\
And all I was, in ashes. Can thy love,\\
Thy beauty, make amends, though even now,\\
Close over us, the silver star, thy guide,\\
Shines in those tremulous eyes that fill with tears\\
To hear me? Let me go: take back thy gift:\\
Why should a man desire in any way\\
To vary from the kindly race of men\\
Or pass beyond the goal of ordinance\\*
Where all should pause, as is most meet for all?\\!

A soft air fans the cloud apart; there comes\\*
A glimpse of that dark world where I was born.\\
Once more the old mysterious glimmer steals\\
From thy pure brows, and from thy shoulders pure,\\
And bosom beating with a heart renewed.\\
Thy cheek begins to redden through the gloom,\\
Thy sweet eyes brighten slowly close to mine,\\
Ere yet they blind the stars, and the wild team\\
Which love thee, yearning for thy yoke, arise,\\
And shake the darkness from their loosened manes,\\*
And beat the twilight into flakes of fire.\\!

Lo! ever thus thou growest beautiful\\*
In silence, then before thine answer given\\*
Departest, and thy tears are on my cheek.\\!

Why wilt thou ever scare me with thy tears,\\*
And make me tremble lest a saying learned,\\
In days far-off, on that dark earth, be true?\\*
'The Gods themselves cannot recall their gifts.'\\!

Ay me! ay me! with what another heart\\*
In days far-off, and with what other eyes\\
I used to watch -- if I be he that watched --\\
The lucid outline forming round thee; saw\\
The dim curls kindle into sunny rings;\\
Changed with thy mystic change, and felt my blood\\
Glow with the glow that slowly crimsoned all\\
Thy presence \& thy portals, while I lay,\\
Mouth, forehead, eyelids, growing dewy-warm\\
With kisses balmier than half-opening buds\\
Of april, and could hear the lips that kissed\\
Whispering I knew not what of wild and sweet,\\
Like that strange song I heard \textit{Apollo} sing,\\*
While \textsc{Ilion} like a mist rose into towers.\\!

Yet hold me not for ever in thine east:\\*
How can my nature longer mix with thine?\\
Coldly thy rosy shadows bathe me, cold\\
Are all thy lights, and cold my wrinkled feet\\
Upon thy glimmering thresholds, when the steam\\
Floats up from those dim fields about the homes\\
Of happy men that have the power to die,\\
And grassy barrows of the happier dead.\\
Release me, and restore me to the ground;\\
Thou seest all things: thou wilt see my grave:\\
Thou wilt renew thy beauty morn by morn;\\
I earth in earth forget these empty courts,\\*
And thee returning on thy silver wheels.
\end{verse}

\subsection{}

\blfootnote{`Crossing the Bar', Alfred Tennyson, 1st Baron Tennyson, Poet Laureate (1809 -- 1892), \cite{norton}. The `bar' in question refers to the sandbars which often lurk in the waters near to the breakwaters of ports. Lord Tennyson is said to have written this poem while on a ferry to the Isle of Wight. Shortly before his death, he decreed that all editions of his works should close with these verses. \P 13. Lord Tennyson explained, `The pilot has been on board all the while, but in the dark I have not seen him.'}\settowidth{\versewidth}{For though from out our bourne of time place}
\begin{verse}[\versewidth]
Sunset \& evening star,\\*
\vin And one clear call for me!\\
And may there be no moaning of the bar\\*
\vin When I put out to sea,\\!

But such a tide as moving seems asleep,\\*
\vin Too full for sound \& foam,\\
When that which drew from out the boundless deep\\*
\vin Turns again home.\\!

Twilight \& evening bell,\\*
\vin And after that the dark!\\
And may there be no sadness of farewell,\\*
\vin When I embark;\\!

For though from out our bourne of time \& place\\*
\vin The flood may bear me far,\\
I hope to see my pilot face to face\\*
\vin When I have crossed the bar.
\end{verse}

\subsection{}

\blfootnote{Miss Jane Austen (1775 -- 1817), \cite{odq}.}We do not look in great cities for our best morality.

\section{}

\subsection{}

\blfootnote{`To a Waterfowl', William Bryant (1794 -- 1878), \cite{norton}.}\settowidth{\versewidth}{While glow the heavens with the last steps of day,}
\begin{verse}[\versewidth]
\vin Whither, 'midst falling dew,\\*
While glow the heavens with the last steps of day,\\
Far, through their rosy depths, dost thou pursue\\*
\vin Thy solitary way?\\!

\vin Vainly the fowler's eye\\*
Might mark thy distant flight, to do thee wrong,\\
As, darkly seen against the crimson sky,\\*
\vin Thy figure floats along.\\!

\vin Seek'st thou the plashy brink\\*
Of weedy lake, or marge of river wide,\\
Or where the rocking billows rise \& sink\\*
\vin On the chaf\`{e}d ocean side?\\!

\vin There is a Power, whose care\\*
Teaches thy way along that pathless coast,\\
The desert and illimitable air\\*
\vin Lone wandering, but not lost.\\!

\vin All day thy wings have fanned,\\*
At that far height, the cold thin atmosphere;\\
Yet stoop not, weary, to the welcome land,\\*
\vin Though the dark night is near.\\!

\vin And soon that toil shall end,\\*
Soon shalt thou find a summer home, and rest,\\
And scream among thy fellows; reeds shall bend,\\*
\vin Soon, o'er thy sheltered nest.\\!

\vin Thou'rt gone, the abyss of heaven\\*
Hath swallowed up thy form, yet, on my heart\\
Deeply hath sunk the lesson thou hast given,\\*
\vin And shall not soon depart.\\!

\vin He, who, from zone to zone,\\*
Guides through the boundless sky thy certain flight,\\
In the long way that I must trace alone\\*
\vin Will lead my steps aright.
\end{verse}

\subsection{}

\blfootnote{`In Time of ``The Breaking of Nations{''}', Thomas Hardy (1840 -- 1928), \cite{norton}. This poem might well hold the record in the English canon for the longest gestation; for Hardy began working on it in 1870 following the Battle of Sedan, and only finished it in 1915 during the First World War. The title is an allusion to Jeremiah 51.20.}\settowidth{\versewidth}{    From the heaps of couch-grass;}
\begin{verse}[\versewidth]
Only a man harrowing clods\\*
\vin In a slow silent walk\\
With an old horse that stumbles \& nods\\*
\vin Half asleep as they stalk.\\!

Only thin smoke without flame\\*
\vin From the heaps of couch-grass;\\
Yet this will go onward the same\\*
\vin Though dynasties pass.\\!

Yonder a maid \& her wight\\*
\vin Come whispering by:\\
War's annals will cloud into night\\*
\vin Ere their story die.
\end{verse}

\subsection{}

\blfootnote{Robert Browning (1828 -- 1889), \cite{odq}.}Leave \emph{now} for dogs and apes! Man has forever.

\section{}

\subsection{}

\blfootnote{`Telling the Bees', John Whittier (1807 -- 1892), \cite{norton}. It is ancient custom for a beekeeper to tell his bees of significant events in his life, for fear that they might otherwise migrate.}\settowidth{\versewidth}{There's the same sweet clover-smell in the breeze;}
\begin{verse}[\versewidth]
Here is the place; right over the hill\\*
\vin Runs the path I took;\\
You can see the gap in the old wall still,\\*
\vin And the stepping-stones in the shallow brook.\\!

There is the house, with the gate red-barred,\\*
\vin And the poplars tall;\\
And the barn's brown length, and the cattle-yard,\\*
\vin And the white horns tossing above the wall.\\!

There are the beehives ranged in the sun;\\*
\vin And down by the brink\\
Of the brook are her poor flowers, weed-o'errun,\\*
\vin Pansy and daffodil, rose and pink.\\!

A year has gone, as the tortoise goes,\\*
\vin Heavy \& slow;\\
And the same rose blows, and the same sun glows,\\*
\vin And the same brook sings of a year ago.\\!

There's the same sweet clover-smell in the breeze;\\*
\vin And the june sun warm\\
Tangles his wings of fire in the trees,\\*
\vin Setting, as then, over \textsc{Fernside} farm.\\!

I mind me how with a lover's care\\*
\vin From my sunday coat\\
I brushed off the burrs, and smoothed my hair,\\*
\vin And cooled at the brookside my brow and throat.\\!

Since we parted, a month had passed --\\*
\vin To love, a year;\\
Down through the beeches I looked at last\\*
\vin On the little red gate and the well-sweep near.\\!

I can see it all now: the slantwise rain\\*
\vin Of light through the leaves,\\
The sundown’s blaze on her window-pane,\\*
\vin The bloom of her roses under the eaves.\\!

Just the same as a month before,\\*
\vin The house and the trees,\\
The barn's brown gable, the vine by the door,\\*
\vin Nothing changed but the hives of bees.\\!

Before them, under the garden wall,\\*
\vin Forward \& back,\\
Went drearily singing the chore-girl small,\\*
\vin Draping each hive with a shred of black.\\!

Trembling, I listened: the summer sun\\*
\vin Had the chill of snow;\\
For I knew she was telling the bees of one\\*
\vin Gone on the journey we all must go!\\!

Then I said to myself, My \textit{Mary} weeps\\*
\vin For the dead today:\\
Haply her blind old grandsire sleeps\\*
\vin The fret and the pain of his age away.\\!

But her dog whined low; on the doorway sill,\\*
\vin With his cane to his chin,\\
The old man sat; and the chore-girl still\\*
\vin Sung to the bees stealing out and in.\\!

And the song she was singing ever since\\*
\vin In my ear sounds on:\\
Stay at home, pretty bees; fly not hence!\\*
\vin Mistress \textit{Mary} is dead \& gone!
\end{verse}

\subsection{}

\blfootnote{`Parting, without a Sequel', John Ransom (1888 -- 1974), \cite{norton}.}\settowidth{\versewidth}{Saying to the blue-capped functioner of doom,}
\begin{verse}[\versewidth]
\vin She has finished \& sealed the letter\\*
At last, which he so richly has deserved,\\
With characters venomous \& hatefully curved,\\*
\vin And nothing could be better.\\!

\vin But even as she gave it\\*
Saying to the blue-capped functioner of doom,\\
`Into his hands,' she hoped the leering groom\\*
\vin Might somewhere lose \& leave it.\\!

\vin Then all the blood\\*
Forsook the face. She was too pale for tears,\\
Observing the ruin of her younger years.\\*
\vin She went and stood\\!

\vin Under her father's vaunting oak\\*
Who kept his peace in wind \& sun and glistened\\
Stoical in the rain; to whom she listened\\*
\vin If he spoke.\\!

\vin And now the agitation of the rain\\*
Rasped his sere leaves, and he talked low \& gentle\\
Reproaching the wan daughter by the lintel;\\*
\vin Ceasing \& beginning again.\\!

\vin Away went the messenger's bicycle;\\*
His serpent’s track went up the hill forever,\\
And all the time she stood there hot as fever\\*
\vin And cold as any icicle.
\end{verse}

\subsection{}

\blfootnote{Robert Browning (1828 -- 1889), \cite{odq}.}\settowidth{\versewidth}{Open my heart and you will see}
\begin{verse}[\versewidth]
Open my heart and you will see\\*
Graved inside of it: Italy.
\end{verse}

\section{}

\subsection{}

\blfootnote{`Andrea del Sarto', Robert Browning (1828 -- 1889), \cite{norton}. Subtitle: `Called ``The Faultless Painter{''}'. Andrea del Sarto was an artist of the Italian Renaissance. He was thought to be one of the best painters in the world in his own time, but his reputation has fared less well than those of his contemporaries Michelangelo, da Vinci and Raphael. \P 12. Fiesole is a small town overlooking Florence.}\settowidth{\versewidth}{Yet the will's somewhat -- somewhat, too, the power --}
\begin{verse}[\versewidth]
But do not let us quarrel any more,\\*
No, my \textit{Lucrezia}; bear with me for once:\\
Sit down and all shall happen as you wish.\\
You turn your face, but does it bring your heart?\\
I'll work then for your friend's friend, never fear,\\
Treat his own subject after his own way,\\
Fix his own time, accept too his own price,\\
And shut the money into this small hand\\
When next it takes mine. Will it? tenderly?\\
O I'll content him, but tomorrow, love!\\
I often am much wearier than you think,\\
This evening more than usual, and it seems\\
As if -- forgive now -- should you let me sit\\
Here by the window with your hand in mine\\
And look a \sfrac{$1$}{$2$} hour forth on \textsc{Fiesole},\\
Both of one mind, as married people use,\\
Quietly, quietly the evening through,\\
I might get up to-morrow to my work\\
Cheerful \& fresh as ever. Let us try.\\
Tomorrow, how you shall be glad for this!\\
Your soft hand is a woman of itself,\\
And mine the man's bared breast she curls inside.\\
Don't count the time lost, neither; you must serve\\
For each of the five pictures we require:\\
It saves a model. So! keep looking so --\\
My serpentining beauty, rounds on rounds!\\
How could you ever prick those perfect ears\\
Even to put the pearl there?! O so sweet --\\
My face, my moon, my everybody's moon,\\
Which everybody looks on and calls his,\\
And, I suppose, is looked on by in turn,\\
While she looks -- no one's: very dear, no less.\\
You smile? Why, there's my picture ready made,\\
There's what we painters call our harmony!\\
A common greyness silvers everything,\\
All in a twilight, you \& I alike\\
You, at the point of your first pride in me\\
(That's gone you know), but I, at every point;\\
My youth, my hope, my art, being all toned down\\
To yonder sober pleasant \textsc{Fiesole}.\\
There's the bell clinking from the chapel-top;\\
That length of convent-wall across the way\\
Holds the trees safer, huddled more inside;\\
The last monk leaves the garden; days decrease,\\
And autumn grows, autumn in everything.\\
Eh? the whole seems to fall into a shape\\
As if I saw alike my work and self\\
And all that I was born to be \& do,\\
A twilight-piece. Love, we are in God's hand.\\
How strange now, looks the life he makes us lead;\\
So free we seem, so fettered fast we are!\\
I feel he laid the fetter: let it lie!\\
This chamber for example -- turn your head --\\
All that's behind us! You don't understand\\
Nor care to understand about my art,\\
But you can hear at least when people speak:\\
And that cartoon, the second from the door\\
-- It is the thing, love! so such things should be --\\
Behold Madonna! I am bold to say.\\
I can do with my pencil what I know,\\
What I see, what at bottom of my heart\\
I wish for, if I ever wish so deep --\\
Do easily, too -- when I say, perfectly,\\
I do not boast, perhaps: yourself are judge,\\
Who listened to the legate's talk last week,\\
And just as much they used to say in France.\\
At any rate 'tis easy, all of it!\\
No sketches first, no studies, that's long past:\\
I do what many dream of, all their lives,\\
Dream? strive to do, and agonize to do,\\
And fail in doing. I could count twenty such\\
On twice your fingers, and not leave this town,\\
Who strive -- you don't know how the others strive\\
To paint a little thing like that you smeared\\
Carelessly passing with your robes afloat --\\
Yet do much less, so much less, someone says,\\
(I know his name, no matter) -- so much less!\\
Well, less is more, \textit{Lucrezia}: I am judged.\\
There burns a truer light of God in them,\\
In their vexed beating stuffed \& stopped-up brain,\\
Heart, or whate'er else, than goes on to prompt\\
This low-pulsed forthright craftsman's hand of mine.\\
Their works drop groundward, but themselves, I know,\\
Reach many a time a heaven that's shut to me,\\
Enter and take their place there sure enough,\\
Though they come back and cannot tell the world.\\
My works are nearer heaven, but I sit here.\\
The sudden blood of these men! at a word --\\
Praise them, it boils, or blame them, it boils too.\\
I, painting from myself and to myself,\\
Know what I do, am unmoved by men's blame\\
Or their praise either. Somebody remarks\\
\textit{Morello}'s outline there is wrongly traced,\\
His hue mistaken; what of that? or else,\\
Rightly traced and well ordered; what of that?\\
Speak as they please, what does the mountain care?\\
Ah, but a man's reach should exceed his grasp,\\
Or what's a heaven for? All is silver-grey,\\
Placid \& perfect with my art: the worse!\\
I know both what I want and what might gain,\\
And yet how profitless to know, to sigh,\\
`Had I been two, another and myself,\\
Our head would have o'erlooked the world!' No doubt.\\
Yonder's a work now, of that famous youth\\
The urbinate who died five years ago.\\
('Tis copied, \textit{George Vasari} sent it me.)\\
Well, I can fancy how he did it all,\\
Pouring his soul, with kings \& popes to see,\\
Reaching, that heaven might so replenish him,\\
Above and through his art -- for it gives way;\\
That arm is wrongly put -- and there again --\\
A fault to pardon in the drawing's lines,\\
Its body, so to speak: its soul is right,\\
He means right -- that, a child may understand.\\
Still, what an arm! and I could alter it:\\
But all the play, the insight and the stretch --\\
(Out of me, out of me! And wherefore out?\\
Had you enjoined them on me, given me soul,\\
We might have risen to \textit{Rafael}, I \& you!\\
Nay, love, you did give all I asked, I think --\\
More than I merit, yes, by many times.\\
But had you -- O with the same perfect brow,\\
And perfect eyes, and more than perfect mouth,\\
And the low voice my soul hears, as a bird\\
The fowler's pipe, and follows to the snare --\\
Had you, with these the same, but brought a mind!\\
Some women do so. Had the mouth there urged,\\
`God and the glory! never care for gain.\\
The present by the future, what is that?\\
Live for fame, side by side with \textit{Agnolo}!\\
\textit{Rafael} is waiting: up to God, all three!'\\
I might have done it for you. So it seems:\\
Perhaps not. All is as God over-rules.\\
Beside, incentives come from the soul's self;\\
The rest avail not. Why do I need you?\\
What wife had \textit{Rafael}, or has \textit{Agnolo}?\\
In this world, who can do a thing, will not;\\
And who would do it, cannot, I perceive:\\
Yet the will's somewhat -- somewhat, too, the power --\\
And thus we \sfrac{$1$}{$2$} men struggle. At the end,\\
God, I conclude, compensates, punishes.\\
'Tis safer for me, if the award be strict,\\
That I am something underrated here,\\
Poor this long while, despised, to speak the truth.\\
I dared not, do you know, leave home all day,\\
For fear of chancing on the \textsc{Paris} lords.\\
The best is when they pass \& look aside;\\
But they speak sometimes; I must bear it all.\\
Well may they speak! That \textit{Francis}, that first time,\\
And that long festal year at \textsc{Fontainebleau}!\\
I surely then could sometimes leave the ground,\\
Put on the glory, \textit{Rafael}'s daily wear,\\
In that humane great monarch's golden look --\\
One finger in his beard or twisted curl\\
Over his mouth's good mark that made the smile,\\
One arm about my shoulder, round my neck,\\
The jingle of his gold chain in my ear,\\
I painting proudly with his breath on me,\\
All his court round him, seeing with his eyes,\\
Such frank french eyes, and such a fire of souls\\
Profuse, my hand kept plying by those hearts --\\
And, best of all, this, this, this face beyond,\\
This in the background, waiting on my work,\\
To crown the issue with a last reward!\\
A good time, was it not, my kingly days?\\
And had you not grown restless... but I know --\\
'Tis done \& past: 'twas right, my instinct said:\\
Too live the life grew, golden and not grey,\\
And I'm the weak-eyed bat no sun should tempt\\
Out of the grange whose four walls make his world.\\
How could it end in any other way?\\
You called me, and I came home to your heart.\\
The triumph was -- to reach and stay there; since\\
I reached it ere the triumph, what is lost?\\
Let my hands frame your face in your hair's gold,\\
You beautiful \textit{Lucrezia} that are mine!\\
`\textit{Rafael} did this, \textit{Andrea} painted that;\\
The roman's is the better when you pray,\\
But still the other's Virgin was his wife' --\\
Men will excuse me. I am glad to judge\\
Both pictures in your presence; clearer grows\\
My better fortune, I resolve to think.\\
For, do you know, \textit{Lucrezia}, as God lives,\\
Said one day \textit{Agnolo}, his very self,\\
To \textit{Rafael}... I have known it all these years...\\
(When the young man was flaming out his thoughts\\
Upon a palace-wall for \textsc{Rome} to see,\\
Too lifted up in heart because of it.)\\
`Friend, there's a certain sorry little scrub\\
Goes up \& down our \textsc{Florence}, none cares how,\\
Who, were he set to plan \& execute\\
As you are, pricked on by your popes \& kings,\\
Would bring the sweat into that brow of yours!'\\
To \textit{Rafael}'s! And indeed the arm is wrong.\\
I hardly dare... Yet, only you to see,\\
Give the chalk here -- quick, thus, the line should go!\\
Aye, but the soul! he's \textit{Rafael}! Rub it out!\\
Still, all I care for, if he spoke the truth,\\
(What he? why, who but \textit{Michel Agnolo}?\\
Do you forget already words like those?)\\
If really there was such a chance, so lost --\\
Is, whether you're -- not grateful -- but more pleased.\\
Well, let me think so. And you smile indeed!\\
This hour has been an hour! Another smile?\\
If you would sit thus by me every night\\
I should work better, do you comprehend?\\
I mean that I should earn more, give you more.\\
See, it is settled dusk now; there's a star;\\
\textit{Morello}'s gone, the watch-lights show the wall,\\
The cue-owls speak the name we call them by.\\
Come from the window, love; come in, at last,\\
Inside the melancholy little house\\
We built to be so gay with. God is just.\\
King \textit{Francis} may forgive me: oft at nights\\
When I look up from painting, eyes tired out,\\
The walls become illumined, brick from brick\\
Distinct, instead of mortar, fierce bright gold,\\
That gold of his I did cement them with!\\
Let us but love each other. Must you go?\\
That cousin here again? He waits outside?\\
Must see you? You, and not with me? Those loans?\\
More gaming debts to pay? You smiled for that?\\
Well, let smiles buy me! Have you more to spend?\\
While hand \& eye \& something of a heart\\
Are left me, work's my ware, and what's it worth?\\
I'll pay my fancy. Only let me sit\\
The grey remainder of the evening out,\\
Idle, you call it, and muse perfectly\\
How I could paint, were I but back in France,\\
One picture, just one more -- the Virgin's face,\\
Not yours this time! I want you at my side\\
To hear them -- that is, \textit{Michel Agnolo} --\\
Judge all I do and tell you of its worth.\\
Will you? Tomorrow, satisfy your friend.\\
I take the subjects for his corridor,\\
Finish the portrait out of hand -- there, there,\\
And throw him in another thing or two\\
If he demurs; the whole should prove enough\\
To pay for this same cousin's freak. Beside,\\
What's better and what's all I care about,\\
Get you the 13 {\hoskeroe scudi} for the ruff!\\
Love, does that please you? Ah, but what does he,\\*
The cousin? What does he to please you more?\\!

I am grown peaceful as old age tonight.\\*
I regret little; I would change still less.\\
Since there my past life lies, why alter it?\\
The very wrong to \textit{Francis}! It is true\\
I took his coin, was tempted and complied,\\
And built this house and sinned, and all is said.\\
My father \& my mother died of want.\\
Well, had I riches of my own? You see\\
How one gets rich! Let each one bear his lot.\\
They were born poor, lived poor, and poor they died:\\
And I have laboured somewhat in my time\\
And not been paid profusely. Some good son\\
Paint my 200 pictures -- let him try!\\
No doubt, there's something strikes a balance. Yes,\\
You loved me quite enough. it seems to-night.\\
This must suffice me here. What would one have?\\
In heaven, perhaps, new chances, one more chance --\\
Four great walls in the \textsc{New Jerusalem},\\
Meted on each side by the angel's reed,\\
For \textit{Leonard}, \textit{Rafael}, \textit{Agnolo} and me\\
To cover -- the three first without a wife,\\
While I have mine! So -- still they overcome\\*
Because there's still \textit{Lucrezia}, as I choose.\\!

Again the cousin's whistle! Go, my love.
\end{verse}

\subsection{}

\blfootnote{Miss Emily Brontë (1818 -- 1848), \cite{wuthering}.}\settowidth{\versewidth}{My walk home was lengthened by a diversion in the direction of the kirk. When beneath its walls, I perceived decay had made progress, even in seven months: many a window showed black gaps deprived of glass; and slates jutted off here there, beyond the right line of the roof, to be gradually worked off in coming autumn storms.}
\begin{verse}[\versewidth]
My walk home was lengthened by a diversion in the direction of the kirk. When beneath its walls, I perceived decay had made progress, even in seven months: many a window showed black gaps deprived of glass; and slates jutted off here \& there, beyond the right line of the roof, to be gradually worked off in coming autumn storms.\\!

I sought, and soon discovered, the three headstones on the slope next the moor: the middle one grey, and \sfrac{$1$}{$2$} buried in the heath; \textit{Edgar Linton}'s only harmonized by the turf and moss creeping up its foot; \textit{Heathcliff}'s still bare.\\!

I lingered round them, under that benign sky: watched the moths fluttering among the heath \& harebells, listened to the soft wind breathing through the grass, and wondered how any one could ever imagine unquiet slumbers for the sleepers in that quiet earth.
\end{verse}

\subsection{}

\blfootnote{Robert Browning (1828 -- 1889), \cite{norton}. These words are taken from Browning's \refpoem{Last Ride Together}.}Who knows but the world may end tonight?

\section{}

\subsection{}

\blfootnote{$\mathbb{R}$ Prof Alfred Housman (1859 -- 1936), \cite{norton}.}\settowidth{\versewidth}{    The land you used to plough.}
\begin{verse}[\versewidth]
`Is my team ploughing,\\*
\vin That I was used to drive\\
And hear the harness jingle\\*
\vin When I was man alive?'\\!

Ay, the horses trample,\\*
\vin The harness jingles now;\\
No change though you lie under\\*
\vin The land you used to plough.\\!

`Is football playing\\*
\vin Along the river shore,\\
With lads to chase the leather,\\*
\vin Now I stand up no more?'\\!

Ay, the ball is flying;\\*
\vin The lads play heart \& soul;\\
The goal stands up, the keeper\\*
\vin Stands up to keep the goal.\\!

`Is my girl happy,\\*
\vin That I thought hard to leave,\\
And has she tired of weeping\\*
\vin As she lies down at eve?'\\!

Ay, she lies down lightly;\\*
\vin She lies not down to weep.\\
Your girl is well contented.\\*
\vin Be still, my lad, and sleep.\\!

`Is my friend hearty,\\*
\vin Now I am thin \& pine,\\
And has he found to sleep in\\*
\vin A better bed than mine?'\\!

Yes, lad, I lie easy;\\*
\vin I lie as lads would choose;\\
I cheer a dead man's sweetheart;\\*
\vin Never ask me whose.
\end{verse}

\subsection{}

\blfootnote{Prof Alfred Housman (1859 -- 1936), \cite{norton}.}\settowidth{\versewidth}{And the bridegroom all night through}
\begin{verse}[\versewidth]
When I watch the living meet,\\*
\vin And the moving pageant file\\
Warm \& breathing through the street\\*
\vin Where I lodge a little while,\\!

If the heats of hate \& lust\\*
\vin In the house of flesh are strong,\\
Let me mind the house of dust\\*
\vin Where my sojourn shall be long.\\!

In the nation that is not\\*
\vin Nothing stands that stood before;\\
There revenges are forgot,\\*
\vin And the hater hates no more;\\!

Lovers lying two \& two\\*
\vin Ask not whom they sleep beside,\\
And the bridegroom all night through\\*
\vin Never turns him to the bride.
\end{verse}

\subsection{}

\blfootnote{Robert Browning (1828 -- 1889), \cite{norton}. These words are from Browning's \refpoem{Bishop Blougram's Apology}.}\settowidth{\versewidth}{Our interest's on the dangerous edge of things:}
\begin{verse}[\versewidth]
Our interest's on the dangerous edge of things:\\*
The honest thief, the tender murderer,\\*
The superstitious atheist.
\end{verse}

\section{}

\subsection{}

\blfootnote{Edward Thomas (1878 -- 1917), \cite{norton}.}\settowidth{\versewidth}{And screwed along the furrow till the brass flashed}
\begin{verse}[\versewidth]
As the team's head-brass flashed out on the turn\\*
The lovers disappeared into the wood.\\
I sat among the boughs of the fallen elm\\
That strewed an angle of the fallow, and\\
Watched the plough narrowing a yellow square\\
Of charlock. Every time the horses turned\\
Instead of treading me down, the ploughman leaned\\
Upon the handles to say or ask a word,\\
About the weather, next about the war.\\
Scraping the share he faced towards the wood,\\
And screwed along the furrow till the brass flashed\\
Once more. The blizzard felled the elm whose crest\\
I sat in, by a woodpecker's round hole,\\
The ploughman said. `When will they take it away?'\\
`When the war's over.' So the talk began --\\
One minute \& an interval of 10,\\
A minute more \& the same interval.\\
`Have you been out?' `No.' `And don’t want to, perhaps?'\\
`If I could only come back again, I should.\\
I could spare an arm. I shouldn't want to lose\\
A leg. If I should lose my head, why, so,\\
I should want nothing more... Have many gone\\
From here?' `Yes.' `Many lost?' `Yes, a good few.\\
Only two teams work on the farm this year.\\
One of my mates is dead. The second day\\
In France they killed him. It was back in march,\\
The very night of the blizzard, too. Now if\\
He had stayed here we should have moved the tree.'\\
`And I should not have sat here. Everything\\
Would have been different. For it would have been\\
Another world.' `Ay, \& a better, though,\\
If we could see all, all might seem good.' Then\\
The lovers came out of the wood again:\\
The horses started and for the last time\\
I watched the clods crumble \& topple over\\*
After the ploughshare and the stumbling team.
\end{verse}

\subsection{}

\blfootnote{Frederick Tuckerman (1821 -- 1873), \cite{norton}.}\settowidth{\versewidth}{Thin little leaves of wood fern, ribbed toothed,}
\begin{verse}[\versewidth]
Thin little leaves of wood fern, ribbed \& toothed,\\*
Long curved sail needles of the green pitch pine,\\
With common sandgrass, skirt the horizon line,\\
And over these the incorruptible blue!\\
Here let me gently lie and softly view\\
All world asperities, lightly touched \& smoothed\\
As by his gracious hand, the great Bestower.\\
What though the year be late? some colors run\\
Yet through the dry, some links of melody.\\
Still let me be, by such, assuaged \& soothed\\
And happier made, as when, our schoolday done,\\
We hunted on from flower to frosty flower,\\
Tattered and dim, the last red butterfly,\\*
Or the old grasshopper molasses-mouthed.
\end{verse}

\subsection{}

\blfootnote{Arthur Balfour, 1st Earl of Balfour (1848 -- 1930), \cite{odq}.}Nothing matters very much and very few things matter at all.

\chapter{October}

\section{}

\subsection{}

\blfootnote{John 1.1-5,9-13, \cite{kjv}. A tradition is floating around that this translation of John 1.1 was first suggested by Sir Thomas More -- ironically enough, since More was so fanatically opposed to Tyndale, the chief architect of the King James Version -- although the Almanacker has been unable to pin down a precise source.}In the beginning was the Word, and the Word was with God, and the Word was God.

The same was in the beginning with God.

All things were made by him; and without him was not any thing made that was made.

In him was life; and the life was the light of men.

And the light shineth in darkness; and the darkness comprehended it not.

That was the true light, which lighteth every man that cometh into the world.

He was in the world, and the world was made by him, and the world knew him not.

He came unto his own, and his own received him not.

But as many as received him, to them gave he power to become the sons of God, even to them that believe on his name:

Which were born, not of blood, nor of the will of the flesh, nor of the will of man, but of God.

\subsection{}

\blfootnote{Numbers 6.24-26, \cite{kjv}.}The {\hoskeroe LORD} bless thee, and keep thee:

The {\hoskeroe LORD} make his face shine upon thee, and be gracious unto thee:

The {\hoskeroe LORD} lift up his countenance upon thee, and give thee peace.

\subsection{}

\blfootnote{Genesis 9.6, \cite{kjv}.}Whoso sheddeth man's blood, by man shall his blood be shed: for in the image of God made he man.

\section{}

\subsection{}

\blfootnote{Psalm 3, \cite{bcp}.}Lord, how are they increased that trouble me; $\wp$ many are they that rise against me.

Many one there is to say of my soul, $\wp$ there is no help for him in his God.

But thou, O Lord art my defender; $\wp$ thou art my worship, and the lifter up of my head.

I did call upon the Lord with my voice, $\wp$ and he heard me out of his holy hill.

I laid me down and slept, and rose up again, $\wp$ for the Lord sustained me.

I will not be afraid for ten thousands of the people $\wp$ that have set themselves against me round about.

Up, Lord, and help me, O my God, for thou smitest all mine enemies upon the cheekbone; $\wp$ thou hast broken the teeth of the ungodly.

Salvation belongeth unto the Lord, $\wp$ and thy blessing is upon thy people.

\subsection{}

\blfootnote{`For Aid Against All Perils', \cite{bcp}. The original prayer concludes, `for the love of thy only Son, our Saviour Jesus Christ. Amen.'}Lighten our darkness, we beseech thee, O Lord; and by thy great mercy defend us from all perils and dangers of this night. Amen.

\subsection{}

\blfootnote{Matthew 19.6, \cite{kjv}.}{\color{red} What therefore God hath joined together, let not man put assunder.}

\section{}

\subsection{}

\blfootnote{Psalm 6, \cite{bcp}.}O Lord, rebuke me not in thine indignation, $\wp$ neither chasten me in thy displeasure.

Have mercy on me, O Lord, for I am weak; $\wp$ O Lord, heal me, for my bones are vexed.

My soul also is sore troubled; $\wp$ but, Lord, how long wilt thou punish me?

Turn thee, O Lord, and deliver my soul; $\wp$ O save me for thy mercy's sake.

For in death no man remembereth thee, $\wp$ and who will give thee thanks in the pit?

I am weary of my groaning; every night I wash my bed, $\wp$ and water my couch with my tears.

My beauty is gone for very trouble, $\wp$ and worn away because of all my enemies.

Away from me, all ye that work vanity, $\wp$ for the Lord hath heard the voice of my weeping.

The Lord hath heard my petition; $\wp$ the Lord will receive my prayer.

All my enemies shall be confounded, and sore vexed; $\wp$ they shall be turned back, and put to shame suddenly.

\subsection{}

\blfootnote{`For Peace', \cite{bcp}. The original prayer concludes, `through the merits of Jesus Christ our Saviour. Amen.'}God, from whom all holy desires, all good counsels, and all just works do proceed: give unto thy servants that peace which the world cannot give; that our hearts may be set to obey thy commandments, and also that by thee we being defended from the fear of our enemies may pass our time in rest \& quietness. Amen.

\subsection{}

\blfootnote{`Joshua 23.14', Joshua 23.14, \cite{kjv}.}I am going the way of all the earth.

\section{}

\subsection{}

\blfootnote{Job 19.2-29, \cite{kjv}.}I called my servant, and he gave me no answer; I entreated him with my mouth.

My breath is strange to my wife, though I entreated for the children's sake of mine own body.

Yea, young children despised me; I arose, \& they spake against me.

All my inward friends abhorred me: and they whom I loved are turned against me.

My bone cleaveth to my skin and to my flesh, and I am escaped with the skin of my teeth.

Have pity upon me, have pity upon me, O ye my friends; for the hand of God hath touched me.

Why do ye persecute me as God, and are not satisfied with my flesh?

O that my words were now written! O that they were printed in a book!

That they were graven with an iron pen \& lead in the rock for ever!

For I know that my redeemer liveth, and that he shall stand at the latter day upon the earth:

And though after my skin worms destroy this body, yet in my flesh shall I see God:

Whom I shall see for myself, and mine eyes shall behold, and not another; though my reins be consumed within me.

But ye should say, Why persecute we him, seeing the root of the matter is found in me?

Be ye afraid of the sword: for wrath bringeth the punishments of the sword, that ye may know there is a judgment.

\subsection{}

\blfootnote{Solomon 2.23-24, 3.1-4,7-8, \cite{kjv}.}God created man to be immortal, and made him to be an image of his own eternity.

Nevertheless, through envy of the devil came death into the world: and they that do hold of his side do find it.

But the souls of the righteous are in the hand of God, and there shall no torment touch them.

In the sight of the unwise they seemed to die: and their departure is taken for misery,

And their going from us to be utter destruction: but they are in peace.

For though they be punished in the sight of men, yet is their hope full of immortality.

In the time of their visitation they shall shine, and run to \& fro like sparks among the stubble.

They shall judge the nations, and have dominion over the peoples, and their Lord shall reign for ever.

\subsection{}

\blfootnote{Deuteronomy 30.15, \cite{kjv}.}I have set before thee this day life and good, and death and evil.

\section{}

\subsection{}

\blfootnote{Job 38-41, 42.1-6, \cite{kjv}. \P 98. Where the Almanacker has put `ox', the KJV reads `unicorn'.}Then the {\hoskeroe LORD} answered \textit{Job} out of the whirlwind, and said,

Who is this that darkeneth counsel by words without knowledge? Gird up now thy loins like a man;

For I will demand of thee, and answer thou me.

Where wast thou when I laid the foundations of the earth? Declare, if thou hast understanding.

Who hath laid the measures thereof, if thou knowest? Or who hath stretched the line upon it?

Whereupon are the foundations thereof fastened? Or who laid the corner stone thereof;

When the morning stars sang together, and all the sons of God shouted for joy?

Or who shut up the sea with doors, when it brake forth, as if it had issued out of the womb?

When I made the cloud the garment thereof, and thick darkness a swaddlingband for it,

And brake up for it my decreed place, and set bars and doors,

And said, Hitherto shalt thou come, but no further: and here shall thy proud waves be stayed?

Hast thou commanded the morning since thy days; and caused the dayspring to know his place;

That it might take hold of the ends of the earth, that the wicked might be shaken out of it?

It is turned as clay to the seal; and they stand as a garment.

And from the wicked their light is withholden, and the high arm shall be broken.

Hast thou entered into the springs of the sea? Or hast thou walked in the search of the depth?

Have the gates of death been opened unto thee? Or hast thou seen the doors of the shadow of death?

Hast thou perceived the breadth of the earth? Declare if thou knowest it all.

Where is the way where light dwelleth? And as for darkness, where is the place thereof,

That thou shouldest take it to the bound thereof, and that thou shouldest know the paths to the house thereof?

Knowest thou it, because thou wast then born? Or because the number of thy days is great?

Hast thou entered into the treasures of the snow? Or hast thou seen the treasures of the hail,

Which I have reserved against the time of trouble, against the day of battle \& war?

By what way is the light parted, which scattereth the east wind upon the earth?

Who hath divided a watercourse for the overflowing of waters, or a way for the lightning of thunder;

To cause it to rain on the earth, where no man is; on the wilderness, wherein there is no man;

To satisfy the desolate and waste ground; and to cause the bud of the tender herb to spring forth?

Hath the rain a father? Or who hath begotten the drops of dew?

Out of whose womb came the ice? And the hoary frost of heaven, who hath gendered it?

The waters are hid as with a stone, and the face of the deep is frozen.

Canst thou bind the sweet influences of \textit{Pleiades}, or loose the bands of \textit{Orion}?

Canst thou bring forth \textit{Mazzaroth} in his season? Or canst thou guide \textit{Arcturus} with his sons?

Knowest thou the ordinances of heaven? Canst thou set the dominion thereof in the earth?

Canst thou lift up thy voice to the clouds, that abundance of waters may cover thee?

Canst thou send lightnings, that they may go, and say unto thee, Here we are?

Who hath put wisdom in the inward parts? Or who hath given understanding to the heart?

Who can number the clouds in wisdom? Or who can stay the bottles of heaven,

When the dust groweth into hardness, and the clods cleave fast together?

Wilt thou hunt the prey for the lion? Or fill the appetite of the young lions,

When they couch in their dens, and abide in the covert to lie in wait?

Who provideth for the raven his food? When his young ones cry unto God, they wander for lack of meat.

Knowest thou the time when the wild goats of the rock bring forth? Or canst thou mark when the hinds do calve?

Canst thou number the months that they fulfil? Or knowest thou the time when they bring forth?

They bow themselves, they bring forth their young ones, they cast out their sorrows.

Their young ones are in good liking, they grow up with corn; they go forth, and return not unto them.

Who hath sent out the wild ass free? Or who hath loosed the bands of the wild ass?

Whose house I have made the wilderness, and the barren land his dwellings.

He scorneth the multitude of the city, neither regardeth he the crying of the driver.

The range of the mountains is his pasture, and he searcheth after every green thing.

Will the ox be willing to serve thee, or abide by thy crib?

Canst thou bind the ox with his band in the furrow? Or will he harrow the valleys after thee?

Wilt thou trust him, because his strength is great? Or wilt thou leave thy labour to him?

Wilt thou believe him, that he will bring home thy seed, and gather it into thy barn?

Gavest thou the goodly wings unto the peacocks? Or wings and feathers unto the ostrich,

Which leaveth her eggs in the earth, and warmeth them in dust, and forgetteth that the foot may crush them, or that the wild beast may break them?

She is hardened against her young ones, as though they were not hers: her labour is in vain without fear;

Because God hath deprived her of wisdom, neither hath he imparted to her understanding.

What time she lifteth up herself on high, she scorneth the horse and his rider.

Hast thou given the horse strength? Hast thou clothed his neck with thunder?

Canst thou make him afraid as a grasshopper? The glory of his nostrils is terrible.

He paweth in the valley, and rejoiceth in his strength: he goeth on to meet the armed men.

He mocketh at fear, and is not affrighted; neither turneth he back from the sword.

The quiver rattleth against him, the glittering spear and the shield.

He swalloweth the ground with fierceness and rage: neither believeth he that it is the sound of the trumpet.

He saith among the trumpets, Ha, ha; and he smelleth the battle afar off,

The thunder of the captains, and the shouting.

Doth the hawk fly by thy wisdom, and stretch her wings toward the south?

Doth the eagle mount up at thy command, and make her nest on high?

She dwelleth and abideth on the rock, upon the crag of the rock, and the strong place.

From thence she seeketh the prey, and her eyes behold afar off.

Her young ones also suck up blood: and where the slain are, there is she.

Moreover, the {\hoskeroe LORD} answered \textit{Job}, and said,

Shall he that contendeth with the Almighty instruct him? He that reproveth God, let him answer it.

Then \textit{Job} answered the {\hoskeroe LORD}, and said,

Behold, I am vile; what shall I answer thee? I will lay mine hand upon my mouth.

Once have I spoken; but I will not answer: yea, twice; but I will proceed no further.

Then answered the {\hoskeroe LORD} unto \textit{Job} out of the whirlwind, and said,

Gird up thy loins now like a man:

I will demand of thee, and declare thou unto me.

Wilt thou also disannul my judgment? Wilt thou condemn me, that thou mayest be righteous?

Hast thou an arm like God? Or canst thou thunder with a voice like him?

Deck thyself now with majesty and excellency; and array thyself with glory \& beauty.

Cast abroad the rage of thy wrath: and behold every one that is proud, and abase him.

Look on every one that is proud, and bring him low; and tread down the wicked in their place.

Hide them in the dust together; and bind their faces in secret.

Then will I also confess unto thee yhat thine own right hand can save thee.

Behold now \textit{Behemoth}, which I made with thee; he eateth grass as an ox.

Lo now, his strength is in his loins, and his force is in the navel of his belly.

He moveth his tail like a cedar: the sinews of his stones are wrapped together.

His bones are as strong pieces of brass; his bones are like bars of iron.

He is the chief of the ways of God: he that made him can make his sword to approach unto him.

Surely the mountains bring him forth food, where all the beasts of the field play.

He lieth under the shady trees, in the covert of the reed, and fens.

The shady trees cover him with their shadow; the willows of the brook compass him about.

Behold, he drinketh up a river, and hasteth not: he trusteth that he can draw up \textsc{Jordan} into his mouth.

He taketh it with his eyes: his nose pierceth through snares.

Canst thou draw out leviathan with an hook, or his tongue with a cord which thou lettest down?

Canst thou put an hook into his nose, or bore his jaw through with a thorn?

Will he make many supplications unto thee? Will he speak soft words unto thee?

Will he make a covenant with thee? Wilt thou take him for a servant for ever?

Wilt thou play with him as with a bird? Or wilt thou bind him for thy maidens?

Shall the companions make a banquet of him? Shall they part him among the merchants?

Canst thou fill his skin with barbed irons, or his head with fish spears?

Lay thine hand upon him; remember the battle, do no more.

Behold, the hope of him is in vain: shall not one be cast down even at the sight of him?

None is so fierce that dare stir him up: who then is able to stand before me?

Who hath prevented me, that I should repay him? Whatsoever is under the whole heaven is mine.

I will not conceal his parts, nor his power, nor his comely proportion.

Who can discover the face of his garment? Or who can come to him with his double bridle?

Who can open the doors of his face? His teeth are terrible round about.

His scales are his pride, Shut up together as with a close seal.

One is so near to another, that no air can come between them.

They are joined one to another, they stick together, that they cannot be sundered.

By his neesings a light doth shine, and his eyes are like the eyelids of the morning.

Out of his mouth go burning lamps, and sparks of fire leap out.

Out of his nostrils goeth smoke, as out of a seething pot or caldron.

His breath kindleth coals, and a flame goeth out of his mouth.

In his neck remaineth strength, and sorrow is turned into joy before him.

The flakes of his flesh are joined together: they are firm in themselves; they cannot be moved.

His heart is as firm as a stone; yea, as hard as a piece of the nether millstone.

When he raiseth up himself, the mighty are afraid: by reason of breakings they purify themselves.

The sword of him that layeth at him cannot hold: the spear, the dart, nor the habergeon.

He esteemeth iron as straw, and brass as rotten wood.

The arrow cannot make him flee: slingstones are turned with him into stubble.

Darts are counted as stubble: he laugheth at the shaking of a spear.

Sharp stones are under him: he spreadeth sharp pointed things upon the mire.

He maketh the deep to boil like a pot: he maketh the sea like a pot of ointment.

He maketh a path to shine after him; one would think the deep to be hoary.

Upon earth there is not his like, who is made without fear.

He beholdeth all high things: he is a king over all the children of pride.

Then \textit{Job} answered the {\hoskeroe LORD}, and said,

I know that thou canst do every thing, snd that no thought can be withholden from thee.

Who is he that hideth counsel without knowledge? Therefore have I uttered that I understood not;

Things too wonderful for me, which I knew not.

Hear, I beseech thee, and I will speak: I will demand of thee, and declare thou unto me.

I have heard of thee by the hearing of the ear: but now mine eye seeth thee.

Wherefore I abhor myself, and repent in dust and ashes.

\subsection{}

\blfootnote{`For Purity', \cite{bcp}.}Almighty God, unto whom all hearts be open, all desires known, and from whom no secrets are hid: cleanse the thoughts of our hearts by the inspiration of thy Holy Spirit, that we may perfectly love thee, and worthily magnify thy holy name. Amen.

\subsection{}

\blfootnote{Job 1.21, \cite{kjv}.}The {\hoskeroe LORD} gave, and the {\hoskeroe LORD} hath taken away; blessed be the name of the {\hoskeroe LORD}.

\section{}

\subsection{}

\blfootnote{Psalm 19, \cite{bcp}.}The heavens declare the glory of God, $\wp$ and the firmament sheweth his handywork.

One day telleth another, $\wp$ and one night certifieth another.

There is neither speech nor language, $\wp$ but their voices are heard among them.

Their sound is gone out into all lands, $\wp$ and their words into the ends of the world.

In them hath he set a tabernacle for the sun, $\wp$ which cometh forth as a bridegroom out of his chamber, and rejoiceth as a giant to run his course.

It goeth forth from the uttermost part of the heaven, and runneth about unto the end of it again, $\wp$ and there is nothing hid from the heat thereof.

The law of the Lord is an undefiled law, converting the soul; $\wp$ the testimony of the Lord is sure, and giveth wisdom unto the simple.

The statutes of the Lord are right, and rejoice the heart; $\wp$ the commandment of the Lord is pure, and giveth light unto the eyes.

The fear of the Lord is clean, and endureth for ever; $\wp$ the judgements of the Lord are true, and righteous altogether.

More to be desired are they than gold, yea, than much fine gold, $\wp$ sweeter also than honey, and the honey-comb.

\subsection{}

\blfootnote{Psalm 7.12-17, \cite{bcp}.}God is a righteous judge, strong \& patient; $\wp$ and God is provoked every day.

If a man will not turn, he will whet his sword; $\wp$ he hath bent his bow, and made it ready.

He hath prepared for him the instruments of death; $\wp$ he ordaineth his arrows against the persecutors.

Behold, the ungodly travaileth with iniquity; $\wp$ he hath conceived mischief, and brought forth falsehood.

He hath graven \& digged up a pit, $\wp$ and is fallen himself into the destruction that he made for other.

For his travail shall come upon his own head, $\wp$ and his wickedness shall fall on his own pate.

\subsection{}

\blfootnote{Job 32.9, \cite{kjv}.}Great men are not always wise.

\section{}

\subsection{}

\blfootnote{$\mathbb{R}$ Psalm 24, \cite{kjv}.}The earth is the {\hoskeroe LORD}'s, and the fulness thereof; the world, and they that dwell therein.

For he hath founded it upon the seas, and established it upon the floods.

Who shall ascend into the hill of the {\hoskeroe LORD}? Or who shall stand in his holy place?

He that hath clean hands, and a pure heart; who hath not lifted up his soul unto vanity, nor sworn deceitfully.

He shall receive the blessing from the {\hoskeroe LORD}, and righteousness from the God of his salvation.

This is the generation of them that seek him, that seek thy face, O Jacob. {\hoskeroe Selah}.

Lift up your heads, O ye gates; and be ye lift up, ye everlasting doors; and the King of glory shall come in.

Who is this King of glory? The {\hoskeroe LORD} strong and mighty, the {\hoskeroe LORD} mighty in battle.

Lift up your heads, O ye gates; even lift them up, ye everlasting doors; and the King of glory shall come in.

Who is this King of glory? The {\hoskeroe LORD} of hosts, he is the King of glory. {\hoskeroe Selah}.

\subsection{}

\blfootnote{Psalm 10.1-2,6,8-9,13,17, \cite{bcp}.}Why standest thou so far off, O Lord, $\wp$ and hidest thy face in the needful time of trouble?

The ungodly, for his own lust, doth persecute the poor; $\wp$ let them be taken in the crafty wiliness that they have imagined.

For he hath said in his heart, Tush, I shall never be cast down; $\wp$ there shall no harm happen unto me.

He sitteth lurking in the thievish corners of the streets, $\wp$ and privily in his lurking dens doth he murder the innocent; his eyes are set against the poor.

For he lieth waiting secretly; even as a lion lurketh he in his den, $\wp$ that he may ravish the poor.

Arise, O Lord God, and lift up thine hand; $\wp$ forget not the poor.

Break thou the power of the ungodly \& malicious; $\wp$ search out his ungodliness, until thou find none.

\subsection{}

\blfootnote{Psalm 145.9, \cite{kjv}.}The {\hoskeroe LORD} is loving unto every man, and his mercy is over all his works.

\section{}

\subsection{}

\blfootnote{Psalm 33.1-11, \cite{bcp}.}Rejoice in the Lord, O ye righteous, $\wp$ for it becometh well the just to be thankful.

Praise the Lord with harp; $\wp$ sing praises unto him with the lute, and instrument of ten strings.

Sing unto the Lord a new song; $\wp$ sing praises lustily unto him with a good courage.

For the word of the Lord is true, $\wp$ and all his works are faithful.

He loveth righteousness and judgement; $\wp$ the earth is full of the goodness of the Lord.

By the word of the Lord were the heavens made, $\wp$ and all the hosts of them by the breath of his mouth.

He gathereth the waters of the sea together, as it were upon an heap, $\wp$ and layeth up the deep, as in a treasure-house.

Let all the earth fear the Lord; $\wp$ stand in awe of him, all ye that dwell in the world.

For he spake, and it was done; $\wp$ he commanded, and it stood fast.

The Lord bringeth the counsel of the heathen to nought, $\wp$ and maketh the devices of the people to be of none effect, and casteth out the counsels of princes.

The counsel of the Lord shall endure for ever, $\wp$ and the thoughts of his heart from generation to generation.

Blessed are the people, whose God is the Lord, $\wp$ and blessed are the folk, that he hath chosen to him to be his inheritance.

\subsection{}

\blfootnote{Psalm 4.6-8, \cite{kjv}.}There be many that say, Who will show us any good? {\hoskeroe LORD}, lift thou up the light of thy countenance upon us.

Thou hast put gladness in my heart, more than in the time that their corn \& their wine increased.

I will both lay me down in peace, \& sleep: for thou, {\hoskeroe LORD}, only makest me dwell in safety.

\subsection{}

\blfootnote{Proverbs 16.33, \cite{kjv}.}The lot is cast into the lap; but the whole disposing thereof is of the {\hoskeroe LORD}.

\section{}

\subsection{}

\blfootnote{Psalm 42, \cite{bcp}.}Like as the hart desireth the water-brooks, $\wp$ so longeth my soul after thee, O God.

My soul is athirst for God, yea, even for the living God; $\wp$ when shall I come to appear before the presence of God?

My tears have been my meat day \& night, $\wp$ while they daily say unto me, Where is now thy God?

Now when I think thereupon, I pour out my heart by myself, $\wp$ for I went with the multitude, and brought them forth into the house of God;

In the voice of praise \& thanksgiving, $\wp$ among such as keep holy-day.

Why art thou so full of heaviness, O my soul $\wp$ and why art thou so disquieted within me?

Put thy trust in God, $\wp$ for I will yet give him thanks for the help of his countenance.

My God, my soul is vexed within me, $\wp$ therefore will I remember thee concerning the land of \textsc{Jordan}, and the little hill of \textsc{Hermon}.

One deep calleth another, because of the noise of the water-pipes, $\wp$ all thy waves \& storms are gone over me.

The Lord hath granted his loving-kindness in the day-time, $\wp$ and in the night-season did I sing of him, and made my prayer unto the God of my life.

I will say unto the God of my strength, Why hast thou forgotten me? $\wp$ Why go I thus heavily, while the enemy oppresseth me?

My bones are smitten asunder as with a sword, $\wp$ while mine enemies that trouble me cast me in the teeth;

Namely, while they say daily unto me, $\wp$ Where is now thy God?

Why art thou so vexed, O my soul, $\wp$ and why art thou so disquieted within me?

O put thy trust in God, $\wp$ for I will yet thank him, which is the help of my countenance, and my God.

\subsection{}

\blfootnote{Psalm 11.1-5,7-8, \cite{bcp}.}In the Lord put I my trust; $\wp$ how say ye then to my soul, that she should flee as a bird unto the hill?

For lo, the ungodly bend their bow, and make ready their arrows within the quiver, $\wp$ that they may privily shoot at them which are true of heart.

If the foundations be destroyed, $\wp$ what hath the righteous done?

The Lord is in his holy temple; $\wp$ the Lord's seat is in heaven.

His eyes consider the poor, $\wp$ and his eyelids try the children of men.

Upon the ungodly he shall rain snares, fire \& brimstone, storm and tempest: $\wp$ this shall be their portion to drink.

For the righteous Lord loveth righteousness; $\wp$ his countenance will behold the thing that is just.

\subsection{}

\blfootnote{Psalm 141.2, \cite{kjv}.}Let my prayer be set forth before thee as incense; and the lifting up of my hands as the evening sacrifice.

\section{}

\subsection{}

\blfootnote{Psalm 46, \cite{bcp}.}God is our refuge \& strength, $\wp$ a very present help in trouble.

Therefore will not we fear, though the earth be removed, $\wp$ and though the mountains be carried into the midst of the sea;

Though the waters thereof roar \& be troubled, $\wp$ though the mountains shake with the swelling thereof. {\hoskeroe Selah}.

There is a river, the streams whereof shall make glad the city of God, $\wp$ the holy place of the tabernacles of the Most High.

God is in the midst of her; she shall not be moved: $\wp$ God shall help her, and that right early.

The heathen raged; the kingdoms were moved; $\wp$ he uttered his voice; the earth melted.

The Lord of hosts is with us; $\wp$ the God of Jacob is our refuge. {\hoskeroe Selah}.

Come, behold the works of the Lord, $\wp$ what desolations he hath made in the earth.

He maketh wars to cease unto the end of the earth; $\wp$ he breaketh the bow, and cutteth the spear in sunder; he burneth the chariot in the fire.

Be still, and know that I am God: $\wp$ I will be exalted among the heathen; I will be exalted in the earth.

The Lord of hosts is with us; $\wp$ the God of Jacob is our refuge. {\hoskeroe Selah}.

\subsection{}

\blfootnote{Psalm 23, \cite{kjv}.}The {\hoskeroe LORD} is my shepherd; I shall not want.

He maketh me to lie down in green pastures: he leadeth me beside the still waters.

He restoreth my soul: he leadeth me in the paths of righteousness for his name's sake.

Yea, though I walk through the valley of the shadow of death, I will fear no evil: for thou art with me; thy rod \& thy staff they comfort me.

Thou preparest a table before me in the presence of mine enemies: thou anointest my head with oil; my cup runneth over.

Surely goodness and mercy shall follow me all the days of my life: and I will dwell in the house of the {\hoskeroe LORD} for ever.

\subsection{}

\blfootnote{Psalm 18.8, \cite{bcp}.}Keep me as the apple of an eye; hide me under the shadow of thy wings.

\section{}

\subsection{}

\blfootnote{Psalm 51, \cite{bcp}.}Have mercy upon me, O God, after thy great goodness; $\wp$ according to the multitude of thy mercies do away mine offences.

Wash me thoroughly from my wickedness, $\wp$ and cleanse me from my sin.

For I acknowledge my faults, $\wp$ and my sin is ever before me.

Against thee only have I sinned, and done this evil in thy sight, $\wp$ that thou mightest be justified in thy saying, and clear when thou art judged.

Behold, I was shapen in wickedness, $\wp$ and in sin hath my mother conceived me.

But lo, thou requirest truth in the inward parts, $\wp$ and shalt make me to understand wisdom secretly.

Thou shalt purge me with hyssop, and I shall be clean; $\wp$ thou shalt wash me, and I shall be whiter than snow.

Thou shalt make me hear of joy \& gladness, $\wp$ that the bones which thou hast broken may rejoice.

Turn thy face from my sins, $\wp$ and put out all my misdeeds.

Make me a clean heart, O God, $\wp$ and renew a right spirit within me.

Cast me not away from thy presence, $\wp$ and take not thy Holy Spirit from me.

O give me the comfort of thy help again, $\wp$ and stablish me with thy free Spirit.

Then shall I teach thy ways unto the wicked, $\wp$ and sinners shall be converted unto thee.

Deliver me from blood-guiltiness, O God, thou that art the God of my health, $\wp$ and my tongue shall sing of thy righteousness.

Thou shalt open my lips, O Lord, $\wp$ and my mouth shall shew thy praise.

For thou desirest no sacrifice, else would I give it thee, $\wp$ but thou delightest not in burnt offerings.

The sacrifice of God is a troubled spirit; $\wp$ a broken and contrite heart, O God, shalt thou not despise.

\subsection{}

\blfootnote{Psalm 10.18-20, \cite{bcp}.}The Lord is king for ever \& ever, $\wp$ and the heathen are perished out of the land.

Lord, thou hast heard the desire of the poor; $\wp$ thou preparest their heart, and thine ear hearkeneth;

To help the fatherless \& poor unto their right, $\wp$ that the man of the earth be no more exalted against them.

\subsection{}

\blfootnote{Matthew 6.34, \cite{kjv}.}{\color{red} Sufficient unto the day is the evil thereof.}

\section{}

\subsection{}

\blfootnote{Psalm 104, \cite{kjv}.}Bless the {\hoskeroe LORD}, O my soul. O {\hoskeroe LORD} my God, thou art very great; thou art clothed with honour and majesty.

Who coverest thyself with light as with a garment: who stretchest out the heavens like a curtain:

Who layeth the beams of his chambers in the waters: who maketh the clouds his chariot: who walketh upon the wings of the wind:

Who maketh his angels spirits; his ministers a flaming fire:

Who laid the foundations of the earth, that it should not be removed for ever.

Thou coveredst it with the deep as with a garment: the waters stood above the mountains.

At thy rebuke they fled; at the voice of thy thunder they hasted away.

They go up by the mountains; they go down by the valleys unto the place which thou hast founded for them.

Thou hast set a bound that they may not pass over; that they turn not again to cover the earth.

He sendeth the springs into the valleys, which run among the hills.

They give drink to every beast of the field: the wild asses quench their thirst.

By them shall the fowls of the heaven have their habitation, which sing among the branches.

He watereth the hills from his chambers: the earth is satisfied with the fruit of thy works.

He causeth the grass to grow for the cattle, and herb for the service of man: that he may bring forth food out of the earth;

And wine that maketh glad the heart of man, and oil to make his face to shine, and bread which strengtheneth man's heart.

The trees of the {\hoskeroe LORD} are full of sap; the cedars of Lebanon, which he hath planted;

Where the birds make their nests: as for the stork, the fir trees are her house.

The high hills are a refuge for the wild goats; and the rocks for the conies.

He appointed the moon for seasons: the sun knoweth his going down.

Thou makest darkness, and it is night: wherein all the beasts of the forest do creep forth.

The young lions roar after their prey, and seek their meat from God.

The sun ariseth, they gather themselves together, and lay them down in their dens.

Man goeth forth unto his work and to his labour until the evening.

O {\hoskeroe LORD}, how manifold are thy works! in wisdom hast thou made them all: the earth is full of thy riches.

So is this great \& wide sea, wherein are things creeping innumerable, both small and great beasts.

There go the ships: there is that leviathan, whom thou hast made to play therein.

These wait all upon thee; that thou mayest give them their meat in due season.

That thou givest them they gather: thou openest thine hand, they are filled with good.

Thou hidest thy face, they are troubled: thou takest away their breath, they die, and return to their dust.

Thou sendest forth thy spirit, they are created: and thou renewest the face of the earth.

The glory of the {\hoskeroe LORD} shall endure for ever: the {\hoskeroe LORD} shall rejoice in his works.

He looketh on the earth, and it trembleth: he toucheth the hills, and they smoke.

I will sing unto the {\hoskeroe LORD} as long as I live: I will sing praise to my God while I have my being.

My meditation of him shall be sweet: I will be glad in the {\hoskeroe LORD}.

Let the sinners be consumed out of the earth, and let the wicked be no more.

Bless thou the {\hoskeroe LORD}, O my soul. Praise ye the {\hoskeroe LORD}.

\subsection{}

\blfootnote{Ruth 1.16-17, \cite{kjv}. This passage from Ruth often forms part of the Jewish marriage ceremony.}Intreat me not to leave thee, or to return from following after thee:

For whither thou goest, I will go; and where thou lodgest, I will lodge:

Thy people shall be my people, and thy God my God:

Where thou diest, will I die, and there will I be buried:

The {\hoskeroe LORD} do so to me, and more also, if ought but death part thee \& me.

\subsection{}

\blfootnote{Matthew 4.4, \cite{kjv}. Christ here is quoting Deuteronomy 8.3.}Man shall not live by bread alone, but by every word that proceedeth out of the mouth of God.

\section{}

\subsection{}

\blfootnote{Psalm 139, \cite{bcp}. \P 19. This seems to be a mistranslation. The full couplet ought to read something like, `If I say, ``Let only darkness cover me,/ And the light about me be night...{''}' (as the RSV renders it). \P 30. The Almanacker has deleted the couplet immediately following `And in thy book were all my members written.'}O Lord, thou hast searched me out \& known me; $\wp$ thou knowest my down-sitting \& mine up-rising, thou understandest my thoughts long before.

Thou art about my path, \& about my bed, $\wp$ and spiest out all my ways.

For lo, there is not a word in my tongue, $\wp$ but thou, O Lord, knowest it altogether.

Thou hast fashioned me behind \& before, $\wp$ and laid thine hand upon me.

Such knowledge is too wonderful \& excellent for me; $\wp$ I cannot attain unto it.

Whither shall I go then from thy Spirit, $\wp$ or whither shall I go then from thy presence?

If I climb up into heaven, thou art there; $\wp$ If I go down to hell, thou art there also.

If I take the wings of the morning, $\wp$ and remain in the uttermost parts of the sea;

Even there also shall thy hand lead me, $\wp$ and thy right hand shall hold me.

If I say, Peradventure the darkness shall cover me, $\wp$ then shall my night be turned to day.

Yea, the darkness is no darkness with thee, but the night is as clear as the day; $\wp$ the darkness \& light to thee are both alike.

For my reins are thine; $\wp$ thou hast covered me in my mother's womb.

I will give thanks unto thee, for I am fearfully and wonderfully made; $\wp$ marvellous are thy works, and that my soul knoweth right well.

My bones are not hid from thee, $\wp$ though I be made secretly, and fashioned beneath in the earth.

Thine eyes did see my substance, yet being unperfect, $\wp$ and in thy book were all my members written.

How dear are thy counsels unto me, O God! $\wp$ O how great is the sum of them!

If I tell them, they are more in number than the sand; when I wake up I am present with thee.

Wilt thou not slay the wicked, O God? $\wp$ Depart from me, ye blood-thirsty men.

For they speak unrighteously against thee, $\wp$ and thine enemies take thy name in vain.

Do not I hate them, O Lord, that hate thee, $\wp$ and am not I grieved with those that rise up against thee?

Yea, I hate them right sore, $\wp$ even as though they were mine enemies.

Try me, O God, and seek the ground of my heart; $\wp$ prove me, and examine my thoughts.

Look well if there be any way of wickedness in me, $\wp$ and lead me in the way everlasting.

\subsection{}

\blfootnote{Psalm 8, \cite{kjv}. The Almanacker has replaced the KJV's `our Lord' with the \refbook{BCP}'s `our Governor' in the first and last verses of this psalm.}O {\hoskeroe LORD} our governor, how excellent is thy name in all the earth, who hast set thy glory above the heavens.

Out of the mouth of babes \& sucklings hast thou ordained strength because of thine enemies that thou mightest still the enemy \& the avenger.

When I consider thy heavens, the work of thy fingers, the moon \& the stars, which thou hast ordained;

What is man, that thou art mindful of him, and the son of man, that thou visitest him?

For thou hast made him a little lower than the angels, and hast crowned him with glory \& honour.

Thou madest him to have dominion over the works of thy hands; thou hast put all things under his feet:

All sheep \& oxen, yea, and the beasts of the field;

The fowl of the air, \& the fish of the sea, and whatsoever passeth through the paths of the seas.

O {\hoskeroe LORD} our governor, how excellent is thy name in all the earth.

\subsection{}

\blfootnote{John 8.32, \cite{kjv}.}{\color{red} The truth shall make you free.}

\section{}

\subsection{}

\blfootnote{Psalm 147.1-11, \cite{kjv}.}Praise ye the {\hoskeroe LORD}: for it is good to sing praises unto our God; for it is pleasant; and praise is comely.

The {\hoskeroe LORD} doth build up \textsc{Jerusalem}: he gathereth together the outcasts of Israel.

He healeth the broken in heart, and bindeth up their wounds.

He telleth the number of the stars; he calleth them all by their names.

Great is our {\hoskeroe LORD}, and of great power: his understanding is infinite.

The {\hoskeroe LORD} lifteth up the meek: he casteth the wicked down to the ground.

Sing unto the {\hoskeroe LORD} with thanksgiving; sing praise upon the harp unto our God:

Who covereth the heaven with clouds, who prepareth rain for the earth, who maketh grass to grow upon the mountains.

He giveth to the beast his food, and to the young ravens which cry.

He delighteth not in the strength of the horse: he taketh not pleasure in the legs of a man.

The {\hoskeroe LORD} taketh pleasure in them that fear him, in those that hope in his mercy.

\subsection{}

\blfootnote{Psalm 15, \cite{bcp}. For self-evident reasons, some call this `the gentelman's psalm'.}Lord, who shall dwell in thy tabernacle, $\wp$ or who shall rest upon thy holy hill?

Even he that leadeth an uncorrupt life, $\wp$ and doeth the thing which is right, and speaketh the truth from his heart.

He that hath used no deceit in his tongue, nor done evil to his neighbour, $\wp$ and hath not slandered his neighbour.

He that setteth not by himself, but is lowly in his own eyes, $\wp$ and maketh much of them that fear the Lord.

He that sweareth unto his neighbour, and disappointeth him not, $\wp$ though it were to his own hindrance.

He that hath not given his money upon usury, $\wp$ nor taken reward against the innocent.

Whoso doeth these things $\wp$ shall never fall.

\subsection{}

\blfootnote{Mark 2.27, \cite{kjv}.}{\color{red} The sabbath was made for man, and not man for the sabbath.}

\section{}

\subsection{}

\blfootnote{Psalm 148.1-13, \cite{bcp}.}O praise the Lord of heaven; $\wp$ praise him in the height.

Praise him, all ye angels of his; $\wp$ praise him, all his host.

Praise him, sun \& moon; $\wp$ praise him, all ye stars \& light.

Praise him, all ye heavens, $\wp$ and ye waters that are above the heavens.

Let them praise the name of the Lord; $\wp$ for he spake the word, and they were made; he commanded, and they were created.

He hath made them fast for ever \& ever; $\wp$ he hath given them a law which shall not be broken.

Praise the Lord upon earth, $\wp$ ye dragons, and all deeps;

Fire \& hail, snow \& vapours, $\wp$ wind \& storm, fulfilling his word;

Mountains \& all hills, $\wp$ fruitful trees \& all cedars;

Beasts \& all cattle; $\wp$ worms \& feathered fowls;

Kings of the earth \& all people, $\wp$ princes \& all judges of the world;

Young men \& maidens, old men \& children, praise the name of the Lord, $\wp$ for his name only is excellent, and his praise above heaven \& earth.

\subsection{}

\blfootnote{Psalm 150, \cite{kjv}.}Praise ye the {\hoskeroe LORD}. Praise God in his sanctuary: praise him in the firmament of his power.

Praise him for his mighty acts: praise him according to his excellent greatness.

Praise him with the sound of the trumpet: praise him with the psaltery \& harp.

Praise him with the timbrel \& dance: praise him with stringed instruments \& organs.

Praise him upon the loud cymbals: praise him upon the high sounding cymbals.

Let every thing that hath breath praise the {\hoskeroe LORD}. praise ye the {\hoskeroe LORD}.

\subsection{}

\blfootnote{Matthew 7.3, \cite{kjv}.}{\color{red} Why beholdest thou the mote that is in thy brother's eye, but considerest not the beam that is in thine own eye?}

\section{}

\subsection{}

\blfootnote{`The Magnificat', \cite{bcp}. This song is a translation of Luke 1.46-55.}My soul doth magnify the Lord, $\wp$ and my spirit hath rejoiced in God my saviour.

For he hath regarded: $\wp$ the lowliness of his hand-maiden.

For, behold, from henceforth $\wp$ all generations shall call me blessed.

For he that is mighty hath magnified me, $\wp$ and holy is his name.

And his mercy is on them that fear him $\wp$ throughout all generations.

He hath showed strength with his arm; $\wp$ he hath scattered the proud in the imagination of their hearts.

He hath put down the mighty from their seat, $\wp$ and hath exalted the humble \& meek.

He hath filled the hungry with good things: $\wp$ and the rich he hath sent empty away.

He remembering of his mercy hath holpen his servant Israel: $\wp$ as he promised to our forefathers, \textit{Abraham} \& his seed, for ever.

\subsection{}

\blfootnote{`The Nunc Dimittis', \cite{bcp}. These lines are a translation of Luke 2.29-32.}Lord, now lettest thou thy servant depart in peace, $\wp$ according to thy word.

For mine eyes have seen $\wp$ thy salvation,

Which thou hast prepared $\wp$ before the face of all peoples;

To be a light to lighten the gentiles, $\wp$ and the glory of thy people Israel.

\subsection{}

\blfootnote{Matthew 6.24, \cite{kjv}.}{\color{red} No man can serve two masters.}

\section{}

\subsection{}

\blfootnote{`Benedictus', \cite{bcp}. This song is a translation of Luke 1.68-79.}Bless\`{e}d be the Lord God of Israel: $\wp$ for he hath visited and redeemed his people;

And he hath raised up a mighty salvation for us $\wp$ in the house of his servant \textit{David};

As he spoke by the mouth of his holy prophets $\wp$ which have been since the world began;

That we should be saved from our enemies $\wp$ and from the hand of all that hate us;

To perform the mercy promised to our forefathers, $\wp$ and to remember his holy covenant;

To perform the oath which he sware to our forefather \textit{Abraham} $\wp$ that he would give us;

That we being delivered out of the hands of our enemies: $\wp$ might serve him without fear;

In holiness \& righteousness before him $\wp$ all the days of our life.

And thou, child, shalt be called the prophet of the Highest: $\wp$ for thou shalt go before the face of the Lord to prepare his ways;

To give knowledge of salvation unto his people $\wp$ for the remission of their sins,

Through the tender mercy of our God, $\wp$ whereby the day spring from on high hath visited us;

To give light to them that sit in darkness, and in the shadow of death: $\wp$ to guide our feet into the way of peace.

\subsection{}

\blfootnote{`The Lord's Prayer', \cite{bcp}. The first four lines are a translation of Matthew 6.9-13. The doxology is of uncertain origin; a version of it appears in the Διδαχή, and in certain Byzantine manuscripts of Matthew.}{\color{red} Our Father, which art in heaven, hallowed be thy name; thy kingdom come; thy will be done in earth, as it is in heaven: give us this day our daily bread; and forgive us our trespasses, as we forgive them that trespass against us; and lead us not into temptation, but deliver us from evil. For thine is the kingdom, the power, and the glory, for ever \& ever. Amen.}

\subsection{}

\blfootnote{Matthew 6.3, \cite{kjv}.}{\color{red} Let not thy left hand know what thy right hand doeth.}

\section{}

\subsection{}

\blfootnote{Matthew 5.2-10, \cite{kjv}. These verses are sometimes called the Beatitudes, a term derived from the Latin word \textit{beatitudo}, meaning \textit{blessedness}, which was sometimes printed in the marginalia of the Vulgate at this point in Matthew's Gospel.}{\color{red} Bless\`{e}d are the poor in spirit: for theirs is the kingdom of heaven.

Bless\`{e}d are they that mourn: for they shall be comforted.

Bless\`{e}d are the meek: for they shall inherit the earth.

Bless\`{e}d are they which do hunger \& thirst after righteousness: for they shall be filled.

Bless\`{e}d are the merciful: for they shall obtain mercy.

Bless\`{e}d are the pure in heart: for they shall see God.

Bless\`{e}d are the peacemakers: for they shall be called children of God.

Bless\`{e}d are they which are persecuted for righteousness' sake: for theirs is the kingdom of heaven.}

\subsection{}

\blfootnote{Matthew 25.34-36,40, \cite{kjv}.}{\color{red} Come, ye blessed of my Father; inherit the kingdom prepared for you from the foundation of the world:

For I was an hungred, and ye gave me meat: I was thirsty, and ye gave me drink:

I was a stranger, and ye took me in: naked, and ye clothed me:

I was sick, and ye visited me: I was in prison, and ye came unto me.

Verily I say unto you, Inasmuch as ye have done it unto one of the least of these my brethren, ye have done it unto me.}

\subsection{}

\blfootnote{Matthew 7.1, \cite{kjv}.}{\color{red} Judge not, that ye be not judged.}

\section{}

\subsection{}

\blfootnote{$\mathbb{R}$ 1 Corinthians 13, \cite{kjv}. The original reads `charity' where the Almanacker has put `love'.}Though I speak with the tongues of men and of angels, and have not love, I am become as sounding brass, or a tinkling cymbal.

And though I have the gift of prophecy, and understand all mysteries, and all knowledge; and though I have all faith, so that I could remove mountains, and have not love, I am nothing.

And though I bestow all my goods to feed the poor, and though I give my body to be burned, and have not love, it profiteth me nothing.

Love suffereth long, and is kind; charity envieth not; love vaunteth not itself, is not puffed up,

Doth not behave itself unseemly, seeketh not her own, is not easily provoked, thinketh no evil;

Rejoiceth not in iniquity, but rejoiceth in the truth;

Beareth all things, believeth all things, hopeth all things, endureth all things.

Love never faileth: but whether there be prophecies, they shall fail; whether there be tongues, they shall cease; whether there be knowledge, it shall vanish away.

For we know in part, and we prophesy in part.

But when that which is perfect is come, then that which is in part shall be done away.

When I was a child, I spake as a child, I understood as a child, I thought as a child: but when I became a man, I put away childish things.

For now we see through a glass, darkly; but then face to face: now I know in part; but then shall I know even as also I am known.

And now abideth faith, hope, love, these three; but the greatest of these is love.

\subsection{}

\blfootnote{`Veni Creator Spiritus', John Cosin, Bishop of Durham (1594 -- 1672), \cite{bcp}. The hymn is ancient, being composed by a Frankish monk, Rabanus Maurus, in the ninth century, and translated into English by the Bishop of Durham. The original hymn concludes with a further verse and a half.}\settowidth{\versewidth}{Where thou art guide, no ill can come.}
\begin{verse}[\versewidth]
Come, Holy Ghost, our souls inspire,\\*
And lighten with celestial fire.\\
Thou the anointing spirit art,\\*
Who dost thy sevenfold gifts impart.\\!

Thy blessed unction from above\\*
Is comfort, life, and fire of love.\\
Enable with perpetual light\\*
The dullness of our blinded sight.\\!

Anoint \& cheer our soil\`{e}d face\\*
With the abundance of thy grace.\\
Keep far from foes, give peace at home:\\*
Where thou art guide, no ill can come.
\end{verse}

\subsection{}

\blfootnote{Matthew 13.57, \cite{kjv}.}{\color{red} A prophet is not without honour, save in his own country.}

\section{}

\subsection{}

\blfootnote{Ephesians 6.10-20, \cite{kjv}.}Be strong in the Lord, and in the power of his might.

Put on the whole armour of God, that ye may be able to stand against the wiles of the devil.

For we wrestle not against flesh and blood, but against principalities, against powers, against the rulers of the darkness of this world, against spiritual wickedness in high places.

Wherefore take unto you the whole armour of God, that ye may be able to withstand in the evil day, and having done all, to stand.

Stand therefore, having your loins girt about with truth, and having on the breastplate of righteousness;

And your feet shod with the preparation of the gospel of peace;

Above all, taking the shield of faith, wherewith ye shall be able to quench all the fiery darts of the wicked.

And take the helmet of salvation, and the sword of the Spirit, which is the word of God:

Praying always with all prayer \& supplication in the Spirit, and watching thereunto with all perseverance and supplication for all saints;

And for me, that utterance may be given unto me, that I may open my mouth boldly, to make known the mystery of the gospel,

For which I am an ambassador in bonds: that therein I may speak boldly, as I ought to speak.

\subsection{}

\blfootnote{James 1.16-21, \cite{kjv}.}Do not err, my beloved brethren.

Every good gift \& every perfect gift is from above, and cometh down from the Father of lights, with whom is no variableness, neither shadow of turning.

Of his own will begat he us with the word of truth, that we should be a kind of first fruits of his creatures.

Wherefore, my beloved brethren, let every man be swift to hear, slow to speak, slow to wrath:

For the wrath of man worketh not the righteousness of God.

Wherefore lay apart all filthiness and superfluity of naughtiness, and receive with meekness the engrafted word, which is able to save your souls.

\subsection{}

\blfootnote{Matthew 20.21, \cite{kjv}.}{\color{red} Render therefore unto Caesar the things which are Caesar's; and unto God the things that are God's.}

\section{}

\subsection{}

\blfootnote{Isaiah 6.1-8, \cite{kjv}.}In the year that King \textit{Uzziah} died I saw also the Lord sitting upon a throne, high \& lifted up, and his train filled the temple.

Above it stood the seraphims: each one had six wings; with twain he covered his face, and with twain he covered his feet, and with twain he did fly.

And one cried unto another, and said, Holy, holy, holy, is the {\hoskeroe LORD} of hosts: the whole earth is full of his glory.

And the posts of the door moved at the voice of him that cried, and the house was filled with smoke.

Then said I, Woe is me! For I am undone; because I am a man of unclean lips, and I dwell in the midst of a people of unclean lips: for mine eyes have seen the King, the {\hoskeroe LORD} of hosts.

Then flew one of the seraphims unto me, having a live coal in his hand, which he had taken with the tongs from off the altar:

And he laid it upon my mouth, and said, Lo, this hath touched thy lips; and thine iniquity is taken away, and thy sin purged.

Also I heard the voice of the Lord, saying,`Whom shall I send, and who will go for us? Then said I, Here am I; send me.

\subsection{}

\blfootnote{Isaiah 9.2-7, \cite{kjv}.}The people that walked in darkness have seen a great light: they that dwell in the land of the shadow of death, upon them hath the light shined.

Thou hast multiplied the nation, and not increased the joy: they joy before thee according to the joy in harvest, and as men rejoice when they divide the spoil.

For thou hast broken the yoke of his burden, and the staff of his shoulder, the rod of his oppressor, as in the day of Midian.

For every battle of the warrior is with confused noise, and garments rolled in blood; but this shall be with burning and fuel of fire.

For unto us a child is born, unto us a son is given: and the government shall be upon his shoulder: and his name shall be called Wonderful, Counseller, the mighty God, the everlasting Father, the Prince of Peace.

Of the increase of his government and peace there shall be no end, upon the throne of \textit{David}, and upon his kingdom, to order it, and to establish it with judgment and with justice from henceforth even for ever. The zeal of the {\hoskeroe LORD} of hosts will perform this.

\subsection{}

\blfootnote{Matthew 19.24, \cite{kjv}.}{\color{red} It is easier for a camel to go through the eye of a needle, than for a rich man to enter into the kingdom of God.}

\section{}

\subsection{}

\blfootnote{Isaiah 35.3-10, \cite{kjv}.}Strengthen ye the weak hands, and confirm the feeble knees.

Say to them that are of a fearful heart, Be strong, fear not: behold, your God will come with vengeance, even God with a recompence; he will come and save you.

Then the eyes of the blind shall be opened, and the ears of the deaf shall be unstopped.

Then shall the lame man leap as an hart, and the tongue of the dumb sing: for in the wilderness shall waters break out, and streams in the desert.

And the parched ground shall become a pool, and the thirsty land springs of water: in the habitation of dragons, where each lay, shall be grass with reeds \& rushes.

And an highway shall be there, and a way, and it shall be called The way of holiness; the unclean shall not pass over it; but it shall be for those: the wayfaring men, though fools, shall not err therein.

No lion shall be there, nor any ravenous beast shall go up thereon, it shall not be found there; but the redeemed shall walk there:

And the ransomed of the {\hoskeroe LORD} shall return, and come to \textsc{Zion} with songs and everlasting joy upon their heads: they shall obtain joy \& gladness, and sorrow \& sighing shall flee away.

\subsection{}

\blfootnote{Micah 3.1-4, \cite{kjv}.}But in the last days it shall come to pass, that the mountain of the house of the {\hoskeroe LORD} shall be established in the top of the mountains, and it shall be exalted above the hills; and people shall flow unto it.

And many nations shall come, and say, Come, and let us go up to the mountain of the {\hoskeroe LORD}, and to the house of the God of Jacob; and he will teach us of his ways, and we will walk in his paths: for the law shall go forth of \textsc{Zion}, and the word of the {\hoskeroe LORD} from \textsc{Jerusalem}.

And he shall judge among many people, and rebuke strong nations afar off; and they shall beat their swords into plowshares, and their spears into pruning-hooks: nation shall not lift up a sword against nation, neither shall they learn war any more.

But they shall sit every man under his vine and under his fig tree; and none shall make them afraid: for the mouth of the {\hoskeroe LORD} of hosts hath spoken it.

\subsection{}

\blfootnote{Matthew 5.39, \cite{kjv}.}{\color{red} Whosoever shall smite thee on thy right cheek, turn to him the other also.}

\section{}

\subsection{}

\blfootnote{Isaiah 40, \cite{kjv}.}Comfort ye, comfort ye my people, saith your God.

Speak ye comfortably to \textsc{Jerusalem}, and cry unto her, that her warfare is accomplished, that her iniquity is pardoned: for she hath received of the {\hoskeroe LORD}'s hand double for all her sins.

The voice of him that crieth in the wilderness, Prepare ye the way of the {\hoskeroe LORD}, make straight in the desert a highway for our God.

Every valley shall be exalted, and every mountain and hill shall be made low: and the crooked shall be made straight, and the rough places plain:

And the glory of the {\hoskeroe LORD} shall be revealed, and all flesh shall see it together: for the mouth of the {\hoskeroe LORD} hath spoken it.

The voice said, Cry. And he said, What shall I cry? All flesh is grass, and all the goodliness thereof is as the flower of the field:

The grass withereth, the flower fadeth: because the spirit of the {\hoskeroe LORD} bloweth upon it: surely the people is grass.

The grass withereth, the flower fadeth: but the word of our God shall stand for ever.

O \textsc{Zion}, that bringest good tidings, get thee up into the high mountain; O \textsc{Jerusalem}, that bringest good tidings, lift up thy voice with strength; lift it up, be not afraid; say unto the cities of Judah, Behold your God!

Behold, the Lord {\hoskeroe GOD} will come with strong hand, and his arm shall rule for him: behold, his reward is with him, and his work before him.

He shall feed his flock like a shepherd: he shall gather the lambs with his arm, and carry them in his bosom, and shall gently lead those that are with young.

Who hath measured the waters in the hollow of his hand, and meted out heaven with the span, and comprehended the dust of the earth in a measure, and weighed the mountains in scales, and the hills in a balance?

Who hath directed the Spirit of the {\hoskeroe LORD}, or being his counseller hath taught him?

With whom took he counsel, and who instructed him, and taught him in the path of judgment, and taught him knowledge, and shewed to him the way of understanding?

Behold, the nations are as a drop of a bucket, and are counted as the small dust of the balance: behold, he taketh up the isles as a very little thing.

And Lebanon is not sufficient to burn, nor the beasts thereof sufficient for a burnt offering.

All nations before him are as nothing; and they are counted to him less than nothing, and vanity.

To whom then will ye liken God? or what likeness will ye compare unto him?

The workman melteth a graven image, and the goldsmith spreadeth it over with gold, and casteth silver chains.

He that is so impoverished that he hath no oblation chooseth a tree that will not rot; he seeketh unto him a cunning workman to prepare a graven image, that shall not be moved.

Have ye not known? have ye not heard? hath it not been told you from the beginning? have ye not understood from the foundations of the earth?

It is he that sitteth upon the circle of the earth, and the inhabitants thereof are as grasshoppers; that stretcheth out the heavens as a curtain, and spreadeth them out as a tent to dwell in:

That bringeth the princes to nothing; he maketh the judges of the earth as vanity.

Yea, they shall not be planted; yea, they shall not be sown: yea, their stock shall not take root in the earth: and he shall also blow upon them, and they shall wither, and the whirlwind shall take them away as stubble.

To whom then will ye liken me, or shall I be equal? saith the Holy One.

Lift up your eyes on high, and behold who hath created these things, that bringeth out their host by number: he calleth them all by names by the greatness of his might, for that he is strong in power; not one faileth.

Why sayest thou, O Jacob, and speakest, O Israel, My way is hid from the {\hoskeroe LORD}, and my judgment is passed over from my God?

Hast thou not known? hast thou not heard, that the everlasting God, the {\hoskeroe LORD}, the Creator of the ends of the earth, fainteth not, neither is weary? there is no searching of his understanding.

He giveth power to the faint; and to them that have no might he increaseth strength.

Even the youths shall faint and be weary, and the young men shall utterly fall:

But they that wait upon the {\hoskeroe LORD} shall renew their strength; they shall mount up with wings as eagles; they shall run, and not be weary; and they shall walk, and not faint.

\subsection{}

\blfootnote{Isaiah 42.5-8, \cite{kjv}.}Thus saith God the {\hoskeroe LORD}, he that created the heavens, and stretched them out; he that spread forth the earth, and that which cometh out of it; he that giveth breath unto the people upon it, and spirit to them that walk therein:

I the {\hoskeroe LORD} have called thee in righteousness, and will hold thine hand, and will keep thee, and give thee for a covenant of the people, for a light of the gentiles;

To open the blind eyes, to bring out the prisoners from the prison, and them that sit in darkness out of the prison house.

I am the {\hoskeroe LORD}: that is my name: and my glory will I not give to another, neither my praise to graven images.

\subsection{}

\blfootnote{1 John 4.20, \cite{kjv}.}If a man say, I love God, and hateth his brother, he is a liar.

\section{}

\subsection{}

\blfootnote{Isaiah 53, \cite{kjv}.}Who hath believed our report? And to whom is the arm of the {\hoskeroe LORD} revealed?

For he shall grow up before him as a tender plant, and as a root out of a dry ground: he hath no form nor comeliness; and when we shall see him, there is no beauty that we should desire him.

He is despised and rejected of men; a man of sorrows, and acquainted with grief: and we hid as it were our faces from him; he was despised, and we esteemed him not.

Surely he hath borne our griefs, and carried our sorrows: yet we did esteem him stricken, smitten of God, and afflicted.

But he was wounded for our transgressions, he was bruised for our iniquities: the chastisement of our peace was upon him; and with his stripes we are healed.

All we like sheep have gone astray; we have turned every one to his own way; and the {\hoskeroe LORD} hath laid on him the iniquity of us all.

He was oppressed, and he was afflicted, yet he opened not his mouth: he is brought as a lamb to the slaughter, and as a sheep before her shearers is dumb, so he openeth not his mouth.

He was taken from prison and from judgment: and who shall declare his generation? for he was cut off out of the land of the living: for the transgression of my people was he stricken.

And he made his grave with the wicked, and with the rich in his death; because he had done no violence, neither was any deceit in his mouth.

Yet it pleased the {\hoskeroe LORD} to bruise him; he hath put him to grief: when thou shalt make his soul an offering for sin, he shall see his seed, he shall prolong his days, and the pleasure of the {\hoskeroe LORD} shall prosper in his hand.

He shall see of the travail of his soul, and shall be satisfied: by his knowledge shall my righteous servant justify many; for he shall bear their iniquities.

Therefore will I divide him a portion with the great, and he shall divide the spoil with the strong; because he hath poured out his soul unto death: and he was numbered with the transgressors; and he bare the sin of many, and made intercession for the transgressors.

\subsection{}

\blfootnote{Isaiah 43.1-7, \cite{kjv}.}But now thus saith the {\hoskeroe LORD} that created thee, O Jacob, and he that formed thee, O Israel, Fear not: for I have redeemed thee, I have called thee by thy name; thou art mine.

When thou passest through the waters, I will be with thee; and through the rivers, they shall not overflow thee: when thou walkest through the fire, thou shalt not be burned; neither shall the flame kindle upon thee.

For I am the {\hoskeroe LORD} thy God, the Holy One of Israel, thy Saviour: I gave Egypt for thy ransom, Ethiopia and Seba for thee.

Since thou wast precious in my sight, thou hast been honourable, and I have loved thee: therefore will I give men for thee, and people for thy life.

Fear not: for I am with thee: I will bring thy seed from the east, and gather thee from the west;

I will say to the north, Give up; and to the south, Keep not back: bring my sons from far, and my daughters from the ends of the earth;

Even every one that is called by my name: for I have created him for my glory, I have formed him; yea, I have made him.

\subsection{}

\blfootnote{Acts 10.34, \cite{kjv}.}God is no respecter of persons.

\section{}

\subsection{}

\blfootnote{Isaiah 55.6-13, \cite{kjv}.}Seek ye the {\hoskeroe LORD} while he may be found, call ye upon him while he is near:

Let the wicked forsake his way, and the unrighteous man his thoughts: and let him return unto the {\hoskeroe LORD}, and he will have mercy upon him; and to our God, for he will abundantly pardon.

For my thoughts are not your thoughts, neither are your ways my ways, saith the {\hoskeroe LORD}.

For as the heavens are higher than the earth, so are my ways higher than your ways, and my thoughts than your thoughts.

For as the rain cometh down, and the snow from heaven, and returneth not thither, but watereth the earth, and maketh it bring forth and bud, that it may give seed to the sower, and bread to the eater:

So shall my word be that goeth forth out of my mouth: it shall not return unto me void, but it shall accomplish that which I please, and it shall prosper in the thing whereto I sent it.

For ye shall go out with joy, and be led forth with peace: the mountains and the hills shall break forth before you into singing, and all the trees of the field shall clap their hands.

Instead of the thorn shall come up the fir tree, and instead of the brier shall come up the myrtle tree: and it shall be to the {\hoskeroe LORD} for a name, for an everlasting sign that shall not be cut off.

\subsection{}

\blfootnote{Isaiah 52.7-10, \cite{kjv}.}How beautiful upon the mountains are the feet of him that bringeth good tidings, that publisheth peace; that bringeth good tidings of good, that publisheth salvation; that saith unto \textsc{Zion}, Thy God reigneth!

Thy watchmen shall lift up the voice; with the voice together shall they sing: for they shall see eye to eye, when the {\hoskeroe LORD} shall bring again \textsc{Zion}.

Break forth into joy, sing together, ye waste places of \textsc{Jerusalem}: for the {\hoskeroe LORD} hath comforted his people, he hath redeemed \textsc{Jerusalem}.

The {\hoskeroe LORD} hath made bare his holy arm in the eyes of all the nations; and all the ends of the earth shall see the salvation of our God.

\subsection{}

\blfootnote{1 John 4.8, \cite{kjv}.}God is love.

\section{}

\subsection{}

\blfootnote{Isaiah 65.17-25, \cite{kjv}.}For, behold, I create new heavens and a new earth: and the former shall not be remembered, nor come into mind.

But be ye glad and rejoice for ever in that which I create: for, behold, I create \textsc{Jerusalem} a rejoicing, and her people a joy.

And I will rejoice in \textsc{Jerusalem}, and joy in my people: and the voice of weeping shall be no more heard in her, nor the voice of crying.

There shall be no more thence an infant of days, nor an old man that hath not filled his days: for the child shall die an 100 years old; but the sinner being an 100 years old shall be accursed.

And they shall build houses, and inhabit them; and they shall plant vineyards, and eat the fruit of them.

They shall not build, and another inhabit; they shall not plant, and another eat: for as the days of a tree are the days of my people, and mine elect shall long enjoy the work of their hands.

They shall not labour in vain, nor bring forth for trouble; for they are the seed of the blessed of the {\hoskeroe LORD}, and their offspring with them.

And it shall come to pass, that before they call, I will answer; and while they are yet speaking, I will hear.

The wolf and the lamb shall feed together, and the lion shall eat straw like the bullock: and dust shall be the serpent's meat. They shall not hurt nor destroy in all my holy mountain, saith the {\hoskeroe LORD}.

\subsection{}

\blfootnote{Isaiah 60.1-5, \cite{kjv}.}Arise, shine; for thy light is come, and the glory of the {\hoskeroe LORD} is risen upon thee.

For, behold, the darkness shall cover the earth, and gross darkness the people: but the {\hoskeroe LORD} shall arise upon thee, and his glory shall be seen upon thee.

And the gentiles shall come to thy light, and kings to the brightness of thy rising.

Lift up thine eyes round about, and see: all they gather themselves together, they come to thee: thy sons shall come from far, and thy daughters shall be nursed at thy side.

Then thou shalt see, and flow together, and thine heart shall fear, and be enlarged; because the abundance of the sea shall be converted unto thee, the forces of the gentiles shall come unto thee.

\subsection{}

\blfootnote{Hebrew 13.2, \cite{kjv}.}Some have entertained angels unawares.

\section{}

\subsection{}

\blfootnote{Zecheriah 12, 13.1,14.1-12, \cite{kjv}.}Behold, I will make \textsc{Jerusalem} a cup of trembling unto all the people round about, when they shall be in the siege both against Judah and against \textsc{Jerusalem}.

And in that day will I make \textsc{Jerusalem} a burdensome stone for all people: all that burden themselves with it shall be cut in pieces, though all the people of the earth be gathered together against it.

In that day, saith the {\hoskeroe LORD}, I will smite every horse with astonishment, and his rider with madness: and I will open mine eyes upon the house of Judah, and will smite every horse of the people with blindness.

And the governors of Judah shall say in their heart, The inhabitants of \textsc{Jerusalem} shall be my strength in the {\hoskeroe LORD} of hosts their God.

In that day will I make the governors of Judah like an hearth of fire among the wood, and like a torch of fire in a sheaf; and they shall devour all the people round about, on the right hand \& on the left: and \textsc{Jerusalem} shall be inhabited again in her own place, even in \textsc{Jerusalem}.

The {\hoskeroe LORD} also shall save the tents of Judah first, that the glory of the house of \textit{David} and the glory of the inhabitants of \textsc{Jerusalem} do not magnify themselves against Judah.

In that day shall the {\hoskeroe LORD} defend the inhabitants of \textsc{Jerusalem}; and he that is feeble among them at that day shall be as \textit{David}; and the house of \textit{David} shall be as God, as the angel of the {\hoskeroe LORD} before them.

And it shall come to pass in that day, that I will seek to destroy all the nations that come against \textsc{Jerusalem}.

And I will pour upon the house of \textit{David}, and upon the inhabitants of \textsc{Jerusalem}, the spirit of grace \& of supplications: and they shall look upon me whom they have pierced, and they shall mourn for him, as one mourneth for his only son, and shall be in bitterness for him, as one that is in bitterness for his firstborn.

In that day shall there be a great mourning in \textsc{Jerusalem}, as the mourning of \textsc{Hadadrimmon} in the valley of \textsc{Megiddon}.

And the land shall mourn, every family apart; the family of the house of \textit{David} apart, and their wives apart; the family of the house of \textit{Nathan} apart, and their wives apart;
The family of the house of \textit{Levi} apart, and their wives apart; the family of \textit{Shimei} apart, and their wives apart;

All the families that remain, every family apart, and their wives apart.

Behold, the day of the {\hoskeroe LORD} cometh, and thy spoil shall be divided in the midst of thee.

For I will gather all nations against \textsc{Jerusalem} to battle; and the city shall be taken, and the houses rifled, and the women ravished; and half of the city shall go forth into captivity, and the residue of the people shall not be cut off from the city.

Then shall the {\hoskeroe LORD} go forth, and fight against those nations, as when he fought in the day of battle.

And his feet shall stand in that day upon the \textsc{Mount of Olives}, which is before \textsc{Jerusalem} on the east, and the \textsc{Mount of Olives} shall cleave in the midst thereof toward the east and toward the west, and there shall be a very great valley; and half of the mountain shall remove toward the north, and half of it toward the south.

And ye shall flee to the valley of the mountains; for the valley of the mountains shall reach unto \textsc{Azal}: yea, ye shall flee, like as ye fled from before the earthquake in the days of \textit{Uzziah}, King of Judah: and the {\hoskeroe LORD} my God shall come, and all the saints with thee.

And it shall come to pass in that day, that the light shall not be clear, nor dark:

But it shall be one day which shall be known to the {\hoskeroe LORD}, not day, nor night: but it shall come to pass, that at evening time it shall be light.

And it shall be in that day, that living waters shall go out from \textsc{Jerusalem}; half of them toward the former sea, and half of them toward the hinder sea: in summer and in winter shall it be.

And the {\hoskeroe LORD} shall be king over all the earth: in that day shall there be one {\hoskeroe LORD}, and his name one.

All the land shall be turned as a plain from \textsc{Geba} to \textsc{Rimmon} south of \textsc{Jerusalem}: and it shall be lifted up, and inhabited in her place, from \textsc{Benjamin's Gate} unto the place of the first gate, unto the corner gate, and from the \textsc{Tower of Hananeel} unto the king's winepresses.

And men shall dwell in it, and there shall be no more utter destruction; but \textsc{Jerusalem} shall be safely inhabited.

And this shall be the plague wherewith the {\hoskeroe LORD} will smite all the people that have fought against \textsc{Jerusalem}; their flesh shall consume away while they stand upon their feet, and their eyes shall consume away in their holes, and their tongue shall consume away in their mouth.

\subsection{}

\blfootnote{Isaiah 61.1-4, \cite{kjv}.}The Spirit of the Lord {\hoskeroe GOD} is upon me; because the {\hoskeroe LORD} hath anointed me to preach good tidings unto the meek; he hath sent me to bind up the broken-hearted, to proclaim liberty to the captives, and the opening of the prison to them that are bound;

To proclaim the acceptable year of the {\hoskeroe LORD}, and the day of vengeance of our God; to comfort all that mourn;

To appoint unto them that mourn in \textsc{Zion}, to give unto them beauty for ashes, the oil of joy for mourning, the garment of praise for the spirit of heaviness; that they might be called trees of righteousness, the planting of the {\hoskeroe LORD}, that he might be glorified.

And they shall build the old wastes, they shall raise up the former desolations, and they shall repair the waste cities, the desolations of many generations.

\subsection{}

\blfootnote{1 Timothy 6.10, \cite{kjv}.}The love of money is the root of all evil.

\section{}

\subsection{}

\blfootnote{Malachi 3-4, \cite{kjv}.}Behold, I will send my messenger, and he shall prepare the way before me: and the Lord, whom ye seek, shall suddenly come to his temple, even the messenger of the covenant, whom ye delight in: behold, he shall come, saith the {\hoskeroe LORD} of hosts.

But who may abide the day of his coming? and who shall stand when he appeareth? for he is like a refiner's fire, and like fullers' soap:

And he shall sit as a refiner \& purifier of silver: and he shall purify the sons of \textit{Levi}, and purge them as gold \& silver, that they may offer unto the {\hoskeroe LORD} an offering in righteousness.

Then shall the offering of Judah \& \textsc{Jerusalem} be pleasant unto the {\hoskeroe LORD}, as in the days of old, and as in former years.

And I will come near to you to judgement; and I will be a swift witness against the sorcerers, and against the adulterers, and against false swearers, and against those that oppress the hireling in his wages, the widow, and the fatherless, and that turn aside the stranger from his right, and fear not me, saith the {\hoskeroe LORD} of hosts.

For I am the {\hoskeroe LORD}, I change not; therefore ye sons of Jacob are not consumed.

Even from the days of your fathers ye are gone away from mine ordinances, and have not kept them. Return unto me, and I will return unto you, saith the {\hoskeroe LORD} of hosts. But ye said, Wherein shall we return?

Will a man rob God? Yet ye have robbed me. But ye say, Wherein have we robbed thee? In tithes \& offerings.

Ye are cursed with a curse: for ye have robbed me, even this whole nation.

Bring ye all the tithes into the storehouse, that there may be meat in mine house, and prove me now herewith, saith the {\hoskeroe LORD} of hosts, if I will not open you the windows of heaven, and pour you out a blessing, that there shall not be room enough to receive it.

And I will rebuke the devourer for your sakes, and he shall not destroy the fruits of your ground; neither shall your vine cast her fruit before the time in the field, saith the {\hoskeroe LORD} of hosts.

And all nations shall call you blessed: for ye shall be a delightsome land, saith the {\hoskeroe LORD} of hosts.

Your words have been stout against me, saith the {\hoskeroe LORD}. Yet ye say, What have we spoken so much against thee?

Ye have said, It is vain to serve God: and what profit is it that we have kept his ordinance, and that we have walked mournfully before the {\hoskeroe LORD} of hosts?

And now we call the proud happy; yea, they that work wickedness are set up; yea, they that tempt God are even delivered.

Then they that feared the {\hoskeroe LORD} spake often one to another: and the {\hoskeroe LORD} hearkened, and heard it, and a book of remembrance was written before him for them that feared the {\hoskeroe LORD}, and that thought upon his name.

And they shall be mine, saith the {\hoskeroe LORD} of hosts, in that day when I make up my jewels; and I will spare them, as a man spareth his own son that serveth him.

Then shall ye return, and discern between the righteous and the wicked, between him that serveth God and him that serveth him not.

For, behold, the day cometh, that shall burn as an oven; and all the proud, yea, and all that do wickedly, shall be stubble: and the day that cometh shall burn them up, saith the {\hoskeroe LORD} of hosts, that it shall leave them neither root nor branch.

But unto you that fear my name shall the Sun of righteousness arise with healing in his wings; and ye shall go forth, and grow up as calves of the stall.

And ye shall tread down the wicked; for they shall be ashes under the soles of your feet in the day that I shall do this, saith the {\hoskeroe LORD} of hosts.

Remember ye the law of \textit{Moses} my servant, which I commanded unto him in \textsc{Horeb} for all Israel, with the statutes \& judgments.

Behold, I will send you \textit{Elijah} the prophet before the coming of the great \& dreadful day of the {\hoskeroe LORD}:

And he shall turn the heart of the fathers to the children, and the heart of the children to their fathers, lest I come and smite the earth with a curse.

\subsection{}

\blfootnote{Isaiah 66.1-2, \cite{kjv}.}Thus saith the {\hoskeroe LORD}: the heaven is my throne, and the earth is my footstool: where is the house that ye build unto me? and where is the place of my rest?

For all those things hath mine hand made, and all those things have been, saith the {\hoskeroe LORD}: but to this man will I look, even to him that is poor and of a contrite spirit, and trembleth at my word.

\subsection{}

\blfootnote{James 3.5, \cite{kjv}.}How great a matter a little fire kindleth.

\section{}

\subsection{}

\blfootnote{Revelation 21,22.1-5, \cite{kjv}.}And I saw a new heaven and a new earth: for the first heaven and the first earth were passed away; and there was no more sea.

And I \textit{John} saw the holy city, \textsc{New Jerusalem}, coming down from God out of heaven, prepared as a bride adorned for her husband.

And I heard a great voice out of heaven saying, Behold, the tabernacle of God is with men, and he will dwell with them, and they shall be his people, and God himself shall be with them, and be their God.

And God shall wipe away all tears from their eyes; and there shall be no more death, neither sorrow, nor crying, neither shall there be any more pain: for the former things are passed away.

And he that sat upon the throne said, Behold, I make all things new. And he said unto me, Write: for these words are true and faithful.

And he said unto me, It is done. I am Alpha and Omega, the beginning and the end. I will give unto him that is athirst of the fountain of the water of life freely.

He that overcometh shall inherit all things; and I will be his God, and he shall be my son.

But the fearful, and unbelieving, and the abominable, and murderers, and whoremongers, and sorcerers, and idolaters, and all liars, shall have their part in the lake which burneth with fire and brimstone: which is the second death.

And there came unto me one of the seven angels which had the seven vials full of the seven last plagues, and talked with me, saying, Come hither, I will shew thee the bride, the Lamb's wife.

And he carried me away in the spirit to a great and high mountain, and shewed me that great city, the holy \textsc{Jerusalem}, descending out of heaven from God,

Having the glory of God: and her light was like unto a stone most precious, even like a jasper stone, clear as crystal;

And had a wall great and high, and had twelve gates, and at the gates twelve angels, and names written thereon, which are the names of the twelve tribes of the children of Israel:

On the east three gates; on the north three gates; on the south three gates; and on the west three gates.

And the wall of the city had twelve foundations, and in them the names of the twelve apostles of the Lamb.

And he that talked with me had a golden reed to measure the city, and the gates thereof, and the wall thereof.

And the city lieth foursquare, and the length is as large as the breadth: and he measured the city with the reed, twelve thousand furlongs. The length and the breadth and the height of it are equal.

And he measured the wall thereof, an hundred and forty and four cubits, according to the measure of a man, that is, of the angel.

And the building of the wall of it was of jasper: and the city was pure gold, like unto clear glass.

And the foundations of the wall of the city were garnished with all manner of precious stones. The first foundation was jasper; the second, sapphire; the third, a chalcedony; the fourth, an emerald;

The fifth, sardonyx; the sixth, sardius; the seventh, chrysolite; the eighth, beryl; the ninth, a topaz; the tenth, a chrysoprasus; the eleventh, a jacinth; the twelfth, an amethyst.

And the twelve gates were twelve pearls; every several gate was of one pearl: and the street of the city was pure gold, as it were transparent glass.

And I saw no temple therein: for the Lord God Almighty and the Lamb are the temple of it.

And the city had no need of the sun, neither of the moon, to shine in it: for the glory of God did lighten it, and the Lamb is the light thereof.

And the nations of them which are saved shall walk in the light of it: and the kings of the earth do bring their glory and honour into it.

And the gates of it shall not be shut at all by day: for there shall be no night there.

And they shall bring the glory and honour of the nations into it.

And there shall in no wise enter into it any thing that defileth, neither whatsoever worketh abomination, or maketh a lie: but they which are written in the Lamb's book of life.

And he shewed me a pure river of water of life, clear as crystal, proceeding out of the throne of God and of the Lamb.

In the midst of the street of it, and on either side of the river, was there the tree of life, which bare twelve manner of fruits, and yielded her fruit every month: and the leaves of the tree were for the healing of the nations.

And there shall be no more curse: but the throne of God and of the Lamb shall be in it; and his servants shall serve him:

And they shall see his face; and his name shall be in their foreheads.

And there shall be no night there; and they need no candle, neither light of the sun; for the Lord God giveth them light: and they shall reign for ever and ever.

\subsection{}

\blfootnote{Revelation 3.14-22, \cite{kjv}.}These things saith the Amen, the faithful and true witness, the beginning of the creation of God;

I know thy works, that thou art neither cold nor hot: I would thou wert cold or hot.

So then because thou art lukewarm, and neither cold nor hot, I will spue thee out of my mouth.

Because thou sayest, I am rich, and increased with goods, and have need of nothing; and knowest not that thou art wretched, and miserable, and poor, and blind, and naked:

I counsel thee to buy of me gold tried in the fire, that thou mayest be rich; and white raiment, that thou mayest be clothed, and that the shame of thy nakedness do not appear; and anoint thine eyes with eyesalve, that thou mayest see.

As many as I love, I rebuke and chasten: be zealous therefore, and repent.

Behold, I stand at the door, and knock: if any man hear my voice, and open the door, I will come in to him, and will sup with him, and he with me.

To him that overcometh will I grant to sit with me in my throne, even as I also overcame, and am set down with my Father in his throne.

He that hath an ear, let him hear what the Spirit saith unto the churches.

\subsection{}

\blfootnote{1 Timothy 6.7, \cite{kjv}.}We brought nothing into this world, and it is certain we can carry nothing out.

\chapter{November}

\section{}

\subsection{}

\blfootnote{`The Things that Matter', Mrs Edith Bland (1858 -- 1924), \cite{oxfordlarkin}. Mrs Bland lived for another nineteen years after this poem was first published.}\settowidth{\versewidth}{What charms will cure your different pains,}
\begin{verse}[\versewidth]
Now that I've nearly done my days,\\*
\vin And grown too stiff to sweep or sew,\\
I sit \& think, till I'm amaze,\\
\vin About what lots of things I know:\\
Things as I've found out one by one --\\
\vin And when I’m fast down in the clay,\\
My knowing things and how they're done\\*
\vin Will all be lost \& thrown away.\\!

There's things, I know, as won't be lost,\\*
\vin Things as folks write \& talk about:\\
The way to keep your roots from frost,\\
\vin And how to get your ink spots out.\\
What medicine's good for sores \& sprains,\\
\vin What way to salt your butter down,\\
What charms will cure your different pains,\\*
\vin And what will bright your faded gown.\\!

But more important things than these,\\*
\vin They can't be written in a book:\\
How fast to boil your greens \& peas,\\
\vin And how good bacon ought to look;\\
The feel of real good wearing stuff,\\
\vin The kind of apple as will keep,\\
The look of bread that's rose enough,\\*
\vin And how to get a child asleep.\\!

Whether the jam is fit to pot,\\*
\vin Whether the milk is going to turn,\\
Whether a hen will lay or not,\\
\vin Is things as some folks never learn.\\
I know the weather by the sky;\\
\vin I know what herbs grow in what lane;\\
And if sick men are going to die,\\*
\vin Or if they'll get about again.\\!

Young wives come in, a-smiling, grave,\\*
\vin With secrets that they itch to tell:\\
I know what sort of times they'll have,\\
\vin And if they'll have a boy or gell.\\
And if a lad is ill to bind,\\
\vin Or some young maid is hard to lead,\\
I know when you should speak 'em kind,\\*
\vin And when it's scolding as they need.\\!

I used to know where birds'd set,\\*
\vin And likely spots for trout or hare,\\
And God may want me to forget\\
\vin The way to set a line or snare;\\
But not the way to truss a chick,\\
\vin To fry a fish, or baste a roast,\\
Nor how to tell, when folks are sick,\\*
\vin What kind of herb will ease them most.\\!

Forgetting seems such silly waste.\\*
\vin I know so many little things,\\
And now the angels will make haste\\
\vin To dust it all away with wings.\\
O God, you made me like to know,\\
\vin You kept the things straight in my head,\\
Please God, if you can make it so,\\*
\vin Let me know something when I'm dead.
\end{verse}

\subsection{}

\blfootnote{Anonymous, \cite{oxfordlocal}. `Inscribed on a well, Derry Hill, near Chippenham, Wiltshire'. The middle couplet alludes to Matthew 6.3, and the final couplet to Matthew 10.42/Mark 9.41.}\settowidth{\versewidth}{Here quench your thirst, and mark in me}
\begin{verse}[\versewidth]
Here quench your thirst, and mark in me\\*
An emblem of true charity,\\
Who while my bounty I bestow\\
Am neither seen nor heard to flow;\\
Repaid by fresh supplies from heaven\\*
For every cup of water given.
\end{verse}

\subsection{}

\blfootnote{Francis Bacon, Viscount St Alban (1561 -- 1626), \cite{odq}.}A little philosophy inclineth man's mind to atheism, but depth in philosophy bringeth men's minds about to religion.

\section{}

\subsection{}

\blfootnote{`A Palinode', Edmund Bolton (1575 -- 1633), \cite{elizabethan}. The title is delightfully obscure. Dr Johnson defines a \textit{palinode} in two words: `A recantation.' The term seems to have originated with the (very) ancient Greek poet Stesichorus, who wrote a poem called the Παλινῳδία, wherein he retracted the charge -- which he had made in an earlier poem -- that Helen was responsible for the horrors of the Trojan War.}\settowidth{\versewidth}{    As shine by fountains, bubbles, flowers, or snow?}
\begin{verse}[\versewidth]
As withereth the primose by the river,\\*
As fadeth summer's sun from gliding fountains,\\
As vanisheth the light-blown bubble ever,\\
As melteth snow upon the mossy mountains:\\
So melts, so vanisheth, so fades, so withers,\\
The rose, the shine, the bubble, and the snow,\\
Of praise, pomp, glory, joy, which short life gathers,\\
Fair praise, vain pomp, sweet glory, brittle joy.\\
The withered primrose by the mourning river,\\
The faded summer's sun from weeping fountains,\\
The light-blown bubble vanish\`{e}d for ever,\\
The molten snow upon the naked mountains,\\
\vin Are emblems that the treasures we up-lay\\*
\vin Soon wither, vanish, fade, and melt away.\\!

For as the snow, whose lawn did overspread\\*
Th' ambitious hills, which giant-like did threat\\
To pierce the heavens with their aspiring head,\\
Naked \& bare doth leave their craggy seat;\\
Whenas the bubble, which did empty fly,\\
The dalliance of the undiscern\`{e}d wind,\\
On whose calm rolling waves it did rely,\\
Hath shipwrack made, where it did dalliance find;\\
And when the sunshine which dissolved the snow,\\
Coloured the bubble with a pleasant vary,\\
And made the rathe and timely primrose grow,\\
Swarth clouds withdrawn, which longer time do tarry:\\
\vin O what is praise, pomp, glory, joy, but so\\*
\vin As shine by fountains, bubbles, flowers, or snow?
\end{verse}

\subsection{}

\blfootnote{Anonymous, \cite{oxfordlocal}. `An inscription on the pendulum of the tower clock, St Lawrence's Church, Bidborough, Kent'.}\settowidth{\versewidth}{When as a youth I dreamed and talked,}
\begin{verse}[\versewidth]
When as a child I laughed and wept,\\*
\vin \vin \vin Time crept.\\
When as a youth I dreamed and talked,\\
\vin \vin \vin Time walked.\\
When I became a full grown man\\
\vin \vin \vin Time ran.\\
And later as I older grew,\\
\vin \vin \vin Time flew.\\
Soon I shall find when travelling on\\
\vin \vin \vin Time gone.\\
Will \textit{Christ} have saved my soul by then?\\*
\vin \vin \vin Amen.
\end{verse}

\subsection{}

\blfootnote{Francis Bacon, Viscount St Alban (1561 -- 1626), \cite{odq}.}All colours will agree in the dark.

\section{}

\subsection{}

\blfootnote{`The Rolling English Road', Gilbert Chesterton, Knight (1874 -- 1936), \cite{oxfordlarkin}. Anyone with a rudimentary grasp of British geography could tell you that the routes suggested in the last line of each verse are unlikely to be the most direct.}\settowidth{\versewidth}{But the wild rose was above him when they found him in the ditch.}
\begin{verse}[\versewidth]
Before the roman came to \textsc{Rye} or out to \textsc{Severn} strode,\\*
The rolling english drunkard made the rolling english road.\\
A reeling road, a rolling road, that rambles round the shire,\\
And after him the parson ran, the sexton \& the squire;\\
A merry road, a mazy road, and such as we did tread\\*
The night we went to \textsc{Birmingham} by way of \textsc{Beachy Head}.\\!

I knew no harm of \textit{Bonaparte} and plenty of the squire,\\*
And for to fight the frenchman I did not much desire;\\
But I did bash their bayonets because they came arrayed\\
To straighten out the crooked road an english drunkard made,\\
Where you \& I went down the lane with ale-mugs in our hands,\\*
The night we went to \textsc{Glastonbury} by way of \textsc{Goodwin Sands}.\\!

His sins they were forgiven him; or why do flowers run\\*
Behind him; and the hedges all strengthening in the sun?\\
The wild thing went from left to right and knew not which was which,\\
But the wild rose was above him when they found him in the ditch.\\
God pardon us, nor harden us; we did not see so clear\\*
The night we went to \textsc{Bannockburn} by way of \textsc{Brighton Pier}.\\!

My friends, we will not go again or ape an ancient rage,\\*
Or stretch the folly of our youth to be the shame of age,\\
But walk with clearer eyes and ears this path that wandereth,\\
And see undrugged in evening light the decent inn of death;\\
For there is good news yet to hear and fine things to be seen,\\*
Before we go to paradise by way of \textsc{Kensal Green}.
\end{verse}

\subsection{}

\blfootnote{Hilaire Belloc (1870 -- 1953), \cite{oxfordlarkin}.}\settowidth{\versewidth}{The world's a stage. The trifling entrance fee}
\begin{verse}[\versewidth]
The world's a stage. The trifling entrance fee\\*
\vin Is paid (by proxy) to the registrar.\\
The orchestra is very loud \& free\\
\vin But plays no music in particular.\\
They do not print a programme, that I know.\\
\vin The cast is large. There isn't any plot.\\
The acting of the piece is far below\\
\vin The very worst of modernistic rot.\\
The only part about it I enjoy\\
\vin Is what was called in english the `foyay'.\\
There will I stand apart awhile and toy\\
\vin With thought, and set my cigarette alight;\\
And then -- without returning to the play --\\*
\vin On with my coat and out into the night.
\end{verse}

\subsection{}

\blfootnote{Benjamin Disraeli, Earl of Beaconsfield (1804 -- 1881), \cite{odq}. Many individuals have been claimed as the origin of this maxim, but, as far as the Almanacker has been been able to ascertain, Lord Beaconsfield's utterance was the earliest. Elbert Hubbard provides an amusing gloss: `Never explain -- your friends do not need it and your enemies will not believe you anyway.'}Never complain and never explain.

\section{}

\subsection{}

\blfootnote{$\mathbb{R}$ `The Secret People', Gilbert Chesterton, Knight (1874 -- 1936), \cite{oxfordlarkin}.}\settowidth{\versewidth}{The new grave lords closed round him, that had eaten the abbey's fruits,}
\begin{verse}[\versewidth]
Smile at us; pay us; pass us; but do not quite forget;\\*
For we are the people of England, that never have spoken yet.\\
There is many a fat farmer that drinks less cheerfully;\\
There is many a free french peasant who is richer \& sadder than we.\\
There are no folk in the whole world so helpless or so wise.\\
There is hunger in our bellies; there is laughter in our eyes;\\
You laugh at us and love us; both mugs \& eyes are wet:\\*
Only you do not know us. For we have not spoken yet.\\!

The fine french kings came over in a flutter of flags \& dames.\\*
We liked their smiles \& battles, but we never could say their names.\\
The blood ran red to \textsc{Bosworth} and the high french lords went down;\\
There was naught but a naked people under a naked crown.\\
And the eyes of the king's servants turned terribly every way,\\
And the gold of the king's servants rose higher every day.\\
They burned the homes of the shaven men, that had been quaint \& kind,\\
Till there was no bed in a monk's house, nor food that man could find.\\
The inns of God where no man paid, that were the wall of the weak,\\*
The king's servants ate them all. And still we did not speak.\\!

And the face of the king's servants grew greater than the king:\\*
He tricked them, and they trapped him, and stood round him in a ring.\\
The new grave lords closed round him, that had eaten the abbey's fruits,\\
And the men of the new religion, with their bibles in their boots,\\
We saw their shoulders moving, to menace or discuss,\\
And some were pure and some were vile; but none took heed of us.\\
We saw the king as they killed him, and his face was proud \& pale;\\*
And a few men talked of freedom, while England talked of ale.\\!

A war that we understood not came over the world and woke\\*
Americans, frenchmen, irish; but we knew not the things they spoke.\\
They talked about rights \& nature \& peace \& the people's reign:\\
And the squires, our masters, bade us fight; and scorned us never again.\\
Weak if we be for ever, could none condemn us then;\\
Men called us serfs \& drudges; men knew that we were men.\\
In foam \& flame at \textsc{Trafalgar}, on \textsc{Albuera} plains,\\
We did \& died like lions, to keep ourselves in chains;\\
We lay in living ruins; firing \& fearing not\\
The strange fierce face of the frenchmen who knew for what they fought,\\
And the man who seemed to be more than a man we strained against \& broke;\\*
And we broke our own rights with him. And still we never spoke.\\!

Our patch of glory ended; we never heard guns again.\\*
But the squire seemed struck in the saddle; he was foolish, as if in pain;\\
He leaned on a staggering lawyer; he clutched a cringing jew;\\
He was stricken; it may be, after all, he was stricken at \textsc{Waterloo}.\\
Or perhaps the shades of the shaven men, whose spoil is in his house,\\
Come back in shining shapes at last to spoil his last carouse:\\
We only know the last sad squires rode slowly towards the sea,\\*
And a new people takes the land: and still it is not we.\\!

They have given us into the hand of new unhappy lords,\\*
Lords without anger or honour, who dare not carry their swords.\\
They fight by shuffling papers; they have bright dead alien eyes;\\
They look at our labour \& laughter as a tired man looks at flies.\\
And the load of their loveless pity is worse than the ancient wrongs,\\*
Their doors are shut in the evening; and they know no songs.\\!

We hear men speaking for us of new laws strong \& sweet,\\*
Yet is there no man speaketh as we speak in the street.\\
It may be we shall rise the last as frenchmen rose the first,\\
Our wrath come after Russia's wrath and our wrath be the worst.\\
It may be we are meant to mark with our riot \& our rest\\
God's scorn for all men governing. It may be beer is best.\\
But we are the people of England; and we have not spoken yet.\\*
Smile at us; pay us; pass us. But do not quite forget.
\end{verse}

\subsection{}

\blfootnote{`The Little Dancers', Prof Laurence Binyon (1869 -- 1943), \cite{oxfordlarkin}.}\settowidth{\versewidth}{From a tavern--window; there, to the brisk measure}
\begin{verse}[\versewidth]
Lonely, save for a few faint stars, the sky\\*
Dreams; and lonely, below, the little street\\
Into its gloom retires, secluded and shy.\\
Scarcely the dumb roar enters this soft retreat;\\
And all is dark, save where come flooding rays\\
From a tavern--window; there, to the brisk measure\\
Of an organ that down in an alley merrily plays,\\
Two children, all alone and no one by,\\
Holding their tattered frocks, through an airy maze\\
Of motion lightly threaded with nimble feet\\
Dance sedately; face to face they gaze,\\*
Their eyes shining, grave with a perfect pleasure.
\end{verse}

\subsection{}

\blfootnote{Benjamin Disraeli, Earl of Beaconsfield (1804 -- 1881), \cite{odq}.}The palace is not safe when the cottage is not happy.

\section{}

\subsection{}

\blfootnote{$\mathbb{R}$ `The Eolian Harp', Samuel Coleridge (1772 -- 1834), \cite{norton}. An Aeolian harp is a kind of stringed instrument, designed so that it can be played entirely by the wind. Coleridge's own note indicates that this poem was written at Clevedon in Somerset. \P 1. The Sara in question is generally accepted as being the poet's (future) wife. Confusingly, he also had a crush (Sara Hutchnison) and a daughter of that name.}\settowidth{\versewidth}{Snatched from yon bean-field! And the world so hushed!}
\begin{verse}[\versewidth]
My pensive \textit{Sara}, thy soft cheek reclined\\*
Thus on mine arm, most soothing sweet it is\\
To sit beside our cot, our cot o'ergrown\\
With white-flowered jasmine, and the broad-leaved myrtle,\\
(Meet emblems they of innocence and love!)\\
And watch the clouds, that late were rich with light,\\
Slow saddening round, and mark the star of eve\\
Serenely brilliant (such would wisdom be)\\
Shine opposite! How exquisite the scents\\
Snatched from yon bean-field! And the world s\'{o} hushed!\\
The stilly murmur of the distant sea\\
Tells us of silence. And that simplest lute,\\
Placed length-ways in the clasping casement, hark!\\
How by the desultory breeze caressed,\\
Like some coy maid \sfrac{$1$}{$2$} yielding to her lover,\\
It pours such sweet upbraiding, as must needs\\
Tempt to repeat the wrong! And now, its strings\\
Boldlier swept, the long sequacious notes\\
Over delicious surges sink \& rise,\\
Such a soft floating witchery of sound\\
As twilight elfins make, when they at eve\\
Voyage on gentle gales from fairy land,\\
Where melodies round honey-dropping flowers,\\
Footless and wild, like birds of paradise,\\
Nor pause, nor perch, hovering on untamed wing!\\
O the one life within us and abroad,\\
Which meets all motion and becomes its soul,\\
A light in sound, a sound-like power in light,\\
Rhythm in all thought, and joyance everywhere --\\
Methinks, it should have been impossible\\
Not to love all things in a world so filled;\\
Where the breeze warbles, and the mute still air\\*
Is music slumbering on her instrument.\\!

And thus, my love! as on the midway slope\\*
Of yonder hill I stretch my limbs at noon,\\
Whilst through my half-closed eyelids I behold\\
The sunbeams dance, like diamonds, on the main,\\
And tranquil muse upon tranquility:\\
Full many a thought uncalled and undetained,\\
And many idle flitting phantasies,\\
Traverse my indolent \& passive brain,\\
As wild and various as the random gales\\*
That swell \& flutter on this subject lute!\\!

And what if all of animated nature\\*
Be but organic harps diversely framed,\\
That tremble into thought, as o'er them sweeps\\
Plastic and vast, one intellectual breeze,\\*
At once the soul of each, and God of all?\\!

But thy more serious eye a mild reproof\\*
Darts, O belov\`{e}d woman! nor such thoughts\\
Dim \& unhallowed dost thou not reject,\\
And biddest me walk humbly with my God.\\
Meek daughter in the family of \textit{Christ}!\\
Well hast thou said and holily dispraised\\
These shapings of the unregenerate mind;\\
Bubbles that glitter as they rise and break\\
On vain philosophy's aye-babbling spring.\\
For never guiltless may I speak of him,\\
The incomprehensible! save when with awe\\
I praise him, and with faith that inly f\'{e}els;\\
Who with his saving mercies heal\`{e}d me,\\
A sinful \& most miserable man,\\
Wildered \& dark, and gave me to possess\\*
Peace, and this cot, and thee, heart-honored maid!
\end{verse}

\subsection{}

\blfootnote{`Ah! Sun-Flower', William Blake (1757 -- 1827), \cite{blakea}.}\settowidth{\versewidth}{    And the pale virgin shrouded in snow:}
\begin{verse}[\versewidth]
Ah sunflower, weary of time,\\*
\vin Who countest the steps of the sun:\\
Seeking after that sweet golden clime\\*
\vin Where the traveller's journey is done.\\!

Where the youth pined away with desire,\\*
\vin And the pale virgin shrouded in snow:\\
Arise from their graves and aspire,\\*
\vin Where my sunflower wishes to go.
\end{verse}

\subsection{}

\blfootnote{William Blake (1757 -- 1827), \cite{blakea}. This is one of Blake's `Proverbs of Hell' from \refbook{The Marriage of Heaven and Hell}.}Eternity is in love with the productions of time.

\section{}

\subsection{}

\blfootnote{$\mathbb{R}$ Sir John Denham (1615 -- 1669), \cite{obev}. This is from Sir John's long poem, \refbook{Cooper's Hill}.}\settowidth{\versewidth}{Though deep yet clear, though gentle yet not dull,}
\begin{verse}[\versewidth]
My eye, descending from the hill, surveys\\*
Where \textsc{Thames} among the wanton valleys strays.\\
\textsc{Thames}, the most loved of all the ocean's sons,\\
By his old sire, to his embraces runs,\\
Hasting to pay his tribute to the sea,\\
Like mortal life to meet eternity;\\
Though with those streams he no resemblance hold,\\
Whose foam is amber, and their gravel gold:\\
His genuine \& less guilty wealth t'explore,\\
Search not his bottom, but survey his shore,\\
O'er which he kindly spreads his spacious wing,\\
And hatches plenty for th'ensuing spring;\\
Nor then destroys it with too fond a stay,\\
Like mothers which their infants overlay;\\
Nor with a sudden \& impetuous wave,\\
Like profuse kings, resumes the wealth he gave.\\
No unexpected inundations spoil\\
The mower's hopes, nor mock the ploughman's toil;\\
But godlike his unwearied bounty flows;\\
First loves to do, then loves the good he does.\\
Nor are his blessings to his banks confined,\\
But free \& common as the sea or wind;\\
When he, to boast or to disperse his stores,\\
Full of the tributes of his grateful shores,\\
Visits the world, and in his flying tow'rs\\
Brings home to us, and makes both Indies ours;\\
Finds wealth where 'tis, bestows it where it wants,\\
Cities in deserts, woods in cities, plants.\\
So that to us no thing, no place, is strange,\\
While his fair bosom is the world's exchange.\\
O could I flow like thee, and make thy stream\\
My great example, as it is my theme,\\
Though deep yet clear, though gentle yet not dull,\\*
Strong without rage, without o'erflowing full!
\end{verse}

\subsection{}

\blfootnote{`Cradle Song', William Blake (1757 -- 1827), \cite{blakea}. The Almanacker has omitted the last three verses, being of inferior quality than the rest.}\settowidth{\versewidth}{While o'er thee doth mother weep.}
\begin{verse}[\versewidth]
Sweet dreams form a shade\\*
O'er my lovely infant's head.\\
Sweet dreams of pleasant streams\\*
By happy, silent, moony beams.\\!

Sweet sleep, with soft down\\*
Weave thy brows an infant crown.\\
Sweet sleep, angel mild,\\*
Hover o'er my happy child.\\!

Sweet smiles, in the night\\*
Hover over my delight.\\
Sweet smiles, mother's smile,\\*
All the livelong night beguile.\\!

Sweet moans, dovelike sighs,\\*
Chase not slumber from thine eyes.\\
Sweet moan, sweeter smiles:\\*
All the dovelike moans beguile.\\!

Sleep, sleep, happy child.\\*
(All creation slept \& smiled.)\\
Sleep, sleep, happy sleep,\\*
While o'er thee doth mother weep.
\end{verse}

\subsection{}

\blfootnote{William Blake (1757 -- 1827), \cite{blakea}. This is one of Blake's `Proverbs of Hell' from \refbook{The Marriage of Heaven and Hell}.}If others had not been foolish, we should be so.

\section{}

\subsection{}

\blfootnote{`A Valediction: Forbidding Mourning', The Very Rev Dr John Donne (1572 -- 1631), \cite{obev}.}\settowidth{\versewidth}{    The breath goes now, and some say, No:}
\begin{verse}[\versewidth]
As virtuous men pass mildly away,\\*
\vin And whisper to their souls to go,\\
Whilst some of their sad friends do say\\*
\vin The breath goes now, and some say, No:\\!

So let us melt, and make no noise,\\*
\vin No tear-floods, nor sigh-tempests move;\\
'Twere profanation of our joys\\*
\vin To tell the laity our love.\\!

Moving of th'earth brings harms \& fears,\\*
\vin Men reckon what it did, and meant;\\
But trepidation of the spheres,\\*
\vin Though greater far, is innocent.\\!

Dull sublunary lovers' love\\*
\vin (Whose soul is sense) cannot admit\\
Absence, because it doth remove\\*
\vin Those things which elemented it.\\!

But we by a love so much refined,\\*
\vin That our selves know not what it is,\\
Inter-assur\`{e}d of the mind,\\*
\vin Care less, eyes, lips, \& hands to miss.\\!

Our two souls therefore, which are one,\\*
\vin Though I must go, endure not yet\\
A breach, but an expansion,\\*
\vin Like gold to airy thinness beat.\\!

If they be two, they are two so\\*
\vin As stiff twin compasses are two;\\
Thy soul, the fixed foot, makes no show\\*
\vin To move, but doth, if the other do.\\!

And though it in the center sit,\\*
\vin Yet when the other far doth roam,\\
It leans \& hearkens after it,\\*
\vin And grows erect, as that comes home.\\!

Such wilt thou be to me, who must,\\*
\vin Like th'other foot, obliquely run;\\
Thy firmness makes my circle just,\\*
\vin And makes me end where I begun.
\end{verse}

\subsection{}

\blfootnote{$\mathbb{R}$ Robert Burns (1759 -- 1796), \cite{obev}. These lines appear to be Burns's own translation of Ecclesiastes 7.1-6. He includes them as the epitaph to his \refpoem{Address to the Unco Guid}.}\settowidth{\versewidth}{    May ha'e some piles o' caff in;}
\begin{verse}[\versewidth]
My son, these maxims make a rule,\\*
\vin An' lump them ay thegither:\\
The rigid righteous is a fool,\\
\vin The rigid wise anither;\\
The cleanest corn that e'er was dight\\
\vin May ha'e some piles o' caff in;\\
So ne'er a fellow-creature slight\\*
\vin For random fits o' daffin.
\end{verse}

\subsection{}

\blfootnote{William Blake (1757 -- 1827), \cite{blakea}. This is one of Blake's `Proverbs of Hell' from \emph{The Marriage of Heaven and Hell}.}In seed-time learn; in harvest teach; in winter enjoy.

\section{}

\subsection{}

\blfootnote{$\mathbb{R}$ `The Good Morrow', The Very Rev Dr John Donne (1572 -- 1631), \cite{pbev}.}\settowidth{\versewidth}{That thou lov'st me, as thou say'st,}
\begin{verse}[\versewidth]
Sweetest love, I do not go,\\*
\vin For weariness of thee,\\
Nor in hope the world can show\\
\vin A fitter love for me;\\
\vin \vin But since that I\\
Must die at last, 'tis best\\
To use myself in jest\\*
\vin Thus by feigned deaths to die.\\!

Yesternight the sun went hence,\\*
\vin And yet is here today;\\
He hath no desire nor sense,\\
\vin Nor \sfrac{$1$}{$2$} so short a way:\\
\vin \vin Then fear not me,\\
But believe that I shall make\\
Speedier journeys, since I take\\*
\vin More wings \& spurs than he.\\!

O how feeble is man's power,\\*
\vin That if good fortune fall,\\
Cannot add another hour,\\
\vin Nor a lost hour recall.\\
\vin \vin But come bad chance,\\
And we join to'it our strength,\\
And we teach it art \& length,\\*
\vin Itself o'er us to'advance.\\!

When thou sigh'st, thou sigh'st not wind,\\*
\vin But sigh'st my soul away;\\
When thou weep'st, unkindly kind,\\
\vin My life's blood doth decay.\\
\vin \vin It cannot be\\
That thou lov'st me, as thou say'st,\\
If in thine my life thou waste,\\*
\vin That art the best of me.\\!

Let not thy divining heart\\*
\vin Forethink me any ill;\\
Destiny may take thy part,\\
\vin And may thy fears fulfil;\\
\vin \vin But think that we\\
Are but turned aside to sleep;\\
They who one another keep\\*
\vin Alive, ne'er parted be.
\end{verse}

\subsection{}

\blfootnote{John Clare (1793 -- 1864), \cite{obev}.}\settowidth{\versewidth}{The wild duck startles like a sudden thought,}
\begin{verse}[\versewidth]
The wild duck startles like a sudden thought,\\*
And heron slow as if it might be caught.\\
The flopping crows on weary wings go by\\
And grey-beard jackdaws noising as they fly.\\
The crowds of starnels whizz and hurry by,\\
And darken like a clod the evening sky.\\
The larks like thunder rise and suthy round,\\
Then drop and nestle in the stubble ground.\\
The wild swan hurries high and noises loud\\
With white neck peering to the evening cloud.\\
The weary rooks to distant woods are gone.\\
With lengths of tail the magpie winnows on\\
To neighbouring tree, and leaves the distant crow\\*
While small birds nestle in the edge below.
\end{verse}

\subsection{}

\blfootnote{`What is Man!', William Blake (1757 -- 1827), \cite{blakea}.}\settowidth{\versewidth}{The sun's light when he unfolds it}
\begin{verse}[\versewidth]
The sun's light when he unfolds it\\*
Depends on the organ that beholds it.
\end{verse}

\section{}

\subsection{}

\blfootnote{`Elegy Written in a Country Churchyard', Prof Thomas Gray (1716 -- 1771), \cite{treasury}.}\settowidth{\versewidth}{    The swallow twitt'ring from the straw-built shed,}
\begin{verse}[\versewidth]
The curfew tolls the knell of parting day,\\*
\vin The lowing herd wind slowly o'er the lea,\\
The plowman homeward plods his weary way,\\*
\vin And leaves the world to darkness and to me.\\!

Now fades the glimm'ring landscape on the sight,\\*
\vin And all the air a solemn stillness holds,\\
Save where the beetle wheels his droning flight,\\*
\vin And drowsy tinklings lull the distant folds;\\!

Save that from yonder ivy-mantled tow'r\\*
\vin The moping owl does to the moon complain\\
Of such, as wand'ring near her secret bow'r,\\*
\vin Molest her ancient solitary reign.\\!

Beneath those rugged elms, that yew-tree's shade,\\*
\vin Where heaves the turf in many a mould'ring heap,\\
Each in his narrow cell for ever laid,\\*
\vin The rude forefathers of the hamlet sleep.\\!

The breezy call of incense-breathing morn,\\*
\vin The swallow twitt'ring from the straw-built shed,\\
The cock's shrill clarion, or the echoing horn,\\*
\vin No more shall rouse them from their lowly bed.\\!

For them no more the blazing hearth shall burn,\\*
\vin Or busy housewife ply her evening care:\\
No children run to lisp their sire's return,\\*
\vin Or climb his knees the envied kiss to share.\\!

Oft did the harvest to their sickle yield,\\*
\vin Their furrow oft the stubborn glebe has broke;\\
How jocund did they drive their team afield!\\*
\vin How bowed the woods beneath their sturdy stroke!\\!

Let not ambition mock their useful toil,\\*
\vin Their homely joys, and destiny obscure;\\
Nor grandeur hear with a disdainful smile\\*
\vin The short \& simple annals of the poor.\\!

The boast of heraldry, the pomp of pow'r,\\*
\vin And all that beauty, all that wealth e'er gave,\\
Awaits alike th'inevitable hour.\\*
\vin The paths of glory lead but to the grave.\\!

Nor you, ye proud, impute to these the fault,\\*
\vin If mem'ry o'er their tomb no trophies raise,\\
Where thro' the long-drawn aisle \& fretted vault\\*
\vin The pealing anthem swells the note of praise.\\!

Can storied urn or animated bust\\*
\vin Back to its mansion call the fleeting breath?\\
Can honour's voice provoke the silent dust,\\*
\vin Or flatt'ry soothe the dull cold ear of death?\\!

Perhaps in this neglected spot is laid\\*
\vin Some heart once pregnant with celestial fire;\\
Hands, that the rod of empire might have swayed,\\*
\vin Or waked to ecstasy the living lyre.\\!

But knowledge to their eyes her ample page\\*
\vin Rich with the spoils of time did ne'er unroll;\\
Chill penury repressed their noble rage,\\*
\vin And froze the genial current of the soul.\\!

Full many a gem of purest ray serene,\\*
\vin The dark unfathomed caves of ocean bear:\\
Full many a flow'r is born to blush unseen,\\*
\vin And waste its sweetness on the desert air.\\!

Some village-\textit{Hampden}, that with dauntless breast\\*
\vin The little tyrant of his fields withstood;\\
Some mute inglorious \textit{Milton} here may rest,\\*
\vin Some \textit{Cromwell} guiltless of his country's blood.\\!

Th'applause of list'ning senates to command,\\*
\vin The threats of pain \& ruin to despise,\\
To scatter plenty o'er a smiling land,\\*
\vin And read their hist'ry in a nation's eyes,\\!

Their lot forbade: nor circumscribed alone\\*
\vin Their growing virtues, but their crimes confined;\\
Forbade to wade through slaughter to a throne,\\*
\vin And shut the gates of mercy on mankind,\\!

The struggling pangs of conscious truth to hide,\\*
\vin To quench the blushes of ingenuous shame,\\
Or heap the shrine of luxury \& pride\\*
\vin With incense kindled at the muse's flame.\\!

Far from the madding crowd's ignoble strife,\\*
\vin Their sober wishes never learned to stray;\\
Along the cool sequestered vale of life\\*
\vin They kept the noiseless tenor of their way.\\!

Yet ev'n these bones from insult to protect,\\*
\vin Some frail memorial still erected nigh,\\
With uncouth rhymes \& shapeless sculpture decked,\\*
\vin Implores the passing tribute of a sigh.\\!

Their name, their years, spelt by th'unlettered muse,\\*
\vin The place of fame and elegy supply:\\
And many a holy text around she strews,\\*
\vin That teach the rustic moralist to die.\\!

For who to dumb forgetfulness a prey,\\*
\vin This pleasing anxious being e'er resigned,\\
Left the warm precincts of the cheerful day,\\*
\vin Nor cast one longing, ling'ring look behind?\\!

On some fond breast the parting soul relies,\\*
\vin Some pious drops the closing eye requires;\\
Ev'n from the tomb the voice of nature cries,\\*
\vin Ev'n in our ashes live their wonted fires.\\!

For thee, who mindful of th'unhonoured dead\\*
\vin Dost in these lines their artless tale relate;\\
If chance, by lonely contemplation led,\\*
\vin Some kindred spirit shall inquire thy fate,\\!

Haply some hoary-headed swain may say,\\*
\vin `Oft have we seen him at the peep of dawn\\
Brushing with hasty steps the dews away\\*
\vin To meet the sun upon the upland lawn.\\!

`There at the foot of yonder nodding beech\\*
\vin That wreathes its old fantastic roots so high,\\
His listless length at noontide would he stretch,\\*
\vin And pore upon the brook that babbles by.\\!

`Hard by yon wood, now smiling as in scorn,\\*
\vin Mutt'ring his wayward fancies he would rove,\\
Now drooping, woeful wan, like one forlorn,\\*
\vin Or crazed with care, or crossed in hopeless love.\\!

`One morn I missed him on the customed hill,\\*
\vin Along the heath \& near his fav'rite tree;\\
Another came; nor yet beside the rill,\\*
\vin Nor up the lawn, nor at the wood was he;\\!

`The next with dirges due in sad array\\*
\vin Slow thro' the church-way path we saw him borne.\\
Approach \& read (for thou canst read) the lay,\\*
\vin Graved on the stone beneath yon aged thorn.'\\!

{\itshape
\flagverse{\footnotesize The Epitaph} Here rests his head upon the lap of earth,\\
\vin A youth to fortune \& to fame unknown.\\
Fair science frowned not on his humble birth,\\*
\vin And melancholy marked him for her own.}\\!

{\itshape
Large was his bounty, and his soul sincere,\\
\vin Heav'n did a recompense as largely send:\\
He gave to mis'ry all he had, a tear,\\*
\vin He gained from heav'n ('twas all he wished) a friend.}\\!

{\itshape
No farther seek his merits to disclose,\\
\vin Or draw his frailties from their dread abode,\\
(There they alike in trembling hope repose)\\*
\vin The bosom of his Father \& his God.}
\end{verse}

\subsection{}

\blfootnote{The Rev George Crabbe (1754 -- 1832), \cite{obev}.}\settowidth{\versewidth}{So thin, so pale, is yet of gold:}
\begin{verse}[\versewidth]
The ring so worn, as you behold,\\*
So thin, so pale, is yet of gold:\\
The passion such it was to prove;\\*
Worn with life's care, love yet was love.
\end{verse}

\subsection{}

\blfootnote{George Noel, 6th Baron Byron (1788 -- 1824), \cite{odq}. These words are taken from the fourteenth canto of \refbook{Don Juan}.}Truth is always strange; stranger than fiction.

\section{}

\subsection{}

\blfootnote{$\mathbb{R}$ William Habington (1605 -- 1654), \cite{obev}. The title is a quotation from the Vulgate, Psalm 19.2 (or 18.2, using the Vulgate numbering of the Psalms), which means, `Night to night shows knowledge.'}\settowidth{\versewidth}{So rich with jewels hung, that night}
\begin{verse}[\versewidth]
\vin When I survey the bright\\*
\vin \vin Celestial sphere;\\
So rich with jewels hung, that night\\*
Doth like an ethiop bride appear:\\!

\vin My soul her wings doth spread\\*
\vin \vin And heavenward flies,\\
Th'Almighty's mysteries to read\\*
In the large volumes of the skies.\\!

\vin For the bright firmament\\*
\vin \vin Shoots forth no flame\\
So silent, but is eloquent\\*
In speaking the Creator's name.\\!

\vin No unregarded star\\*
\vin \vin Contracts its light\\
Into so small a character,\\*
Removed far from our human sight,\\!

\vin But if we steadfast look\\*
\vin \vin We shall discern\\
In it, as in some holy book,\\*
How man may heavenly knowledge learn.\\!

\vin It tells the conqueror\\*
\vin \vin That far-stretched power,\\
Which his proud dangers traffic for,\\*
Is but the triumph of an hour:\\!

\vin That from the farthest north,\\*
\vin \vin Some nation may,\\
Yet undiscovered, issue forth,\\*
And o'er his new-got conquest sway:\\!

\vin Some nation yet shut in\\*
\vin \vin With hills of ice\\
May be let out to scourge his sin,\\*
Till they shall equal him in vice.\\!

\vin And then they likewise shall\\*
\vin \vin Their ruin have;\\
For as yourselves your empires fall,\\*
And every kingdom hath a grave.\\!

\vin Thus those celestial fires,\\*
\vin \vin Though seeming mute,\\
The fallacy of our desires\\*
And all the pride of life confute:\\!

\vin For they have watched since first\\*
\vin \vin The world had birth:\\
And found sin in itself accursed,\\*
And nothing permanent on earth.
\end{verse}

\subsection{}

\blfootnote{`The Villain', William Davies (1871 -- 1940), \cite{oxfordlarkin}.}\settowidth{\versewidth}{While joy gave clouds the light of stars,}
\begin{verse}[\versewidth]
While joy gave clouds the light of stars,\\*
\vin That beamed wher'er they looked;\\
And calves \& lambs had tottering knees,\\
\vin Excited, while they sucked;\\
While every bird enjoyed his song,\\
Without one thought of harm or wrong --\\
I turned my head and saw the wind,\\
\vin Not far from where I stood,\\
Dragging the corn by her golden hair,\\*
\vin Into a dark \& lonely wood.
\end{verse}

\subsection{}

\blfootnote{Charles, by the Grace of God, King of England, Scotland, France and Ireland, Defender of the Faith (1600 -- 1649), \cite{odq}.}I go from a corruptible to an incorruptible crown.

\section{}

\subsection{}

\blfootnote{`Cancer's a Funny Thing', Prof John Haldane (1892 -- 1964), \cite{oxfordlarkin}. \P 37. The poem here includes a footnote, also in verse, discussing the deities of India which also have more than one face.}\settowidth{\versewidth}{And, when sufficient had been pressed in,}
\begin{verse}[\versewidth]
I wish I had the voice of \textit{Homer}\\*
To sing of rectal carcinoma,\\
Which kills a lot more chaps, in fact,\\*
Than were bumped off when \textsc{Troy} was sacked.\\!

I noticed I was passing blood\\*
(Only a few drops, not a flood).\\
So pausing on my homeward way\\
From \textsc{Tallahassee} to \textsc{Bombay}\\
I asked a doctor, now my friend,\\
To peer into my hinder end,\\
To prove or to disprove the rumour\\
That I had a malignant tumour.\\
They pumped in BaS0\textsubscript{4}\\
Till I could really stand no more,\\
And, when sufficient had been pressed in,\\
They photographed my large intestine.\\
In order to decide the issue\\
They next scraped out some bits of tissue.\\
(Before they did so, some good pal\\
Had knocked me out with pentothal,\\
Whose action is extremely quick,\\
And does not leave me feeling sick.)\\
The microscope returned the answer\\
That I had certainly got cancer,\\
So I was wheeled into the theatre\\
Where holes were made to make me better.\\
One set is in my perineum\\
Where I can feel, but can't yet see 'em.\\
Another made me like a kipper\\
Or female prey of \textit{Jack the Ripper}.\\
Through this incision, I don't doubt,\\
The neoplasm was taken out,\\
Along with colon, \& lymph nodes\\
Where cancer cells might find abodes.\\
A third much smaller hole is meant\\
To function as a ventral vent:\\
So now I am like two-faced \textit{Janus}\\
The only god who sees his anus.\\
I'll swear, without the risk of perjury,\\
It was a snappy bit of surgery.\\
My rectum is a serious loss to me,\\
But I've a very neat colostomy,\\
And hope, as soon as I am able,\\*
To make it keep a fixed time-table.\\!

So do not wait for aches \& pains\\*
To have a surgeon mend your drains;\\
If he says cancer, you're a dunce\\
Unless you have it out at once,\\
For if you wait it's sure to swell,\\
And may have progeny as well.\\
My final word, before I'm done,\\
Is: cancer can be rather fun.\\
Thanks to the nurses \& \textit{Nye Bevan}\\
The NHS is quite like heaven\\
Provided one confronts the tumour\\
With a sufficient sense of humour.\\
I know that cancer often kills,\\
But so do cars \& sleeping pills;\\
And it can hurt one till one sweats,\\
So can bad teeth \& unpaid debts.\\
A spot of laughter, I am sure,\\
Often accelerates one's cure;\\
So let us patients do our bit\\*
To help the surgeons make us fit.
\end{verse}

\subsection{}

\blfootnote{`Song', The Rev Canon Dr Richard Dixon (1833 -- 1900), \cite{londonbook}.}\settowidth{\versewidth}{    Above the swelling stream;}
\begin{verse}[\versewidth]
The feathers of the willow\\*
Are \sfrac{$1$}{$2$} of them grown yellow\\
\vin Above the swelling stream;\\
And ragged are the bushes,\\
And rusty now the rushes,\\*
\vin And wild the clouded gleam.\\!

The thistle now is older;\\*
His stalk begins to moulder;\\
\vin His head is white as snow;\\
The branches all are barer;\\
The linnet's song is rarer;\\*
\vin The robin pipeth now.
\end{verse}

\subsection{}

\blfootnote{Philip Stanhope, 4th Earl of Chesterfield (1694 -- 1773), \cite{odq}.}Speak of the moderns without contempt, and of the ancients without idolatry.

\section{}

\subsection{}

\blfootnote{`After the Visit', Thomas Hardy (1840 -- 1928), \cite{pbev}.}\settowidth{\versewidth}{And I marked not the charm in the changes of day}
\begin{verse}[\versewidth]
\vin \vin Come again to the place\\*
Where your presence was as a leaf that skims\\
Down a drouthy way whose ascent bedims\\*
\vin The bloom on the farer's face.\\!

\vin \vin Come again, with the feet\\*
That were light on the green as a thistledown ball,\\
And those mute ministrations to one \& to all\\*
\vin Beyond a man's saying sweet.\\!

\vin \vin Until then the faint scent\\*
Of the bordering flowers swam unheeded away,\\
And I marked not the charm in the changes of day\\*
\vin As the cloud colours came and went.\\!

\vin \vin Through the dark corridors\\*
Your walk was so soundless I did not know\\
Your form from a phantom's of long ago\\*
\vin Said to pass on the ancient floors,\\!

\vin \vin Till you drew from the shade\\*
And I saw the large luminous living eyes\\
Regard me in fixed inquiring-wise\\*
\vin As those of a soul that weighed,\\!

\vin \vin Scarce consciously,\\*
The eternal question of what life was,\\
And why we were there, and by whose strange laws\\*
\vin That which mattered most could not be.
\end{verse}

\subsection{}

\blfootnote{Holy Sonnet 10, The Very Rev Dr John Donne (1572 -- 1631), \cite{obev}.}\settowidth{\versewidth}{    Much pleasure; then from thee much more must flow,}
\begin{verse}[\versewidth]
Death, be not proud, though some have called thee\\*
\vin Mighty \& dreadful, for thou art not so;\\
\vin For those whom thou think'st thou dost overthrow\\
Die not, poor death, nor yet canst thou kill me.\\
From rest \& sleep, which but thy pictures be,\\
\vin Much pleasure; then from thee much more must flow,\\
\vin And soonest our best men with thee do go,\\
Rest of their bones, \& soul's delivery.\\
Thou art slave to fate, chance, kings, \& desperate men,\\
\vin And dost with poison, war, \& sickness dwell,\\
\vin And poppy or charms can make us sleep as well\\
And better than thy stroke; why swell'st thou then?\\
\vin \vin One short sleep past, we wake eternally\\*
\vin \vin And death shall be no more; death, thou shalt die.
\end{verse}

\subsection{}

\blfootnote{Gilbert Chesterton, Knight (1874 -- 1936), \cite{odq}.}Bigotry may be roughly defined as the anger of men who have no opinions.

\section{}

\subsection{}

\blfootnote{`Ruth', Thomas Hood (1799 -- 1845), \cite{newlove}.}\settowidth{\versewidth}{She stood breast-high amid the corn,}
\begin{verse}[\versewidth]
She stood breast-high amid the corn,\\*
Clasped by the golden light of morn,\\
Like the sweetheart of the sun,\\*
Who many a glowing kiss had won.\\!

On her cheek an autumn flush,\\*
Deeply ripened; such a blush\\
In the midst of brown was born,\\*
Like red poppies grown with corn.\\!

Round her eyes her tresses fell;\\*
Which were blackest none could tell,\\
But long lashes veiled a light,\\*
That had else been all too bright.\\!

And her hat, with shady brim,\\*
Made her tressy forehead dim;\\
Thus she stood amid the stooks,\\*
Praising God with sweetest looks:\\!

Sure, I said, heaven did not mean,\\*
Where I reap thou shouldst but glean;\\
Lay thy sheaf adown and come,\\*
Share my harvest \& my home.
\end{verse}

\subsection{}

\blfootnote{`Melancholy', William Habington (1605 -- 1654), \cite{londonbook}.}\settowidth{\versewidth}{We still have land in ken; and 'cause our boat}
\begin{verse}[\versewidth]
Were but that sigh a penitential breath\\*
That thou art mine: it would blow with it death,\\
T'inclose me in my marble: where I'd be\\
Slave to the tyrant worms to set thee free.\\
What should we envy? Though with larger sail\\
Some dance upon the ocean; yet more frail\\
And faithlese is that wave, than where we glide,\\
Blest in the safety of a private tide.\\
We still have land in ken; and 'cause our boat\\
Dares not affront the weather, we'll ne'er float\\
Far from the shore. To daring them each cloud\\
Is big with thunder; every wind speaks loud;\\
And rough wild rocks about the shore appeare;\\*
Yet virtue will find room to anchor there.
\end{verse}

\subsection{}

\blfootnote{Gilbert Chesterton, Knight (1874 -- 1936), \cite{odq}.}Democrats object to men being disqualified by the accident of birth; tradition objects to their being disqualified by the accident of death.

\section{}

\subsection{}

\blfootnote{Prof Alfred Housman (1859 -- 1936), \cite{ptmgmc}.}\settowidth{\versewidth}{Tell me not here; it needs not saying,}
\begin{verse}[\versewidth]
Tell me not here; it needs not saying,\\*
\vin What tune the enchantress plays\\
In aftermaths of soft september\\
\vin Or under blanching mays,\\
For she \& I were long acquainted\\*
\vin And I knew all her ways.\\!

On russet floors, by waters idle,\\*
\vin The pine lets fall its cone;\\
The cuckoo shouts all day at nothing\\
\vin In leafy dells alone;\\
And traveller's joy beguiles in autumn\\*
\vin Hearts that have lost their own.\\!

On acres of the seeded grasses\\*
\vin The changing burnish heaves;\\
Or marshalled under moons of harvest\\
\vin Stand still all night the sheaves;\\
Or beeches strip in storms for winter\\*
\vin And stain the wind with leaves.\\!

Possess, as I possessed a season,\\*
\vin The countries I resign,\\
Where over elmy plains the highway\\
\vin Would mount the hills and shine,\\
And full of shade the pillared forest\\*
\vin Would murmur and be mine.\\!

For nature, heartless, witless nature,\\*
\vin Will neither care nor know\\
What stranger's feet may find the meadow\\
\vin And trespass there and go,\\
Nor ask amid the dews of morning\\*
\vin If they are mine or no.
\end{verse}

\subsection{}

\blfootnote{`In Tenebris', Thomas Hardy (1840 -- 1928), \cite{pbev}.}\settowidth{\versewidth}{But, since it once hath been,}
\begin{verse}[\versewidth]
\vin Wintertime nighs;\\*
But my bereavement pain\\
It cannot bring again:\\*
\vin Twice no one dies.\\!

\vin Flower petals flee;\\*
But, since it once hath been,\\
No more that severing scene\\*
\vin Can harrow me.\\!

\vin Birds faint in dread:\\*
I shall not lose old strength\\
In the lone frost's black length:\\*
\vin Strength long since fled.\\!

\vin Leaves freeze to dun;\\*
But friends cannot turn cold\\
This season as of old\\*
\vin For him with none.\\!

\vin Tempests may scath;\\*
But love cannot make smart\\
Again this year his heart\\*
\vin Who no heart hath.\\!

\vin Black is night's cope;\\*
But death will not appal\\
One who, past doubtings all,\\*
\vin Waits in unhope.
\end{verse}

\subsection{}

\blfootnote{Gilbert Chesterton, Knight (1874 -- 1936), \cite{odq}.}To be clever enough to get all that money, one has to be stupid enough to want it.

\section{}

\subsection{}

\blfootnote{Prof Alfred Housman (1859 -- 1936), \cite{oxfordlarkin}. This is \refbook{A Shropshire Lad} \S XXIII.}\settowidth{\versewidth}{And then one could talk with them friendly and wish them farewell}
\begin{verse}[\versewidth]
The lads in their hundreds to \textsc{Ludlow} come in for the fair;\\*
\vin There's men from the barn \& the forge \& the mill \& the fold;\\
The lads for the girls \& the lads for the liquor are there,\\*
\vin And there with the rest are the lads that will never be old.\\!

There's chaps from the town \& the field \& the till \& the cart,\\*
\vin And many to count are the stalwart, and many the brave,\\
And many the handsome of face \& the handsome of heart,\\*
\vin And few that will carry their looks or their truth to the grave.\\!

I wish one could know them; I wish there were tokens to tell\\*
\vin The fortunate fellows that now you can never discern;\\
And then one could talk with them friendly and wish them farewell\\*
\vin And watch them depart on the way that they will not return.\\!

But now you may stare as you like and there's nothing to scan;\\*
\vin And brushing your elbow unguessed-at and not to be told\\
They carry back bright to the coiner the mintage of man,\\*
\vin The lads that will die in their glory \& never be old.
\end{verse}

\subsection{}

\blfootnote{`The Self-Unseeing', Thomas Hardy (1840 -- 1928), \cite{pbev}.}\settowidth{\versewidth}{Childlike, I danced in a dream;}
\begin{verse}[\versewidth]
Here is the ancient floor,\\*
\vin Footworn \& hollowed \& thin,\\
Here was the former door\\*
\vin Where the dead feet walked in.\\!

She sat here in her chair,\\*
\vin Smiling into the fire;\\
He who played stood there,\\*
\vin Bowing it higher \& higher.\\!

Childlike, I danced in a dream;\\*
\vin Blessings emblazoned that day;\\
Everything glowed with a gleam;\\*
\vin Yet we were looking away.
\end{verse}

\subsection{}

\blfootnote{Gilbert Chesterton, Knight (1874 -- 1936), \cite{odq}.}We make our friends; we make our enemies; but God makes our next-door neighbour.

\section{}

\subsection{}

\blfootnote{John Keats (1795 -- 1821), \cite{obev}. These are the opening lines of Keats's \refbook{Hyperion: A Fragment}.}\settowidth{\versewidth}{Robs not one light seed from the feathered grass,}
\begin{verse}[\versewidth]
Deep in the shady sadness of a vale\\*
Far sunken from the healthy breath of morn,\\
Far from the fiery noon, and eve's one star,\\
Sat grey-haired \textit{Saturn}, quiet as a stone,\\
Still as the silence round about his lair;\\
Forest on forest hung about his head\\
Like cloud on cloud. No stir of air was there,\\
Not so much life as on a summer's day\\
Robs not one light seed from the feathered grass,\\
But where the dead leaf fell, there did it rest.\\
A stream went voiceless by, still deadened more\\
By reason of his fallen divinity\\
Spreading a shade: the naiad 'mid her reeds\\*
Pressed her cold finger closer to her lips.\\!

Along the margin and large foot-marks went,\\*
No further than to where his feet had strayed,\\
And slept there since. Upon the sodden ground\\
His old right hand lay nerveless, listless, dead,\\
Unsceptred; and his realmless eyes were closed;\\
While his bowed head seemed listening to the earth,\\*
His ancient mother, for some comfort yet.
\end{verse}

\subsection{}

\blfootnote{$\mathbb{R}$ `To his ever-loving God', Robert Herrick (1591 -- 1674), \cite{obev}.}\settowidth{\versewidth}{That slack my pace, but yet not make me stay?}
\begin{verse}[\versewidth]
Can I not come to thee, my God, for these\\*
So very-many-meeting hindrances,\\
That slack my pace, but yet not make me stay?\\
Who slowly goes, rids, in the end, his way.\\
Clear thou my paths, or shorten thou my miles;\\
Remove the bars, or lift me o'er the stiles;\\
Since rough the way is, help me when I call,\\
And take me up; or else prevent the fall.\\
I ken my home, and it affords some ease\\
To see far off the smoking villages.\\
Fain would I rest, yet covet not to die\\
For fear of future biting penury:\\
No, no, my God -- thou know'st my wishes be\\*
To leave this life not loving it, but thee.
\end{verse}

\subsection{}

\blfootnote{Irvin Cobb (1876 -- 1944), \cite{odq}.}A good storyteller is a person who has a good memory and hopes other people haven't.

\section{}

\subsection{}

\blfootnote{`The Way through the Woods', Rudyard Kipling (1865 -- 1936), \cite{oxfordlarkin}. The Almanacker was forced to compose a pastiche of this poem as a small child.}\settowidth{\versewidth}{You will hear the beat of a horse's feet,}
\begin{verse}[\versewidth]
They shut the road through the woods\\*
Seventy years ago.\\
Weather \& rain have undone it again,\\
And now you would never know\\
There was once a road through the woods\\
Before they planted the trees.\\
It is underneath the coppice \& heath,\\
And the thin anemones.\\
Only the keeper sees\\
That, where the ring-dove broods,\\
And the badgers roll at ease,\\*
There was once a road through the woods.\\!

Yet, if you enter the woods\\*
Of a summer evening late,\\
When the night-air cools on the trout-ringed pools\\
Where the otter whistles his mate\\
(They fear not men in the woods,\\
Because they see so few),\\
You will hear the beat of a horse's feet,\\
And the swish of a skirt in the dew,\\
Steadily cantering through\\
The misty solitudes,\\
As though they perfectly knew\\
The old lost road through the woods...\\*
But there is no road through the woods.
\end{verse}

\subsection{}

\blfootnote{`Heaven-Haven', Fr Gerard Hopkins (1844 -- 1889), \cite{londonbook}. Fr Hopkins's own subtitle reads: `A nun takes the veil'.}\settowidth{\versewidth}{To fields where flies no sharp sided hail}
\begin{verse}[\versewidth]
\vin I have desired to go\\*
\vin \vin Where springs not fail,\\
To fields where flies no sharp \& sided hail\\*
\vin And a few lilies blow.\\!

\vin And I have asked to be\\*
\vin \vin Where no storms come,\\
Where the green swell is in the havens dumb,\\*
\vin And out of the swing of the sea.
\end{verse}

\subsection{}

\blfootnote{The Rev Charles Colton (1777 -- 1832), \cite{odq}.}The greatest fool may ask more than the wisest man can answer.

\section{}

\subsection{}

\blfootnote{`Lycidas', John Milton (1608 -- 1674), \cite{treasury}. In his \refbook{Poems} of 1645, Milton gave this preface: `In this monody the author bewails a learned friend [Edward King], unfortunately drowned in his passage from Chester on the Irish seas, 1637. And by occasion foretells the ruin of our corrupted clergy, then in their height.'}\settowidth{\versewidth}{Blind mouths! that scarce themselves know how to hold}
\begin{verse}[\versewidth]
Yet once more, O ye laurels, and once more\\*
Ye myrtles brown, with ivy never sere,\\
I come to pluck your berries harsh \& crude,\\
And with forced fingers rude\\
Shatter your leaves before the mellowing year.\\
Bitter constraint \& sad occasion dear\\
Compels me to disturb your season due;\\
For \textit{Lycidas} is dead, dead ere his prime,\\
Young \textit{Lycidas}, and hath not left his peer.\\
Who would not sing for \textit{Lycidas}? he knew\\
Himself to sing, and build the lofty rhyme.\\
He must not float upon his watery bier\\
Unwept, and welter to the parching wind,\\*
Without the meed of some melodious tear.\\!

Begin then, sisters of the sacred well\\*
That from beneath the seat of \textit{Jove} doth spring;\\
Begin, and somewhat loudly sweep the string.\\
Hence with denial vain \& coy excuse!\\
So may some gentle muse\\
With lucky words favour my destined urn,\\
And as he passes turn\\*
And bid fair peace be to my sable shroud.\\!

For we were nursed upon the selfsame hill,\\*
Fed the same flock, by fountain, shade, \& rill;\\
Together both, ere the high lawns appeared\\
Under the opening eyelids of the morn,\\
We drove afield, and both together heard\\
What time the gray-fly winds her sultry horn,\\
Battening our flocks with the fresh dews of night,\\
Oft till the star that rose at evening bright\\
Toward heav'n's descent had sloped his westering wheel.\\
Meanwhile the rural ditties were not mute,\\
Tempered to th'oaten flute;\\
Rough satyrs danced, \& fauns with cloven heel,\\
From the glad sound would not be absent long;\\*
And old \textit{Damoetas} loved to hear our song.\\!

But O the heavy change now thou art gone,\\*
Now thou art gone, and never must return.\\
Thee, shepherd, thee the woods \& desert caves,\\
With wild thyme \& the gadding vine o'ergrown,\\
And all their echoes mourn.\\
The willows \& the hazel copses green\\
Shall now no more be seen\\
Fanning their joyous leaves to thy soft lays.\\
As killing as the canker to the rose,\\
Or taint-worm to the weanling herds that graze,\\
Or frost to flowers that their gay wardrobe wear\\
When first the white thorn blows:\\*
Such, \textit{Lycidas}, thy loss to shepherd's ear.\\!

Where were ye, nymphs, when the remorseless deep\\*
Closed o'er the head of your loved \textit{Lycidas}?\\
For neither were ye playing on the steep\\
Where your old bards, the famous druids, lie,\\
Nor on the shaggy top of \textsc{Mona} high,\\
Nor yet where \textsc{Deva} spreads her wizard stream.\\
Aye me! I fondly dream\\
Had ye been there -- for what could that have done?\\
What could the muse herself that \textit{Orpheus} bore,\\
The muse herself, for her enchanting son,\\
Whom universal nature did lament,\\
When by the rout that made the hideous roar\\
His gory visage down the stream was sent,\\*
Down the swift \textsc{Hebrus} to the lesbian shore?\\!

Alas! what boots it with incessant care\\*
To tend the homely, slighted shepherd's trade,\\
And strictly meditate the thankless muse?\\
Were it not better done, as others use,\\
To sport with \textit{Amaryllis} in the shade,\\
Or with the tangles of \textit{Neaera}'s hair?\\
Fame is the spur that the clear spirit doth raise\\
(That last infirmity of noble mind)\\
To scorn delights \& live laborious days;\\
But the fair guerdon when we hope to find,\\
And think to burst out into sudden blaze,\\
Comes the blind fury with th'abhorred shears,\\
And slits the thin-spun life. `But not the praise,'\\
\textit{Phoebus} replied, and touched my trembling ears;\\
`Fame is no plant that grows on mortal soil,\\
Nor in the glistering foil\\
Set off to th'world, nor in broad rumour lies,\\
But lives \& spreads aloft by those pure eyes\\
And perfect witness of all-judging \textit{Jove};\\
As he pronounces lastly on each deed,\\*
Of so much fame in heav'n expect thy meed.'\\!

O fountain \textsc{Arethuse}, and thou honoured flood,\\*
Smooth-sliding \textsc{Mincius}, crowned with vocal reeds,\\
That strain I heard was of a higher mood.\\
But now my oat proceeds,\\
And listens to the herald of the sea,\\
That came in \textit{Neptune}'s plea.\\
He asked the waves, and asked the felon winds,\\
`What hard mishap hath doomed this gentle swain?'\\
And questioned every gust of rugged wings\\
That blows from off each beaked promontory.\\
They knew not of his story;\\
And sage \textit{Hippotades} their answer brings,\\
That not a blast was from his dungeon strayed;\\
The air was calm, and on the level brine\\
Sleek \textit{Panope} with all her sisters played.\\
It was that fatal \& perfidious bark,\\
Built in th'eclipse, and rigged with curses dark,\\*
That sunk so low that sacred head of thine.\\!

Next \textit{Camus}, reverend sire, went footing slow,\\*
His mantle hairy, \& his bonnet sedge,\\
Inwrought with figures dim, and on the edge\\
Like to that sanguine flower inscribed with woe.\\
`Ah! who hath reft,' quoth he, `my dearest pledge?'\\
Last came, and last did go,\\
The pilot of the galilean lake;\\
Two massy keys he bore of metals twain\\
(The golden opes, the iron shuts amain).\\
He shook his mitred locks, and stern bespake:\\
`How well could I have spared for thee, young swain,\\
Enow of such as for their bellies' sake\\
Creep \& intrude, and climb into the fold?\\
Of other care they little reckoning make\\
Than how to scramble at the shearers' feast\\
And shove away the worthy bidden guest.\\
Blind mouths! that scarce themselves know how to hold\\
A sheep-hook, or have learned aught else the least\\
That to the faithful herdman's art belongs!\\
What recks it them? What need they? They are sped;\\
And when they list their lean \& flashy songs\\
Grate on their scrannel pipes of wretched straw,\\
The hungry sheep look up, and are not fed,\\
But, swoll'n with wind \& the rank mist they draw,\\
Rot inwardly, and foul contagion spread;\\
Besides what the grim wolf with privy paw\\
Daily devours apace, and nothing said,\\
But that two-handed engine at the door\\*
Stands ready to smite once, and smite no more.'\\!

Return, \textit{Alpheus}: the dread voice is past\\*
That shrunk thy streams; return, sicilian muse,\\
And call the vales \& bid them hither cast\\
Their bells \& flowerets of 1000 hues.\\
Ye valleys low, where the mild whispers use\\
Of shades \& wanton winds, \& gushing brooks,\\
On whose fresh lap the swart star sparely looks,\\
Throw hither all your quaint enamelled eyes,\\
That on the green turf suck the honeyed showers\\
And purple all the ground with vernal flowers.\\
Bring the rathe primrose that forsaken dies,\\
The tufted crow-toe, \& pale jessamine,\\
The white pink, \& the pansy freaked with jet,\\
The glowing violet,\\
The musk-rose, \& the well attired woodbine,\\
With cowslips wan that hang the pensive head,\\
And every flower that sad embroidery wears;\\
Bid amaranthus all his beauty shed,\\
And daffadillies fill their cups with tears,\\
To strew the laureate hearse where \textit{Lycid} lies.\\
For so to interpose a little ease,\\
Let our frail thoughts dally with false surmise.\\
Aye me! Whilst thee the shores \& sounding seas\\
Wash far away, where'er thy bones are hurled;\\
Whether beyond the stormy Hebrides,\\
Where thou perhaps under the whelming tide\\
Visit'st the bottom of the monstrous world,\\
Or whether thou, to our moist vows denied,\\
Sleep'st by the fable of \textit{Bellerus} old,\\
Where the great vision of the guarded mount\\
Looks toward \textsc{Namancos} \& \textsc{Bayona}'s hold:\\
Look homeward angel now, and melt with ruth;\\*
And, O ye dolphins, waft the hapless youth.\\!

Weep no more, woeful shepherds, weep no more,\\*
For \textit{Lycidas}, your sorrow, is not dead,\\
Sunk though he be beneath the watery floor;\\
So sinks the day-star in the ocean bed,\\
And yet anon repairs his drooping head,\\
And tricks his beams, and with new-spangled ore\\
Flames in the forehead of the morning sky:\\
So \textit{Lycidas} sunk low, but mounted high\\
Through the dear might of him that walked the waves;\\
Where, other groves \& other streams along,\\
With nectar pure his oozy locks he laves,\\
And hears the unexpressive nuptial song,\\
In the blest kingdoms meek of joy \& love.\\
There entertain him all the saints above,\\
In solemn troops, \& sweet societies,\\
That sing, and singing in their glory move,\\
And wipe the tears for ever from his eyes.\\
Now, \textit{Lycidas}, the shepherds weep no more:\\
Henceforth thou art the genius of the shore,\\
In thy large recompense, and shalt be good\\*
To all that wander in that perilous flood.\\!

Thus sang the uncouth swain to th'oaks \& rills,\\*
While the still morn went out with sandals gray;\\
He touched the tender stops of various quills,\\
With eager thought warbling his doric lay;\\
And now the sun had stretched out all the hills,\\
And now was dropped into the western bay;\\
At last he rose, and twitched his mantle blue:\\*
Tomorrow to fresh woods, \& pastures new.
\end{verse}

\subsection{}

\blfootnote{Prof Alfred Housman (1859 -- 1936), \cite{londonbook}.}\settowidth{\versewidth}{        And wears the turning globe.}
\begin{verse}[\versewidth]
The night is freezing fast;\\*
\vin Tomorrow comes december;\\
\vin \vin And winterfalls of old\\
Are with me from the past;\\
\vin And chiefly I remember\\*
\vin \vin How \textit{Dick} would hate the cold.\\!

Fall, winter, fall; for he,\\*
\vin Prompt hand \& headpiece clever,\\
\vin \vin Has woven a winter robe,\\
And made of earth \& sea\\
\vin His overcoat for ever,\\*
\vin \vin And wears the turning globe.
\end{verse}

\subsection{}

\blfootnote{William Congreve (1670 -- 1729), \cite{londonbook}. This couplet is taken from a verse epistle to Richard Temple, 1st Viscount Cobham.}\settowidth{\versewidth}{For virtue now is neither more or less,}
\begin{verse}[\versewidth]
For virtue now is neither more or less,\\*
And vice is only varied in the dress.
\end{verse}

\section{}

\subsection{}

\blfootnote{`Any Soul to Any Body', Cosmo Monkhouse (1840 -- 1901), \cite{obev}.}\settowidth{\versewidth}{    Who clove to me so close, whate'er the weather,}
\begin{verse}[\versewidth]
So we must part, my body, you \& I\\*
\vin Who've spent so many pleasant years together.\\
'Tis sorry work to lose your company\\
\vin Who clove to me so close, whate'er the weather,\\
From winter unto winter, wet or dry;\\
\vin But you have reached the limit of your tether,\\
And I must journey on my way alone,\\*
And leave you quietly beneath a stone.\\!

They say that you are altogether bad\\*
\vin (Forgive me; 'tis not my experience),\\
And think me very wicked to be sad\\
\vin At leaving you, a clod, a prison, whence\\
To get quite free I should be very glad.\\
\vin Perhaps I may be so, some few days hence,\\
But now, methinks, 'twere graceless not to spend\\*
A tear or two on my departing friend.\\!

Now our long partnership is near completed,\\*
\vin And I look back upon its history;\\
I greatly fear I have not always treated\\
\vin You with the honesty you showed to me.\\
And I must own that you have oft defeated\\
\vin Unworthy schemes by your sincerity,\\
And by a blush or stammering tongue have tried\\*
To make me think again before I lied.\\!

'Tis true you're not so handsome as you were,\\*
\vin But that's not your fault and is partly mine.\\
You might have lasted longer with more care,\\
\vin And still looked something like your first design;\\
And even now, with all your wear \& tear,\\
\vin 'Tis pitiful to think I must resign\\
You to the friendless grave, the patient prey\\*
Of all the hungry legions of decay.\\!

But you must stay, dear body, and I go.\\*
\vin And I was once so very proud of you:\\
You made my mother's eyes to overflow\\
\vin When first she saw you, wonderful \& new.\\
And now, with all your faults, 'twere hard to find\\
\vin A slave more willing or a friend more true.\\
Ay -- even they who say the worst about you\\*
Can scarcely tell what I shall do without you.
\end{verse}

\subsection{}

\blfootnote{$\mathbb{R}$ Henry King, Bishop of Chichester (1592 -- 1669), \cite{obev}.}\settowidth{\versewidth}{From whence no triumph ever came,}
\begin{verse}[\versewidth]
Tell me no more how fair she is.\\*
\vin I have no mind to hear\\
The story of that distant bliss\\
\vin I never shall come near:\\
By sad experience I have found\\*
That her perfection is my wound.\\!

And tell me not how fond I am\\*
\vin To tempt a daring fate,\\
From whence no triumph ever came,\\
\vin But to repent too late:\\
There is some hope ere long I may\\*
In silence dote my self away.\\!

I ask no pity, love, from thee,\\*
\vin Nor will thy justice blame,\\
So that thou wilt not envy me\\
\vin The glory of my flame:\\
Which crowns my heart when ere it dies,\\*
In that it falls her sacrifice.
\end{verse}

\subsection{}

\blfootnote{William Congreve (1670 -- 1729), \cite{odq}.}I am always of the opinion with the learned, if they speak first.

\section{}

\subsection{}

\blfootnote{George Peele (1556 -- 1596), \cite{londonbook}. This poem is frequently anthologised under the title \refpoem{A Farewell to Arms}, which is the ultimate source of the title of Hemingway's novel. These lines conclude \refbook{Polyhymnia}.}\settowidth{\versewidth}{    But spurned in vain; youth waneth by increasing:}
\begin{verse}[\versewidth]
His golden locks time hath to silver turned;\\*
\vin O time too swift, O swiftness never ceasing!\\
His youth 'gainst time \& age hath ever spurned,\\
\vin But spurned in vain; youth waneth by increasing:\\
Beauty, strength, youth, are flowers but fading seen;\\*
Duty, faith, love are roots, and ever green.\\!

His helmet now shall make a hive for bees;\\*
\vin And lovers' sonnets turned to holy psalms;\\
A man-at-arms must now serve on his knees,\\
\vin And feed on prayers, which are age his alms:\\
But though from court to cottage he depart,\\*
His saint is sure of his unspotted heart.\\!

And when he saddest sits in homely cell,\\*
\vin He'll teach his swains this carol for a song:\\
Blest be the hearts that wish my sovereign well;\\
\vin Cursed be the souls that think her any wrong.\\
Goddess, allow this ag{\`{e}}d man his right\\*
To be your beadsman now that was your knight.
\end{verse}

\subsection{}

\blfootnote{`Autumn Ploughing', Dr John Masefield, Poet Laureate (1878 -- 1967), \cite{obev}.}\settowidth{\versewidth}{    The last spilled seed corn left upon the ground;}
\begin{verse}[\versewidth]
After the ranks of stubble have laid bare,\\*
\vin And field mice \& finches' beaks have found\\
\vin The last spilled seed corn left upon the ground;\\*
And no more swallows miracle in air;\\!

When the green tuft no longer hides the hare,\\*
\vin And dropping starling flights at evening come;\\
\vin When birds, except the robin, have gone dumb,\\*
And leaves are rustling downwards everywhere;\\!

Then out, with the great horses, come the ploughs,\\*
\vin And all day long the slow procession goes,\\
\vin \vin Darkening the stubble fields with broadening strips.\\
Grey sea-gulls settle after to carouse:\\
\vin Harvest prepares upon the harvest's close,\\*
\vin \vin Before the blackbird pecks the scarlet hips.
\end{verse}

\subsection{}

\blfootnote{William Cowper (1731 -- 1800), \cite{pbev}. This is a line from Book I of Cowper's \refbook{The Task}.}God made the country, and man made the town.

\section{}

\subsection{}

\blfootnote{`Epistle to Miss Blount, On Her Leaving the Town, After the Coronation', Alexander Pope (1688 -- 1744), \cite{newlove}. The Miss Blount in question must have been one of the two Blount sisters, Teresa and Martha, with whom Pope was friendly. Martha Blount is more likely, since, as Robert Carruthers argued in his \refbook{Life} of 1857, Pope and she were particularly close, perhaps even lovers, and indeed Pope made her his principal heir. \P 24. The word `whisk' as used here is an archaic name for the card-game whist.}\settowidth{\versewidth}{Or with his hound comes hollowing from the stable,}
\begin{verse}[\versewidth]
As some fond virgin, whom her mother's care\\*
Drags from the town to wholesome country air,\\
Just when she learns to roll a melting eye,\\
And hear a spark, yet think no danger nigh;\\
From the dear man unwillingly she must sever,\\
Yet takes one kiss before she parts for ever:\\
Thus from the world fair \textit{Zephalinda} flew,\\
Saw others happy, and with sighs withdrew;\\
Not that their pleasures caused her discontent:\\
She sighed not that thzzy stayed, but that shzz went.\\
She went, to plain-work, and to purling brooks,\\
Old-fashioned halls, dull aunts, \& croaking rooks;\\
She went from opera, park, assembly, play,\\
To morning walks, \& prayers three hours a day;\\
To pass her time 'twixt reading \& bohea,\\
To muse, and spill her solitary tea,\\
Or o'er cold coffee trifle with the spoon,\\
Count the slow clock, and dine exact at noon;\\
Divert her eyes with pictures in the fire,\\
Hum \sfrac{$1$}{$2$} a tune, tell stories to the squire;\\
Up to her godly garret after seven;\\
There starve \& pray, for that's the way to heaven.\\
Some squire, perhaps, you take a delight to rack;\\
Whose game is whisk, whose treat a toast in sack,\\
Who visits with a gun, presents you birds,\\
Then gives a smacking buss, \& cries, No words!\\
Or with his hound comes hollowing from the stable,\\
Makes love with nods, \& knees beneath a table;\\
Whose laughs are hearty, though his jests are coarse,\\
And loves you best of all things -- but his horse.\\
In some fair evening, on your elbow laid,\\
Your dream of triumphs in the rural shade;\\
In pensive thought recall the fancied scene,\\
See coronations rise on every green;\\
Before you pass th'imaginary sights\\
Of lords \& earls \& dukes \& gartered knights;\\
While the spread fan o'ershades your closing eyes;\\
Then give one flirt, and all the vision flies.\\
Thus vanish scepters, coronets \& balls,\\
And leave you in lone woods, or empty walls.\\
So when your slave, at some dear, idle time,\\
(Not plagued with headaches, or the want of rhyme)\\
Stands in the streets, abstracted from the crew,\\
And while he seems to study, thinks of you:\\
Just when his fancy points your sprightly eyes,\\
Or sees the blush of soft \textit{Parthenia} rise,\\
Gay pats my shoulder, and you vanish quite;\\
Streets, chairs, and coxcombs rush upon my sight;\\
Vexed to be still in town, I knit my brow,\\*
Look sour and hum a tune -- as you may now.
\end{verse}

\subsection{}

\blfootnote{`The Emigrant', Dr John Masefield, Poet Laureate (1878 -- 1967), \cite{masefield}.}\settowidth{\versewidth}{Old sea-boots stamping, shuffling, it brought the bitter tears;}
\begin{verse}[\versewidth]
Going by \textit{Daly}'s shanty I heard the boys within\\*
Dancing the spanish hornpipe to \textit{Driscoll}'s violin;\\
I heard the sea-boots shaking the rough planks of the floor,\\*
But I was going westward, I hadn't heart for more.\\!

All down the windy village the noise rang in my ears;\\*
Old sea-boots stamping, shuffling, it brought the bitter tears;\\
The old tune piped \& quavered; the lilts came clear and strong;\\*
But I was going westward, I couldn't join the song.\\!

There were the grey stone houses, the night wind blowing keen,\\*
The hill-sides pale with moonlight, the young corn springing green,\\
The hearth nooks lit \& kindly, with dear friends good to see,\\*
But I was going westward, and the ship waited me.
\end{verse}

\subsection{}

\blfootnote{Daniel Defoe (1660 -- 1731), \cite{odq}.}The good of subjects is the end of kings.

\section{}

\subsection{}

\blfootnote{`Eloisa to Abelard', Alexander Pope (1688 -- 1744), \cite{londonbook}. This poem plays a significant part in the plot of \refbook{Sharpe's Enemy}.}\settowidth{\versewidth}{Ev'n thought meets thought, ere from the lips it part,}
\begin{verse}[\versewidth]
In these deep solitudes \& awful cells,\\*
Where heav'nly-pensive contemplation dwells,\\
And ever-musing melancholy reigns;\\
What means this tumult in a vestal's veins?\\
Why rove my thoughts beyond this last retreat?\\
Why feels my heart its long-forgotten heat?\\
Yet, yet I love! From \textit{Abelard} it came,\\*
And \textit{Eloisa} yet must kiss the name.\\!

Dear fatal name, rest ever unrevealed,\\*
Nor pass these lips in holy silence sealed.\\
Hide it, my heart, within that close disguise,\\
Where mixed with God's, his loved idea lies:\\
O write it not, my hand -- the name appears\\
Already written -- wash it out, my tears!\\
In vain lost \textit{Eloisa} weeps and prays,\\*
Her heart still d\'{i}ctates, and her hand obeys.\\!

Relentless walls, whose darksome round contains\\*
Repentant sighs, \& voluntary pains:\\
Ye rugged rocks, which holy knees have worn;\\
Ye grots \& caverns shagged with horrid thorn!\\
Shrines, where their vigils pale-eyed virgins keep,\\
And pitying saints, whose statues learn to weep!\\
Though cold like you, unmoved, and silent grown,\\
I have not yet forgot myself to stone.\\
All is not heav'n's while \textit{Abelard} has part,\\
Still rebel nature holds out \sfrac{$1$}{$2$} my heart;\\
Nor pray'rs nor fasts its stubborn pulse restrain,\\*
Nor tears, for ages, taught to flow in vain.\\!

Soon as thy letters trembling I unclose,\\*
That well-known name awakens all my woes.\\
O name for ever sad, for ever dear,\\
Still breathed in sighs, still ushered with a tear.\\
I tremble too, where'er my own I find,\\
Some dire misfortune follows close behind.\\
Line after line my gushing eyes o'erflow,\\
Led through a sad variety of woe:\\
Now warm in love, now with'ring in thy bloom,\\
Lost in a convent's solitary gloom!\\
There stern religion quenched th'unwilling flame;\\*
There died the best of passions, love \& fame.\\!

Yet write, O write me all, that I may join\\*
Griefs to thy griefs, and echo sighs to thine.\\
Nor foes nor fortune take this pow'r away;\\
And is my \textit{Abelard} less kind than they?\\
Tears still are mine, and those I need not spare,\\
Love but demands what else were shed in prayer;\\
No happier task these faded eyes pursue;\\*
To read and weep is all they now can do.\\!

Then share thy pain, allow that sad relief;\\*
Ah, more than share it; give me all thy grief.\\
Heav'n first taught letters for some wretch's aid,\\
Some banished lover, or some captive maid;\\
They live; they speak; they breathe what love inspires,\\
Warm from the soul, and faithful to its fires;\\
The virgin's wish without her fears impart;\\
Excuse the blush, and pour out all the heart;\\
Speed the soft intercourse from soul to soul,\\*
And waft a sigh from \textsc{Indus} to the \textsc{Pole}.\\!

Thou know'st how guiltless first I met thy flame,\\*
When love approached me under friendship's name;\\
My fancy formed thee of angelic kind,\\
Some emanation of th'all-beauteous mind.\\
Those smiling eyes, attemp'ring every day,\\
Shone sweetly lambent with celestial day.\\
Guiltless I gazed; heav'n listened while you sung;\\
And truths divine came mended from that tongue.\\
From lips like those what precept failed to move?\\
Too soon they taught me 'twas no sin to love.\\
Back through the paths of pleasing sense I ran,\\
Nor wished an angel whom I loved a man.\\
Dim \& remote the joys of saints I see;\\*
Nor envy them, that heav'n I lose for thee.\\!

How oft, when pressed to marriage, have I said,\\*
`Curse on all laws but those which love has made!'\\
Love, free as air, at sight of human ties,\\
Spreads his light wings, and in a moment flies;\\
Let wealth, let honour, wait the wedded dame,\\
\'{A}ugust her deed, and sacred be her fame;\\
Before true passion all those views remove.\\
Fame, wealth \& honour, what are you to love?\\
The jealous God, when we profane his fires,\\
Those restless passions in revenge inspires;\\
And bids them make mistaken mortals groan,\\
Who seek in love for aught but love alone.\\
Should at my feet the world's great master fall,\\
Himself, his throne, his world, I'd scorn 'em all:\\
Not Caesar's empress would I deign to prove;\\
No, make me mistress to the man I love;\\
If there be yet another name more free,\\
More fond than mistress, make me that to thee!\\
Oh happy state, when souls each other draw,\\
When love is liberty, and nature, law:\\
All then is full, possessing, \& possessed,\\
No craving void left aching in the breast:\\
Ev'n thought meets thought, ere from the lips it part,\\
And each warm wish springs mutual from the heart.\\
This sure is bliss (if bliss on earth there be)\\*
And once the lot of \textit{Abelard} \& me.\\!

Alas, how changed! What sudden horrors rise!\\*
A naked lover bound \& bleeding lies!\\
Where, where was Eloise? Her voice, her hand,\\
Her poniard, had opposed the dire command.\\
Barbarian, stay! That bloody stroke restrain;\\
The crime was common; common be the pain.\\
I can no more; by shame, by rage suppressed;\\*
Let tears, \& burning blushes speak the rest.\\!

Canst thou forget that sad, that solemn day,\\*
When victims at yon altar's foot we lay?\\
Canst thou forget what tears that moment fell,\\
When, warm in youth, I bade the world farewell?\\
As with cold lips I kissed the sacred veil,\\
The shrines all trembled, and the lamps grew pale:\\
Heav'n scarce believed the conquest it surveyed,\\
And saints with wonder heard the vows I made.\\
Yet then, to those dread altars as I drew,\\
Not on the cross my eyes were fixed, but you:\\
Not grace, or zeal, love only was my call,\\
And if I lose thy love, I lose my all.\\
Come, with thy looks, thy words, relieve my woe;\\
Those still at least are left thee to bestow.\\
Still on that breast enamoured let me lie;\\
Still drink delicious poison from thy eye,\\
Pant on thy lip, and to thy heart be pressed;\\
Give all thou canst -- and let me dream the rest.\\
Ah no! Instruct me other joys to prize;\\
With other beauties charm my partial eyes;\\
Full in my view set all the bright abode,\\*
And make my soul quit \textit{Abelard} for God.\\!

Ah, think at least thy flock deserves thy care,\\*
Plants of thy hand, \& children of thy prayer.\\
From the false world in early youth they fled,\\
By thee to mountains, wilds, \& deserts led.\\
You raised these hallowed walls; the desert smiled,\\
And paradise was opened in the wild.\\
No weeping orphan saw his father's stores\\
Our shrines irradiate, or emblaze the floors;\\
No silver saints, by dying misers giv'n,\\
Here bribed the rage of ill-requited heav'n:\\
But such plain roofs as piety could raise,\\
And only vocal with the maker's praise.\\
In these lone walls (their days eternal bound)\\
These moss-grown domes with spiry turrets crowned,\\
Where awful arches make a noonday night,\\
And the dim windows shed a solemn light;\\
Thy eyes diffused a reconciling ray,\\
And gleams of glory brightened all the day.\\
But now no face divine contentment wears,\\
'Tis all blank sadness, or continual tears.\\
See how the force of others' prayers I try,\\
(O pious fraud of am'rous charity!)\\
But why should I on others' prayers depend?\\
Come thou, my father, brother, husband, friend!\\
Ah let thy handmaid, sister, daughter move,\\
And all those tender names in one, thy love!\\
The darksome pines that o'er yon rocks reclined\\
Wave high, and murmur to the hollow wind,\\
The wand'ring streams that shine between the hills,\\
The grots that echo to the tinkling rills,\\
The dying gales that pant upon the trees,\\
The lakes that quiver to the curling breeze;\\
No more these scenes my meditation aid,\\
Or lull to rest the visionary maid.\\
But o'er the twilight groves \& dusky caves,\\
Long-sounding aisles, \& intermingled graves,\\
Black melancholy sits, and round her throws\\
A death-like silence, \& a dread repose:\\
Her gloomy presence saddens all the scene,\\
Shades ev'ry flower, and darkens ev'ry green,\\
Deepens the murmur of the falling floods,\\*
And breathes a browner horror on the woods.\\!

Yet here for ever, ever must I stay;\\*
Sad proof how well a lover can obey!\\
Death, only death, can break the lasting chain;\\
And here, ev'n then, shall my cold dust remain,\\
Here all its frailties, all its flames resign,\\*
And wait till 'tis no sin to mix with thine.\\!

Ah wretch, believed the spouse of God in vain,\\*
Confessed within the slave of love \& man.\\
Assist me, heav'n! But whence arose that prayer?\\
Sprung it from piety, or from despair?\\
Ev'n here, where frozen chastity retires,\\
Love finds an altar for forbidden fires.\\
I ought to grieve, but cannot what I ought;\\
I mourn the lover, not lament the fault;\\
I view my crime, but kindle at the view,\\
Repent old pleasures, and solicit new;\\
Now turned to heav'n, I weep my past offence,\\
Now think of thee, and curse my innocence.\\
Of all affliction taught a lover yet,\\
'Tis sure the hardest science to forget!\\
How shall I lose the sin, yet keep the sense,\\
And love th'offender, yet detest th'offence?\\
How the dear object from the crime remove,\\
Or how distinguish penitence from love?\\
Unequal task, a passion to resign,\\
For hearts so touched, so pierced, so lost as mine.\\
Ere such a soul regains its peaceful state,\\
How often must it love, how often hate!\\
How often hope, despair, resent, regret,\\
Conceal, disdain -- do all things but forget.\\
But let heav'n seize it, all at once 'tis fired;\\
Not touched, but rapt; not wakened, but inspired!\\
O come, O teach me nature to subdue,\\
Renounce my love, my life, myself -- \& you.\\
Fill my fond heart with God alone, for he\\*
Alone can rival, can succeed to thee.\\!

How happy is the blameless vestal's lot!\\*
The world forgetting, by the world forgot.\\
Eternal sunshine of the spotless mind!\\
Each prayer accepted, and each wish resigned;\\
Labour \& rest, that equal periods keep;\\
`Obedient slumbers that can wake and weep;'\\
Desires composed, affections ever ev'n,\\
Tears that delight, and sighs that waft to heav'n.\\
Grace shines around her with serenest beams,\\
And whisp'ring angels prompt her golden dreams.\\
For her th'unfading rose of \textsc{Eden} blooms,\\
And wings of seraphs shed divine perfumes,\\
For her the spouse prepares the bridal ring;\\
For her white virgins hymeneals sing;\\
To sounds of heav'nly harps she dies away,\\*
And melts in visions of eternal day.\\!

Far other dreams my erring soul employ,\\*
Far other raptures, of unholy joy:\\
When at the close of each sad, sorrowing day,\\
Fancy restores what vengeance snatched away,\\
Then conscience sleeps, and leaving nature free,\\
All my loose soul unbounded springs to thee.\\
O cursed, dear horrors of all-conscious night!\\
How glowing guilt exalts the keen delight!\\
Provoking demons all restraint remove,\\
And stir within me ev'ry source of love.\\
I hear thee, view thee, gaze o'er all thy charms,\\
And round thy phantom glue my clasping arms.\\
I wake -- no more I hear; no more I view;\\
The phantom flies me, as unkind as you.\\
I call aloud; it hears not what I say;\\
I stretch my empty arms; it glides away.\\
To dream once more I close my willing eyes;\\
Ye soft illusions, dear deceits, arise!\\
Alas, no more -- methinks we wand'ring go\\
Through dreary wastes, and weep each other's woe,\\
Where round some mould'ring tower pale ivy creeps,\\
And low-browed rocks hang nodding o'er the deeps.\\
Sudden you mount; you beckon from the skies;\\
Clouds interpose, waves roar and winds arise.\\
I shriek, start up, the same sad prospect find,\\*
And wake to all the griefs I left behind.\\!

For thee the fates, severely kind, ordain\\*
A cool suspense from pleasure \& from pain;\\
Thy life a long, dead calm of fixed repose;\\
No pulse that riots, and no blood that glows.\\
Still as the sea, ere winds were taught to blow,\\
Or moving spirit bade the waters flow;\\
Soft as the slumbers of a saint forgiv'n,\\*
And mild as opening gleams of promised heav'n.\\!

Come, \textit{Abelard}, for what hast thou to dread?\\*
The torch of \textit{Venus} burns not for the dead.\\
Nature stands checked; religion disapproves;\\
Ev'n thou art cold -- yet \textit{Eloisa} loves.\\
Ah hopeless, lasting flames, like those that burn\\*
To light the dead, and warm th'unfruitful urn!\\!

What scenes appear where'er I turn my view?\\*
The dear ideas, where I fly, pursue,\\
Rise in the grove, before the altar rise,\\
Stain all my soul, and wanton in my eyes.\\
I waste the matin lamp in sighs for thee,\\
Thy image steals between my God \& me,\\
Thy voice I seem in ev'ry hymn to hear,\\
With ev'ry bead I drop too soft a tear.\\
When from the censer clouds of fragrance roll,\\
And swelling organs lift the rising soul,\\
One thought of thee puts all the pomp to flight,\\
Priests, tapers, temples, swim before my sight:\\
In seas of flame my plunging soul is drowned,\\*
While altars blaze, and angels tremble round.\\!

While prostrate here in humble grief I lie,\\*
Kind, virtuous drops just gath'ring in my eye,\\
While praying, trembling, in the dust I roll,\\
And dawning grace is op'ning on my soul:\\
Come, if thou dar'st, all charming as thou art!\\
Oppose thyself to heav'n; dispute my heart;\\
Come, with one glance of those deluding eyes\\
Blot out each bright idea of the skies;\\
Take back that grace, those sorrows, \& those tears;\\
Take back my fruitless penitence \& prayers;\\
Snatch me, just mounting, from the blest abode;\\*
Assist the fiends, and tear me from my God!\\!

No, fly me, fly me, far as pole from pole;\\*
Rise alps between us, and whole oceans roll!\\
Ah, come not, write not, think not once of me,\\
Nor share one pang of all I felt for thee.\\
Thy oaths I quit, thy memory resign;\\
Forget, renounce me, hate whate'er was mine.\\
Fair eyes, \& tempting looks (which yet I view!)\\
Long loved, adored ideas, all adieu!\\
O grace serene! O virtue heav'nly fair!\\
Divine oblivion of low-thoughted care!\\
Fresh blooming hope, gay daughter of the sky!\\
And faith, our early immortality!\\
Enter, each mild, each amicable guest;\\*
Receive, and wrap me in eternal rest!\\!

See in her cell sad \textit{Eloisa} spread,\\*
Propped on some tomb, a neighbour of the dead.\\
In each low wind methinks a spirit calls,\\
And more than echoes talk along the walls.\\
Here, as I watched the dying lamps around,\\
From yonder shrine I heard a hollow sound.\\
`Come, sister, come!' it said, or seem'd to say.\\
`Thy place is here, sad sister; come away!\\
Once like thyself, I trembled, wept, and prayed,\\
Love's victim then, though now a sainted maid:\\
But all is calm in this eternal sleep;\\
Here grief forgets to groan, and love to weep,\\
Ev'n superstition loses ev'ry fear:\\*
For God, not man, absolves our frailties here.'\\!

I come; I come! Prepare your roseate bow'rs,\\*
Celestial palms, and ever-blooming flow'rs.\\
Thither, where sinners may have rest, I go,\\
Where flames refined in breasts seraphic glow:\\
Thou, \textit{Abelard}, the last sad office pay,\\
And smooth my passage to the realms of day;\\
See my lips tremble, and my eyeballs roll;\\
Suck my last breath, and catch my flying soul!\\
Ah no -- in sacred vestments may'st thou stand,\\
The hallowed taper trembling in thy hand;\\
Present the cross before my lifted eye;\\
Teach me at once, and learn of me to die.\\
Ah then, thy once-loved \textit{Eloisa} see!\\
It will be then no crime to gaze on me.\\
See from my cheek the transient roses fly!\\
See the last sparkle languish in my eye!\\
Till ev'ry motion, pulse, \& breath be o'er;\\
And ev'n my \textit{Abelard} be loved no more.\\
O death all-eloquent, you only prove\\*
What dust we dote on, when 'tis man we love.\\!

Then too, when fate shall thy fair frame destroy,\\*
(That cause of all my guilt, \& all my joy)\\
In trance ecstatic may thy pangs be drowned,\\
Bright clouds descend, and angels watch thee round;\\
From op'ning skies may streaming glories shine,\\*
And saints embrace thee with a love like mine.\\!

May one kind grave unite each hapless name,\\*
And graft my love immortal on thy fame!\\
Then, ages hence, when all my woes are o'er,\\
When this rebellious heart shall beat no more;\\
If ever chance two wand'ring lovers brings\\
To Paraclete's white walls \& silver springs,\\
O'er the pale marble shall they join their heads,\\
And drink the falling tears each other sheds;\\
Then sadly say, with mutual pity moved,\\*
`Oh may we never love as these have loved!'\\!

From the full choir when loud hosannas rise,\\*
And swell the pomp of dreadful sacrifice,\\
Amid that scene if some relenting eye\\
Glance on the stone where our cold relics lie,\\
Devotion's self shall steal a thought from heav'n;\\
One human tear shall drop and be forgiv'n.\\
And sure, if fate some future bard shall join\\
In sad similitude of griefs to mine,\\
Condemned whole years in absence to deplore,\\
And image charms he must behold no more;\\
Such if there be, who loves so long, so well;\\
Let him our sad, our tender story tell;\\
The well-sung woes will soothe my pensive ghost;\\*
He best can paint 'em, who shall feel 'em most.
\end{verse}

\subsection{}

\blfootnote{`An Epilogue', Dr John Masefield, Poet Laureate (1878 -- 1967), \cite{oxfordlarkin}.}\settowidth{\versewidth}{And kind things done by men with ugly faces,}
\begin{verse}[\versewidth]
I have seen flowers come in stony places\\*
And kind things done by men with ugly faces,\\
And the gold cup won by the worst horse at the races,\\*
So I trust, too.
\end{verse}

\subsection{}

\blfootnote{George Douglas (1868 -- 1952), \cite{odq}. The quotation continues: `To keep him, two.'}To find a friend one must close one eye.

\section{}

\subsection{}

\blfootnote{`Like as the Damask Rose', Francis Quarles (1592 -- 1644), \cite{londonbook}. The attribution to Quarles has been questioned. This poem is was set to music by Sir Edward Elgar as \refpoem{Like \emph{to} the Damask Rose}, who made minor changes to the text.}\settowidth{\versewidth}{The swan's near death; man's life is done.}
\begin{verse}[\versewidth]
Like as the damask rose you see,\\*
Or like the blossom on the tree,\\
Or like the dainty flower of may,\\
Or like the morning to the day,\\
Or like the sun or like the shade,\\*
Or like the gourd which \textit{Jonas} had:\\!

Even such is man, whose thread is spun,\\*
Drawn out, and cut, and so is done.\\
The rose withers; the blossom blasteth;\\
The flower fades; the morning hasteth;\\
The sun sets; the shadow flies;\\*
The gourd consumes; and man he dies.\\!

Like to the grass that's newly sprung,\\*
Or like the tale that's new begun,\\
Or like the bird that's here today,\\
Or like the pearl{\`{e}}d dew of may,\\
Or like an hour, or like a span,\\*
Or like the singing of a swan:\\!

Even such is man, who lives by breath,\\*
Is here, now there, in life, in death.\\
The grass withers; the tale is ended;\\
The bird is flown; the dew's descended;\\
The hour is short, the span not long;\\*
The swan's near death; man's life is done.\\!

Like to the bubble in the brook,\\*
Or, in a glass, much like a look,\\
Or like a shuttle in weaver's hand,\\
Or like the writing on the sand,\\
Or like a thought, or like a dream,\\*
Or like the gliding of the stream:\\!

Even such is man, who lives by breath,\\*
Is here, now there, in life \& death.\\
The bubble's cut; the look's forgot;\\
The shuttle's flung; the writing's blot;\\
The thought is past; the dream is gone;\\*
The water glides; man's life is done.\\!

Like to an arrow from the bow,\\*
Or like swift course of wat'ry flow,\\
Or like the time 'twixt flood \& ebb,\\
Or like the spider's tender web,\\
Or like a race, or like a goal,\\*
Or like the dealing of a dole:\\!

Even such is man, whose brittle state\\*
Is always subject unto fate.\\
The arrow's shot; the flood soon spent,\\
The time no time, the web soon rent,\\
The race soon run, the goal soon won,\\*
The dole soon dealt, man's life first done.\\!

Like to the lightning from the sky,\\*
Or like a post that quick doth die,ust\\
Or like a quaver in short song,\\
Or like a journey three days long,\\
Or like the snow when summer's come,\\*
Or like the pear, or like the plum:\\!

Even such is man, who heaps up sorrow,\\*
Lives but this day, and dies tomorrow.\\
The lightning's past; the post must go;\\
The song is short; the journey's so;\\
The pear doth rot; the plum doth fall;\\*
The snow dissolves, and so must all.\\!

Like to the seed put in earth's womb,\\*
Or like dead \textit{Lazarus} in his tomb,\\
Or like \textit{Tabitha}, being asleep,\\
Or \textit{Jonas}-like within the deep,\\
Or like the night, or stars by day,\\*
Which seem to vanish clean away:\\!

Even so this death, man's life bereaves,\\*
But, being dead, man death deceives.\\
The seed it springeth; \textit{Lazarus} standeth;\\
\textit{Tabitha} walks, and \textit{Jonas} landeth;\\
The night is passed; the stars remain:\\*
So man that dies shall live again.
\end{verse}

\subsection{}

\blfootnote{`Old Shepherd's Prayer', Miss Charlotte Mew (1869 -- 1928), \cite{oxfordlarkin}. There is a Withybush in Haverfordwest (or Hwlffordd) in Pembrokeshire, and perhaps Church-Town could be St David's? But this is pure speculation. There is nothing especially Welsh -- other than the profession of sheep-farming itself -- about the shepherd and his dialect, and the place-names are probably just invented.}\settowidth{\versewidth}{    And down in yard, fit to burst his chain, yapping out at Sue I do hear young Mac.}
\begin{verse}[\versewidth]
Up to the bed by the window, where I be lyin',\\*
\vin Comes bells \& bleat of the flock wi' they two children's clack.\\
Over, from under the eaves there's the starlings flyin',\\*
\vin And down in yard, fit to burst his chain, yapping out at \textit{Sue} I do hear young \textit{Mac}.\\!

Turning around like a falled-over sack\\*
\vin I can see team ploughin' in \textsc{Whithy-bush} field \& meal carts startin' up road to \textsc{Church-Town};\\
Saturday afternoon the men goin' back\\*
\vin And the women from market, trapin' home over the down.\\!

Heavenly Master, I would like to wake to they same green places\\*
\vin Where I be knowed for breakin' dogs \& follerin' sheep.\\
And if I may not walk in th'old ways and look on th'old faces\\*
\vin I would sooner sleep.
\end{verse}

\subsection{}

\blfootnote{John Dryden, Poet Laureate (1631 -- 1700), \cite{odq}.}A thing well said will be wit in all languages.

\section{}

\subsection{}

\blfootnote{`A Song of a Young Lady to Her Ancient Lover', John Wilmot, 2nd Earl of Rochester (1647 -- 1680), \cite{newlove}. \P 10. Other sources give `heart' instead of `heat'. \P 18. Other sources give `their' instead `his'.}\settowidth{\versewidth}{    But still continue as thou art,}
\begin{verse}[\versewidth]
Ancient person, for whom I\\*
All the flattering youth defy,\\
Long be it ere thou grow old,\\
Aching, shaking, crazy, cold;\\
\vin But still continue as thou art,\\*
\vin Ancient person of my heart.\\!

On thy withered lips \& dry,\\*
Which like barren furrows lie,\\
Brooding kisses I will pour\\
Shall thy youthful heat restore\\
(Such kind showers in autumn fall,\\
And a second spring recall);\\
\vin Nor from thee will ever part,\\*
\vin Ancient person of my heart.\\!

Thy nobler part, which but to name\\*
In our sex would be counted shame,\\
By age's frozen grasp possessed,\\
From his ice shall be released,\\
And soothed by my reviving hand,\\
In former warmth \& vigour stand.\\
All a lover's wish can reach\\
For thy joy my love shall teach,\\
And for they pleasure shall improve\\
All that art can add to love.\\
\vin Yet still I love thee without art,\\*
\vin Ancient person of my heart.
\end{verse}

\subsection{}

\blfootnote{$\mathbb{R}$ Mary Herbert, Countess of Pembroke (1561 -- 1621), \cite{obev}. This is a translation of Psalm 117. In Lady Pembroke's time, the letter E was sometimes omitted when spelling the word \textit{praise}.}\settowidth{\versewidth}{Praise him that aye}
\begin{verse}[\versewidth]
Praise him that aye\\*
Remains the same:\\
All tongues display\\
\textit{Iehova}'s fame.\\
Sing all that share\\
This earthly ball:\\
His mercies are\\
Exposed to all.\\
Like as the word\\
Once he doth give,\\
Rolled in rec\'{o}rd,\\*
Doth time outlive.
\end{verse}

\subsection{}

\blfootnote{Dr William du Bois (1868 -- 1963), \cite{odq}.}The cost of liberty is less than the price of repression.

\section{}

\subsection{}

\blfootnote{$\mathbb{R}$ `La Belle Confidente', Sir Thomas Stanley (1625 -- 1678), \cite{pbev}.}\settowidth{\versewidth}{You earthly souls that court a wanton flame,}
\begin{verse}[\versewidth]
You earthly souls that court a wanton flame,\\*
\vin \vin Whose pale weak influence\\
Can rise no higher then the humble name\\
\vin \vin And narrow laws of sense,\\
\vin Learn by our friendship to create\\
\vin \vin An immaterial fire,\\
\vin Whose brightness angels may admire,\\*
\vin \vin But cannot emulate.\\!

Sickness may fright the roses from her cheek,\\*
\vin \vin Or make the lilies fade,\\
But all the subtle ways that death doth seek\\
\vin \vin Cannot my love invade:\\
\vin Flames that are kindled by the eye,\\
\vin \vin Through time \& age expire;\\
\vin But ours that boast a reach far higher\\*
\vin \vin Can nor decay nor die.\\!

For when we must resign our vital breath,\\*
\vin \vin Our loves by fate benighted,\\
We by this friendship shall survive in death,\\
\vin \vin Even in divorce united.\\
\vin Weak love through fortune or distrust\\
\vin \vin In time forgets to burn,\\
\vin But this pursues us to the urn,\\*
\vin \vin And marries either's dust.
\end{verse}

\subsection{}

\blfootnote{$\mathbb{R}$ `Wedded', Isaac Rosenberg (1890 -- 1918), \cite{pbev}.}\settowidth{\versewidth}{    Love, fall from sheltering tresses.}
\begin{verse}[\versewidth]
They leave their love-lorn haunts,\\*
\vin Their sigh-warm floating \textsc{Eden};\\
And they are mute at once,\\
\vin Mortals by God unheeden,\\*
\vin By their past kisses chidden.\\!

But they have kissed and known\\*
\vin Clear things we dim by guesses --\\
Spirit to spirit grown:\\
\vin Heaven, born in hand-caresses.\\*
\vin Love, fall from sheltering tresses.\\!

And they are dumb \& strange:\\*
\vin Bared trees bowed from each other.\\
Their last green interchange\\
\vin What lost dreams shall discover?\\*
\vin Dead, strayed, to love-strange lover.
\end{verse}

\subsection{}

\blfootnote{The Rt Hon Edmund Burke (1729 -- 1797), \cite{odq}.}By hating vices too much, they come to love men too little.

\section{}

\subsection{}

\blfootnote{Henry Howard, Earl of Surrey (1517 -- 1547), \cite{pbev}. These are the opening lines of Morris' poem \refpoem{October}.}\settowidth{\versewidth}{    The grey tower sings a strange old tinkling tune.}
\begin{verse}[\versewidth]
O love, turn from the changing sea and gaze,\\*
\vin Down these grey slopes, upon the year grown old,\\
A-dying 'mid the autumn-scented haze\\
\vin That hangeth o'er the hollow in the wold,\\
\vin Where the wind-bitten ancient elms infold\\
Grey church, long barn, orchard, and red-roofed stead,\\*
Wrought in dead days for men a long while dead.\\!

Come down, O love; may not our hands still meet,\\*
\vin Since still we live today, forgetting june,\\
Forgetting may, deeming october sweet?\\
\vin O hearken, hearken: through the afternoon\\
\vin The grey tower sings a strange old tinkling tune.\\
Sweet, sweet, \& sad, the toiling year's last breath,\\*
To satiate of life, to strive with death.\\!

And we too -- will it not be soft \& kind,\\*
\vin That rest from life, from patience, \& from pain,\\
That rest from bliss we know not when we find,\\
\vin That rest from love which ne'er the end can gain?\\
\vin Hark: how the tune swells, that erewhile did wane.\\
Look up, love. Ah cling close, and never move.\\*
How can I have enough of life \& love?
\end{verse}

\subsection{}

\blfootnote{`Ozymandias of Egypt', Percy Shelley (1792 -- 1822), \cite{treasury}. Shelley seems to have written this sonnet as one half of a sonnet-writing competition with his friend Horace Smith, who published a very similar, if clearly inferior, poem in the same journal a month later. \P 11. Professor Holloway, in his introduction to the \textit{Oxford Book of Local Verses}, highlights the double meaning in this line: `Time, the poet intimates, invites the proud of a later age, as they gaze upon the forgotten ruins of that spurious grandeur, to despair in a deeper sense.'}\settowidth{\versewidth}{The hand that mocked them and the heart that fed.}
\begin{verse}[\versewidth]
I met a traveller from an antique land\\*
Who said: Two vast \& trunkless legs of stone\\
Stand in the desert. Near them on the sand,\\
Half sunk, a shattered visage lies, whose frown\\
And wrinkled lip \& sneer of cold command\\
Tell that its sculptor well those passions read\\
Which yet survive, stamped on these lifeless things,\\
The hand that mocked them and the heart that fed.\\
And on the pedestal these words appear:\\
\textit{My name is \textit{Ozymandias}, king of kings:}\\
\textit{Look on my works, ye mighty, and despair!}\\
Nothing beside remains: round the decay\\
Of that colossal wreck, boundless \& bare,\\*
The lone \& level sands stretch far away.
\end{verse}

\subsection{}

\blfootnote{The Rt Hon Edmund Burke (1729 -- 1797), \cite{odq}.}Man is by his constitution a religious animal.

\section{}

\subsection{}

\blfootnote{$\mathbb{R}$ Alfred Tennyson, 1st Baron Tennyson, Poet Laureate (1809 -- 1892), \cite{pbev}. These are lines 407-440 of \refpoem{The Passing of Arthur}.}\settowidth{\versewidth}{Moved from the brink, like some full-breasted swan}
\begin{verse}[\versewidth]
And slowly answered \textit{Arthur} from the barge:\\*
The old order changeth, yielding place to new,\\
And God fulfils himself in many ways,\\
Lest one good custom should corrupt the world.\\
Comfort thyself: what comfort is in me?\\
I have lived my life, and that which I have done\\
May he within himself make pure, but thou,\\
If thou shouldst never see my face again,\\
Pray for my soul. More things are wrought by prayer\\
Than this world dreams of. Wherefore, let thy voice\\
Rise like a fountain for me night \& day.\\
For what are men better than sheep or goats\\
That nourish a blind life within the brain,\\
If, knowing God, they lift not hands of prayer\\
Both for themselves \& those who call them friend?\\
For so the whole round earth is every way\\
Bound by gold chains about the feet of God.\\
But now farewell. I am going a long way\\
With these thou seest -- if indeed I go\\
(For all my mind is clouded with a doubt) --\\
To the island-valley of \textsc{Avilion};\\
Where falls not hail, or rain, or any snow,\\
Nor ever wind blows loudly; but it lies\\
Deep-meadowed, happy, fair with orchard lawns\\
And bowery hollows crowned with summer sea,\\*
Where I will heal me of my grievous wound.\\!

So said he, and the barge with oar \& sail\\*
Moved from the brink, like some full-breasted swan\\
That, fluting a wild carol ere her death,\\
Ruffles her pure cold plume, and takes the flood\\
With swarthy webs. Long stood Sir \textit{Bedivere}\\
Revolving many memories, till the hull\\
Looked one black dot against the verge of dawn,\\*
And on the mere the wailing died away.
\end{verse}

\subsection{}

\blfootnote{`Destiny', Sir Thomas Stanley (1625 -- 1678), \cite{londonbook}.}\settowidth{\versewidth}{Or think the hours do move too slow;}
\begin{verse}[\versewidth]
\vin Chide, chide no more away\\*
The fleeting daughters of the day,\\
Nor with impatient thoughts outrun\\
\vin \vin The lazy sun,\\
Or think the hours do move too slow;\\
\vin \vin Delay is kind,\\
\vin And we too soon shall find\\*
That which we seek, yet fear to know.\\!

\vin The mystic dark decrees\\*
Unfold not of the destinies,\\
Nor boldly seek to antedate\\
\vin \vin The laws of fate;\\
Thy anxious search awhile forbear,\\
\vin \vin Suppress thy haste\\
\vin And know that time at last\\*
Will crown thy hope, or fix thy fear.
\end{verse}

\subsection{}

\blfootnote{Samuel Butler (1835 -- 1902), \cite{twoaf}. The full quotation is, `Autumn is the mellower season, and what we lose in flowers we more than gain in fruits.'}What we lose in flowers we more than gain in fruits.

\section{}

\subsection{}

\blfootnote{`Old Man', Edward Thomas (1878 -- 1917), \cite{obev}. The scientific name of the plant described is \textit{Artemisia abrotanum}.}\settowidth{\versewidth}{Old man, or lad's love -- in the name there's nothing}
\begin{verse}[\versewidth]
Old man, or lad's love -- in the name there's nothing\\*
To one that knows not lad's love, or old man,\\
The hoar-green feathery herb, almost a tree,\\
Growing with rosemary \& lavender.\\
Even to one that knows it well, the names\\
Half decorate, \sfrac{$1$}{$2$} perplex, the thing it is:\\
At least, what that is clings not to the names\\*
In spite of time. And yet I like the names.\\!

The herb itself I like not, but for certain\\*
I love it, as some day the child will love it\\
Who plucks a feather from the door-side bush\\
Whenever she goes in or out of the house.\\
Often she waits there, snipping the tips and shrivelling\\
The shreds at last on to the path, perhaps\\
Thinking, perhaps of nothing, till she sniffs\\
Her fingers and runs off. The bush is still\\
But \sfrac{$1$}{$2$} as tall as she, though it is as old;\\
So well she clips it. Not a word she says;\\
And I can only wonder how much hereafter\\
She will remember, with that bitter scent,\\
Of garden rows, \& ancient damson trees\\
Topping a hedge, a bent path to a door,\\
A low thick bush beside the door, and me\\*
Forbidding her to pick.\\!

\textcolor{white}{Forbidding her to pick.} As for myself,\\*
Where first I met the bitter scent is lost.\\
I, too, often shrivel the grey shreds,\\
Sniff them and think and sniff again and try\\
Once more to think what it is I am remembering,\\
Always in vain. I cannot like the scent,\\
Yet I would rather give up others more sweet,\\*
With no meaning, than this bitter one.\\!

I have mislaid the key. I sniff the spray\\*
And think of nothing; I see and I hear nothing;\\
Yet seem, too, to be listening, lying in wait\\
For what I should, yet never can, remember:\\
No garden appears, no path, no hoar-green bush\\
Of lad's love, or old man, no child beside,\\
Neither father nor mother, nor any playmate;\\*
Only an avenue, dark, nameless, without end.
\end{verse}

\subsection{}

\blfootnote{`Requiem', Robert Stevenson (1850 -- 1894), \cite{ptmgmc}. These verses are inscribed, according to Stevenson's wishes, on his tomb on Upolu, an island now part of the Independent State of Samoa. Philip Larkin's infamous \refpoem{This Be the Verse} is presumably a response to this poem. \P 8. What has come to be regarded as the standard version of this poem gives the penultimate line as, `Home is the sailor, home from sea', i.e. without the second ``the''. However, the Almanacker prefers the version with said ``the'', and, indeed, this is the version found on the aforementioned tomb.}\settowidth{\versewidth}{This be the verse you 'grave for me:}
\begin{verse}[\versewidth]
Under the wide \& starry sky,\\*
Dig the grave and let me lie.\\
Glad did I live and gladly die,\\*
\vin And I laid me down with a will.\\!

This be the verse you 'grave for me:\\*
\textit{Here he lies where he longed to be;}\\
\textit{Home is the sailor, home from the sea,}\\*
\vin \textit{And the hunter home from the hill.}
\end{verse}

\subsection{}

\blfootnote{Samuel Butler (1835 -- 1902), \cite{odq}.}You can do very little with faith, but you can do nothing without it.

\section{}

\subsection{}

\blfootnote{Walt Whitman (1819 -- 1892), \cite{ptmgmc}.}\settowidth{\versewidth}{The skies of day night, colours, densities, forms, maybe these are (as doubtless they are) only apparitions, and the real something has yet to be known,}
\begin{verse}[\versewidth]
Of the terrible doubt of appearances,\\*
Of the uncertainty after all, that we may be deluded,\\
That maybe reliance \& hope are but speculations after all,\\
That maybe identity beyond the grave is a beautiful fable only,\\
Maybe the things I perceive, the animals, plants, men, hills, shining \& flowing waters,\\
The skies of day \& night, colours, densities, forms, maybe these are (as doubtless they are) only apparitions, and the real something has yet to be known,\\
(How often they dart out of themselves as if to confound me \& mock me!\\
How often I think neither I know, nor any man knows, aught of them)\\
Maybe seeming to me what they are (as doubtless they indeed but seem) as from my present point of view, and might prove (as of course they would) nought of what they appear, or nought anyhow, from entirely changed points of view;\\
To me these \& the like of these are curiously answered by my lovers, my dear friends,\\
When he whom I love travels with me or sits a long while holding me by the hand,\\
When the subtle air, the impalpable, the sense that words \& reason hold not, surround us \& pervade us,\\
Then I am charged with untold \& untellable wisdom, I am silent; I require nothing further;\\
I cannot answer the question of appearances or that of identity beyond the grave,\\
But I walk or sit indifferent; I am satisfied;\\*
He ahold of my hand has completely satisfied me.
\end{verse}

\subsection{}

\blfootnote{Edward Thomas (1878 -- 1917), \cite{oxfordlarkin}.}\settowidth{\versewidth}{    Rose, currant, raspberry, or goutweed,}
\begin{verse}[\versewidth]
\vin Today I think\\*
Only with scents- scents dead leaves yield,\\
\vin And bracken, and wild carrot's seed,\\*
And the square mustard field;\\!

\vin Odours that rise\\*
When the spade wounds the roots of tree,\\
\vin Rose, currant, raspberry, or goutweed,\\*
Rhubarb or celery;\\!

\vin The smoke's smell, too,\\*
Flowing from where a bonfire burns\\
\vin The dead, the waste, the dangerous,\\*
And all to sweetness turns.\\!

\vin It is enough\\*
To smell, to crumble the dark earth,\\
\vin While the robin sings over again\\*
Sad songs of autumn mirth.
\end{verse}

\subsection{}

\blfootnote{Percy Shelley (1792 -- 1822), \cite{pbev}. This is a line from Shelley's \refbook{Adonais}, his elegy for John Keats.}He hath awakened from the dream of life.

\chapter{December}

\section{}

\subsection{}

\blfootnote{`On Anothers Sorrow', William Blake (1757 -- 1827), \cite{blakea}.}\settowidth{\versewidth}{Weep, nor be with sorrow filled?}
\begin{verse}[\versewidth]
Can I see another's woe,\\*
And not be in sorrow too?\\
Can I see another's grief,\\*
And not seek for kind relief?\\!

Can I see a falling tear,\\*
And not feel my sorrow's share?\\
Can a father see his child\\*
Weep, nor be with sorrow filled?\\!

Can a mother sit and hear\\*
An infant groan, an infant fear?\\
No, no. Never can it be.\\*
Never, never can it be.\\!

And can he who smiles on all\\*
Hear the wren with sorrows small,\\
Hear the small bird's grief \& care,\\*
Hear the woes that infants bear\\!

And not sit beside the nest,\\*
Pouring pity in their breast?\\
And not sit the cradle near,\\*
Weeping tear on infant's tear?\\!

And not sit both night \& day,\\*
Wiping all our tears away?\\
O no. Never can it be.\\*
Never, never can it be.\\!

He doth give his joy to all:\\*
He becomes an infant small:\\
He becomes a man of woe:\\*
He doth feel the sorrow too.\\!

Think not thou canst sigh a sigh,\\*
And thy Maker is not by:\\
Think not thou canst weep a tear,\\*
And thy Maker is not near.\\!

O he gives to us his joy\\*
That our grief he may destroy:\\
Till our grief is fled \& gone\\*
He doth sit by us and moan.
\end{verse}

\subsection{}

\blfootnote{`My Pretty Rose Tree', William Blake (1757 -- 1827), \cite{blakea}.}\settowidth{\versewidth}{    And I passed the sweet flower o'er.}
\begin{verse}[\versewidth]
A flower was offered to me,\\*
\vin Such a flower as may never bore.\\
But I said I've a pretty rose tree,\\*
\vin And I passed the sweet flower o'er.\\!

Then I went to my pretty rose tree,\\*
\vin To tend her by day and by night.\\
But my rose turned away with jealousy,\\*
\vin And her thorns were my only delight.
\end{verse}

\subsection{}

\blfootnote{Mrs Elizabeth Browning (1806 -- 1861), \cite{londonbook}. This is the final line of Mrs Browning's sonnet \refpoem{Grief}; `it' is a statue.}If it could weep, it could arise and go.

\section{}

\subsection{}

\blfootnote{`William Bond', William Blake (1757 -- 1827), \cite{blakea}.}\settowidth{\versewidth}{    And Mary fell down on the right-hand floor,}
\begin{verse}[\versewidth]
I wonder whether the girls are mad,\\*
\vin And I wonder whether they mean to kill,\\
And I wonder if \textit{William Bond} will die,\\*
\vin For assuredly he is very ill.\\!

He went to church in a may morning,\\*
\vin Attended by fairies, one, two \& three;\\
But the angels of providence drove them away,\\*
\vin And he returned home in misery.\\!

He went not out to the field nor fold,\\*
\vin He went not out to the village nor town,\\
But he came home in a black, black cloud,\\*
\vin And took to his bed \& there lay down.\\!

And an angel of providence at his feet,\\*
\vin And an angel of providence at his head,\\
And in the midst a black, black cloud,\\*
\vin And in the midst the sick man on his bed.\\!

And on his right hand was \textit{Mary Green},\\*
\vin And on his left hand was his sister \textit{Jane},\\
And their tears fell through the black, black cloud\\*
\vin To drive away the sick man's pain.\\!

`O \textit{William}, if thou dost another love,\\*
\vin Dost another love better than poor \textit{Mary},\\
Go \& take that other to be thy wife,\\*
\vin And \textit{Mary Green} shall her servant be.'\\!

`Yes, \textit{Mary}, I do another love,\\*
\vin Another I love far better than thee,\\
And another I will have for my wife;\\*
\vin Then what have I to do with thee?\\!

`For thou art melancholy pale,\\*
\vin And on thy head is the cold moon's shine,\\
But she is ruddy \& bright as day,\\*
\vin And the sunbeams dazzle from her eyne.'\\!

\textit{Mary} trembled \& \textit{Mary} chilled,\\*
\vin And \textit{Mary} fell down on the right-hand floor,\\
That \textit{William Bond} \& his sister \textit{Jane}\\*
\vin Scarce could recover \textit{Mary} more.\\!

When \textit{Mary} woke \& found her laid\\*
\vin On the right hand of her \textit{William} dear,\\
On the right hand of his loved bed,\\*
\vin And saw her \textit{William Bond} so near,\\!

The fairies that fled from \textit{William Bond}\\*
\vin Danc\`{e}d around her shining head;\\
They danc\`{e}ed over the pillow white,\\*
\vin And the angels of providence left the bed.\\!

I thought love lived in the hot sunshine,\\*
\vin But O he lives in the moony light!\\
I thought to find love in the heat of day,\\*
\vin But sweet love is the comforter of night.\\!

Seek love in the pity of others' woe,\\*
\vin In the gentle relief of another's care,\\
In the darkness of night and the winter's snow;\\*
\vin In the naked \& outcast, seek love there!
\end{verse}

\subsection{}

\blfootnote{`The Chimney Sweeper', William Blake (1757 -- 1827), \cite{blakea}. The Almanacker has omitted the first verse.}\settowidth{\versewidth}{They clothed me in the clothes of death,}
\begin{verse}[\versewidth]
Because I was happy upon the heath,\\*
\vin And smiled among the winter's snow,\\
They clothed me in the clothes of death,\\*
\vin And taught me to sing the notes of woe.\\!

And because I am happy \& dance \& sing,\\*
\vin They think they have done me no injury,\\
And are gone to praise God \& his Priest \& King,\\*
\vin Who make up a heaven of our misery.
\end{verse}

\subsection{}

\blfootnote{William Blake (1757 -- 1827), \cite{blakea}. This is one of Blake's `Proverbs of Hell' from \refbook{The Marriage of Heaven and Hell}.}Excess of sorrow laughs. Excess of joy weeps.

\section{}

\subsection{}

\blfootnote{`London Snow', Dr Robert Bridges, Poet Laureate (1844 -- 1930), \cite{oxfordlarkin}.}\settowidth{\versewidth}{The eye marvelled -- marvelled at the dazzling whiteness;}
\begin{verse}[\versewidth]
When men were all asleep the snow came flying,\\*
\vin In large white flakes falling on the city brown,\\
Stealthily \& perpetually settling \& loosely lying,\\
\vin Hushing the latest traffic of the drowsy town;\\
Deadening, muffling, stifling its murmurs failing;\\
\vin Lazily \& incessantly floating down \& down:\\
Silently sifting \& veiling road, roof \& railing;\\
\vin Hiding difference, making unevenness even,\\
Into angles \& crevices softly drifting \& sailing.\\
\vin All night it fell, and when full inches seven\\
It lay in the depth of its uncompacted lightness,\\
\vin The clouds blew off from a high \& frosty heaven;\\
And all woke earlier for the unaccustomed brightness\\
\vin Of the winter dawning, the strange unheavenly glare:\\
The eye marvelled -- marvelled at the dazzling whiteness;\\
\vin The ear hearkened to the stillness of the solemn air;\\
No sound of wheel rumbling nor of foot falling,\\
\vin And the busy morning cries came thin \& spare.\\
Then boys I heard, as they went to school, calling,\\
\vin They gathered up the crystal manna to freeze\\
Their tongues with tasting, their hands with snowballing;\\
\vin Or rioted in a drift, plunging up to the knees;\\
Or peering up from under the white-mossed wonder,\\
\vin O look at the trees! they cried, O look at the trees!\\
With lessened load a few carts creak and blunder,\\
\vin Following along the white deserted way,\\
A country company long dispersed asunder:\\
\vin When now already the sun, in pale display\\
Standing by \textsc{Paul's} high dome, spread forth below\\
\vin His sparkling beams, and awoke the stir of the day.\\
For now doors open, and war is waged with the snow;\\
\vin And trains of sombre men, past tale of number,\\
Tread long brown paths, as toward their toil they go:\\
\vin But even for them awhile no cares encumber\\
Their minds diverted; the daily word is unspoken,\\
\vin The daily thoughts of labour \& sorrow slumber\\*
At the sight of the beauty that greets them, for the charm they have broken.
\end{verse}

\subsection{}

\blfootnote{`The Sick Rose', William Blake (1757 -- 1827), \cite{newlove}.}\settowidth{\versewidth}{    In the howling storm}
\begin{verse}[\versewidth]
O rose, thou art sick.\\*
\vin The invisible worm\\
That flies in the night\\*
\vin In the howling storm\\!

Has found out thy bed\\*
\vin Of crimson joy:\\
And his dark secret love\\*
\vin Does thy life destroy.
\end{verse}

\subsection{}

\blfootnote{Henry St John, 1st Viscount Bolingbroke (1678 -- 1751), \cite{odq}.}The great mistake is that of looking upon men as virtuous.

\section{}

\subsection{}

\blfootnote{`Thanatopsis', William Bryant (1794 -- 1878), \cite{norton}. A \refbook{thanatopsis} is a meditation on death. \P 52. Barca is the name of an ancient city in Libya, hence `barcan wilderness'. \P 54. Oregon was the original name of the Columbia river in the State of Oregon.}\settowidth{\versewidth}{Thy image. Earth, that nourished thee, shall claim}
\begin{verse}[\versewidth]
To him who in the love of nature holds\\*
Communion with her visible forms, she speaks\\
A various language; for his gayer hours\\
She has a voice of gladness, and a smile\\
And eloquence of beauty, and she glides\\
Into his darker musings, with a mild\\
And healing sympathy, that steals away\\
Their sharpness, ere he is aware. When thoughts\\
Of the last bitter hour come like a blight\\
Over thy spirit, and sad images\\
Of the stern agony, and shroud, and pall,\\
And breathless darkness, and the narrow house,\\
Make thee to shudder, and grow sick at heart;\\
Go forth, under the open sky, and list\\
To nature's teachings, while from all around --\\
Earth \& her waters, \& the depths of air --\\*
Comes a still voice --\\!

\textcolor{white}{Comes a still voice --} Yet a few days, and thee\\*
The all-beholding sun shall see no more\\
In all his course; nor yet in the cold ground,\\
Where thy pale form was laid, with many tears,\\
Nor in the embrace of ocean, shall exist\\
Thy image. Earth, that nourished thee, shall claim\\
Thy growth, to be resolved to earth again,\\
And, lost each human trace, surrendering up\\
Thine individual being, shalt thou go\\
To mix for ever with the elements,\\
To be a brother to the insensible rock\\
And to the sluggish clod, which the rude swain\\
Turns with his share, and treads upon. The oak\\
Shall send his roots abroad, and pierce thy mould.\\
Yet not to thine eternal resting-place\\
Shalt thou retire alone, nor couldst thou wish\\
Couch more magnificent. Thou shalt lie down\\
With patriarchs of the infant world, with kings,\\
The powerful of the earth, the wise, the good,\\
Fair forms, \& hoary seers of ages past,\\
All in one mighty sepulchre. The hills\\
Rock-ribbed \& ancient as the sun, the vales\\
Stretching in pensive quietness between;\\
The venerable woods -- rivers that move\\
In majesty, and the complaining brooks\\
That make the meadows green; and, poured round all,\\
Old ocean's grey \& melancholy waste,\\
Are but the solemn decorations all\\
Of the great tomb of man. The golden sun,\\
The planets, all the infinite host of heaven,\\
Are shining on the sad abodes of death,\\
Through the still lapse of ages. All that tread\\
The globe are but a handful to the tribes\\
That slumber in its bosom. Take the wings\\
Of morning, pierce the barcan wilderness,\\
Or lose thyself in the continuous woods\\
Where rolls the \textsc{Oregon}, and hears no sound,\\
Save his own dashings -- yet the dead are there:\\
And 1,000,000s in those solitudes, since first\\
The flight of years began, have laid them down\\
In their last sleep—the dead reign there alone.\\
So shalt thou rest, and what if thou withdraw\\
In silence from the living, and no friend\\
Take note of thy departure? All that breathe\\
Will share thy destiny. The gay will laugh\\
When thou art gone, the solemn brood of care\\
Plod on, and each one as before will chase\\
His favorite phantom; yet all these shall leave\\
Their mirth \& their employments, and shall come\\
And make their bed with thee. As the long train\\
Of ages glide away, the sons of men,\\
The youth in life's green spring, and he who goes\\
In the full strength of years, matron \& maid,\\
The speechless babe, and the grey-headed man\\
Shall one by one be gathered to thy side,\\
By those, who in their turn shall follow them.\\
So live, that when thy summons comes to join\\
The innumerable caravan, which moves\\
To that mysterious realm, where each shall take\\
His chamber in the silent halls of death,\\
Thou go not, like the quarry-slave at night,\\
Scourged to his dungeon, but, sustained \& soothed\\
By an unfaltering trust, approach thy grave,\\
Like one who wraps the drapery of his couch\\*
About him, and lies down to pleasant dreams.
\end{verse}

\subsection{}

\blfootnote{Witter Bynner (1881 -- 1968), \cite{norton}.}\settowidth{\versewidth}{To the earth's green motions lying warm below}
\begin{verse}[\versewidth]
Winter uncovers distances, I find;\\*
And so the cold and so the wintry mind\\
Takes leaves away, till there is left behind\\
A wide cold world. And so the heart grows blind\\
To the earth's green motions lying warm below\\*
Field upon field, field upon field, of snow.
\end{verse}

\subsection{}

\blfootnote{Henry St John, 1st Viscount Bolingbroke (1678 -- 1751), \cite{odq}.}Truth lies within a little and certain compass, but error is immense.

\section{}

\subsection{}

\blfootnote{Dr Thomas Campion (1567 -- 1620), \cite{londonbook}.}\settowidth{\versewidth}{    Ordained to pine in sorrow's endless bands,}
\begin{verse}[\versewidth]
The cypress curtain of the night is spread,\\*
\vin And over all a silent dew is cast.\\
The weaker cares by sleep are conquer{\`{e}}d.\\
\vin But I alone with hideous grief aghast,\\
In spite of \textit{Morpheus}' charms a watch do keep\\*
Over mine eyes to banish careless sleep.\\!

Yet oft my trembling eyes through faintness close;\\*
\vin And then the map of hell before me stands,\\
Which ghosts do see and I am one of those\\
\vin Ordained to pine in sorrow's endless bands,\\
Since from my wretched soul all hopes are reft,\\*
And now no cause of life to me is left.\\!

Grief, sieze my soul for that will still endure\\*
\vin When my crazed body is consumed and gone;\\
Bear it to thy black den; there keep it sure,\\
\vin Where thou 10,000 souls dost tire upon:\\
Yet all do not afford such food to thee\\*
All this poor one, the worser part of me.
\end{verse}

\subsection{}

\blfootnote{`The Old Woman', Joseph Campbell (1879 -- 1944), \cite{oxfordlarkin}.}\settowidth{\versewidth}{    With her travail done,}
\begin{verse}[\versewidth]
As a white candle\\*
\vin In a holy place,\\
So is the beauty\\*
\vin Of an aged face.\\!

As the spent radiance\\*
\vin Of the winter sun,\\
So is a woman\\*
\vin With her travail done,\\!

Her brood gone from her,\\*
\vin And her thoughts as still\\
As the waters\\*
\vin Under a ruined mill.
\end{verse}

\subsection{}

\blfootnote{Heywood Broun (1888 -- 1939), \cite{odq}.}Everybody favours free speech in the slack moments when no axes are being ground.

\section{}

\subsection{}

\blfootnote{`Badger', John Clare (1793 -- 1864), \cite{norton}.}\settowidth{\versewidth}{He comes and hears. They let the strongest loose.}
\begin{verse}[\versewidth]
When midnight comes a host of dogs \& men\\*
Go out and track the badger to his den,\\
And put a sack within the hole, and lie\\
Till the old grunting badger passes by.\\
He comes and hears. They let the strongest loose.\\
The old fox hears the noise and drops the goose.\\
The poacher shoots and hurries from the cry,\\
And the old hare half wounded buzzes by.\\
They get a forked stick to bear him down\\
And clap the dogs and take him to the town,\\
And bait him all the day with many dogs,\\
And laugh \& shout \& fright the scampering hogs.\\
He runs along and bites at all he meets:\\*
They shout \& hollo down the noisy streets.\\!

He turns about to face the loud uproar\\*
And drives the rebels to their very door.\\
The frequent stone is hurled where'er they go;\\
When badgers fight, then everyone's a foe.\\
The dogs are clapped and urged to join the fray;\\
The badger turns and drives them all away.\\
Though scarcely half as big, demure \& small,\\
He fights with dogs for bones and beats them all.\\
The heavy mastiff, savage in the fray,\\
Lies down and licks his feet and turns away.\\
The bulldog knows his match and waxes cold,\\
The badger grins and never leaves his hold.\\
He drives the crowd and follows at their heels\\*
And bites them through; the drunkard swears \& reels.\\!

The frighted women take the boys away,\\*
The blackguard laughs and hurries on the fray.\\
He tries to reach the woods, an awkward race,\\
But sticks \& cudgels quickly stop the chase.\\
He turns againn and drives the noisy crowd\\
And beats the many dogs in noises loud.\\
He drives away and beats them every one,\\
And then they loose them all and set them on.\\
He falls as dead and kicked by boys \& men,\\
Then starts and grins and drives the crowd again;\\
Till kicked \& torn \& beaten out he lies\\*
And leaves his hold and cackles, groans, \& dies.
\end{verse}

\subsection{}

\blfootnote{$\mathbb{R}$ `Upon a Young Married Couple Dead and Buryed Together', The Rev Canon Richard Crashaw (1613 -- 1649), \cite{pbev}. There is another, somewhat longer, version of this poem.}\settowidth{\versewidth}{Because they both lived but one life.}
\begin{verse}[\versewidth]
To these, whom death again did wed,\\*
This grave's their second marriage bed.\\
For though the hand of fate could force\\
'Twixt soul \& body a divorce,\\
It could not sunder man \& wife,\\
Because they both lived but one life.\\
Peace, good reader. Do not weep.\\
Peace, the lovers are asleep.\\
They, sweet turtles, folded lie\\*
In the last knot that love could tie.
\end{verse}

\subsection{}

\blfootnote{Heywood Broun (1888 -- 1939), \cite{odq}.}Posterity is as likely to be wrong as anybody else.

\section{}

\subsection{}

\blfootnote{William Cowper (1731 -- 1800), \cite{norton}. This is an excerpt from a long poem, \refbook{The Task}, a series of urbane reflections.}\settowidth{\versewidth}{Were tasked to his full strength, absorbed lost.}
\begin{verse}[\versewidth]
Just when our drawing-rooms begin to blaze\\*
With lights by clear reflection multiplied\\
From many a mirror (in which he of \textsc{Gath},\\
\textit{Goliath}, might have seen his giant bulk\\
Whole without stooping, tow'ring crest \& all),\\
My pleasures too begin. But me perhaps\\
The glowing hearth may satisfy awhile\\
With faint illumination that uplifts\\
The shadow to the ceiling, there by fits\\
Dancing uncouthly to the quiv'ring flame.\\
Not undelightful is an hour to me\\
So spent in parlour twilight; such a gloom\\
Suits well the thoughtful or unthinking mind,\\
The mind contemplative, with some new theme\\
Pregnant, or indisposed alike to all.\\
Laugh ye, who boast your more mercurial pow'rs\\
That never feel a stupor, know no pause\\
Nor need one. I am conscious, and confess\\
Fearless, a soul that does not always think.\\
Me oft has fancy ludicrous \& wild\\
Soothed with a waking dream of houses, tow'rs,\\
Trees, churches, \& strange visages expressed\\
In the red cinders, while with poring eye\\
I gazed, myself creating what I saw.\\
Nor less amused have I quiescent watched\\
The sooty films that play upon the bars --\\
Pendulous, \& foreboding in the view\\
Of superstition, prophesying still,\\
Though still deceived, some stranger's near approach.\\
'Tis thus the understanding takes repose\\
In indolent vacuity of thought,\\
And sleeps \& is refrshed. Meanwhile the face\\
Conceals the mood lethargic with a mask\\
Of deep deliberation, as the man\\
Were tasked to his full strength, absorbed \& lost.\\
Thus oft reclined at ease, I lose an hour\\
At evening, till at length the freezing blast\\
That sweeps the bolted shutter, summons home\\
The recollected powers and, snapping short\\
The glassy threads with which the fancy weaves\\
Her brittle toys, restores me to myself.\\
How calm is my recess, and how the frost\\
Raging abroad, and the rough wind, endear\\
The silence \& the warmth enjoyed within.\\
I saw the woods \& fields at close of day,\\
A variegated show; the meadows green\\
Though faded, and the lands where lately waved\\
The golden harvest, of a mellow brown,\\
Upturned so lately by the forceful share.\\
I saw far off the weedy fallows smile\\
With verdure not unprofitable, grazed\\
By flocks fast-feeding \& selecting each\\
His fav'rite herb; while all the leafless groves\\
That skirt th'horizon wore a sable hue\\
Scarce noticed in the kindred dusk of eve.\\
Tomorrow brings a change, a total change\\
Which even now -- though silently performed\\
And slowly, and by most unfelt -- the face\\
Of universal nature undergoes.\\
Fast falls a fleecy show'r. The downy flakes\\
Descending, and with never-ceasing lapse\\
Softly alighting upon all below,\\
Assimilate all objects. Earth receives\\
Gladly the thick'ning mantle, \& the green\\
And tender blade that feared the chilling blast\\*
Escapes unhurt beneath so warm a veil.
\end{verse}

\subsection{}

\blfootnote{$\mathbb{R}$ Samuel Daniel (1562 -- 1619), \cite{pbev}. This is one of Daniel's sonnets \refpoem{To Delia}.}\settowidth{\versewidth}{Soon doth it fade, that makes the fairest flourish;}
\begin{verse}[\versewidth]
Beauty, sweet love, is like the morning dew;\\*
\vin Whose short refresh upon the tender green,\\
Cheers for a time, but till the sun doth show:\\
\vin And straight 'tis gone, as it had never been.\\
Soon doth it fade, that makes the fairest flourish;\\
\vin Short is the glory of the blushing rose:\\
The hue which thou so carefully dost nourish;\\
\vin Yet which, at length, thou must be forced to lose.\\
When thou, surcharged with burden of thy years,\\
\vin Shalt bend thy wrinkles homeward to the earth;\\
When time hath made a passport for thy fears,\\
\vin Dated in age, the Kalends of our death:\\
But, ah, no more. This hath been often told;\\*
And women grieve to think they must be old.
\end{verse}

\subsection{}

\blfootnote{Robert Browning (1828 -- 1889), \cite{odq}. This is a line from \refpoem{A Toccata of Galuppi's}.}In you come with your cold music till I creep through every nerve.

\section{}

\subsection{}

\blfootnote{`The Wood-Pile', Robert Frost, Poet Laureate of Vermont (1874 -- 1963), \cite{norton}.}\settowidth{\versewidth}{One flight out sideways would have undeceived him.}
\begin{verse}[\versewidth]
Out walking in the frozen swamp one grey day,\\*
I paused and said, `I will turn back from here.\\
No, I will go on farther -- and we shall see.'\\
The hard snow held me, save where now \& then\\
One foot went through. The view was all in lines\\
Straight up \& down of tall slim trees\\
Too much alike to mark or name a place by\\
So as to say for certain I was here\\
Or somewhere else: I was just far from home.\\
A small bird flew before me. He was careful\\
To put a tree between us when he lighted,\\
And say no word to tell me who he was\\
Who was so foolish as to think what he thought.\\
He thought that I was after him for a feather --\\
The white one in his tail; like one who takes\\
Everything said as personal to himself.\\
One flight out sideways would have undeceived him.\\
And then there was a pile of wood for which\\
I forgot him and let his little fear\\
Carry him off the way I might have gone,\\
Without so much as wishing him good night.\\
He went behind it to make his last stand.\\
It was a cord of maple, cut \& split\\
And piled -- and measured, four by four by eight.\\
And not another like it could I see.\\
No runner tracks in this year's snow looped near it.\\
And it was older sure than this year's cutting,\\
Or even last year's or the year's before.\\
The wood was grey \& the bark warping off it\\
And the pile somewhat sunken. Clematis\\
Had wound strings round \& round it like a bundle.\\
What held it though on one side was a tree\\
Still growing, and on one a stake \& prop,\\
These latter about to fall. I thought that only\\
Someone who lived in turning to fresh tasks\\
Could so forget his handiwork on which\\
He spent himself, the labor of his axe,\\
And leave it there far from a useful fireplace\\
To warm the frozen swamp as best it could\\*
With the slow smokeless burning of decay.
\end{verse}

\subsection{}

\blfootnote{$\mathbb{R}$ `The Philosopher and the Lover; to a Mistress dying', Sir William Davenant (1606 -- 1668), \cite{obev}.}\settowidth{\versewidth}{    Where love is cold and beauty blind.'}
\begin{verse}[\versewidth]
Your beauty, ripe \& calm \& fresh\\*
\vin As eastern summers are,\\
Must now, forsaking time and flesh,\\*
\vin Add light to some small star.\\!

`Whilst she yet lives, were stars decayed,\\*
\vin Their light by hers relief might find;\\
But death will lead her to a shade\\*
\vin Where love is cold and beauty blind.'\\!

Lovers, whose priests all poets are,\\*
\vin Think every mistress, when she dies,\\
Is changed at least into a star:\\*
\vin And who dares doubt the poets wise?\\!

`But ask not bodies doomed to die\\*
\vin To what abode they go;\\
Since knowledge is but sorrow's spy,\\*
\vin It is not safe to know.'
\end{verse}

\subsection{}

\blfootnote{George Noel, 6th Baron Byron (1788 -- 1824), \cite{odq}. This is a line from the third canto of \refbook{Childe Harold's Pilgrimage}.}I have not loved the world, nor the world me.

\section{}

\subsection{}

\blfootnote{`The Drove-Road', Wilfrid Gibson (1878 -- 1962), \cite{oxfordlarkin}. \P 23. The term `Irish Channel' is an archaic name for the North Channel which connects the Irish Sea with the Atlantic.}\settowidth{\versewidth}{That rustling in his ears... and drifting, drifting...}
\begin{verse}[\versewidth]
'Twas going to snow. 'Twas snowing. Curse his luck.\\*
And 15 mile to travel. Here was he\\
With nothing but an empty pipe to suck,\\
And \sfrac{$1$}{$2$} a flask of rum -- but that would be\\
More welcome later on. He'd had a drink\\
Before he left, and that would keep him warm\\
A tidy while; and 'twould be good to think\\
He'd something to fall back on if the storm\\
Should come to much. You never knew with snow.\\
A sup of rain he didn't mind at all,\\
But snow was different with so far to go --\\
Full 15 mile, and not a house of call.\\
Ay, snow was quite another story, quite --\\
Snow on the fell-tops with a north-east wind\\
Behind it, blowing steadily with a bite\\
That made you feel that you were stark \& skinned.\\
And those poor beasts -- and they just off the boat\\
A day or so, and hardly used to land --\\
Still dizzy with the sea, their wits afloat.\\
When they first reached the dock they scarce could stand,\\
They'd been so joggled. It's gey bad to cross,\\
After a long day's jolting in the train,\\
Thon Irish Channel, always pitch \& toss --\\
And, heads or tails, not much for them to gain.\\
And then the market, and the throng \& noise\\
Of yapping dogs; and they stung mad with fear,\\
Welted with switches by those senseless boys --\\
He'd like to dust their jackets. But 'twas queer,\\
A beast's life, when you came to think of it,\\
From start to finish -- queerer, ay, a lot\\
Than any man's, and chancier a good bit.\\
With his ash-sapling at their heels they'd got\\
To travel before night those 15 miles\\
Of hard fell road against the driving snow,\\
Half-blinded, on \& on. He thought at whiles\\
'Twas just as well for them they couldn't know...\\
Though, as for that, 'twas little that he knew\\
Himself what was in store for him. He took\\
Things as they came: 'twas all a man could do;\\
And he'd kept going somehow by hook or crook.\\
And here he was, with 15 mile of fell,\\
And snow and... God, but it was blowing stiff.\\
And no tobacco. Blessed if he could tell\\
Where he had lost it -- but for \sfrac{$1$}{$2$} a whiff\\
He'd swop the very jacket off his back --\\
Not that he'd miss the cobweb of old shreds\\
That held the holes together. Thon cheap-jack\\
Who'd sold it him had said it was Lord \textit{Ted}'s,\\
And london-cut: but \textit{Teddy} had grown fat\\
Since he'd been made an alderman... His bid?\\
And did the gentleman not want a hat\\
To go with it, a topper? If he did,\\
Here was the very... Hell, but it was cold,\\
And driving dark it was -- nigh dark as night.\\
He'd almost think he must be getting old\\
To feel the wind so. And long out of sight\\
The beasts had trotted. Well, what odds. The way\\
Ran straight for 10 miles on, and they'd go straight:\\
They'd never heed a by-road. Many a day\\
He'd had to trudge on, trusting them to fate,\\
And always found them safe. They scamper fast,\\
But in the end a man could walk them down:\\
They're showy trotters, but they cannot last:\\
He'd race the fastest beast for half-a-crown\\
On a day's journey. Beasts were never made\\
For steady travelling -- drive them 20 mile\\
And they were done; while he was not afraid\\
To travel twice that distance with a smile.\\
But not a day like this. He'd never felt\\
A wind with such an edge. 'Twas like the blade\\
Of the rasper in the pocket of his belt\\
He kept for easy shaving. In his trade\\
You'd oft to make your toilet under a dyke --\\
And he was always one for a clean chin,\\
And carried soap. He'd never felt the like --\\
That wind, it cut clean through you to the skin.\\
He might be mother-naked, walking bare,\\
For all the use his clothes were, with the snow\\
Half blinding him and clagging to his hair\\
And trickling down his spine. He'd like to know\\
What was the sense of pegging steadily,\\
Chilled to the marrow, after a daft herd\\
Of draggled beasts he couldn't even see.\\
But that was him all over -- just a word,\\
A nod, a wink, the price of \sfrac{$1$}{$2$} \& \sfrac{$1$}{$2$},\\
And he'd be setting out for God-knows-where\\
With no more notion than a yearling calf\\
Where he would find himself when he got there.\\
And he'd been travelling hard on 60 year\\
The same old road, the same old giddy gait;\\
And he'd be walking, for a pint of beer,\\
Into his coffin one day, soon or late --\\
But not with such a tempest in his teeth,\\
Half-blinded and \sfrac{$1$}{$2$} dothered, that he hoped.\\
He'd met a sight of weather on the heath,\\
But this beat all. 'Twas worse than when he'd groped\\
His way that evening down the \textsc{Mallerstang} --\\
Thon was a blizzard, thon, and he was done\\
And almost dropping, when he came a bang\\
Against a house -- slap-bang, and like to stun.\\
Though that just saved his senses: and right there\\
He saw a lighted window he'd not seen,\\
Although he'd nearly staggered through its glare\\
Into a goodwife's kitchen, where she'd been\\
Baking hot griddle-cakes upon the peat...\\
And he could taste them now, and feel the glow\\
Of steady, aching, tingly, drowsy heat\\
As he sat there and let the caking snow\\
Melt off his boots, staining the sanded floor.\\
And that brown jug she took down from the shelf --\\
And every time he'd finished fetching more\\
And piping, Now reach up and help yourself!\\
She was a wonder, thon, the gay old wife --\\
But no such luck this journey. Things like that\\
Could hardly happen every day of life,\\
Or no one would be dying but the fat\\
And oily undertakers, starved to death\\
For want of custom... Hell, but he would soon\\
Be giving them a job... It caught your breath,\\
That throttling wind. And it was not yet noon.\\
And he'd be travelling through it until dark.\\
Dark! 'Twas already dark, and might be night\\
For all that he could see. And not a spark\\
Of comfort for him. Just to strike a light\\
And press the kindling shag down in the bowl,\\
Keeping the flame well shielded with his hand,\\
And puff \& puff! He'd give his very soul\\
For \sfrac{$1$}{$2$} a pipe. He couldn't understand\\
How he had come to lose it. He'd the rum --\\
Ay, that was safe enough, but it would keep\\
A while: you never knew what chance might come\\
In such a storm... If he could only sleep...\\
If he could only sleep... That rustling sound\\
Of drifting snow, it made him sleepy-like --\\
Drowsy \& dizzy, dithering round \& round...\\
If he could only curl up under a dyke\\
And sleep \& sleep. ... It dazzled him, that white,\\
Drifting \& drifting round \& round \& round...\\
Just \sfrac{$1$}{$2$} a moment's snooze... He'd be all right.\\
It made his head quite dizzy, that dry sound\\
Of rustling snow: it made his head go round,\\
That rustling in his ears... and drifting, drifting...\\
If only he could sleep... He would sleep sound...\\
God, he was nearly gone... The storm was lifting;\\
And he'd run into something soft \& warm --\\
Slap into his own beasts, and never knew.\\
Huddled they were, bamboozled by the storm --\\
And little wonder either when it blew\\
A blasted blizzard. Still, they'd got to go:\\
They couldn't stand there snoozing until night.\\
But they were sniffing something in the snow:\\
'Twas that had stopped them, something big \& white --\\
A bundle -- nay, a woman... And she slept --\\
But it was death to sleep. He'd nearly dropped\\
Asleep himself. 'Twas well that he had kept\\
That rum, and lucky that the beasts had stopped.\\
Ay, it was well that he had kept the rum:\\
He liked his drink, but he had never cared\\
For soaking by himself \& sitting mum:\\*
Even the best rum tasted better shared.
\end{verse}

\subsection{}

\blfootnote{`Exile', Ernest Dowson (1867 -- 1900), \cite{newlove}. The Almanacker has excised the third and fourth verses. Dowson dedicated this poem to the Anglo-Irish dramatist and novelist known as Conal Holmes O'Connell O'Riordan, who outlived him by almost a half-century.}\settowidth{\versewidth}{The sound of the waters of separation}
\begin{verse}[\versewidth]
By the sad waters of separation\\*
\vin Where we have wandered by divers ways,\\
I have but the shadow \& imitation\\*
\vin Of the old memorial days.\\!

In music I have no consolation;\\*
\vin No roses are pale enough for me;\\
The sound of the waters of separation\\*
\vin Surpasseth roses \& melody.\\!

No man knoweth our desolation;\\*
\vin Memory pales of the old delight;\\
While the sad waters of separation\\*
\vin Bear us on to the ultimate night.
\end{verse}

\subsection{}

\blfootnote{George Noel, 6th Baron Byron (1788 -- 1824), \cite{odq}. This is a line from the fourth canto of \refbook{Childe Harold's Pilgrimage}.}Of its beauty is the mind diseased.

\section{}

\subsection{}

\blfootnote{`A Frosty Night', Prof Robert Graves (1895 -- 1985), \cite{oxfordlarkin}.}\settowidth{\versewidth}{    Through green boughs of june.}
\begin{verse}[\versewidth]
\textit{Alice}, dear, what ails you,\\*
\vin Dazed \& lost \& shaken?\\
Has the chill night numbed you?\\*
\vin Is it fright you have taken?\\!

`Mother, I am very well;\\*
\vin I was never better.\\
Mother, do not hold me so,\\*
\vin Let me write my letter.'\\!

Sweet, my dear, what ails you?\\*
\vin `No, but I am well.\\
The night was cold \& frosty --\\*
\vin There's no more to tell.'\\!

Ay, the night was frosty,\\*
\vin Coldly gaped the moon,\\
Yet the birds seemed twittering\\*
\vin Through green boughs of june.\\!

`Soft \& thick the snow lay;\\*
\vin Stars danced in the sky --\\
Not all the lambs of may-day\\*
\vin Skip so bold \& high.'\\!

Your feet were dancing, \textit{Alice},\\*
\vin Seemed to dance on air,\\
You looked a ghost or angel\\*
\vin In the star-light there.\\!

Your eyes were frosted star-light;\\*
\vin Your heart, fire \& snow.\\
Who was it said, I love you?\\*
\vin `Mother, let me go!'
\end{verse}

\subsection{}

\blfootnote{Prof Robert Graves (1895 -- 1985), \cite{newlove}.}\settowidth{\versewidth}{She tells her love while half asleep,}
\begin{verse}[\versewidth]
She tells her love while \sfrac{$1$}{$2$} asleep,\\*
In the dark hours,\\
With \sfrac{$1$}{$2$}-words whispered low:\\
As earth stirs in her winter sleep\\
And puts out grass \& flowers\\
Despite the snow,\\*
Despite the falling snow.
\end{verse}

\subsection{}

\blfootnote{Gilbert Chesterton, Knight (1874 -- 1936), \cite{odq}. This is the last line of Chesterton's poem \refpoem{English Graves}.}They died to save their country and they only saved the world.

\section{}

\subsection{}

\blfootnote{$\mathbb{R}$ `After a Journey', Thomas Hardy (1840 -- 1928), \cite{oxfordlarkin}. The ghost in question is that of Hardy's first wife, Emma.}\settowidth{\versewidth}{    For the stars close their shutters the dawn whitens hazily.}
\begin{verse}[\versewidth]
Hereto I come to view a voiceless ghost;\\*
\vin Whither, O whither will its whim now draw me?\\
Up the cliff, down, till I'm lonely, lost,\\
\vin And the unseen waters' ejaculations awe me.\\
Where you will next be there's no knowing,\\
\vin Facing round about me everywhere,\\
\vin \vin With your nut-coloured hair,\\*
And grey eyes, and rose-flush coming \& going.\\!

Yes: I have re-entered your olden haunts at last;\\*
\vin Through the years, through the dead scenes I have tracked you;\\
What have you now found to say of our past --\\
\vin Scanned across the dark space wherein I have lacked you?\\
Summer gave us sweets, but autumn wrought division?\\
\vin Things were not lastly as firstly well\\
\vin \vin With us twain, you tell?\\*
But all's closed now, despite time's derision.\\!

I see what you are doing: you are leading me on\\*
\vin To the spots we knew when we haunted here together,\\
The waterfall, above which the mist-bow shone\\
\vin At the then fair hour in the then fair weather,\\
And the cave just under, with a voice still so hollow\\
\vin That it seems to call out to me from forty years ago,\\
\vin \vin When you were all aglow,\\*
And not the thin ghost that I now frailly follow.\\!

Ignorant of what there is flitting here to see,\\*
\vin The waked birds preen \& the seals flop lazily,\\
Soon you will have, dear, to vanish from me,\\
\vin For the stars close their shutters \& the dawn whitens hazily.\\
Trust me, I mind not, though life lours,\\
\vin The bringing me here; nay, bring me here again.\\
\vin \vin I am just the same as when\\*
Our days were a joy, \& our paths through flowers.
\end{verse}

\subsection{}

\blfootnote{`On the Death of Mr Richard West', Prof Thomas Gray (1716 -- 1771), \cite{norton}. Richard West is an obscure figure, the son of another Richard West, who was briefly Lord Chancellor of Ireland.}\settowidth{\versewidth}{    Or cheerful fields resume their green attire;}
\begin{verse}[\versewidth]
In vain to me the smiling mornings shine,\\*
\vin And reddening \textit{Phoebus} lifts his golden fire;\\
The birds in vain their amorous descant join;\\
\vin Or cheerful fields resume their green attire;\\
These ears -- alas! -- for other notes repine,\\
\vin A different object do these eyes require;\\
My lonely anguish melts no heart but mine;\\
\vin And in my breast the imperfect joys expire.\\
Yet morning smiles the busy race to cheer,\\
\vin And new-born pleasure brings to happier men;\\
The fields to all their wonted tribute bear;\\
\vin To warm their little loves the birds complain;\\
I fruitless mourn to him that cannot hear,\\*
\vin And weep the more because I weep in vain.
\end{verse}

\subsection{}

\blfootnote{Mrs Lydia Child (1802 -- 1880), \cite{odq}.}We first crush people to the earth, and then claim the right of trampling on them forever, because they are prostrate.

\section{}

\subsection{}

\blfootnote{Thomas Hood (1799 -- 1845), \cite{treasury}. Prof Philip Larkin's own \refpoem{I Remember, I Remember} is a sour response Hood's poem.}\settowidth{\versewidth}{To know I'm farther off from heav'n}
\begin{verse}[\versewidth]
I remember, I remember\\*
\vin The house where I was born,\\
The little window where the sun\\
\vin Came peeping in at morn;\\
He never came a wink too soon,\\
\vin Nor brought too long a day,\\
But now, I often wish the night\\*
\vin Had borne my breath away!\\!

I remember, I remember,\\*
\vin The roses, red \& white,\\
The violets, \& the lily-cups,\\
\vin Those flowers made of light!\\
The lilacs where the robin built,\\
\vin And where my brother set\\
The laburnum on his birthday --\\*
\vin The tree is living yet!\\!

I remember, I remember,\\*
\vin Where I was used to swing,\\
And thought the air must rush as fresh\\
\vin To swallows on the wing;\\
My spirit flew in feathers then,\\
\vin That is so heavy now,\\
And summer pools could hardly cool\\*
\vin The fever on my brow!\\!

I remember, I remember,\\*
\vin The fir trees dark \& high;\\
I used to think their slender tops\\
\vin Were close against the sky:\\
It was a childish ignorance,\\
\vin But now 'tis little joy\\
To know I'm farther off from heav'n\\*
\vin Than when I was a boy.
\end{verse}

\subsection{}

\blfootnote{`Former Beauties', Thomas Hardy (1840 -- 1928), \cite{pbev}. \P 7. `Froom' = the river Frome.}\settowidth{\versewidth}{These market-dames, mid-aged, with lips thin-drawn,}
\begin{verse}[\versewidth]
These market-dames, mid-aged, with lips thin-drawn,\\*
\vin \vin And tissues sere,\\
Are they the ones we loved in years agone,\\*
\vin \vin And courted here?\\!

Are these the muslined pink young things to whom\\*
\vin \vin We vowed \& swore\\
In nooks on summer sundays by the \textsc{Froom},\\*
\vin \vin Or \textsc{Budmouth} shore?\\!

Do they remember those gay tunes we trod\\*
\vin \vin Clasped on the green;\\
Aye; trod till moonlight set on the beaten sod\\*
\vin \vin A satin sheen?\\!

They must forget, forget. They cannot know\\*
\vin \vin What once they were,\\
Or memory would transfigure them, and show\\*
\vin \vin Them always fair.
\end{verse}

\subsection{}

\blfootnote{The Rev Charles Churchill (1732 -- 1764), \cite{odq}. This is a line from Churchill's \refpoem{Epistle to William Hogarth}.}No crime's so great as daring to excel.

\section{}

\subsection{}

\blfootnote{`The Eve of St Agnes', John Keats (1795 -- 1821), \cite{norton}.}\settowidth{\versewidth}{And 'tween the curtains peeped, where -- lo! -- how fast she slept.}
\begin{verse}[\versewidth]
\vin St \textit{Agnes}' Eve -- ah bitter chill it was!\\*
\vin The owl, for all his feathers, was a-cold;\\
\vin The hare limped, trembling through the frozen grass,\\
\vin And silent was the flock in woolly fold:\\
\vin Numb were the beadsman's fingers, while he told\\
\vin His rosary, and while his frosted breath,\\
\vin Like pious incense from a censer old,\\
\vin Seemed taking flight for heaven, without a death,\\*
Past the sweet Virgin's picture, while his prayer he saith.\\!

\vin His prayer he saith, this patient, holy man;\\*
\vin Then takes his lamp, and riseth from his knees,\\
\vin And back returneth, meagre, barefoot, wan,\\
\vin Along the chapel aisle by slow degrees:\\
\vin The sculptured dead, on each side, seem to freeze,\\
\vin Emprisoned in black, purgatorial rails:\\
\vin Knights, ladies, praying in dumb orat'ries,\\
\vin He passeth by; and his weak spirit fails\\*
To think how they may ache in icy hoods \& mails.\\!

\vin Northward he turneth through a little door,\\*
\vin And scarce three steps, ere music's golden tongue\\
\vin Flattered to tears this ag\`{e}d man \& poor;\\
\vin But no -- already had his deathbell rung;\\
\vin The joys of all his life were said \& sung:\\
\vin His was harsh penance on St \textit{Agnes}' Eve:\\
\vin Another way he went, and soon among\\
\vin Rough ashes sat he for his soul's reprieve,\\*
And all night kept awake, for sinners' sake to grieve.\\!

\vin That ancient beadsman heard the prelude soft;\\*
\vin And so it chanced, for many a door was wide,\\
\vin From hurry to \& fro. Soon, up aloft,\\
\vin The silver, snarling trumpets 'gan to chide:\\
\vin The level chambers, ready with their pride,\\
\vin Were glowing to receive a 1000 guests:\\
\vin The carv\`{e}d angels, ever eager-eyed,\\
\vin Star, where upon their heads the cornice rests,\\*
With hair blown back, and wings put cross-wise on their breasts.\\!

\vin At length burst in the argent revelry,\\*
\vin With plume, tiara, \& all rich array,\\
\vin Numerous as shadows haunting faerily\\
\vin The brain, new-stuffed, in youth, with triumphs gay\\
\vin Of old romance. These let us wish away,\\
\vin And turn, sole-thoughted, to one lady there,\\
\vin Whose heart had brooded, all that wintry day,\\
\vin On love, and winged St \textit{Agnes}' saintly care,\\*
As she had heard old dames full many times declare.\\!

\vin They told her how, upon St \textit{Agnes}' Eve,\\*
\vin Young virgins might have visions of delight,\\
\vin And soft adorings from their loves receive\\
\vin Upon the honeyed middle of the night,\\
\vin If ceremonies due they did aright;\\
\vin As, supperless to bed they must retire,\\
\vin And couch supine their beauties, lily white;\\
\vin Nor look behind, nor sideways, but require\\*
Of heaven with upward eyes for all that they desire.\\!

\vin Full of this whim was thoughtful \textit{Madeline}:\\*
\vin The music, yearning like a God in pain,\\
\vin She scarcely heard: her maiden eyes divine,\\
\vin Fixed on the floor, saw many a sweeping train\\
\vin Pass by -- she heeded not at all: in vain\\
\vin Came many a tiptoe, amorous cavalier,\\
\vin And back retired; not cooled by high disdain,\\
\vin But she saw not: her heart was otherwhere:\\*
She sighed for \textit{Agnes}' dreams, the sweetest of the year.\\!

\vin She danced along with vague, regardless eyes,\\*
\vin Anxious her lips, her breathing quick \& short:\\
\vin The hallowed hour was near at hand: she sighs\\
\vin Amid the timbrels, and the thronged resort\\
\vin Of whisperers in anger, or in sport;\\
\vin 'Mid looks of love, defiance, hate, \& scorn,\\
\vin Hoodwinked with faery fancy; all amort,\\
\vin Save to St \textit{Agnes} \& her lambs unshorn,\\*
And all the bliss to be before tomorrow morn.\\!

\vin So, purposing each moment to retire,\\*
\vin She lingered still. Meantime, across the moors,\\
\vin Had come young \textit{Porphyro}, with heart on fire\\
\vin For \textit{Madeline}. Beside the portal doors,\\
\vin Buttressed from moonlight, stands he, and implores\\
\vin All saints to give him sight of \textit{Madeline},\\
\vin But for one moment in the tedious hours,\\
\vin That he might gaze \& worship all unseen;\\*
Perchance speak, kneel, touch, kiss -- in sooth such things have been.\\!

\vin He ventures in: let no buzzed whisper tell:\\*
\vin All eyes be muffled, or a 100 swords\\
\vin Will storm his heart, love's fev'rous citadel:\\
\vin For him, those chambers held barbarian hordes,\\
\vin Hyena foemen, \& hot-blooded lords,\\
\vin Whose very dogs would execrations howl\\
\vin Against his lineage: not one breast affords\\
\vin Him any mercy, in that mansion foul,\\*
Save one old beldame, weak in body \& in soul.\\!

\vin Ah -- happy chance! -- the ag\`{e}d creature came,\\*
\vin Shuffling along with ivory-headed wand,\\
\vin To where he stood, hid from the torch's flame,\\
\vin Behind a broad half-pillar, far beyond\\
\vin The sound of merriment \& chorus bland:\\
\vin He startled her; but soon she knew his face,\\
\vin And grasped his fingers in her palsied hand,\\
\vin Saying, `Mercy, \textit{Porphyro}! hie thee from this place;\\*
They are all here to-night, the whole blood-thirsty race!\\!

\vin `Get hence! Get hence! There's dwarfish \textit{Hildebrand};\\*
\vin He had a fever late, and in the fit\\
\vin He curs\`{e}d thee \& thine, both house \& land:\\
\vin Then there's that old Lord Maurice, not a whit\\
\vin More tame for his gray hairs -- Alas me! Flit!\\
\vin Flit like a ghost away.' `Ah, gossip dear,\\
\vin We're safe enough; here in this armchair sit,\\
\vin And tell me how' -- `Good saints! Not here, not here;\\*
Follow me, child, or else these stones will be thy bier.'\\!

\vin He followed through a lowly arch\`{e}d way,\\*
\vin Brushing the cobwebs with his lofty plume,\\
\vin And as she muttered, `Well-a—well-a-day!'\\
\vin He found him in a little moonlight room,\\
\vin Pale, latticed, chill, \& silent as a tomb.\\
\vin `Now tell me where is \textit{Madeline},' said he,\\
\vin `O tell me, \textit{Angela}, by the holy loom\\
\vin Which none but secret sisterhood may see,\\*
When they St \textit{Agnes}' wool are weaving piously.'\\!

\vin `St \textit{Agnes}! Ah! It is St \textit{Agnes}' Eve --\\*
\vin Yet men will murder upon holy days:\\
\vin Thou must hold water in a witch's sieve,\\
\vin And be liege-lord of all the elves \& fays,\\
\vin To venture so: it fills me with amaze\\
\vin To see thee, \textit{Porphyro}! St \textit{Agnes}' Eve!\\
\vin God's help! my lady fair the conjuror plays\\
\vin This very night: good angels her deceive!\\*
But let me laugh awhile, I've mickle time to grieve.'\\!

\vin Feebly she laugheth in the languid moon,\\*
\vin While \textit{Porphyro} upon her face doth look,\\
\vin Like puzzled urchin on an ag\`{e}d crone\\
\vin Who keepeth closed a wond'rous riddle-book,\\
\vin As spectacled she sits in chimney nook.\\
\vin But soon his eyes grew brilliant, when she told\\
\vin His lady's purpose; and he scarce could brook\\
\vin Tears, at the thought of those enchantments cold,\\*
And \textit{Madeline} asleep in lap of legends old.\\!

\vin Sudden a thought came like a full-blown rose,\\*
\vin Flushing his brow, and in his pain\`{e}d heart\\
\vin Made purple riot: then doth he propose\\
\vin A stratagem, that makes the beldame start:\\
\vin `A cruel man \& impious thou art:\\
\vin Sweet lady, let her pray, and sleep, \& dream\\
\vin Alone with her good angels, far apart\\
\vin From wicked men like thee. Go! Go! I deem\\*
Thou canst not surely be the same that thou didst seem.'\\!

\vin `I will not harm her, by all saints I swear,'\\*
\vin Quoth \textit{Porphyro}: `O may I ne'er find grace\\
\vin When my weak voice shall whisper its last prayer,\\
\vin If one of her soft ringlets I displace,\\
\vin Or look with ruffian passion in her face:\\
\vin Good \textit{Angela}, believe me by these tears;\\
\vin Or I will, even in a moment's space,\\
\vin Awake, with horrid shout, my foemen's ears,\\*
And beard them, though they be more fanged than wolves \& bears.'\\!

\vin `Ah! why wilt thou affright a feeble soul?\\*
\vin A poor, weak, palsy-stricken, churchyard thing,\\
\vin Whose passing-bell may ere the midnight toll;\\
\vin Whose prayers for thee, each morn \& evening,\\
\vin Were never missed.' Thus plaining, doth she bring\\
\vin A gentler speech from burning \textit{Porphyro};\\
\vin So woful, and of such deep sorrowing,\\
\vin That \textit{Angela} gives promise she will do\\*
Whatever he shall wish, betide her weal or woe.\\!

\vin Which was, to lead him, in close secrecy,\\*
\vin Even to \textit{Madeline}'s chamber, and there hide\\
\vin Him in a closet, of such privacy\\
\vin That he might see her beauty unespyed,\\
\vin And win perhaps that night a peerless bride,\\
\vin While legioned faeries paced the coverlet,\\
\vin And pale enchantment held her sleepy-eyed.\\
\vin Never on such a night have lovers met,\\*
Since \textit{Merlin} paid his demon all the monstrous debt.\\!

\vin `It shall be as thou wishest,' said the dame:\\*
\vin `All cates \& dainties shall be stor\`{e}d there\\
\vin Quickly on this feast-night: by the tambour frame\\
\vin Her own lute thou wilt see: no time to spare,\\
\vin For I am slow \& feeble, and scarce dare\\
\vin On such a catering trust my dizzy head.\\
\vin Wait here, my child, with patience; kneel in prayer\\
\vin The while: Ah! thou must needs the lady wed,\\*
Or may I never leave my grave among the dead.'\\!

\vin So saying, she hobbled off with busy fear.\\*
\vin The lover's endless minutes slowly passed;\\
\vin The dame returned, and whispered in his ear\\
\vin To follow her; with ag\`{e}d eyes aghast\\
\vin From fright of dim espial. Safe at last,\\
\vin Through many a dusky gallery, they gain\\
\vin The maiden's chamber, silken, hushed, \& chaste;\\
\vin Where \textit{Porphyro} took covert, pleased amain.\\*
His poor guide hurried back with agues in her brain.\\!

\vin Her falt'ring hand upon the balustrade,\\*
\vin Old \textit{Angela} was feeling for the stair,\\
\vin When \textit{Madeline}, St \textit{Agnes}' charm\`{e}d maid,\\
\vin Rose, like a missioned spirit, unaware:\\
\vin With silver taper's light, \& pious care,\\
\vin She turned, and down the ag\`{e}d gossip led\\
\vin To a safe level matting. Now prepare,\\
\vin Young \textit{Porphyro}, for gazing on that bed;\\*
She comes, she comes again, like ring-dove frayed \& fled.\\!

\vin Out went the taper as she hurried in;\\*
\vin Its little smoke, in pallid moonshine, died:\\
\vin She closed the door, she panted, all akin\\
\vin To spirits of the air, \& visions wide:\\
\vin No uttered syllable, or, woe betide!\\
\vin But to her heart, her heart was voluble,\\
\vin Paining with eloquence her balmy side;\\
\vin As though a tongueless nightingale should swell\\*
Her throat in vain, and die, heart-stifled, in her dell.\\!

\vin A casement high \& triple-arched there was,\\*
\vin All garlanded with carven imag'ries\\
\vin Of fruits, \& flowers, \& bunches of knot-grass,\\
\vin And diamonded with panes of quaint device,\\
\vin Innumerable of stains \& splendid dyes,\\
\vin As are the tiger-moth's deep-damasked wings;\\
\vin And in the midst, 'mong 1000 heraldries,\\
\vin And twilight saints, \& dim emblazonings,\\*
A shielded scutcheon blushed with blood of queens \& kings.\\!

\vin Full on this casement shone the wintry moon,\\*
\vin And threw warm gules on \textit{Madeline}'s fair breast,\\
\vin As down she knelt for heaven's grace \& boon;\\
\vin Rose-bloom fell on her hands, together prest,\\
\vin And on her silver cross soft amethyst,\\
\vin And on her hair a glory, like a saint:\\
\vin She seemed a splendid angel, newly dressed,\\
\vin Save wings, for heaven: \textit{Porphyro} grew faint:\\*
She knelt, so pure a thing, so free from mortal taint.\\!

\vin Anon his heart revives: her vespers done,\\*
\vin Of all its wreath\`{e}d pearls her hair she frees;\\
\vin Unclasps her warm\`{e}d jewels one by one;\\
\vin Loosens her fragrant boddice; by degrees\\
\vin Her rich attire creeps rustling to her knees:\\
\vin Half-hidden, like a mermaid in sea-weed,\\
\vin Pensive awhile she dreams awake, and sees,\\
\vin In fancy, fair St Agnes in her bed,\\*
But dares not look behind, or all the charm is fled.\\!

\vin Soon, trembling in her soft \& chilly nest,\\*
\vin In sort of wakeful swoon, perplexed she lay,\\
\vin Until the poppied warmth of sleep oppressed\\
\vin Her sooth\`{e}d limbs, \& soul fatigued away;\\
\vin Flown, like a thought, until the morrow-day;\\
\vin Blissfully havened both from joy \& pain;\\
\vin Clasped like a missal where swart paynims pray;\\
\vin Blinded alike from sunshine \& from rain,\\*
As though a rose should shut, and be a bud again.\\!

\vin Stol'n to this paradise, and so entranced,\\*
\vin \textit{Porphyro} gazed upon her empty dress,\\
\vin And listened to her breathing, if it chanced\\
\vin To wake into a slumberous tenderness;\\
\vin Which when he heard, that minute did he bless,\\
\vin And breathed himself: then from the closet crept,\\
\vin Noiseless as fear in a wide wilderness,\\
\vin And over the hushed carpet, silent, stept,\\*
And 'tween the curtains peeped, where -- lo! -- how fast she slept.\\!

\vin Then by the bed-side, where the faded moon\\*
\vin Made a dim, silver twilight, soft he set\\
\vin A table, and, \sfrac{$1$}{$2$} anguished, threw thereon\\
\vin A cloth of woven crimson, gold, \& jet:\\
\vin O for some drowsy morphean amulet!\\
\vin The boisterous, midnight, festive clarion,\\
\vin The kettle-drum, \& far-heard clarinet,\\
\vin Affray his ears, though but in dying tone:\\*
The hall-door shuts again, and all the noise is gone.\\!

\vin And still she slept an azure-lidded sleep,\\*
\vin In blanch\`{e}d linen, smooth, \& lavendered,\\
\vin While he forth from the closet brought a heap\\
\vin Of candied apple, quince, \& plum, \& gourd;\\
\vin With jellies soother than the creamy curd,\\
\vin And lucent syrops, tinct with cinnamon;\\
\vin Manna \& dates, in argosy transferred\\
\vin From \textsc{Fez}; \& spic\`{e}d dainties, every one,\\*
From silken \textsc{Samarcand} to cedared Lebanon.\\!

\vin These delicates he heaped with glowing hand\\*
\vin On golden dishes \& in baskets bright\\
\vin Of wreath\`{e}d silver: sumptuous they stand\\
\vin In the retir\`{e}d quiet of the night,\\
\vin Filling the chilly room with perfume light.\\
\vin `And now, my love, my seraph fair, awake!\\
\vin Thou art my heaven, and I thine eremite:\\
\vin Open thine eyes, for meek St \textit{Agnes}' sake,\\*
Or I shall drowse beside thee, so my soul doth ache.'\\!

\vin Thus whispering, his warm, unnerv\`{e}d arm\\*
\vin Sank in her pillow. Shaded was her dream\\
\vin By the dusk curtains:—'twas a midnight charm\\
\vin Impossible to melt as ic\`{e}d stream:\\
\vin The lustrous salvers in the moonlight gleam;\\
\vin Broad golden fringe upon the carpet lies:\\
\vin It seemed he never, never could redeem\\
\vin From such a stedfast spell his lady's eyes;\\*
So mused awhile, entoiled in woof\`{e}d phantasies.\\!

\vin Awakening up, he took her hollow lute,\\*
\vin Tumultuous, and, in chords that tenderest be,\\
\vin He played an ancient ditty, long since mute,\\
\vin In Provence called, ``La belle dame sans mercy'':\\
\vin Close to her ear touching the melody;\\
\vin Wherewith disturbed, she uttered a soft moan:\\
\vin He ceased -- she panted quick -- and suddenly\\
\vin Her blue affray\`{e}d eyes wide open shone:\\*
Upon his knees he sank, pale as smooth-sculptured stone.\\!

\vin Her eyes were open, but she still beheld,\\*
\vin Now wide awake, the vision of her sleep:\\
\vin There was a painful change, that nigh expelled\\
\vin The blisses of her dream so pure \& deep\\
\vin At which fair \textit{Madeline} began to weep,\\
\vin And moan forth witless words with many a sigh;\\
\vin While still her gaze on \textit{Porphyro} would keep;\\
\vin Who knelt, with join\`{e}d hands \& piteous eye,\\*
Fearing to move or speak, she looked so dreamingly.\\!

\vin `Ah, \textit{Porphyro}!' said she, `But even now\\*
\vin Thy voice was at sweet tremble in mine ear,\\
\vin Made tuneable with every sweetest vow;\\
\vin And those sad eyes were spiritual \& clear:\\
\vin How changed thou art! how pallid, chill, \& drear!\\
\vin Give me that voice again, my \textit{Porphyro},\\
\vin Those looks immortal, those complainings dear!\\
\vin O leave me not in this eternal woe,\\*
For if thy diest, my love, I know not where to go.'\\!

\vin Beyond a mortal man impassioned far\\*
\vin At these voluptuous accents, he arose\\
\vin Ethereal, flushed, and like a throbbing star\\
\vin Seen 'mid the sapphire heaven's deep repose;\\
\vin Into her dream he melted, as the rose\\
\vin Blendeth its odour with the violet --\\
\vin Solution sweet: meantime the frost-wind blows\\
\vin Like love's alarum pattering the sharp sleet\\*
Against the window-panes; St \textit{Agnes}' moon hath set.\\!

\vin 'Tis dark: quick pattereth the flaw-blown sleet:\\*
\vin `This is no dream, my bride, my \textit{Madeline}!'\\
\vin 'Tis dark: the ic\`{e}d gusts still rave \& beat:\\
\vin `No dream, Alas! Alas! And woe is mine!\\
\vin \textit{Porphyro} will leave me here to fade \& pine.\\
\vin Cruel! what traitor could thee hither bring?\\
\vin I curse not, for my heart is lost in thine,\\
\vin Though thou forsakest a deceiv\`{e}d thing;\\*
A dove forlorn \& lost with sick unprun\`{e}d wing.'\\!

\vin `My \textit{Madeline}! sweet dreamer! lovely bride!\\*
\vin Say, may I be for aye thy vassal blest?\\
\vin Thy beauty's shield, heart-shaped \& vermeil-dyed?\\
\vin Ah silver shrine, here will I take my rest\\
\vin After so many hours of toil \& quest,\\
\vin A famished pilgrim, saved by miracle.\\
\vin Though I have found, I will not rob thy nest\\
\vin Saving of thy sweet self; if thou think'st well\\*
To trust, fair \textit{Madeline}, to no rude infidel.\\!

\vin `Hark! 'Tis an elfin-storm from faery land,\\*
\vin Of haggard seeming, but a boon indeed:\\
\vin Arise! Arise! the morning is at hand;\\
\vin The bloated wassaillers will never heed:\\
\vin Let us away, my love, with happy speed;\\
\vin There are no ears to hear, or eyes to see,\\
\vin Drowned all in rhenish \& the sleepy mead:\\
\vin Awake! Arise, my love, and fearless be,\\*
For o'er the southern moors I have a home for thee.'\\!

\vin She hurried at his words, beset with fears,\\*
\vin For there were sleeping dragons all around,\\
\vin At glaring watch, perhaps, with ready spears,\\
\vin Down the wide stairs a darkling way they found.\\
\vin In all the house was heard no human sound.\\
\vin A chain-drooped lamp was flickering by each door;\\
\vin The arras, rich with horseman, hawk, \& hound,\\
\vin Fluttered in the besieging wind's uproar;\\*
And the long carpets rose along the gusty floor.\\!

\vin They glide, like phantoms, into the wide hall;\\*
\vin Like phantoms, to the iron porch, they glide;\\
\vin Where lay the porter, in uneasy sprawl,\\
\vin With a huge empty flaggon by his side:\\
\vin The wakeful bloodhound rose, and shook his hide,\\
\vin But his sagacious eye an inmate owns:\\
\vin By one, \& one, the bolts full easy slide:\\
\vin The chains lie silent on the footworn stones;\\*
The key turns, and the door upon its hinges groans.\\!

\vin And they are gone: ay, ages long ago\\*
\vin These lovers fled away into the storm.\\
\vin That night the baron dreamt of many a woe,\\
\vin And all his warrior-guests, with shade \& form\\
\vin Of witch, \& demon, \& large coffin-worm,\\
\vin Were long be-nightmared. \textit{Angela} the old\\
\vin Died palsy-twitched, with meagre face deform;\\
\vin The beadsman, after 1000 aves told,\\*
For aye unsought for slept among his ashes cold.
\end{verse}

\subsection{}

\blfootnote{Prof Alfred Housman (1859 -- 1936), \cite{ptmgmc}.}\settowidth{\versewidth}{How hopeless under ground}
\begin{verse}[\versewidth]
Ensanguining the skies\\*
How heavily it dies\\
\vin Into the west away;\\
Past touch \& sight \& sound,\\
Not further to be found,\\
How hopeless under ground\\*
\vin Falls the remorseful day.
\end{verse}

\subsection{}

\blfootnote{John Clare (1793 -- 1864), \cite{odq}.}\settowidth{\versewidth}{The present is the funeral of the past,}
\begin{verse}[\versewidth]
The present is the funeral of the past,\\*
And man the living sepulchre of life.
\end{verse}

\section{}

\subsection{}

\blfootnote{`On an Infant Dying as Soon as Born', Charles Lamb (1775 -- 1834), \cite{treasury}.}\settowidth{\versewidth}{(With her nine moons' long workings sickened)}
\begin{verse}[\versewidth]
I saw wherein the shroud did lurk\\*
A curious frame of nature's work.\\
A flow'ret crush\`{e}d in the bud,\\
A nameless piece of babyhood,\\
Was in a cradle-coffin lying;\\
Extinct, with scarce the sense of dying;\\
So soon to exchange the imprisoning womb\\
For darker closets of the tomb!\\
She did but ope an eye, and put\\
A clear beam forth, then strait up shut\\
For the long dark: ne'er more to see\\
Through glasses of mortality.\\
Riddle of destiny, who can show\\
What thy short visit meant, or know\\
What thy errand here below?\\
Shall we say, that nature blind\\
Checked her hand, \& changed her mind,\\
Just when she had exactly wrought\\
A finished pattern without fault?\\
Could she flag, or could she tire,\\
Or lacked she the promethean fire\\
(With her nine moons' long workings sickened)\\
That should thy little limbs have quickened?\\
Limbs so firm, they seemed to assure\\
Life of health, and days mature:\\
Woman's self in miniature!\\
Limbs so fair, they might supply\\
(Themselves now but cold imagery)\\
The sculptor to make beauty by.\\
Or did the stern-eyed fate descry,\\
That babe, or mother, one must die;\\
So in mercy left the stock,\\
And cut the branch; to save the shock\\
Of young years widowed; and the pain,\\
When single state comes back again\\
To the lone man who, 'reft of wife,\\
Thenceforward drags a maimed life?\\
The economy of heaven is dark;\\
And wisest clerks have missed the mark,\\
Why human buds, like this, should fall,\\
More brief than fly ephemeral,\\
That has his day; while shrivelled crones\\
Stiffen with age to stocks \& stones;\\
And crabb\`{e}d use the conscience sears\\
In sinners of an hundred years.\\
Mother's prattle, mother's kiss,\\
Baby fond, thou ne'er wilt miss.\\
Rites, which custom does impose,\\
Silver bells \& baby clothes;\\
Coral redder than those lips,\\
Which pale death did late eclipse;\\
Music framed for infants' glee,\\
Whistle never tuned for thee;\\
Though thou want'st not, thou shalt have them,\\
Loving hearts were they which gave them.\\
Let not one be missing; nurse,\\
See them laid upon the hearse\\
Of infant slain by doom perverse.\\
Why should kings \& nobles have\\
Pictured trophies to their grave;\\
And we, churls, to thee deny\\
Thy pretty toys with thee to lie,\\*
A more harmless vanity?
\end{verse}

\subsection{}

\blfootnote{Prof Alfred Housman (1859 -- 1936), \cite{oxfordlarkin}.}\settowidth{\versewidth}{Where there's neither heat nor cold.}
\begin{verse}[\versewidth]
Others -- I am not the first --\\*
Have willed more mischief than they durst:\\
If in the breathless night I too\\*
Shiver now, 'tis nothing new.\\!

More than I, if truth were told,\\*
Have stood and sweated hot \& cold,\\
And through their reins in ice \& fire\\*
Fear contended with desire.\\!

Agued once like me were they,\\*
But I like them shall win my way\\
Lastly to the bed of mould\\*
Where there's neither heat nor cold.\\!

But from my grave across my brow\\*
Plays no wind of healing now,\\
And fire \& ice within me fight\\*
Beneath the suffocating night.
\end{verse}

\subsection{}

\blfootnote{William Congreve (1670 -- 1729), \cite{odq}.}\settowidth{\versewidth}{Heaven has no rage, like love to hatred turned;}
\begin{verse}[\versewidth]
Heaven has no rage, like love to hatred turned;\\*
Nor hell a fury, like a woman scorned.
\end{verse}

\section{}

\subsection{}

\blfootnote{Walter Landor (1775 -- 1864), \cite{londonbook}. These are lines 271-93 of Landor's \refpoem{Helen and Corythos}. The Helen in question is the same woman whose beauty sparked the Trojan War. Corythos -- the name is usually Latinised as Corythus, from the Greek Κόρυθος -- was the son of Paris and Oenone; his mother sent the young man to Troy, where he and Helen fell in love with each other; Paris, not recognising his own son, killed him out of jealousy.}\settowidth{\versewidth}{And fondness he refreshed: her anxious thoughts}
\begin{verse}[\versewidth]
Her failing spirits with derisive glee\\*
And fondness he refreshed: her anxious thoughts\\
Followed, and upon \textit{Corythos} they dwelt.\\
Often he met her eyes, nor shunned they his.\\
For, royal as she was and born of \textit{Zeus},\\
She was compassionate, and bowed her head\\
To share her smiles \& griefs with those below.\\
All in her sight were level, for she stood\\
High above all within the sea-girt world.\\
At last she questioned \textit{Corythos} what brought\\
His early footsteps through such dangerous ways.\\
And from abode so peaceable \& safe.\\
At once he told her why he came: she held\\
Her hand to \textit{Corythos}: he stood ashamed\\
Not to have hated her: he looked; he sighed.\\
He hung upon her words. What gentle words!\\
How chaste her countenance.\\
\textcolor{white}{How chaste her countenance.} What open brows\\
The brave \& beauteous ever have! thought she,\\
But even the hardiest, when above their heads\\
Death is impending, shudder at the sight\\
Of barrows on the sands and bones exposed\\
And whitening in the wind, and cypresses\\*
From \textsc{Ida} waiting for dissevered friends.
\end{verse}

\subsection{}

\blfootnote{$\mathbb{R}$ `What News', Walter Landor (1775 -- 1864), \cite{newlove}.}\settowidth{\versewidth}{    If there be change, no change I see;}
\begin{verse}[\versewidth]
Here, ever since you went abroad,\\*
\vin If there be change, no change I see;\\
I only walk our wonted road;\\*
\vin The road is only walked by me.\\!

Yes; I forgot; a change there is;\\*
\vin Was it of that you bade me tell?\\
I catch at times, at times I miss\\*
\vin The sight, the tone, I know so well.\\!

Only two months since you stood here!\\*
\vin Two shortest months! Then tell me why\\
Voices are harsher than they were,\\*
\vin And tears are longer ere they dry.
\end{verse}

\subsection{}

\blfootnote{Captain Joseph Conrad (1857 -- 1924), \cite{odq}.}The terrorist and the policeman both come from the same basket.

\section{}

\subsection{}

\blfootnote{$\mathbb{R}$ Prof Henry Longfellow (1807 -- 1882), \cite{norton}.}\settowidth{\versewidth}{Seize them, and whirl them aloft, and sprinkle them far o'er the ocean.}
\begin{verse}[\versewidth]
This is the forest primeval. The murmuring pines \& the hemlocks,\\*
Bearded with moss, and in garments green, indistinct in the twilight,\\
Stand like druids of old, with voices sad \& prophetic,\\
Stand like harpers hoar, with beards that rest on their bosoms\\
Loud from its rocky caverns, the deep-voiced neighboring ocean\\*
Speaks, and in accents disconsolate answers the wail of the forest.\\!

This is the forest primeval; but where are the hearts that beneath it\\*
Leaped like the roe, when he hears in the woodland the voice of the huntsman?\\
Where is the thatch-roofed village, the home of acadian farmers,\\
Men whose lives glided on like rivers that water the woodlands,\\
Darkened by shadows of earth, but reflecting an image of heaven?\\
Waste are those pleasant farms, and the farmers forever departed.\\
Scattered like dust \& leaves, when the mighty blasts of october\\
Seize them, and whirl them aloft, and sprinkle them far o'er the ocean.\\*
Naught but tradition remains of the beautiful village of \textsc{Grand-Przz}.
\end{verse}

\subsection{}

\blfootnote{`Shadows', David Lawrence (1885 -- 1930), \cite{oxfordlarkin}. \P 31. Prof Larkin places the `of' in this line in square brackets, presumably because of its absence in some manuscript from which the text is ultimately drawn.}\settowidth{\versewidth}{And the silence of short days, the silence of the year, the shadow,}
\begin{verse}[\versewidth]
And if tonight my soul may find her peace\\*
In sleep, and sink in good oblivion,\\
And in the morning wake like a new-opened flower\\*
Then I have been dipped again in God, and new-created.\\!

And if, as weeks go round, in the dark of the moon\\*
My spirit darkens and goes out, and soft strange gloom\\
Pervades my movements \& my thoughts \& words\\
Then I shall know that I am walking still\\*
With God, we are close together now the moon's in shadow.\\!

And if, as autumn deepens \& darkens\\*
I feel the pain of falling leaves, and stems that break in storms\\
And trouble \& dissolution \& distress\\
And then the softness of deep shadows folding,\\
Folding around my soul \& spirit, around my lips\\
So sweet, like a swoon, or more like the drowse of a low, sad song\\
Singing darker than the nightingale, on, on to the solstice\\
And the silence of short days, the silence of the year, the shadow,\\
Then I shall know that my life is moving still\\
With the dark earth, and drenched\\*
With the deep oblivion of earth's lapse \& renewal.\\!

And if, in the changing phases of man's life,\\*
I fall in sickness \& in misery;\\
My wrists seem broken and my heart seems dead\\
And strength is gone, and my life\\*
Is only the leavings of a life:\\!

And still, among it all, snatches of lovely oblivion, and snatches\\*
Of renewal\\
Odd, wintry flowers upon the withered stem, yet new, strange flowers\\*
Such as my life has not brought forth before, new blossoms of me\\!

Then I must know that still\\*
I am in the hands of the unknown God;\\
He is breaking me down to his own oblivion\\*
To send me forth on a new morning, a new man.
\end{verse}

\subsection{}

\blfootnote{Clarence Darrow (1857 -- 1938), \cite{odq}. From a speech concerning`the Negro race'.}The law has made him equal, but man has not.

\section{}

\subsection{}

\blfootnote{`Inseparable', Philip Marston (1850 -- 1887), \cite{newlove}. Other sources give the first line as `When I and thou are dead, my dearest'.}\settowidth{\versewidth}{Yet one thought is, I deem, more kind,}
\begin{verse}[\versewidth]
When thou \& I are dead, my dear,\\*
\vin The earth above us lain,\\
When we no more in autumn hear\\
\vin The fall of leaves \& rain,\\
Or round the snow-enshrouded year\\*
\vin The midnight winds complain;\\!

When we no more in green mid-spring,\\*
\vin Its sights \& sounds may mind;\\
The warm wet leaves set quivering\\
\vin With touches of the wind,\\
The birds at morn, \& birds that sing\\*
\vin When day is left behind;\\!

When over all the moonlight lies,\\*
\vin Intensely bright \& still;\\
When some meandering brooklet sighs,\\
\vin At parting from its hill;\\
And scents from voiceless gardens rise,\\*
\vin The peaceful air to fill;\\!

When we no more through summer light\\*
\vin The deep, dim woods discern,\\
Nor hear the nightingales at night,\\
\vin In vehement singing, yearn\\
To stars \& moon, that, dumb \& bright,\\*
\vin In nightly vigil burn;\\!

When smiles, \& hopes, \& joys, and fears,\\*
\vin And words that lovers say,\\
And sighs of love, \& passionate tears\\
\vin Are lost to us for aye,\\
What thing of all our love appears,\\*
\vin In cold \& coffined clay?\\!

When all their kisses, sweet \& close,\\*
\vin Our lips shall quite forget;\\
When, where the day upon us rose,\\
\vin The day shall rise \& set,\\
While we for love's sublime repose\\*
\vin Shall have not one regret;\\!

O this true comfort is, I think,\\*
\vin That, be death near or far,\\
When we have crossed the fatal brink,\\
\vin And found nor moon nor star --\\
To know not, when in death we sink,\\*
\vin The lifeless things we are.\\!

Yet one thought is, I deem, more kind,\\*
\vin That when we sleep so well,\\
On memories that we leave behind,\\
\vin When kindred spirits dwell,\\
My name to thine in words they'll bind\\*
\vin Of love inseparable.
\end{verse}

\subsection{}

\blfootnote{`The Bride', David Lawrence (1885 -- 1930), \cite{oxfordlarkin}.}\settowidth{\versewidth}{Nay, but she sleeps like a bride, and dreams her dreams}
\begin{verse}[\versewidth]
My love looks like a girl tonight,\\*
\vin But she is old.\\
The plaits that lie along her pillow\\
\vin Are not gold,\\
But threaded with filigree silver\\*
\vin And uncanny cold.\\!

She looks like a young maiden, since her brow\\*
\vin Is smooth \& fair;\\
Her cheeks are very smooth; her eyes are closed.\\
\vin She sleeps a rare\\*
Still winsome sleep, so still, and so composed.\\!

Nay, but she sleeps like a bride, and dreams her dreams\\*
\vin Of perfect things.\\
She lies at last, the darling, in the shape of her dream.\\
\vin And her dead mouth sings\\*
By its shape, like the thrushes in clear evenings.
\end{verse}

\subsection{}

\blfootnote{The Rev Charles Dodgson (1832 -- 1898), \cite{odq}.}It takes all the running you can do, to keep in the same place; if you want to get somewhere else, you must run at least twice as fast as that.

\section{}

\subsection{}

\blfootnote{Sir William McGonagall (1825 -- 1902), \cite{mcgonagall}. William McGonagall (his knighthood would seem to have been self-bestowed; but where's the harm in that?) is often said to be the worst poet in the English language, and this his worst poem. Yet the Almanacker cannot help but discern a particular kind of genius in his works, rarely seen outside of the writings of Joseph Smith and L Ron Hubbard. The disaster described was indeed a genuine tragedy, and remains the most lethal British railway disaster to this day.}\settowidth{\versewidth}{Because none of the passengers were saved to tell the tale}
\begin{verse}[\versewidth]
Beautiful \textsc{Railway Bridge} of the silvery \textsc{Tay}!\\*
Alas! I am very sorry to say\\
That 90 lives have been taken away\\
On the last sabbath day of eighteen seventy-nine,\\*
Which will be remembered for a very long time.\\!

'Twas about seven o'clock at night,\\*
And the wind it blew with all its might,\\
And the rain came pouring down,\\
And the dark clouds seemed to frown,\\
And the demon of the air seemed to say,\\*
I'll blow down the \textsc{Bridge of Tay}.\\!

When the train left \textsc{Edinburgh}\\*
The passengers' hearts were light \& felt no sorrow,\\
But \textit{Boreas} blew a terrific gale,\\
Which made their hearts for to quail,\\
And many of the passengers with fear did say,\\*
I hope God will send us safe across the \textsc{Bridge of Tay}.\\!

But when the train came near to \textsc{Wormit Bay},\\*
\textit{Boreas} he did loud \& angry bray,\\
And shook the central girders of the \textsc{Bridge of Tay}\\
On the last sabbath day of eighteen seventy-nine,\\*
Which will be remembered for a very long time.\\!

So the train sped on with all its might,\\*
And bonny \textsc{Dundee} soon hove in sight,\\
And the passengers' hearts felt light,\\
Thinking they would enjoy themselves on the New Year,\\
With their friends at home they loved most dear,\\*
And wish them all a happy New Year.\\!

So the train moved slowly along the \textsc{Bridge of Tay},\\*
Until it was about midway,\\
Then the central girders with a crash gave way,\\
And down went the train \& passengers into the \textsc{Tay}!\\
The storm fiend did loudly bray,\\
Because 90 lives had been taken away,\\
On the last sabbath day of eighteen seventy-nine,\\*
Which will be remembered for a very long time.\\!

As soon as the catastrophe came to be known\\*
The alarm from mouth to mouth was blown,\\
And the cry rang out all o'er the town:\\
Good Heavens! The \textsc{Tay Bridge} is blown down,\\
And a passenger train from Edinburgh,\\
Which filled all the people's hearts with sorrow,\\
And made them for to turn pale,\\
Because none of the passengers were saved to tell the tale\\
How the disaster happened on the last sabbath day of eighteen seventy-nine,\\*
Which will be remembered for a very long time.\\!

It must have been an awful sight,\\*
To witness in the dusky moonlight,\\
While the storm fiend did laugh, and angry did bray,\\
Along the \textsc{Railway Bridge} of the silvery \textsc{Tay}.\\
O ill-fated \textsc{Bridge} of the silvery \textsc{Tay},\\
I must now conclude my lay\\
By telling the world fearlessly without the least dismay,\\
That your central girders would not have given way,\\
At least many sensible men do say,\\
Had they been supported on each side with buttresses,\\
At least many sensible men confesses,\\
For the stronger we our houses do build,\\*
The less chance we have of being killed.
\end{verse}

\subsection{}

\blfootnote{`Napoleon', Walter de la Mare (1873 -- 1956), \cite{obev}. This poem presumably concerns Napoleon's disastrous invasion of Russia.}\settowidth{\versewidth}{What is the world, O soldiers?}
\begin{verse}[\versewidth]
What is the world, O soldiers?\\*
\vin It is I:\\
I, this incessant snow,\\
\vin This northern sky;\\
Soldiers, this solitude\\
Through which we go\\*
\vin Is I.
\end{verse}

\subsection{}

\blfootnote{The Rev Charles Dodgson (1832 -- 1898), \cite{odq}.}Jam tomorrow and jam yesterday -- but never jam today.

\section{}

\subsection{}

\blfootnote{`The Farmer's Bride', Miss Charlotte Mew (1869 -- 1928), \cite{norton}. Miss Mew never married, and was fond of wearing male clothing.}\settowidth{\versewidth}{        One night, in the fall, she runned away.}
\begin{verse}[\versewidth]
\vin Three summers since I chose a maid,\\*
\vin Too young maybe -- but more's to do\\
\vin At harvest-time than bide \& woo.\\
\vin \vin When us was wed she turned afraid\\
\vin Of love \& me \& all things human;\\
\vin Like the shut of a winter's day\\
\vin Her smile went out, and 'twasn't a woman --\\
\vin More like a little frightened fay.\\*
\vin \vin One night, in the fall, she runned away.\\!

\vin Out 'mong the sheep, her be, they said;\\*
\vin 'Should properly have been abed;\\
\vin But sure enough she wasn't there\\
\vin Lying awake with her wide brown stare.\\
So over seven-acre field \& up-along across the down\\
\vin We chased her, flying like a hare\\
\vin Before out lanterns. To \textsc{Church-Town}\\
\vin \vin All in a shiver \& a scare\\
\vin We caught her, fetched her home at last\\*
\vin \vin And turned the key upon her, fast.\\!

\vin She does the work about the house\\*
\vin As well as most, but like a mouse:\\
\vin Happy enough to chat \& play\\
\vin \vin With birds \& rabbits \& such as they,\\
\vin \vin So long as men-folk keep away.\\
\vin Not near, not near! her eyes beseech\\
\vin When one of us comes within reach.\\
\vin \vin The women say that beasts in stall\\
\vin \vin Look round like children at her call.\\*
\vin \vin I've hardly heard her speak at all.\\!

\vin Shy as a leveret, swift as he,\\*
\vin Straight \& slight as a young larch tree,\\
\vin Sweet as the first wild violets, she,\\*
\vin To her wild self. But what to me?\\!

\vin The short days shorten \& the oaks are brown;\\*
\vin \vin The blue smoke rises to the low grey sky;\\
\vin One leaf in the still air falls slowly down;\\
\vin \vin A magpie's spotted feathers lie\\
\vin On the black earth spread white with rime;\\
\vin The berries redden up to Christmas-time.\\
\vin \vin What's Christmas-time without there be\\*
\vin \vin Some other in the house than we!\\!

\vin \vin She sleeps up in the attic there\\*
\vin \vin Alone, poor maid. 'Tis but a stair\\
\vin Betwixt us. Oh! my God! the down,\\
\vin The soft young down of her, the brown,\\*
The brown of her -- her eyes, her hair, her hair!
\end{verse}

\subsection{}

\blfootnote{Dr John McCrae (1872 -- 1918), \cite{norton}. The argument of the poem -- that the living should give their lives to avenge the dead -- is clearly stupid. Where would the killing end before the whole world was sacrificed to this quasi-religion of military honour? And indeed the First World War provided a kind of answer to that question. But it remains a fine poem, and was popular with the ordinary soldiers of that most terrible of wars.}\settowidth{\versewidth}{    The torch; be yours to hold it high.}
\begin{verse}[\versewidth]
In Flanders fields the poppies grow\\*
Between the crosses, row on row,\\
\vin That mark our place; and in the sky\\
\vin The larks, still bravely singing, fly\\*
Scarce heard amid the guns below.\\!

We are the dead. Short days ago\\*
We lived, felt dawn, saw sunset glow,\\
\vin Loved \& were loved, and now we lie\\*
\vin \vin \vin In Flanders fields.\\!

Take up our quarrel with the foe:\\*
To you from failing hands we throw\\
\vin The torch; be yours to hold it high.\\
\vin If ye break faith with us who die\\
We shall not sleep, though poppies grow\\*
\vin \vin \vin In Flanders fields.
\end{verse}

\subsection{}

\blfootnote{Michael Drayton (1563 -- 1631), \cite{odq}. These words can be found in canto 2 of \refbook{The Baron's Wars}.}Comfort's a cripple.

\section{}

\subsection{}

\blfootnote{$\mathbb{R}$ `On the Welsh Language', Mrs Katherine Philips (1632 -- 1664), \cite{norton}.}\settowidth{\versewidth}{Where's Athens now, to whom Rome learning owes,}
\begin{verse}[\versewidth]
If honour to an ancient name be due,\\*
Or riches challenge it for one that's new,\\
The british language claims in either sense\\
Both for its age, and for its opulence.\\
But all great things must be from us removed,\\
To be with higher reverence beloved.\\
So landscapes which in prospects distant lie,\\
With greater wonder draw the pleas\`{e}d eye.\\
Is not great \textsc{Troy} to one dark ruin hurled?\\
Once the famed scene of all fighting world.\\
Where's \textsc{Athens} now, to whom \textsc{Rome} learning owes,\\
And the safe laurels that adorned her brows?\\
A strange reverse of fate she did endure,\\
Never once greater, than she's now obscure.\\
Even \textsc{Rome} herself can but some footsteps show\\
Of \textit{Scipio}'s times, or those of \textit{Cicero}.\\
And as the roman \& the grecian state,\\
The british fell, the spoil of time \& fate.\\
But though the language hath the beauty lost,\\
Yet she has still some great remains to boast.\\
For 'twas in that, the sacred bards of old,\\
In deathless numbers did their thoughts unfold.\\
In groves, by rivers, and on fertile plains,\\
They civilized \& taught the listening swains;\\
Whilst with high raptures, and as great success,\\
Virtue they clothed in music's charming dress.\\
This \textit{Merlin} spoke, who in his gloomy cave,\\
Even destiny herself seemed to enslave.\\
For to his sight the future time was known,\\
Much better than to others is their own;\\
And with such state, predictions from him fell,\\
As if he did decree, and not foretell.\\
This spoke King \textit{Arthur}, who, if fame be true,\\
Could have compelled mankind to speak it too.\\
In this once \textit{Boadicca} valour taught,\\
And spoke more nobly than her soldiers fought:\\
Tell me what hero could be more than she,\\
Who fell at once for fame \& liberty?\\
Nor could a greater sacrifice belong,\\
Or to her children's, or her country's wrong.\\
This spoke \textit{Caractacus}, who was so brave,\\
That to the roman fortune check he gave:\\
And when their yoke he could decline no more,\\
He it so decently \& nobly wore,\\
That \textsc{Rome} herself with blushes did believe,\\
A briton would the law of honour give;\\
And hastily his chains away she threw,\\*
Lest her own captive else should her subdue.
\end{verse}

\subsection{}

\blfootnote{`Shadwell Stair', Wilfred Owen (1893 -- 1918), \cite{faber20th}. Shadwell Stair is an obscure alleyway leading down to the Thames, in Rotherhithe, a district on the outskirts of London proper. The place is said to have been, in Owen's time, one of those spots where gay men pick each other up for sexual encounters; although it has to be countered that the same could be said of a great many locations.}\settowidth{\versewidth}{    And dawn creeps up the Shadwell Stair.}
\begin{verse}[\versewidth]
I am the ghost of \textsc{Shadwell Stair}.\\*
\vin Along the wharves by the waterhouse,\\
\vin And through the cavernous slaughterhouse,\\*
I am the shadow that walks there.\\!

Yet I have flesh both firm \& cool,\\*
\vin And eyes tumultuous as the gems\\
\vin Of moons \& lamps in the full \textsc{Thames}\\*
When dusk sails wavering down the pool.\\!

Shuddering the purple street arc burns\\*
\vin Where I watch always; from the banks\\
\vin Dolorously the shipping clanks\\*
And after me a strange tide turns.\\!

I walk till the stars of \textsc{London} wane\\*
\vin And dawn creeps up the \textsc{Shadwell Stair}.\\
\vin But when the crowing sirens blare\\*
I with another ghost am lain.
\end{verse}

\subsection{}

\blfootnote{John Dryden, Poet Laureate (1631 -- 1700), \cite{odq}. This is a line from the epilogue to \refbook{Constantine the Great}.}Good men starve for want of impudence.

\section{}

\subsection{}

\blfootnote{$\mathbb{R}$ Alexander Pope (1688 -- 1744), \cite{pbev}. These are lines 37-60 of Pope's longer poem, \refpoem{Windsor-Forest}.}\settowidth{\versewidth}{See from the brake the whirring pheasant springs,}
\begin{verse}[\versewidth]
See from the brake the whirring pheasant springs,\\*
And mounts exulting on triumphant wings;\\
Short is his joy. He feels the fiery wound,\\
Flutters in blood, and panting beats the ground.\\
Ah what avail his glossy, varying dyes,\\
His purple crest, \& scarlet-circled eyes,\\
The vivid green his shining plumes unfold;\\
His painted wings, \& breast that flames with gold?\\
Nor yet, when moist \textit{Arcturus} clouds the sky,\\
The woods \& fields their pleasing toils deny.\\
To plains with well-breathed beagles we repair,\\
And trace the mazes of the circling hare.\\
(Beasts, urged by us, their fellow beasts pursue,\\
And learn of man each other to undo.)\\
With slaughtering guns th'unwearied fowler roves,\\
When frosts have whitened all the naked groves;\\
Where doves in flocks the leafless trees o'ershade,\\
And lonely woodcocks haunt the watery glade.\\
He lifts the tube, and levels with his eye;\\
Strait a short thunder breaks the frozen sky.\\
Oft, as in airy rings they skim the heath,\\
The clamorous lapwings feel the leaden death:\\
Oft as the mounting larks their notes prepare,\\*
They fall, and leave their little lives in air.
\end{verse}

\subsection{}

\blfootnote{Gabriel Rossetti (1828 -- 1882), \cite{newlove}. This sonnet is from Rossetti's sequence, \refbook{The House of Life}.}\settowidth{\versewidth}{    Which, brought together, would find loving voice;}
\begin{verse}[\versewidth]
Two separate divided silences,\\*
\vin Which, brought together, would find loving voice;\\
\vin Two glances which together would rejoice\\
In love, now lost like stars beyond dark trees;\\
Two hands apart whose touch alone gives ease;\\
\vin Two bosoms which, heart-shrined with mutual flame,\\
\vin Would, meeting in one clasp, be made the same;\\
Two souls, the shore wave-mocked of sundering seas:\\
Such are we now. Ah may our hope forecast\\
\vin Indeed one hour again, when on this stream\\
\vin Of darkened love once more the light shall gleam?\\
An hour how slow to come -- how quickly past --\\
Which blooms \& fades, and only leaves at last,\\*
\vin Faint as shed flowers, the attenuated dream.
\end{verse}

\subsection{}

\blfootnote{John Dryden, Poet Laureate (1631 -- 1700), \cite{obev}. This is from line 35 of Dryden's \refbook{Sixth Satyr of Juvenal}.}Whores and silver in one age were born.

\section{}

\subsection{}

\blfootnote{`For a Dead Lady', Edwin Robinson (1869 -- 1935), \cite{pbev}.}\settowidth{\versewidth}{    Shall fill the eyes that now are faded,}
\begin{verse}[\versewidth]
No more with overflowing light\\*
\vin Shall fill the eyes that now are faded,\\
Nor shall another's fringe with night\\
\vin Their woman-hidden world as they did.\\
\vin \vin No more shall quiver down the days\\
\vin \vin The flowing wonder of her ways,\\
Whereof no language may requite\\*
\vin The shifting \& the many-shaded.\\!

The grace, divine, definitive,\\*
\vin Clings only as a faint forestalling;\\
The laugh that love could not forgive\\
\vin Is hushed, and answers to no calling;\\
\vin \vin The forehead \& the little ears\\
\vin \vin Have gone where \textit{Saturn} keeps the years;\\
The breast where roses could not live\\*
\vin Has done with rising and with falling.\\!

The beauty, shattered by the laws\\*
\vin That have creation in their keeping,\\
No longer trembles at applause,\\
\vin Or over children that are sleeping;\\
\vin \vin And we who delve in beauty's lore\\
\vin \vin Know all that we have known before\\
Of what inexorable cause\\*
\vin Makes time so vicious in his reaping.
\end{verse}

\subsection{}

\blfootnote{`The General', Siegfried Sassoon (1886 -- 1967), \cite{oxfordlarkin}. Arras was the site of a battle between the British and German Empires in 1917, which resulted in some three hundred thousand casualties.}\settowidth{\versewidth}{And we're cursing his staff for incompetent swine.}
\begin{verse}[\versewidth]
Good morning! Good morning! the General said\\*
When we met him last week on our way to the line.\\
Now the soldiers he smiled at are most of 'em dead,\\
And we're cursing his staff for incompetent swine.\\
He's a cheery old card, grunted \textit{Harry} to \textit{Jack}\\*
As they slogged up to \textsc{Arras} with rifle \& pack.\\!

But he did for them both by his plan of attack.
\end{verse}

\subsection{}

\blfootnote{The Rt Hon Edmund Burke (1729 -- 1797), \cite{odq}.}A nation is not to be governed, which is perpetually to be conquered.

\section{}

\subsection{}

\blfootnote{`The Dying Man in His Garden', Dr George Sewell (1687 -- 1726), \cite{treasury}.}\settowidth{\versewidth}{The bean-flower's deep-embosomed sweet}
\begin{verse}[\versewidth]
Why, \textit{Damon}, with the forward day\\*
Dost thou thy little spot survey,\\
From tree to tree, with doubtful cheer,\\
Pursue the progress of the year,\\
What winds arise, what rains descend,\\*
When thou before that year shalt end?\\!

What do thy noontide walks avail,\\*
To clear the leaf, \& pick the snail,\\
Then wantonly to death decree\\
An insect usefuller than thee?\\
Thou \& the worm are brother-kind,\\*
As low, as earthy, \& as blind.\\!

Vain wretch! canst thou expect to see\\*
The downy peach make court to thee?\\
Or that thy sense shall ever meet\\
The bean-flower's deep-embosomed sweet\\
Exhaling with an evening blast?\\*
Thy evenings then will all be past!\\!

Thy narrow pride, thy fancied green\\*
(For vanity's in little seen),\\
All must be left when death appears,\\
In spite of wishes, groans, \& tears;\\
Nor one of all thy plants that grow\\*
But rosemary will with thee go.
\end{verse}

\subsection{}

\blfootnote{`The Pride of Youth', Sir Walter Scott, 1st Baronet (1771 -- 1832), \cite{treasury}.}\settowidth{\versewidth}{The owl from the steeple sing,}
\begin{verse}[\versewidth]
Proud \textit{Maisie} is in the wood,\\*
\vin Walking so early;\\
Sweet robin sits on the bush,\\*
\vin Singing so rarely.\\!

Tell me, thou bonny bird,\\*
\vin When shall I marry me?\\
When six braw gentlemen\\*
\vin Kirk-ward shall carry ye.\\!

Who makes the bridal bed,\\*
\vin Birdie, say truly?\\
The gray-headed sexton\\*
\vin That delves the grave duly.\\!

The glowworm o'er grave \& stone\\*
\vin Shall light thee steady;\\
The owl from the steeple sing,\\*
\vin Welcome, proud lady.
\end{verse}

\subsection{}

\blfootnote{The Rt Hon Edmund Burke (1729 -- 1797), \cite{odq}.}It is a general popular error to imagine that the loudest complainers for the public to be the most anxious for its welfare.

\section{}

\subsection{}

\blfootnote{`A Dream of the Unknown', Percy Shelley (1792 -- 1822), \cite{treasury}.}\settowidth{\versewidth}{The sod scarce heaved; and that tall flower that wets --}
\begin{verse}[\versewidth]
I dreamed that, as I wandered by the way,\\*
\vin Bare winter suddenly was changed to spring,\\
And gentle odours led my steps astray,\\
\vin Mixed with a sound of waters murmuring\\
Along a shelving bank of turf, which lay\\
\vin Under a copse, and hardly dared to fling\\
Its green arms round the bosom of the stream,\\*
But kissed it and then fled, as thou mightest in dream.\\!

There grew pied wind-flowers \& violets,\\*
\vin Daisies, those pearled Arcturi of the earth,\\
The constellated flower that never sets;\\
\vin Faint oxlips; tender bluebells, at whose birth\\
The sod scarce heaved; and that tall flower that wets --\\
\vin Like a child, \sfrac{$1$}{$2$} in tenderness \& mirth --\\
Its mother's face with heaven's collected tears,\\*
When the low wind, its playmate's voice, it hears.\\!

And in the warm hedge grew lush eglantine,\\*
\vin Green cowbind \& the moonlight-coloured may,\\
And cherry-blossoms, and white cups, whose wine\\
\vin Was the bright dew, yet drained not by the day;\\
And wild roses, and ivy serpentine,\\
\vin With its dark buds \& leaves, wandering astray;\\
And flowers azure, black, \& streaked with gold,\\*
Fairer than any wakened eyes behold.\\!

And nearer to the river's trembling edge\\*
\vin There grew broad flag-flowers, purple pranked with white,\\
And starry river buds among the sedge,\\
\vin And floating water-lilies, broad \& bright,\\
Which lit the oak that overhung the hedge\\
\vin With moonlight beams of their own watery light;\\
And bulrushes, and reeds of such deep green\\*
As soothed the dazzled eye with sober sheen.\\!

Methought that of these visionary flowers\\*
\vin I made a nosegay, bound in such a way\\
That the same hues, which in their natural bowers\\
\vin Were mingled or opposed, the like array\\
Kept these imprisoned children of the hours\\
\vin Within my hand -- and then, elate \& gay,\\
I hastened to the spot whence I had come,\\*
That I might there present it! O! to whom?
\end{verse}

\subsection{}

\blfootnote{$\mathbb{R}$ William Shakespeare (1564 -- 1616), \cite{obev}. This song is sung in \refbook{Love's Labours Lost} V.2. \P 2. Blowing one's nail means breathing on one's hands to warm them up. \P 11. The parson's saw is more likely his sermon than an implement for cutting wood; likewise the crabs hissing in the bowl are more likely crab apples than sea creatures.}\settowidth{\versewidth}{    And milk comes frozen home in pail,}
\begin{verse}[\versewidth]
When icicles hang by the wall,\\*
\vin And \textit{Dick} the shepherd blows his nail,\\
And \textit{Tom} bears logs into the hall,\\
\vin And milk comes frozen home in pail,\\
When blood is nipped and ways be foul,\\
Then nightly sings the staring owl,\\
\vin \vin Tu-whit;\\
\vin \vin Tu-who, a merry note,\\*
While greasy \textit{Joan} doth keel the pot.\\!

When all aloud the wind doth blow,\\*
\vin And coughing drowns the parson's saw,\\
And birds sit brooding in the snow,\\
\vin And \textit{Marian}'s nose looks red \& raw,\\
When roasted crabs hiss in the bowl,\\
Then nightly sings the staring owl,\\
\vin \vin Tu-whit;\\
\vin \vin Tu-who, a merry note,\\*
While greasy \textit{Joan} doth keel the pot.
\end{verse}

\subsection{}

\blfootnote{The Rt Hon Edmund Burke (1729 -- 1797), \cite{odq}.}Kings will be tyrants from policy when subjects are rebels from principle.

\section{}

\subsection{}

\blfootnote{$\mathbb{R}$ Henry Howard, Earl of Surrey (1517 -- 1547), \cite{pbev}. These are verses 254-259 of Dryden's longer poem \refbook{Annus Mirabilis}.}\settowidth{\versewidth}{    To a last lodging call their wandering friends:}
\begin{verse}[\versewidth]
Night came, but without darkness or repose,\\*
\vin A dismal picture of the general doom;\\
Where souls distracted when the trumpet blows\\*
\vin And half unready with their bodies come.\\!

Those who have homes, when home they do repair,\\*
\vin To a last lodging call their wandering friends:\\
Their short uneasy sleeps are broke with care,\\*
\vin To look how near their own destruction tends.\\!

Those who have none, sit round where once it was,\\*
\vin And with full eyes each wonted room require:\\
Haunting the yet warm ashes of the place,\\*
\vin As murdered men walk where they did expire.\\!

Some stir up coals, and watch the vestal fire,\\*
\vin Others in vain from sight of ruin run;\\
And while through burning labyrinths they retire,\\*
\vin With loathing eyes repeat what they would shun.\\!

The most in fields like herded beasts lie down,\\*
\vin To dews obnoxious on the grassy floor;\\
And while their babes in sleep their sorrows drown,\\*
\vin Sad parents watch the remnants of their store.\\!

While by the motion of the flames they guess\\*
\vin What streets are burning now, and what are near,\\
An infant waking to the paps would press,\\*
\vin And meets, instead of milk, a falling tear.
\end{verse}

\subsection{}

\blfootnote{$\mathbb{R}$ `A Complaint by Night of the Lover not Beloved', Sir Philip Sidney (1554 -- 1586), \cite{pbev}.}\settowidth{\versewidth}{    Who seeketh heav'n, and comes of heav'nly breath.}
\begin{verse}[\versewidth]
Leave me, O love, which reachest but to dust;\\*
\vin And thou, my mind, aspire to higher things;\\
Grow rich in that which never taketh rust;\\
\vin Whatever fades but fading pleasure brings.\\
Draw in thy beams and humble all thy might\\
\vin To that sweet yoke where lasting freedoms be;\\
Which breaks the clouds and opens forth the light,\\
\vin That doth both shine and give us sight to see.\\
O take fast hold; let that light be thy guide\\
\vin In this small course which birth draws out to death,\\
And think how evil becometh him to slide,\\
\vin Who seeketh heav'n, and comes of heav'nly breath.\\
Then farewell, world; thy uttermost I see:\\*
Eternal love, maintain thy life in me.
\end{verse}

\subsection{}

\blfootnote{The Rt Hon Edmund Burke (1729 -- 1797), \cite{odq}.}There is, however, a limit at which forebearance ceases to be a virtue.

\section{}

\subsection{}

\blfootnote{Alfred Tennyson, 1st Baron Tennyson, Poet Laureate (1809 -- 1892), \cite{norton}.}\settowidth{\versewidth}{    That sweeps with all its autumn bowers,}
\begin{verse}[\versewidth]
Calm is the morn without a sound,\\*
\vin Calm as to suit a calmer grief,\\
\vin And only through the faded leaf,\\*
The chestnut pattering to the ground:\\!

Calm \& deep peace on this high wold,\\*
\vin And on these dews that drench the furze.\\
\vin And all the silvery gossamers\\*
That twinkle into green \& gold:\\!

Calm \& still light on yon great plain\\*
\vin That sweeps with all its autumn bowers,\\
\vin And crowded farms \& lessening towers,\\*
To mingle with the bounding main:\\!

Calm \& deep peace in this wide air,\\*
\vin These leaves that redden to the fall;\\
\vin And in my heart, if calm at all,\\*
If any calm, a calm despair:\\!

Calm on the seas, \& silver sleep,\\*
\vin And waves that sway themselves in rest,\\
\vin And dead calm in that noble breast\\*
Which heaves but with the heaving deep.
\end{verse}

\subsection{}

\blfootnote{Alfred Tennyson, 1st Baron Tennyson, Poet Laureate (1809 -- 1892), \cite{norton}.}\settowidth{\versewidth}{On the bald street breaks the blank day.}
\begin{verse}[\versewidth]
Dark house, by which once more I stand\\*
\vin Here in the long unlovely street,\\
\vin Doors, where my heart was used to beat\\*
So quickly, waiting for a hand,\\!

A hand that can be clasped no more --\\*
\vin Behold me, for I cannot sleep,\\
\vin And like a guilty thing I creep\\*
At earliest morning to the door.\\!

He is not here; but far away\\*
\vin The noise of life begins again,\\
\vin And ghastly through the drizzling rain\\*
On the bald street breaks the blank day.
\end{verse}

\subsection{}

\blfootnote{Alexander Pope (1688 -- 1744), \cite{obev}. This is line 215 of Pope's \refbook{Essay on Criticism}.}A little learning is a dangerous thing.

\section{}

\subsection{}

\blfootnote{Alfred Tennyson, 1st Baron Tennyson, Poet Laureate (1809 -- 1892), \cite{norton}.}\settowidth{\versewidth}{The casement slowly grows a glimmering square;}
\begin{verse}[\versewidth]
Tears, idle tears, I know not what they mean;\\*
Tears from the depth of some divine despair\\
Rise in the heart, and gather to the eyes,\\
In looking on the happy autumn fields,\\*
And thinking of the days that are no more.\\!

Fresh as the first beam glittering on a sail,\\*
That brings our friends up from the underworld,\\
Sad as the last which reddens over one\\
That sinks with all we love below the verge;\\*
So sad, so fresh, the days that are no more.\\!

Ah sad \& strange as in dark summer dawns\\*
The earliest pipe of \sfrac{$1$}{$2$} awakened birds\\
To dying ears, when unto dying eyes\\
The casement slowly grows a glimmering square;\\*
So sad, so strange, the days that are no more.\\!

Dear as remembered kisses after death,\\*
And sweet as those by hopeless fancy feigned\\
On lips that are for others; deep as love,\\
Deep as first love, and wild with all regret;\\*
O death in life, the days that are no more.
\end{verse}

\subsection{}

\blfootnote{Edward Thomas (1878 -- 1917), \cite{obev}.}\settowidth{\versewidth}{Arrives, and all else is drowned;}
\begin{verse}[\versewidth]
Out in the dark over the snow\\*
The fallow fawns invisible go\\
With the fallow doe;\\
And the winds blow\\*
Fast as the stars are slow.\\!

Stealthily the dark haunts round\\*
And, when the lamp goes, without sound\\
At a swifter bound\\
Than the swiftest hound,\\*
Arrives, and all else is drowned;\\!

And star \& I \& wind \& deer\\*
Are in the dark together -- near,\\
Yet far -- and fear\\
Drums on my ear\\*
In that sage company drear.\\!

How weak \& little is the light,\\*
All the universe of sight,\\
Love \& delight,\\
Before the might,\\*
If you love it not, of night.
\end{verse}

\subsection{}

\blfootnote{Oscar Wilde (1854 -- 1900), \cite{odq}. The original, in Wilde's preface to his \refbook{Portrait of Dorian Gray}, begins, `The nineteenth century dislike of realism...'}Dislike of realism is the rage of Caliban seeing his own face in the glass.

\section{}

\subsection{}

\blfootnote{Alfred Tennyson, 1st Baron Tennyson, Poet Laureate (1809 -- 1892), \cite{norton}.}\settowidth{\versewidth}{    The darkened heart that beat no more;}
\begin{verse}[\versewidth]
The \textsc{Danube} to the \textsc{Severn} gave\\*
\vin The darkened heart that beat no more;\\
\vin They laid him by the pleasant shore,\\*
And in the hearing of the wave.\\!

There twice a day the \textsc{Severn} fills;\\*
\vin The salt sea-water passes by,\\
\vin And hushes half the babbling \textsc{Wye},\\*
And makes a silence in the hills.\\!

The \textsc{Wye} is hushed nor moved along,\\*
\vin And hushed my deepest grief of all,\\
\vin When filled with tears that cannot fall,\\*
I brim with sorrow drowning song.\\!

The tide flows down, the wave again\\*
\vin Is vocal in its wooded walls;\\
\vin My deeper anguish also falls,\\*
And I can speak a little then.
\end{verse}

\subsection{}

\blfootnote{Dr William Wordsworth, Poet Laureate (1770 -- 1850), \cite{treasury}.}\settowidth{\versewidth}{No motion has she now, no force;}
\begin{verse}[\versewidth]
A slumber did my spirit seal;\\*
\vin I had no human fears:\\
She seemed a thing that could not feel\\*
\vin The touch of earthly years.\\!

No motion has she now, no force;\\*
\vin She neither hears nor sees;\\
Rolled round in earth's diurnal course\\*
\vin With rocks, \& stones, \& trees.
\end{verse}

\subsection{}

\blfootnote{Oscar Wilde (1854 -- 1900), \cite{odq}. This is a line from \refbook{The Ballad of Reading Gaol}.}Each man kills the thing he loves.

\section{}

\subsection{}

\blfootnote{James Thomson (1700 -- 1748), \cite{pbev}. These lines are from poet's longer poem \refpoem{Winter}, which seems to exist in several versions.}\settowidth{\versewidth}{Descends th'ethereal force, and ploughs its waves,}
\begin{verse}[\versewidth]
Late in the lowering sky red fiery streaks\\*
Begin to flush about; the reeling clouds\\
Stagger with dizzy aim, as doubting yet\\
Which master to obey: while rising, slow,\\
Sad, in the leaden-coloured east, the moon\\
Wears a black circle round her sullied orb.\\
Then issues forth the storm, with loud control,\\
And the thin fabrick of the pillared air\\
O'erturns, at once. Prone, on th'uncertain main,\\
Descends th'ethereal force, and ploughs its waves,\\
With dreadful rift: from the mid-deep appears\\
Surge after surge, the rising watery war.\\
Whitening, the angry billows rowl immense,\\
And roar their rerrors, through the shuddering soul\\
Of feeble man, amidst their fury caught,\\
And dashed upon his fate: then, o'er the cliff,\\
Where dwells the sea-mew, unconfined, they fly,\\*
And, hurrying, swallow up the sterile shore.\\!

The mountain growls; and all its sturdy sons\\*
Stoop to the bottom of the rocks they shade:\\
Lone on its midnight side, and all aghast,\\
The dark wayfaring stranger, breathless, toils\\
And climbs against the blast --\\
Low waves the rooted forest, vexed, and sheds\\
What of its leafy honours yet remains.\\
Thus, struggling through the dissipated grove,\\
The whirling tempest raves along the plain;\\
And, on the cottage thatched, or lordly dome,\\
Keen-fastening, shakes 'em to the solid base.\\
Sleep, frighted, flies; the hollow chimney howls,\\*
The windows rattle, and the hinges creak.
\end{verse}

\subsection{}

\blfootnote{`Mutability', Dr William Wordsworth, Poet Laureate (1770 -- 1850), \cite{norton}.}\settowidth{\versewidth}{    Of awful notes, whose concord shall not fail;}
\begin{verse}[\versewidth]
From low to high doth dissolution climb,\\*
\vin And sink from high to low, along a scale\\
\vin Of awful notes, whose concord shall not fail;\\
A musical but melancholy chime,\\
Which they can hear who meddle not with crime,\\
\vin Nor avarice, nor over-anxious care.\\
\vin Truth fails not; but her outward forms that bear\\
The longest date do melt like frosty rime,\\
That in the morning whitened hill \& plain\\
\vin And is no more; drop like the tower sublime\\
\vin \vin Of yesterday, which royally did wear\\
His crown of weeds, but could not even sustain\\
\vin \vin Some casual shout that broke the silent air,\\*
\vin Or the unimaginable touch of time.
\end{verse}

\subsection{}

\blfootnote{Oscar Wilde (1854 -- 1900), \cite{criticasartist}.}It is well for his peace that the saint goes to his martyrdom; he is spared the sight of the horror of his harvest.

\chapter{Unodecember}

\section{}

\subsection{}

\blfootnote{$\mathbb{R}$ `The Wife A-Lost', The Rev William Barnes (1801 -- 1886), \cite{newlove}.}\settowidth{\versewidth}{Since I do miss your voice an' face}
\begin{verse}[\versewidth]
Since I no more do see your face\\*
\vin Up stairs or down below,\\
I'll sit me in the lonesome place\\
\vin Where flat-boughed beech do grow;\\
Below the beeches' bough, my love,\\
\vin Where you did never come,\\
An' I don't look to meet ye now\\*
\vin As I do look at home.\\!

Since you no more be at my side\\*
\vin In walks in summer het\\
I'll go alone where mist do ride,\\
\vin Through trees a-drippin' wet;\\
Below the rain-wet bough, my love,\\
\vin Where you did never come,\\
An' I don't grieve to miss ye now\\*
\vin As I do grieve at home.\\!

Since now beside my dinner-board\\*
\vin Your voice do never sound,\\
I'll eat the bit I can afford,\\
\vin A-yield upon the ground;\\
Below the darksome bough, my love,\\
\vin Where you did never dine,\\
An' I don't grieve to miss ye now\\*
\vin As I at home do pine.\\!

Since I do miss your voice an' face\\*
\vin In prayer at eventide,\\
I'll pray wi' one sad voice for grace\\
\vin To go where you do bide;\\
Above the tree an' bough, my love,\\
\vin Where you be gone afore,\\
An' be a waitin' for me now\\*
\vin To come for evermore.
\end{verse}

\subsection{}

\blfootnote{$\mathbb{R}$ `Oh! Cruel', Anonymous, \cite{bod8227}. This is the first verse in a ballad; sadly the others are much inferior. The `Oh!' has been excised from the front of the first line, and the `And' has been excised from the front of the middle two lines.}\settowidth{\versewidth}{Cruel was the little ship that rowed him off the strand,}
\begin{verse}[\versewidth]
Cruel were my parents to tear my love from me.\\*
Cruel were the press-gang that took him off to sea.\\
Cruel was the little ship that rowed him off the strand,\\*
And cruel was the great big ship that sailed from the land.
\end{verse}

\subsection{}

\blfootnote{2 Samuel 22.22, \cite{kjv}. King David says this of his dead son, the first child Bathsheba bore him.}I shall go to him but he shall not return to me.

\section{}

\subsection{}

\blfootnote{`For the Fallen', Prof Laurence Binyon (1869 -- 1943), \cite{oxfordlarkin}.}\settowidth{\versewidth}{To the innermost heart of their own land they are known}
\begin{verse}[\versewidth]
With proud thanksgiving, a mother for her children,\\*
\vin England mourns for her dead across the sea.\\
Flesh of her flesh they were, spirit of her spirit,\\*
\vin Fallen in the cause of the free.\\!

Solemn the drums thrill; death august \& royal\\*
\vin Sings sorrow up into immortal spheres,\\
There is music in the midst of desolation\\*
\vin And a glory that shines upon our tears.\\!

They went with songs to the battle; they were young,\\*
\vin Straight of limb, true of eye, steady \& aglow.\\
They were staunch to the end against odds uncounted;\\*
\vin They fell with their faces to the foe.\\!

They shall grow not old, as we that are left grow old:\\*
\vin Age shall not weary them, nor the years condemn.\\
At the going down of the sun \& in the morning\\*
\vin We will remember them.\\!

They mingle not with their laughing comrades again;\\*
\vin They sit no more at familiar tables of home;\\
They have no lot in our labour of the day-time;\\*
\vin They sleep beyond England's foam.\\!

But where our desires are \& our hopes profound,\\*
\vin Felt as a well-spring that is hidden from sight,\\
To the innermost heart of their own land they are known\\*
\vin As the stars are known to the night;\\!

As the stars that shall be bright when we are dust,\\*
\vin Moving in marches upon the heavenly plain;\\
As the stars that are starry in the time of our darkness,\\*
\vin To the end, to the end, they remain.
\end{verse}

\subsection{}

\blfootnote{`Out of Work', Kenneth Ashley (1885 -- 1969), \cite{ptmgmc}.}\settowidth{\versewidth}{I'd tramped the length breadth of the fen;}
\begin{verse}[\versewidth]
Alone at the shut of day was I,\\*
With a star or two in a frost-clear sky,\\
\vin And the byre smell in the air.\\
I'd tramped the length \& breadth of the fen;\\
But never a farmer wanted men;\\*
\vin Naught doing anywhere.\\!

A great calm moon rose back of the mill,\\*
And I told myself it was God's will\\
\vin Who went hungry and who went fed.\\
I tried to whistle; I tried to be brave;\\
But the new-ploughed fields smelt dank as the grave;\\*
\vin And I wished I were dead.
\end{verse}

\subsection{}

\blfootnote{Matthew 27.46, \cite{kjv}. Christ is quoting here the first line of Psalm 22. The Aramaic, following the KJV's transliteration, is `Eli, Eli, lama sabachthani?'.}{\color{red} My God, my God, why hast thou forsaken me?}

\section{}

\subsection{}

\blfootnote{`Before the Birth of One of Her Children', Mrs Anne Bradstreet (1612 -- 1672), \cite{norton}.}\settowidth{\versewidth}{Who with salt tears this last farewell did take.}
\begin{verse}[\versewidth]
All things within this fading world hath end;\\*
Adversity doth still our joys attend;\\
No ties so strong, no friends so dear \& sweet,\\
But with death's parting blow is sure to meet.\\
The sentence past is most irrevocable,\\
A common thing, yet O inevitable.\\
How soon, my dear, death may my steps attend\\
How soon't may be thy lot to lose thy friend,\\
We are both ignorant, yet love bids me\\
These farewell lines to recommend to thee,\\
That when that knot's untied that made us one,\\
I may seem thine, who in effect am none.\\
And if I see not \sfrac{$1$}{$2$} my days that's due,\\
What nature would, God grant to yours \& you;\\
The many faults that well you know I have\\
Let be interred in my oblivious grave;\\
If any worth or virtue were in me,\\
Let that live freshly in thy memory\\
And when thou feel'st no grief, as I no harms,\\
Yet love thy dead, who long lay in thine arms.\\
And when thy loss shall be repaid with gains,\\
Look to my little babes, my dear remains.\\
And if thou love thyself, or loved'st me,\\
These O protect from step-dames' injury.\\
And if chance to thine eyes shall bring this verse,\\
With some sad sighs honour my absent hearse;\\
And kiss this paper for thy loves dear sake,\\*
Who with salt tears this last farewell did take.
\end{verse}

\subsection{}

\blfootnote{`Hunger', Prof Laurence Binyon (1869 -- 1943), \cite{oxfordlarkin}. Prof Binyon was clearly inspired by the Old English riddles, such as are found in the \refbook{Exeter Book}. The Almanacker has excised the last two lines, since these give the game away.}\settowidth{\versewidth}{None sees me, but they look on one another,}
\begin{verse}[\versewidth]
I come among the peoples like a shadow.\\*
I sit down by each man's side.\\!

None sees me, but they look on one another,\\*
And know that I am there.\\!

My silence is like the silence of the tide\\*
That buries the playground of children;\\!

Like the deepening of frost in the slow night,\\*
When birds are dead in the morning.\\!

Armies trample, invade, destroy,\\*
With guns roaring from earth \& air.\\!

I am more terrible than armies;\\*
I am more feared than the cannon.\\!

Kings and chancellors give commands;\\*
I give no command to any;\\!

But I am listened to more than kings\\*
And more than passionate orators.\\!

I unswear words, and undo deeds.\\*
Naked things know me.
\end{verse}

\subsection{}

\blfootnote{Jeremiah 8.20, \cite{kjv}.}Summer is ended, and we are not saved.

\section{}

\subsection{}

\blfootnote{George Noel, 6th Baron Byron (1788 -- 1824), \cite{norton}.}\settowidth{\versewidth}{    Sunk chill on my brow --}
\begin{verse}[\versewidth]
When we two parted\\*
\vin In silence \& tears,\\
Half broken-hearted\\
\vin To sever for years,\\
Pale grew thy cheek \& cold,\\
\vin Colder thy kiss;\\
Truly that hour foretold\\*
\vin Sorrow to this.\\!

The dew of the morning\\*
\vin Sunk chill on my brow --\\
It felt like the warning\\
\vin Of what I feel now.\\
Thy vows are all broken,\\
\vin And light is thy fame;\\
I hear thy name spoken,\\*
\vin And share in its shame.\\!

They name thee before me,\\*
\vin A knell to mine ear;\\
A shudder comes o'er me --\\
\vin Why wert thou so dear?\\
They know not I knew thee,\\
\vin Who knew thee too well --\\
Long, long shall I rue thee,\\*
\vin Too deeply to tell.\\!

In secret we met --\\*
\vin In silence I grieve,\\
That thy heart could forget,\\
\vin Thy spirit deceive.\\
If I should meet thee\\
\vin After long years,\\
How should I greet thee?\\*
\vin With silence \& tears.
\end{verse}

\subsection{}

\blfootnote{`To the Countess of Blessington', George Noel, 6th Baron Byron (1788 -- 1824), \cite{byron}. The Almanacker has excised the first two verses.}\settowidth{\versewidth}{    The string which was worthy the strain.}
\begin{verse}[\versewidth]
I am ashes where once I was fire,\\*
\vin And the bard in my bosom is dead;\\
What I loved I now merely admire,\\*
\vin And my heart is as grey as my head.\\!

My life is not dated by years --\\*
\vin There are moments which act as a plough;\\
And there is not a furrow appears\\*
\vin But is deep in my soul as my brow.\\!

Let the young \& the brilliant aspire\\*
\vin To sing what I gaze on in vain;\\
For sorrow has torn from my lyre\\*
\vin The string which was worthy the strain.
\end{verse}

\subsection{}

\blfootnote{Matthew 3.10, \cite{kjv}. The KJV's rendering of Matthew 3.10 in full: `And now also the axe is laid unto the root of the trees: therefore every tree which bringeth not forth good fruit is hewn down, and cast into the fire.'}The axe is laid unto the root of the trees.

\section{}

\subsection{}

\blfootnote{Dr Thomas Campion (1567 -- 1620), \cite{norton}.}\settowidth{\versewidth}{Though love all his pleasures are but toys,}
\begin{verse}[\versewidth]
\vin Now winter nights enlarge\\*
\vin \vin The number of their hours;\\
\vin And clouds their storms discharge\\
\vin \vin Upon the airy towers.\\
\vin Let now the chimneys blaze\\
\vin \vin And cups o'erflow with wine;\\
\vin Let well-turned words amaze\\
\vin With harmony divine.\\
\vin Now yellow waxen lights\\
\vin \vin Shall wait on honey love\\
While youthful revels, masques \& courtly sights\\*
\vin \vin Sleep's leaden spells remove.\\!

\vin This time doth well dispense\\*
\vin \vin With lovers' long discourse;\\
\vin Much speech hath some defense,\\
\vin \vin Though beauty no remorse.\\
\vin All do not all things well;\\
\vin \vin Some measures comely tread,\\
\vin Some knotted riddles tell,\\
\vin \vin Some poems smoothly read.\\
\vin The summer hath his joys,\\
\vin \vin And winter his delights;\\
Though love \& all his pleasures are but toys,\\*
\vin \vin They shorten tedious nights.
\end{verse}

\subsection{}

\blfootnote{George Noel, 6th Baron Byron (1788 -- 1824), \cite{norton}. Here Lord Byron is codifying, in a manner not dissimilar to Burns, an ancient English folk song, known in one of its variations as \refpoem{The Maid of Amsterdam}.}\settowidth{\versewidth}{Though the night was made for loving,}
\begin{verse}[\versewidth]
So we'll go no more a roving\\*
\vin So late into the night,\\
Though the heart be still as loving,\\*
\vin And the moon be still as bright.\\!

For the sword outwears its sheath,\\*
\vin And the soul wears out the breast,\\
And the heart must pause to breathe,\\*
\vin And love itself have rest.\\!

Though the night was made for loving,\\*
\vin And the day returns too soon,\\
Yet we'll go no more a roving\\*
\vin By the light of the moon.
\end{verse}

\subsection{}

\blfootnote{Hosea 8.7, \cite{kjv}.}They have sown the wind, and they shall reap the whirlwind.

\section{}

\subsection{}

\blfootnote{$\mathbb{R}$ `The Stricken Deer', William Cowper (1731 -- 1800), \cite{londonbook}. \P 144. Pilgrims to the oracle at Delphi would first wash themselves in the Castalian spring. Drinking therefrom was said to induce poetical inspiration. \P 150. Themis was a Greek goddess of prophecy, named in some sources as the mother of Prometheus. \P 151. `Immortal Hale' = Sir Matthew Hale.}\settowidth{\versewidth}{Friends in the friends of science, and true prayer}
\begin{verse}[\versewidth]
I was a stricken deer that left the herd\\*
Long since, with many an arrow deep infixed\\
My panting side was charged, when I withdrew\\
To seek a tranquil death in distant shades.\\
There was I found by one who had himself\\
Been hurt by th'archers. In his side he bore,\\
And in his hands \& feet, the cruel scars.\\
With gentle force soliciting the darts,\\
He drew them forth, and healed and bade me live.\\
Since then, with few associates, in remote\\
And silent woods I wander, far from those,\\
My former partners of the peopled scene;\\
With few associates, and not wishing more.\\
Here much I ruminate, as much I may,\\
With other views of men and manners now\\
Than once, and others of a life to come.\\
I see that all are wanderers, gone astray\\
Each in his own delusions; they are lost\\
In chase of fancied happiness, still wooed\\
And never won. Dream after dream ensues,\\
And still they dream that they shall still succeed,\\
And still are disappointed. Rings the world\\
With the vain stir. I sum up \sfrac{$1$}{$2$} mankind\\
And add two thirds of the remaining \sfrac{$1$}{$2$},\\
And find the total of their hopes \& fears\\
Dreams, empty dreams. The million flit as gay,\\
As if created only like the fly,\\
That spreads his motley wings in the eye of noon\\
To sport their season and be seen no more.\\
The rest aro sober dreamers, grave \& wise\\
And pregnant with discoveries new \& rare.\\
Somo write a narrative of wars \& feats,\\
Of heroes little known, and call the rant\\
A history; describe the man, of whom\\
His own coevals took but little note,\\
And paint his person, character, \& views,\\
As they had know him from his mother womb.\\
They disentangle from the puzzled skein\\
In which obscurity has wrapped them up,\\
The threads of politic \& shrewd design\\
That ran through all his purposes, and charge\\
His mind with meanings that he never had\\
Or having, kept concealed. Some drill \& bore\\
The solid earth, and from the strata there\\
Extract a register, by which we learn\\
That he, who made it and revealed its date\\
To \textit{Moses}, was mistaken in its age.\\
Some, more acute \& more industrious still,\\
Contrive creation, travel nature up\\
To the sharp peak of her sublimest height,\\
Tell us whence the stars, why some are fixed,\\
And planetary some, what gave them first\\
Rotation, from what fountain flowed their light.\\
Great contest follows, and much learned dust\\
Involves the combatants, each claiming truth,\\
And truth disclaiming both. And thus they spend\\
The little wick of life's poor shallow lamp\\
In playing tricks with nature, giving laws\\
To distant worlds, and trifling in their own.\\
Is't not a pity now, that tickling rheums\\
Should ever tease the lungs, and blear the sight\\
Of oracles like these. Great pity too\\
That having wielded the elements, and built\\
A 1000 systems, each in his own way,\\
They should go out in fame and be forgot!\\
Ah what is life thus spent? And what are they\\
But frantic who thus spend it? All for smoke --\\
Eternity for bubbles proves at last\\
A senseless bargain. When I see such games\\
Played by the creatures of a power who swears\\
That he will judge the earth, and call the fool\\
To a sharp reckoning that has lived in vain;\\
And when I weigh this seeming wisdom well,\\
And prove it the infallible result\\
So hollow and so false -- I feel my heart\\
Dissolve in pity, and account the learned,\\
If this be learning, most of all deceived.\\
Great crimes alarm the conscience, but it sleeps\\
While thoughtful man is plausibly amused.\\
Defend me therefore, common sense, say I,\\
Prom reveries so airy, from the toil\\
Of dropping buckets into empty wells,\\*
And growing old in drawing nothing up!\\!

'Twere well says one sage erudite, profound,\\*
Terribly arched \& aquiline his nose,\\
And overbuilt with most impending brows,\\
'Twere well could you permit the world to live\\
As the world pleases. What's the world to you?\\
Much. I was born of woman, and drew milk,\\
As sweet as charity, from human breasts.\\
I think, articulate; I laugh and weep\\
And exercise all functions of a man.\\
How then should I and any man that lives\\
Be strangers to each other? Pierce my vein;\\
Take of the crimson stream meandering there,\\
And catechise it well. Apply your glass,\\
Search it, and prove now if it be not blood\\
Congenial with thine own: and if it be,\\
With edge of subtlety canst thou suppose\\
Keen enough, wise \& skilful as thou art,\\
To cut the link of brotherhood, by which\\
One common maker bound me to the kind.\\
True; I am no proficient, I confess,\\
In arts like yours I cannot call the swift\\
And perilous lightnings from the angry clouds,\\
And bid them hide themselves in th'earth beneath;\\
I cannot analyse the air, nor catch\\
The parallax of yonder luminous point\\
That seems \sfrac{$1$}{$2$} quenched in the immense abyss;\\
Such powers I boast not -- neither can I rest\\
A silent witness of the headlong rage,\\
Or heedless folly by which thousands die,\\*
Bone of my bone, \& kindred souls to mine.\\!

God never meant that man should scale the heavens\\*
By strides of human wisdom. In his works,\\
Though wondrous, he commands us in his word\\
To seek him rather where his mercy shines.\\
The mind indeed, enlightened from above,\\
Views him in all; ascribes to the grand cause\\
The grand eflect, acknowledges with joy\\
His manner, and with rapture tastes his style.\\
But never yet did philosophic tube\\
That brings the planets home into the eye\\
Of observation, and discovers, else\\
Not visible, his family of worlds,\\
Discover him that rules them: such a veil\\
Hangs oyer mortal eyes, blind from the birth,\\
And dark in things divine. Full often too\\
Our wayward intellect, the more we learn\\
Of nature, overlooks her author more;\\
From instrumental causes proud to draw\\
Conclusions retrograde, and mad mistake.\\
But if his word once teach us, shoot a ray\\
Through all the heart's dark chambers, and reveal\\
Truths undiscernd but by that holy light,\\
Then all is plain. Philosophy baptised\\
In the pure fountain of eternal love\\
Has eyes indeed; and viewing all she sees,\\
As meant to indicate a God to man,\\
Gives him his praise, and forfeits not her own.\\
Learning has borne such fruit in other days\\
On all her branches: piety has found\\
Friends in the friends of science, and true prayer\\
Has flowed from lips wet with castalian dews.\\
Such was thy wisdom, \textit{Newton}, childlike sage,\\
Sagacious reader of the works of God,\\
And in his word sagacious. Such too thine,\\
\textit{Milton}, whose genius had angelic wings,\\
And fed on manna. And such thine, in whom\\
Our british \textit{Themis} gloried with just cause,\\
Immortal \textit{Hale}, for deep discernment praised,\\
And sound integrity not more, than famed\\*
For sanctity of manners undefiled.
\end{verse}

\subsection{}

\blfootnote{Catherine Dyer, Lady Dyer (1590 -- 1654), \cite{newlove}. Lady Dyer had this remarkable epitaph inscribed on the monument of her late husband, Sir William Dyer (1583 -- 1621), which can be found in St Denys' Church in the village of Colmworth, Bedfordshire. This sonnet is actually just the second half of the complete epitaph. \P 12. There is some ambiguity in this line: some books gives `my blood grows cold', while others give `my beloved grows' cold. The Almanacker is no scholar of seventeenth century English orthography, but he has seen the original monument himself, and can report that it reads `MY BLOVD GROWES COLD', and thus he concludes that either interpretation may be correct.}\settowidth{\versewidth}{Shall soon repose her by thy slumb'ring side.}
\begin{verse}[\versewidth]
My dearest dust, could not thy hasty day\\*
Afford thy drowsy patience leave to stay\\
One hour longer: so that we might either\\
Sit up, or gone to bed together?\\
But since thy finished labour hath possessed\\
Thy weary limbs with early rest,\\
Enjoy it sweetly: and thy widow bride\\
Shall soon repose her by thy slumb'ring side.\\
Whose business, now, is only to prepare\\
My nightly dress, and call to prayer:\\
Mine eyes wax heavy and the day grows old.\\
The dew falls thick; my blood grows cold.\\
Draw, draw the clos\`{e}d curtains: and make room:\\*
My dear, my dearest dust; I come, I come.
\end{verse}

\subsection{}

\blfootnote{Daniel 5.27, \cite{kjv}. This is a portion of Daniel's interpretation of the famous writing on the wall at Belshazzar's feast.}Thou art weighed in the balances and art found wanting.

\section{}

\subsection{}

\blfootnote{`To the Memory of Mr Oldham', John Dryden, Poet Laureate (1631 -- 1700), \cite{norton}. \P 10. Nisus is a character from the \textit{Aeneid}, who, having slipped and fallen during a race, and seeing that he can't recover his lead, tackles one of the other competitors to ensure his friend's victory. \P 23. The name Marcellus refers to a number of figures from Roman history, although Dryden is probably referring here to Marcus Claudius Marcellus, the nephew and proposed heir of Augustus, whose death at nineteen years of age is a good example of a man who died before his youthful promise could be realised -- just like John Oldham, the subject of this elegy.}\settowidth{\versewidth}{Thy generous fruits, though gathered ere their prime}
\begin{verse}[\versewidth]
Farewell, too little \& too lately known,\\*
Whom I began to think \& call my own;\\
For sure our souls were near allied; and thine\\
Cast in the same poetic mould with mine.\\
One common note on either lyre did strike,\\
And knaves \& fools we both abhorred alike:\\
To the same goal did both our studies drive,\\
The last set out the soonest did arrive.\\
Thus \textit{Nisus} fell upon the slippery place,\\
While his young friend performed and won the race.\\
O early ripe! to thy abundant store\\
What could advancing age have added more?\\
It might (what nature never gives the young)\\
Have taught the numbers of thy native tongue.\\
But satire needs not those, and wit will shine\\
Through the harsh cadence of a rugged line.\\
A noble error, and but seldom made,\\
When poets are by too much force betrayed.\\
Thy generous fruits, though gathered ere their prime\\
Still showed a quickness; and maturing time\\
But mellows what we write to the dull sweets of rhyme.\\
Once more, hail \& farewell; farewell thou young,\\
But ah too short, \textit{Marcellus} of our tongue;\\
Thy brows with ivy, and with laurels bound;\\*
But fate \& gloomy night encompass thee around.
\end{verse}

\subsection{}

\blfootnote{John Clare (1793 -- 1864), \cite{obev}.}\settowidth{\versewidth}{And try to shake their fleeces from the snows.}
\begin{verse}[\versewidth]
The sheep get up and make their many tracks\\*
And bear a load of snow upon their backs,\\
And gnaw the frozen turnip to the ground\\
With sharp quick bite, and then go noising round\\
The boy that pecks the turnips all the day\\
And knocks his hands to keep the cold away\\
And laps his legs in straw to keep them warm\\
And hides behind the hedges from the storm.\\
The sheep, as tame as dogs, go where he goes\\
And try to shake their fleeces from the snows.\\
Then leave their frozen meal and wander round\\
The stubble stack that stands beside the ground,\\
And lie all night and face the drizzling storm\\*
And shun the hovel where they might be warm.
\end{verse}

\subsection{}

\blfootnote{John Dalberg-Acton, 1st Baron Acton (1834 -- 1902), \cite{odq}.}Beware of too much explaining, lest we end by too much excusing.

\section{}

\subsection{}

\blfootnote{`The Impulse', Robert Frost, Poet Laureate of Vermont (1874 -- 1963), \cite{norton}. This is part of a sequence of poems called \refpoem{The Hill Wife}.}\settowidth{\versewidth}{And didn't answer -- didn't speak --}
\begin{verse}[\versewidth]
It was too lonely for her there,\\*
\vin And too wild,\\
And since there were but two of them,\\*
\vin And no child,\\!

And work was little in the house,\\*
\vin She was free,\\
And followed where he furrowed field,\\*
\vin Or felled tree.\\!

She rested on a log and tossed\\*
\vin The fresh chips,\\
With a song only to herself\\*
\vin On her lips.\\!

And once she went to break a bough\\*
\vin Of black alder.\\
She strayed so far she scarcely heard\\*
\vin When he called her --\\!

And didn't answer -- didn't speak --\\*
\vin Or return.\\
She stood, and then she ran \& hid\\*
\vin In the fern.\\!

He never found her, though he looked\\*
\vin Everywhere,\\
And he asked at her mother's house\\*
\vin Was she there.\\!

Sudden \& swift \& light as that\\*
\vin The ties gave,\\
And he learned of finalities\\*
\vin Besides the grave.
\end{verse}

\subsection{}

\blfootnote{`A Dirge', Miss Christina Rossetti (1830 -- 1894), \cite{obev}.}\settowidth{\versewidth}{Why were you born when the snow was falling?}
\begin{verse}[\versewidth]
Why were you born when the snow was falling?\\*
You should have come to the cuckoo's calling,\\
Or when grapes are green in the cluster,\\
Or, at least, when lithe swallows muster\\
\vin For their far off flying\\*
\vin From summer dying.\\!

Why did you die when the lambs were cropping?\\*
You should have died at the apples' dropping,\\
When the grasshopper comes to trouble,\\
And the wheat-fields are sodden stubble,\\
\vin And all winds go sighing\\*
\vin For sweet things dying.
\end{verse}

\subsection{}

\blfootnote{William Blake (1757 -- 1827), \cite{blakea}. This is one of Blake's `Proverbs of Hell' from \refbook{The Marriage of Heaven and Hell}.}Every harlot was a virgin once.

\section{}

\subsection{}

\blfootnote{`Birches', Robert Frost, Poet Laureate of Vermont (1874 -- 1963), \cite{norton}.}\settowidth{\versewidth}{They are dragged to the withered bracken by the load,}
\begin{verse}[\versewidth]
When I see birches bend to left \& right\\*
Across the lines of straighter darker trees,\\
I like to think some boy's been swinging them.\\
But swinging doesn't bend them down to stay\\
As ice-storms do. Often you must have seen them\\
Loaded with ice a sunny winter morning\\
After a rain. They click upon themselves\\
As the breeze rises, and turn many-coloured\\
As the stir cracks \& crazes their enamel.\\
Soon the sun's warmth makes them shed crystal shells\\
Shattering \& avalanching on the snow-crust --\\
Such heaps of broken glass to sweep away\\
You'd think the inner dome of heaven had fallen.\\
They are dragged to the withered bracken by the load,\\
And they seem not to break; though once they are bowed\\
So low for long, they never right themselves:\\
You may see their trunks arching in the woods\\
Years afterwards, trailing their leaves on the ground\\
Like girls on hands \& knees that throw their hair\\
Before them over their heads to dry in the sun.\\
But I was going to say when truth broke in\\
With all her matter-of-fact about the ice-storm\\
I should prefer to have some boy bend them\\
As he went out \& in to fetch the cows --\\
Some boy too far from town to learn baseball,\\
Whose only play was what he found himself,\\
Summer or winter, and could play alone.\\
One by one he subdued his father's trees\\
By riding them down over \& over again\\
Until he took the stiffness out of them,\\
And not one but hung limp, not one was left\\
For him to conquer. He learned all there was\\
To learn about not launching out too soon\\
And so not carrying the tree away\\
Clear to the ground. He always kept his poise\\
To the top branches, climbing carefully\\
With the same pains you use to fill a cup\\
Up to the brim, and even above the brim.\\
Then he flung outward, feet first, with a swish,\\
Kicking his way down through the air to the ground.\\
So was I once myself a swinger of birches.\\
And so I dream of going back to be.\\
It's when I'm weary of considerations,\\
And life is too much like a pathless wood\\
Where your face burns and tickles with the cobwebs\\
Broken across it, and one eye is weeping\\
From a twig's having lashed across it open.\\
I'd like to get away from earth awhile\\
And then come back to it and begin over.\\
May no fate wilfully misunderstand me\\
And \sfrac{$1$}{$2$} grant what I wish \& snatch me away\\
Not to return. Earth's the right place for love:\\
I don't know where it's likely to go better.\\
I'd like to go by climbing a birch tree,\\
And climb black branches up a snow-white trunk\\
Tow\'{a}rd heaven, till the tree could bear no more,\\
But dipped its top and set me down again.\\
That would be good both going \& coming back.\\*
One could do worse than be a swinger of birches.
\end{verse}

\subsection{}

\blfootnote{$\mathbb{R}$ Samuel Daniel (1562 -- 1619), \cite{pbev}. This is one of Daniel's sonnets \refpoem{To Delia}. Beaumont and Fletcher borrowed heavily from these lines in writing one of the songs in their play \refbook{Valentinian}.}\settowidth{\versewidth}{    Brother to death, in silent darkness born:}
\begin{verse}[\versewidth]
Care-charmer sleep, son of the sable night,\\*
\vin Brother to death, in silent darkness born:\\
Relieve my languish, and restore the light,\\
\vin With dark forgetting of my cares, return;\\
And let the day be time enough to mourn\\
\vin The shipwreck of my ill-adventured youth:\\
Let waking eyes suffice to wail their scorn,\\
\vin Without the torment of the night's untruth.\\
Cease dreams, th'imagery of our day-desires,\\
\vin To model forth the passions of the morrow;\\
Never let rising sun approve you liars,\\
\vin To add more grief to aggravate my sorrow.\\
Still let me sleep, embracing clouds in vain;\\*
And never wake to feel the day's disdain.
\end{verse}

\subsection{}

\blfootnote{William Blake (1757 -- 1827), \cite{blakea}. This is one of Blake's `Proverbs of Hell' from \refbook{The Marriage of Heaven and Hell}.}One law for the lion and ox is oppression.

\section{}

\subsection{}

\blfootnote{$\mathbb{R}$ `Long Tom', Wilfrid Gibson (1878 -- 1962), \cite{oxfordlarkin}.}\settowidth{\versewidth}{And nothing would keep out that gnawing cold --}
\begin{verse}[\versewidth]
He talked of \textsc{Delhi} brothels \sfrac{$1$}{$2$} the night,\\*
Quaking with fever; and then, dragging tight\\
The frouzy blankets to his chattering chin,\\
Cursed for an hour because they were so thin\\
And nothing would keep out that gnawing cold --\\
Scarce 40 years of age, and yet so old,\\
Haggard and worn with burning eyes set deep --\\*
Until at last he cursed himself asleep.\\!

Before I'd shut my eyes reveille came;\\*
And as I dressed by the one candle-flame\\
The mellow golden light fell on his face\\
Still sleeping, touching it to tender grace,\\
Rounding the features life had scarred so deep,\\
Till youth came back to him in quiet sleep:\\
And then what women saw in him I knew\\*
And why they'd love him all his brief life through.
\end{verse}

\subsection{}

\blfootnote{`Lament', Wilfrid Gibson (1878 -- 1962), \cite{oxfordlarkin}. Although Gibson joined the British Army during the First World War, he never served abroad. Due to some kind of medical defect, possibly poor eyesight, he ultimately served as a clerk, which -- when one compares his lifespan to that of Edward Thomas, who was born in the same year -- turned out to be a good career move.}\settowidth{\versewidth}{Their lives for us loved, too, the sun rain?}
\begin{verse}[\versewidth]
We who are left, how shall we look again\\*
Happily on the sun or feel the rain\\
\vin Without remembering how they who went\\
\vin Ungrudgingly and spent\\*
Their lives for us loved, too, the sun \& rain?\\!

A bird among the rain-wet lilac sings --\\*
But we, how shall we turn to little things\\
\vin And listen to the birds \& winds \& streams\\
\vin Made holy by their dreams,\\*
Nor feel the heart-break in the heart of things?
\end{verse}

\subsection{}

\blfootnote{The Rt Hon John Bright (1811 -- 1889), \cite{odq}.}The angel of death has been abroad throughout the land; you may also hear the beating of his wings.

\section{}

\subsection{}

\blfootnote{`The Convergence of the Twain', Thomas Hardy (1840 -- 1928), \cite{norton}. Hardy wrote this poem in response to the sinking of the RMS \textit{Titanic}. His ideas about the `immanent will' seem to owe a debt to the philosophy of Arthur Schopenhauer.}\settowidth{\versewidth}{And the pride of life that planned her, stilly couches she.}
\begin{verse}[\versewidth]
\vin \vin In a solitude of the sea\\*
\vin \vin Deep from human vanity,\\*
And the pride of life that planned her, stilly couches she.\\!

\vin \vin Steel chambers, late the pyres\\*
\vin \vin Of her salamandrine fires,\\*
Cold currents thrid, and turn to rhythmic tidal lyres.\\!

\vin \vin Over the mirrors meant\\*
\vin \vin To glass the opulent\\*
The sea-worm crawls -- grotesque, slimed, dumb, indifferent.\\!

\vin \vin Jewels in joy designed\\*
\vin \vin To ravish the sensuous mind\\*
Lie lightless, all their sparkles bleared \& black \& blind.\\!

\vin \vin Dim moon-eyed fishes near\\*
\vin \vin Gaze at the gilded gear\\*
And query, What does this vaingloriousness down here?\\!

\vin \vin Well: while was fashioning\\*
\vin \vin This creature of cleaving wing,\\*
The immanent will that stirs \& urges everything\\!

\vin \vin Prepared a sinister mate\\*
\vin \vin For her -- so gaily great --\\*
A shape of ice, for the time far \& dissociate.\\!

\vin \vin And as the smart ship grew\\*
\vin \vin In stature, grace, \& hue,\\*
In shadowy silent distance grew the iceberg too.\\!

\vin \vin Alien they seemed to be;\\*
\vin \vin No mortal eye could see\\*
The intimate welding of their later history,\\!

\vin \vin Or sign that they were bent\\*
\vin \vin By paths coincident\\*
On being anon twin halves of one august event,\\!

\vin \vin Till the spinner of the years\\*
\vin \vin Said, Now! And each one hears,\\*
And consummation comes, and jars two hemispheres.
\end{verse}

\subsection{}

\blfootnote{`Behind the Line', Ivor Gurney (1890 -- 1937), \cite{obev}. \P 5. An \textit{estimanet} is a French word for a kind of caf\'e which serves alcohol.}\settowidth{\versewidth}{Black printed columns give news, but no longer the fear}
\begin{verse}[\versewidth]
I suppose France this morning is as white as here:\\*
High white clouds veiling the sun, and the mere\\
Cabbage fields \& potato plants lovely to see,\\*
Back behind at \textsc{Robecq} there with the day free.\\!

In the {\hoskeroe estaminets} I suppose the air as cool, and the floor\\*
Grateful dark red; the beer and the different store\\
Of citron, grenadine, red wine as surely delectable\\*
As in 1916; with the round stains on the dark table.\\!

Journals Fran{\c{c}}ais tell the same news and the queer\\*
Black printed columns give news, but no longer the fear\\
Of shrapnel or any evil metal torments.\\*
High white morning as here one is sure is on France.
\end{verse}

\subsection{}

\blfootnote{Miss Emily Brontë (1818 -- 1848), \cite{pbev}. This is a line from Miss Bronte's poem \refpoem{Remembrance}.}Sweet love of youth, forgive, if I forget thee.

\section{}

\subsection{}

\blfootnote{`Thoughts of Phena', Thomas Hardy (1840 -- 1928), \cite{oxfordlarkin}. Subtitle: `At News of Her Death'.The Phena in question was a Tryphena Sparks, Hardy's probable lover and cousin (or possibly niece) and at one time his intended bride. Prof Larkin commented once that reading this poem brought about his conversion to the genuinely English tradition of poetry, and away from Yeats's shoddy school.}\settowidth{\versewidth}{Her full day-star; unease, or regret, or forebodings, or fears}
\begin{verse}[\versewidth]
\vin Not a line of her writing have I,\\*
\vin \vin Not a thread of her hair,\\
No mark of her late time as dame in her dwelling, whereby\\
\vin \vin I may picture her there;\\
\vin And in vain do I urge my unsight\\
\vin \vin To conceive my lost prize\\
At her close, whom I knew when her dreams were upbrimming with light,\\*
\vin \vin And with laughter her eyes.\\!

\vin What scenes spread around her last days,\\*
\vin \vin Sad, shining, or dim?\\
Did her gifts \& compassions enray \& enarch her sweet ways\\
\vin \vin With an aureat nimb?\\
\vin Or did life-light decline from her years,\\
\vin \vin And mischances control\\
Her full day-star; unease, or regret, or forebodings, or fears\\*
\vin \vin Disennoble her soul?\\!

\vin Thus I do but the phantom retain\\*
\vin \vin Of the maiden of yore\\
As my relic; yet haply the best of her -- fined in my brain\\
\vin \vin It may be the more\\
\vin That no line of her writing have I,\\
\vin \vin Nor a thread of her hair,\\
No mark of her late time as dame in her dwelling, whereby\\*
\vin \vin I may picture her there.
\end{verse}

\subsection{}

\blfootnote{She to Him 1, Thomas Hardy (1840 -- 1928), \cite{oxfordlarkin}.}\settowidth{\versewidth}{    That sportsman time but rears his brood to kill,}
\begin{verse}[\versewidth]
When you shall see me in the toils of time,\\*
\vin My lauded beauties carried off from me,\\
My eyes no longer stars as in their prime,\\
\vin My name forgot of maiden fair \& free;\\
When, in your being, heart concedes to mind,\\
\vin And judgment, though you scarce its process know,\\
Recalls the excellencies I once enshrined,\\
\vin And you are irked that they have withered so;\\
Remembering mine the loss is, not the blame,\\
\vin That sportsman time but rears his brood to kill,\\
Knowing me in my soul the very same\\
\vin One who would die to spare you touch of ill,\\
Will you not grant to old affection's claim\\*
\vin The hand of friendship down life's sunless hill?
\end{verse}

\subsection{}

\blfootnote{John Bunyan (1628 -- 1688), \cite{bunyan}. The quotation continues: `and one sin will destroy a sinner.' Bunyan was no sailor, or else he would have written: one \emph{flood} will cause a ship to \emph{founder}.}One leak will sink a ship.

\section{}

\subsection{}

\blfootnote{$\mathbb{R}$ `At Castle Boterel', Thomas Hardy (1840 -- 1928), \cite{ptmgmc}. Kastel Boterel is the Cornish-language name for the English village of Boscastle. The village is named after a nearby castle (of which very little survives) which itself was named after the Barons Botreaux (pronounced like the English word \textit{buttery}), a title in the Peerage of England in abeyance at the time of writing.}\settowidth{\versewidth}{What we did as we climbed, and what we talked of}
\begin{verse}[\versewidth]
As I drive to the junction of lane \& highway,\\*
\vin And the drizzle bedrenches the waggonette,\\
I look behind at the fading byway,\\
\vin And see on its slope, now glistening wet,\\*
\vin \vin Distinctly yet\\!

Myself and a girlish form benighted\\*
\vin In dry march weather. We climb the road\\
Beside a chaise. We had just alighted\\
\vin To ease the sturdy pony's load\\*
\vin \vin When he sighed \& slowed.\\!

What we did as we climbed, and what we talked of\\*
\vin Matters not much, nor to what it led --\\
Something that life will not be balked of\\
\vin Without rude reason till hope is dead,\\*
\vin \vin And feeling fled.\\!

It filled but a minute. But was there ever\\*
\vin A time of such quality, since or before,\\
In that hill's story ? To one mind never,\\
\vin Though it has been climbed, foot-swift, foot-sore,\\*
\vin \vin By 1000s more.\\!

Primeval rocks form the road's steep border,\\*
\vin And much have they faced there, first \& last,\\
Of the transitory in earth's long order;\\
\vin But what they record in colour \& cast\\*
\vin \vin Is -- that we two passed.\\!

And to me, though time's unflinching rigour,\\*
\vin In mindless rote, has ruled from sight\\
The substance now, one phantom figure\\
\vin Remains on the slope, as when that night\\*
\vin \vin Saw us alight.\\!

I look \& see it there, shrinking, shrinking;\\*
\vin I look back at it amid the rain\\
For the very last time; for my sand is sinking,\\
\vin And I shall traverse old love's domain\\*
\vin \vin Never again.
\end{verse}

\subsection{}

\blfootnote{`Very Old Man', Dr James Henry (1798 -- 1876), \cite{obev}.}\settowidth{\versewidth}{From chair to chair about my mother's chamber,}
\begin{verse}[\versewidth]
I well remember how some threescore years\\*
And 10 ago, a helpless babe, I toddled\\
From chair to chair about my mother's chamber,\\
Feeling, as 'twere, my way in the new world\\
And foolishly afraid of, or, as 't might be,\\
Foolishly pleased with, th' unknown objects round me.\\
And now with stiffened joints I sit all day\\
In one of those same chairs, as foolishly\\
Hoping or fearing something from me hid\\
Behind the thick, dark veil which I see hourly\\
And minutely on every side round closing\\*
And from my view all objects shutting out.
\end{verse}

\subsection{}

\blfootnote{George Noel, 6th Baron Byron (1788 -- 1824), \cite{odq}. Lord Byron is said to have uttered these words to his wife.}You should have a softer pillow than my heart.

\section{}

\subsection{}

\blfootnote{`On the Departure Platform', Thomas Hardy (1840 -- 1928), \cite{oxfordlarkin}.}\settowidth{\versewidth}{We kissed at the barrier; and passing through}
\begin{verse}[\versewidth]
We kissed at the barrier; and passing through\\*
She left me, and moment by moment got\\
Smaller \& smaller, until to my view\\*
\vin She was but a spot;\\!

A wee white spot of muslin fluff\\*
That down the diminishing platform bore\\
Through hustling crowds of gentle \& rough\\*
\vin To the carriage door.\\!

Under the lamplight's fitful glowers,\\*
Behind dark groups from far \& near,\\
Whose interests were apart from ours,\\*
\vin She would disappear,\\!

Then show again, till I ceased to see\\*
That flexible form, that nebulous white;\\
And she who was more than my life to me\\*
\vin Had vanished quite.\\!

We have penned new plans since that fair fond day,\\*
And in season she will appear again\\
-- Perhaps in the same soft white array --\\*
\vin But never as then.\\!

And why, young man, must eternally fly\\*
A joy you'll repeat, if you love her well?\\
O friend, nought happens twice thus; why,\\*
\vin I cannot tell.
\end{verse}

\subsection{}

\blfootnote{`To Dianeme', Robert Herrick (1591 -- 1674), \cite{treasury}.}\settowidth{\versewidth}{Sweet, be not proud of those two eyes,}
\begin{verse}[\versewidth]
Sweet, be not proud of those two eyes,\\*
Which starlike sparkle in their skies;\\
Nor be you proud that you can see\\
All hearts your captives, yours yet free;\\
Be you not proud of that rich hair\\
Which wantons with the lovesick air;\\
Whenas that ruby which you wear,\\
Sunk from the tip of your soft ear,\\
Will last to be a precious stone\\*
When all your world of beauty's gone.
\end{verse}

\subsection{}

\blfootnote{James Cabell (1879 -- 1958), \cite{odq}.}The optimist proclaims that we live in the best of all possible worlds; and the pessimist fears this is true.

\section{}

\subsection{}

\blfootnote{`Neutral Tones', Thomas Hardy (1840 -- 1928), \cite{norton}.}\settowidth{\versewidth}{    Your face, and the God-cursed sun, a tree,}
\begin{verse}[\versewidth]
We stood by a pond that winter day,\\*
\vin And the sun was white, as though chidden of God,\\
\vin And a few leaves lay on the starving sod;\\*
They had fallen from an ash, and were grey.\\!

Your eyes on me were as eyes that rove\\*
\vin Over tedious riddles of years ago;\\
\vin And some words played between us to \& fro\\*
On which lost the more by our love.\\!

The smile on your mouth was the deadest thing\\*
\vin Alive enough to have strength to die;\\
\vin And a grin of bitterness swept thereby\\*
Like an ominous bird a-wing...\\!

Since then, keen lessons that love deceives,\\*
\vin And wrings with wrong, have shaped to me\\
\vin Your face, and the God-cursed sun, \& a tree,\\*
And a pond edged with greyish leaves.
\end{verse}

\subsection{}

\blfootnote{`Sonnet', Thomas Hood (1799 -- 1845), \cite{londonbook}.}\settowidth{\versewidth}{That thoughts shall cease, and the immortal spright}
\begin{verse}[\versewidth]
It is not death, that sometime in a sigh\\*
\vin This eloquent breath shall take its speechless flight;\\
That sometime these bright stars, that now reply\\
\vin In sunlight to the sun, shall set in night;\\
That this warm conscious flesh shall perish quite,\\
\vin And all life's ruddy springs forget to flow;\\
That thoughts shall cease, and the immortal spright\\
\vin Be lapped in alien clay and laid below;\\
It is not death to know this -- but to know\\
\vin That pious thoughts, which visit at new graves\\
In tender pilgrimage, will cease to go\\
\vin So duly and so oft -- and when grass waves\\
Over the past-away, there may be then\\*
No resurrection in the minds of men.
\end{verse}

\subsection{}

\blfootnote{Thomas Carlyle (1795 -- 1881), \cite{odq}.}Experience is the best of schoolmasters, only the school fees are heavy.

\section{}

\subsection{}

\blfootnote{`Afterwards', Thomas Hardy (1840 -- 1928), \cite{norton}.}\settowidth{\versewidth}{If, when hearing that I have been stilled at last, they stand at the door,}
\begin{verse}[\versewidth]
When the present has latched its postern behind my tremulous stay,\\*
\vin And the may month flaps its glad green leaves like wings,\\
Delicate-filmed as new-spun silk, will the neighbours say,\\*
\vin `He was a man who used to notice such things'?\\!

If it be in the dusk when, like an eyelid's soundless blink,\\*
\vin The dewfall-hawk comes crossing the shades to alight\\
Upon the wind-warped upland thorn, a gazer may think,\\*
\vin `To him this must have been a familiar sight.'\\!

If I pass during some nocturnal blackness, mothy \& warm,\\*
\vin When the hedgehog travels furtively over the lawn,\\
One may say, `He strove that such innocent creatures should come to no harm,\\*
\vin But he could do little for them; and now he is gone.'\\!

If, when hearing that I have been stilled at last, they stand at the door,\\*
\vin Watching the full-starred heavens that winter sees,\\
Will this thought rise on those who will meet my face no more,\\*
\vin `He was one who had an eye for such mysteries'?\\!

And will any say when my bell of quittance is heard in the gloom,\\*
\vin And a crossing breeze cuts a pause in its outrollings,\\
Till they rise again, as they were a new bell's boom,\\*
\vin `He hears it not now, but used to notice such things'?
\end{verse}

\subsection{}

\blfootnote{Prof Alfred Housman (1859 -- 1936), \cite{oxfordlarkin}.}\settowidth{\versewidth}{What are those blue remembered hills?}
\begin{verse}[\versewidth]
Into my heart an air that kills\\*
\vin From yon far country blows:\\
What are those blue remembered hills?\\*
\vin What spires, what farms are those?\\!

That is the land of lost content\\*
\vin -- I see it shining plain --\\
The happy highways where I went\\*
\vin And cannot come again.
\end{verse}

\subsection{}

\blfootnote{Miss Wilella Cather (1873 -- 1947), \cite{odq}.}The heart of another is a dark forest.

\section{}

\subsection{}

\blfootnote{`Harp Song of the Dane Women', Rudyard Kipling (1865 -- 1936), \cite{pbev}. These lines are found in Kipling's story \refpoem{The Knights of the Joyous Venture}, which is in the collection \refbook{Puck of Pook's Hill}.}\settowidth{\versewidth}{And the sound of your oar-blades, falling hollow,}
\begin{verse}[\versewidth]
What is a woman that you forsake her,\\*
And the hearth-fire and the home-acre,\\*
To go with the old grey widow-maker?\\!

She has no house to lay a guest in --\\*
But one chill bed for all to rest in,\\*
That the pale suns \& the stray bergs nest in.\\!

She has no strong white arms to fold you,\\*
But the 10-times-fingering weed to hold you --\\*
Out on the rocks where the tide has rolled you.\\!

Yet, when the signs of summer thicken,\\*
And the ice breaks, and the birch-buds quicken,\\*
Yearly you turn from our side, and sicken --\\!

Sicken again for the shouts \& the slaughters.\\*
You steal away to the lapping waters,\\*
And look at your ship in her winter quarters.\\!

You forget our mirth, and talk at the tables,\\*
The kine in the shed \& the horse in the stables --\\*
To pitch her sides and go over her cables.\\!

Then you drive out where the storm clouds swallow,\\*
And the sound of your oar-blades, falling hollow,\\*
Is all we have left through the months to follow.\\!

Ah what is woman that you forsake her,\\*
And the hearth-fire and the home-acre,\\*
To go with the old grey widow-maker?
\end{verse}

\subsection{}

\blfootnote{`Dirce', Walter Landor (1775 -- 1864), \cite{obev}. Dirce is an obscure figure in Greek mythology, the wicked aunt -- according to Pseudo-Apollodorus -- of Zeus' twin sons Amphion and Zethus. The myth seems to have only a tangential connection to this poem.}\settowidth{\versewidth}{Stand close around, ye stygian set,}
\begin{verse}[\versewidth]
Stand close around, ye stygian set,\\*
\vin With \textit{Dirce} in one boat conveyed;\\
Or \textit{Charon}, seeing, may forget\\*
\vin That he is old and she a shade.
\end{verse}

\subsection{}

\blfootnote{Mrs Susanna Centlivre (1669 -- 1723), \cite{odq}.}Nothing to be done without a bribe I find, in love as well as law.

\section{}

\subsection{}

\blfootnote{$\mathbb{R}$ George Meredith (1828 -- 1909), \cite{norton}. This is taken from \refbook{Modern Love}, Meredith's sequence of poems describing the breakdown of his first marriage.}\settowidth{\versewidth}{    Fast, sweet, golden, shows the marriage-knot.}
\begin{verse}[\versewidth]
At dinner, she is hostess; I am host.\\*
\vin Went the feast ever cheerfuller? She keeps\\
\vin The topic over intellectual deeps\\
In buoyancy afloat. They see no ghost.\\
With sparkling surface-eyes we ply the ball:\\
\vin It is in truth a most contagious game:\\
\vin ``Hiding the Skeleton'' shall be its name.\\
Such play as this the devils might appal!\\
But here's the greater wonder; in that we,\\
\vin Enamoured of an acting nought can tire,\\
\vin Each other, like true hypocrites, admire;\\
Warm-lighted looks, love's ephemerae,\\
Shoot gaily o'er the dishes \& the wine.\\
\vin We waken envy of our happy lot.\\
\vin Fast, sweet, \& golden, shows the marriage-knot.\\*
Dear guests, you now have seen love's corpse-light shine.
\end{verse}

\subsection{}

\blfootnote{`Villanelle', Roland Leighton (1895 -- 1915), \cite{brittain}. The Almanacker heard this poem recited over Leighton's grave when he was sixteen. \P 1. Plug Street was the name adopted by the British soldiers for Ploegsteert, a Belgian village.}\settowidth{\versewidth}{(It is strange they should be blue,}
\begin{verse}[\versewidth]
Violets from \textsc{Plug Street Wood},\\*
Sweet, I send you from oversea.\\
(It is strange they should be blue,\\
Blue, when his soaked blood was red,\\
For they grew around his head:\\*
It is strange they should be blue.)\\!

Violets from \textsc{Plug Street Wood},\\*
Think what they have meant to me --\\
Life \& hope \& love \& you.\\
(And you did not see them grow,\\
Where his mangled body lay,\\
Hiding horror from the day;\\*
Sweetest it was better so.)\\!

Violets from oversea,\\*
To your dear, far, forgetting land,\\
These I send in memory,\\*
Knowing you will understand.
\end{verse}

\subsection{}

\blfootnote{Mrs Mary Chesnut (1823 -- 1886), \cite{odq}.}There is no hope but we will try to have no fear.

\section{}

\subsection{}

\blfootnote{George Meredith (1828 -- 1909), \cite{norton}. This is taken from \refbook{Modern Love}, Meredith's sequence of poems describing the breakdown of his first marriage.}\settowidth{\versewidth}{    The strange low sobs that shook their common bed}
\begin{verse}[\versewidth]
By this he knew she wept with waking eyes:\\*
\vin That, at his hand's light quiver by her head,\\
\vin The strange low sobs that shook their common bed\\
Were called into her with a sharp surprise,\\
And strangled mute, like little gaping snakes,\\
\vin Dreadfully venomous to him. She lay\\
\vin Stone-still, and the long darkness flowed away\\
With muffled pulses. Then, as midnight makes\\
Her giant heart of memory \& tears\\
\vin Drink the pale drug of silence, and so beat\\
\vin Sleep's heavy measure, they from head to feet\\
Were moveless, looking through their dead black years,\\
By vain regret scrawled over the blank wall.\\
\vin Like sculptured effigies they might be seen\\
\vin Upon their marriage-tomb, the sword between;\\*
Each wishing for the sword that severs all.
\end{verse}

\subsection{}

\blfootnote{`Sonnet', Dr John Masefield, Poet Laureate (1878 -- 1967), \cite{obev}. This is a translation of a poem by Don Francisco de Quevedo.}\settowidth{\versewidth}{    Their strength destroyed by this new age's way}
\begin{verse}[\versewidth]
I saw the ramparts of my native land,\\*
\vin One time so strong, now dropping in decay,\\
\vin Their strength destroyed by this new age's way\\
That has worn out and rotted what was grand.\\
I went into the fields: there I could see\\
\vin The sun drink up the waters newly thawed,\\
\vin And on the hills the moaning cattle pawed;\\*
Their miseries robbed the day of light for me.\\!

I went into my house: I saw how spotted,\\*
\vin Decaying things made that old home their prize.\\
\vin \vin My withered walking-staff had come to bend;\\
I felt the age had won; my sword was rotted,\\
\vin And there was nothing on which I set my eyes\\*
\vin \vin That was not a reminder of the end.
\end{verse}

\subsection{}

\blfootnote{Gilbert Chesterton, Knight (1874 -- 1936), \cite{odq}.}It isn't that they can't see the solution; it is that they can't see the problem.

\section{}

\subsection{}

\blfootnote{George Meredith (1828 -- 1909), \cite{norton}. This is taken from \refbook{Modern Love}, Meredith's sequence of poems describing the breakdown of his first marriage.}\settowidth{\versewidth}{    That night he learned how silence best can speak}
\begin{verse}[\versewidth]
He found her by the ocean's moaning verge,\\*
\vin Nor any wicked change in her discerned;\\
\vin And she believed his old love had returned,\\
Which was her exultation, \& her scourge.\\
She took his hand, and walked with him, and seemed\\
\vin The wife he sought, though shadow-like \& dry.\\
\vin She had one terror, lest her heart should sigh,\\
And tell her loudly that she no longer dreamed.\\
She dared not say, `This is my breast: look in.'\\
\vin But there's a strength to help the desperate weak.\\
\vin That night he learned how silence best can speak\\
The awful things when pity pleads for sin.\\
About the middle of the night her call\\
\vin Was heard, and he came wondering to the bed.\\
\vin Now kiss me, dear! It may be, now! she said.\\*
\textsc{Lethe} had passed those lips, and he knew all.
\end{verse}

\subsection{}

\blfootnote{Sonnet 23, John Milton (1608 -- 1674), \cite{norton}. \P 2. Alcestis was the wife of Admetus. Having given her life to ensure her husband's survival, Heracles broke into the underworld and returned her to her home.}\settowidth{\versewidth}{    Whom Jove's great son to her glad husband gave,}
\begin{verse}[\versewidth]
Methought I saw my late espous\`{e}d saint\\*
\vin Brought to me, like \textit{Alcestis}, from the grave,\\
\vin Whom \textit{Jove}'s great son to her glad husband gave,\\
Rescued from death by force, though pale \& faint.\\
Mine, as whom washed from spot of child-bed taint\\
\vin Purification in the old law did save,\\
\vin And such as yet once more I trust to have\\
Full sight of her in heaven without restraint,\\
Came vested all in white, pure as her mind;\\
\vin Her face was veiled, yet to my fancied sight\\
Love, sweetness, goodness, in her person shined\\
\vin So clear as in no face with more delight.\\
But O as to embrace me she inclined,\\*
\vin I waked, she fled, and day brought back my night.
\end{verse}

\subsection{}

\blfootnote{Gilbert Chesterton, Knight (1874 -- 1936), \cite{odq}.}The poor have sometimes objected to being governed badly; the rich have always objected to being governed at all.

\section{}

\subsection{}

\blfootnote{$\mathbb{R}$ `In Nunhead Cemetery', Miss Charlotte Mew (1869 -- 1928), \cite{norton}.}\settowidth{\versewidth}{Tomorrow I will tell you about the eyes of the Crystal Palace train}
\begin{verse}[\versewidth]
It is the clay what makes the earth stick to his spade;\\*
\vin He fills in holes like this year after year;\\
The others have gone; they were tired, and \sfrac{$1$}{$2$} afraid\\*
\vin But I would rather be standing here;\\!

There is nowhere else to go. I have seen this place\\*
\vin From the windows of the train that's going past\\
Against the sky. This is rain on my face;\\*
\vin It was raining here when I saw it last.\\!

There is something horrible about a flower;\\*
\vin This, broken in my hand, is one of those\\
He threw it in just now; it will not live another hour;\\*
\vin There are 1000s more; you do not miss a rose.\\!

One of the children hanging about\\*
\vin Pointed at the whole dreadful heap and smiled\\
This morning after th\'{a}t was carried out;\\*
\vin There is something terrible about a child.\\!

We were like children last week, in the \textsc{Strand};\\*
\vin That was the day you laughed at me\\
Because I tried to make you understand\\
\vin The cheap, stale chap I used to be\\*
\vin Before I saw the things you made me see.\\!

This is not a real place; perhaps by \& by\\*
\vin I shall wake -- I am getting drenched with all this rain:\\
Tomorrow I will tell you about the eyes of the \textsc{Crystal Palace} train\\*
\vin Looking down on us, and you will laugh \& I shall see what you see again.\\!

\vin \vin Not here, not now. We said, Not yet\\*
\vin \vin Across our low stone parapet\\*
\vin Will the quick shadows of the sparrows fall.\\!

\vin \vin But still it was a lovely thing\\*
\vin \vin Through the grey months to wait for spring\\
\vin \vin With the birds that go a-gypsying\\
\vin In the parks till the blue seas call.\\
\vin \vin And next to these, you used to care\\
\vin \vin For the lions in \textsc{Trafalgar Square},\\
Who'll stand \& speak for \textsc{London} when her bell of judgement tolls --\\
\vin \vin And the gulls at \textsc{Westminster} that were\\
\vin \vin The old sea-captains' souls.\\
Today again the brown tide splashes step by step, the river-stair,\\*
\vin \vin And the gulls are there!\\!

By a month we have missed our day:\\*
\vin The children would have hung about\\
Round the carriage \& over the way\\*
\vin As you \& I came out.\\!

We should have stood on the gulls' black cliffs \& heard the sea\\*
\vin And seen the moon's white track;\\
I would have called; you would have come to me\\*
\vin And kissed me back.\\!

You have never done that: I do not know\\*
\vin Why I stood staring at your bed\\
And heard you, though you spoke so low,\\
\vin But could not reach your hands, your little head;\\
There was nothing we could not do, you said,\\*
\vin And you went, and I let you go!\\!

Now I will burn you back; I will burn you through,\\*
\vin Though I am damned for it we two will lie\\
\vin And burn, here where the starlings fly\\
\vin To these white stones from the wet sky;\\
\vin Dear, you will say this is not I --\\*
It would not be you! It would not be you!\\!

If for only a little while\\*
\vin You will think of it you will understand;\\
\vin \vin If you will touch my sleeve \& smile\\
\vin As you did that morning in the \textsc{Strand}\\
\vin \vin I can wait quietly with you\\
\vin \vin Or go away if you want me to --\\
God! What is God? But your face has gone \& your hand!\\*
\vin Let me stay here too.\\!

\vin \vin When I was quite a little lad\\*
\vin At Christmas time we went \sfrac{$1$}{$2$} mad\\
\vin \vin For joy of all the toys we had,\\
And then we used to sing about the sheep\\
\vin \vin The shepherds watched by night;\\
We used to pray to \textit{Christ} to keep\\
\vin \vin Our small souls safe till morning light;\\
I am scared; I am staying with you tonight --\\*
\vin \vin \vin \vin Put me to sleep.\\!

I shall stay here: here you can see the sky;\\*
The houses in the street are much too high;\\
\vin There is no one left to speak to there;\\
\vin Here they are everywhere,\\
And just above them fields \& fields of roses lie --\\*
If he would dig it all up again they would not die.
\end{verse}

\subsection{}

\blfootnote{`Epitaph on the Earl of Strafford', The Rev Dr Clement Paman (1612 -- 1664), \cite{obev}. Strafford, one of the Charles I's ministers, offered himself up to the king as a sacrifice to appease a certain faction in the House of Commons, and was duly beheaded. Sir Christoper notes that others have attributed this poem to John Cleveland.}\settowidth{\versewidth}{Strafford, who was hurried hence}
\begin{verse}[\versewidth]
Here lies wise and valiant dust\\*
Huddled up 'twixt fit \& just,\\
\textit{Strafford}, who was hurried hence\\
'Twixt treason \& convenience.\\
He spent his time here in a mist,\\
A papist, yet a calvinist;\\
His prince's nearest joy \& grief,\\
He had, yet wanted all relief;\\
The prop \& ruin of the state;\\
The people's violent love \& hate;\\
One in extremes loved \& abhorred.\\
Riddles lie here, or in a word,\\
Here lies blood; and let it lie\\*
Speechless still and never cry.
\end{verse}

\subsection{}

\blfootnote{Gilbert Chesterton, Knight (1874 -- 1936), \cite{odq}.}What we all dread most... is a maze with no centre.

\section{}

\subsection{}

\blfootnote{`The Send-Off', Wilfred Owen (1893 -- 1918), \cite{oxfordlarkin}.}\settowidth{\versewidth}{Down the close, darkening lanes they sang their way}
\begin{verse}[\versewidth]
Down the close, darkening lanes they sang their way\\*
To the siding-shed,\\*
And lined the train with faces grimly gay.\\!

Their breasts were stuck all white with wreath \& spray\\*
As men's are, dead.\\!

Dull porters watched them, and a casual tramp\\*
Stood staring hard,\\
Sorry to miss them from the upland camp.\\
Then, unmoved, signals nodded, and a lamp\\*
Winked to the guard.\\!

So secretly, like wrongs hushed-up, they went.\\*
They were not ours:\\*
We never heard to which front these were sent.\\!

Nor there if they yet mock what women meant\\*
Who gave them flowers.\\!

Shall they return to beatings of great bells\\*
In wild trainloads?\\
A few, a few, too few for drums \& yells,\\
May creep back, silent, to still village wells\\*
Up \sfrac{$1$}{$2$} known roads.
\end{verse}

\subsection{}

\blfootnote{William Shakespeare (1564 -- 1616), \cite{treasury}. This song is sung by Feste in \refbook{Twelfth Night} II.4.}\settowidth{\versewidth}{My shroud of white, stuck all with yew,}
\begin{verse}[\versewidth]
\vin Come away, come away, death,\\*
And in sad cypress let me be laid;\\
\vin Fly away, fly away, breath;\\
I am slain by a fair cruel maid.\\
My shroud of white, stuck all with yew,\\
\vin \vin O prepare it!\\
My part of death, no one so true\\*
\vin \vin Did share it.\\!

\vin Not a flower, not a flower sweet\\*
On my black coffin let there be strown;\\
\vin Not a friend, not a friend greet\\
My poor corpse, where my bones shall be thrown:\\
A 1000 1000 sighs to save,\\
\vin \vin Lay me O where\\
Sad true lover never find my grave,\\*
\vin \vin To weep there.
\end{verse}

\subsection{}

\blfootnote{Dr Charles Darwin (1809 -- 1882), \cite{odq}.}Ignorance more frequently begets confidence than does knowledge.

\section{}

\subsection{}

\blfootnote{`The Mill', Edwin Robinson (1869 -- 1935), \cite{norton}.}\settowidth{\versewidth}{    The tea was cold; the fire was dead;}
\begin{verse}[\versewidth]
The miller's wife had waited long;\\*
\vin The tea was cold; the fire was dead;\\
And there might yet be nothing wrong\\
\vin In how he went \& what he said:\\
`There are no millers any more,'\\
\vin Was all that she had heard him say;\\
And he had lingered at the door\\*
\vin So long that it seemed yesterday.\\!

Sick with a fear that had no form\\*
\vin She knew that she was there at last;\\
And in the mill there was a warm\\
\vin And mealy fragrance of the past.\\
What else there was would only seem\\
\vin To say again what he had meant;\\
And what was hanging from a beam\\*
\vin Would not have heeded where she went.\\!

And if she thought it followed her,\\*
\vin She may have reasoned in the dark\\
That one way of the few there were\\
\vin Would hide her \& would leave no mark:\\
Black water, smooth above the weir\\
\vin Like starry velvet in the night,\\
Though ruffled once, would soon appear\\*
\vin The same as ever to the sight.
\end{verse}

\subsection{}

\blfootnote{William Shakespeare (1564 -- 1616), \cite{treasury}. These lines are sung by the king's two sons in \refbook{Cymbeline} IV.2.}\settowidth{\versewidth}{    Thou art past the tyrant's stroke;}
\begin{verse}[\versewidth]
Fear no more the heat o' the sun,\\*
\vin Nor the furious winter's rages;\\
Thou thy worldy task hast done,\\
\vin Home art gone, and ta'en thy wages:\\
Golden lads \& girls all must,\\*
As chimney-sweepers, come to dust.\\!

Fear no more the frown o' the great;\\*
\vin Thou art past the tyrant's stroke;\\
Care no more to clothe \& eat;\\
\vin To thee the reed is as the oak:\\
The sceptre, learning, physic must\\*
All follow this, and come to dust.\\!

Fear no more the lightning flash,\\*
\vin Nor the all-dreaded thunder stone;\\
Fear not slander, censure rash;\\
\vin Thou hast finished joy \& moan:\\
All lovers young, all lovers must\\*
Consign to thee, and come to dust.
\end{verse}

\subsection{}

\blfootnote{Daniel Defoe (1660 -- 1731), \cite{odq}.}The good die early, and the bad die late.

\section{}

\subsection{}

\blfootnote{$\mathbb{R}$ William Shakespeare (1564 -- 1616), \cite{obev}. These lines are spoken by Portia in \refbook{The Merchant of Venice} IV.1}\settowidth{\versewidth}{Which if thou follow, this strict court of Venice}
\begin{verse}[\versewidth]
The quality of mercy is not strained;\\*
It droppeth as the gentle rain from heaven\\
Upon the place beneath. It is twice blest;\\
It blesseth him that gives \& him that takes:\\
'Tis mightiest in the mightiest; it becomes\\
The throned monarch better than his crown:\\
His sceptre shows the force of temporal power,\\
The attribute to awe \& majesty,\\
Wherein doth sit the dread \& fear of kings;\\
But mercy is above this sceptred sway;\\
It is enthron\`{e}d in the hearts of kings;\\
It is an attribute to God himself;\\
And earthly power doth then show likest God's\\
When mercy seasons justice. Therefore, jew,\\
Though justice be thy plea, consider this,\\
That, in the course of justice, none of us\\
Should see salvation: we do pray for mercy;\\
And that same prayer doth teach us all to render\\
The deeds of mercy. I have spoke thus much\\
To mitigate the justice of thy plea;\\
Which if thou follow, this strict court of \textsc{Venice}\\*
Must needs give sentence 'gainst the merchant there.
\end{verse}

\subsection{}

\blfootnote{William Shakespeare (1564 -- 1616), \cite{treasury}.}\settowidth{\versewidth}{    From thee, the pleasure of the fleeting year!}
\begin{verse}[\versewidth]
How like a winter hath my absence been\\*
\vin From thee, the pleasure of the fleeting year!\\
What freezings have I felt, what dark days seen,\\*
\vin What old december's bareness everywhere!\\!

And yet this time removed was summer's time:\\*
\vin The teeming autumn, big with rich increase,\\
Bearing the wanton burden of the prime\\*
\vin Like widowed wombs after their lords' decease:\\!

Yet this abundant issue seemed to me\\*
\vin But hope of orphans, \& unfathered fruit;\\
For summer \& his pleasures wait on thee,\\*
\vin And, thou away, the very birds are mute;\\!

Or if they sing, 'tis with so dull a cheer,\\*
That leaves look pale, dreading the winter's near.
\end{verse}

\subsection{}

\blfootnote{Thomas de Quincey (1785 -- 1859), \cite{odq}.}A conscience is a more expensive encumbrance than a wife or a carriage.

\section{}

\subsection{}

\blfootnote{$\mathbb{R}$ William Shakespeare (1564 -- 1616), \cite{obev}. This is a dialogue between Isabella and Claudio from \refbook{Measure for Measure} III.1.}\settowidth{\versewidth}{What says my brother? `Death is a fearful thing.'}
\begin{verse}[\versewidth]
What says my brother? `Death is a fearful thing.'\\*
And sham\`{e}d life a hateful.\\
`Ay, but to die, and go we know not where;\\
To lie in cold obstruction and to rot;\\
This sensible warm motion to become\\
A kneaded clod; and the delighted spirit\\
To bathe in fiery floods, or to reside\\
In thrilling region of thick-ribb\`{e}d ice;\\
To be imprisoned in the viewless winds,\\
And blown with restless violence round about\\
The pendent world; or to be worse than worst\\
Of those that lawless \& incertain thought\\
Imagine howling: 'tis too horrible!\\
The weariest \& most loath\`{e}d worldly life\\
That age, ache, penury \& imprisonment\\
Can lay on nature is a paradise\\*
To what we fear of death.'
\end{verse}

\subsection{}

\blfootnote{$\mathbb{R}$ William Shakespeare (1564 -- 1616), \cite{obev}. These lines form the eponymous villain's lament for his wife from \refbook{Macbeth} V.5.}\settowidth{\versewidth}{There would have been a time for such a word.}
\begin{verse}[\versewidth]
She should have died hereafter;\\*
There would have been a time for such a word.\\
Tomorrow \& tomorrow \& tomorrow,\\
Creeps in this petty pace from day to day\\
To the last syllable of recorded time,\\
And all our yesterdays have lighted fools\\
The way to dusty death. Out, out, brief candle!\\
Life's but a walking shadow, a poor player\\
That struts \& frets his hour upon the stage\\
And then is heard no more: it is a tale\\
Told by an idiot, full of sound \& fury,\\*
Signifying nothing.
\end{verse}

\subsection{}

\blfootnote{Charles Dickens (1812 -- 1870), \cite{odq}. These words are last two lines of Miss Dickinson's poem beginning `My life closed twice before its close'.}Parting is all who know of heaven, and all we need of hell.

\section{}

\subsection{}

\blfootnote{`The Flight of Love', Percy Shelley (1792 -- 1822), \cite{treasury}.}\settowidth{\versewidth}{Love first leaves the well-built nest;}
\begin{verse}[\versewidth]
\vin When the lamp is shattered\\*
The light in the dust lies dead --\\
\vin When the cloud is scattered\\
The rainbow's glory is shed.\\
\vin When the lute is broken,\\
Sweet tones are remembered not;\\
\vin When the lips have spoken,\\*
Loved accents are soon forgot.\\!

\vin As music \& splendor\\*
Survive not the lamp \& the lute,\\
\vin The heart's echoes render\\
No song when the spirit is mute:--\\
\vin No song but sad dirges,\\
Like the wind through a ruined cell,\\
\vin Or the mournful surges\\*
That ring the dead seaman's knell.\\!

\vin When hearts have once mingled\\*
Love first leaves the well-built nest;\\
\vin The weak one is singled\\
To endure what it once possessed.\\
\vin O love! who bewailest\\
The frailty of all things here,\\
\vin Why choose you the frailest\\*
For your cradle, your home, and your bier?\\!

\vin Its passions will rock thee\\*
As the storms rock the ravens on high;\\
\vin Bright reason will mock thee,\\
Like the sun from a wintry sky.\\
\vin From thy nest every rafter\\
Will rot, and thine eagle home\\
\vin Leave thee naked to laughter,\\*
When leaves fall \& cold winds come.
\end{verse}

\subsection{}

\blfootnote{Percy Shelley (1792 -- 1822), \cite{treasury}.}\settowidth{\versewidth}{A widow bird sate mourning for her love}
\begin{verse}[\versewidth]
A widow bird sate mourning for her love\\*
\vin Upon a wintry bough;\\
The frozen wind crept on above,\\*
\vin The freezing stream below.\\!

There was no leaf upon the forest bare.\\*
\vin No flower upon the ground,\\
And little motion in the air\\*
\vin Except the mill-wheel's sound.
\end{verse}

\subsection{}

\blfootnote{Charles Dickens (1812 -- 1870), \cite{odq}. These words are uttered by Mr Tappertit in \refbook{Barnaby Rudge}, chapter 22.}There are strings... in the human heart that had better not be vibrated.

\section{}

\subsection{}

\blfootnote{`The third and fourth verses of this poem appear in Mrs Mary Shelley's \emph{Frankenstein}, without attribution to her husband; it is unclear whether or not she was their genuine author.', Percy Shelley (1792 -- 1822), \cite{norton}.}\settowidth{\versewidth}{    Night closes round, and they are lost forever:}
\begin{verse}[\versewidth]
We are as clouds that veil the midnight moon;\\*
\vin How restlessly they speed, \& gleam, \& quiver,\\
Streaking the darkness radiantly -- yet soon\\*
\vin Night closes round, and they are lost forever:\\!

Or like forgotten lyres, whose dissonant strings\\*
\vin Give various response to each varying blast,\\
To whose frail frame no second motion brings\\*
\vin One mood or modulation like the last.\\!

We rest. A dream has power to poison sleep;\\*
\vin We rise. One wandering thought pollutes the day;\\
We feel, conceive or reason, laugh or weep;\\*
\vin Embrace fond woe, or cast our cares away:\\!

It is the same. For, be it joy or sorrow,\\*
\vin The path of its departure still is free:\\
Man's yesterday may ne'er be like his morrow;\\*
\vin Nought may endure but mutability.
\end{verse}

\subsection{}

\blfootnote{`A Complaint by Night of the Lover not Beloved', Percy Shelley (1792 -- 1822), \cite{pbev}. The title Somerset Maugham's novel \refbook{The Painted Veil} is drawn from this sonnet.}\settowidth{\versewidth}{    Their shadows, o'er the chasm, sightless drear.}
\begin{verse}[\versewidth]
Lift not the painted veil which those who live\\*
\vin Call life: though unreal shapes be pictured there,\\
And it but mimic all we would believe\\
\vin With colours idly spread -- behind, lurk fear\\
And hope, twin destinies; who ever weave\\
\vin Their shadows, o'er the chasm, sightless \& drear.\\
I knew one who had lifted it -- he sought,\\
\vin For his lost heart was tender, things to love,\\
But found them not, alas, nor was there aught\\
\vin The world contains, the which he could approve.\\
\vin Through the unheeding many he did move,\\
\vin \vin A splendour among shadows, a bright blot\\
\vin Upon this gloomy scene, a spirit that strove\\*
\vin \vin For truth, and like the preacher found it not.
\end{verse}

\subsection{}

\blfootnote{Lorenzo Dow (1777 -- 1834), \cite{odq}. Dow was speaking specifically about the Calvinist doctrine of }You will be damned if you do, and you will be damned if you don't.

\section{}

\subsection{}

\blfootnote{`The Leper', Algernon Swinburne (1837 -- 1909), \cite{obev}. As Swinburne's own note indicates, this poem is a retelling of a digression in the \refbook{Grand Chroniques de France, 1505}.}\settowidth{\versewidth}{    This curse to plague her, a curse of his.}
\begin{verse}[\versewidth]
Nothing is better, I well think,\\*
\vin Than love; the hidden well-water\\
Is not so delicate to drink:\\*
\vin This was well seen of me \& her.\\!

I served her in a royal house;\\*
\vin I served her wine \& curious meat.\\
For will to kiss between her brows,\\*
\vin I had no heart to sleep or eat.\\!

Mere scorn God knows she had of me,\\*
\vin A poor scribe, nowise great or fair,\\
Who plucked his clerk's hood back to see\\*
\vin Her curled-up lips \& amorous hair.\\!

I vex my head with thinking this.\\*
\vin Yea, though God always hated me,\\
And hates me now that I can kiss\\*
\vin Her eyes, plait up her hair to see\\!

How she then wore it on the brows,\\*
\vin Yet am I glad to have her dead\\
Here in this wretched wattled house\\*
\vin Where I can kiss her eyes \& head.\\!

Nothing is better, I well know,\\*
\vin Than love; no amber in cold sea\\
Or gathered berries under snow:\\*
\vin That is well seen of her \& me.\\!

Three thoughts I make my pleasure of:\\*
\vin First I take heart \& think of this:\\
That knight's gold hair she chose to love,\\*
\vin His mouth she had such will to kiss.\\!

Then I remember that sundawn\\*
\vin I brought him by a privy way\\
Out at her lattice, and thereon\\*
\vin What gracious words she found to say.\\!

(Cold rushes for such little feet —\\*
\vin Both feet could lie into my hand.\\
A marvel was it of my sweet\\*
\vin Her upright body could so stand.)\\!

`Sweet friend, God give you thank \& grace;\\*
\vin Now am I clean \& whole of shame,\\
Nor shall men burn me in the face\\*
\vin For my sweet fault that scandals them.'\\!

I tell you over word by word.\\*
\vin She, sitting edgewise on her bed,\\
Holding her feet, said thus. The third,\\*
\vin A sweeter thing than these, I said.\\!

God, that makes time and ruins it\\*
\vin And alters not, abiding God,\\
Changed with disease her body sweet,\\*
\vin The body of love wherein she abode.\\!

Love is more sweet \& comelier\\*
\vin Than a dove's throat strained out to sing.\\
All they spat out and cursed at her\\*
\vin And cast her forth for a base thing.\\!

They cursed her, seeing how God had wrought\\*
\vin This curse to plague her, a curse of his.\\
Fools were they surely, seeing not\\*
\vin How sweeter than all sweet she is.\\!

He that had held her by the hair,\\*
\vin With kissing lips blinding her eyes,\\
Felt her bright bosom, strained \& bare,\\*
\vin Sigh under him, with short mad cries\\!

Out of her throat \& sobbing mouth\\*
\vin And body broken up with love,\\
With sweet hot tears his lips were loth\\*
\vin Her own should taste the savour of,\\!

Yea, he inside whose grasp all night\\*
\vin Her fervent body leapt or lay,\\
Stained with sharp kisses red \& white,\\*
\vin Found her a plague to spurn away.\\!

I hid her in this wattled house,\\*
\vin I served her water \& poor bread.\\
For joy to kiss between her brows\\*
\vin Time upon time I was nigh dead.\\!

Bread failed; we got but well-water\\*
\vin And gathered grass with dropping seed.\\
I had such joy of kissing her,\\*
\vin I had small care to sleep or feed.\\!

Sometimes when service made me glad\\*
\vin The sharp tears leapt between my lids,\\
Falling on her, such joy I had\\*
\vin To do the service God forbids.\\!

`I pray you let me be at peace;\\*
\vin Get hence, make room for me to die.'\\
She said that: her poor lip would cease,\\*
\vin Put up to mine, and turn to cry.\\!

I said, `Bethink yourself how love\\*
\vin Fared in us twain, what either did;\\
Shall I unclothe my soul thereof?\\*
\vin That I should do this, God forbid.'\\!

Yea, though God hateth us, he knows\\*
\vin That hardly in a little thing\\
Love faileth of the work it does\\*
\vin Till it grow ripe for gathering.\\!

Six months, and now my sweet is dead\\*
\vin A trouble takes me; I know not\\
If all were done well, all well said,\\*
\vin No word or tender deed forgot.\\!

Too sweet, for the least part in her,\\*
\vin To have shed life out by fragments; yet,\\
Could the close mouth catch breath and stir,\\*
\vin I might see something I forget.\\!

Six months, and I sit still and hold\\*
\vin In two cold palms her cold two feet.\\
Her hair, \sfrac{$1$}{$2$} grey \sfrac{$1$}{$2$} ruined gold,\\*
\vin Thrills me and burns me in kissing it.\\!

Love bites and stings me through, to see\\*
\vin Her keen face made of sunken bones.\\
Her worn-off eyelids madden me,\\*
\vin That were shot through with purple once.\\!

She said, `Be good with me; I grow\\*
\vin So tired for shame's sake, I shall die\\
If you say nothing:' even so.\\*
\vin And she is dead now, and shame put by.\\!

Yea, and the scorn she had of me\\*
\vin In the old time, doubtless vexed her then.\\
I never should have kissed her. See\\*
\vin What fools God's anger makes of men!\\!

She might have loved me a little too,\\*
\vin Had I been humbler for her sake.\\
But that new shame could make love new\\*
\vin She saw not -- yet her shame did make.\\!

I took too much upon my love,\\*
\vin Having for such mean service done\\
Her beauty \& all the ways thereof,\\*
\vin Her face \& all the sweet thereon.\\!

Yea, all this while I tended her,\\*
\vin I know the old love held fast his part:\\
I know the old scorn waxed heavier,\\*
\vin Mixed with sad wonder, in her heart.\\!

It may be all my love went wrong --\\*
\vin A scribe's work writ awry and blurred,\\
Scrawled after the blind evensong --\\*
\vin Spoilt music with no perfect word.\\!

But surely I would fain have done\\*
\vin All things the best I could. Perchance\\
Because I failed, came short of one,\\*
\vin She kept at heart that other man's.\\!

I am grown blind with all these things:\\*
\vin It may be now she hath in sight\\
Some better knowledge; still there clings\\*
\vin The old question. Will not God do right?
\end{verse}

\subsection{}

\blfootnote{``A Land Dirge'', John Webster (1580 -- 1634), \cite{treasury}.}\settowidth{\versewidth}{To rear him hillocks that shall keep him warm}
\begin{verse}[\versewidth]
Call for the robin-redbreast \& the wren,\\*
Since o'er shady groves they hover\\
And with leaves \& flowers do cover\\
The friendless bodies of unburied men.\\
Call unto his funeral dole\\
The ant, the field-mouse, \& the mole\\
To rear him hillocks that shall keep him warm\\
And (when gay tombs are robbed) sustain no harm:\\
But keep the wolf far thence, that's foe to men,\\*
For with his nails he'll dig them up again.
\end{verse}

\subsection{}

\blfootnote{Sir Arthur Doyle (1859 -- 1930), \cite{odq}.}Where there is no imagination there is no horror.

\section{}

\subsection{}

\blfootnote{`Rain', Edward Thomas (1878 -- 1917), \cite{norton}.}\settowidth{\versewidth}{Rain, midnight rain, nothing but the wild rain}
\begin{verse}[\versewidth]
Rain, midnight rain, nothing but the wild rain\\*
On this bleak hut, \& solitude, and me\\
Remembering again that I shall die\\
And neither hear the rain nor give it thanks\\
For washing me cleaner than I have been\\
Since I was born into solitude.\\
Bless\`{e}d are the dead that the rain rains upon:\\
But here I pray that none whom once I loved\\
Is dying tonight or lying still awake\\
Solitary, listening to the rain,\\
Either in pain or thus in sympathy\\
Helpless among the living and the dead,\\
Like a cold water among broken reeds,\\
Myriads of broken reeds all still \& stiff,\\
Like me who have no love which this wild rain\\
Has not dissolved except the love of death,\\
If love it be towards what is perfect and\\*
Cannot, the tempest tells me, disappoint.
\end{verse}

\subsection{}

\blfootnote{Dr William Wordsworth, Poet Laureate (1770 -- 1850), \cite{norton}. \P 13. Proteus and Triton are minor aquatic deities from Greek mythology who appear in Homer's {\textit Ὀδύσσεια} and Hesiod's {\textit Θεογονία} respectively.}\settowidth{\versewidth}{We have given our hearts away, a sordid boon!}
\begin{verse}[\versewidth]
The world is too much with us; late \& soon,\\*
Getting \& spending, we lay waste our powers;\\
Little we see in nature that is ours;\\
We have given our hearts away, a sordid boon!\\
This sea that bares her bosom to the moon,\\
The winds that will be howling at all hours,\\
And are up-gathered now like sleeping flowers,\\
For this, for everything, we are out of tune;\\
It moves us not. Great God! I'd rather be\\
A pagan suckled in a creed outworn;\\
So might I, standing on this pleasant lea,\\
Have glimpses that would make me less forlorn;\\
Have sight of \textit{Proteus} rising from the sea;\\*
Or hear old \textit{Triton} blow his wreath\`{e}d horn.
\end{verse}

\subsection{}

\blfootnote{Gabriel Rossetti (1828 -- 1882), \cite{odq}. This quotation forms the refrain of a ballad, which was translated from a French poem, \refpoem{Ballad des dames du temps jadis}, by Fran\c{c}ois Villon. The French is, `Mais o\`u sont les neiges d'antan?'}Where are the snows of yesteryear?

\chapter{Duodecember}

\section{}

\subsection{}

\blfootnote{`A Nocturnal upon St Lucy's Day, Being the Shortest Day', The Very Rev Dr John Donne (1572 -- 1631), \cite{norton}. St Lucy's day falls on the thirteenth day of December in both the Julian and Gregorian calendars. Christmas, the twenty-fifth of December, being a kind of successor to a Roman festival in honour of the sun, was intended to fall on (or very close to) the winter solstice; however, due to the slight failings of the Julian calendar, by the seventeenth century the solstice actually occurred on the thirteenth. The Gregorian reforms essentially rectified the situation, although they've also anachronised a rather beautiful poem. \P 39. The `goat' refers primarily to Aries, the sign of the zodiac corresponding to spring.}\settowidth{\versewidth}{Of absence, darkness, death: things which are not.}
\begin{verse}[\versewidth]
'Tis the year's midnight, and it is the day's,\\*
\textit{Lucy}'s, who scarce seven hours herself unmasks;\\
\vin The sun is spent, and now his flasks\\
\vin Send forth light squibs, no constant rays;\\
\vin \vin The world's whole sap is sunk;\\
The general balm th'hydroptic earth hath drunk,\\
Whither, as to the bed's feet, life is shrunk,\\
Dead \& interred; yet all these seem to laugh,\\*
Compared with me, who am their epitaph.\\!

Study me then, you who shall lovers be\\*
At the next world, that is, at the next spring;\\
\vin For I am every dead thing,\\
\vin In whom love wrought new alchemy.\\
\vin \vin For his art did express\\
A quintessence even from nothingness,\\
From dull privations, and lean emptiness;\\
He ruined me, and I am re-begot\\*
Of absence, darkness, death: things which are not.\\!

All others, from all things, draw all that's good,\\*
Life, soul, form, spirit, whence they being have;\\
\vin I, by love's limbeck, am the grave\\
\vin Of all that's nothing. Oft a flood\\
\vin \vin Have we two wept, and so\\
Drowned the whole world, us two; oft did we grow\\
To be two chaoses, when we did show\\
Care to aught else; and often absences\\*
Withdrew our souls, and made us carcasses.\\!

But I am by her death (which word wrongs her)\\*
Of the first nothing the elixir grown;\\
\vin Were I a man, that I were one\\
\vin I needs must know; I should prefer,\\
\vin \vin If I were any beast,\\
Some ends, some means; yea plants, yea stones detest,\\
And love; all, all some properties invest;\\
If I an ordinary nothing were,\\*
As shadow, a light and body must be here.\\!

But I am none; nor will my sun renew.\\*
You lovers, for whose sake the lesser sun\\
\vin At this time to the goat is run\\
\vin To fetch new lust, and give it you,\\
\vin \vin Enjoy your summer all;\\
Since she enjoys her long night's festival,\\
Let me prepare towards her, and let me call\\
This hour her vigil, and her eve, since this\\*
Both the year's, and the day's deep midnight is.
\end{verse}

\subsection{}

\blfootnote{`On My First Son', Ben Jonson (1572 -- 1637), \cite{norton}. The name Benjamin, the name of the departed child, means `son of my right hand'.}\settowidth{\versewidth}{Seven years tho' wert lent to me, and I thee pay,}
\begin{verse}[\versewidth]
Farewell, thou child of my right hand, and joy;\\*
My sin was too much hope of thee, loved boy.\\
Seven years tho' wert lent to me, and I thee pay,\\
Exacted by thy fate, on the just day.\\
O could I lose all father now! For why\\
Will man lament the state he should envy?\\
To have so soon 'scaped world's \& flesh's rage,\\
And if no other misery, yet age?\\
Rest in soft peace, and, asked, say, Here doth lie\\
\textit{Ben Jonson} his best piece of poetry --\\
For whose sake henceforth all his vows be such,\\*
As what he loves may never like too much.
\end{verse}

\subsection{}

\blfootnote{The Rev Robert Burton (1577 -- 1640), \cite{odq}.}A nightingale... dies for shame if another bird sings better.

\section{}

\subsection{}

\blfootnote{The Very Rev Dr John Donne (1572 -- 1631), \cite{norton}. The line about the `mandrake root' is puzzling. Surely it should be, `Get with child \emph{by} mandrake root' since mandrakes were once believed to have aphrodisiac and fertilising qualities (as per Genesis 30), an old wives' tale similar to the legends about Ulysses and the sirens or, indeed, a faithful beautiful woman? Or is the Very Rev Dr Donne genuinely inviting the reader to ejaculate into a plant?}\settowidth{\versewidth}{    Till age snow white hairs on thee;}
\begin{verse}[\versewidth]
Go \& catch a falling star;\\*
\vin Get with child a mandrake root;\\
Tell me where all past years are,\\
\vin Or who cleft the devil's foot;\\
Teach me to hear mermaids singing,\\
Or to keep off envy's stinging,\\
\vin \vin And find\\
\vin \vin What wind\\*
Serves to advance an honest mind.\\!

If thou be'st born to strange sights,\\*
\vin Things invisible to see,\\
Ride 10,000 days \& nights,\\
\vin Till age snow white hairs on thee;\\
Thou, when thou return'st, wilt tell me,\\
All strange wonders that befell thee,\\
\vin \vin And swear:\\
\vin \vin Nowhere\\*
Lives a woman true, and fair.\\!

If thou find'st one, let me know;\\*
\vin Such a pilgrimage were sweet;\\
Yet do not; I would not go,\\
\vin Though at next door we might meet;\\
Though she were true when you met her,\\
And last till you write your letter,\\
\vin \vin Yet she\\
\vin \vin Will be\\*
False, ere I come, to two, or three.
\end{verse}

\subsection{}

\blfootnote{`On My First Daughter', Ben Jonson (1572 -- 1637), \cite{norton}.}\settowidth{\versewidth}{Yet all heaven's gifts being heaven's due,}
\begin{verse}[\versewidth]
Here lies, to each her parents' ruth,\\*
\textit{Mary}, the daughter of their youth;\\
Yet all heaven's gifts being heaven's due,\\
It makes the father less to rue.\\
At six months' end she parted hence\\
With safety of her innocence;\\
Whose soul heaven's queen, whose name she bears,\\
In comfort of her mother's tears,\\
Hath placed amongst her virgin-train:\\
Where, while that severed doth remain,\\
This grave partakes the fleshly birth;\\*
Which cover lightly, gentle earth.
\end{verse}

\subsection{}

\blfootnote{The Rt Hon Edmund Burke (1729 -- 1797), \cite{odq}.}In the groves of their academy, at the end of every vista, you see nothing but the gallows.

\section{}

\subsection{}

\blfootnote{`The Relic', The Very Rev Dr John Donne (1572 -- 1631), \cite{norton}. \P 80. The Almanacker cannot help noticing that ``Jesus Christ'' would scan just as well as `something else', though this is pure speculation.}\settowidth{\versewidth}{Who thought that this device might be some way}
\begin{verse}[\versewidth]
\vin When my grave is broke up again,\\*
\vin Some second guest to entertain\\
\vin (For graves have learned that womanhead,\\
\vin To be to more than one a bed)\\
\vin \vin And he that digs it spies\\
A bracelet of bright hair about the bone,\\
\vin \vin Will he not let'us alone,\\
And think that there a loving couple lies,\\
Who thought that this device might be some way\\
To make their souls, at the last busy day,\\*
Meet at this grave, and make a little stay?\\!

\vin If this fall in a time, or land,\\*
\vin Where misdevotion doth command,\\
\vin Then he, that digs us up, will bring\\
\vin Us to the bishop \& the king,\\
\vin \vin To make us relics; then\\
Thou shalt be a \textit{Mary Magdalen}, and I\\
\vin \vin A something else thereby;\\
All women shall adore us, and some men;\\
And since at such time miracles are sought,\\
I would have that age by this paper taught\\*
What miracles we harmless lovers wrought.\\!

\vin First, we loved well and faithfully,\\*
\vin Yet knew not what we loved, nor why;\\
\vin Difference of sex no more we knew\\
\vin Than our guardian angels do;\\
\vin \vin Coming \& going, we\\
Perchance might kiss, but not between those meals;\\
\vin \vin Our hands ne'er touched the seals\\
Which nature, injured by late law, sets free;\\
These miracles we did, but now alas,\\
All measure, and all language, I should pass,\\*
Should I tell what a miracle she was.
\end{verse}

\subsection{}

\blfootnote{`Upon a Child that Died', Robert Herrick (1591 -- 1674), \cite{norton}. This little poem was clearly influenced by Ben Jonson's \refpoem{On My First Daughter}.}\settowidth{\versewidth}{Give her strewings, but not stir}
\begin{verse}[\versewidth]
Here she lies, a pretty bud,\\*
Lately made of flesh \& blood,\\
Who as soon fell fast asleep\\
As her little eyes did peep.\\
Give her strewings, but not stir\\*
The earth that lightly covers her.
\end{verse}

\subsection{}

\blfootnote{Prof William Goodwin (1831 -- 1912), \cite{agamemnon}. Professor Goodwin is here translating a line from Aeschylus's Ἀγαμέμνων.}Justice brings knowledge within the reach of those who have suffered.

\section{}

\subsection{}

\blfootnote{The Rev Charles Wolfe (1791 -- 1823), \cite{treasury}. Lieutenant General Sir John Moore died of wounds, having led his men into battle, at the battle of Corunna in the Peninsular War.}\settowidth{\versewidth}{But we steadfastly gazed on the face that was dead,}
\begin{verse}[\versewidth]
Not a drum was heard, not a funeral note\\*
\vin As his corpse to the rampart we hurried;\\
Not a soldier discharged his farewell shot\\*
\vin O'er the grave where our hero we buried.\\!

We buried him darkly at dead of night,\\*
\vin The sods with our bayonets turning,\\
By the struggling moonbeam's misty light\\*
\vin And the lantern dimly burning.\\!

No useless coffin enclosed his breast,\\*
\vin Not in sheet or in shroud we wound him;\\
But he lay like a warrior taking his rest\\*
\vin With his martial cloak around him.\\!

Few \& short were the prayers we said,\\*
\vin And we spoke not a word of sorrow;\\
But we steadfastly gazed on the face that was dead,\\*
\vin And we bitterly thought of the morrow.\\!

We thought, as we hollowed his narrow bed\\*
\vin And smoothed down his lonely pillow,\\
That the foe \& the stranger would tread o'er his head,\\*
\vin And we far away on the billow!\\!

Lightly they'll talk of the spirit that's gone,\\*
\vin And o'er his cold ashes upbraid him --\\
But little he'll reck, if they let him sleep on\\*
\vin In the grave where a briton has laid him.\\!

But \sfrac{$1$}{$2$} of our heavy task was done\\*
\vin When the clock struck the hour for retiring;\\
And we heard the distant \& random gun\\*
\vin That the foe was sullenly firing.\\!

Slowly \& sadly we laid him down,\\*
\vin From the field of his fame fresh \& gory;\\
We carved not a line, and we raised not a stone,\\*
\vin But we left him alone with his glory.
\end{verse}

\subsection{}

\blfootnote{`To His Love', Ivor Gurney (1890 -- 1937), \cite{norton}.}\settowidth{\versewidth}{    Thing I must somehow forget.}
\begin{verse}[\versewidth]
He's gone, and all our plans\\*
\vin Are useless indeed.\\
We'll walk no more on Cotswold\\
\vin Where the sheep feed\\*
\vin Quietly and take no heed.\\!

His body that was so quick\\*
\vin Is not as you\\
Knew it, on \textsc{Severn River}\\
\vin Under the blue\\*
\vin Driving our small boat through.\\!

You would not know him now...\\*
\vin But still he died\\
Nobly, so cover him over\\
\vin With violets of pride\\*
\vin Purple from \textsc{Severn} side.\\!

Cover him! Cover him soon!\\*
\vin And with thick-set\\
Masses of memoried flowers\\
\vin Hide that red wet\\*
\vin Thing I must somehow forget.
\end{verse}

\subsection{}

\blfootnote{Dion Boursiquot (1820 -- 1890), \cite{odq}.}Men talk of killing time, while time quietly kills them.

\section{}

\subsection{}

\blfootnote{`Another: A Black patch on Lucasta's Face', Col Richard Lovelace (1617 -- 1657), \cite{obev}.}\settowidth{\versewidth}{Whose ugly night seemed masked with days' skreen.}
\begin{verse}[\versewidth]
As I beheld a winter's evening air,\\*
Curled in her court false locks of living hair,\\*
Buttered with jessamine the sun left there.\\!

Galliard \& clinquant she appeared to give,\\*
A serenade or ball to us that grieve,\\*
And teach us {\hoskeroe \`{a} la mode} more gently live.\\!

But as a moor, who to her cheeks prefers\\*
White spots, t'allure her black idolaters,\\*
Me thought she looked all o'er bepatched with stars.\\!

Like the dark front of some ethiopian queen,\\*
Veiled all o'er with gems of red, blue, green,\\*
Whose ugly night seemed masked with days' skreen.\\!

Whilst the fond people offered sacrifice\\*
To sapphires, 'stead of veins \& arteries,\\*
And bowed unto the diamonds, not her eyes.\\!

Behold \textit{Lucasta}'s face, how't glows like noon!\\*
A sun entire is her complexion,\\*
And formed of one whole constellation.\\!

So gently shining, so serene, so clear,\\*
Her look doth universal nature cheer;\\*
Only a cloud or two hangs here \& there.
\end{verse}

\subsection{}

\blfootnote{`Stopping by Woods on a Snowy Evening', Robert Frost, Poet Laureate of Vermont (1874 -- 1963), \cite{norton}.}\settowidth{\versewidth}{Whose woods these are I think I know.}
\begin{verse}[\versewidth]
Whose woods these are I think I know.\\*
His house is in the village though;\\
He will not see me stopping here\\*
To watch his woods fill up with snow.\\!

My little horse must think it queer\\*
To stop without a farmhouse near\\
Between the woods \& frozen lake\\*
The darkest evening of the year.\\!

He gives his harness bells a shake\\*
To ask if there is some mistake.\\
The only other sound's the sweep\\*
Of easy wind \& downy flake.\\!

The woods are lovely, dark \& deep,\\*
But I have promises to keep,\\
And miles to go before I sleep,\\*
And miles to go before I sleep.
\end{verse}

\subsection{}

\blfootnote{Mrs Daisy Devlin (1881 -- 1972), \cite{odq}.}My life will be sour grapes and ashes without you.

\section{}

\subsection{}

\blfootnote{`Dulce et Decorum Est', Wilfred Owen (1893 -- 1918), \cite{ptmgmc}. The Latin phrase is from Horace (\refbook{Carmina} III.2); it means, `It is sweet and fitting to die for one's country.' Although credited to Owen, the poem was written in close collaboration with Siegried Sassoon. It is sometimes said to be a response to the poetry of Jessie Pope.}\settowidth{\versewidth}{But limped on, blood-shod. All went lame; all blind;}
\begin{verse}[\versewidth]
Bent double, like old beggars under sacks,\\*
Knock-kneed, coughing like hags, we cursed through sludge,\\
Till on the haunting flares we turned our backs,\\
And towards our distant rest began to trudge.\\
Men marched asleep. Many had lost their boots,\\
But limped on, blood-shod. All went lame; all blind;\\
Drunk with fatigue; deaf even to the hoots\\*
Of gas-shells dropping softly behind.\\!

Gas! Gas! Quick, boys! An ecstasy of fumbling\\*
Fitting the clumsy helmets just in time,\\
But someone still was yelling out and stumbling\\
And flound'ring like a man in fire or lime.\\
Dim through the misty panes \& thick green light,\\*
As under a green sea, I saw him drowning.\\!

In all my dreams before my helpless sight,\\*
He plunges at me, guttering, choking, drowning.\\!

If in some smothering dreams, you too could pace\\*
Behind the wagon that we flung him in,\\
And watch the white eyes writhing in his face,\\
His hanging face, like a devil's sick of sin;\\
If you could hear, at every jolt, the blood\\
Come gargling from the froth-corrupted lungs,\\
Obscene as cancer, bitter as the cud\\
Of vile, incurable sores on innocent tongues --\\
My friend, you would not tell with such high zest\\
To children ardent for some desperate glory\\
The old lie: {\hoskeroe Dulce et decorum est\\*
Pro patria mori}.
\end{verse}

\subsection{}

\blfootnote{`The Oft-Repeated Dream', Robert Frost, Poet Laureate of Vermont (1874 -- 1963), \cite{norton}. This is part of a sequence of poems called \refpoem{The Hill Wife}.}\settowidth{\versewidth}{Was afraid in an oft-repeated dream}
\begin{verse}[\versewidth]
She had no saying dark enough\\*
\vin For the dark pine that kept\\
Forever trying the window latch\\*
\vin Of the room where they slept.\\!

The tireless but ineffectual hands\\*
\vin That with every futile pass\\
Made the great tree seem as a little bird\\*
\vin Before the mystery of glass.\\!

It never had been inside the room,\\*
\vin And only one of the two\\
Was afraid in an oft-repeated dream\\*
\vin Of what the tree might do.
\end{verse}

\subsection{}

\blfootnote{John Dalberg-Acton, 1st Baron Acton (1834 -- 1902), \cite{odq}.}Power tends to corrupt and absolute power corrupts absolutely.

\section{}

\subsection{}

\blfootnote{`Disabled', Wilfred Owen (1893 -- 1918), \cite{oxfordlarkin}. \P 12. Owen is generally considered to have been a homosexual, but this line forces the Almanacker to doubt -- to reconsider at least -- this hypothesis. \P 19. Nineteen was a significant age, since this was the youngest age at which a soldier could be sent to the front line.}\settowidth{\versewidth}{Girls' waists are, or how warm their subtle hands.}
\begin{verse}[\versewidth]
He sat in a wheeled chair, waiting for dark,\\*
And shivered in his ghastly suit of grey,\\
Legless, sewn short at elbow. Through the park\\
Voices of boys rang saddening like a hymn,\\
Voices of play \& pleasure after day,\\*
Till gathering sleep had mothered them from him.\\!

About this time town used to swing so gay\\*
When glow-lamps budded in the light blue trees,\\
And girls glanced lovelier as the air grew dim --\\
In the old times, before he threw away his knees.\\
Now he will never feel again how slim\\
Girls' waists are, or how warm their subtle hands.\\*
All of them touch him like some queer disease.\\!

There was an artist silly for his face,\\*
For it was younger than his youth, last year.\\
Now, he is old; his back will never brace;\\
He's lost his colour very far from here,\\
Poured it down shell-holes till the veins ran dry,\\
And \sfrac{$1$}{$2$} his lifetime lapsed in the hot race\\*
And leap of purple spurted from his thigh.\\!

One time he liked a blood-smear down his leg,\\*
After the matches, carried shoulder-high.\\
It was after football, when he'd drunk a peg,\\
He thought he'd better join. He wonders why.\\
Someone had said he'd look a god in kilts,\\
That's why; and maybe, too, to please his \textit{Meg},\\
Aye, that was it, to please the giddy jilts\\
He asked to join. He didn't have to beg;\\
Smiling they wrote his lie: aged 19 years.\\
Germans he scarcely thought of; all their guilt,\\
And Austria's, did not move him. And no fears\\
Of fear came yet. He thought of jewelled hilts\\
For daggers in plaid socks; of smart salutes;\\
And care of arms; and leave; and pay arrears;\\
{\hoskeroe Esprit de corps}; and hints for young recruits.\\*
And soon, he was drafted out with drums \& cheers.\\!

Some cheered him home, but not as crowds cheer, Goal!\\*
Only a solemn man who brought him fruits\\*
Th\'{a}nked him; and then enquired about his soul.\\!

Now, he will spend a few sick years in institutes,\\*
And do what things the rules consider wise,\\
And take whatever pity they may dole.\\
Tonight he noticed how the women's eyes\\
Passed from him to the strong men that were whole.\\
How cold \& late it is. Why don't they come\\*
And put him into bed? Why don't they come?
\end{verse}

\subsection{}

\blfootnote{Prof Alfred Housman (1859 -- 1936), \cite{norton}.}\settowidth{\versewidth}{    But young men think it is, and we were young.}
\begin{verse}[\versewidth]
Here dead lie we because we did not choose\\*
\vin To live and shame the land from which we sprung.\\
Life, to be sure, is nothing much to lose,\\*
\vin But young men think it is, and we were young.
\end{verse}

\subsection{}

\blfootnote{Mrs Amelia Barr (1831 -- 1919), \cite{odq}.}The fate of love is that it always seems too little or too much.

\section{}

\subsection{}

\blfootnote{Sir Thomas Wyatt (1503 -- 1542), \cite{treasury}.}\settowidth{\versewidth}{    That sometime they put themselves in danger}
\begin{verse}[\versewidth]
They flee from me that sometime did me seek\\*
\vin With naked foot stalking in my chamber.\\
I have seen them gentle, tame and meek\\
\vin That now are wild and do not remember.\\
\vin That sometime they put themselves in danger\\
To take bread at my hand; and now they range\\*
Busily seeking with a continual change.\\!

Thanked be fortune, it hath been otherwise\\*
\vin Twenty times better, but once in special\\
In thin array after a pleasant guise\\
\vin When her loose gown from her shoulders did fall,\\
\vin And me she caught in her arms long \& small;\\
Therewithal sweetly did me kiss,\\*
And softly said, Dear heart, how like you this?\\!

It was no dream: I lay broad waking.\\*
\vin But all is turned thorough my gentleness\\
Into a strange fashion of forsaking;\\
\vin And I have leave to go of her goodness\\
\vin And she also to use newfangleness.\\
But since that I so kindly am served,\\*
I would fain know what she hath deserved.
\end{verse}

\subsection{}

\blfootnote{Prof Alfred Housman (1859 -- 1936), \cite{norton}.}\settowidth{\versewidth}{    The lightfoot boys are laid;}
\begin{verse}[\versewidth]
With rue my heart is laden\\*
\vin For golden friends I had,\\
For many a rose-lipped maiden\\*
\vin And many a lightfoot lad.\\!

By brooks too broad for leaping\\*
\vin The lightfoot boys are laid;\\
The rose-lipped girls are sleeping\\*
\vin In fields where roses fade.
\end{verse}

\subsection{}

\blfootnote{The Rt Hon Joseph Addison (1672 -- 1719), \cite{odq}.}The post of honour is a private station.

\section{}

\subsection{}

\blfootnote{`Dover Beach', Prof Matthew Arnold (1822 -- 1888), \cite{norton}.}\settowidth{\versewidth}{Of pebbles which the waves draw back, and fling,}
\begin{verse}[\versewidth]
The sea is calm tonight.\\*
The tide is full; the moon lies fair\\
Upon the straits; on the french coast the light\\
Gleams \& is gone; the cliffs of England stand,\\
Glimmering \& vast, out in the tranquil bay.\\
Come to the window; sweet is the night-air.\\
Only, from the long line of spray\\
Where the sea meets the moon-blanched land,\\
Listen. You hear the grating roar\\
Of pebbles which the waves draw back, and fling,\\
At their return, up the high strand,\\
Begin, and cease, and then again begin,\\
With tremulous cadence slow, and bring\\*
The eternal note of sadness in.\\!

\textit{Sophocles} long ago\\*
Heard it on the Aegean, and it brought\\
Into his mind the turbid ebb \& flow\\
Of human misery; we\\
Find also in the sound a thought,\\*
Hearing it by this distant northern sea.\\!

The sea of faith\\*
Was once, too, at the full, and round earth's shore\\
Lay like the folds of a bright girdle furled.\\
But now I only hear\\
Its melancholy, long, withdrawing roar,\\
Retreating, to the breath\\
Of the night-wind, down the vast edges drear\\*
And naked shingles of the world.\\!

Ah, love, let us be true\\*
To one another, for the world, which seems\\
To lie before us like a land of dreams,\\
So various, so beautiful, so new,\\
Hath really neither joy, nor love, nor light,\\
Nor certitude, nor peace, nor help for pain;\\
And we are here as on a darkling plain\\
Swept with confused alarms of struggle \& flight,\\*
Where ignorant armies clash by night.
\end{verse}

\subsection{}

\blfootnote{Dr Thomas Campion (1567 -- 1620), \cite{norton}.}\settowidth{\versewidth}{    Of masques revels which sweet youth did make,}
\begin{verse}[\versewidth]
When thou must home to shades of underground,\\*
\vin And there arrived, a new admir\`{e}d guest,\\
The beauteous spirits do engirt thee round,\\
\vin White \textit{Iope}, blithe \textit{Helen}, \& the rest,\\
To hear the stories of thy finished love\\*
From that smooth tongue whose music hell can move;\\!

Then wilt thou speak of banqueting delights,\\*
\vin Of masques \& revels which sweet youth did make,\\
Of tourneys \& great challenges of knights,\\
\vin And all these triumphs for thy beauty's sake:\\
When thou hast told these honours done to thee,\\*
Then tell, O tell, how thou didst murder me.
\end{verse}

\subsection{}

\blfootnote{Samuel Butler (1835 -- 1902), \cite{odq}.}The public... takes in its milk on the principle that it is cheaper to do this than to keep a cow. So it is, but the milk is more likely to be watered.

\section{}

\subsection{}

\blfootnote{`Porphria's Lover', Robert Browning (1828 -- 1889), \cite{norton}. Porphyria is a blood disorder commonly thought to induce madness, although this complication is actually very rare, even when one takes into account the rarity of the disease itself.}\settowidth{\versewidth}{    Which done, she rose, and from her form}
\begin{verse}[\versewidth]
The rain set early in tonight;\\*
\vin The sullen wind was soon awake;\\
It tore the elm-tops down for spite,\\
\vin And did its worst to vex the lake:\\
\vin I listened with heart fit to break.\\
When glided in \textit{Porphyria}; straight\\
\vin She shut the cold out \& the storm,\\
And kneeled and made the cheerless grate\\
\vin Blaze up, \& all the cottage warm;\\
\vin Which done, she rose, and from her form\\
Withdrew the dripping cloak \& shawl,\\
\vin And laid her soiled gloves by, untied\\
Her hat \& let the damp hair fall,\\
\vin And, last, she sat down by my side\\
\vin And called me. When no voice replied,\\
She put my arm about her waist,\\
\vin And made her smooth white shoulder bare,\\
And all her yellow hair displaced,\\
\vin And, stooping, made my cheek lie there,\\
\vin And spread, o'er all, her yellow hair,\\
Murmuring how she loved me -- she\\
\vin Too weak, for all her heart's endeavour,\\
To set its struggling passion free\\
\vin From pride, \& vainer ties dissever,\\
\vin And give herself to me for ever.\\
But passion sometimes would prevail,\\
\vin Nor could tonight's gay feast restrain\\
A sudden thought of one so pale\\
\vin For love of her, and all in vain:\\
\vin So, she was come through wind \& rain.\\
Be sure I looked up at her eyes\\
\vin Happy \& proud; at last I knew\\
\textit{Porphyria} worshipped me; surprise\\
\vin Made my heart swell, and still it grew\\
\vin While I debated what to do.\\
That moment she was mine, mine, fair,\\
\vin Perfectly pure \& good: I found\\
A thing to do, and all her hair\\
\vin In one long yellow string I wound\\
\vin Three times her little throat around,\\
And strangled her. No pain felt she;\\
\vin I am quite sure she felt no pain.\\
As a shut bud that holds a bee,\\
\vin I warily oped her lids: again\\
\vin Laughed the blue eyes without a stain.\\
And I untightened next the tress\\
\vin About her neck; her cheek once more\\
Blushed bright beneath my burning kiss:\\
\vin I propped her head up as before,\\
\vin Only, this time my shoulder bore\\
Her head, which droops upon it still:\\
\vin The smiling rosy little head,\\
So glad it has its utmost will,\\
\vin That all it scorned at once is fled,\\
\vin And I, its love, am gained instead!\\
\textit{Porphyria}'s love: she guessed not how\\
\vin Her darling one wish would be heard.\\
And thus we sit together now,\\
\vin And all night long we have not stirred,\\*
\vin And yet God has not said a word!
\end{verse}

\subsection{}

\blfootnote{`Rose Aylmer', Walter Landor (1775 -- 1864), \cite{norton}. Landor was inspired to write these verses by The Hon Rose Aylmer, daughter of the 4th Baron Aylmer; she is an obscure character, who appears to have died in 1800.}\settowidth{\versewidth}{Ah what avails the sceptred race,}
\begin{verse}[\versewidth]
Ah what avails the sceptred race,\\*
\vin Ah what the form divine!\\
What every virtue, every grace!\\
\vin \textit{Rose Aylmer}, all were thine.\\
\textit{Rose Aylmer}, whom these wakeful eyes\\
\vin May weep, but never see,\\
A night of memories \& of sighs\\*
\vin I consecrate to thee.
\end{verse}

\subsection{}

\blfootnote{The Rev Robert Burton (1577 -- 1640), \cite{odq}.}All poets are mad.

\section{}

\subsection{}

\blfootnote{Anonymous, \cite{newlove}.}\settowidth{\versewidth}{Her breasts, that welled so plump high,}
\begin{verse}[\versewidth]
She lay all naked in her bed,\\*
\vin And I myself lay by;\\
No veil but curtains about her spread,\\
\vin No covering but I:\\
Her head upon her shoulders seeks\\
\vin To hang in careless wise,\\
And full of blushes was her cheeks,\\*
\vin And of wishes were her eyes.\\!

Her blood still fresh into her face,\\*
\vin As on a message came,\\
To say that in another place\\
\vin It meant another game;\\
Her cherry lip moist, plump \& fair,\\
\vin Millions of kisses crown,\\
Which ripe \& uncropped dangle there,\\*
\vin And weigh the branches down.\\!

Her breasts, that welled so plump \& high,\\*
\vin Bred pleasant pain in me;\\
For all the world I do defy\\
\vin The like felicity;\\
Her thighs \& belly, soft \& fair,\\
\vin To me were only shown:\\
To have seen such meat, and not to have eat,\\*
\vin Would have angered any stone.\\!

Her knees lay upward gently bent,\\*
\vin And all lay hollow under,\\
As if on easy terms, they meant\\
\vin To fall unforced asunder;\\
Just so the cyprian queen did lie,\\
\vin Expecting in her bower;\\
When too long stay had kept the boy\\*
\vin Beyond his promised hour.\\!

Dull clown, quoth she, why dost delay\\*
\vin Such proffered bliss to take?\\
Canst thou find out no other way\\
\vin Similitudes to make?\\
Mad with delight I thundering\\
\vin Throw my arms about her,\\
But -- pox upon't -- 'twas but a dream;\\*
\vin And so I lay without her.
\end{verse}

\subsection{}

\blfootnote{Miss Christina Rossetti (1830 -- 1894), \cite{norton}.}\settowidth{\versewidth}{    You tell me of our future that you planned:}
\begin{verse}[\versewidth]
Remember me when I am gone away,\\*
\vin Gone far away into the silent land;\\
\vin When you can no more hold me by the hand,\\
Nor I \sfrac{$1$}{$2$} turn to go yet turning stay.\\
Remember me when no more day by day\\
\vin You tell me of our future that you planned:\\
\vin Only remember me; you understand\\
It will be late to counsel then or pray.\\
Yet if you should forget me for a while\\
\vin And afterwards remember, do not grieve:\\
\vin For if the darkness \& corruption leave\\
\vin \vin A vestige of the thoughts that once I had,\\
Better by far you should forget, and smile\\*
\vin \vin Than that you should remember \& be sad.
\end{verse}

\subsection{}

\blfootnote{The Rev Robert Burton (1577 -- 1640), \cite{odq}.}Naught so sweet as melancholy.

\section{}

\subsection{}

\blfootnote{Prof Laurence Binyon (1869 -- 1943), \cite{oxfordlarkin}. Prof Larkin calls these four verses \refpoem{The Burning of the Leaves}, although in other books they are the first of five parts of a longer poem of the same name.}\settowidth{\versewidth}{        On stubborn stalks that crackle as they resist.}
\begin{verse}[\versewidth]
Now is the time for the burning of the leaves.\\*
\vin They go to the fire; the nostril pricks with smoke\\
\vin \vin Wandering slowly into a weeping mist.\\
Brittle \& blotched, ragged \& rotten sheaves.\\
\vin A flame seizes the smouldering ruin and bites\\*
\vin \vin On stubborn stalks that crackle as they resist.\\!

The last hollyhock's fallen tower is dust;\\*
\vin All the spices of june are a bitter reek,\\
\vin \vin All the extravagant riches spent \& mean.\\
All burns. The reddest rose is a ghost;\\
\vin Sparks whirl up, to expire in the mist: the wild\\*
\vin \vin Fingers of fire are making corruption clean.\\!

Now is the time for stripping the spirit bare,\\*
\vin Time for the burning of days ended \& done,\\
\vin \vin Idle solace of things that have gone before:\\
Rootless hope \& fruitless desire are there;\\
\vin Let them go to the fire, with never a look behind.\\*
\vin \vin The world that was ours is a world that is ours no more.\\!

They will come again, the leaf \& the flower, to arise\\*
\vin From squalor of rottenness into the old splendour,\\
\vin \vin And magical scents to a wondering memory bring;\\
The same glory, to shine upon different eyes.\\
\vin Earth cares for her own ruins, naught for ours.\\*
\vin \vin Nothing is certain, only the certain spring.
\end{verse}

\subsection{}

\blfootnote{Miss Christina Rossetti (1830 -- 1894), \cite{norton}.}\settowidth{\versewidth}{I shall not hear the nightingale}
\begin{verse}[\versewidth]
When I am dead, my dearest,\\*
\vin Sing no sad songs for me;\\
Plant thou no roses at my head,\\
\vin Nor shady cypress tree:\\
Be the green grass above me\\
\vin With showers \& dewdrops wet;\\
And if thou wilt, remember,\\*
\vin And if thou wilt, forget.\\!

I shall not see the shadows;\\*
\vin I shall not feel the rain;\\
I shall not hear the nightingale\\
\vin Sing on, as if in pain:\\
And dreaming through the twilight\\
\vin That doth not rise nor set,\\
Haply I may remember,\\*
\vin And haply may forget.
\end{verse}

\subsection{}

\blfootnote{The Rev Robert Burton (1577 -- 1640), \cite{odq}.}One was never married, and that's his hell; another is, and that's his...

\section{}

\subsection{}

\blfootnote{$\mathbb{R}$ `A Leave-Taking', Algernon Swinburne (1837 -- 1909), \cite{pbev}.}\settowidth{\versewidth}{    Flowers without scent, and fruits that would not grow,}
\begin{verse}[\versewidth]
Let us go hence, my songs; she will not hear.\\*
Let us go hence together without fear;\\
\vin Keep silence now, for singing-time is over,\\
And over all old things \& all things dear.\\
\vin She loves not you nor me as all we love her.\\
Yea, though we sang as angels in her ear,\\*
\vin \vin She would not hear.\\!

Let us rise up and part; she will not know.\\*
Let us go seaward as the great winds go,\\
\vin Full of blown sand \& foam; what help is here?\\
There is no help, for all these things are so,\\
\vin And all the world is bitter as a tear.\\
And how these things are, though ye strove to show,\\*
\vin \vin She would not know.\\!

Let us go home \& hence; she will not weep.\\*
We gave love many dreams \& days to keep,\\
\vin Flowers without scent, and fruits that would not grow,\\
Saying, If thou wilt, thrust in thy sickle and reap.\\
\vin All is reaped now; no grass is left to mow;\\
And we that sowed, though all we fell on sleep,\\*
\vin \vin She would not weep.\\!

Let us go hence and rest; she will not love.\\*
She shall not hear us if we sing hereof,\\
\vin Nor see love's ways, how sore they are \& steep.\\
Come hence, let be, lie still; it is enough.\\
\vin Love is a barren sea, bitter \& deep;\\
And though she saw all heaven in flower above,\\*
\vin \vin She would not love.\\!

Let us give up, go down; she will not care.\\*
Though all the stars made gold of all the air,\\
\vin And the sea moving saw before it move\\
One moon-flower making all the foam-flowers fair;\\
\vin Though all those waves went over us, and drove\\
Deep down the stifling lips \& drowning hair,\\*
\vin \vin She would not care.\\!

Let us go hence, go hence; she will not see.\\*
Sing all once more together; surely she,\\
\vin She too, remembering days \& words that were,\\
Will turn a little toward us, sighing; but we,\\
\vin We are hence, we are gone, as though we had not been there.\\
Nay, and though all men seeing had pity on me,\\*
\vin \vin She would not see.
\end{verse}

\subsection{}

\blfootnote{`Lucifer in Starlight', George Meredith (1828 -- 1909), \cite{norton}.}\settowidth{\versewidth}{        Around the ancient track marched, rank on rank,}
\begin{verse}[\versewidth]
On a starred night Prince \textit{Lucifer} uprose.\\*
\vin Tired of his dark dominion swung the fiend\\
\vin Above the rolling ball in cloud part screened,\\
Where sinners hugged their spectre of repose.\\
Poor prey to his hot fit of pride were those.\\
\vin And now upon his western wing he leaned;\\
\vin Now his huge bulk o'er Afric's sands careened;\\
Now the black planet shadowed arctic snows.\\
Soaring through wider zones that pricked his scars\\
\vin With memory of the old revolt from Awe,\\
He reached a middle height, and at the stars,\\
\vin \vin Which are the brain of heaven, he looked, and sank.\\
\vin \vin Around the ancient track marched, rank on rank,\\*
\vin The army of unalterable law.
\end{verse}

\subsection{}

\blfootnote{The Rev Robert Burton (1577 -- 1640), \cite{odq}.}The pen is worse than the sword.

\section{}

\subsection{}

\blfootnote{$\mathbb{R}$ `Auguries of Innocence', William Blake (1757 -- 1827), \cite{blakea}.}\settowidth{\versewidth}{Hold infinity in the palm of your hand}
\begin{verse}[\versewidth]
To see a world in a grain of sand\\*
And a heaven in a wild flower,\\
Hold infinity in the palm of your hand\\
And eternity in an hour:\\
A robin redbreast in a cage\\
Puts all heaven in a rage.\\
A dove-house filled with doves \& pigeons\\
Shudders hell through all its regions.\\
A dog starved at his master's gate\\
Predicts the ruin of the state.\\
A horse misused upon the road\\
Calls to heaven for human blood.\\
A skylark wounded in the wing,\\
A cherubim does cease to sing.\\
The game cock clipped \& armed for fight\\
Does the rising sun affright.\\
Every wolf's \& lion's howl\\
Raises from hell a human soul.\\
The wild deer, wandering here \& there\\
Keeps the human soul from care.\\
The lamb misused breeds public strife\\
And yet forgives the butcher's knife.\\
The bat that flits at close of eve\\
Has left the brain that won't believe.\\
The owl that calls upon the night\\
Speaks the unbeliever's fright.\\
He who shall hurt the little wren\\
Shall never be beloved by men.\\
He who the ox to wrath has moved\\
Shall never be by woman loved.\\
The wanton boy that kills the fly\\
Shall feel the spider's enmity.\\
He who torments the chafer's sprite\\
Weaves a bower in endless night.\\
The beggar's dog \& widow's cat,\\
Feed them \& thou wilt grow fat.\\
The gnat that sings his summer's song\\
Poison gets from slander's tongue.\\
The poison of the snake \& newt\\
Is the sweat of envy's foot.\\
The poison of the honey bee\\
Is the artist's jealousy.\\
The prince's robes \& beggar's rags\\
Are toadstools on the miser's bags.\\
A truth that's told with bad intent\\
Beats all the lies you can invent.\\
The soldier armed with sword \& gun\\
Palsied strikes the summer's sun.\\
The poor man's farthing is worth more\\
Than all the gold on Afric's shore.\\
One mite wrung from the lab'rer's hands\\
Shall buy \& sell the miser's lands,\\
Or if protected from on high\\
Does that whole nation sell \& buy.\\
He who mocks the infant's faith\\
Shall be mocked in age \& death.\\
He who shall teach the child to doubt\\
The rotting grave shall ne'er get out.\\
He who respects the infant's faith\\
Triumphs over hell \& death.\\
The child's toys \& the old man's reasons\\
Are the fruits of the two seasons.\\
The questioner who sits so sly\\
Shall never know how to reply.\\
He who replies to words of doubt\\
Doth put the light of knowledge out.\\
The strongest poison ever known\\
Came from caesar's laurel crown.\\
Nought can deform the human race\\
Like to the armour's iron brace.\\
When gold \& gems adorn the plough\\
To peaceful arts shall envy bow.\\
A riddle or the cricket's cry\\
Is to doubt a fit reply.\\
The emmet's inch \& eagle's mile\\
Make lame philosophy to smile.\\
He who doubts from what he sees\\
Will ne'er believe do what you please.\\
God appears \& God is light\\
To those poor souls who dwell in night,\\
But does a human form display\\*
To those who dwell in realms of day.
\end{verse}

\subsection{}

\blfootnote{`Farewell', John Clare (1793 -- 1864), \cite{norton}.}\settowidth{\versewidth}{In the prison yard nothing builds, blackbirds or thrushes.}
\begin{verse}[\versewidth]
Farewell to the bushy clump close to the river\\*
And the flags where the butter-bump hides in for ever;\\
Farewell to the weedy nook, hemmed in by waters;\\
Farewell to the miller's brook \& his three bonny daughters;\\
Farewell to them all while in prison I lie --\\*
In the prison a thrall sees nought but the sky.\\!

Shut out are the green fields and birds in the bushes;\\*
In the prison yard nothing builds, blackbirds or thrushes.\\
Farewell to the old mill \& dash of the waters,\\*
To the miller \&, dearer still, to his three bonny daughters.\\!

In the nook, the large burdock grows near the green willow;\\*
In the flood, round the moorcock dashes under the billow;\\
To the old mill farewell, to the lock, pens, \& waters,\\*
To the miller himsel', \& his three bonny daughters.
\end{verse}

\subsection{}

\blfootnote{The Rev Robert Burton (1577 -- 1640), \cite{odq}.}All places are distant from heaven alike.

\section{}

\subsection{}

\blfootnote{$\mathbb{R}$ William Shakespeare (1564 -- 1616), \cite{obev}. These lines are spoken by Jaques in \refbook{As You Like It}, II.7.}\settowidth{\versewidth}{Even in the cannon's mouth. And then the justice,}
\begin{verse}[\versewidth]
\textcolor{white}{Wherein we play in.} All the world's a stage,\\*
And all the men \& women merely players;\\
They have their exits \& their entrances,\\
And one man in his time plays many parts,\\
His acts being seven ages. At first, the infant,\\
Mewling \& puking in the nurse's arms.\\
Then the whining schoolboy, with his satchel\\
And shining morning face, creeping like snail\\
Unwillingly to school. And then the lover,\\
Sighing like furnace, with a woeful ballad\\
Made to his mistress' eyebrow. Then a soldier,\\
Full of strange oaths and bearded like the pard,\\
Jealous in honor, sudden \& quick in quarrel,\\
Seeking the bubble reputation\\
Even in the cannon's mouth. And then the justice,\\
In fair round belly with good capon lined,\\
With eyes severe \& beard of formal cut,\\
Full of wise saws \& modern instances;\\
And so he plays his part. The sixth age shifts\\
Into the lean \& slippered pantaloon,\\
With spectacles on nose \& pouch on side;\\
His youthful hose, well saved, a world too wide\\
For his shrunk shank, and his big manly voice,\\
Turning again toward childish treble, pipes\\
And whistles in his sound. Last scene of all,\\
That ends this strange eventful history,\\
Is second childishness \& mere oblivion,\\*
Sans teeth, sans eyes, sans taste, sans everything.
\end{verse}

\subsection{}

\blfootnote{`Gypsies', John Clare (1793 -- 1864), \cite{norton}.}\settowidth{\versewidth}{Then feels the heat too strong and goes aloof;}
\begin{verse}[\versewidth]
The snow falls deep; the forest lies alone:\\*
The boy goes hasty for his load of brakes,\\
Then thinks upon the fire and hurries back;\\
The gypsy knocks his hands and tucks them up,\\
And seeks his squalid camp, \sfrac{$1$}{$2$} hid in snow,\\
Beneath the oak, which breaks away the wind,\\
And bushes close, with snow like hovel warm:\\
There stinking mutton roasts upon the coals,\\
And the \sfrac{$1$}{$2$}-roasted dog squats close and rubs,\\
Then feels the heat too strong and goes aloof;\\
He watches well, but none a bit can spare,\\
And vainly waits the morsel thrown away:\\
'Tis thus they live -- a picture to the place;\\*
A quiet, pilfering, unprotected race.
\end{verse}

\subsection{}

\blfootnote{The Rev Robert Burton (1577 -- 1640), \cite{odq}.}Were it not that they are loath to lay out money for a rope, they would be hanged forthwith.

\section{}

\subsection{}

\blfootnote{$\mathbb{R}$ William Shakespeare (1564 -- 1616), \cite{obev}. These lines are spoken by Othello over the sleeping Desdemona in \refbook{Othello} V.2.}\settowidth{\versewidth}{And love thee after. One more, and that's the last.}
\begin{verse}[\versewidth]
It is the cause, it is the cause, my soul.\\*
Let me not name it to you, you chaste stars;\\
It is the cause. Yet I'll not shed her blood,\\
Nor scar that whiter skin of hers than snow\\
And smooth as monumental alabaster.\\
Yet she must die, else she'll betray more men.\\
Put out the light, and then put out the light.\\
If I quench thee, thou flaming minister,\\
I can again thy former light restore\\
Should I repent me. But once put out thy light,\\
Thou cunning'st pattern of excelling nature,\\
I know not where is that promethean heat\\
That can thy light relume. When I have plucked thy rose\\
I cannot give it vital growth again,\\
It must needs wither. I'll smell thee on the tree.\\
O balmy breath, that dost almost persuade\\
Justice to break her sword! One more, one more.\\
Be thus when thou art dead and I will kill thee\\
And love thee after. One more, and that's the last.\\
So sweet was ne'er so fatal. I must weep,\\
But they are cruel tears. This sorrow's heavenly,\\*
It strikes where it doth love. She wakes.
\end{verse}

\subsection{}

\blfootnote{William Shakespeare (1564 -- 1616), \cite{treasury}. This song is sung by Ariel in \refbook{The Tempest} I.2.}\settowidth{\versewidth}{Full fathom five thy father lies:}
\begin{verse}[\versewidth]
Full fathom five thy father lies:\\*
\vin Of his bones are coral made;\\
Those are pearls that were his eyes:\\
\vin Nothing of him that doth fade\\
But doth suffer a sea-change\\
Into something rich \& strange.\\
Sea-nymphs hourly ring his knell:\\
Hark! now I hear them --\\*
\vin \vin Ding-dong, bell.
\end{verse}

\subsection{}

\blfootnote{William Shakespeare (1564 -- 1616), \cite{shakespeare}. This is uttered by Miranda in \refbook{The Tempest} I.2.}Good wombs have borne bad sons.

\section{}

\subsection{}

\blfootnote{George Meredith (1828 -- 1909), \cite{newlove}. This is taken from \refbook{Modern Love}, Meredith's sequence of poems describing the breakdown of his first marriage.}\settowidth{\versewidth}{    Her cheek was salt against my kiss, and swift}
\begin{verse}[\versewidth]
In our old shipwrecked days there was an hour,\\*
\vin When, in the firelight steadily aglow,\\
\vin Joined slackly, we beheld the red chasm grow\\
Among the clicking coals. Our library-bower\\
That eve was left to us: and hushed we sat\\
\vin As lovers to whom time is whispering.\\
\vin From sudden-opened doors we heard them sing:\\
The nodding elders mixed good wine with chat.\\
Well knew we that life's greatest treasure lay\\
\vin With us, and of it was our talk. `Ah, yes!\\
\vin Love dies!' I said: I never thought it less.\\
She yearned to me that sentence to unsay.\\
Then when the fire domed blackening, I found\\
\vin Her cheek was salt against my kiss, and swift\\
\vin Up the sharp scale of sobs her breast did lift:\\*
Now am I haunted by that taste, that sound.
\end{verse}

\subsection{}

\blfootnote{William Shakespeare (1564 -- 1616), \cite{treasury}.}\settowidth{\versewidth}{So shalt thou feed on death, that feeds on men;}
\begin{verse}[\versewidth]
Poor soul, the centre of my sinful earth,\\*
Foiled by those rebel powers that thee array,\\
Why dost thou pine within, and suffer dearth,\\*
Painting thy outward walls so costly gay?\\!

Why so large cost, having so short a lease,\\*
Dost thou upon thy fading mansion spend?\\
Shall worms, inheritors of this excess,\\*
Eat up thy charge? is this thy body's end?\\!

Then, soul, live thou upon thy servant's loss,\\*
And let that pine to aggravate thy store;\\
Buy terms divine in selling hours of dross;\\*
Within be fed, without be rich no more:--\\!

So shalt thou feed on death, that feeds on men;\\*
And death once dead, there's no more dying then.
\end{verse}

\subsection{}

\blfootnote{William Shakespeare (1564 -- 1616), \cite{shakespeare}. This is uttered by Ariel in \refbook{The Tempest} I.2.}Hell is empty, and all the devils are here.

\section{}

\subsection{}

\blfootnote{John Dryden, Poet Laureate (1631 -- 1700), \cite{newlove}.}\settowidth{\versewidth}{Till we have lost our treasure,}
\begin{verse}[\versewidth]
Farewell, ungrateful traitor;\\*
\vin Farewell, my perjured swain.\\
Let never injured creature\\
\vin Believe a man again.\\
The pleasure of possessing\\
Surpasses all expressing,\\
But 'tis too short a blessing,\\*
\vin And love too long a pain.\\!

'Tis easy to deceive us\\*
\vin In pity of your pain,\\
But when we love you leave us\\
\vin To rail at you in vain.\\
Before we have descried it,\\
There is no bliss beside it,\\
But she that once has tried it\\*
\vin Will never love again.\\!

The passion you pretended\\*
\vin Was only to obtain,\\
But when the charm is ended\\
\vin The charmer you disdain.\\
Your love by ours we measure\\
Till we have lost our treasure,\\
But dying is a pleasure,\\*
\vin When living is a pain.
\end{verse}

\subsection{}

\blfootnote{William Shakespeare (1564 -- 1616), \cite{norton}.}\settowidth{\versewidth}{    And sable curls all silvered o'er with white;}
\begin{verse}[\versewidth]
When I do count the clock that tells the time,\\*
\vin And see the brave day sunk in hideous night;\\
When I behold the violet past prime,\\
\vin And sable curls all silvered o'er with white;\\
When lofty trees I see barren of leaves\\
\vin Which erst from heat did canopy the herd,\\
And summer's green all girded up in sheaves\\
\vin Borne on the bier with white \& bristly beard,\\
Then of thy beauty do I question make,\\
\vin That thou among the wastes of time must go,\\
Since sweets \& beauties do themselves forsake\\
\vin And die as fast as they see others grow;\\
And nothing 'gainst time's scythe can make defence\\*
Save breed, to brave him when he takes thee hence.
\end{verse}

\subsection{}

\blfootnote{William Shakespeare (1564 -- 1616), \cite{shakespeare}. This is uttered by Trinculo in \refbook{The Tempest} II.2.}Misery acquaints a man with strange bedfellows.

\section{}

\subsection{}

\blfootnote{`Tichborne's Lament', Chidiock Tichborne (1562 -- 1586), \cite{norton}. The ultimate written source for this poem is a letter which Tichborne wrote to his wife on the night before he was hanged, drawn and quartered for his part in a conspiracy against Elizabeth I. Tichborne was part of the same family which provided the fourteen Tichborne baronets (of Tichborne in the County of Hampshire) who held the title from its creation in 1621 until its extinction in 1968. He was also a distant cousin of Henry Tichborne, 1st Baron Ferrard and 1st Baronet (of Beaulieu in the County of Meath), who sadly left no heirs to his titles. \P 17. The word `glass' refers here to an hourglass, rather than a drinking vessel.}\settowidth{\versewidth}{    And all my good is but vain hope of gain;}
\begin{verse}[\versewidth]
My prime of youth is but a frost of cares;\\*
\vin My feast of joy is but a dish of pain;\\
My crop of corn is but a field of tares;\\
\vin And all my good is but vain hope of gain;\\
The day is past, and yet I saw no sun,\\*
And now I live, and now my life is done.\\!

My tale was heard and yet it was not told;\\*
\vin My fruit is fallen, and yet my leaves are green;\\
My youth is spent and yet I am not old;\\
\vin I saw the world and yet I was not seen;\\
My thread is cut and yet it is not spun,\\*
And now I live, and now my life is done.\\!

I sought my death and found it in my womb;\\*
\vin I looked for life and saw it was a shade;\\
I trod the earth and knew it was my tomb,\\
\vin And now I die, and now I was but made;\\
My glass is full, and now my glass is run,\\*
And now I live, and now my life is done.
\end{verse}

\subsection{}

\blfootnote{`To the Moon', Percy Shelley (1792 -- 1822), \cite{treasury}.}\settowidth{\versewidth}{Of climbing heaven, and gazing on the earth,}
\begin{verse}[\versewidth]
\vin Art thou pale for weariness\\*
Of climbing heaven, and gazing on the earth,\\
\vin Wandering companionless\\
Among the stars that have a different birth --\\
And ever-changing, like a joyless eye\\*
That finds no object worth its constancy?
\end{verse}

\subsection{}

\blfootnote{`This is a line from Sonnet 35.', William Shakespeare (1564 -- 1616), \cite{norton}.}Roses have thorns, and silver fountains mud.

\section{}

\subsection{}

\blfootnote{`The Listeners', Walter de la Mare (1873 -- 1956), \cite{norton}. This poem was said to be a favourite of St Teresa of Calcutta.}\settowidth{\versewidth}{Stood thronging the faint moonbeams on the dark stair,}
\begin{verse}[\versewidth]
Is there anybody there? said the traveller,\\*
\vin Knocking on the moonlit door;\\
And his horse in the silence champed the grasses\\
\vin Of the forest's ferny floor:\\
And a bird flew up out of the turret,\\
\vin Above the traveller's head:\\
And he smote upon the door again a second time;\\
\vin Is there anybody there? he said.\\
But no one descended to the traveller;\\
\vin No head from the leaf-fringed sill\\
Leaned over \& looked into his grey eyes,\\
\vin Where he stood perplexed \& still.\\
But only a host of phantom listeners\\
\vin That dwelt in the lone house then\\
Stood listening in the quiet of the moonlight\\
\vin To that voice from the world of men:\\
Stood thronging the faint moonbeams on the dark stair,\\
\vin That goes down to the empty hall,\\
Hearkening in an air stirred \& shaken\\
\vin By the lonely traveller's call.\\
And he felt in his heart their strangeness,\\
\vin Their stillness answering his cry,\\
While his horse moved, cropping the dark turf,\\
\vin 'Neath the starred \& leafy sky;\\
For he suddenly smote on the door, even\\
\vin Louder, and lifted his head:\\
Tell them I came, and no one answered,\\
\vin That I kept my word, he said.\\
Never the least stir made the listeners,\\
\vin Though every word he spake\\
Fell echoing through the shadowiness of the still house\\
\vin From the one man left awake:\\
Ay, they heard his foot upon the stirrup,\\
\vin And the sound of iron on stone,\\
And how the silence surged softly backward,\\*
\vin When the plunging hoofs were gone.
\end{verse}

\subsection{}

\blfootnote{`Dying Speech of an Old Philosopher', Walter Landor (1775 -- 1864), \cite{norton}.}\settowidth{\versewidth}{I strove with none, for none was worth my strife:}
\begin{verse}[\versewidth]
I strove with none, for none was worth my strife:\\*
\vin Nature I loved, and, next to nature, art:\\
I warmed both hands before the fire of life;\\*
\vin It sinks; and I am ready to depart.
\end{verse}

\subsection{}

\blfootnote{The Rt Hon Joseph Addison (1672 -- 1719), \cite{odq}.}There is no living with thee, nor without thee.

\section{}

\subsection{}

\blfootnote{`The Rime of the Ancient Mariner', Samuel Coleridge (1772 -- 1834), \cite{norton}. Coleridge kicks off his \refpoem{Rime} with a lengthy quotation from a seventeenth century theologian, Thomas Burnet, and the original text is peppered with margin notes throughout.}\settowidth{\versewidth}{`Down dropped the breeze, the sails dropt down;}
\begin{verse}[\versewidth]
\flagverse{\footnotesize Part I} It is an ancient mariner\\*
\vin And he stoppeth one of three.\\
`By thy long grey beard \& glittering eye,\\*
\vin Now wherefore stopp'st thou me?\\!

`The bridegroom's doors are opened wide,\\*
\vin And I am next of kin;\\
The guests are met; the feast is set:\\*
\vin May'st hear the merry din.'\\!

He holds him with his skinny hand,\\*
\vin 'There was a ship,' quoth he.\\
'Hold off! Unhand me, grey-beard loon!'\\*
\vin Eftsoons his hand dropped he.\\!

He holds him with his glittering eye --\\*
\vin The wedding-guest stood still,\\
And listens like a three years' child:\\*
\vin The mariner hath his will.\\!

The wedding-guest sat on a stone:\\*
\vin He cannot choose but hear;\\
And thus spake on that ancient man,\\*
\vin The bright-eyed mariner.\\!

`The ship was cheered, the harbour cleared,\\*
\vin Merrily did we drop\\
Below the kirk, below the hill,\\*
\vin Below the lighthouse-top.\\!

`The sun came up upon the left,\\*
\vin Out of the sea came he.\\
And he shone bright, and on the right\\*
\vin Went down into the sea.\\!

`Higher \& higher every day,\\*
\vin Till over the mast at noon --'\\
The wedding-guest here beat his breast,\\*
\vin For he heard the loud bassoon.\\!

`The bride hath paced into the hall,\\*
\vin Red as a rose is she;\\
Nodding their heads before her goes\\*
\vin The merry minstrelsy.\\!

`The wedding-guest he beat his breast,\\*
\vin Yet he cannot choose but hear;\\
And thus spake on that ancient man,\\*
\vin The bright-eyed mariner.\\!

`And now the storm-blast came, and he\\*
\vin Was tyrannous \& strong:\\
He struck with his o'ertaking wings,\\*
\vin And chased us south along.\\!

`With sloping masts \& dipping prow,\\*
As who pursued with yell \& blow\\
Still treads the shadow of his foe,\\
\vin And forward bends his head,\\
The ship drove fast, loud roared the blast,\\*
\vin And southward aye we fled.\\!

`And now there came both mist \& snow,\\*
\vin And it grew wondrous cold:\\
And ice, mast-high, came floating by,\\*
\vin As green as emerald.\\!

`And through the drifts the snowy clifts\\*
\vin Did send a dismal sheen:\\
Nor shapes of men nor beasts we ken --\\*
\vin The ice was all between.\\!

`The ice was here; the ice was there;\\*
\vin The ice was all around:\\
It cracked \& growled, and roared \& howled,\\*
\vin Like noises in a swound!\\!

`At length did cross an albatross,\\*
\vin Thorough the fog it came;\\
As if it had been a christian soul,\\*
\vin We hailed it in God's name.\\!

`It ate the food it ne'er had eat,\\*
\vin And round \& round it flew.\\
The ice did split with a thunder-fit;\\*
\vin The helmsman steered us through.\\!

`And a good south wind sprung up behind;\\*
\vin The albatross did follow,\\
And every day, for food or play,\\*
\vin Came to the mariner's hollo!\\!

`In mist or cloud, on mast or shroud,\\*
\vin It perched for vespers nine;\\
Whiles all the night, through fog-smoke white,\\*
\vin Glimmered the white Moon-shine.'\\!

'God save thee, ancient Mariner!\\*
\vin From the fiends, that plague thee thus!\\
Why look'st thou so?' `With my cross-bow\\*
\vin I shot the albatross.\\!

\flagverse{\footnotesize Part II} `The sun now rose upon the right:\\*
\vin Out of the sea came he,\\
Still hid in mist, and on the left\\*
\vin Went down into the sea.\\!

`And the good south wind still blew behind,\\*
\vin But no sweet bird did follow,\\
Nor any day for food or play\\*
\vin Came to the mariner's hollo.\\!

`And I had done a hellish thing,\\*
\vin And it would work 'em woe:\\
For all averred, I had killed the bird\\
\vin hat made the breeze to blow.\\
``Ah wretch!'' said they, ``The bird to slay,\\*
\vin That made the breeze to blow!''\\!

`Nor dim nor red, like God's own head,\\*
\vin The glorious sun uprist:\\
Then all averred, I had killed the bird\\
\vin That brought the fog and mist.\\
``'Twas right,'' said they, ``Such birds to slay,\\*
\vin That bring the fog \& mist.''\\!

`The fair breeze blew; the white foam flew;\\*
\vin The furrow followed free;\\
We were the first that ever burst\\*
\vin Into that silent sea.\\!

`Down dropped the breeze, the sails dropt down;\\*
\vin 'Twas sad as sad could be;\\
And we did speak only to break\\*
\vin The silence of the sea!\\!

`All in a hot \& copper sky,\\*
\vin The bloody sun, at noon,\\
Right up above the mast did stand,\\*
\vin No bigger than the moon.\\!

`Day after day, day after day,\\*
\vin We stuck, nor breath nor motion;\\
As idle as a painted ship\\*
\vin Upon a painted ocean.\\!

`Water, water, every where,\\*
\vin And all the boards did shrink;\\
Water, water, every where,\\*
\vin Nor any drop to drink.\\!

`The very deep did rot: O \textit{Christ}!\\*
\vin That ever this should be!\\
Yea, slimy things did crawl with legs\\*
\vin Upon the slimy sea.\\!

`About, about, in reel \& rout\\*
\vin The death-fires danced at night;\\
The water, like a witch's oils,\\*
\vin Burnt green, and blue \& white.\\!

`And some in dreams assur\`{e}d were\\*
\vin Of the spirit that plagued us so;\\
Nine fathom deep he had followed us\\*
\vin From the land of mist \& snow.\\!

`And every tongue, through utter drought,\\*
\vin Was withered at the root;\\
We could not speak, no more than if\\*
\vin We had been choked with soot.\\!

`Ah well a-day what evil looks\\*
\vin Had I from old \& young!\\
Instead of the cross, the albatross\\*
\vin About my neck was hung.\\!

\flagverse{\footnotesize Part III} `There passed a weary time. Each throat\\*
\vin Was parched, and glazed each eye.\\
A weary time! a weary time!\\
\vin How glazed each weary eye,\\
When looking westward, I beheld\\*
\vin A something in the sky.\\!

`At first it seemed a little speck,\\*
\vin And then it seemed a mist;\\
It moved \& moved, and took at last\\*
\vin A certain shape, I wist.\\!

`A speck, a mist, a shape, I wist!\\*
\vin And still it neared \& neared:\\
As if it dodged a water-sprite,\\*
\vin It plunged \& tacked \& veered.\\!

`With throats unslaked, with black lips baked,\\*
\vin We could nor laugh nor wail;\\
Through utter drought all dumb we stood.\\
I bit my arm; I sucked the blood,\\*
\vin And cried, ``A sail! A sail!''\\!

`With throats unslaked, with black lips baked,\\*
\vin Agape they heard me call:\\
``Gramercy!'' they for joy did grin,\\
And all at once their breath drew in.\\*
\vin As they were drinking all.\\!

`{``}See! See!'' I cried. ``She tacks no more\\*
\vin Hither to work us weal;\\
Without a breeze, without a tide,\\*
\vin She steadies with upright keel!''\\!

`The western wave was all aflame.\\*
\vin The day was well nigh done.\\
Almost upon the western wave\\
\vin Rested the broad bright sun;\\
When that strange shape drove suddenly\\*
\vin Betwixt us and the sun.\\!

`And straight the sun was flecked with bars,\\*
\vin (Heaven's mother send us grace!)\\
As if through a dungeon-grate he peered\\*
\vin With broad \& burning face.\\!

`{``}Alas!'' thought I, and my heart beat loud,\\*
\vin ``How fast she nears \& nears!\\
Are those her sails that glance in the sun,\\*
\vin Like restless gossameres?\\!

`{``}Are those her ribs through which the sun\\*
\vin Did peer, as through a grate?\\
And is that woman all her crew?\\
Is that a death? and are there two?\\*
\vin Is death that woman's mate?''\\!

`Her lips were red; her looks were free;\\*
\vin Her locks were yellow as gold:\\
Her skin was as white as leprosy;\\
The nightmare life-in-death was she,\\*
\vin Who thicks man's blood with cold.\\!

`The naked hulk alongside came,\\*
\vin And the twain were casting dice;\\
``The game is done! I've won! I've won!''\\*
\vin Quoth she, and whistles thrice.\\!

`The sun's rim dips; the stars rush out;\\*
\vin At one stride comes the dark;\\
With far-heard whisper, o'er the sea,\\*
\vin Off shot the spectre-bark.\\!

`We listened \& looked sideways up!\\*
Fear at my heart, as at a cup,\\
\vin My life-blood seemed to sip!\\
The stars were dim, and thick the night,\\
The steersman's face by his lamp gleamed white;\\
\vin From the sails the dew did drip --\\
Till clomb above the eastern bar\\
The horn\`{e}d moon, with one bright star\\*
\vin Within the nether tip.\\!

`One after one, by the star-dogged moon,\\*
\vin Too quick for groan or sigh,\\
Each turned his face with a ghastly pang,\\*
\vin And cursed me with his eye.\\!

`Four times 50 living men,\\*
\vin (And I heard nor sigh nor groan)\\
With heavy thump, a lifeless lump,\\*
\vin They dropped down one by one.\\!

`The souls did from their bodies fly --\\*
\vin They fled to bliss or woe!\\
And every soul, it passed me by,\\*
\vin Like the whizz of my cross-bow!'\\!

\flagverse{\footnotesize Part IV}`I fear thee, ancient mariner!\\*
\vin I fear thy skinny hand!\\
And thou art long, \& lank, \& brown,\\*
\vin As is the ribbed sea-sand.\\!

`I fear thee and thy glittering eye,\\*
\vin And thy skinny hand, so brown.'\\
`Fear not, fear not, thou wedding-guest!\\*
\vin This body dropped not down.\\!

`Alone, alone, all, all alone,\\*
\vin Alone on a wide wide sea!\\
And never a saint took pity on\\*
\vin My soul in agony.\\!

`The many men, so beautiful!\\*
\vin And they all dead did lie:\\
And a 1000 1000 slimy things\\*
\vin Lived on; and so did I.\\!

`I looked upon the rotting sea,\\*
\vin And drew my eyes away;\\
I looked upon the rotting deck,\\*
\vin And there the dead men lay.\\!

`I looked to heaven, and tried to pray;\\*
\vin But or ever a prayer had gushed,\\
A wicked whisper came, and made\\*
\vin My heart as dry as dust.\\!

`I closed my lids, and kept them close,\\*
\vin And the balls like pulses beat;\\
For the sky \& the sea, and the sea \& the sky\\
Lay dead like a load on my weary eye,\\*
\vin And the dead were at my feet.\\!

`The cold sweat melted from their limbs,\\*
\vin Nor rot nor reek did they:\\
The look with which they looked on me\\*
\vin Had never passed away.\\!

`An orphan's curse would drag to hell\\*
\vin A spirit from on high;\\
But O more horrible than that\\
\vin Is the curse in a dead man's eye!\\
Seven days, seven nights, I saw that curse,\\*
\vin And yet I could not die.\\!

`The moving moon went up the sky,\\*
\vin And no where did abide:\\
Softly she was going up,\\*
\vin And a star or two beside --\\!

`Her beams bemocked the sultry main,\\*
\vin Like april hoar-frost spread;\\
But where the ship's huge shadow lay,\\
The charm\`{e}d water burnt alway\\*
\vin A still \& awful red.\\!

`Beyond the shadow of the ship,\\*
\vin I watched the water-snakes:\\
They moved in tracks of shining white,\\
And when they reared, the elfish light\\*
\vin Fell off in hoary flakes.\\!

`Within the shadow of the ship\\*
\vin I watched their rich attire:\\
Blue, glossy green, and velvet black,\\
They coiled \& swam; and every track\\*
\vin Was a flash of golden fire.\\!

`O happy living things, no tongue\\*
\vin Their beauty might declare:\\
A spring of love gushed from my heart,\\
\vin And I blessed them unaware:\\
Sure my kind saint took pity on me,\\*
\vin And I blessed them unaware.\\!

`The selfsame moment I could pray;\\*
\vin And from my neck so free\\
The albatross fell off, and sank\\*
\vin Like lead into the sea.\\!

\flagverse{\footnotesize Part V}`O sleep, it is a gentle thing,\\*
\vin Beloved from pole to pole!\\
To \textit{Mary} Queen the praise be given!\\
She sent the gentle sleep from heaven,\\*
\vin That slid into my soul.\\!

`The silly buckets on the deck,\\*
\vin That had so long remained,\\
I dreamt that they were filled with dew;\\*
\vin And when I awoke, it rained.\\!

`My lips were wet; my throat was cold;\\*
\vin My garments all were dank;\\
Sure I had drunken in my dreams,\\*
\vin And still my body drank.\\!

`I moved, and could not feel my limbs:\\*
\vin I was so light -- almost\\
I thought that I had died in sleep,\\*
\vin And was a bless\`{e}d ghost.\\!

`And soon I heard a roaring wind:\\*
\vin It did not come anear;\\
But with its sound it shook the sails,\\*
\vin That were so thin \& sere.\\!

`The upper air burst into life!\\*
\vin And a 100 fire-flags sheen,\\
To \& fro they were hurried about!\\
And to \& fro, and in \& out,\\*
\vin The wan stars danced between.\\!

`And the coming wind did roar more loud,\\*
\vin And the sails did sigh like sedge,\\
And the rain poured down from one black cloud;\\*
\vin The moon was at its edge.\\!

`The thick black cloud was cleft, and still\\*
\vin The moon was at its side:\\
Like waters shot from some high crag,\\
The lightning fell with never a jag,\\*
\vin A river steep \& wide.\\!

`The loud wind never reached the ship,\\*
\vin Yet now the ship moved on!\\
Beneath the lightning and the moon\\*
\vin The dead men gave a groan.\\!

`They groaned; they stirred; they all uprose,\\*
\vin Nor spake, nor moved their eyes;\\
It had been strange, even in a dream,\\*
\vin To have seen those dead men rise.\\!

`The helmsman steered, the ship moved on;\\*
\vin Yet never a breeze up-blew;\\
The mariners all 'gan work the ropes,\\
\vin Where they were wont to do;\\
They raised their limbs like lifeless tools --\\*
\vin We were a ghastly crew.\\!

`The body of my brother's son\\*
\vin Stood by me, knee to knee:\\
The body \& I pulled at one rope,\\*
\vin But he said nought to me.'\\!

`I fear thee, ancient mariner!'\\*
\vin `Be calm, thou wedding-guest!\\
'Twas not those souls that fled in pain,\\
Which to their corses came again,\\*
\vin But a troop of spirits blest:\\!

`For when it dawned -- they dropped their arms,\\*
\vin And clustered round the mast;\\
Sweet sounds rose slowly through their mouths,\\*
\vin And from their bodies passed.\\!

`Around, around, flew each sweet sound,\\*
\vin Then darted to the sun;\\
Slowly the sounds came back again,\\*
\vin Now mixed, now one by one.\\!

`Sometimes a-dropping from the sky\\*
\vin I heard the sky-lark sing;\\
Sometimes all little birds that are,\\
How they seemed to fill the sea and air\\*
\vin With their sweet jargoning!\\!

`And now 'twas like all instruments,\\*
\vin Now like a lonely flute;\\
And now it is an angel's song,\\*
\vin That makes the heavens be mute.\\!

`It ceased; yet still the sails made on\\*
\vin A pleasant noise till noon,\\
A noise like of a hidden brook\\
\vin In the leafy month of june,\\
That to the sleeping woods all night\\*
\vin Singeth a quiet tune.\\!

`Till noon we quietly sailed on,\\*
\vin Yet never a breeze did breathe:\\
Slowly \& smoothly went the ship,\\*
\vin Moved onward from beneath.\\!

`Under the keel nine fathom deep,\\*
\vin From the land of mist \& snow,\\
The spirit slid: and it was he\\
\vin That made the ship to go.\\
The sails at noon left off their tune,\\*
\vin And the ship stood still also.\\!

`The sun, right up above the mast,\\*
\vin Had fixed her to the ocean:\\
But in a minute she 'gan stir,\\
\vin With a short uneasy motion --\\
Backwards \& forwards \sfrac{$1$}{$2$} her length\\*
\vin With a short uneasy motion.\\!

`Then like a pawing horse let go,\\*
\vin She made a sudden bound:\\
It flung the blood into my head,\\*
\vin And I fell down in a swound.\\!

`How long in that same fit I lay,\\*
\vin I have not to declare;\\
But ere my living life returned,\\
I heard and in my soul discerned\\*
\vin Two voices in the air.\\!

`{``}Is it he?'' quoth one, ``Is this the man?\\*
\vin By him who died on cross,\\
With his cruel bow he laid full low\\*
\vin The harmless albatross.\\!

`{``}The spirit who bideth by himself\\*
\vin In the land of mist \& snow,\\
He loved the bird that loved the man\\*
\vin Who shot him with his bow.''\\!

`The other was a softer voice,\\*
\vin As soft as honey-dew:\\
Quoth he, ``The man hath penance done,\\*
\vin And penance more will do.''\\!

\flagverse{\footnotesize Part VI}`{``}But tell me! Tell me! Speak again,\\*
\flagverse{\footnotesize First Voice}\vin Thy soft response renewing --\\
What makes that ship drive on so fast?\\*
\vin What is the ocean doing?''\\!

\flagverse{\footnotesize Second Voice}`{``}Still as a slave before his lord,\\*
\vin The ocean hath no blast;\\
His great bright eye most silently\\*
\vin Up to the moon is cast --\\!

`{``}If he may know which way to go;\\*
\vin For she guides him smooth or grim.\\
See, brother, see! how graciously\\*
\vin She looketh down on him.''\\!

\flagverse{\footnotesize First Voice}`{``}But why drives on that ship so fast,\\*
\vin Without or wave or wind?''\\!

\flagverse{\footnotesize Second Voice}`{``}The air is cut away before,\\*
\vin And closes from behind.\\!

`{``}Fly, brother, fly! More high, more high!\\*
\vin Or we shall be belated:\\
For slow \& slow that ship will go,\\*
\vin When the mariner's trance is abated.''\\!

`I woke, and we were sailing on\\*
\vin As in a gentle weather:\\
'Twas night, calm night, the moon was high;\\*
\vin The dead men stood together.\\!

`All stood together on the deck,\\*
\vin For a charnel-dungeon fitter:\\
All fixed on me their stony eyes,\\*
\vin That in the moon did glitter.\\!

`The pang, the curse, with which they died,\\*
\vin Had never passed away:\\
I could not draw my eyes from theirs,\\*
\vin Nor turn them up to pray.\\!

`And now this spell was snapped: once more\\*
\vin I viewed the ocean green,\\
And looked far forth, yet little saw\\*
\vin Of what had else been seen --\\!

`Like one, that on a lonesome road\\*
\vin Doth walk in fear \& dread,\\
And having once turned round walks on,\\
\vin And turns no more his head;\\
Because he knows, a frightful fiend\\*
\vin Doth close behind him tread.\\!

`But soon there breathed a wind on me,\\*
\vin Nor sound nor motion made:\\
Its path was not upon the sea,\\*
\vin In ripple or in shade.\\!

`It raised my hair; it fanned my cheek\\*
\vin Like a meadow-gale of spring --\\
It mingled strangely with my fears,\\*
\vin Yet it felt like a welcoming.\\!

`Swiftly, swiftly flew the ship,\\*
\vin Yet she sailed softly too:\\
Sweetly, sweetly blew the breeze --\\*
\vin On me alone it blew.\\!

`O dream of joy, is this indeed\\*
\vin The light-house top I see?\\
Is this the hill? Is this the kirk?\\*
\vin Is this mine own country?\\!

`We drifted o'er the harbour-bar,\\*
\vin And I with sobs did pray --\\
``O let me be awake, my God!\\*
\vin Or let me sleep alway.''\\!

`The harbour-bay was clear as glass,\\*
\vin So smoothly it was strewn.\\
And on the bay the moonlight lay,\\*
\vin And the shadow of the moon.\\!

`The rock shone bright, the kirk no less,\\*
\vin That stands above the rock:\\
The moonlight steeped in silentness\\*
\vin The steady weathercock.\\!

`And the bay was white with silent light,\\*
\vin Till rising from the same,\\
Full many shapes, that shadows were,\\*
\vin In crimson colours came.\\!

`A little distance from the prow\\*
\vin Those crimson shadows were:\\
I turned my eyes upon the deck --\\*
\vin O \textit{Christ}! What saw I there!\\!

`Each corse lay flat, lifeless \& flat,\\*
\vin And, by the holy rood!\\
A man all light, a seraph-man,\\*
\vin On every corse there stood.\\!

`This seraph-band, each waved his hand:\\*
\vin It was a heavenly sight!\\
They stood as signals to the land,\\*
\vin Each one a lovely light;\\!

`This seraph-band, each waved his hand,\\*
\vin No voice did they impart --\\
No voice; but O the silence sank\\*
\vin Like music on my heart.\\!

`But soon I heard the dash of oars,\\*
\vin I heard the pilot's cheer;\\
My head was turned perforce away\\*
\vin And I saw a boat appear.\\!

`The pilot \& the pilot's boy,\\*
\vin I heard them coming fast:\\
Dear Lord in Hhaven! it was a joy\\*
\vin The dead men could not blast.\\!

`I saw a third -- I heard his voice:\\*
\vin It is the hermit good!\\
He singeth loud his godly hymns\\
\vin That he makes in the wood.\\
He'll shrieve my soul, he'll wash away\\*
\vin The albatross's blood.\\!

\flagverse{\footnotesize Part VII}`This hermit good lives in that wood\\*
\vin Which slopes down to the sea.\\
How loudly his sweet voice he rears!\\
He loves to talk with marineres\\*
\vin That come from a far country.\\!

`He kneels at morn, and noon, and eve --\\*
\vin He hath a cushion plump:\\
It is the moss that wholly hides\\*
\vin The rotted old oak-stump.\\!

`The skiff-boat neared: I heard them talk,\\*
\vin ``Why, this is strange, I trow!\\
Where are those lights so many \& fair,\\*
\vin That signal made but now?''\\!

`{``}Strange, by my faith!'' the hermit said,\\*
\vin ``And they answered not our cheer!\\
The planks looked warped! And see those sails,\\
\vin How thin they are \& sere!\\
I never saw aught like to them,\\*
\vin Unless perchance it were\\!

`{``}Brown skeletons of leaves that lag\\*
\vin My forest-brook along;\\
When the ivy-tod is heavy with snow,\\
And the owlet whoops to the wolf below,\\*
\vin That eats the she-wolf's young.''\\!

`{``}Dear Lord, it hath a fiendish look --''\\*
\vin The pilot made reply,\\
``I am afeared.'' ``Push on! Push on!''\\*
\vin Said the hermit cheerily.\\!

`The boat came closer to the ship,\\*
\vin But I nor spake nor stirred;\\
The boat came close beneath the ship,\\*
\vin And straight a sound was heard.\\!

`Under the water it rumbled on,\\*
\vin Still louder \& more dread:\\
It reached the ship; it split the bay;\\*
\vin The ship went down like lead.\\!

`Stunned by that loud \& dreadful sound,\\*
\vin Which sky \& ocean smote,\\
Like one that hath been seven days drowned\\
\vin My body lay afloat;\\
But swift as dreams, myself I found\\*
\vin Within the pilot's boat.\\!

`Upon the whirl, where sank the ship,\\*
\vin The boat spun round \& round;\\
And all was still, save that the hill\\*
\vin Was telling of the sound.\\!

`I moved my lips -- the pilot shrieked\\*
\vin And fell down in a fit;\\
The holy hermit raised his eyes,\\*
\vin And prayed where he did sit.\\!

`I took the oars: the pilot's boy,\\*
\vin Who now doth crazy go,\\
Laughed loud \& long, and all the while\\
\vin His eyes went to \& fro.\\
``Ha ha!'' quoth he, ``Full plain I see,\\*
\vin The devil knows how to row.''\\!

`And now, all in my own country,\\*
\vin I stood on the firm land.\\
The hermit stepped forth from the boat,\\*
\vin And scarcely he could stand.\\!

`{``}O shrieve me! Shrieve me, holy man!''\\*
\vin The hermit crossed his brow.\\
``Say quick,' quoth he, ``I bid thee say --\\*
\vin What manner of man art thou?''\\!

`Forthwith this frame of mine was wrenched\\*
\vin With a woeful agony,\\
Which forced me to begin my tale;\\*
\vin And then it left me free.\\!

`Since then, at an uncertain hour,\\*
\vin That agony returns:\\
And till my ghastly tale is told,\\*
\vin This heart within me burns.\\!

`I pass, like night, from land to land;\\*
\vin I have strange power of speech;\\
That moment that his face I see,\\
I know the man that must hear me:\\*
\vin To him my tale I teach.\\!

`What loud uproar bursts from that door!\\*
\vin The wedding-guests are there:\\
But in the garden-bower the bride\\
\vin And bride-maids singing are:\\
And hark the little vesper bell,\\*
\vin Which biddeth me to prayer!\\!

`O wedding-guest, this soul hath been\\*
\vin Alone on a wide wide sea:\\
So lonely 'twas, that God himself\\*
\vin Scarce seem\`{e}d there to be.\\!

`O sweeter than the marriage-feast,\\*
\vin 'Tis sweeter far to me,\\
To walk together to the kirk\\*
\vin With a goodly company!\\!

`To walk together to the kirk,\\*
\vin And all together pray,\\
While each to his great Father bends,\\
Old men, and babes, and loving friends\\*
\vin And youths \& maidens gay!\\!

`Farewell, farewell! But this I tell\\*
\vin To thee, thou wedding-guest!\\
He prayeth well, who loveth well\\*
\vin Both man \& bird \& beast.\\!

`He prayeth best, who loveth best\\*
\vin All things both great \& small;\\
For the dear God who loveth us,\\*
\vin He made \& loveth all.'\\!

The mariner, whose eye is bright,\\*
\vin Whose beard with age is hoar,\\
Is gone: and now the wedding-guest\\*
\vin Turned from the bridegroom's door.\\!

He went like one that hath been stunned,\\*
\vin And is of sense forlorn:\\
A sadder \& a wiser man,\\*
\vin He rose the morrow morn.
\end{verse}

\subsection{}

\blfootnote{`R\'esum\'e', Mrs Dorothy Parker (1893 -- 1967), \cite{norton}.}\settowidth{\versewidth}{    And drugs cause cramp.}
\begin{verse}[\versewidth]
Razors pain you;\\*
\vin Rivers are damp;\\
Acids stain you;\\
\vin And drugs cause cramp.\\
Guns aren't lawful;\\
\vin Nooses give;\\
Gas smells awful;\\*
\vin You might as well live.
\end{verse}

\subsection{}

\blfootnote{Sir Thomas Browne (1605 -- 1682), \cite{odq}.}For the world, I count it not an inn, but an hospital, and a place, not to live, but to die in.

\section{}

\subsection{}

\blfootnote{`Frost at Midnight', Samuel Coleridge (1772 -- 1834), \cite{norton}. In folklore, the flakes of ash floating up the flue were said to predict the arrival of strangers, and thus Coleridge refers to them as such.}\settowidth{\versewidth}{Of my sweet birth-place, and the old church-tower,}
\begin{verse}[\versewidth]
The frost performs its secret ministry,\\*
Unhelped by any wind. The owlet's cry\\
Came loud -- and hark, again, loud as before.\\
The inmates of my cottage, all at rest,\\
Have left me to that solitude, which suits\\
Abstruser musings: save that at my side\\
My cradled infant slumbers peacefully.\\
'Tis calm indeed, so calm that it disturbs\\
And vexes meditation with its strange\\
And extreme silentness. Sea, hill, \& wood,\\
This populous village! Sea, \& hill, \& wood,\\
With all the numberless goings-on of life,\\
Inaudible as dreams! The thin blue flame\\
Lies on my low-burnt fire, and quivers not;\\
Only that film, which fluttered on the grate,\\
Still flutters there, the sole unquiet thing.\\
Methinks, its motion in this hush of nature\\
Gives it dim sympathies with me who live,\\
Making it a companionable form,\\
Whose puny flaps \& freaks the idling spirit\\
By its own moods interprets, everywhere\\
Echo or mirror seeking of itself,\\*
And makes a toy of thought.\\!

\textcolor{white}{And makes a toy of thought.} But O how oft,\\*
How oft, at school, with most believing mind,\\
Presageful, have I gazed upon the bars,\\
To watch that fluttering str\'{a}nger, and as oft\\
With unclosed lids, already had I dreamt\\
Of my sweet birth-place, and the old church-tower,\\
Whose bells, the poor man's only music, rang\\
From morn to evening, all the hot fair-day,\\
So sweetly, that they stirred \& haunted me\\
With a wild pleasure, falling on mine ear\\
Most like articulate sounds of things to come.\\
So gazed I, till the soothing things, I dreamt,\\
Lulled me to sleep, and sleep prolonged my dreams.\\
And so I brooded all the following morn,\\
Awed by the stern preceptor's face, mine eye\\
Fixed with mock study on my swimming book:\\
Save if the door half opened, and I snatched\\
A hasty glance, and still my heart leaped up,\\
For still I hoped to see the str\'{a}nger's face,\\
Townsman, or aunt, or sister more beloved,\\*
My play-mate when we both were clothed alike.\\!

Dear babe, that sleepest cradled by my side,\\*
Whose gentle breathings, heard in this deep calm,\\
Fill up the intersperséd vacancies\\
And momentary pauses of the thought.\\
My babe so beautiful, it thrills my heart\\
With tender gladness, thus to look at thee,\\
And think that thou shalt learn far other lore,\\
And in far other scenes. For I was reared\\
In the great city, pent 'mid cloisters dim,\\
And saw nought lovely but the sky \& stars.\\
But th\'{o}u, my babe, shalt wander like a breeze\\
By lakes \& sandy shores, beneath the crags\\
Of ancient mountain, and beneath the clouds,\\
Which image in their bulk both lakes \& shores\\
And mountain crags: so shalt thou see \& hear\\
The lovely shapes \& sounds intelligible\\
Of that eternal language, which thy God\\
Utters, who from eternity doth teach\\
Himself in all, and all things in himself.\\
Great universal Teacher, he shall mould\\*
Thy spirit, and by giving make it ask.\\!

Therefore all seasons shall be sweet to thee,\\*
Whether the summer clothe the general earth\\
With greenness, or the redbreast sit \& sing\\
Betwixt the tufts of snow on the bare branch\\
Of mossy apple-tree, while the night-thatch\\
Smokes in the sun-thaw; whether the eave-drops fall\\
Heard only in the trances of the blast,\\
Or if the secret ministry of frost\\
Shall hang them up in silent icicles,\\*
Quietly shining to the quiet moon.
\end{verse}

\subsection{}

\blfootnote{`The Cross of Snow', Prof Henry Longfellow (1807 -- 1882), \cite{norton}. Prof Longfellow survived both of his wives. The first, Elizabeth, died at twenty-two following a miscarriage. The second, Frances, having given him six children, died in an horrific accident; her dress caught fire while she was sealing envelopes with melted wax, and, although Prof Longfellow heroically tried to smother the flames with his own body, she was burned to death. Naturally, the professor was badly burned himself, which perhaps explains the `cross of snow... I wear upon my breast'.}\settowidth{\versewidth}{    Looks at me from the wall, where round its head}
\begin{verse}[\versewidth]
In the long, sleepless watches of the night,\\*
\vin A gentle face -- the face of one long dead --\\
\vin Looks at me from the wall, where round its head\\
The night-lamp casts a halo of pale light.\\
Here in this room she died; and soul more white\\
\vin Never through martyrdom of fire was led\\
\vin To its repose; nor can in books be read\\
The legend of a life more benedight.\\
There is a mountain in the distant west\\
\vin That, sun-defying, in its deep ravines\\
\vin \vin Displays a cross of snow upon its side.\\
Such is the cross I wear upon my breast\\
\vin These 18 years, through all the changing scenes\\*
\vin \vin And seasons, changeless since the day she died.
\end{verse}

\subsection{}

\blfootnote{\cite{bcp}.}There was never any thing by the wit of man so well devised, or so sure established, which in continuance of time hath not been corrupted.

\section{}

\subsection{}

\blfootnote{$\mathbb{R}$ `Ode to a Nightingale', John Keats (1795 -- 1821), \cite{treasury}. Where the Almanacker gives `foreign', Palgrave gives `alien'; `foreign' is the Almanacker's invention, but `alien' is an intolerable metrical sin.}\settowidth{\versewidth}{        And with thee fade away into the forest dim:}
\begin{verse}[\versewidth]
My heart aches, and a drowsy numbness pains\\*
\vin My sense, as though of hemlock I had drunk,\\
Or emptied some dull opiate to the drains\\
\vin One minute past, and \textsc{Lethe}-wards had sunk:\\
'Tis not through envy of thy happy lot,\\
\vin But being too happy in thine happiness --\\
\vin \vin That thou, light-wing\`{e}d dryad of the trees\\
\vin \vin \vin \vin In some melodious plot\\
\vin Of beechen green, \& shadows numberless,\\*
\vin \vin Singest of summer in full-throated ease.\\!

O for a draught of vintage! that hath been\\*
\vin Cooled a long age in the deep-delv\`{e}d earth,\\
Tasting of \textit{Flora} \& the country green,\\
\vin Dance, \& provencal song, \& sunburnt mirth!\\
O for a beaker full of the warm south,\\
\vin Full of the true, the blushful \textsc{Hippocrene},\\
\vin \vin With beaded bubbles winking at the brim,\\
\vin \vin \vin \vin And purple-stain\`{e}d mouth;\\
\vin That I might drink, and leave the world unseen,\\*
\vin \vin And with thee fade away into the forest dim:\\!

Fade far away, dissolve, and quite forget\\*
\vin What thou among the leaves hast never known,\\
The weariness, the fever, \& the fret\\
\vin Here, where men sit and hear each other groan;\\
Where palsy shakes a few, sad, last gray hairs,\\
\vin Where youth grows pale, \& spectre-thin, and dies;\\
\vin \vin Where but to think is to be full of sorrow\\
\vin \vin \vin \vin And leaden-eyed despairs,\\
\vin Where beauty cannot keep her lustrous eyes,\\*
\vin \vin Or new Love pine at them beyond tomorrow.\\!

Away! away! for I will fly to thee,\\*
\vin Not charioted by \textit{Bacchus} \& his pards,\\
But on the viewless wings of poesy,\\
\vin Though the dull brain perplexes \& retards:\\
Already with thee! tender is the night,\\
\vin And haply the queen-moon is on her throne,\\
\vin \vin Clustered around by all her starry fays;\\
\vin \vin \vin \vin But here there is no light,\\
\vin Save what from heaven is with the breezes blown\\*
\vin \vin Through verdurous glooms \& winding mossy ways.\\!

I cannot see what flowers are at my feet,\\*
\vin Nor what soft incense hangs upon the boughs,\\
But, in embalm\`{e}d darkness, guess each sweet\\
\vin Wherewith the seasonable month endows\\
The grass, the thicket, \& the fruit-tree wild;\\
\vin White hawthorn, \& the pastoral eglantine;\\
\vin \vin Fast fading violets covered up in leaves;\\
\vin \vin \vin \vin And mid-may's eldest child,\\
\vin The coming musk-rose, full of dewy wine,\\*
\vin \vin The murmurous haunt of flies on summer eves.\\!

Darkling I listen; and, for many a time\\*
\vin I have been \sfrac{$1$}{$2$} in love with easeful death,\\
Called him soft names in many a mus\`{e}d rhyme,\\
\vin To take into the air my quiet breath;\\
Now more than ever seems it rich to die,\\
\vin To cease upon the midnight with no pain,\\
\vin \vin While thou art pouring forth thy soul abroad\\
\vin \vin \vin \vin In such an ecstasy!\\
\vin Still wouldst thou sing, and I have ears in vain --\\*
\vin \vin To thy high requiem become a sod.\\!

Thou wast not born for death, immortal bird!\\*
\vin No hungry generations tread thee down;\\
The voice I hear this passing night was heard\\
\vin In ancient days by emperor \& clown:\\
Perhaps the selfsame song that found a path\\
\vin Through the sad heart of \textit{Ruth}, when, sick for home,\\
\vin \vin She stood in tears amid the foreign corn;\\
\vin \vin \vin \vin The same that oft-times hath\\
\vin Charmed magic casements, opening on the foam\\*
\vin \vin Of perilous seas, in fairy lands forlorn.\\!

Forlorn! The very word is like a bell\\*
\vin To toll me back from thee to my sole self!\\
Adieu! The fancy cannot cheat so well\\
\vin As she is famed to do, deceiving elf.\\
Adieu! Adieu! Thy plaintive anthem fades\\
\vin Past the near meadows, over the still stream,\\
\vin \vin Up the hillside; and now 'tis buried deep\\
\vin \vin \vin \vin In the next valley-glades:\\
\vin Was it a vision, or a waking dream?\\*
\vin \vin Fled is that music. Do I wake or sleep?
\end{verse}

\subsection{}

\blfootnote{Wallace Stevens (1879 -- 1955), \cite{norton}. This is the thireenth of Stevens's famous \refpoem{Thirteen Ways of Looking at a Blackbird}.}\settowidth{\versewidth}{It was evening all afternoon.}
\begin{verse}[\versewidth]
It was evening all afternoon.\\*
It was snowing\\
And it was going to snow.\\
The blackbird sat\\*
In the cedar-limbs.
\end{verse}

\subsection{}

\blfootnote{Mrs Aphra Behn (1640 -- 1689), \cite{odq}.}Come away; poverty's catching.

\section{}

\subsection{}

\blfootnote{`Ode on Melancholy', John Keats (1795 -- 1821), \cite{treasury}.}\settowidth{\versewidth}{    Wolf's-bane, tight-rooted, for its poisonous wine;}
\begin{verse}[\versewidth]
No, no, go not to \textsc{Lethe}, neither twist\\*
\vin Wolf's-bane, tight-rooted, for its poisonous wine;\\
Nor suffer thy pale forehead to be kissed\\
\vin By nightshade, ruby grape of \textit{Proserpine};\\
Make not your rosary of yew-berries,\\
\vin Nor let the beetle, nor the death-moth be\\
\vin \vin Your mournful \textit{Psyche}, nor the downy owl\\
A partner in your sorrow's mysteries;\\
\vin For shade to shade will come too drowsily,\\*
\vin \vin And drown the wakeful anguish of the soul.\\!

But when the melancholy fit shall fall\\*
\vin Sudden from heaven like a weeping cloud,\\
That fosters the droop-headed flowers all,\\
\vin And hides the green hill in an april shroud;\\
Then glut thy sorrow on a morning rose,\\
\vin Or on the rainbow of the salt sand-wave,\\
\vin \vin Or on the wealth of glob\`{e}d peonies;\\
Or if thy mistress some rich anger shows,\\
\vin Emprison her soft hand, and let her rave,\\*
\vin \vin And feed deep, deep upon her peerless eyes.\\!

She dwells with beauty -- beauty that must die;\\*
\vin And joy, whose hand is ever at his lips\\
Bidding adieu; and aching pleasure nigh,\\
\vin Turning to poison while the bee-mouth sips:\\
Ay, in the very temple of delight\\
\vin Veiled melancholy has her sov'reign shrine,\\
\vin \vin Though seen of none save him whose strenuous tongue\\
\vin Can burst joy's grape against his palate fine;\\
His soul shalt taste the sadness of her might,\\*
\vin \vin And be among her cloudy trophies hung.
\end{verse}

\subsection{}

\blfootnote{Miss Emily Brontë (1818 -- 1848), \cite{norton}.}\settowidth{\versewidth}{    Half the sweet enchanting smile;}
\begin{verse}[\versewidth]
Long neglect has worn away\\*
\vin Half the sweet enchanting smile;\\
Time has turned the bloom to grey;\\*
\vin Mould \& damp the face defile.\\!

But that lock of silky hair,\\*
\vin Still beneath the picture twined,\\
Tells what once those features were,\\*
\vin Paints their image on the mind.\\!

Fair the hand that traced that line,\\*
\vin `Dearest, ever deem me true';\\
Swiftly flew the fingers fine\\*
\vin When the pen that motto drew.
\end{verse}

\subsection{}

\blfootnote{Miss Emily Brontë (1818 -- 1848), \cite{odq}.}Proud people breed sad sorrows for themselves.

\section{}

\subsection{}

\blfootnote{$\mathbb{R}$ Psalm 137, \cite{bcp}. \P 11. The Almanacker has reversed the order of the clauses in this line. \P 18. The \textit{BCP} gives `throweth' in this line, but the Almanacker prefers the KJV's `dasheth'.}\settowidth{\versewidth}{Let my tongue cleave to the roof of my mouth, if I do not remember thee; | yea, if I prefer not Jerusalem in my mirth.}
\begin{verse}[\versewidth]
By the waters of \textsc{Babylon} we sat down and wept $\wp$ when we remembered thee, O \textsc{Zion}.\\*
As for our harps, we hanged them up $\wp$ upon the trees that are therein.\\*
For they that led us away captive required of us then a song, and melody in our heaviness: $\wp$ sing us one of the songs of \textsc{Zion}.\\!

How shall we sing the Lord's song $\wp$ in a strange land?\\*
If I forget thee, O \textsc{Jerusalem}, $\wp$ let my right hand forget her cunning.\\*
Let my tongue cleave to the roof of my mouth, if I do not remember thee; $\wp$ yea, if I prefer not \textsc{Jerusalem} in my mirth.\\!

Remember the children of Edom, O Lord, $\wp$ in the day of \textsc{Jerusalem},\\*
How they said, Down with it, down with it, $\wp$ even to the ground.\\
O daughter of \textsc{Babylon}, wasted with misery, $\wp$ yea, happy shall he be that rewardeth thee, as thou hast served us.\\*
Blessed shall he be that taketh thy children $\wp$ and dasheth them against the stones.
\end{verse}

\subsection{}

\blfootnote{`The Silver Swan', This brief poem was made into a famous madrigal by Orlando Gibbons. The identity of the author of the words is unclear, although it may have been Gibbons himself or his patron Sir Christopher Hatton., Anonymous, \cite{norton}.}\settowidth{\versewidth}{When death approached, unlocked her silent throat.}
\begin{verse}[\versewidth]
The silver swan, who, living, had no note,\\*
When death approached, unlocked her silent throat.\\
Leaning her breast upon the reedy shore,\\
Thus sang her first \& last, and sang no more:\\
Farewell, all joys. O death, come close mine eyes.\\*
More geese than swans now live, more fools than wise.
\end{verse}

\subsection{}

\blfootnote{Miss Emily Brontë (1818 -- 1848), \cite{odq}.}The tyrant grinds down his slaves and they don't turn against him; they crush those beneath them.

\section{}

\subsection{}

\blfootnote{Thomas Hardy (1840 -- 1928), \cite{newlove}.}\settowidth{\versewidth}{    Lay not in a heart that could breathe such blame.}
\begin{verse}[\versewidth]
In the vaulted way, where the passage turned\\*
\vin To the shadowy corner that none could see,\\
\vin You paused for our parting -- plaintively:\\
Though overnight had come words that burned\\*
\vin My fond frail happiness out of me.\\!

And then I kissed you -- despite my thought\\*
\vin That our spell must end when reflection came\\
\vin On what you had deemed me, whose one long aim\\
Had been to serve you; that what I sought\\*
\vin Lay not in a heart that could breathe such blame.\\!

But yet I kissed you: whereon you again\\*
\vin As of old kissed me. Why, why was it so?\\
\vin Do you cleave to me after that light-tongued blow?\\
If you scorned me at eventide, how love then?\\*
\vin The thing is dark, dear. I do not know.
\end{verse}

\subsection{}

\blfootnote{Thomas Hardy (1840 -- 1928), \cite{norton}.}\settowidth{\versewidth}{And say, Would God it came to pass}
\begin{verse}[\versewidth]
I look into my glass\\*
\vin And view my wasting skin,\\
And say, Would God it came to pass\\*
\vin My heart had shrunk as thin!\\!

For then, I, undistressed\\*
\vin By hearts grown cold to me,\\
Could lonely wait my endless rest\\*
\vin With equanimity.\\!

But time, to make me grieve,\\*
\vin Part steals, lets part abide;\\
And shakes this fragile frame at eve\\*
\vin With throbbings of noontide.
\end{verse}

\subsection{}

\blfootnote{Abigail Adams, First Lady of the United States (1744 -- 1818), \cite{odq}.}All men would be tyrants if they could.

\section{}

\subsection{}

\blfootnote{Thomas Hardy (1840 -- 1928), \cite{newlove}. \P 16. Grey's Bridge is a bridge over the River Frome just outside of Dorchester, and Durnover Lea is a nearby meadow.}\settowidth{\versewidth}{I shall sit by the fire and wait dreaming}
\begin{verse}[\versewidth]
In the black winter morning\\*
\vin No light will be struck near my eyes\\
While the clock in the stairway is warning\\*
\vin For five, when he used to rise.\\!

{\itshape
Leave the door unbarred,\\
\vin The clock unwound;\\
Make my lone bed hard;\\*
\vin Would 'twere underground!}\\!

When the summer dawns clearly,\\*
\vin And the apple tree tops seem alight,\\
Who will undraw the curtain and cheerly\\*
\vin Call out that the morning is bright?\\!

When I tarry at market\\*
\vin No form will cross \textsc{Durnover Lea}\\
In the gathering darkness, to hark at\\*
\vin \textsc{Grey's Bridge} for the pit-pat o' me.\\!

When the supper crock's steaming,\\*
\vin And the time is the time of his tread,\\
I shall sit by the fire and wait dreaming\\*
\vin In a silence as of the dead.\\!

{\itshape
Leave the door unbarred,\\
\vin The clock unwound;\\
Make my lone bed hard;\\*
\vin Would 'twere underground!}
\end{verse}

\subsection{}

\blfootnote{$\mathbb{R}$ `Anthem for Doomed Youth', Wilfred Owen (1893 -- 1918), \cite{norton}.}\settowidth{\versewidth}{What passing-bells for these who die as cattle?}
\begin{verse}[\versewidth]
What passing-bells for these who die as cattle?\\*
\vin Only the monstrous anger of the guns.\\
Only the stuttering rifles' rapid rattle\\
\vin Can patter out their hasty orisons.\\
No mockeries now for them; no prayers nor bells;\\
\vin Nor any voice of mourning save the choirs,\\
The shrill, demented choirs of wailing shells;\\*
\vin And bugles calling for them from sad shires.\\!

What candles may be held to speed them all?\\*
\vin Not in the hands of boys, but in their eyes\\
\vin Shall shine the holy glimmers of good-bys.\\
The pallor of girls' brows shall be their pall;\\
Their flowers the tenderness of patient minds,\\*
And each slow dusk a drawing-down of blinds.
\end{verse}

\subsection{}

\blfootnote{Washington Irving (1783 -- 1859), \cite{odq}. Irving attributed this quotation to Ayesha, the mother of Sultan Muhammad XII of Granada (called Boabdil by the Spanish), the last Muslim ruler on the Iberian peninsular.}You do well to weep as a woman over what you could not defend as a man.

\section{}

\subsection{}

\blfootnote{`The Newcomer's Wife', Thomas Hardy (1840 -- 1928), \cite{oxfordlarkin}.}\settowidth{\versewidth}{That night there was the splash of a fall}
\begin{verse}[\versewidth]
He paused on the sill of a door ajar\\*
That screened a lively liquor bar,\\
For the name had reached him through the door\\*
Of her he had married the week before.\\!

`We called her the hack of the parade;\\*
But she was discreet in the games she played;\\
If slightly worn, she's pretty yet,\\*
And gossips, after all, forget.\\!

`And he knows nothing of her past;\\*
I am glad the girl's in luck at last;\\
Such ones, though stale to native eyes,\\*
Newcomers snatch at as a prize.'\\!

`Yes, being a stranger he sees her blent\\*
Of all that's fresh \& innocent,\\
Nor dreams how many a love campaign\\*
She had enjoyed before his reign!'\\!

That night there was the splash of a fall\\*
Over the slimy harbour-wall:\\
They searched, and at the deepest place\\*
Found him with crabs upon his face.
\end{verse}

\subsection{}

\blfootnote{`Futility', Wilfred Owen (1893 -- 1918), \cite{norton}.}\settowidth{\versewidth}{At home, whispering of fields half-sown.}
\begin{verse}[\versewidth]
Move him into the sun --\\*
\vin Gently its touch awoke him once,\\
At home, whispering of fields \sfrac{$1$}{$2$}-sown.\\
\vin Always it woke him, even in France,\\
Until this morning \& this snow.\\
If anything might rouse him now\\*
The kind old sun will know.\\!

Think how it wakes the seeds --\\*
\vin Woke once the clays of a cold star.\\
Are limbs, so dear-achieved, are sides\\
\vin Full-nerved, still warm, too hard to stir?\\
Was it for this the clay grew tall?\\
O what made fatuous sunbeams toil\\*
To break earth's sleep at all?
\end{verse}

\subsection{}

\blfootnote{Samuel Butler (1835 -- 1902), \cite{odq}.}All animals, except man, know that the principal business of life is to enjoy it.

\section{}

\subsection{}

\blfootnote{`The Darkling Thrush', Thomas Hardy (1840 -- 1928), \cite{norton}. Hardy began writing this poem on the thirty-first day of December (of the New Style) of 1900.}\settowidth{\versewidth}{The land's sharp features seemed to be}
\begin{verse}[\versewidth]
I leant upon a coppice gate\\*
\vin When frost was spectre-grey,\\
And winter's dregs made desolate\\
\vin The weakening eye of day.\\
The tangled bine-stems scored the sky\\
\vin Like strings of broken lyres,\\
And all mankind that haunted nigh\\*
\vin Had sought their household fires.\\!

The land's sharp features seemed to be\\*
\vin The century's corpse outleant,\\
His crypt the cloudy canopy,\\
\vin The wind his death-lament.\\
The ancient pulse of germ \& birth\\
\vin Was shrunken hard \& dry,\\
And every spirit upon earth\\*
\vin Seemed fervourless as I.\\!

At once a voice arose among\\*
\vin The bleak twigs overhead\\
In a full-hearted evensong\\
\vin Of joy illimited;\\
An ag\`{e}d thrush, frail, gaunt, \& small,\\
\vin In blast-beruffled plume,\\
Had chosen thus to fling his soul\\*
\vin Upon the growing gloom.\\!

So little cause for carolings\\*
\vin Of such ecstatic sound\\
Was written on terrestrial things\\
\vin Afar or nigh around,\\
That I could think there trembled through\\
\vin His happy good-night air\\
Some blessed hope, whereof he knew\\*
\vin And I was unaware.
\end{verse}

\subsection{}

\blfootnote{`God Knows', Miss Minnie Haskins (1875 -- 1957), \cite{haskins}. The Almanacker has excised all but the first verse. George VI recited the first five lines of this poem in the Royal Christmas Message of 1939.}\settowidth{\versewidth}{And he led me towards the hills the breaking of day in the lone east.}
\begin{verse}[\versewidth]
And I said to the man who stood at the gate of the year,\\*
Give me a light that I may tread safely into the unknown.\\
And he replied:\\
Go out into the darkness and put your hand into the hand of God.\\
That shall be to you better than light and safer than a known way.\\
So I went forth, and finding the hand of God, trod gladly into the night.\\*
And he led me towards the hills \& the breaking of day in the lone east.
\end{verse}

\subsection{}

\blfootnote{`This is the penultimate line of \textit{Pilgrim's Progress} (or, more precisely, the first part thereof -- the second part being a kind of sequel). The remaining prose reads: `as well as from the City of Destruction. So I awoke, and behold it was a dream.'', John Bunyan (1628 -- 1688), \cite{odq}.}Then I saw that there was a way to hell, even from the gates of heaven.

\chapter{Intercalaris}

\section{}

\subsection{}

\blfootnote{`Mus{\'e}e des Beaux Arts', Prof Wystan Auden (1907 -- 1973), \cite{audena}. The Mus{\'e}e des Beaux Arts in questions is to be found in Brussels.}\settowidth{\versewidth}{Where the dogs go on with their doggy life and the torturer's horse}
\begin{verse}[\versewidth]
About suffering they were never wrong,\\*
The Old Masters: how well they understood\\
Its human position; how it takes place;\\
While someone else is eating or opening a window or just walking dully along;\\
How, when the ag{\`{e}}d are reverently, passionately waiting\\
For the miraculous birth, there must always be\\
Children who did not specially want it to happen, skating\\
On a pond at the edge of the wood:\\
They never forgot\\
That even the dreadful martyrdom must run its course\\
Anyhow in a corner, some untidy spot\\
Where the dogs go on with their doggy life and the torturer's horse\\*
Scratches its innocent behind on a tree.\\!

In \textit{Bruegel}'s {\hoskeroe Icarus}, for instance: how everything turns away\\*
Quite leisurely from the disaster; the ploughman may\\
Have heard the splash, the forsaken cry,\\
But for him it was not an important failure; the sun shone\\
As it had to on the white legs disappearing into the green\\
Water; and the expensive delicate ship that must have seen\\
Something amazing, a boy falling out of the sky,\\*
Had somewhere to get to and sailed calmly on.
\end{verse}

\subsection{}

\blfootnote{Prof Wystan Auden (1907 -- 1973), \cite{audena}.}\settowidth{\versewidth}{Sings agreeably, agreeably, agreeably of love.}
\begin{verse}[\versewidth]
Carry her over the water,\\*
\vin And set her down under a tree,\\
Where the culvers white all day \& all night\\
\vin And the winds from every \sfrac{$1$}{$4$}\\*
Sing agreeably, agreeably, agreeably of love.\\!

Put a gold ring on her finger\\*
\vin And press her close to your heart,\\
While the fish in the lake their snapshots take,\\
\vin And the frog, that sanguine singer,\\*
Sings agreeably, agreeably, agreeably of love.\\!

The streets shall flock to your marriage,\\*
\vin The houses turn round to look,\\
The tables \& chairs say suitable prayers,\\
\vin And the horses drawing your carriage\\*
Sing agreeably, agreeably, agreeably of love.
\end{verse}

\subsection{}

\blfootnote{Prof Wystan Auden (1907 -- 1973), \cite{audenb}. These are two lines taken from Auden's early poem \refpoem{Missing}.}Heroes are buried who did not believe in death.

\section{}

\subsection{}

\blfootnote{`Their Lonely Betters', Prof Wystan Auden (1907 -- 1973), \cite{audena}.}\settowidth{\versewidth}{While rustling flowers for some third party waited}
\begin{verse}[\versewidth]
As I listened from a beach-chair in the shade\\*
To all the noises that my garden made,\\
It seemed to me only proper that words\\*
Should be withheld from vegetables \& birds.\\!

A robin with no christian name ran through\\*
The robin-anthem which was all it knew,\\
While rustling flowers for some third party waited\\*
To say which pairs, if any, should get mated.\\!

Not one of them was capable of lying;\\*
There was not one of them which knew that it was dying,\\
Or could have with a rhythm or a rhyme\\*
Assumed responsibility for time.\\!

Let them leave language to their lonely betters\\*
Who count some days and long for certain letters;\\
We, too, make noises when we laugh or weep:\\*
Words are for those with promises to keep.
\end{verse}

\subsection{}

\blfootnote{`The Secret Agent', Prof Wystan Auden (1907 -- 1973), \cite{audena}. The text here follows that of the earliest published version. In later editions, Auden amended the final line to read: `Parting easily two that were never joined.'}\settowidth{\versewidth}{For a bogus guide, seduced with the old tricks.}
\begin{verse}[\versewidth]
Control of the passes was, he saw, the key\\*
To this new district, but who would get it?\\
He, the trained spy, had walked into the trap\\*
For a bogus guide, seduced with the old tricks.\\!

At \textsc{Greenhearth} was a fine site for a dam\\*
And easy power, had they pushed the rail\\
Some stations nearer. They ignored his wires.\\*
The bridges were unbuilt and trouble coming.\\!

The street music seemed gracious now to one\\*
For weeks up in the desert. Woken by water\\
Running away in the dark, he often had\\
Reproached the night for a companion\\
Dreamed of already. They would shoot, of course,\\*
Parting easily who were never joined.
\end{verse}

\subsection{}

\blfootnote{Prof Wystan Auden (1907 -- 1973), \cite{audenb}. These words are drawn from two lines of Auden's early poem \refpoem{Let History Be My Judge}.}There could be no question of living if we did not win.

\section{}

\subsection{}

\blfootnote{`Address to the Beasts', Prof Wystan Auden (1907 -- 1973), \cite{audena}.}\settowidth{\versewidth}{Even when we can't see or hear you,}
\begin{verse}[\versewidth]
For us who, from the moment\\*
We are first worlded,\\*
Lapse into disarray,\\!

Who seldom know exactly\\*
What we are up to,\\*
And, as a rule, don't want to,\\!

What a joy to know,\\*
Even when we can't see or hear you,\\*
That you are around,\\!

Though very few of you\\*
Find us worth looking at,\\*
Unless we come too close.\\!

To you all scents are sacred\\*
Except our smell \& those\\*
We manufacture.\\!

How promptly \& ably\\*
You execute nature's policies,\\*
And are never\\!

Lured into misconduct\\*
Except by some unlucky\\*
Chance imprinting.\\!

Endowed from birth with good manners,\\*
You wag no snobbish elbows,\\*
Don't leer,\\!

Don't look down your nostrils,\\*
Nor poke them into another\\*
Creature's business.\\!

Your own habitations\\*
Are cosy \& private, not\\*
Pretentious temples.\\!

Of course, you have to take lives\\*
To keep your own, but never\\*
Kill for applause.\\!

Compared with even your greediest,\\*
How non-U\\*
Our hunting gentry seem.\\!

Exempt from taxation,\\*
You have never felt the need\\*
To become literate,\\!

But your oral cultures\\*
Have inspired our poets to pen\\*
Dulcet verses,\\!

And, though unconscious of God,\\*
Your sung eucharists\\*
Are more hallowed than ours.\\!

Instinct is commonly said\\*
To rule you: I would call it\\*
Common sense.\\!

If you cannot engender\\*
A genius like \textit{Mozart},\\*
Neither can you\\!

Plague the earth\\*
With brilliant sillies like \textit{Hegel}\\*
Or clever nasties like \textit{Hobbes}.\\!

Shall we ever become adulted,\\*
As you all soon do?\\*
It seems unlikely.\\!

Indeed, one balmy day,\\*
We might all become\\*
Not fossils, but vapour.\\!

Distinct now,\\*
In the end we shall join you\\*
(How soon all corpses look alike),\\!

But you exhibit no signs\\*
Of knowing that you are sentenced.\\*
Now, could that be why\\!

We upstarts are often\\*
Jealous of your innocence,\\*
But never envious.
\end{verse}

\subsection{}

\blfootnote{Prof Wystan Auden (1907 -- 1973), \cite{audena}. This is the last verse of \refpoem{The Lesson}.}\settowidth{\versewidth}{I woke. You were not there. But as I dressed}
\begin{verse}[\versewidth]
I woke. You were not there. But as I dressed\\*
Anxiety turned to shame, feeling all three\\
Intended one rebuke. For had not each\\
In its own way tried to teach\\
My will to love you that it cannot be,\\
As I think, of such consequence to want\\*
What anyone is given, if they want?
\end{verse}

\subsection{}

\blfootnote{Prof Wystan Auden (1907 -- 1973), \cite{audenb}.}\settowidth{\versewidth}{I'm afraid there's many a spectacled sod}
\begin{verse}[\versewidth]
I'm afraid there's many a spectacled sod\\*
Prefers the \textsc{British Museum} to God.
\end{verse}

\section{}

\subsection{}

\blfootnote{`No, Plato, No', Prof Wystan Auden (1907 -- 1973), \cite{audena}.}\settowidth{\versewidth}{    My ductless glands for instance,}
\begin{verse}[\versewidth]
I can't imagine anything\\*
\vin That I would less like to be\\
Than a disincarnate spirit,\\
\vin Unable to chew or sip\\
Or make contact with surfaces\\
\vin Or breathe the scents of summer\\
Or comprehend speech or music\\
\vin Or gaze at what lies beyond.\\
No, God has placed me exactly\\
\vin Where I'd have chosen to be:\\
The sub-lunar world is such fun,\\
\vin Where man is male or female\\*
And gives proper names to all things.\\!

\vin I can, however, conceive\\*
That the organs nature gave me,\\
\vin My ductless glands for instance,\\
Slaving 24 hours a day\\
\vin With no show of resentment\\
To gratify me, their master,\\
\vin And keep me in proper shape,\\
(Not that I give them their orders;\\
\vin I wouldn't know what to yell)\\
Dream of another existence\\
\vin Than that they have known so far.\\
Yes, it could well be that my flesh,\\
\vin Is praying for `him' to die,\\
So setting her free to become\\*
\vin Irresponsible matter.
\end{verse}

\subsection{}

\blfootnote{`Short Ode to the Cuckoo', Prof Wystan Auden (1907 -- 1973), \cite{audena}.}\settowidth{\versewidth}{Such as the merle, your two-note act is kid-stuff:}
\begin{verse}[\versewidth]
No one now imagines you answer idle questions\\*
-- How long shall I live? How long remain single?\\
Will butter be cheaper? -- nor does your shout make\\*
\vin Husbands uneasy.\\!

Compared with arias by the great performas\\*
Such as the merle, your two-note act is kid-stuff:\\
Our most hardened crooks are sincerely shocked by\\*
\vin Your nesting habits.\\!

Science, aesthetics, ethics may huff \& puff but they\\*
Cannot extinguish your magic: you marvel\\
The commuter as you wondered the savage.\\*
\vin Hence, in my diary,\\!

Where I normally enter nothing but social\\*
Engagements and, lately, the death of friends, I\\
Scribble year after year when I first hear you,\\*
\vin Of a holy moment.
\end{verse}

\subsection{}

\blfootnote{Prof Wystan Auden (1907 -- 1973), \cite{audenb}. This is a line from Prof Auden's sonnet \refpoem{The Ship}.}One doubts the virtue, one the beauty of his wife.

\section{}

\subsection{}

\blfootnote{`In Praise of Limestone', Prof Wystan Auden (1907 -- 1973), \cite{audena}.}\settowidth{\versewidth}{    For effects that bring down the house, could happen to all}
\begin{verse}[\versewidth]
If it form the landscape that we, the inconstant ones,\\*
\vin Are constantly homesick for, this is chiefly\\
Because it dissolves in water. Mark these rounded slopes\\
\vin With their surface fragrance of thyme and, beneath,\\
A secret system of caves \& conduits; hear the springs\\
\vin That spurt out everywhere with a chuckle,\\
Each filling a private pool for its fish and carving\\
\vin Its own little ravine whose cliffs entertain\\
The butterfly \& the lizard; examine this region\\
\vin Of short distances \& definite places:\\
What could be more like mother or a fitter background\\
\vin For her son, the flirtatious male who lounges\\
Against a rock in the sunlight, never doubting\\
\vin That for all his faults he is loved; whose works are but\\
Extensions of his power to charm? From weathered outcrop\\
\vin To hill-top temple, from appearing waters to\\
Conspicuous fountains, from a wild to a formal vineyard,\\
\vin Are ingenious but short steps that a child's wish\\
To receive more attention than his brothers, whether\\*
\vin By pleasing or teasing, can easily take.\\!

Watch, then, the band of rivals as they climb up \& down\\*
\vin The steep stone gennels in twos \& threes, at times\\
Arm in arm, but never, thank God, in step; or engaged\\
\vin On a shady side of a square at midday in\\
Voluble discourse, knowing each other too well to think\\
\vin There are any important secrets, unable\\
To conceive a god whose temper-tantrums are moral\\
\vin And not to be pacified by a clever line\\
Or a good lay: for, accustomed to a stone that responds,\\
\vin They have never had to veil their faces in awe\\
Of a crater whose blazing fury could not be fixed;\\
\vin Adjusted to the local needs of valleys\\
Where everything can be touched or reached by walking,\\
\vin Their eyes have never looked into infinite space\\
Through the lattice-work of a nomad's comb; born lucky,\\
\vin Their legs have never encountered the fungi\\
And insects of the jungle, the monstrous forms \& lives\\
\vin With which we have nothing, we like to think, in common.\\
So, when one of them goes to the bad, the way his mind works\\
\vin Remains comprehensible: to become a pimp\\
Or deal in fake jewellery or ruin a fine tenor voice\\
\vin For effects that bring down the house, could happen to all\\
But the best \& the worst of us... That is why, I suppose,\\
\vin The best \& worst never stayed here long but sought\\
Immoderate soils where the beauty was not so external,\\
\vin The light less public and the meaning of life\\
Something more than a mad camp. `Come!' cried the granite wastes,\\
\vin `How evasive is your humour, how accidental\\
Your kindest kiss, how permanent is death.' (Saints-to-be\\
\vin Slipped away sighing.) `Come!' purred the clays \& gravels,\\
`On our plains there is room for armies to drill; rivers\\
\vin Wait to be tamed and slaves to construct you a tomb\\
In the grand manner: soft as the earth is mankind and both\\
\vin Need to be altered.' (Intendant Caesars rose and\\
Left, slamming the door.) But the really reckless were fetched\\
\vin By an older colder voice, the oceanic whisper:\\
`I am the solitude that asks \& promises nothing;\\
\vin That is how I shall set you free. There is no love;\\*
There are only the various envies, all of them sad.\\!

\vin They were right, my dear; all those voices were right\\*
And still are; this land is not the sweet home that it looks,\\
\vin Nor its peace the historical calm of a site\\
Where something was settled once \& for all: a backward\\
\vin And dilapidated province, connected\\
To the big busy world by a tunnel, with a certain\\
\vin Seedy appeal, is that all it is now? Not quite:\\
It has a worldly duty which in spite of itself\\
\vin It does not neglect, but calls into question\\
All the great powers assume; it disturbs our rights. The poet,\\
\vin Admired for his earnest habit of calling\\
The sun the sun, his mind puzzle, is made uneasy\\
\vin By these marble statues which so obviously doubt\\
His anti-mythological myth; and these gamins,\\
\vin Pursuing the scientist down the tiled colonnade\\
With such lively offers, rebuke his concern for nature's\\
\vin Remotest aspects: I, too, am reproached, for what\\
And how much you know. Not to lose time, not to get caught,\\
\vin Not to be left behind, not -- please! -- to resemble\\
The beasts who repeat themselves or a thing like water\\
\vin Or stone whose conduct can be predicted, these\\
Are our {\hoskeroe Common Prayer}, whose greatest comfort is music\\
\vin Which can be made anywhere, is invisible,\\
And does not smell. In so far as we have to look forward\\
\vin To death as a fact, no doubt we are right: but if\\
Sins can be forgiven, if bodies rise from the dead,\\
\vin These modifications of matter into\\
Innocent athletes \& gesticulating fountains,\\
\vin Made solely for pleasure, make a further point:\\
The bless\`{e}d will not care what angle they are regarded from,\\
\vin Having nothing to hide. Dear, I know nothing of\\
Either, but when I try to imagine a faultless love\\
\vin Or the life to come, what I hear is the murmur\\*
Of underground streams, what I see is a limestone landscape.
\end{verse}

\subsection{}

\blfootnote{`Epitaph on a Tyrant', Prof Wystan Auden (1907 -- 1973), \cite{audena}.}\settowidth{\versewidth}{And the poetry he invented was easy to understand;}
\begin{verse}[\versewidth]
Perfection, of a kind, was what he was after,\\*
And the poetry he invented was easy to understand;\\
He knew human folly like the back of his hand,\\
As was greatly interested in armies \& fleets;\\
When he laughed, respectable senators burst with laughter,\\*
And when he cried the little children died in the streets.
\end{verse}

\subsection{}

\blfootnote{Prof Wystan Auden (1907 -- 1973), \cite{audenb}. These are two lines from Prof Auden's sonnet \refpoem{Macao}.}Churches alongside brothels testify that faith can pardon natural behaviour.

\section{}

\subsection{}

\blfootnote{`Recitative by Death', Prof Wystan Auden (1907 -- 1973), \cite{audena}.}\settowidth{\versewidth}{    Crashed the sound-barrier, and may very soon}
\begin{verse}[\versewidth]
Ladies \& gentlemen, you have made most remarkable\\*
\vin Progress, and progress, I agree, is a boon;\\
You have built more automobiles than are parkable,\\
\vin Crashed the sound-barrier, and may very soon\\
\vin Be setting up juke-boxes on the moon:\\
But I beg to remind you that, despite all that,\\*
I, death, am \& will always be cosmocrat.\\!

Still I sport with the young \& daring; at my whim\\*
\vin The climber steps upon the rotten boulder;\\
The undertow catches boys as they swim;\\
\vin The speeder steers onto the slippery shoulder:\\
\vin With others I wait until they are older,\\
Before assigning, according to my humour,\\*
To one a coronary, to another a tumour.\\!

Liberal my views on religion \& race;\\*
Tax-posture, credit-rating, social ambition\\
Cut no ice with me. We shall meet face to face\\
\vin Despite the drugs \& lies of your physician,\\
\vin The costly euphemisms of the mortician:\\
\textsc{Westchester} matron and \textsc{Bowery} bum,\\*
Both shall dance with me when I rattle my drum.
\end{verse}

\subsection{}

\blfootnote{`August 1968', Prof Wystan Auden (1907 -- 1973), \cite{audena}. The Soviet-led invasion of Czechoslovakia occurred in August 1968.}\settowidth{\versewidth}{But one prize is beyond his reach;}
\begin{verse}[\versewidth]
The ogre does what ogres can,\\*
Deeds quite impossible for man,\\
But one prize is beyond his reach;\\
The ogre cannot master speech.\\
About a subjugated plain,\\
Among its desperate \& slain,\\
The ogre stands with hands on hips,\\*
While drivel gushes from his lips.
\end{verse}

\subsection{}

\blfootnote{Prof Wystan Auden (1907 -- 1973), \cite{audenb}. This is the last line of Prof Auden's sonnet \refpoem{Macao}.}And nothing serious can happen here.

\section{}

\subsection{}

\blfootnote{`Since', Prof Wystan Auden (1907 -- 1973), \cite{audena}.}\settowidth{\versewidth}{Then, carrying candles, climbed}
\begin{verse}[\versewidth]
On a mid-december day,\\*
Frying sausages\\
For myself, I abruptly\\
Felt under fingers\\
Thirty years younger the rim\\
Of a steering-wheel,\\
On my cheek the parching wind\\
Of an august noon,\\
As passenger beside me\\*
You as then you were.\\!

Slap across a veg'-growing\\*
Alluvial plain\\
We raced in clouds of white dust,\\
And geese fled screaming\\
As we missed them by inches,\\
Making a bee-line\\
For mountains gradually\\
Enlarging eastward,\\
Joyfully certain nightfall\\*
Would occasion joy.\\!

It did. In a flagged kitchen\\*
We were served broiled trout\\
And a rank cheese: for a while\\
We talked by the fire,\\
Then, carrying candles, climbed\\
Steep stairs. Love was made\\
Then \& there: so halcyoned,\\
Soon we fell asleep\\
To the sound of a river\\*
Swabbling through a gorge.\\!

Since then, other enchantments\\*
Have blazed \& faded,\\
Enemies changed their address,\\
And war made ugly\\
An uncountable number\\
Of unknown neighbours,\\
Precious as us to themselves:\\
But round your image\\
There is no fog, and the earth\\*
Can still astonish.\\!

Of what, then, should I complain,\\*
Pottering about\\
A neat suburban kitchen?\\
Solitude? Rubbish!\\
It's social enough with real\\
Faces \& landscapes\\
For whose friendly countenance\\
I at least can learn\\
To live with obesity\\*
And a little fame.
\end{verse}

\subsection{}

\blfootnote{`This Lunar Beauty', Prof Wystan Auden (1907 -- 1973), \cite{audenb}. Prof Auden made minor, but semantically helpful, amendments to the punctuation of this poem in later editions.}\settowidth{\versewidth}{And the heart's changes}
\begin{verse}[\versewidth]
This lunar beauty\\*
Has no history,\\
Is complete \& early;\\
If beauty later\\
Bear any feature,\\
It had a lover\\*
And is another.\\!

This like a dream\\*
Keeps other time,\\
And daytime is\\
The loss of this;\\
For time is inches\\
And the heart's changes\\
Where ghost has haunted,\\*
Lost \& wanted.\\!

But this was never\\*
A ghost's endeavour\\
Nor, finished this,\\
Was ghost at ease;\\
And till it pass\\
Love shall not near\\
The sweetness here\\
Nor sorrow take\\*
His endless look.
\end{verse}

\subsection{}

\blfootnote{Prof Wystan Auden (1907 -- 1973), \cite{audenb}. This is the last line of Prof Auden's \refpoem{Another Time}.}Another time has other lives to live.

\section{}

\subsection{}

\blfootnote{`O What Is That Sound', Prof Wystan Auden (1907 -- 1973), \cite{audena}. Note that Prof Auden consistently declined to add a question mark to the title of this poem.}\settowidth{\versewidth}{    What are they doing this morning, this morning?}
\begin{verse}[\versewidth]
O what is that sound which so thrills the ear\\*
\vin Down in the valley drumming, drumming?\\
Only the scarlet soldiers, dear,\\*
\vin \vin The soldiers coming.\\!

O what is that light I see flashing so clear\\*
\vin Over the distance brightly, brightly?\\
Only the sun on their weapons, dear,\\*
\vin \vin As they step lightly.\\!

O what are they doing with all that gear?\\*
\vin What are they doing this morning, this morning?\\
Only the usual manoeuvres, dear,\\*
\vin \vin Or perhaps a warning.\\!

O why have they left the road down there;\\*
\vin Why are they suddenly wheeling, wheeling?\\
Perhaps a change in the orders, dear;\\*
\vin \vin Why are you kneeling?\\!

O haven't they stopped for the doctor's care;\\*
\vin Haven't they reined their horses, their horses?\\
Why, they are none of them wounded, dear,\\*
\vin \vin None of these forces.\\!

O is it the parson they want with white hair;\\*
\vin Is it the parson, is it, is it?\\
No, they are passing his gateway, dear,\\*
\vin \vin Without a visit.\\!

O it must be the farmer who lives so near;\\*
\vin It must be the farmer so cunning, so cunning.\\
They have passed the farm already, dear,\\*
\vin \vin And now they are running.\\!

O where are you going? Stay with me here!\\*
\vin Were the vows you swore me deceiving, deceiving.\\
No, I promised to love you, dear,\\*
\vin \vin But I must be leaving.\\!

O it's broken the lock \& splintered the door;\\*
\vin O it's the gate where they're turning, turning;\\
Their feet are heavy on the floor\\*
\vin \vin And their eyes are burning.
\end{verse}

\subsection{}

\blfootnote{Prof Wystan Auden (1907 -- 1973), \cite{audenb}.}\settowidth{\versewidth}{No one guesses you are weak.}
\begin{verse}[\versewidth]
Pick a quarrel, go to war,\\*
Leave the hero in the bar;\\
Hunt the lion, climb the peak:\\*
No one guesses you are weak.
\end{verse}

\subsection{}

\blfootnote{Prof Wystan Auden (1907 -- 1973), \cite{audenb}. This couplet was taken from one of the \refpoem{Shorts} Prof Auden composed around 1940.}\settowidth{\versewidth}{Any heaven we think it decent to enter}
\begin{verse}[\versewidth]
Any heaven we think it decent to enter\\*
Must be ptolemaic with ourselves at the centre.
\end{verse}

\section{}

\subsection{}

\blfootnote{`Epilogue', Prof Wystan Auden (1907 -- 1973), \cite{audena}. Prof Auden was certainly inspired by the third Child Ballad (\refpoem{The Fause Knight on the Road}) in writing this poem. It is an epilogue in the sense that it is the final piece in \refbook{The Orators}, the most enigmatic of his anthologies.}\settowidth{\versewidth}{O what was that bird?' said horror to hearer.}
\begin{verse}[\versewidth]
O where are you going? said reader to rider.\\*
That valley is fatal where furnaces burn;\\
Younder's the midden whose odours will madden;\\*
That gap is the grave where the tall return.\\!

O do you imagine, said fearer to farer,\\*
That dusk will delay on your path to the pass,\\
Your diligent looking discover the lacking\\*
Your footsteps feel from granite to grass?\\!

O what was that bird?' said horror to hearer.\\*
Did you see that shape in the twisted trees?\\
Behind you swiftly the figure comes softly;\\*
That spot on your skin is a shocking disease.\\!

Out of this house, said rider to reader;\\*
Yours never will, said farer to fearer;\\
They're looking for you, said hearer to horror\\*
As he left him there, as he left him there.
\end{verse}

\subsection{}

\blfootnote{Prof Wystan Auden (1907 -- 1973), \cite{audenb}.}\settowidth{\versewidth}{I'm beginning to lose patience}
\begin{verse}[\versewidth]
I'm beginning to lose patience\\*
With my personal relations:\\
They are not deep,\\*
And they are not cheap.
\end{verse}

\subsection{}

\blfootnote{Prof Wystan Auden (1907 -- 1973), \cite{audenb}. This is a line from Prof Auden's \refpoem{Leap Before You Look}.}Look if you like, but you will have to leap.

\section{}

\subsection{}

\blfootnote{`Funeral Blues', Prof Wystan Auden (1907 -- 1973), \cite{audena}.}\settowidth{\versewidth}{I thought that love would last forever: I was wrong.}
\begin{verse}[\versewidth]
Stop all the clocks, cut off the telephone,\\*
Prevent the dog from barking with a juicy bone,\\
Silence the pianos and with muffled drum\\*
Bring out the coffin, let the mourners come.\\!

Let aeroplanes circle, moaning overhead,\\*
Scribbling on the sky the message, He is dead;\\
Put {\hoskeroe cr\^{e}pe} bows round the white necks of the public doves;\\*
Let the traffic policemen wear black cotton gloves.\\!

He was my north, my south, my east \& west,\\*
My working week and my sunday rest,\\
My noon, my midnight, my talk, my song;\\*
I thought that love would last forever: I was wrong.\\!

The stars are not wanted now; put out every one,\\*
Pack up the moon and dismantle the sun;\\
Pour away the ocean and sweep up the wood;\\*
For nothing now can ever come to any good.
\end{verse}

\subsection{}

\blfootnote{Prof Wystan Auden (1907 -- 1973), \cite{audenb}. The terms `vertical' and `horizontal' in this short poem refer to the living and the dead respectively.}\settowidth{\versewidth}{Let us honour if we can}
\begin{verse}[\versewidth]
Let us honour if we can\\*
The vertical man,\\
Though we value none\\*
But the horizontal one.
\end{verse}

\subsection{}

\blfootnote{Prof Wystan Auden (1907 -- 1973), \cite{audenb}. This is the title of a poem.}Music is international.

\section{}

\subsection{}

\blfootnote{`Taller Today', Prof Wystan Auden (1907 -- 1973), \cite{audena}. The text here follows that of the earliest published version. In later editions, Auden excised the second and third verses.}\settowidth{\versewidth}{Where the brook runs over the gravel, far from the glacier.}
\begin{verse}[\versewidth]
Taller today, we remember similar evenings, evenings,\\*
Walking together in the windless orchard\\*
Where the brook runs over the gravel, far from the glacier.\\!

Again in the room with the sofa hiding the grate,\\*
Look down to the river when the rain is over,\\
See him turn to the window, hearing our last\\*
Of Captain \textit{Ferguson}.\\!

It is seen how excellent hands have turned to commonness.\\*
One staring too long, went blind in a tower,\\*
One sold all his manors to fight, broke through, and faltered.\\!

Nights come bringing the snow, and the dead howl\\*
Under the headlands in their windy dwelling\\
Because the Adversary put too easy questions\\*
On lonely roads\\!

But happy now, though no nearer each other,\\*
We see the farms lighted all along the valley;\\
Down at the mill-shed the hammering stops\\*
And men go home.\\!

Noises at dawn will bring\\*
Freedom for some, but not this peace\\
No bird can contradict: passing, but is sufficient now\\*
For something fulfilled this hour, loved or endured.
\end{verse}

\subsection{}

\blfootnote{Prof Wystan Auden (1907 -- 1973), \cite{audenb}.}\settowidth{\versewidth}{These had stopped seeking}
\begin{verse}[\versewidth]
These had stopped seeking\\*
But went on speaking,\\
Have not contributed,\\*
But have diluted.\\!

These ordered light\\*
But had no right,\\
And handed on\\*
War \& a son.
\end{verse}

\subsection{}

\blfootnote{Prof Wystan Auden (1907 -- 1973), \cite{audenb}. This was Prof Auden's suggestion for an epitaph for a tomb of the unknown soldier.}\settowidth{\versewidth}{To save your world, you asked this man to die:}
\begin{verse}[\versewidth]
To save your world, you asked this man to die:\\*
Would this man, could he see you now, ask why?
\end{verse}

\section{}

\subsection{}

\blfootnote{`The Fall of Rome', Prof Wystan Auden (1907 -- 1973), \cite{audena}. Prof Auden dedicated the poem to Cyril Connolly. Each verse seems to consider a reason frequently given by historians for the collapse of the Roman Empire. In particular, the last verse concerns the following theory: changes in the climate forced the migration of certain species (amongst them, reindeer) on the steppes of Eastern Europe, which in turn forced the migration of those tribes which depended on said species. This led to a chain of tribal migrations, culminating in the barbarian invasions of Late Antiquity which brought about the Empire's demise.}\settowidth{\versewidth}{The piers are pummelled by the waves;}
\begin{verse}[\versewidth]
The piers are pummelled by the waves;\\*
In a lonely field the rain\\
Lashes an abandoned train;\\*
Outlaws fill the mountain caves.\\!

Fantastic grow the evening gowns;\\*
Agents of the Fisc pursue\\
Absconding tax-defaulters through\\*
The sewers of provincial towns.\\!

Private rites of magic send\\*
The temple prostitutes to sleep;\\
All the literati keep\\*
An imaginary friend.\\!

Cerebrotonic \textit{Cato} may\\*
Extol the ancient disciplines,\\
But the muscle-bound marines\\*
Mutiny for food \& pay.\\!

Caesar's double bed is warm\\*
While an unimportant clerk\\
Writes, `I do not like my work,'\\*
On a pink official form.\\!

Unendowed with wealth or pity,\\*
Little birds with scarlet legs,\\
Sitting on their speckled eggs,\\*
Eye each flu-infected city.\\!

Altogether elsewhere, vast\\*
Herds of reindeer move across\\
Miles \& miles of golden moss,\\*
Silently and very fast.
\end{verse}

\subsection{}

\blfootnote{Prof Wystan Auden (1907 -- 1973), \cite{audenb}.}\settowidth{\versewidth}{Outgrows his nervous laugh,}
\begin{verse}[\versewidth]
That night when joy began\\*
Our narrowest veins to flush,\\
We waited for the flash\\*
Of morning's levelled gun.\\!

But morning let us pass,\\*
And day by day relief\\
Outgrows his nervous laugh,\\*
Grown credulous of peace,\\!

As mile by mile is seen\\*
No trespasser's reproach,\\
And love's best glasses reach\\*
No fields but are his own.
\end{verse}

\subsection{}

\blfootnote{Prof Wystan Auden (1907 -- 1973), \cite{audenb}. This is a line from Prof Auden's \refpoem{Secondary Epic}.}Hindsight as foresight makes no sense.

\section{}

\subsection{}

\blfootnote{Prof Wystan Auden (1907 -- 1973), \cite{audena}. These lines constitute the third section of Prof Auden's \refpoem{In Memory of W B Yeats}, omitting the first verse. They were written shortly after Yeats's death in 1939.}\settowidth{\versewidth}{Teach the free man how to praise.}
\begin{verse}[\versewidth]
Time that is intolerant\\*
Of the brave \& innocent,\\
And indifferent in a week\\*
To a beautiful physique,\\!

Worships language and forgives\\*
Everyone by whom it lives.;\\
Pardons cowardice, conceit,\\*
Lays its honours at their feet.\\!

Time that which this strange excuse\\*
Pardoned \textit{Kipling} \& his views,\\
And will pardon \textit{Paul Claudel},\\*
Pardons him for writing well.\\!

In the nightmare of the dark\\*
All the dogs of Europe bark,\\
And the living nations wait,\\*
Each sequestered in its hate;\\!

Intellectual disgrace\\*
Stares from every human face,\\
And the seas of pity lie\\*
Locked \& frozen in each eye.\\!

Follow, poet, follow right\\*
To the bottom of the night;\\
With your unconstraining voice\\*
Still persuade us to rejoice;\\!

With the farming of a verse\\*
Make a vineyard of the curse;\\
Sing of human unsuccess\\*
In a rapture of distress;\\!

In the deserts of the heart\\*
Let the healing fountain start;\\
In the prison of his days\\*
Teach the free man how to praise.
\end{verse}

\subsection{}

\blfootnote{Prof Wystan Auden (1907 -- 1973), \cite{audenb}. These lines constitute the first verse of \refpoem{Half Way}.}\settowidth{\versewidth}{And dismissed the greater part of your friends,}
\begin{verse}[\versewidth]
Having abdicated with comparative ease\\*
And dismissed the greater part of your friends,\\
Escaping by submarine\\
In a false beard, \sfrac{$1$}{$2$} hoping the ports were watched,\\
You have got here, and it isn't snowing:\\*
How shall we celebrate your arrival?
\end{verse}

\subsection{}

\blfootnote{Prof Wystan Auden (1907 -- 1973), \cite{audenb}. This is the title of one of Prof Auden's poems.}The truest poetry is the most feigning.

\section{}

\subsection{}

\blfootnote{`The Watershed', Prof Wystan Auden (1907 -- 1973), \cite{audenb}.}\settowidth{\versewidth}{But seldom this. Near you, taller than the grass,}
\begin{verse}[\versewidth]
Who stands, the crux left of the watershed,\\*
On the wet road between the chafing grass\\
Below him sees dismantled washing-floors,\\
Snatches of tramline running to a wood,\\
An industry already comatose,\\
Yet sparsely living. A ramshackle engine\\
At \textsc{Cashwell} raises water; for 10 years\\
It lay in flooded workings until this,\\
Its latter office, grudgingly performed.\\
And, further, here and there, though many dead\\
Lie under the poor soil, some acts are chosen,\\
Taken from recent winters; two there were\\
Cleaned out a damaged shaft by hand, clutching\\
The winch a gale would tear them from; one died\\
During a storm, the fells impassable,\\
Not at his village, but in wooden shape\\
Through long abandoned levels nosed his way\\*
And in his final valley went to ground.\\!

Go home, now, stranger, proud of your young stock,\\*
Stranger, turn back again, frustrate \& vexed:\\
This land, cut off, will not communicate,\\
Be no accessory content to one\\
Aimless for faces rather there than here.\\
Beams from your car may cross a bedroom wall,\\
They wake no sleeper; you may hear the wind\\
Arriving driven from the ignorant sea\\
To hurt itself on pane, on bark of elm\\
Where sap unbaffled rises, being spring;\\
But seldom this. Near you, taller than the grass,\\*
Ears poise before decision, scenting danger.
\end{verse}

\subsection{}

\blfootnote{`Who's Who', Prof Wystan Auden (1907 -- 1973), \cite{audenb}.}\settowidth{\versewidth}{    Though giddy, climbed new mountains; named a sea;}
\begin{verse}[\versewidth]
A shilling life will give you all the facts:\\*
\vin How father beat him, how he ran away,\\
What were the struggles of his youth, what acts\\
\vin Made him the greatest figure of his day;\\
Of how he fought, fished, hunted, worked all night,\\
\vin Though giddy, climbed new mountains; named a sea;\\
Some of the last researchers even write\\
\vin Love made him weep his pints like you \& me.\\
With all his honours on, he sighed for one\\
\vin Who, say astonished critics, lived at home;\\
\vin \vin Did little jobs about the house with skill\\
\vin \vin And nothing else; could whistle; would sit still\\
\vin Or potter round the garden; answered some\\*
Of his long marvellous letters but kept none.
\end{verse}

\subsection{}

\blfootnote{Prof Wystan Auden (1907 -- 1973), \cite{audenb}. This is a line from Prof Auden's \refpoem{The Truest Poetry is the Most Feigning}.}Good poets have a weakness for bad puns.

\section{}

\subsection{}

\blfootnote{`Venus Will Now Say a Few Words', Prof Wystan Auden (1907 -- 1973), \cite{audenb}.}\settowidth{\versewidth}{But joy is mine not yours -- to have come so far,}
\begin{verse}[\versewidth]
Since you are going to begin today\\*
Let us consider what it is you do.\\
You are the one whose part it is to lean,\\
For whom it is not good to be alone.\\
Laugh warmly turning shyly in the hall\\
Or climb with bare knees the volcanic hill,\\
Acquire that flick of wrist and after strain\\
Relax in your darling’s arms like a stone,\\
Remembering everything you can confess,\\
Making the most of firelight, of hours and fuss;\\
But joy is mine not yours -- to have come so far,\\
Whose cleverest invention was lately fur;\\
Lizards my best once who took years to breed,\\
Could not control the temperature of blood.\\
To reach that shape for your face to assume,\\
Pleasure to many and despair to some,\\
I shifted ranges, lived epochs handicapped\\
By climate, wars, or what the young men kept,\\
Modified theories on the types of dross,\\
Altered desire and history of dress.\\
You in the town now call the exile fool\\
That writes home once a year as last leaves fall,\\
Think -- romans had a language in their day\\
And ordered roads with it, but it had to die:\\
Your culture can but leave -- forgot as sure\\
As place-name origins in favorite shire --\\
Jottings for stories, some often-mentioned \textit{Jack},\\
And references in letters to a private joke,\\
Equipment rusting in unweeded lanes,\\
Virtues still advertised on local lines;\\
And your conviction shall help none to fly,\\*
Cause rather a perversion on next floor.\\!

Nor even in despair your own, when swiftly\\*
Comes general assault on your ideas of safety:\\
That sense of famine, central anguish felt\\
For goodness wasted at peripheral fault,\\
Your shutting up the house and taking prow\\
To go into the wilderness to pray,\\
Means that I wish to leave and to pass on,\\
Select another form, perhaps your son;\\
Though he reject you, join opposing team\\
Be late or early at another time,\\
My treatment will not differ -- he will be tipped,\\
Found weeping, signed for, made to answer, topped.\\
Do not imagine you can abdicate;\\
Before you reach the frontier you are caught;\\
Others have tried it and will try again\\
To finish that which they did not begin:\\
Their fate must always be the same as yours,\\
To suffer the loss they were afraid of, yes,\\*
Holders of one position, wrong for years.
\end{verse}

\subsection{}

\blfootnote{Prof Wystan Auden (1907 -- 1973), \cite{audenb}. This is the first verse of Prof Auden's \refpoem{Danse Macabre}.}\settowidth{\versewidth}{It's farewell to the drawing room's mannerly cry,}
\begin{verse}[\versewidth]
It's farewell to the drawing room's mannerly cry,\\*
The professor's logical whereto \& why,\\
The frock-coated diplomat's polished aplomb,\\*
Now matters are settled with gas \& with bomb.
\end{verse}

\subsection{}

\blfootnote{Prof Wystan Auden (1907 -- 1973), \cite{audenb}. These are three lines from Prof Auden's \refpoem{Horae Canonicae}.}You need not see what someone is doing to know if it is his vocation; you have only to watch his eyes.

\section{}

\subsection{}

\blfootnote{Prof Wystan Auden (1907 -- 1973), \cite{audenb}. This is the fourth and final section of Auden's \refpoem{1929}, written in the autumn of that year.}\settowidth{\versewidth}{The chairs are being brought in from the garden,}
\begin{verse}[\versewidth]
It is time for the destruction of error.\\*
The chairs are being brought in from the garden,\\
The summer talk stopped on that savage coast\\
Before the storms, after the guests \& birds:\\
In sanatoriums they laugh less \& less,\\
Less certain of cure; and the loud madman\\*
Sinks now into a more terrible calm.\\!

The falling leaves know it, the children,\\*
At play on the fuming alkali-tip\\
Or by the flooded football ground, know it --\\
This is the dragon's day, the devourer's:\\
Orders are given to the enemy for a time\\
With underground proliferation of mould,\\
With constant whisper \& the casual question,\\
To haunt the poisoned in his shunned house,\\
To destroy the efflorescence of the flesh,\\
To censor the play of the mind, to enforce\\*
Conformity with the orthodox bone,\\!

With organised fear, the articulated skeleton.\\*
You whom I gladly walk with, touch,\\
Or wait for as one certain of good,\\
We know it, we know that love\\
Needs more than the admiring excitement of union,\\
More than the abrupt self-confident farewell,\\
The heel on the finishing blade of grass,\\
The self-confidence of the falling root,\\
Needs death, death of the grain, our death.\\
Death of the old gang; would leave them\\
In sullen valley where is made no friend,\\
The old gang to be forgotten in the spring,\\
The hard bitch and the riding-master,\\
Stiff underground; deep in clear lake\\*
The lolling bridegroom, beautiful, there.
\end{verse}

\subsection{}

\blfootnote{Prof Wystan Auden (1907 -- 1973), \cite{audenb}. This is the first verse of the tenth sonnet in Prof Auden's sequence \refpoem{The Quest}. The rest of the poem, sadly, does not live up the promise of these marvellous first four lines.}\settowidth{\versewidth}{    A high percentage had an ugly face.}
\begin{verse}[\versewidth]
They noticed that virginity was needed\\*
\vin To trap the unicorn in every case,\\
But not that, of those virgins who succeeded,\\*
\vin A high percentage had an ugly face.
\end{verse}

\subsection{}

\blfootnote{Prof Wystan Auden (1907 -- 1973), \cite{audenb}. These are the closing words of Prof Auden's \refpoem{Good-Bye to the Mezzogiorno}.}\settowidth{\versewidth}{There is no forgetting that one was.}
\begin{verse}[\versewidth]
Though one cannot always\\*
Rmember exactly why one has been happy,\\*
There is no forgetting that one was.
\end{verse}

\section{}

\subsection{}

\blfootnote{Prof Wystan Auden (1907 -- 1973), \cite{audenb}. The first of Prof Auden's \refpoem{Five Songs}.}\settowidth{\versewidth}{Go through the motions of exploring the familiar;}
\begin{verse}[\versewidth]
What's in your mind, my dove, my coney?\\*
Do thoughts grow like feathers, the dead end of life?\\
Is it making of love or counting of money,\\*
Or raid on the jewels, the plans of a thief?\\!

Open your eyes, my dearest dallier;\\*
Let hunt with your hands for escaping me;\\
Go through the motions of exploring the familiar;\\*
Stand on the brink of the warm white day.\\!

Rise with the wind, my great big serpent;\\*
Silence the birds and darken the air;\\
Change me with terror, alive in a moment;\\*
Strike for the heart and have me there.
\end{verse}

\subsection{}

\blfootnote{Prof Wystan Auden (1907 -- 1973), \cite{audenb}.}\settowidth{\versewidth}{        Of what will come of this?}
\begin{verse}[\versewidth]
My second thoughts condemn\\*
\vin And wonder how I dare\\
\vin \vin To look you in the eye.\\
\vin What right have I to swear\\
Even at one AM\\*
\vin \vin To love you till I die?\\!

Earth meets too many crimes\\*
\vin For fibs to interest her;\\
\vin \vin If I can give my word,\\
\vin Forgiveness can recur\\
Any number of times\\*
\vin \vin In time. Which is absurd.\\!

{\hoskeroe Tempus fugit.} Quite.\\*
\vin So finish up your drink.\\
\vin \vin `All flesh is grass.' It is.\\
\vin But who on earth can think\\
With heavy heart or light\\*
\vin \vin Of what will come of this?
\end{verse}

\subsection{}

\blfootnote{Prof Wystan Auden (1907 -- 1973), \cite{audenb}. These words are a haiku from Prof Auden's \refpoem{Thanksgiving for a Habitat}.}Money cannot buy the fuel of love: but is excellent kindling.

\section{}

\subsection{}

\blfootnote{`Consider', Prof Wystan Auden (1907 -- 1973), \cite{audenb}.}\settowidth{\versewidth}{Those handsome and diseased youngsters, those women}
\begin{verse}[\versewidth]
Consider this and in our time\\*
As the hawk sees it or the helmeted airman:\\
The clouds rift suddenly -- look there\\
At cigarette-end smouldering on a border\\
At the first garden party of the year.\\
Pass on, admire the view of the massif\\
Through plate-glass windows of the \textsc{Sport Hotel};\\
Join there the insufficient units\\
Dangerous, easy, in furs, in uniform\\
And constellated at reserved tables\\
Supplied with feelings by an efficient band\\
Relayed elsewhere to farmers and their dogs\\*
Sitting in kitchens in the stormy fens.\\!

Long ago, supreme antagonist,\\*
More powerful than the great northern whale\\
Ancient and sorry at life's limiting defect,\\
In Cornwall, Mendip, or the Pennine moor\\
Your comments on the highborn mining-captains,\\
Found they no answer, made them wish to die\\
- Lie since in barrows out of harm.\\
You talk to your admirers every day\\
By silted harbours, derelict works,\\
In strangled orchard, and the silent comb\\
Where dogs have worried or a bird was shot.\\
Order the ill that they attack at once:\\
Visit the ports and, interrupting\\
The leisurely conversation in the bar\\
Within a stone's throw of the sunlit water,\\
Beckon your chosen out. Summon\\
Those handsome and diseased youngsters, those women\\
Your solitary agents in the country parishes;\\
And mobilise the powerful forces latent\\
In soils that make the farmer brutal\\
In the infected sinus, and the eyes of stoats.\\
Then, ready, start your rumour, soft\\
But horrifying in its capacity to disgust\\
Which, spreading magnified, shall come to be\\
A polar peril, a prodigious alarm,\\
Scattering the people, as torn up paper\\
Rags and utensils in a sudden gust,\\*
Seized with immeasurable neurotic dread.\\!

Seekers after happiness, all who follow\\*
The convolutions of your simple wish,\\
It is later than you think; nearer that day\\
Far other than that distant afternoon\\
Amid rustle of frocks and stamping feet\\
They gave the prizes to the ruined boys.\\
You cannot be away, then, no\\
Not though you pack to leave within an hour,\\
Escaping humming down arterial roads:\\
The date was yours; the prey to fugues,\\
Irregular breathing and alternate ascendancies\\
After some haunted migratory years\\
To disintegrate on an instant in the explosion of mania\\*
Or lapse forever into a classic fatigue.
\end{verse}

\subsection{}

\blfootnote{`The More Loving One', Prof Wystan Auden (1907 -- 1973), \cite{audenb}.}\settowidth{\versewidth}{Looking up at the stars, I know quite well}
\begin{verse}[\versewidth]
Looking up at the stars, I know quite well\\*
That, for all they care, I can go to hell,\\
But on earth indifference is the least\\*
We have to dread from man or beast.\\!

How should we like it were stars to burn\\*
With a passion for us we could not return?\\
If equal affection cannot be,\\*
Let the more loving one be me.\\!

Admirer as I think I am\\*
Of stars that do not give a damn,\\
I cannot, now I see them, say\\*
I missed one terribly all day.\\!

Were all stars to disappear or die,\\*
I should learn to look at an empty sky\\
And feel its total dark sublime,\\*
Though this might take me a little time.
\end{verse}

\subsection{}

\blfootnote{Prof Wystan Auden (1907 -- 1973), \cite{audenb}. This is a line from Prof Auden's \refpoem{At the Party}.}No one hears his own remarks as prose.

\section{}

\subsection{}

\blfootnote{`The Two', Prof Wystan Auden (1907 -- 1973), \cite{audenb}.}\settowidth{\versewidth}{Tell your stories of fishing other men's wives:}
\begin{verse}[\versewidth]
You are the town \& we are the clock.\\*
We are the guardians of the gate in the rock,\\
\vin The two.\\
On your left \& on your right,\\
In the day \& in the night,\\*
\vin We are watching you.\\!

Wiser not to ask just what has occurred\\*
To them who disobeyed our word;\\
\vin To those\\
We were the whirlpool, we were the reef,\\
We were the formal nightmare, grief\\*
\vin And the unlucky rose.\\!

Climb up the crane, learn the sailor's words\\*
When the ships from the islands laden with birds\\
\vin Come in.\\
Tell your stories of fishing \& other men's wives:\\
The expansive moments of constricted lives\\*
\vin In the lighted inn.\\!

But do not imagine we do not know\\*
Nor that what you hide with such care won't show\\
\vin At a glance.\\
Nothing is done, nothing is said,\\
But don't make the mistake of believing us dead:\\*
\vin I shouldn't dance.\\!

We're afraid in that case you'll have a fall.\\*
We've been watching you over the garden wall\\
\vin For hours.\\
The sky is darkening like a stain;\\
Something is going to fall like rain\\*
\vin And it won't be flowers.\\!

When the green field comes off like a lid\\*
Revealing what was much better hid:\\
\vin Unpleasant.\\
And look, behind you without a sound\\
The woods have come up and are standing round\\*
\vin In deadly crescent.\\!

The bolt is sliding in its groove;\\*
Outside the window is the black remov-\\
\vin -er's van.\\
And now with sudden swift emergence\\
Come the hooded women, humpbacked surgeons\\*
\vin And the scissor man.\\!

This might happen any day;\\*
So be careful what you say\\
\vin And do:\\
Be clean, be tidy, oil the lock,\\
Trim the garden, wind the clock;\\*
\vin Remember the two.
\end{verse}

\subsection{}

\blfootnote{`Elegy for JFK', Prof Wystan Auden (1907 -- 1973), \cite{audenb}.}\settowidth{\versewidth}{What he is fated to become}
\begin{verse}[\versewidth]
Why th\'{e}n, why th\'{e}re,\\*
Why th\'{u}s, we cry, did he die?\\*
The heavens are silent.\\!

What he was, he was:\\*
What he is fated to become\\*
Depends on us.\\!

Remembering his death,\\*
How we choose to live\\*
Will decide its meaning.\\!

When a just man dies,\\*
Lamentation \& praise,\\*
Sorrow \& joy, are one.
\end{verse}

\subsection{}

\blfootnote{Prof Wystan Auden (1907 -- 1973), \cite{audenb}. This is a verse from Prof Auden's \refpoem{The Art of Healing}, an elegy for a doctor.}To some, ill health is a way to be important, others are stoics, a few fanatics who won't feel happy until they are cut open.

\section{}

\subsection{}

\blfootnote{`Paysage Moralis\'e', Prof Wystan Auden (1907 -- 1973), \cite{audenb}. According to an article by one Harry Eyres, published in the \refbook{Financial Times} in 2012: `The art historian Erwin Panofsky coined the phrase paysage moralis\'e to describe the kind of renaissance painting in which aspects of landscape have moral significance. The example he took was Piero di Cosimo's \refbook{The Discovery of Honey by Bacchus}, ``where the antithesis between Virtue and Pleasure is symbolised by the contrast between an easy road winding through beautiful country and a steep stony path leading up to a forbidding rock''. Panofsky was apparently rather chuffed that his coinage provided W. H. Auden with the title for one of his most anthologised poems'.}\settowidth{\versewidth}{That brought them desperate to the brink of valleys;}
\begin{verse}[\versewidth]
Hearing of harvests rotting in the valleys,\\*
Seeing at end of street the barren mountains,\\
Round corners coming suddenly on water,\\
Knowing them shipwrecked who were launched for islands,\\
We honour founders of these starving cities\\*
Whose honour is the image of our sorrow,\\!

Which cannot see its likeness in their sorrow\\*
That brought them desperate to the brink of valleys;\\
Dreaming of evening walks through learned cities\\
They reined their violent horses on the mountains,\\
Those fields like ships to castaways on islands,\\*
Visions of green to them who craved for water.\\!

They built by rivers and at night the water\\*
Running past windows comforted their sorrow;\\
Each in his little bed conceived of islands\\
Where every day was dancing in the valleys\\
And all the green trees blossomed on the mountains\\*
Where love was innocent, being far from cities.\\!

But dawn came back and they were still in cities;\\*
No marvellous creature rose up from the water;\\
There was still gold \& silver in the mountains\\
But hunger was a more immediate sorrow,\\
Although to moping villagers in valleys\\*
Some waving pilgrims were describing islands...\\!

`The gods,' they promised, `visit us from islands,\\*
Are stalking, head-up, lovely, through our cities;\\
Now is the time to leave your wretched valleys\\
And sail with them across the lime-green water,\\
Sitting at their white sides, forget your sorrow,\\*
The shadow cast across your lives by mountains.'\\!

So many, doubtful, perished in the mountains,\\*
Climbing up crags to get a view of islands;\\
So many, fearful, took with them their sorrow\\
Which stayed them when they reached unhappy cities;\\
So many, careless, dived \& drowned in water;\\*
So many, wretched, would not leave their valleys.\\!

It is our sorrow. Shall it melt? Ah water\\*
Would gush, flush, green these mountains \& these valleys,\\*
And we rebuild our cities, not dream of islands.
\end{verse}

\subsection{}

\blfootnote{Prof Wystan Auden (1907 -- 1973), \cite{audenb}.}\settowidth{\versewidth}{God never makes knots,}
\begin{verse}[\versewidth]
God never makes knots,\\*
But is expert, if asked to,\\*
At untying them.
\end{verse}

\subsection{}

\blfootnote{Prof Wystan Auden (1907 -- 1973), \cite{audenb}. These words are taken from Prof Auden's \refpoem{Moon Landing}. The correctness of `worth while' as two separate words in this context is not universally acknowledged.}An adventure it would not have occurred to women to think worth while.

\section{}

\subsection{}

\blfootnote{$\mathbb{R}$ `The Watchers', Prof Wystan Auden (1907 -- 1973), \cite{audenb}.}\settowidth{\versewidth}{Here. You can see the marks. They lay in wait,'}
\begin{verse}[\versewidth]
Now from my window-sill I watch the night,\\*
The church clock's yellow face, the green pier light\\
Burn for a new imprudent year;\\
The silence buzzes in my ear;\\*
The lights of near-by families are out.\\!

Under the darkness nothing seems to stir;\\*
The lilac bush like a conspirator\\
Shams dead upon the lawn, and there\\
Above the flagstaff the great bear\\*
Hangs as a portent over \textsc{Helensburgh}.\\!

O lords of limit, training dark \& light\\*
And setting a taboo 'twixt left \& right,\\
The influential quiet twins\\
From whom all property begins,\\*
Look leniently upon us all to-night.\\!

No one has seen you: none can say, `Of late --\\*
Here. You can see the marks. They lay in wait,'\\
But in my thoughts to-night you seem\\
Forms which I saw once in a dream,\\*
The stocky keepers of a wild estate.\\!

With guns beneath your arms, in sun \& wet,\\*
At doorways posted or on ridges set,\\
By cope or bridge we know you there\\
Whose sleepless presences endear\\*
Our peace to us with a perpetual threat.\\!

Look not too closely, be not over-quick;\\*
We have no invitation, but we are sick,\\
Using the mole's device, the carriage\\
Of peacock or rat's desperate courage,\\*
And we shall only pass you by a trick.\\!

Deeper towards the summer the year moves on.\\*
What if the starving visionary have seen\\
The carnival within our gates,\\
Your bodies kicked about the streets,\\*
We need your power still: use it, that none,\\!

O, from their tables break uncontrollably away,\\*
Lunging, insensible to injury,\\
Dangerous in a room or out wild-\\
-ly spinning like a top in the field,\\*
Mopping \& mowing through the sleepless day.
\end{verse}

\subsection{}

\blfootnote{Prof Wystan Auden (1907 -- 1973), \cite{audenb}.}\settowidth{\versewidth}{As the poets have mournfully sung,}
\begin{verse}[\versewidth]
As the poets have mournfully sung,\\*
Death takes the innocent young,\\
\vin The rolling in money,\\
\vin The screamingly funny,\\*
And those who are very well hung.
\end{verse}

\subsection{}

\blfootnote{Prof Wystan Auden (1907 -- 1973), \cite{audenb}. These words are taken from one of Prof Auden's `Shorts'.}Both God and the Accuser speak very softly.

\section{}

\subsection{}

\blfootnote{Prof Wystan Auden (1907 -- 1973), \cite{audenb}.}\settowidth{\versewidth}{    As it always must come in the end;}
\begin{verse}[\versewidth]
At last the secret is out,\\*
\vin As it always must come in the end;\\
The delicious story is ripe\\
\vin To tell to the intimate friend;\\
Over the tea-cups \& in the square\\
\vin The tongue has its desire;\\
Still waters run deep, my dear;\\*
\vin There's never smoke without fire.\\!

Behind the corpse in the reservoir,\\*
\vin Behind the ghost on the links,\\
Behind the lady who dances and\\
\vin The man who madly drinks,\\
Under the look of fatigue,\\
\vin The attack of migraine \& the sigh\\
There is always another story;\\*
\vin There is more than meets the eye.\\!

For the clear voice suddenly singing,\\*
\vin High up in the convent wall,\\
The scent of the elder bushes,\\
\vin The sporting prints in the hall,\\
The croquet matches in summer,\\
\vin The handshake, the cough, the kiss,\\
There is always a wicked secret,\\*
\vin A private reason for this.
\end{verse}

\subsection{}

\blfootnote{Prof Wystan Auden (1907 -- 1973), \cite{audenb}.}\settowidth{\versewidth}{Like some valley cheese,}
\begin{verse}[\versewidth]
A poet's hope: to be,\\*
Like some valley cheese,\\*
Local, but prized elsewhere.
\end{verse}

\subsection{}

\blfootnote{Prof Wystan Auden (1907 -- 1973), \cite{audenb}. This is a line from one of Prof Auden's `Shorts'.}Friendship never ages.

\section{}

\subsection{}

\blfootnote{`As I Walked Out One Evening', Prof Wystan Auden (1907 -- 1973), \cite{audenb}. \P 2. The Bristol Street which Prof Auden had in mind is probably the one in Birmingham -- and the Birmingham in England, not the one in Alabama. \P 12. This is likely an allusion to Burns's famous love lyric: `Till a' the seas gang dry, my dear,/ And the rocks melt wi' the sun'.}\settowidth{\versewidth}{`Where the beggars raffle the banknotes}
\begin{verse}[\versewidth]
As I walked out one evening,\\*
\vin Walking down \textsc{Bristol Street},\\
The crowds upon the pavement\\*
\vin Were fields of harvest wheat.\\!

And down by the brimming river\\*
\vin I heard a lover sing\\
Under an arch of the railway:\\*
\vin `Love has no ending.\\!

`I'll love you, dear, I'll love you\\*
\vin Till China \& Africa meet,\\
And the river jumps over the mountain\\*
\vin And the salmon sing in the street;\\!

`I'll love you till the ocean\\*
\vin Is folded \& hung up to dry\\
And the seven stars go squawking\\*
\vin Like geese about the sky.\\!

`The years shall run like rabbits,\\*
\vin For in my arms I hold\\
The flower of the ages,\\*
\vin And the first love of the world.'\\!

But all the clocks in the city\\*
\vin Began to whirr \& chime:\\
`O let not time deceive you;\\*
\vin You cannot conquer time.\\!

`In the burrows of the nightmare\\*
\vin Where justice naked is,\\
Time watches from the shadow\\*
\vin And coughs when you would kiss.\\!

`In headaches \& in worry\\*
\vin Vaguely life leaks away,\\
And time will have his fancy\\*
\vin To-morrow or to-day.\\!

`Into many a green valley\\*
\vin Drifts the appalling snow;\\
Time breaks the threaded dances\\*
\vin And the diver's brilliant bow.\\!

`O plunge your hands in water,\\*
\vin Plunge them in up to the wrist;\\
Stare, stare in the basin\\*
\vin And wonder what you've missed.\\!

`The glacier knocks in the cupboard,\\*
\vin The desert sighs in the bed,\\
And the crack in the tea-cup opens\\*
\vin A lane to the land of the dead.\\!

`Where the beggars raffle the banknotes\\*
\vin And the giant is enchanting to \textit{Jack},\\
And the lily-white boy is a roarer,\\*
\vin And \textit{Jill} goes down on her back.\\!

`O look, look in the mirror,\\*
\vin O look in your distress:\\
Life remains a blessing\\*
\vin Although you cannot bless.\\!

`O stand, stand at the window\\*
\vin As the tears scald \& start;\\
You shall love your crooked neighbour\\*
\vin With your crooked heart.'\\!

It was late, late in the evening;\\*
\vin The lovers they were gone;\\
The clocks had ceased their chiming,\\*
\vin And the deep river ran on.
\end{verse}

\subsection{}

\blfootnote{Prof Wystan Auden (1907 -- 1973), \cite{audenb}.}\settowidth{\versewidth}{Give me a doctor, partridge-plump,}
\begin{verse}[\versewidth]
Give me a doctor, partridge-plump,\\*
Short in the leg \& broad in the rump,\\
An endomorph with gentle hands\\
Who'll never make absurd demands\\
That I abandon all my vices,\\
Nor pull a long face in a crisis,\\
But with a twinkle in his eye\\*
Will tell me that I have to die.
\end{verse}

\subsection{}

\blfootnote{Prof Wystan Auden (1907 -- 1973), \cite{audenb}. This is the last line in Prof Auden's \refpoem{Archaeology}, and thus the last line in his \refbook{Collected Poems}.}Goodness is timeless.

\section{}

\subsection{}

\blfootnote{Prof Wystan Auden (1907 -- 1973), \cite{audenb}.}\settowidth{\versewidth}{All that lives may love; why longer}
\begin{verse}[\versewidth]
Underneath an abject willow,\\*
\vin Lover, sulk no more:\\
Act from thought should quickly follow.\\
\vin What is thinking for?\\
Your unique and moping station\\
\vin Proves you cold;\\
\vin Stand up and fold\\*
Your map of desolation.\\!

Bells that toll across the meadows\\*
\vin From the sombre spire\\
Toll for these unloving shadows\\
\vin Love does not require.\\
All that lives may love; why longer\\
\vin Bow to loss\\
\vin With arms across?\\*
Strike and you shall conquer.\\!

Geese in flocks above you flying.\\*
\vin Their direction know,\\
Icy brooks beneath you flowing,\\
\vin To their ocean go.\\
Dark \& dull is your distraction:\\
\vin Walk then, come,\\
\vin No longer numb\\*
Into your satisfaction.
\end{verse}

\subsection{}

\blfootnote{Prof Wystan Auden (1907 -- 1973), \cite{audenb}.}\settowidth{\versewidth}{And in due season all men}
\begin{verse}[\versewidth]
River, sooner or later,\\*
All reach some ocean,\\
And in due season all men\\
Arrive at a death bed, but\\*
Neither on purpose.
\end{verse}

\subsection{}

\blfootnote{Prof Wystan Auden (1907 -- 1973), \cite{audenb}. This is a couplet from Prof Auden's \refpoem{In Memory of Sigmund Freud}.}\settowidth{\versewidth}{Sad is Eros, builder of cities,}
\begin{verse}[\versewidth]
Sad is \textit{Eros}, builder of cities,\\*
And weeping anarchic \textit{Aphrodite}.
\end{verse}

\section{}

\subsection{}

\blfootnote{`May', Prof Wystan Auden (1907 -- 1973), \cite{audenb}.}\settowidth{\versewidth}{And the white angel-vampires flit,}
\begin{verse}[\versewidth]
May with its light behaving\\*
Stirs vessel, eye \& limb,\\
The singular \& sad\\
Are willing to recover,\\
And to each swan-delighting river\\
The careless picnics come\\*
In living white \& red.\\!

Our dead, remote \& hooded,\\*
In hollows rest, but we\\
From their vague woods have broken,\\
Forests where children meet\\
And the white angel-vampires flit,\\
Stand now with shaded eye,\\*
The dangerous apple taken.\\!

The real world lies before us,\\*
Brave motions of the young,\\
Abundant wish for death,\\
The pleasing, pleasured, haunted:\\
A dying master sinks tormented\\
In his admirers' ring,\\*
The unjust walk the earth.\\!

And love that makes impatient\\*
Tortoise \& roe, that lays\\
The blonde beside the dark,\\
Urges upon our blood,\\
Before the evil \& the good\\
How insufficient is\\*
Touch, endearment, look.
\end{verse}

\subsection{}

\blfootnote{Prof Wystan Auden (1907 -- 1973), \cite{audenb}. This dedication to Christopher Isherwood and Chester Kallman appears at the beginning of at least two collections of Prof Auden's poetry.}\settowidth{\versewidth}{Although you be, as I am, one of those}
\begin{verse}[\versewidth]
Although you be, as I am, one of those\\*
Who feel a christian ought to write in prose,\\
For poetry is magic: born in sin, you\\*
May read them to exorcise the gentile in you.
\end{verse}

\subsection{}

\blfootnote{Prof Wystan Auden (1907 -- 1973), \cite{audenb}. This is a line from Prof Auden's \refpoem{Memorial for the City}.}History marched to the drums of a clear idea.

\section{}

\subsection{}

\blfootnote{Prof Wystan Auden (1907 -- 1973), \cite{audenb}. This is probably the strongest of Prof Auden's love lyrics. \P 17. Earlier versions give: `O but what worm of guilt'.}\settowidth{\versewidth}{            Did what I never wished,}
\begin{verse}[\versewidth]
Dear, though the night is gone,\\*
\vin Its dream still haunts today,\\
\vin \vin That brought us to a room\\
\vin \vin \vin Cavernous, lofty as\\
\vin \vin \vin A railway terminus,\\
\vin \vin And crowded in that gloom\\
Were beds, and we in one\\*
\vin In a far corner lay.\\!

Our whisper woke no clocks,\\*
\vin We kissed and I was glad\\
\vin \vin At everything you did,\\
\vin \vin \vin Indifferent to those\\
\vin \vin \vin Who sat with hostile eyes\\
\vin \vin In pairs on every bed,\\
Arms round each other's neck,\\*
\vin Inert \& vaguely sad.\\!

What hidden worm of guilt\\*
\vin Or what malignant doubt\\
\vin \vin Am I the victim of,\\
\vin \vin \vin That you then, unabashed,\\
\vin \vin \vin Did what I never wished,\\
\vin \vin Confessed another love;\\
And I, submissive, felt\\*
\vin Unwanted and went out?
\end{verse}

\subsection{}

\blfootnote{Prof Wystan Auden (1907 -- 1973), \cite{audenb}.}\settowidth{\versewidth}{And, gabbling off his rustic rhymes,}
\begin{verse}[\versewidth]
The emperor's favourite concubine\\*
\vin Was in the eunuch's pay.\\
The wardens of the marches turned\\
\vin Their spears the other way.\\
The vases crack; the ladies die;\\
\vin The oracles are wrong.\\
We suck our thumbs or sleep; the show\\*
\vin Is gamey \& too long.\\!

But -- music ho! -- at last it comes,\\*
\vin The transformation scene:\\
A rather scruffy-looking god\\
\vin Descends in a machine,\\
And, gabbling off his rustic rhymes,\\
\vin Misplacing one or two,\\
Commands the prisoners to walk,\\*
\vin The enemies to screw.
\end{verse}

\subsection{}

\blfootnote{Prof Wystan Auden (1907 -- 1973), \cite{audenb}. These are the last words of Prof Auden's \refpoem{Spain 1937}. He quoted them disapprovingly in the introduction to his \refbook{Collected Poems}.}\settowidth{\versewidth}{    History to the defeated}
\begin{verse}[\versewidth]
\vin History to the defeated\\*
May say alas but cannot help nor pardon.
\end{verse}

\section{}

\subsection{}

\blfootnote{`Casino', Prof Wystan Auden (1907 -- 1973), \cite{audenb}.}\settowidth{\versewidth}{As deeper in these hands is grooved their fortune: lucky}
\begin{verse}[\versewidth]
Only their hands are living, to the wheel attracted,\\*
Are moved, as deer trek desperately towards a creek\\
\vin Through the dust \& scrub of a desert, or gently,\\*
\vin \vin As sunflowers turn to the light,\\!

And, as night takes up the cries of feverish children,\\*
The cravings of lions in dens, the loves of dons,\\
\vin Gathers them all and remains the night, the\\*
\vin \vin Great room is full of their prayers.\\!

To a last feast of isolation self-invited,\\*
They flock, and in a rite of disbelief are joined;\\
\vin From numbers all their stars are recreated,\\*
\vin \vin The enchanted, the worldly, the sad.\\!

Without, calm rivers flow among the wholly living\\*
Quite near their trysts, and mountains part them, and birds,\\
\vin Deep in the greens \& moistures of summer,\\*
\vin \vin Sing towards their work.\\!

But here no nymph comes naked to the youngest shepherd;\\*
The fountain is deserted; the laurel will not grow;\\
\vin The labyrinth is safe but endless, and broken\\*
\vin \vin Is \textit{Ariadne}'s thread,\\!

As deeper in these hands is grooved their fortune: lucky\\*
Were few, and it is possible that none was loved,\\
\vin And what was god-like in this generation\\*
\vin \vin Was never to be born.
\end{verse}

\subsection{}

\blfootnote{Prof Wystan Auden (1907 -- 1973), \cite{audenb}.}\settowidth{\versewidth}{These fell asleep}
\begin{verse}[\versewidth]
Wishing no harm\\*
But to be warm,\\
These fell asleep\\*
On the burning heap.
\end{verse}

\subsection{}

\blfootnote{Prof Wystan Auden (1907 -- 1973), \cite{anothertime}. This is a line from Prof Auden's \refpoem{September 1. 1939}. He later amended the line to read, `We must love one another \emph{and} die', and later still omitted the poem altogether from collections of his poems.}We must love one another or die.

\section{}

\subsection{}

\blfootnote{`The Unknown Citizen', Prof Wystan Auden (1907 -- 1973), \cite{audenb}. Prof Auden's subtitle: `To JS/07/M/378 This Marble Monument Is Erected by the State'}\settowidth{\versewidth}{And our teachers report that he never interfered with their education.}
\begin{verse}[\versewidth]
He was found by the Bureau of Statistic to be\\*
One against whom there was no official complaint,\\
And all the reports on his conduct agree\\
That, in the modern sense of an old-fashioned word, he was a saint,\\
For in everything he did he served the Greater Community.\\
Except for the war till the day he retired\\
He worked in a factory and never got fired,\\
But satisfied his employers, Fudge Motors Inc.\\
Yet he wasn't a scab or odd in his views,\\
For his union reports that he paid his dues,\\
(Our report on his union shows it was sound)\\
And our Social Psychology workers found\\
That he was popular with his mates \& liked a drink.\\
The press are convinced that he bought a paper every day\\
And that his reactions to advertisements were normal in every way.\\
Policies taken out in his name prove that he was fully insured,\\
And his health card shows he was once in hospital but left it cured.\\
Both Producers Research and High Grade Living declare\\
He was fully sensible to the advantages of the Installment Plan\\
And had every thing necessary to the Modern Man,\\
A phonograph, a radio, a car \& a frigidaire.\\
Our research ers into Public Opinion are content\\
That he held the proper opinions for the time of year;\\
When there was peace, he was for peace; when there was war, he went.\\
He was married and added five children to the population,\\
Which our eugenist says was the right number for a parent of his generation,\\
And our teachers report that he never interfered with their education.\\
Was he free? Was he happy? The question is absurd:\\*
Had anything been wrong, we should certainly have heard.
\end{verse}

\subsection{}

\blfootnote{`Roman Wall Blues', Prof Wystan Auden (1907 -- 1973), \cite{audenb}. \P 6. The Tungrians were an ancient people within the Roman Empire, who inhabited an ill-defined region centred around the later settlement of Li\`ege.}\settowidth{\versewidth}{The mist creeps over the hard grey stone.}
\begin{verse}[\versewidth]
Over the heather the wet wind blows.\\*
I've lice in my tunic \& a cold in my nose.\\!

The rain comes pattering out of the sky,\\*
I'm a wall soldier. I don't know why.\\!

The mist creeps over the hard grey stone.\\*
My girl's in Tungria; I sleep alone.\\!

\textit{Aulus} goes hanging around her place.\\*
I don't like his manners; I don’t like his face.\\!

\textit{Piso}'s a christian; he worships a fish;\\*
There'd be no kissing if he had his wish.\\!

She gave me a ring but I diced it away;\\*
I want my girl and I want my pay.\\!

When I'm a veteran with only one eye\\*
I shall do nothing but look at the sky.
\end{verse}

\subsection{}

\blfootnote{Prof Wystan Auden (1907 -- 1973), \cite{anothertime}. This is a couplet from Prof Auden's \refpoem{September 1. 1939}.}\settowidth{\versewidth}{Those to whom evil is done}
\begin{verse}[\versewidth]
Those to whom evil is done\\*
Do evil in return.
\end{verse}

\section{}

\subsection{}

\blfootnote{`The Model', Prof Wystan Auden (1907 -- 1973), \cite{audenb}. Whether Prof Auden had a specific model and/or painting in mind is unclear.}\settowidth{\versewidth}{So the painter may please himself; give her an english park,}
\begin{verse}[\versewidth]
Generally, reading palms or handwriting or faces\\*
\vin Is a job of translation, since the kind\\
\vin \vin Gentleman often is\\
\vin A seducer, the frowning schoolgirl may\\
\vin \vin Be dying to be asked to stay;\\*
But the body of this old lady exactly indicates her mind;\\!

\textit{Rorschach} or \textit{Binet} could not add to what a fool can see\\*
\vin From the plain fact that she is alive \& well;\\
\vin \vin For when one is 80\\
\vin Even a teeny-weeny bit of greed\\
\vin \vin Makes one very ill indeed,\\*
And a touch of despair is instantaneously fatal:\\!

Whether the town once drank bubbly out of her shoes or whether\\*
\vin She was a governess with a good name\\
\vin \vin In church circles, if her\\
\vin Husband spoiled her or if she lost her son,\\
\vin \vin Is by this time all one.\\*
She survived whatever happened; she forgave; she became.\\!

So the painter may please himself; give her an english park,\\*
\vin Rice-fields in China, or a slum tenement;\\
\vin \vin Make the sky light or dark;\\
\vin Put green plush behind her or a red brick wall.\\
\vin \vin She will compose them all,\\*
Centering the eye on their essential human element.
\end{verse}

\subsection{}

\blfootnote{Prof Wystan Auden (1907 -- 1973), \cite{audenb}. These lines are a verse from Prof Auden's \refpoem{The Witnesses}.}\settowidth{\versewidth}{Look in your heart and see:}
\begin{verse}[\versewidth]
Look in your heart and see:\\*
\vin There lies the answer,\\
Though the heart like a clever\\
\vin Conjuror or dancer\\
Deceive you with many\\
\vin A curious sleight,\\
And motives like stowaways\\*
\vin Are found too late.
\end{verse}

\subsection{}

\blfootnote{Prof Wystan Auden (1907 -- 1973), \cite{audenb}. This is a line from the epilogue to Prof Auden's \refbook{The Age of Anxiety}.}We would rather be ruined than changed

%\backmatter

\renewcommand\thesection{{\arabic{section}}}
\renewcommand\thefootnote{{\arabic{section}}}

\part{Supplementary Material}

\chapter{Ecclesiastes}

\section*{I of the month}

The words of the Preacher, the son of David, king in Jerusalem.

Vanity of vanities, saith the Preacher, vanity of vanities; all is vanity.

What profit hath a man of all his labour which he taketh under the sun?

One generation passeth away, and another generation cometh: but the earth abideth for ever.

The sun also ariseth, and the sun goeth down, and hasteth to his place where he arose.

The wind goeth toward the south, and turneth about unto the north; it whirleth about continually, and the wind returneth again according to his circuits.

All the rivers run into the sea; yet the sea is not full; unto the place from whence the rivers come, thither they return again.

All things are full of labour; man cannot utter it: the eye is not satisfied with seeing, nor the ear filled with hearing.

The thing that hath been, it is that which shall be; and that which is done is that which shall be done: and there is no new thing under the sun.

Is there any thing whereof it may be said, See, this is new? it hath been already of old time, which was before us.

There is no remembrance of former things; neither shall there be any remembrance of things that are to come with those that shall come after.

\section*{II of the month}

I the Preacher was king over Israel in Jerusalem.

And I gave my heart to seek and search out by wisdom concerning all things that are done under heaven: this sore travail hath God given to the sons of man to be exercised therewith.

I have seen all the works that are done under the sun; and, behold, all is vanity and vexation of spirit.

That which is crooked cannot be made straight: and that which is wanting cannot be numbered.

\section*{III of the month}

I communed with mine own heart, saying, Lo, I am come to great estate, and have gotten more wisdom than all they that have been before me in Jerusalem: yea, my heart had great experience of wisdom and knowledge.

And I gave my heart to know wisdom, and to know madness and folly: I perceived that this also is vexation of spirit.

For in much wisdom is much grief: and he that increaseth knowledge increaseth sorrow.

\section*{IV of the month}

I said in mine heart, Go to now, I will prove thee with mirth, therefore enjoy pleasure: and, behold, this also is vanity.

I said of laughter, It is mad: and of mirth, What doeth it?

I sought in mine heart to give myself unto wine, yet acquainting mine heart with wisdom; and to lay hold on folly, till I might see what was that good for the sons of men, which they should do under the heaven all the days of their life.

I made me great works; I builded me houses; I planted me vineyards:

I made me gardens and orchards, and I planted trees in them of all kind of fruits:

I made me pools of water, to water therewith the wood that bringeth forth trees:

I got me servants and maidens, and had servants born in my house; also I had great possessions of great and small cattle above all that were in Jerusalem before me:

\ding{43} I gathered me also silver and gold, and the peculiar treasure of kings and of the provinces: I gat me men singers and women singers, and the delights of the sons of men, concubines very many.

So I was great, and increased more than all that were before me in Jerusalem: also my wisdom remained with me.

And whatsoever mine eyes desired I kept not from them, I withheld not my heart from any joy; for my heart rejoiced in all my labour: and this was my portion of all my labour.

Then I looked on all the works that my hands had wrought, and on the labour that I had laboured to do: and, behold, all was vanity and vexation of spirit, and there was no profit under the sun.

\section*{V of the month}

And I turned myself to behold wisdom, and madness, and folly: for what can the man do that cometh after the king? even that which hath been already done.

Then I saw that wisdom excelleth folly, as far as light excelleth darkness.

The wise man's eyes are in his head; but the fool walketh in darkness: and I myself perceived also that one event happeneth to them all.

Then said I in my heart, As it happeneth to the fool, so it happeneth even to me; and why was I then more wise? Then I said in my heart, that this also is vanity.

For there is no remembrance of the wise more than of the fool for ever; seeing that which now is in the days to come shall all be forgotten. And how dieth the wise man? as the fool.

Therefore I hated life; because the work that is wrought under the sun is grievous unto me: for all is vanity and vexation of spirit.

\section*{VI of the month}

\ding{43} I hated all my labour which I had taken under the sun: because I should leave it unto the man that shall be after me.

And who knoweth whether he shall be a wise man or a fool? yet shall he have rule over all my labour wherein I have laboured, and wherein I have shewed myself wise under the sun. This is also vanity.

Therefore I went about to cause my heart to despair of all the labour which I took under the sun.

For there is a man whose labour is in wisdom, and in knowledge, and in equity; yet to a man that hath not laboured therein shall he leave it for his portion. This also is vanity and a great evil.

For what hath man of all his labour, and of the vexation of his heart, wherein he hath laboured under the sun?

For all his days are sorrows, and his travail grief; yea, his heart taketh not rest in the night. This is also vanity.

\section*{VII of the month}

There is nothing better for a man, than that he should eat and drink, and that he should make his soul enjoy good in his labour. This also I saw, that it was from the hand of God.

For who can eat, or who else can hasten hereunto, more than I?

For God giveth to a man that is good in his sight wisdom, and knowledge, and joy: but to the sinner he giveth travail, to gather and to heap up, that he may give to him that is good before God. This also is vanity and vexation of spirit.

\section*{VIII of the month}

To every thing there is a season, and a time to every purpose under the heaven:

A time to be born, and a time to die; a time to plant, and a time to pluck up that which is planted;

A time to kill, and a time to heal; a time to break down, and a time to build up;

A time to weep, and a time to laugh; a time to mourn, and a time to dance;

A time to cast away stones, and a time to gather stones together; a time to embrace, and a time to refrain from embracing;

A time to get, and a time to lose; a time to keep, and a time to cast away;

A time to rend, and a time to sew; a time to keep silence, and a time to speak;

\ding{43} A time to love, and a time to hate; a time for war, and a time for peace.

\section*{IX of the month}

What profit hath he that worketh in that wherein he laboureth?

I have seen the travail, which God hath given to the sons of men to be exercised in it.

\ding{43} He hath made every thing beautiful in his time: also he hath set eternity in their heart, so that no man can find out the work that God maketh from the beginning to the end.

I know that there is no good in them, but for a man to rejoice, and to do good in his life.

And also that every man should eat and drink, and enjoy the good of all his labour, it is the gift of God.

I know that, whatsoever God doeth, it shall be for ever: nothing can be put to it, nor any thing taken from it: and God doeth it, that men should fear before him.

\ding{43} That which hath been is now; and that which is to be hath already been; and God seeketh again that which is passed away.

\section*{X of the month}

And moreover I saw under the sun the place of judgment, that wickedness was there; and the place of righteousness, that iniquity was there.

I said in mine heart, God shall judge the righteous and the wicked: for there is a time there for every purpose and for every work.

I said in mine heart concerning the estate of the sons of men, that God might manifest them, and that they might see that they themselves are beasts.

For that which befalleth the sons of men befalleth beasts; even one thing befalleth them: as the one dieth, so dieth the other; yea, they have all one breath; so that a man hath no preeminence above a beast: for all is vanity.

All go unto one place; all are of the dust, and all turn to dust again.

\ding{43} Who knoweth the spirit of man, whether it goeth upward, and the spirit of the beast, whether it goeth downward to the earth?

Wherefore I perceive that there is nothing better, than that a man should rejoice in his own works; for that is his portion: for who shall bring him to see what shall be after him?

\section*{XI of the month}

So I returned, and considered all the oppressions that are done under the sun: and behold the tears of such as were oppressed, and they had no comforter; and on the side of their oppressors there was power; but they had no comforter.

Wherefore I praised the dead which are already dead more than the living which are yet alive.

Yea, better is he than both they, which hath not yet been, who hath not seen the evil work that is done under the sun.

\section*{XII of the month}

Again, I considered all travail, and every right work, that for this a man is envied of his neighbour. This is also vanity and vexation of spirit.

The fool foldeth his hands together, and eateth his own flesh.

Better is an handful with quietness, than both the hands full with travail and vexation of spirit.

\section*{XIII of the month}

Then I returned, and I saw vanity under the sun.

There is one alone, and there is not a second; yea, he hath neither child nor brother: yet is there no end of all his labour; neither is his eye satisfied with riches; neither saith he, For whom do I labour, and bereave my soul of good? This is also vanity, yea, it is a sore travail.

\section*{XIV of the month}

Two are better than one; because they have a good reward for their labour.

For if they fall, the one will lift up his fellow: but woe to him that is alone when he falleth; for he hath not another to help him up.

Again, if two lie together, then they have heat: but how can one be warm alone?

And if one prevail against him, two shall withstand him; and a threefold cord is not quickly broken.

\section*{XV of the month}

Better is a poor and a wise child than an old and foolish king, who will no more be admonished.

For out of prison he cometh to reign; whereas also he that is born in his kingdom becometh poor.

I considered all the living which walk under the sun, with the second child that shall stand up in his stead.

There is no end of all the people, even of all that have been before them: they also that come after shall not rejoice in him. Surely this also is vanity and vexation of spirit.

\section*{XVI of the month}

Keep thy foot when thou goest to the house of God, and be more ready to hear, than to give the sacrifice of fools: for they consider not that they do evil.

Be not rash with thy mouth, and let not thine heart be hasty to utter any thing before God: for God is in heaven, and thou upon earth: therefore let thy words be few.

For a dream cometh through the multitude of business; and a fool's voice is known by multitude of words.

When thou vowest a vow unto God, defer not to pay it; for he hath no pleasure in fools: pay that which thou hast vowed.

Better is it that thou shouldest not vow, than that thou shouldest vow and not pay.

Suffer not thy mouth to cause thy flesh to sin; neither say thou before the angel, that it was an error: wherefore should God be angry at thy voice, and destroy the work of thine hands?

For in the multitude of dreams and many words there are also divers vanities: but fear thou God.

\section*{XVII of the month}

If thou seest the oppression of the poor, and violent perverting of judgment and justice in a province, marvel not at the matter: for he that is higher than the highest regardeth; and there be higher than they.

Moreover the profit of the earth is for all: the king himself is served by the field.

He that loveth silver shall not be satisfied with silver; nor he that loveth abundance with increase: this is also vanity.

When goods increase, they are increased that eat them: and what good is there to the owners thereof, saving the beholding of them with their eyes?

The sleep of a labouring man is sweet, whether he eat little or much: but the abundance of the rich will not suffer him to sleep.

There is a sore evil which I have seen under the sun, namely, riches kept for the owners thereof to their hurt.

But those riches perish by evil travail: and he begetteth a son, and there is nothing in his hand.

As he came forth of his mother's womb, naked shall he return to go as he came, and shall take nothing of his labour, which he may carry away in his hand.

And this also is a sore evil, that in all points as he came, so shall he go: and what profit hath he that hath laboured for the wind?

All his days also he eateth in darkness, and he hath much sorrow and wrath with his sickness.

Behold that which I have seen: it is good and comely for one to eat and to drink, and to enjoy the good of all his labour that he taketh under the sun all the days of his life, which God giveth him: for it is his portion.

Every man also to whom God hath given riches and wealth, and hath given him power to eat thereof, and to take his portion, and to rejoice in his labour; this is the gift of God.

For he shall not much remember the days of his life; because God answereth him in the joy of his heart.

\section*{XVIII of the month}

There is an evil which I have seen under the sun, and it is common among men:

A man to whom God hath given riches, wealth, and honour, so that he wanteth nothing for his soul of all that he desireth, yet God giveth him not power to eat thereof, but a stranger eateth it: this is vanity, and it is an evil disease.

If a man beget an hundred children, and live many years, so that the days of his years be many, and his soul be not filled with good, and also that he have no burial; I say, that an untimely birth is better than he.

For he cometh in with vanity, and departeth in darkness, and his name shall be covered with darkness.

Moreover he hath not seen the sun, nor known any thing: this hath more rest than the other.

Yea, though he live a thousand years twice told, yet hath he seen no good: do not all go to one place?

\section*{XIX of the month}

All the labour of man is for his mouth, and yet the appetite is not filled.

For what hath the wise more than the fool? what hath the poor, that knoweth to walk before the living?

Better is the sight of the eyes than the wandering of the desire: this is also vanity and vexation of spirit.

\ding{43} Whatsoever hath been, the name thereof was given long ago, and it is known what man is; neither may he contend with him that is mightier than he.

Seeing there be many things that increase vanity, what is man the better?

For who knoweth what is good for man in this life, all the days of his vain life which he spendeth as a shadow? for who can tell a man what shall be after him under the sun?

\section*{XX of the month}

A good name is better than precious ointment; and the day of death than the day of one's birth.

It is better to go to the house of mourning, than to go to the house of feasting: for that is the end of all men; and the living will lay it to his heart.

Sorrow is better than laughter: for by the sadness of the countenance the heart is made better.

The heart of the wise is in the house of mourning; but the heart of fools is in the house of mirth.

It is better to hear the rebuke of the wise, than for a man to hear the song of fools.

For as the crackling of thorns under a pot, so is the laughter of the fool: this also is vanity.

\ding{43} Surely oppression maketh a wise man mad; and a bribe destroyeth the heart.

Better is the end of a thing than the beginning thereof: and the patient in spirit is better than the proud in spirit.

Be not hasty in thy spirit to be angry: for anger resteth in the bosom of fools.

Say not thou, What is the cause that the former days were better than these? for thou dost not enquire wisely concerning this.

Wisdom is good with an inheritance: and by it there is profit to them that see the sun.

For wisdom is a defence, and money is a defence: but the excellency of knowledge is, that wisdom giveth life to them that have it.

Consider the work of God: for who can make that straight, which he hath made crooked?

In the day of prosperity be joyful, but in the day of adversity consider: God also hath set the one over against the other, to the end that man should find nothing after him.

\section*{XXI of the month}

All things have I seen in the days of my vanity: there is a just man that perisheth in his righteousness, and there is a wicked man that prolongeth his life in his wickedness.

Be not righteous over much; neither make thyself over wise: why shouldest thou destroy thyself?

Be not over much wicked, neither be thou foolish: why shouldest thou die before thy time?

It is good that thou shouldest take hold of this; yea, also from this withdraw not thine hand: for he that feareth God shall come forth of them all.

Wisdom strengtheneth the wise more than ten mighty men which are in the city.

For there is not a just man upon earth, that doeth good, and sinneth not.

Also take no heed unto all words that are spoken; lest thou hear thy servant curse thee:

For oftentimes also thine own heart knoweth that thou thyself likewise hast cursed others.

All this have I proved by wisdom: I said, I will be wise; but it was far from me.

That which is far off, and exceeding deep, who can find it out?

I applied mine heart to know, and to search, and to seek out wisdom, and the reason of things, and to know the wickedness of folly, even of foolishness and madness:

And I find more bitter than death the woman, whose heart is snares and nets, and her hands as bands: whoso pleaseth God shall escape from her; but the sinner shall be taken by her.

Behold, this have I found, saith the preacher, counting one by one, to find out the account:

Which yet my soul seeketh, but I find not: one man among a thousand have I found; but a woman among all those have I not found.

Lo, this only have I found, that God hath made man upright; but they have sought out many inventions.

\section*{XXII of the month}

Who is as the wise man? and who knoweth the interpretation of a thing? a man's wisdom maketh his face to shine, and the boldness of his face shall be changed.

I counsel thee to keep the king's commandment, and that in regard of the oath of God.

Be not hasty to go out of his sight: stand not in an evil thing; for he doeth whatsoever pleaseth him.

Where the word of a king is, there is power: and who may say unto him, What doest thou?

Whoso keepeth the commandment shall feel no evil thing: and a wise man's heart discerneth both time and judgment.

Because to every purpose there is time and judgment, therefore the misery of man is great upon him.

For he knoweth not that which shall be: for who can tell him when it shall be?

There is no man that hath power over the spirit to retain the spirit; neither hath he power in the day of death: and there is no discharge in that war; neither shall wickedness deliver those that are given to it.

All this have I seen, and applied my heart unto every work that is done under the sun: there is a time wherein one man ruleth over another to his own hurt.

\section*{XXIII of the month}

And so I saw the wicked buried, who had come and gone from the place of the holy, and they were forgotten in the city where they had so done: this is also vanity.

Because sentence against an evil work is not executed speedily, therefore the heart of the sons of men is fully set in them to do evil.

Though a sinner do evil an hundred times, and his days be prolonged, yet surely I know that it shall be well with them that fear God, which fear before him:

But it shall not be well with the wicked, neither shall he prolong his days, which are as a shadow; because he feareth not before God.

There is a vanity which is done upon the earth; that there be just men, unto whom it happeneth according to the work of the wicked; again, there be wicked men, to whom it happeneth according to the work of the righteous: I said that this also is vanity.

Then I commended mirth, because a man hath no better thing under the sun, than to eat, and to drink, and to be merry: for that shall abide with him of his labour the days of his life, which God giveth him under the sun.

When I applied mine heart to know wisdom, and to see the business that is done upon the earth: (for also there is that neither day nor night seeth sleep with his eyes:)

Then I beheld all the work of God, that a man cannot find out the work that is done under the sun: because though a man labour to seek it out, yet he shall not find it; yea farther; though a wise man think to know it, yet shall he not be able to find it.

\section*{XXIV of the month}

\ding{43} For all this I considered in my heart even to explore all this, that the righteous, and the wise, and their works, are in the hand of God: no man knoweth whether it be love or hatred; all is before them.

All things come alike to all: there is one event to the righteous, and to the wicked; to the good and to the clean, and to the unclean; to him that sacrificeth, and to him that sacrificeth not: as is the good, so is the sinner; and he that sweareth, as he that feareth an oath.

This is an evil among all things that are done under the sun, that there is one event unto all: yea, also the heart of the sons of men is full of evil, and madness is in their heart while they live, and after that they go to the dead.

For to him that is joined to all the living there is hope: for a living dog is better than a dead lion.

For the living know that they shall die: but the dead know not any thing, neither have they any more a reward; for the memory of them is forgotten.

\ding{43} Their love, and their hatred, and their envy, is now perished; neither have they any more a portion for ever in any thing that is done under the sun.

Go thy way, eat thy bread with joy, and drink thy wine with a merry heart; for God now accepteth thy works.

Let thy garments be always white; and let thy head lack no ointment.

Live joyfully with the wife whom thou lovest all the days of the life of thy vanity, which he hath given thee under the sun, all the days of thy vanity: for that is thy portion in this life, and in thy labour which thou takest under the sun.

Whatsoever thy hand findeth to do, do it with thy might; for there is no work, nor device, nor knowledge, nor wisdom, in the grave, whither thou goest.

I returned, and saw under the sun, that the race is not to the swift, nor the battle to the strong, neither yet bread to the wise, nor yet riches to men of understanding, nor yet favour to men of skill; but time and chance happeneth to them all.

For man also knoweth not his time: as the fishes that are taken in an evil net, and as the birds that are caught in the snare; so are the sons of men snared in an evil time, when it falleth suddenly upon them.

\section*{XXV of the month}

This wisdom have I seen also under the sun, and it seemed great unto me:

There was a little city, and few men within it; and there came a great king against it, and besieged it, and built great bulwarks against it:

Now there was found in it a poor wise man, and he by his wisdom delivered the city; yet no man remembered that same poor man.

Then said I, Wisdom is better than strength: nevertheless the poor man's wisdom is despised, and his words are not heard.

The words of wise men are heard in quiet more than the cry of him that ruleth among fools.

Wisdom is better than weapons of war: but one sinner destroyeth much good.

\section*{XXVI of the month}

Dead flies cause the ointment of the apothecary to send forth a stinking savour: so doth a little folly him that is in reputation for wisdom and honour.

A wise man's heart is at his right hand; but a fool's heart at his left.

Yea also, when he that is a fool walketh by the way, his wisdom faileth him, and he saith to every one that he is a fool.

If the spirit of the ruler rise up against thee, leave not thy place; for yielding pacifieth great offences.

There is an evil which I have seen under the sun, as an error which proceedeth from the ruler:

Folly is set in great dignity, and the rich sit in low place.

I have seen servants upon horses, and princes walking as servants upon the earth.

He that diggeth a pit shall fall into it; and whoso breaketh an hedge, a serpent shall bite him.

Whoso removeth stones shall be hurt therewith; and he that cleaveth wood shall be endangered thereby.

If the iron be blunt, and he do not whet the edge, then must he put to more strength: but wisdom is profitable to direct.

\ding{43} If the serpent bite before it is enchanted, then is there no advantage in the charmer.

The words of a wise man's mouth are gracious; but the lips of a fool will swallow up himself.

The beginning of the words of his mouth is foolishness: and the end of his talk is mischievous madness.

A fool also is full of words: a man cannot tell what shall be; and what shall be after him, who can tell him?

The labour of the foolish wearieth every one of them, because he knoweth not how to go to the city.

Woe to thee, O land, when thy king is a child, and thy princes eat in the morning!

Blessed art thou, O land, when thy king is the son of nobles, and thy princes eat in due season, for strength, and not for drunkenness!

By much slothfulness the building decayeth; and through idleness of the hands the house droppeth through.

A feast is made for laughter, and wine maketh merry: but money answereth all things.

Curse not the king, no not in thy thought; and curse not the rich in thy bedchamber: for a bird of the air shall carry the voice, and that which hath wings shall tell the matter.

\section*{XXVII of the month}

Cast thy bread upon the waters: for thou shalt find it after many days.

Give a portion to seven, and also to eight; for thou knowest not what evil shall be upon the earth.

If the clouds be full of rain, they empty themselves upon the earth: and if the tree fall toward the south, or toward the north, in the place where the tree falleth, there it shall be.

He that observeth the wind shall not sow; and he that regardeth the clouds shall not reap.

As thou knowest not what is the way of the spirit, nor how the bones do grow in the womb of her that is with child: even so thou knowest not the works of God who maketh all.

In the morning sow thy seed, and in the evening withhold not thine hand: for thou knowest not whether shall prosper, either this or that, or whether they both shall be alike good.

\section*{XXVIII of the month}

Truly the light is sweet, and a pleasant thing it is for the eyes to behold the sun:

But if a man live many years, and rejoice in them all; yet let him remember the days of darkness; for they shall be many. All that cometh is vanity.

Rejoice, O young man, in thy youth; and let thy heart cheer thee in the days of thy youth, and walk in the ways of thine heart, and in the sight of thine eyes: but know thou, that for all these things God will bring thee into judgment.

Therefore remove sorrow from thy heart, and put away evil from thy flesh: for childhood and youth are vanity.

\section*{XXIX of the month}

Remember now thy Creator in the days of thy youth, while the evil days come not, nor the years draw nigh, when thou shalt say, I have no pleasure in them;

While the sun, or the light, or the moon, or the stars, be not darkened, nor the clouds return after the rain:

In the day when the keepers of the house shall tremble, and the strong men shall bow themselves, and the grinders cease because they are few, and those that look out of the windows be darkened,

And the doors shall be shut in the streets, when the sound of the grinding is low, and he shall rise up at the voice of the bird, and all the daughters of musick shall be brought low;

\ding{43} Also when they shall be afraid of that which is high, and fears shall be in the way, and the almond tree shall flourish, and the grasshopper shall be a burden, and desire shall fail: because man goeth to his everlasting home, and the mourners go about the streets:

Or ever the silver cord be loosed, or the golden bowl be broken, or the pitcher be broken at the fountain, or the wheel broken at the cistern.

Then shall the dust return to the earth as it was: and the spirit shall return unto God who gave it.

Vanity of vanities, saith the preacher; all is vanity.

\section*{XXX of the month}

And moreover, because the preacher was wise, he still taught the people knowledge; yea, he gave good heed, and sought out, and set in order many proverbs.

The preacher sought to find out acceptable words: and that which was written was upright, even words of truth.

The words of the wise are as goads, and as nails fastened by the masters of assemblies, which are given from one shepherd.

And further, by these, my son, be admonished: of making many books there is no end; and much study is a weariness of the flesh.

Let us hear the conclusion of the whole matter: Fear God, and keep his commandments: for this is the whole duty of man.

For God shall bring every work into judgment, with every secret thing, whether it be good, or whether it be evil.

\chapter{The Song of Solomon}

\section*{First-day (I, VIII, XV and XXII of the month)}

The song of songs, which is Solomon's.

Let him kiss me with the kisses of his mouth: for thy love is better than wine.

Because of the savour of thy good ointments thy name is as ointment poured forth, therefore do the virgins love thee.

\ding{43} Draw me, we will run after thee: the king hath brought me into his chambers: we will be glad and rejoice in thee, we will remember thy love more than wine: rightly do they love thee.

I am black, but comely, O ye daughters of Jerusalem, as the tents of Kedar, as the curtains of Solomon.

Look not upon me, because I am black, because the sun hath looked upon me: my mother's children were angry with me; they made me the keeper of the vineyards; but mine own vineyard have I not kept.

Tell me, O thou whom my soul loveth, where thou feedest, where thou makest thy flock to rest at noon: for why should I be as one that turneth aside by the flocks of thy companions?

If thou know not, O thou fairest among women, go thy way forth by the footsteps of the flock, and feed thy kids beside the shepherds' tents.

I have compared thee, O my love, to a company of horses in Pharaoh's chariots.

Thy cheeks are comely with rows of jewels, thy neck with chains of gold.

We will make thee borders of gold with studs of silver.

While the king sitteth at his table, my spikenard sendeth forth the smell thereof.

A bundle of myrrh is my well-beloved unto me; he shall lie all night betwixt my breasts.

My beloved is unto me as a cluster of camphire in the vineyards of Engedi.

Behold, thou art fair, my love; behold, thou art fair; thou hast doves' eyes.

Behold, thou art fair, my beloved, yea, pleasant: also our bed is green.

The beams of our house are cedar, and our rafters of fir.

\section*{Second-day (II, IX, XVI and XXII of the month)}

I am the rose of Sharon, and the lily of the valleys.

As the lily among thorns, so is my love among the daughters.

As the apple tree among the trees of the wood, so is my beloved among the sons. I sat down under his shadow with great delight, and his fruit was sweet to my taste.

He brought me to the banqueting house, and his banner over me was love.

\ding{43} Stay me with flagons, comfort me with apples: for I am sick from love.

His left hand is under my head, and his right hand doth embrace me.

I charge you, O ye daughters of Jerusalem, by the roes, and by the hinds of the field, that ye stir not up, nor awake my love, till he please.

The voice of my beloved! behold, he cometh leaping upon the mountains, skipping upon the hills.

My beloved is like a roe or a young hart: behold, he standeth behind our wall, he looketh forth at the windows, shewing himself through the lattice.

My beloved spake, and said unto me, Rise up, my love, my fair one, and come away.

For, lo, the winter is past, the rain is over and gone;

\ding{43} The flowers appear on the earth; the time of the singing of birds is come, and the voice of the turtle dove is heard in our land;

The fig tree putteth forth her green figs, and the vines with the tender grape give a good smell. Arise, my love, my fair one, and come away.

O my dove, that art in the clefts of the rock, in the secret places of the stairs, let me see thy countenance, let me hear thy voice; for sweet is thy voice, and thy countenance is comely.

Take us the foxes, the little foxes, that spoil the vines: for our vines have tender grapes.

My beloved is mine, and I am his: he feedeth among the lilies.

Until the day break, and the shadows flee away, turn, my beloved, and be thou like a roe or a young hart upon the mountains of Bether.

\section*{Third-day (III, X, XVII and XXIV of the month)}

By night on my bed I sought him whom my soul loveth: I sought him, but I found him not.

I will rise now, and go about the city in the streets, and in the broad ways I will seek him whom my soul loveth: I sought him, but I found him not.

The watchmen that go about the city found me: to whom I said, Saw ye him whom my soul loveth?

It was but a little that I passed from them, but I found him whom my soul loveth: I held him, and would not let him go, until I had brought him into my mother's house, and into the chamber of her that conceived me.

I charge you, O ye daughters of Jerusalem, by the roes, and by the hinds of the field, that ye stir not up, nor awake my love, till he please.

Who is this that cometh out of the wilderness like pillars of smoke, perfumed with myrrh and frankincense, with all powders of the merchant?

Behold his bed, which is Solomon's; threescore valiant men are about it, of the valiant of Israel.

They all hold swords, being expert in war: every man hath his sword upon his thigh because of fear in the night.

King Solomon made himself a chariot of the wood of Lebanon.

He made the pillars thereof of silver, the bottom thereof of gold, the covering of it of purple, the midst thereof being paved with love, for the daughters of Jerusalem.

Go forth, O ye daughters of Zion, and behold king Solomon with the crown wherewith his mother crowned him in the day of his espousals, and in the day of the gladness of his heart.

\section*{Fourth-day (IV, XI, XVIII and XXV of the month)}

Behold, thou art fair, my love; behold, thou art fair; thou hast doves' eyes within thy locks: thy hair is as a flock of goats, that appear from mount Gilead.

Thy teeth are like a flock of sheep that are even shorn, which came up from the washing; whereof every one bear twins, and none is barren among them.

Thy lips are like a thread of scarlet, and thy speech is comely: thy temples are like a piece of a pomegranate within thy locks.

Thy neck is like the tower of David builded for an armoury, whereon there hang a thousand bucklers, all shields of mighty men.

Thy two breasts are like two young roes that are twins, which feed among the lilies.

Until the day break, and the shadows flee away, I will get me to the mountain of myrrh, and to the hill of frankincense.

Thou art all fair, my love; there is no spot in thee.

Come with me from Lebanon, my spouse, with me from Lebanon: look from the top of Amana, from the top of Shenir and Hermon, from the lions' dens, from the mountains of the leopards.

Thou hast ravished my heart, my sister, my spouse; thou hast ravished my heart with one of thine eyes, with one chain of thy neck.

How fair is thy love, my sister, my spouse! how much better is thy love than wine! and the smell of thine ointments than all spices!

Thy lips, O my spouse, drop as the honeycomb: honey and milk are under thy tongue; and the smell of thy garments is like the smell of Lebanon.

A garden inclosed is my sister, my spouse; a spring shut up, a fountain sealed.

Thy plants are an orchard of pomegranates, with pleasant fruits; camphire, with spikenard,

Spikenard and saffron; calamus and cinnamon, with all trees of frankincense; myrrh and aloes, with all the chief spices:

A fountain of gardens, a well of living waters, and streams from Lebanon.

Awake, O north wind; and come, thou south; blow upon my garden, that the spices thereof may flow out. Let my beloved come into his garden, and eat his pleasant fruits.

\section*{Fifth-day (V, XII, XIX and XXVI of the month)}

I am come into my garden, my sister, my spouse: I have gathered my myrrh with my spice; I have eaten my honeycomb with my honey; I have drunk my wine with my milk: eat, O friends; drink, yea, drink abundantly, O beloved.

I sleep, but my heart waketh: it is the voice of my beloved that knocketh, saying, Open to me, my sister, my love, my dove, my undefiled: for my head is filled with dew, and my locks with the drops of the night.

I have put off my coat; how shall I put it on? I have washed my feet; how shall I defile them?

\ding{43} My beloved put in his hand by the hole of the door, and my heart was moved for him.

I rose up to open to my beloved; and my hands dropped with myrrh, and my fingers with sweet smelling myrrh, upon the handles of the lock.

I opened to my beloved; but my beloved had withdrawn himself, and was gone: my soul failed when he spake: I sought him, but I could not find him; I called him, but he gave me no answer.

The watchmen that went about the city found me, they smote me, they wounded me; the keepers of the walls took away my veil from me.

\ding{43} I charge you, O daughters of Jerusalem, if ye find my beloved, that ye tell him, that I am sick from love.

What is thy beloved more than another beloved, O thou fairest among women? what is thy beloved more than another beloved, that thou dost so charge us?

My beloved is white and ruddy, the chiefest among ten thousand.

His head is as the most fine gold, his locks are bushy, and black as a raven.

His eyes are as the eyes of doves by the rivers of waters, washed with milk, and fitly set.

His cheeks are as a bed of spices, as sweet flowers: his lips like lilies, dropping sweet smelling myrrh.

His hands are as gold rings set with the beryl: his belly is as bright ivory overlaid with sapphires.

His legs are as pillars of marble, set upon sockets of fine gold: his countenance is as Lebanon, excellent as the cedars.

His mouth is most sweet: yea, he is altogether lovely. This is my beloved, and this is my friend, O daughters of Jerusalem.

\section*{Sixth-day (VI, XIII, XX and XXVII of the month)}

Whither is thy beloved gone, O thou fairest among women? whither is thy beloved turned aside? that we may seek him with thee.

My beloved is gone down into his garden, to the beds of spices, to feed in the gardens, and to gather lilies.

I am my beloved's, and my beloved is mine: he feedeth among the lilies.

Thou art beautiful, O my love, as Tirzah, comely as Jerusalem, terrible as an army with banners.

Turn away thine eyes from me, for they have overcome me: thy hair is as a flock of goats that appear from Gilead.

Thy teeth are as a flock of sheep which go up from the washing, whereof every one beareth twins, and there is not one barren among them.

As a piece of a pomegranate are thy temples within thy locks.

There are threescore queens, and fourscore concubines, and virgins without number.

My dove, my undefiled is but one; she is the only one of her mother, she is the choice one of her that bare her. The daughters saw her, and blessed her; yea, the queens and the concubines, and they praised her.

Who is she that looketh forth as the morning, fair as the moon, clear as the sun, and terrible as an army with banners?

I went down into the garden of nuts to see the fruits of the valley, and to see whether the vine flourished and the pomegranates budded.

Or ever I was aware, my soul made me like the chariots of Amminadib.

Return, return, O Shulamite; return, return, that we may look upon thee. What will ye see in the Shulamite? As it were the company of two armies.

\section*{Seventh-day (VII, XIV, XXI and XXVIII of the month)}

How beautiful are thy feet with shoes, O prince's daughter! the joints of thy thighs are like jewels, the work of the hands of a cunning workman.

Thy navel is like a round goblet, which wanteth not liquor: thy belly is like an heap of wheat set about with lilies.

Thy two breasts are like two young roes that are twins.

Thy neck is as a tower of ivory; thine eyes like the fishpools in Heshbon, by the gate of Bathrabbim: thy nose is as the tower of Lebanon which looketh toward Damascus.

Thine head upon thee is like Carmel, and the hair of thine head like purple; the king is held in the galleries.

How fair and how pleasant art thou, O love, for delights!

This thy stature is like to a palm tree, and thy breasts to clusters of grapes.

I said, I will go up to the palm tree, I will take hold of the boughs thereof: now also thy breasts shall be as clusters of the vine, and the smell of thy breath like apples;

And the roof of thy mouth like the best wine for my beloved, that goeth down sweetly, causing the lips of those that are asleep to speak.

I am my beloved's, and his desire is toward me.

Come, my beloved, let us go forth into the field; let us lodge in the villages.

Let us get up early to the vineyards; let us see if the vine flourish, whether the tender grape appear, and the pomegranates bud forth: there will I give thee my loves.

The mandrakes give a smell, and at our gates are all manner of pleasant fruits, new and old, which I have laid up for thee, O my beloved.

\section*{Eighth-day (XXIX of the month)}

O that thou wert as my brother, that sucked the breasts of my mother! when I should find thee without, I would kiss thee; yea, I should not be despised.

I would lead thee, and bring thee into my mother's house, who would instruct me: I would cause thee to drink of spiced wine of the juice of my pomegranate.

His left hand should be under my head, and his right hand should embrace me.

I charge you, O daughters of Jerusalem, that ye stir not up, nor awake my love, until he please.

Who is this that cometh up from the wilderness, leaning upon her beloved? I raised thee up under the apple tree: there thy mother brought thee forth: there she brought thee forth that bare thee.

Set me as a seal upon thine heart, as a seal upon thine arm: for love is strong as death; jealousy is cruel as the grave: the coals thereof are coals of fire, which hath a most vehement flame.

Many waters cannot quench love, neither can the floods drown it: if a man would give all the substance of his house for love, it would utterly be contemned.

We have a little sister, and she hath no breasts: what shall we do for our sister in the day when she shall be spoken for?

If she be a wall, we will build upon her a palace of silver: and if she be a door, we will inclose her with boards of cedar.

I am a wall, and my breasts like towers: then was I in his eyes as one that found favour.

Solomon had a vineyard at Baalhamon; he let out the vineyard unto keepers; every one for the fruit thereof was to bring a thousand pieces of silver.

My vineyard, which is mine, is before me: thou, O Solomon, must have a thousand, and those that keep the fruit thereof two hundred.

Thou that dwellest in the gardens, the companions hearken to thy voice: cause me to hear it.

Make haste, my beloved, and be thou like to a roe or to a young hart upon the mountains of spices.

\printbibliography[title={Sources}]

\bigskip

{\footnotesize \textsc{Note:} I only give here the details of those texts which provided material for the \textit{Almanack} proper. Details of other texts, such as those quoted in the Introduction, are to be found in the relevant footnotes.}

\end{document}
