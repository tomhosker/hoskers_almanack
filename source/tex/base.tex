\documentclass{amsbook}

\title{Hosker's Almanack (#VERSION_STRING)}

#PACKAGE_LOADOUT

\setcounter{secnumdepth}{4}
\setcounter{tocdepth}{0}

\begin{document}

\frontmatter

\maketitle

\settowidth{\versewidth}{In winter, when the fields are white,}
\begin{verse}[\versewidth]
    In winter, when the fields are white,\\*
    I sing this song for your delight.\\!

    In spring, when woods are getting green,\\*
    I'll try and tell you what I mean.\\!

    In summer, when the days are long,\\*
    Perhaps you'll understand the song.\\!

    In autumn, when the leaves are brown,\\*
    Take pen \& ink, and write it down.
\end{verse}

\tableofcontents

%FRONTMATTER  % Replace % with # to insert introductory prose, and vice versa.

\renewcommand\chaptername{Month}

\mainmatter

\renewcommand\thechapter{\Roman{chapter}}
\renewcommand\thesection{\arabic{section}}
\renewcommand\thesubsection{\thesection.\alph{subsection}}
\renewcommand\thefootnote{{\thesubsection}.}
\makeatletter
    \def\blfootnote{\xdef\@thefnmark{}\@footnotetext}
    \renewcommand{\@makefnmark}{\hbox{{{{\@thefnmark}}}}\hbox{{{{ }}}}}
    \renewcommand\@makefntext[1]{\hspace*{1em}{\@thefnmark} #1}
\makeatother

#MAINMATTER

%\backmatter

\renewcommand\thesection{\arabic{section}}
\renewcommand\thefootnote{\arabic{footnote}}
\makeatletter
    \renewcommand{\@makefnmark}{\textsuperscript{\@thefnmark}}
    \renewcommand\@makefntext[1]{\hspace*{1em}\textsuperscript{\@thefnmark}#1}
\makeatother

#BACKMATTER

\printbibliography[title={Sources}]

\end{document}
