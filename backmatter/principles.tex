\chapter*{Principles for Creating a Canon of English Poetry}

\section*{General}

\begin{enumerate}
    \item{Nothing in these principles shall exonerate any editor, or the manager, patron or controller thereof, of any barbarous, blockheaded, partisan, unimaginative, dogmatic or bigoted selections or omissions.}
    \item{In construing and complying with these principles, due regard shall be had to the majesty, wit, sweetness, subtlety and pity of the best of English poetry, which taken together are the spirit of any anthology worthy of the name, and which may make a departure from these principles necessary in some cases.}
\end{enumerate}

\section*{Definitions}

\begin{enumerate}[resume]
    \item{The \emph{canon} of English poetry is the set of those poems and proverbs most needful for all men to know.\footnote{Alfred the Great wrote to Bishop Waerferth in 890, when English literature was still in its infancy: `Therefore it seems well to me, if ye think so, for us also to translate the books \emph{most needful for all men to know} into the speech which all men know, and, as we are well able if we have peace, to make all the youth in England of free men rich enough to devote themselves to it, to learn while they are unfit for other occupation till they are well able to read English writing.'}}
    \item{A \emph{proverb} is no longer than two lines of poetry or a sentence of prose.}
    \item{A \emph{sonnet} is longer than a proverb, and is no longer sixteen lines of iambic pentameter, or the equivalent number of lines in another meter, or shorter.}
    \item{A \emph{song} longer than a sonnet, and is no longer than Milton's `Lycidas'.}
    \item{\textit{Old English} is the language spoken and written before the Battle of Hastings (14 Oct 1066), then \emph{Middle English} until the Battle of Bosworth Field (22 Aug 1485), \emph{Classical English} until the death of Ted Hughes (28 Oct 1998), and \emph{Late English} thereafter.\footnote{\emph{Archaic English} describes the union of Old and Middle English.}}
    \item{The \emph{golden age} of English poetry lasted from the Battle of Bosworth Field until the end of the First World War, and the \emph{silver age} from the end of the First World War until the death of Ted Hughes.}
    \item{A given author is \emph{of the golden age} if he was born during or after 1500,\footnote{The rule of thumb that Sir Thomas Wyatt -- born in 1503 -- was the first poet to write in modern English is basically sound. The Rev John Skelton, the preceding poet in the chronology of English literature, may have fired a few interesting shots in the direction of modernity, but he clearly belongs to the medieval world all the same.} and before 1900.\footnote{Given that the Armistice of 1918 marks the end of the golden age, and given that British soldiers were not eligible to serve overseas before the age of nineteen, a useful rule of thumb is that no poet born in 1900 or thereafter is to be included in the golden age.}}
    \item{For the purposes of the canon, the border ballads are English.\footnote{Without this appropriation, the body of Scots-Irish folks songs would make the English tradition look all too threadbare by contrast.}}
    \item{\emph{Imperial folk} describes all those folk songs composed in English in territories which were at any time part of the British Empire, or by subjects or citizens belonging to such territories.}
    \item{\emph{Urmonotheismus} is the belief in a supreme being which, according to the Vienna School of ethnology,\footnote{See Wilhelm Schmidt's \refbook{Der Ursprung der Gottesidee} (1912-1954) in particular.} was the starting point for all subsequent religious thought.}
\end{enumerate}

\section*{Structure}

\begin{enumerate}[resume]
    \item{The canon is to be limited in length by assigning poetry to each day of the year; hence the canon is also an almanac.}
    \item{This almanac is to follow the Cyprian calendar.}
    \item{A song, a sonnet and a proverb are to be selected for each day of the calendar year.}
    \item{In addition to the twelve ordinary months, i.e. the four groups of three months for each of the four seasons, it will be necessary, in some years, to add a thirteenth intercalary month in order to synchronise the calendar with the solar year.}
    \item{Only poems and proverbs written by authors of the golden age are to be assigned to the twelve ordinary months.}
    \item{Only poems and proverbs written by Prof Wystan Auden are to be assigned to the thirteenth intercalary month.}
    \item{The three spring months are to be assigned poems and proverbs written in a choleric mood; the three summer months, sanguine; the three autumn months, phlegmatic; and the three winter months, melancholic.}
    \item{The songs assigned to the three summer months are to be folk songs, and the sonnets assigned to the middle month of said three summer months are to be sea shanties.}
    \item{The folk songs assigned to the first of the three summer months are to be English folk; the second, Scots-Irish folk; and the third, imperial folk and hymns.\footnote{Aim for a ratio of two parts imperial folk to one part hymns.}}
    \item{The songs, sonnets and proverbs assigned to the middle month of the three autumn months are to be drawn from the King James Version of the Holy Bible and \refbook{The Book of Common Prayer} only.}
    \item{So far as the above restrictions will allow: the stronger poems and proverbs are to be gathered, in spring, towards the beginning of the season; in summer and autumn, to the middle; and in winter, to the end. Moreover, the weaker poems and proverbs are to be gathered, in spring and summer, towards the end of the season; and in autumn and winter, to the beginning.}
    \item{Wherever it is necessary to add supernumerary poems and proverbs, these are to be taken from the works of authors of the silver age.}
\end{enumerate}

\section*{Selection}

\begin{enumerate}[resume]
    \item{In choosing whether to include a given poem or proverb in the canon, the principle test shall be its memorability.\footnote{But bear in mind that this test is not infallible; irritating jingles can be difficult to forget, and masterpieces are sometimes overlooked on the first reading.}}
    \item{Creating shorter poems out of longer works is to be avoided.\footnote{To be avoided, but not to be excluded altogether: it would be invidious not to include Prince Hamlet's famous soliloquy, just as it would be imprudent to start cutting strips out of \refbook{Paradise Lost}. On the other hand, selecting items from a longer poem already divided into sections, such as Meredith's \refbook{Modern Love}, is entirely permitted.}}
    \item{No poem or proverb which expounds any specific religious doctrine, except for Urmonotheismus, is to be included in the canon.}
    \item{No quota, apportionment, proscription or suppression -- favouring or disfavouring any group defined by immutable characteristics -- shall be used in selecting poems and proverbs for inclusion in the canon.}
    \item{Nothing from either Ecclesiastes or the Song of Solomon shall be assigned to any given day of the lunisolar year.\footnote{Since these books are divided among the days of each month and the days of each week respectively.}}
\end{enumerate}

\section*{Orthography}

\begin{enumerate}[resume]
    \item{Spellings shall conform to an Enhanced Johnsonian Orthography (EnJO), which shall be based on the fourth edition Dr Johnson's famous dictionary,\footnote{That is, the edition of 1773.} and which shall correct the few inconsistencies, oversights, omissions and misunderstandings present in the base document. Said EnJO shall also include any neologisms used in subsequent poetry worthy of inclusion in the canon.}
    \item{No spelling shall be amended to conform to the EnJO such as to alter the pronunciation of the word in question; instead, a variant spelling shall be added to the EnJO.}
    \item{Wherever the spelling of a given poem has been altered in transmission from its recorded source to the canon, that poem shall be marked as redacted.}
    \item{The punctuation of any given poem may be amended without marking it as redacted, except where such an amendment would alter the meaning, or would change the sound of the poem when read aloud.}
    \item{Capitalisation shall follow the Italian use as a rule of thumb.}
\end{enumerate}
