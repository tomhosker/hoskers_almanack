Autumn's processes are very subtle and complex. They produce remarkable effects -- the brilliant colouring of leaves and fruit and the miracle of hibernation, which may be the almost complete suspension of life. One fundamental difference between this season and spring lies in there being no increase in external structure, nor reproduction. The changes -- ripening of fruit, fall of leaf, and hibernation -- are internal and, but for a flash of colour, invisible. Perhaps that is why so much attention is given to charting the progress of spring and summer, of which there is abundant superficial evidence, and little to autumn when practically all is hidden -- indeed, much life goes out of its way to hide -- and so rather mysterious.

Death, a quite universal and unescapable cause of change and, therefore, somewhat displeasing, also plays a great part in autumn. The first frosts kill a myriad insects and other small creatures. It is just too bad if man sweeps up their eggs into a bonfire.

The most spectacular sign of these changes is the colouring and then falling leaves of trees and shrubs. The first of these stages, for some reason known as ``the autumn tints'', reachs its gayest during October. At the joint of each leaf-stalk where it joins the twig a thin layer of is formed between the two. The leaf is gradually cut out of the circulatory system of the tree. Its chemical composition undergoes great changes, and this results in bright colours replacing the former green. Finally it becomes so loosely attached to the twig that it falls at a slight touch. On the ground, woven into a carpet of other leaves, it gradually decays and returns some of the energy that has been employed in its creation back to the earth. The scar on the twig has been healed against the entry of fungus or other enemy.

Changes also occur within the ripening fruit. From our and some birds' and animals' point of view, the development of sugars in some kinds which make them palatable is not the least important. It also serves what to us is a secondaru end -- the distribution of seeds by those who consume the surrounding pulpy mass.

A good number of late-summer plants still flower, particularly annual weeds of cultivated ground. A blaze of dahlias and chrysanthemums remains in the garden until it is blackened by frost. Some of our introduced conifers, such as the cedars, will be opening their catkin-like male flowers and scattering pollen during this month. The arbutus or strawberry-tree, which is believed to have survived as a native of these islands in Ireland during an age when much else perished, bears panicles of little white or pinkish urn-like flowers which turn into strawberries.

But most vegetable vigour is seen in the great army of fungi which throw up their spore-bearing devices -- toadstools, shaggy-caps, puff-balls, bracket fungi, and moulds and mildews -- in great abundance throughout late summer. It is a numerous and powerful army. In this country it includes some seven thousand species against fifteen hundred of ordinary seed-bearing plants.

Once again migration on a big scale excites bird-watchers. Not only do we greet more winter visitors, but see many birds of passage. It is not, perhaps, realised how many og our common and resident kinds are also migrants.

Swallows, house- and sand-martins, as well as the last of the wheatear, leave us early in the month. Fieldfare, redwing, brambling and wigeon are the principal arrivals -- with, of course, geese as the great excitement. It is a month when we may glimpse passing rarities; even osprey have been reported in Surrey. It is also a time when the little birds are flocking. Blackbirds and robins take up their territory.

For many animals it is the month when hibernation begins. In its varying degrees from partial to almost complete lifelessness it is a form of wintering adopted by creatures ranging from butterflies to badgers. Sometimes insect hibernators will be found in colonies -- peacock butterflies and queen wasps, or ladybirds, perhaps in some warm space between boarding; at others they will hide singly in cracks between all kinds of odd material.

Bats also hibernate, haning head downwards, in buildings, caves or hollow trees; sometimes they too are found singly in crevices. Dormice roll up in nests, underground or beneath roots and shrubs, with a good supply of food in case warm spells wake them up. The adder joins with one or two friends in sharing a hollow among feathers or dry grass. Grass snakes collect together in large numbers beneath old roots, under piles of brushwood, or in dry holes. Blindworms burrow with their heads into loose dry soil, or similar material; they do this very early in autumn. Lizards also dig themselves in, often quite deeply, joining together in small colonies. Frogs hibernate in many odd places -- holes, under piles of leaves, hayricks, and so on. Toads become comatose only, and prefer dry holes, well away from water. Newts go underground; numbers are often found twined together in a mass.

One other October occurrence is the rut, or mating season of deer. The stags are savage, and will attack people and dogs. They fight for possession of the hinds. With red deer the season usually starts in September (of the Gregorian calendar) and lasts through October (of the same); fallow deer have a shorter period starting rather later. It is the only time when male deer make any noise.
