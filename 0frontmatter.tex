\documentclass[0main.tex]{subfiles}
\begin{document}

\part*{Introduction}

\chapter*{General Introduction}
\renewcommand*{\theHsection}{ch1.\the\value{section}}

I'm conscious of how strange this book must seem to anyone other than its author. So, to sum it up in a few words: my intention was fix and preserve the canon of English poetry in the same way that the canon of Ancient Hebrew was fixed and preserved so magisterially by the Old Testament.

The comparison with the Old Testament is both instructive and problematic. For, on the one hand, a kind of English scripture is precisely what I'm trying to achieve; but, on the other, so much of the source-material for this project is itself translation from the Hebrew (and Greek) sacred texts.

Where would such a canon be closed? And how would it be prevented from growing to such a size that the whole project became self-defeating? The full answer to these questions comes in many parts, all of which are to be given in -- no doubt painful -- detail in the following pages. But the short answer is this: two poems, a short one and a long one, are to be given for each day of the year. Thus this book is less of an anthology and more of an \emph{Almanack}.

As always, the reader will be the judge. And the proof of the pudding is in the eating.

\bigskip

\begin{flushright}
{\sc tom hosker} \\
\emph{The Almanackist} \\
Tickhill, MMXVI
\end{flushright}

\chapter*{A History of the English Language and Its Poetry}
\renewcommand*{\theHsection}{ch2.\the\value{section}}

This chapter, and, indeed, the entire \emph{Almanack}, is built upon this principle:

\begin{figure}[h]
\caption{Johnsonian Supremacy}
\begin{quote}
Dr Johnson's \emph{Dictionary of the English Language} is authoritative; which is to say, Johnson is always to be treated as being correct, except in those cases where it can be demonstrated that he has made an error of fact.
\end{quote}
\end{figure}

As a consequence of this first Rule, this chapter shall draw heavily from Johnson's `History of the English Language', which is to be found within the introductory pages of the aforementioned \emph{Dictionary}.

\bigskip

\section{An Early History of the English Language}

{\footnotesize Though the \emph{Britains} or \emph{Welsh} were the first possessors of this island, whose names are recorded, and are therefore in civil history always considered as the predecessors of the present inhabitants; yet the deduction of the \emph{English} language, from the earliest times of which we have any knowledge to its present state, requires no mention of them: for we have so few words which can, with any probability, be referred to \emph{British} roots, that we justly regard the \emph{Saxon} and \emph{Welsh} as nations totally distinct. It has been conjectured, that when the \emph{Saxons} seized this country, they suffered the \emph{Britains} to live among them in a state of vassalage, employed in the culture of the ground, and other laborious and ignoble services. But it is scarcely possible, that a nation, however depressed, should have been mixed with another in considerable numbers without some communication of their tongue, and therefore, it may, with great reason, be imagined, that those, who were not sheltered in the mountains, perished by the sword.\footnote{`The History of the English Language'. \emph{A Dictionary of the English Language}.}}

\bigskip

So begins Johnson's essay; and, although his supposition that the ancient Britons were exterminated by the Anglo-Saxons would seem to have been disproved by modern genetics,\footnote{A certain Dr Oppenheimer has penned a number of works arguing that the bulk of the British genome comes from neither the Anglo-Saxons nor the Celts, but from a group of very ancient settlers, many centuries before recorded history. Alas, the Almanackist is not sufficiently knowledgeable to judge their veracity.} nonetheless it remains that, of all the invasions of Great Britain, the Anglo-Saxon conquest has had by far the most profound effect on the language her inhabitants. The language of the Anglo-Saxons provided the canvas onto which all the later developments were painted; thus the same language is appropriately referred to as ``Old English''.

Our knowledge of the Anglo-Saxons prior to their invasion of Great Britain is frustratingly patchy. They seem to have come to our island from what is now known as Friesland, itself divided between the modern nation-states of the Netherlands, Germany and Denmark. And this hypothesis is supported by the remarkable similarities between modern English and modern Frisian; Frisian \emph{de} corresponds to English \emph{the}, \emph{wyt} to \emph{white}, \emph{ear} to \emph{ear}, etc. We know that, during their stay in Friesland, the Saxons occupied themselves with making seaborne raids on Roman Britain, so much so that the Emperor had to create a ``Comes Litoris Saxonici''.\footnote{That is, Count of the Saxon Shore.} Prior to this, the Saxons are difficult to distinguish from other Germanic tribes, who themselves emerged during the Iron Age from the Proto-Indo-Europeans, whose language is the ultimate source of all European languages.\footnote{Basque and Hungarian are not descended from this Proto-Indo-European language, but these are exceptions to a rule which otherwise holds remarkably well.}

Old English had a similar vocabulary to its present-day counterpart, but its grammar would seem strange to a modern Englishman; the language was highly inflected, with the endings of nouns changing in a similar way to Latin or, indeed, modern German. This tendency towards inflection, though, was shed early on -- it was more or less gone by the time of the Norman conquest -- and it could be conjectured that this shedding was due to speakers of mutually unintelligible languages simplifying their speech in order to be understood:\footnote{And, conversely, the grammatical straightforwardness of English must have been an important factor in its becoming the first global lingua franca.} Dark Age England was a melting pot of Anglo-Saxons, Romans, Britons and Danes.

Not only would Old English \emph{sound} strange to a modern Englishman, in its earliest forms, it would \emph{look} strange too. For the language was originally written in runes, quite unlike the letters of this document, which adorn the many stone crosses the ancient English left to us as an heirloom.\footnote{Alas, there are many fewer crosses left than there might have been; for, in what must constitute the most outrageous example of Protestant hubris in world history, most of these beautiful monuments were deliberately demolished in the seventeenth century.} It was only after the arrival of Augustine at Canterbury in 597 -- the first serious effort by the Roman Church to convert the Anglo-Saxons -- that Christian monks, more used to writing in Latin, attempted to write English using Roman characters. Hence the thousand year nightmare of hammering English spelling into some kind of unity; the language is written using an alphabet which was never intended for that purpose.\footnote{Italian schoolchildren, the Almanackist believes, have a much easier time learning to spell.} The unmitigated fiasco of English spelling aside, Augustine's arrival ushered in another, more encouraging, development: the commingling of the priests with the poets. Henceforth, any attempt to disentangle the history of the English Church from the history of the English language is a fool's errand -- but more on that theme later.

The Battle of Hastings marks an obvious turning point in English history, but, as Dr Johnson points out, its immediate impact on the English language was surprisingly slight; it's only by the 1150s that the surviving texts become noticeably closer to French. In the following centuries, the influence of Old French over Old English grew more and more. There will doubtless always be debate over whether poets such as William Langland,\footnote{The name William Langland is used here to refer to the author of \emph{Piers Ploughman}, whoever he might have been.} John Gower and Geoffrey Chaucer (d. 1400) were the cause or the effect of the last great wave of Frenchification, but what is plain for all to see is that the works of these poets are significantly more Frenchified than the works of their predecessors; thus the language in which they wrote is given its own name, ``Middle English''. Once, to coin a phrase, the graves of these Middle English poets were filled in, the language transformed into a recognisably ``Modern'' form.

\bigskip

\section{An Early History of English Poetry}

Before we proceed any further, it would perhaps be helpful at this point to set out another Rule:\footnote{Here the Almanackist is deeply indebted to the German philologists of the nineteenth century, especially Wilhelm Teuffel's \emph{Geschichte der R\"omischen Literatur}.}

\begin{figure}[h]
\caption{English-Latin Analogy}
\begin{quote}
The history of English poetry is analogous to the history of Latin poetry, inasmuch as both consist of three phases: Early, Classical and Late.
\end{quote}
\end{figure}

Thus we might construct the following table:

\begin{table}[h]
\centering
\caption*{{\sc Table.} The Phases of English: Its Language \& Poetry}
\begin{tabular}{c|c}
Language & Poetry \\ \hline \hline
Old & \multirow{2}{*}{Early} \\
Middle & \\ \hline
Modern & Classical \\ \hline
Late Modern & Late \\
\end{tabular}
\end{table}

In the previous section, a history of Old and Middle English was given. In this section, a history of Early English Poetry is provided, and parallels drawn with the history of Latin literature as appropriate.

The Almanackist has already mentioned that the histories of English poetry and English Christianity ought not to be pulled apart; and, indeed, it is commonly accepted\footnote{Since Johnson is silent on this matter, the Almanackist can do no better than follow the judgement of the \emph{Norton Anthology}.} that the earliest English poem is a hymn, composed by an illiterate seventh century farmhand called Caedmon. About a century later,\footnote{Dating \emph{Beowulf} is a tortuous business.} \emph{Beowulf} came into being, along with a handful of short poems: riddles, accounts of battles, `The Seafarer' -- not forgetting an intriguing praise poem for the city of Durham. After this first harvest, very little poetry was produced about which Dr Johnson has anything kind to say until the time of Chaucer.

Now the poets of Chaucer's school were of the opinion -- rightly or wrongly -- that there was something lacking in the literature of their mother tongue, and looked to the continent for a model for improving it. Indeed, Chaucer himself spent a number of years in Italy and France, and his works show a marked and deliberate borrowing from the traditions of those nations. The Almanackist must confess that he himself has no affection for the poetry of the Late Middle Ages, and hence little to say about it; but no man can deny its importance in the history of our literature.

The change which English literature experienced in the fourteenth century after Christ is strikingly similar to the transformation which Latin literature underwent in the second century before; with the treasures of Classical Greek serving as a model for the early Roman poets in the same way that the treasures of Old French and Italian inspired the Chaucerian school. Indeed, Ennius, the father of Latin literature, is reputed to have considered himself a reincarnation of Homer; and, certainly, he made a conscious effort to emulate the {\greektext >Ili'as}\footnote{I.e. the \emph{Iliad}.} in composing his \emph{Annales}. A century later, Horace would sum up this metamorphosis in a famous couplet:\\

\settowidth{\versewidth}{\footnotesize Graecia capta ferum victorem cepit, et artes}
\begin{verse}[\versewidth]
{\footnotesize
\vin Graecia capta ferum victorem cepit, et artes\\
intulit agresti Latio...\footnote{That is, `Captive Greece took her captor captive, and brought the arts to rustic Italy...'. Horace, \emph{Epistulae} II.1.156-157.}\\
}
\end{verse}

\bigskip

Having discussed Early English poetry and drawn the appropriate parallels with Latin, allow the Almanackist now to do the same for the next phase of our literature.

\bigskip

\section{British; or, Good-English}

\subsection{The Golden Age of English Literature}

The poetry of Chaucer and his peers would be somewhat intelligible to the average Englishman at the beginning of the twenty-first century -- but only somewhat -- perhaps just a little more than modern Dutch. Even the poetry of John Skelton (d. 1529) would strike him as essentially foreign. Consider, for instance, these lines:

\bigskip

\settowidth{\versewidth}{\footnotesize Though ye suppose all jeperdys ar paste}
\begin{verse}[\versewidth]
{\footnotesize
Though ye suppose all jeperdys ar paste,\\
\vin And all is done that ye lokyd for before,\\
Ware yet, I rede you, of Fortunes dowble cast,\\
\vin For one fals poynt she is wont to kepe in store,\\
\vin And vnder the fell oft festered is the sore:\\
That when ye thynke all daunger for to pas,\\
Ware of the lesard lyeth lurkyng in the gras.\footnote{From a poem which begins: `\emph{Cuncta licet cecidisse putas discrimina rerum}...'. \emph{The Poetical Works of John Skelton}, Ed. Rev. Alexander Dryce.}\\
}
\end{verse}

\bigskip

Some words aren't all that hard to decode -- `ware' presumably translates as \emph{beware} -- but notice the unfamiliar `rede'. And what on earth is a `lesard'?

With Sir Thomas Wyatt (1503 -- 1542), on the other hand, we encounter the first instance of a poetry which is unmistakably \emph{ours}:

\bigskip

\settowidth{\versewidth}{\footnotesize They flee from me, that sometime did me seek,}
\begin{verse}[\versewidth]
{\footnotesize
They flee from me, that sometime did me seek,\\
\vin With naked foot stalking within my chamber:\\
Once have I seen them gentle, tame, and meek,\\
\vin That now are wild, and do not once remember,\\
\vin That sometime they have put themselves in danger\\
To take bread at my hand; and now they range\\
Busily seeking in continual change.\footnote{From `They Flee from Me'. \emph{The Poetical Works of Sir Thomas Wyatt}, Ed. Sir Nicholas Nicolas.}\\
}
\end{verse}

\bigskip

Now bear in mind that Skelton's and Wyatt's lifetimes actually overlapped by over a quarter-century. How are we to account for such a dramatic change? We might look to a variety of factors, including:

\bigskip

\begin{itemize}
\item{The introduction of the printing press to England by William Caxton;}
\item{The nascent Protestant Reformation;}
\item{The conclusion of both the Hundred Years War and of the Wars of the Roses; and,}
\item{The ongoing growth of national consciousness in Europe generally.}
\end{itemize}

\bigskip

Now all these causes were either brought about or embodied by the so-called Tudor Revolution, which itself began on a specific day: 22nd August 1485, in the wake of the Battle of Bosworth Field. Thus the Almanckist defines the Classical period of English poetry as beginning on this day. And, moreover, the Almanackist affixes to the particular subspecies of Modern English in which this poetry was written a name of his own making: ``Good-English''.\footnote{The Almanackist derives his inspiration here from one of the archaic names of steel: \emph{good-iron}. Note that the stress ought to be on ``Good'' in the same way that the stress of \emph{blackboard} is on ``black''.}

We might also refer to this Good-English literary language as ``British''. For, although said language began as an unambiguously English literary movement, it was to absorb the talents and dialects of Wales, Scotland and Ireland in exactly the same way as those territories were to be incorporated into a United Kingdom via the unions of 1535, 1707 and 1801. Thus the Almanackist regards the terms ``Good-English'' and ``British'' as being equivalent when referring to the literary language, and he shall make every effort to use them interchangeably.

British came into being under the Tudors, but it reached its apogee under the Stuarts. The \emph{First Folio} of Shakespeare, and, crucially, the Authorised Version of the Bible were both published during the reign of King James; and these two books provided a canon around which the rest of the language could be built. Following the Glorious Revolution, English literature entered a noticeable dry spell; the poets of eighteenth century are dwarfed by both their predecessors and their successors. A second explosion of creativity occurred with the birth of Romanticism and the Revolution in France.

\subsection{Latin Analogy, Part I}

Latin literature experienced a Golden Age of its own. Indeed, a century after Latin poetry was fathered by Ennius, it reached its apogee under the care of Ovid, Horace, Propertius and, principally, Virgil.

The version of Latin which Wyatt and Shakespeare must have learned at school was quite deliberately and self-consciously created by the poets of this Golden Age. In just the same way, Shakespeare and Wyatt moulded the version of English which schoolchildren will learn in centuries to come.

\subsection{\emph{Fin de si\`ecle}}

Literary languages are begun by political sea-changes; they are also ended by them. Just as British or Good-English was born at Bosworth Field, it died on the fields of Flanders. Indeed, even as late as September 1914, a good two months into the First World War and subsequent to the bloodletting and stalemate of the Battle of the Frontiers, Laurence Binyon could still write:\\

\settowidth{\versewidth}{\footnotesize Solemn the drums thrill: Death august and royal}
\begin{verse}[\versewidth]
{\footnotesize
Solemn the drums thrill: Death august and royal\\
Sings sorrow up into immortal spheres.\\
There is music in the midst of desolation\\
And a glory that shines upon our tears.\footnote{From `For the Fallen'. \emph{The Cause: Poems of the War}.}\\
}
\end{verse}

\bigskip

But after the 1st July 1916, with its sixty thousand British casualties in one morning, and, sixth months later, the slaughterhouse of Passchendaele, it was no longer possible for a serious and sensitive poet to write in that special language. And in any case, the nation to which the language belonged, the same United Kingdom of Great Britain and Ireland, was to cease to exist in 1922. Subsequent to the Great War, the only options available to an Englishman intent on writing poetry were either nostalgia and anachronism, as per John Betjeman and Philip Larkin, or a wilful embrace of deformity and nonsense, as per Thomas Eliot and Geoffrey Hill. Thus we have our next Rule:

\begin{figure}[h]
\caption{Bosworth to Passchendaele}
\begin{quote}
A poem may only be considered to be written in British or Good-English if, and only if, the poet in question flourished between the Battles of Bosworth Field and Passchendaele (but the converse is not true).
\end{quote}
\end{figure}

Two important caveats to the Rule just outlined ought to be mentioned at this point. Firstly, there are a handful of poets -- Hardy and Housman being the two that come to mind -- who, having flourished before the War, continued to produce a small quantity of poetry for a few years afterwards. Such poetry ought to be considered as having been written in Good-English. Secondly, there is the literary miracle that is the works of Wystan Auden.

In a sad degenerate age, Auden managed to write poetry which is neither nonsensical nor anachronistic, but which, on the contrary, possesses a compelling clarity and vitality. It could be argued that Auden is surpassed at certain points by Shakespeare. (Personally, the Almanackist feels it's an open question.) But Shakespeare was writing in an age in which poets were turning out masterpieces of world literature almost as a matter of course; Auden was not. Reading, for instance, his epilogue to \emph{The Orators}, it's like someone has managed to knock together a piece of architecture with all the careful beauty of an English cathedral amongst the compulsory ugliness of the Bauhaus. Thus we English ought to consider Auden, and not Shakespeare, as our greatest poet; and thus we have our next Rule:

\begin{figure}[h]
\caption{Wystan Auden}
\begin{quote}
The works of Wystan Auden are exempt from the Bosworth to Passchendaele Rule; they are considered to be written in Good-English.
\end{quote}
\end{figure}

Does the Almanackist contend that, with the exception of Wystan Auden, there have been no good British poets since the First World War? Yes and no. Of the writers who made the attempt, some were very good, but none were really poets. Consider these lines of Ronald Thomas, typical of the best late twentieth century British literature:\\

\settowidth{\versewidth}{\footnotesize There was Dai Puw. He was no good.}
\begin{verse}[\versewidth]
{\footnotesize
There was Dai Puw. He was no good.\\
They put him in the fields to dock swedes,\\
And took the knife from him, when he came home\\
At late evening with a grin\\
Like the slash of a knife on his face.

There was Llew Puw, and he was no good.\\
Every evening after the ploughing\\
With the big tractor he would sit in his chair,\\
And stare into the tangled fire garden,\\
Opening his slow lips like a snail.

There was Huw Puw, too. What shall I say?\\
I have heard him whistling in the hedges\\
On and on, as though winter\\
Would never again leave those fields,\\
And all the trees were deformed.

And lastly there was the girl;\\
Beauty under some spell of the beast.\\
Her pale face was the lantern\\
By which they read in life's dark book\\
The shrill sentence: God is love.\footnote{`On the Farm'. \cite{norton}.}\\
}
\end{verse}

\bigskip

Now, what would happen if we were to tamper with these verses slightly?\\

\begin{verse}
{\footnotesize
There was Dai Puw. He was no good. They put him in the fields to dock swedes, and took the knife from him when he came home at late evening with a grin like the slash of a knife on his face.

There was Llew Puw, and he was no good. Every evening after the ploughing with the big tractor he would sit in his chair, and stare into the tangled fire garden, opening his slow lips like a snail.

There was Huw Puw, too. What shall I say? I have heard him whistling in the hedges on and on, as though winter would never again leave those fields, and all the trees were deformed.

And lastly there was the girl; beauty under some spell of the beast. Her pale face was the lantern by which they read in life's dark book -- the shrill sentence -- God is love.\\
}
\end{verse}

\bigskip

By removing the line-breaks, very little, it could be argued, has been removed from the poem.\footnote{The heavy caesura in the last sentence is, admittedly, conspicuous in its absence in the prose version. But this one detail can be supplied quite happily in prose by modifying the punctuation -- as, indeed, the Almanackist has done.} And, conversely, little remains in the prose version to suggest where line-breaks might have been. Conclusion: what Thomas et al. wrote would be more helpfully categorised, not as poetry -- at least, not poetry of the Good-English variety -- but as elegant fragments of prose.

\subsection{Latin Analogy, Part II}

Victory in the First World War allowed the British Empire to reach its greatest territorial extent; but, subsequent to the same war, it proved impossible for new poets to write in the British literary language. Likewise, although the Roman Empire was at its most robust under the rule of Augustus (27 BC -- 14 AD),\footnote{And in fact the Empire only reached its greatest territorial extent over a century later.} the Golden Age of Latin literature was nevertheless buried with him.

As in Hesiod's myth,\footnote{{\it \greektext{>'Erga ka`i <Hmera`i}} (``Works and Days''), lines 109-201.} the Golden Age was followed by the Silver, the outstanding poet of which was Martial. Now a good poem is like a good stout; it builds a person up. Martial's verses, on the other hand, have more in common with watered-down lager; their wit offers a certain short-term mollification, but little real nourishment.

The Silver Age is commonly accepted to have come to an end with the death of Trajan in 117. The literary period which followed is known as Late Latin, and this age produced very little good poetry, except for occasional sparks of interest such as Boethius' \emph{Consolatio} or Jerome's translation of the Bible. It is to be noted that the most important Roman prose writers of this period, e.g. Marcus Aurelius and Cassius Dio, elected to write in Greek.

The Late Latin poets are sometimes referred to as the ``Epigoni'',\footnote{That is, \emph{offspring}.} about whom Auden wrote an amusing poem:

\bigskip

\settowidth{\versewidth}{\footnotesize To their credit, a reader will only perceive}
\begin{verse}[\versewidth]
{\footnotesize
To their credit, a reader will only perceive\\
That the language they loved was coming to grief,\\
Expiring in preposterous mechanical tricks,\\
Epanaleptics, rhopalics, anacyclic acrostics...\footnote{From `The Epigoni', \emph{Homage to Clio}.}\\}
\end{verse}

\bigskip

Auden no doubt intended said poem as food for thought for the poetry of our own age; but such thoughts are the substance of the next subsection.

\bigskip

\subsection{The Future of English Poetry}

We've already been over how the Golden Age of English Poetry was born, blossomed and died. Naturally, it was followed by a (brief) Silver Age. This period was dominated, this side of the Atlantic, by a celebrated triumvirate -- Philip Larkin, Ted Hughes and Thom Gunn -- and, on the other, by Robert Lowell and John Berryman. Lowell and Berryman died in the seventies, Larkin in the eighties; Ted Hughes died in 1998, with whom the Silver Age comes to a close.\footnote{Thom Gunn survived until 2004, but the Almanackist thinks of him as the Lepidus of the three.} Thus the period of English literature in which poetry is currently being written could be referred to as ``Late''.

If the English language continues to trace the same trajectory as Latin, we have every reason to be pessimistic regarding the decades, and indeed the centuries, to come. A handful of interesting poems will be written, a few diverting pieces, but nothing indispensable to the language itself. In the same way, Boethius' \emph{Consolatio} is a well-made book, its poetry not without beauty; but schoolchildren learn Virgil, not Boethius, and a complete understanding of Classical Latin could be put together without that voice.

Thus we can conclude that now, at the close of the Silver Age, is an appropriate time to close the canon of English poetry. And thus we have our next Rule:

\begin{figure}[h]
\caption{\it Cr\`eme de la cr\`eme}
\begin{quote}
Only poetry written in Good-English ought to be considered for the \emph{Almanack}.
\end{quote}
\end{figure}

\section{The Calendars of Man}

One reads in the first chapter of Genesis:

\bigskip

{\footnotesize And God said, `Let there be lights in the firmament of the heavens to separate the day from the night; and let them be for signs and for seasons and for days and years...'\footnote{v. 14.}}

\bigskip

The Scriptures are apt, for the calendars of man have depended almost exclusively on the habits of two heavenly bodies: the sun and the moon. One might say that the history of mankind's calendars is a battle between these two bodies for supremacy.

The general trend of this history is a movement away from the moon and towards the sun. In primitive times, the moon's cycle of twenty-odd days was easily observed, and must have provided a convenient frame of reference for identifying a particular day. The precise day on which a solstice or equinox falls, on the other hand, is much less obvious. However, as convenient as the patterns of the moon might be for an ancient astronomer, their effect on human life is negligible in comparison with the undulations of the sun, particularly at higher latitudes. Thus the vast majority of civilisations begin following a lunar calendar, and then, as scientific knowledge increases, a solar calendar is adopted.

The quintessential example of this process is the Roman calendar. It may well be that the earliest Roman calendars were purely lunar like the Islamic calendar; the Almanckist knows of no compelling evidence either way. In any case, by the time of the late Republic an awkward lunisolar compromise had been reached, wherein a year consisted of the familiar twelve months of our own calendar -- these being defined by the phases of the moon -- with an additional ``intercalary'' month being inserted half way through February at the discretion of the College of Pontiffs in order to keep the calendar year from getting too out of kilter with the sun. Such a tortuous calendar might have been feasible in the life of a city-state, but it proved to be a nightmare for the peoples of an intercontinental empire; it would take many weeks for the decisions of the College to be fully disseminated, leaving the provinces thoroughly confused regarding the correct date.\footnote{Furthermore, the intrinsic flaws of the old calendar were exacerbated by the College's tendency to lengthen or shorten the year according to political, and not astronomical, considerations.}

In 46 BC, Julius Caesar decided that enough was enough, and, like his r\^ole-model Alexander, cut the Gordian Knot. Appointing himself \emph{dictator perpetuo},\footnote{Actually, although he was already dictator, Caesar was only awarded the title \emph{dictator perpetuo} sometime after the adoption of the new calendar. But the spirit, if not quite the letter, of what the Almanackist has written is correct.} he abolished the old calendar, replacing it with a new one which drew on the best practices of the peoples of the ancient world: the Egyptian custom of deriving the calendar purely from the sun, the Greek insight that the length of a solar year was very close to 365\sfrac{$1$}{$4$} days long, and the old Roman names. This calendar, with only the slightest of tinkering, has gone on to be adopted by the whole world.\footnote{This potted history of the Julian Calendar is drawn largely from Plutarch's life of Caesar in \emph{Parallel Lives}.}

\chapter*{Principles of the \emph{Almanack}}
\renewcommand*{\theHsection}{ch3.\the\value{section}}
\setcounter{section}{0}

\section{The Cyprian Calendar}

The Cyprian Calendar is a reconstruction of the Roman lunisolar calendar which preceded the Julian Calendar. It consists of thirteen months:

\begin{table}[h]
\begin{tabular}{l l}
Primilis & (Thirty days)\\
Sectilis & (Twenty-nine days)\\
Tertilis & (Thirty days)\\
Quartilis & (Twenty-nine days)\\
Quintilis & (Thirty days)\\
Sextilis & (Twenty-nine days)\\
September & (Thirty days)\\
October & (Twenty-nine days)\\
November & (Thirty days)\\
December & (Twenty-nine days)\\
Unodecember & (Thirty days)\\
Duodecember & (Twenty-nine or thirty days)\\
Intercalaris & (Adjusted)
\end{tabular}
\end{table}

The first day of each year, I Pri, i.e. the first day of Primilis, is defined as beginning at sunset preceding the night of the new moon following the spring equinox. Each subsequent day begins at the following sunset. The length of Intercalaris is adjusted each year to ensure that the next I Pri falls on the correct day.

The ``King of Cyprus''\footnote{The Almanackist has only picked on Cyprus because there was once a Christian noble family which legitimately claimed the title ``King of Cyprus'', but the House of Lusignan has since died out.} determines when I Pri ought to fall. Years are lettered according to the reign of the current King of Cyprus. Thus the first year of Thomas, the Almanackist's own name, is $\mathfrak{T}_1$, the second of the same, $\mathfrak{T}_2$, etc. If there was a King of Cyprus called John, the first year of his reign would be lettered $\mathfrak{J}_1$; if Timothy, then $\mathfrak{Ti}_1$. If there was a second King of Cyprus called Thomas, he would be known as Thomas II, and the first year of his reign would be lettered $\mathfrak{T}^\text{II}_1$.

Now in the two thousand and fourteenth year of the New Style, the  vernal equinox occurred at three minutes to five in the afternoon of the twentieth day of March, i.e. 20 Mar 2014 (NS), and the subsequent new moon and sunset occurred at 1948 30 Mar and 1932 31 Mar respectively. Thus Year $\mathfrak{T}_1$ of the Cyprian Calendar began at sunset on that day, i.e.

\[\text{I Pri $\mathfrak{T}_1$} \left\{
\begin{array}{c}
\text{began at sunset on 31 Mar 2014 (NS)}\\
\text{ended at sunset on 01 Apr 2014 (NS)}
\end{array}
\right.
\]

\bigskip

Thus it can calculated that the Cyprian Calendar will follow the Hebrew Calendar until at least 2114 (NS), by which time the Almanackist will have occupied his grave a good few years. The Cyprian date of any given day can be calculated using the number of the Hebrew date of that day, and using the following table to convert the month:

\begin{table}[h]
\centering
\begin{tabular}{l l}
Primilis &= $ $ Nisan\\
Sectilis &= $ $ Iyar\\
Tertilis &= $ $ Sivan\\
Quartilis &= $ $ Tammuz\\
Quintilis &= $ $ Av\\
Sextilis &= $ $ Elul\\
September &= $ $ Tishrei\\
October &= $ $ Cheshvan\\
November &= $ $ Kislev\\
December &= $ $ Tevet\\
Unodecember &= $ $ Shevat\\
Duodecember &= $ $ Adar \emph{or} Adar'\\
Intercalaris &= $ $ Adar''
\end{tabular}
\end{table}

\section{Its Structure}

The structure of the \emph{Almanack} is the structure of the aforementioned Cyprian Calendar. Now the months are grouped together in accordance with Hippocrates'\footnote{Humourism is given what is probably its first comprehensive treatment in {\it \greektext{Per'i F'usews Anjr'wpou}} (``On the Nature of Man''). This treatise is traditionally attributed to Hippocrates, although Aristotle and others have disputed this attribution.} notion of the four humours, which, although repudiated from a scientific point of view, retains, the Almanackist believes, a certain insight into the psychology of man. The table on the next page ought to make things clear.

\begin{table}[h]
\centering
\begin{tabular}{l|l|l}
\multicolumn{1}{c}{\sc Month} & \multicolumn{1}{|c|}{\sc Humour} & \multicolumn{1}{c}{\sc Mood} \\ \hline \hline
Primilis & & \\
Sectilis & Yellow bile & Pride, ambition, energy \\
Tertilis & & \\ \hline
Quartilis & & \\
Quintilis & Blood & Joy, friendliness, warmth \\
Sextilis & & \\ \hline
September & & \\
October & Phlegm & Serenity, faith, acceptance \\
November & & \\ \hline
December & & \\
Unodecember & Black bile & Sadness, despair, compassion \\
Duodecember & & \\ \hline
Intercalaris & None & -----
\end{tabular}
\end{table}

The poetry selected for a given day is to correspond to the mood of the time of year. Furthermore, there is to be a continuity of mood, so that, for example, Tertilis is to be characterised by energy mixed with a little warmth, whereas Quartilis is to be characterised by warmth mixed with a little energy.

Now the entry in the \emph{Almanack} for each day shall consist of three elements:

\begin{itemize}
\item[1.]{A longer poem, called the \emph{song};}
\item[2.]{A shorter poem, called the \emph{sonnet}; and,}
\item[3.]{A proverb.}
\end{itemize}

For the sake of argument, a sonnet is defined as consisting of not more than fourteen standard lines -- a standard line being a line of iambic pentameter\footnote{Iambic pentameter being the metre in which the vast majority of Shakespeare's works are written. E.g. `Now is the winter of our discontent' would be a standard line.} -- whereas a song is anything longer.

\section{The Selection of Its Contents}

\subsection{Essay on Criticism}

It's a shame that his poetry belongs to the second, and not quite the first, rank of English poetry; for the Almanackist has a great deal of affection for Basil Bunting. Both spent a brief period at a certain Quaker school in the West Riding of Yorkshire\footnote{That is, Ackworth School.} which nevertheless made a permanent and kindly impression on their approaches to literature. Bunting's short essay, `The Poet's Point of View', expresses such a wise and truthful perspective on literary criticism that it's worthy of extensive quotation:
  
\bigskip
{\footnotesize Poetry, like music, is to be heard. It deals in sound -- long sounds and short sounds, heavy beats and light beats, the tone relations of vowels, the relations of consonants to one another which are like instrumental colour in music. Poetry lies dead on the page, until some voice brings it to life, just as music, on the stave, is no more than instructions to the player. A skilled musician can imagine the sound, more or less, and a skilled reader can try to hear, mentally, what his eyes see in print: but nothing will satisfy either of them till his ears hear it as real sound in the air. Poetry must be read aloud.}

{\footnotesize Reading in silence is the source of half the misconceptions that have caused the public to mistrust poetry. Without the sound, the reader looks at the lines as he looks at prose, seeking a meaning. Prose exists to convey meaning, and no meaning such as prose conveys can be expressed as well in poetry. That is not poetry's business.}

{\footnotesize Poetry is seeking to create, not meaning, but beauty; or if you insist on misusing words, its ``meaning'' is of another kind, and lies in the relation to one another of lines and patterns of sound, perhaps harmonious, perhaps contrasting and clashing, which the hearer feels rather than understands, lines of sound drawn in the air which stir deep emotions which may not even have a name in prose. This needs no explaining to an audience which gets its poetry by ear. It has neither time nor inclination to seek a prose meaning in poetry.}

{\footnotesize Very few artists have clear, analytical minds. They do what they do because they must. Some think about it afterwards in a muddled way and try unskilfully to reason about their art. Thus theories are produced which mislead critics and tyros, and sometimes disfigure the work of artists who try to carry out their own theories.}

{\footnotesize There is no need of any theory for that which gives pleasure through the ear, music or poetry. The theoreticians will follow the artist and fail to explain him.}
\bigskip
  
Bunting then goes on to say certain things with which the Almanackist cannot agree, and so let's skip ahead to where the two are next of one mind:\footnote{But for anyone who wishes to read the unexpurgated version of his essay, it can be found in the Bloodaxe Books edition of \emph{Briggflatts} (2009).}
  
\bigskip
{\footnotesize Do not let the people who set examinations kid you that you are any nearer to understanding a poem when you have parsed and analysed every sentence, scanned every line, looked up the words in the Oxford Dictionary and the allusions in a library of reference books. That sort of knowledge will make it harder to understand the poem because, when you listen to it, you will be distracted by a multitude of irrelevant scraps of knowledge. You will not hear the meaning, which is in the sound.}

{\footnotesize All the arts are plagued by charlatans seeking money, or fame, or just an excuse to idle. The less the public understands the art, the easier it is for charlatans to flourish. Since poetry reading became popular, they have found a new field, and it is not easy for an outsider to distinguish a fraud from a poet. But it is a little less difficult when poetry is read aloud. Claptrap work soon bores. Threadbare work soon sounds thin and broken backed.}

{\footnotesize There were mountebanks at the first Albert Hall meeting, as well as a poet or two, but the worst, most insidious charlatans fill chairs and fellowships at universities, write for the weeklies or work for the BBC or the British Council or some other asylum for obsequious idlers. In the eighteenth century it was the church. If these men had to read aloud in public, their empty lines, without resonance, would soon give them away.}
\bigskip

Being Bunting's disciple, the Almanackist must insist that the \emph{Sitz im Leben}\footnote{That is, \emph{situation in life}.} for which this \emph{Almanack} was devised is that the song for a given day should be read aloud -- or, where a tune is indicated, sung -- in front of a small group of people as a kind of grace before the main meal of the day. (The other material for that day may be read out at some other time.) Hopefully this will allow the poems to be shown off in the best light.

\subsection{Permanence}

Of all the definitions of a poem that the Almanackist has come across, the most convincing is, `A linguistic device for making itself remembered.' Thus we have our next Rule:

\begin{figure}[h]
\caption{Permanence}
\begin{quote}
When selecting poetry for the \emph{Almanack}, the primary test for discerning the best poetry is its persistence in the reader's memory.
\end{quote}
\end{figure}

\subsection{Sources}

The Almanackist has endeavoured to only use as sources for the \emph{Almanack} those books which have earned the lasting affection of the British nation -- e.g. Palgrave's \emph{Golden Treasury}, the King James Bible, Shakespeare's \emph{Complete Works}, etc -- and only, as a last resort, to use less cherished texts.

I've taken the liberty of amending those passages which seemed to cry out for as much. For example, in 1 Corinthians 13 I've substituted \emph{love} where the KJV puts `charity'. All such amendments are indicated in the footnotes. I've also converted certain unfamiliar proper nouns into more familiar forms. For example, I've substituted \emph{Lebanon} for the \emph{BCP}'s `Libanus'. In all the amendments I've made, I've tried to change the original texts as little as possible, only correcting what seemed to be the most egregious faults.

\subsection{Religion}

Religious poetry of course makes up a sizable portion of the best English literature, but the Almanackist has wished to avoid his work becoming the property of any particular faith. The Almanackist distinguishes between Natural Religion and Revealed Religion; the former arises from the \emph{Urmonotheismus}\footnote{This term was coined by anthropologist Wilhelm Schmidt in his twelve volume masterpiece, \emph{Der Ursprung der Gottesidee} (``The Origin of the Idea of God'') wherein he concludes that belief in one almighty Sky-Father is instrinsic to human life.} which all cultures and times have in common, whereas the latter claims a special knowledge of the divine.

The Almanackist has judged that poetry expressing Natural Religion is to be considered for the \emph{Almanack}, but poetry expressing Revealed Religion is not. This is not to denigrate Revealed Religion, but rather to recognise that great literature concerns the whole world, and not any sect in particular. So we have another Rule:

\begin{figure}[h]
\caption{\it Urmonotheismus}
\begin{quote}
Poetry expressing Natural Religion is to be considered for the \emph{Almanack}, but poetry expressing Revealed Religion is not.
\end{quote}
\end{figure}

\section{Orthography \& Typography}

\subsection{Orthography}

In accordance with Rule 1, the spellings followed in the \emph{Almanack} are those of Johnson's \emph{Dictionary}, except where so doing would obviously be barbarous.

This involves, most notably, modifying some of the poems of Robert Burns as they are commonly received. Thus `auld' becomes `old' and `pou'd' becomes `pu'd', whereas `tak'' stays `tak'' and `fiere' stays `fiere'; in the former case, the only divergence from Johnson's spelling is a matter of apostrophes, which are allowed, and, in the latter, `fiere' is sufficiently different from its Johnsonian equivalent \emph{friend} to count as another word.

None of this has anything to do with belittling the Scots (or any other people); Burns is treated in exactly the same way as Barnes.\footnote{William Barnes was an English poet who wrote in the Dorset dialect.} Thus we have our last Rule:

\begin{figure}[h]
\caption{Burns \& Barnes}
\begin{quote}
The spellings of the \emph{Almanack} are to follow Johnson's \emph{Dictionary}, except in those cases where so doing would clearly inflict violence on the text, i.e. where the spelling is so different that a different word has effectively been formed.
\end{quote}
\end{figure}

Where the spelling has been altered in the transmission of a text from the source to the \emph{Almanack}, a zeta ($\zeta$) is to be placed in the footnotes for that poem or proverb. If two or more words are altered in a single poem, a xi ($\xi$) is to be inserted.

If a word not found in Johnson's \emph{Dictionary} is present in a text, a dagger ($\dagger$) is to be placed in the footnotes, with a corresponding entry in the `Supplement to Johnson's \emph{Dictionary}' found in the back matter of the \emph{Almanack}. If two or more such words are to be found in a single poem, a double dagger ($\ddagger$) is to be inserted.

The reader will notice that the Almanackist prefers Johnson's \emph{almanack} to the OED's \emph{almanac}, but, otherwise, the spelling in the introductory front matter of the \emph{Almanack} is to follow the OED.

\subsection{Typography}

The Almanackist has elected to adopt the following conventions in the poetry of the \emph{Almanack}:

\begin{itemize}
\item{Capitalisation is to be according to the Italian style, i.e. \emph{october} rather than \emph{October}, \emph{english} rather than \emph{English}; but \emph{Matthew}, \emph{Mark} and \emph{England} remain as they are.}
\item{The names of people are to be printed in \emph{italics}.}
\item{The names of places are to be printed in {\sc small capitals}.}
\item{Names which, in prose, would be printed in italics (such as the names of books) are to be printed in {\hge Old English}.}
\item{Except where it joins two clauses or begins a line or sentence, or where an especially loose or discordant union is indicated, the word \emph{and} is to be replaced with \emph{\&}.}
\end{itemize}

Changes to punctuation, except where such a change would alter the meaning of the text, are to pass unremarked.

The tetragrammaton, where it appears in the King James Version of the Holy Bible, I have generally rendered as {\hge LORD}. This follows a precedent set by the Septuagint, the Vulgate and, indeed, the King James Version itself. Unfortunately, the authors of the Old Testament were wont to use a construction which, translated literally, would be rendered as \emph{the Lord YHWH}. If we were to follow the same rule here as before, we would be left with \emph{the Lord {\hge LORD}}, which is gratuitously ugly. In such cases, I have followed the King James Version, and rendered the tetragrammaton as {\hge GOD}. The tetragrammaton also appears in the \emph{Book of Common Prayer}'s translation of the Psalms. However, this translation was made according to a different set of rules than the King James Version's strict word-for-word translation philosophy, and, in any case, the tetragrammaton is not marked in any special way in the original text.

Words attributed to Christ himself in any of the four gospels are printed in {\color{red} red}.

\section{Ecclesiastes and the Song of Songs}

The Almanackist has also included a translation\footnote{This is generally the King James Version, but the text has been amended to conform with either the Revised Version or Revised Standard Version where one of those two versions gave a more plausible or more beautiful reading. Any verse which has been amended in this way is marked with a printer's fist (\ding{43}) in the margin.} of Ecclesiastes and the Song of Songs. These two marvellous little books are to be read on a shorter cycle than the rest of the \emph{Almanack}, as explained in the following table:

\begin{table}[h]
\centering
\caption*{{\sc Table.} Reading Cycles for Ecclesiastes and the Song of Songs}
\begin{tabular}{p{3cm}|p{3cm}|p{4cm}}
{\sc Book} & {\sc Divisions} & {\sc To be read every...}\\ \hline \hline
Ecclesiastes & 30 & Cyprian month\\ \hline
Song of Songs & 8 & Cyprian week
\end{tabular}
\end{table}

For Ecclesiastes: in months of twenty-nine days, the thirtieth division is to be omitted.

For the Song of Songs: Each Cyprian month is split into four Cyprian weeks, i.e. from the first day to the seventh day, the first week; from the eighth day to the fourteenth day, the second week; from the fifteenth day to the twenty-first day, the third week; from the twenty-second day to the end of the month, the fourth and final week. The eighth division is only to read during the last week of each Cyprian month. In months of thirty days, i.e. when the last week of the month contains nine days, silent reflection is to be allotted on the ninth day where a portion of the Song of Solomon would otherwise be read.

\chapter*{Future Drafts of the \emph{Almanack}}
\renewcommand*{\theHsection}{ch4.\the\value{section}}
\setcounter{section}{0}

\section{Anticipated Changes}

In this third draft, it's only been possible to consult: 

\begin{enumerate}
\item{Palgrave's \emph{Golden Treasury},}
\item{The King James Version of the Holy Bible and the \emph{Book of Common Prayer},}
\item{\emph{The Norton Anthology of Poetry}, and}
\item{\emph{The Tempest} from Shakespeare's \emph{Complete Works}.\footnote{This is a list of the ``big'' publications consulted. For a list of \emph{every} source, please see the ``Sources'' chapter at the end of this book.}}
\end{enumerate}

So far, as the reader can see, the sources have only yielded enough material to fill five months entirely, rather than a whole year of poems and proverbs. For the fourth draft, I intend to collect poems until absolutely all of the gaps can be filled.

In terms of the aforementioned gaps, I calculate that I need to acquire the following:

\begin{table}[h]
\centering
\caption*{{\sc Table.} Songs, Sonnets and Proverbs Outstanding (A)}
\begin{tabular}{c||c|c|c}
{\sc Month} & {\sc Songs} & {\sc Sonnets} & {\sc Proverbs}\\ \hline \hline
Primilis & 0 & 0 & 0 \\ \hline
Sectilis & 25 & 25 & 25 \\ \hline
Tertilis & 28 & 28 & 28 \\ \hline
Quartilis & 28 & 28 & 28 \\ \hline
Quintilis & 0 & 0 & 0 \\ \hline
Sextilis & 28 & 28 & 28 \\ \hline
September & 0 & 0 & 0 \\ \hline
October & 0 & 0 & 0 \\ \hline
November & 30 & 30 & 30 \\ \hline
December & 25 & 25 & 25 \\ \hline
Unodecember & 27 & 27 & 27 \\ \hline
Duodecember & 0 & 0 & 0
\end{tabular}
\end{table}

To put this information another way, I require:

\newpage

\begin{table}[h]
\centering
\caption*{{\sc Table.} Songs, Sonnets and Proverbs Outstanding (B)}
\begin{tabular}{c||c|c|c}
{\sc Humour} & {\sc Songs} & {\sc Sonnets} & {\sc Proverbs}\\ \hline \hline
Choleric & 53 & 53 & 53 \\ \hline
Sanguine & 0 & 56 & 56 \\ \hline
Phlegmatic & 30 & 30 & 30 \\ \hline
Melancholy & 52 & 52 & 52 \\
\end{tabular}
\end{table}

And also 56 folk songs, for Quartilis and Sextilis.

\section{Procedure for Making Suggestions to the Almanackist}

To whichever hands this book should fall into: please feel free to contact the Almanackist with any general comments or suggestions for the inclusion of a particular poem or song. He can be reached at \url{tomdothosker@gmail.com}.

All enquiries will be read sympathetically.

\bigskip

\section{Final Exhortation}

This life is very short, but nonetheless `is attended with so many evils'.\footnote{\cite{bunyan}.} The Almanackist's hope was, in giving the reader regular and easy exposure to the best of English literature, to help him `better to enjoy life, or better to endure it.'\footnote{Dr Johnson, in a review of Soame Jenyns' \emph{Free Enquiry into the Nature of the Origin of Good and Evil}.} Or, as the Very Reverend Dr Donne put it:

\bigskip

\settowidth{\versewidth}{\footnotesize Since I am coming to that holy room,}
\begin{verse}[\versewidth]
{\footnotesize
Since I am coming to that holy room,\\
\vin Where, with thy choir of saints for evermore,\\
I shall be made thy music; as I come\\
\vin I tune the instrument here at the door,\\
\vin And what I must do then, think here before.\footnote{`Hymn to God, My God, in My Sickness', \emph{Poetical Works}, Ed. by Prof. Sir Herbert Grierson.}\\}
\end{verse}

\bigskip

The same score he has in his hand now will be yours and mine soon enough.

\end{document}
