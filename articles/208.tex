\documentclass[MAIN]{subfiles}
\begin{document}

\settowidth{\versewidth}{Thou first great cause, least understood:}
\begin{verse}[\versewidth]
\poemlines{5}
Thou first great cause, least understood:\footnotetext{`The Universal Prayer', Alexander Pope (1688 -- 1744). \cite{norton}. The Almanackist has taken the liberty of removing some of the weaker verses.}\\* 
\vin Who all my sense confined,\\
To know but this -- that thou art good,\\*
\vin And that myself am blind:\\!

What blessings thy free bounty gives,\\*
\vin Let me not cast away;\\
For God is paid when man receives,\\*
\vin To enjoy is to obey.\\!

Let not this weak, unknowing hand\\*
\vin Presume thy bolts to throw,\\
And deal damnation round the land,\\*
\vin On each I judge thy foe.\\!

Teach me to feel another's woe,\\*
\vin To hide the fault I see;\\
That mercy I to others show,\\*
\vin That mercy show to me.\\!

This day, be bread and peace my lot:\\*
\vin All else beneath the sun,\\
Thou know’st if best bestowed or not,\\*
\vin And let thy will be done.\\!

To thee, whose temple is all space,\\*
\vin Whose altar, earth, sea, skies!\\
One chorus let all being raise!\\*
\vin All nature's incense rise!
\end{verse}

\end{document}