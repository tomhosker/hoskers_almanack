\documentclass[MAIN]{subfiles}
\begin{document}

\settowidth{\versewidth}{Row us out from {\sc Desenzano}; to your {\sc Sirmione} row!}
\begin{verse}[\versewidth]
\poemlines{5}
Row us out from {\sc Desenzano}; to your {\sc Sirmione} row!\footnotetext{`Frater Ave atque Vale', Alfred Tennyson, 1st Baron Tennyson, Poet Laureate (1809 -- 1892). \cite{norton}. The title (which means, ``Greetings, brother, and farewell'') is a quotation from Catullus 101; the same poet wrote of his affection for Sirmione (which he called Sirmio) in Catullus 31, \emph{venusta} meaning \emph{beautiful}.}\\*
So they rowed, and there we landed -- `{\hge O venusta} {\sc Sirmio}' --\\
There to me through all the groves of olive in the summer glow,\\
There beneath the roman ruin where the purple flowers grow,\\
Came that `{\hge ave atque vale}' of the poet's hopeless woe,\\
Tenderest of roman poets 19 hundred years ago,\\
`{\hge Frater, ave atque vale}' -- as we wandered to \& fro\\
Gazing at the lydian laughter of the {\sc Garda Lake} below\\*
Sweet \emph{Catullus}'s all-but-island, olive-silvery {\sc Sirmio}!
\end{verse}

\end{document}