\documentclass[MAIN]{subfiles}
\begin{document}

\settowidth{\versewidth}{\vin White as blanched almonds, or the falling snow,}
\begin{verse}[\versewidth]
\poemlines{5}
\emph{Alice} is tall \& upright as a pine,\footnotetext{`Two Rural Sisters', Charles Cotton (1630 -- 1687). \cite{newlove}.}\\*
\vin White as blanched almonds, or the falling snow,\\
\vin Sweet as the damask roses when they blow,\\
And doubtless fruitful as the swelling vine.\\
Ripe to be cut, \& ready to be pressed,\\
\vin Her full cheeked beauties very well appear,\\
\vin And a year's fruit she loses every year,\\*
Wanting a man to improve her to the best.\\!

Full fain she would be husbanded, and yet,\\*
Alas, she cannot a fit labourer get\\
\vin To cultivate her own content:\\
Fain she would be (God wot) about her task,\\
And yet (forsooth) she is too proud to ask,\\*
\vin And (which is worse) too modest to consent.\\!

\emph{Margaret} is of humbler stature by the head\\*
\vin Is (as oft falls out with yellow hair)\\
\vin Than her fair sister, yet so much more fair,\\
As her pure white is better mixed with red.\\
This, hotter than the other 10 to one,\\
\vin Longs to be put into her mother's trade,\\
\vin And loud proclaims she lives too long a maid,\\*
Wishing for one t'untie her virgin zone.\\!

She finds virginity a kind of ware,\\*
That's very very troublesome to bear,\\
\vin And being gone, she thinks will ne'er be missed:\\
And yet withal, the girl has so much grace,\\
To call for help I know she wants the face,\\*
\vin Though asked, I know not how she would resist.
\end{verse}

\end{document}