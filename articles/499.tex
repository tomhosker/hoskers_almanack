\documentclass[MAIN]{subfiles}
\begin{document}

\settowidth{\versewidth}{When that the poor have cried, \emph{Caesar} hath wept:}
\begin{verse}[\versewidth]
\poemlines{5}
Friends, romans, countrymen, lend me your ears.\footnotetext{$\xi$ William Shakespeare (1564 -- 1616). \cite{obev}. These lines are spoken by Mark Antony in \emph{Julius Caesar}, III.2.}\\*
I come to bury \emph{Caesar}, not to praise him.\\
The evil that men do lives after them;\\
The good is oft interr\`ed with their bones;\\
So let it be with \emph{Caesar}. The noble \emph{Brutus}\\
Hath told you \emph{Caesar} was ambitious:\\
If it were so, it was a grievous fault,\\
And grievously hath \emph{Caesar} answered it.\\
Here, under leave of \emph{Brutus} \& the rest --\\
For \emph{Brutus} is an honourable man,\\
So are they all, all honourable men --\\
Come I to speak in \emph{Caesar}'s funeral.\\
He was my friend, faithful \& just to me:\\
But \emph{Brutus} says he was ambitious;\\
And \emph{Brutus} is an honourable man.\\
He hath brought many captives home to {\sc Rome}\\
Whose ransoms did the general coffers fill:\\
Did this in \emph{Caesar} seem ambitious?\\
When that the poor have cried, \emph{Caesar} hath wept:\\
Ambition should be made of sterner stuff:\\
Yet \emph{Brutus} says he was ambitious;\\
And \emph{Brutus} is an honourable man.\\
You all did see that on the Lupercal\\
I thrice presented him a kingly crown,\\
Which he did thrice refuse: was this ambition?\\
Yet \emph{Brutus} says he was ambitious;\\
And, sure, he is an honourable man.\\
I speak not to disprove what \emph{Brutus} spoke,\\
But here I am to speak what I do know.\\
You all did love him once, not without cause:\\
What cause withholds you then, to mourn for him?\\
O judgment! Thou art fled to brutish beasts,\\
And men have lost their reason. Bear with me;\\
My heart is in the coffin there with \emph{Caesar},\\*
And I must pause till it come back to me.
\end{verse}

\end{document}