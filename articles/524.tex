\documentclass[MAIN]{subfiles}
\begin{document}

\settowidth{\versewidth}{Some care to dresse their wearied horse, and some}
\begin{verse}[\versewidth]
\poemlines{5}
The masters go abroad to view the town,\footnotetext{$\xi$ Sir John Harington (1560 - 1612). \cite{obev}. A good chunk of explanation is perhaps helpful with respect to this poem. Firstly, the lines here printed are extracted from Sir John's translation of \emph{Orlando Furioso}, an Italian epic poem by Ludovico Ariosto (which, at almost forty thousand lines, is one of the longest in world literature), these being verses fifty-five to sixty-five of Book XXVIII. Before our story begins, two friends, Giocondo and Astolfo (who also happens to be the King of Lombardy), go on a kind of lads' holiday, seducing numerous women -- they aim for a thousand each -- in order to console themselves over their wives' infidelity. Believing no single man of being able to satisfy a woman's lust, they form a kind of polyandrous marriage with an innkeeper's daughter, Fiametta, who sleeps each night between the two men. In the lines here printed, Fiametta happens to bump into her Greek childhood sweetheart. The two wish to start a life together; however, Fiametta, being taken, comes up with a consolation prize. She tells her Greek paramour to sneak into her room and make love to her, and tricks both Giacondo and Astolfo into believing that the other is enjoying her instead. After these lines end, the two friends discover Fiametta's infidelity after the fact, but take being re-cuckolded with a surprisingly robust sense of humour, and allow her and the Greek to get married.}\\*
\vin And first the churches for devotions' sake;\\
And then the monuments of most renown,\\
\vin As travellers a common custom take:\\
The girl within the chamber sate her down;\\
\vin The men are busied; some the beds do make;\\
Some care to dress their wearied horse, and some\\*
Make ready meat against their masters come.\\!

In this same house the girl a greek had spied,\\*
\vin That in her father's house a boy had been,\\
And slept full often sweetly by her side,\\
\vin And much good sport had pass\`ed them between;\\
Yet fearing lest their love should be descried,\\
\vin In open talk they durst not to be seen,\\
But when by hap the pages down were gone,\\*
Old love renewed and thus they talk thereon.\\!

The greek demands her whither she was going,\\*
\vin And which of these two great estates her keeps.\\
She told them all; she needs no further wooing,\\
\vin And how a-night between them both she sleeps:\\
`Ah!' quoth the greek. `Thou tellest my undoing,\\
\vin My dear \emph{Fiametta}, and with that he weeps;\\
With these two lords wilt thou from Spain be banished.\\*
Are all my hopes thus into nothing vanished?\\!

`My sweet designments turn\`ed are to sour;\\*
\vin My service long finds little recompense;\\
I made a stock according to my power,\\
\vin By hoarding up my wages, and the pence\\
That guests did give, that came in lucky hour;\\
\vin I meant ere long to have departed hence,\\
And to have asked thy sires good will to marry thee,\\*
And that obtained, unto a house to carry thee.'\\!

The wench of her hard fortune doth complain,\\*
\vin And saith that now she doubts he sues too late;\\
The greek doth sigh \& sob, and part doth fain.\\
\vin `And shall I die,' quoth he, `in this estate?\\
Let me enjoy thy sweetness once again,\\
\vin Before my days draw to their doleful date;\\
One small refreshing ere we quite depart\\*
Will make me die with more contented heart.'\\!

The girl with pity mov`ed, thus replies,\\*
\vin `Think not,' quoth she, `but I desire the same;\\
But hard it is among so many eyes,\\
\vin Without incurring punishment \& shame.'\\
`Ah!' quoth the greek, `some means thou wouldst devise,\\
\vin If thou but felt a \sfrac{$1$}{$4$} of my flame,\\
To meet this night in some convenient place,\\*
And be together but a little space.\\!

`Tush!' answered she. `You sue now out of season,\\*
\vin For every night I lie betwixt them two\\
And they will quickly fear and find the treason,\\
\vin Sith still with one of them I have to do.'\\
`Well,' quoth the greek, `I could refute that reason,\\
\vin If you would put your helping hand thereto;\\
You must,' said he, `some pretty 'scuse devise,\\*
And find occasion from them both to rise.'\\!

She first bethinks herself, and after bad\\*
\vin He should return when all were sound asleep,\\
And learn\`ed him, who was thereof right glad,\\
\vin To go \& come, what order he should keep.\\
Now came the greek, as he his lesson had,\\
\vin When all was hushed, as soft as he could creep,\\
First to the door, which opened when he pushed,\\*
Then to the chamber, which was softly rushed.\\!

He takes a long \& leisureable stride,\\*
\vin And longest on the hinder foot he stayed,\\
So soft he treads, although his steps were wide,\\
\vin As though to tread on eggs he were afraid;\\
And as he goes, he gropes on either side\\
\vin To find the bed, with hands abroad displayed,\\
And having found the bottom of the bed,\\*
He creepeth in, and forward go'th his head.\\!

Between \emph{Fiametta}'s tender thighs he came,\\*
\vin That lay upright, as ready to receive;\\
At last they fell unto their merry game,\\
\vin Embracing sweetly now to take their leave;\\
He rode in post, nor can he bait for shame;\\
\vin The beast was good, and would not him deceive;\\
He thinks her pace so easy \& so sure,\\*
That all the night to ride he could endure.\\!

\emph{Giocundo} and the king do both perceive\\*
\vin The bed to rock, as oft it comes to passe,\\
And both of them one error did deceive,\\
\vin For either thought it his companion was:\\
Now hath the greek taken his latter leave,\\
\vin And as he came, he back again doth passe,\\
And \emph{Phoebus}' beams did now to shine begin;\\*
\emph{Fiametta} rose and let the pages in.
\end{verse}

\end{document}