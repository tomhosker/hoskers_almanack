\documentclass[MAIN]{subfiles}
\begin{document}

\settowidth{\versewidth}{\vin That there's some corner of a foreign field}
\begin{verse}[\versewidth]
\poemlines{5}
If I should die, think only this of me:\footnotetext{Rupert Brooke (1887 -- 1915). \cite{norton}. The Almanackist was encouraged to hate this poem at school as an example of the mindless jingoism that led to the Great War in the first place; and, of course, there is something idiotic about it. But there's something noble and beautiful in it too.}\\*
\vin That there's some corner of a foreign field\\
That is forever England. There shall be\\
\vin In that rich earth a richer dust concealed;\\
A dust whom England bore, shaped, made aware,\\
\vin Gave, once, her flowers to love, her ways to roam,\\
A body of England's, breathing english air,\\*
\vin Washed by the rivers, blest by the suns of home.\\!

And think, this heart, all evil shed away,\\*
\vin A pulse in the eternal mind, no less\\
\vin \vin Gives somewhere back the thoughts by England given;\\
Her sights \& sounds; dreams happy as her day;\\
\vin And laughter, learnt of friends; and gentleness,\\*
\vin \vin In hearts at peace, under an english heaven.
\end{verse}

\end{document}