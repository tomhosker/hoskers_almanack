\documentclass[MAIN]{subfiles}
\begin{document}

\settowidth{\versewidth}{So rich with jewels hung, that night}
\begin{verse}[\versewidth]
\poemlines{5}
\vin When I survey the bright\footnotetext{$\xi$ `Nox nocti indicat Scientam', William Habington (1605 -- 1654). \cite{obev}. The title is a quotation from the Vulgate, Psalm 19.2 (or Psalm 18, using the numbering of the Psalms preferred by Roman Catholics), which means, ``Night to night shows knowledge.''}\\*
\vin \vin Celestial sphere;\\
So rich with jewels hung, that night\\*
Doth like an ethiop bride appear:\\!

\vin My soul her wings doth spread\\*
\vin \vin And heavenward flies,\\
Th'Almighty's mysteries to read\\*
In the large volumes of the skies.\\!

\vin For the bright firmament\\*
\vin \vin Shoots forth no flame\\
So silent, but is eloquent\\*
In speaking the Creator's name.\\!

\vin No unregarded star\\*
\vin \vin Contracts its light\\
Into so small a character,\\*
Removed far from our human sight,\\!

\vin But if we steadfast look\\*
\vin \vin We shall discern\\
In it, as in some holy book,\\*
How man may heavenly knowledge learn.\\!

\vin It tells the conqueror\\*
\vin \vin That far-stretched power,\\
Which his proud dangers traffic for,\\*
Is but the triumph of an hour:\\!

\vin That from the farthest north,\\*
\vin \vin Some nation may,\\
Yet undiscovered, issue forth,\\*
And o'er his new-got conquest sway:\\!

\vin Some nation yet shut in\\*
\vin \vin With hills of ice\\
May be let out to scourge his sin,\\*
Till they shall equal him in vice.\\!

\vin And then they likewise shall\\*
\vin \vin Their ruin have;\\
For as yourselves your empires fall,\\*
And every kingdom hath a grave.\\!

\vin Thus those celestial fires,\\*
\vin \vin Though seeming mute,\\
The fallacy of our desires\\*
And all the pride of life confute:\\!

\vin For they have watched since first\\*
\vin \vin The world had birth:\\
And found sin in itself accursed,\\*
And nothing permanent on earth.
\end{verse}

\end{document}