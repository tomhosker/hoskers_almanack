\documentclass[MAIN]{subfiles}
\begin{document}

\settowidth{\versewidth}{Came loud -- and hark, again, loud as before.}
\begin{verse}[\versewidth]
\poemlines{5}
The frost performs its secret ministry,\footnotetext{`Frost at Midnight', Samuel Coleridge (1772 -- 1834). \cite{norton}. In folklore, the flakes of ash floating up the flue were said to predict the arrival of strangers, and thus Coleridge refers to them as such.}\\*
Unhelped by any wind. The owlet's cry\\
Came loud -- and hark, again, loud as before.\\
The inmates of my cottage, all at rest,\\
Have left me to that solitude, which suits\\
Abstruser musings: save that at my side\\
My cradled infant slumbers peacefully.\\
'Tis calm indeed, so calm that it disturbs\\
And vexes meditation with its strange\\
And extreme silentness. Sea, hill, \& wood,\\
This populous village! Sea, \& hill, \& wood,\\
With all the numberless goings-on of life,\\
Inaudible as dreams! The thin blue flame\\
Lies on my low-burnt fire, and quivers not;\\
Only that film, which fluttered on the grate,\\
Still flutters there, the sole unquiet thing.\\
Methinks, its motion in this hush of nature\\
Gives it dim sympathies with me who live,\\
Making it a companionable form,\\
Whose puny flaps \& freaks the idling spirit\\
By its own moods interprets, everywhere\\
Echo or mirror seeking of itself,\\*
And makes a toy of thought.\\!

{\color{white} And makes a toy of thought.} But O how oft,\\
How oft, at school, with most believing mind,\\
Presageful, have I gazed upon the bars,\\
To watch that fluttering str\'anger, and as oft\\
With unclosed lids, already had I dreamt\\
Of my sweet birth-place, and the old church-tower,\\
Whose bells, the poor man's only music, rang\\
From morn to evening, all the hot fair-day,\\
So sweetly, that they stirred \& haunted me\\
With a wild pleasure, falling on mine ear\\
Most like articulate sounds of things to come.\\
So gazed I, till the soothing things, I dreamt,\\
Lulled me to sleep, and sleep prolonged my dreams.\\
And so I brooded all the following morn,\\
Awed by the stern preceptor's face, mine eye\\
Fixed with mock study on my swimming book:\\
Save if the door half opened, and I snatched\\
A hasty glance, and still my heart leaped up,\\
For still I hoped to see the str\'anger's face,\\
Townsman, or aunt, or sister more beloved,\\*
My play-mate when we both were clothed alike.\\!

Dear babe, that sleepest cradled by my side,\\*
Whose gentle breathings, heard in this deep calm,\\
Fill up the intersperséd vacancies\\
And momentary pauses of the thought.\\
My babe so beautiful, it thrills my heart\\
With tender gladness, thus to look at thee,\\
And think that thou shalt learn far other lore,\\
And in far other scenes. For I was reared\\
In the great city, pent 'mid cloisters dim,\\
And saw nought lovely but the sky \& stars.\\
But th\'ou, my babe, shalt wander like a breeze\\
By lakes \& sandy shores, beneath the crags\\
Of ancient mountain, and beneath the clouds,\\
Which image in their bulk both lakes \& shores\\
And mountain crags: so shalt thou see \& hear\\
The lovely shapes \& sounds intelligible\\
Of that eternal language, which thy God\\
Utters, who from eternity doth teach\\
Himself in all, and all things in himself.\\
Great universal Teacher, he shall mould\\*
Thy spirit, and by giving make it ask.\\!

Therefore all seasons shall be sweet to thee,\\*
Whether the summer clothe the general earth\\
With greenness, or the redbreast sit \& sing\\
Betwixt the tufts of snow on the bare branch\\
Of mossy apple-tree, while the night-thatch\\
Smokes in the sun-thaw; whether the eave-drops fall\\
Heard only in the trances of the blast,\\
Or if the secret ministry of frost\\
Shall hang them up in silent icicles,\\*
Quietly shining to the quiet moon.
\end{verse}

\end{document}