\documentclass[MAIN]{subfiles}
\begin{document}

\bigskip

\begin{center}
{\it Tune: Wild Mountain Thyme}
\end{center}

\bigskip

\settowidth{\versewidth}{O the summer time has come,}
\begin{verse}[\versewidth]
\poemlines{5}
O the summer time has come,\footnotetext{$\dagger$ `Wild Mountain Thyme', Francis McPeake (1885 -- 1971). \cite{corries2}. McPeake seems to have been inspired to compose this song by a poem of Robert Tannahill's.
	
Some interpreters of this song (Kate Rusby et al.) have been known to render the first line as, `O the summer time is coming', which, as a certain learned gentleman pointed out to the Almanackist, shows their ignorance. Heather blooms in late summer, and not in the spring.}\\*
\vin And the trees are sweetly blooming,\\
And wild mountain thyme\\
\vin Grows around the purple heather.\\*
\vin \vin Will you go, lassie, go?\\!

{\it And we'll all go together\\*
\vin To pull wild mountain thyme\\
All around the purple heather.\\*
\vin Will you go, lassie, go?}\\!

I will build my love a bower\\*
\vin By yon clear crystal fountain,\\
And on it I will pile\\
\vin All the flowers of the mountain.\\*
\vin \vin Will you go, lassie, go?\\!

I will range through the wilds\\*
\vin And the deep land so dreary\\
And return with the spoils\\
\vin To the bower o' my dearie.\\*
\vin \vin Will ye go, lassie, go?\\!

If my truelove she'll not come,\\*
\vin Then I'll surely find another\\
To pull wild mountain thyme\\
\vin All around the purple heather.\\*
\vin \vin Will you go, lassie, go?\\!

{\it And we'll all go together\\*
\vin To pull wild mountain thyme\\
All around the purple heather.\\*
\vin Will you go, lassie, go?}
\end{verse}

\end{document}