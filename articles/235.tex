\documentclass[MAIN]{subfiles}
\begin{document}

\settowidth{\versewidth}{While my hair was still cut straight across my forehead}
\begin{verse}[\versewidth]
\poemlines{5}
While my hair was still cut straight across my forehead\footnotetext{`The River-Merchant's Wife: a Letter', Ezra Pound (1885 -- 1972). \cite{norton}. This poem is a translation of what is sometimes called `The Song of Chang'an', by the Chinese king of poets, Li Bai (also called Li Po, and known to Pound and various Japanese scholars as Rihaku). Many of the place-names, e.g. `{\sc Chokan}', `{\sc Ku-to-yen}', seem to be a brew of archaism, misunderstanding and poor transliteration.}\\*
I played about the front gate, pulling flowers.\\ 
You came by on bamboo stilts, playing horse;\\
You walked about my seat, playing with blue plums.\\
And we went on living in the village of {\sc Chokan}:\\
Two small people, without dislike or suspicion.\\
At 14 I married my lord, you.\\
I never laughed, being bashful.\\
Lowering my head, I looked at the wall.\\*
Called to, a 1000 times, I never looked back.\\!

At 15 I stopped scowling;\\*
I desired my dust to be mingled with yours\\
Forever \& forever \& forever.\\*
Why should I climb the look out?\\!

At 16 you departed\\*
You went into far {\sc Ku-to-yen}, by the river of swirling eddies,\\
And you have been gone five months.\\*
The monkeys make sorrowful noise overhead.\\!

You dragged your feet when you went out.\\*
By the gate now, the moss is grown, the different mosses,\\
Too deep to clear them away!\\
The leaves fall early this autumn, in wind.\\
The paired butterflies are already yellow with august\\
Over the grass in the west garden;\\
They hurt me.\\
I grow older.\\
If you are coming down through the narrows of the river {\sc Kiang},\\
Please let me know beforehand,\\
And I will come out to meet you\\*
\vin \vin As far as {\sc Cho-fu-Sa}.
\end{verse}

\end{document}