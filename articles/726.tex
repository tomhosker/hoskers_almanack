\documentclass[MAIN]{subfiles}
\begin{document}

\settowidth{\versewidth}{I come among the peoples like a shadow.}
\begin{verse}[\versewidth]
\poemlines{5}
I come among the peoples like a shadow.\footnotetext{`Hunger', Prof Laurence Binyon (1869 -- 1943). \cite{oxfordlarkin}. Prof Binyon was clearly inspired by the Old English riddles, such as are found in the \emph{Exeter Book}. The Almanackist has excised the last two lines, since these give the game away.}\\*
I sit down by each man's side.\\!

None sees me, but they look on one another,\\*
And know that I am there.\\!

My silence is like the silence of the tide\\*
That buries the playground of children;\\!

Like the deepening of frost in the slow night,\\*
When birds are dead in the morning.\\!

Armies trample, invade, destroy,\\*
With guns roaring from earth \& air.\\!

I am more terrible than armies;\\*
I am more feared than the cannon.\\!

Kings and chancellors give commands;\\*
I give no command to any;\\!

But I am listened to more than kings\\*
And more than passionate orators.\\!

I unswear words, and undo deeds.\\*
Naked things know me.
\end{verse}

\end{document}