\documentclass[MAIN]{subfiles}
\begin{document}

\settowidth{\versewidth}{Beautiful {\sc Railway Bridge} of the silvery {\sc Tay}!}
\begin{verse}[\versewidth]
\poemlines{5}
Beautiful {\sc Railway Bridge} of the silvery {\sc Tay}!\footnotetext{`The Tay Bridge Disaster', Sir William McGonagall, Knight of the White Elephant, Burmah (1825 -- 1902). \cite{mcgonagall}. William McGonagall (his knighthood would seem to have been self-bestowed; but where's the harm in that?) is often said to be the worst poet in the English language, and this his worst poem. Yet the Almanackist cannot help but discern a particular kind of genius in his works, rarely seen outside of the writings of Joseph Smith and L. Ron Hubbard. The disaster described was indeed a genuine tragedy, and remains the most lethal British railway disaster to this day.}\\*
Alas! I am very sorry to say\\
That 90 lives have been taken away\\
On the last sabbath day of eighteen seventy-nine,\\*
Which will be remembered for a very long time.\\!

'Twas about seven o'clock at night,\\*
And the wind it blew with all its might,\\
And the rain came pouring down,\\
And the dark clouds seemed to frown,\\
And the demon of the air seemed to say,\\*
`I'll blow down the {\sc Bridge of Tay}.'\\!

When the train left {\sc Edinburgh}\\*
The passengers' hearts were light \& felt no sorrow,\\
But \emph{Boreas} blew a terrific gale,\\
Which made their hearts for to quail,\\
And many of the passengers with fear did say,\\*
`I hope God will send us safe across the {\sc Bridge of Tay}.'\\!

But when the train came near to {\sc Wormit Bay},\\*
\emph{Boreas} he did loud \& angry bray,\\
And shook the central girders of the {\sc Bridge of Tay}\\
On the last sabbath day of eighteen seventy-nine,\\*
Which will be remembered for a very long time.\\!

So the train sped on with all its might,\\*
And bonny {\sc Dundee} soon hove in sight,\\
And the passengers' hearts felt light,\\
Thinking they would enjoy themselves on the New Year,\\
With their friends at home they loved most dear,\\*
And wish them all a happy New Year.\\!

So the train moved slowly along the {\sc Bridge of Tay},\\*
Until it was about midway,\\
Then the central girders with a crash gave way,\\
And down went the train \& passengers into the {\sc Tay}!\\
The storm fiend did loudly bray,\\
Because 90 lives had been taken away,\\
On the last sabbath day of eighteen seventy-nine,\\*
Which will be remembered for a very long time.\\!

As soon as the catastrophe came to be known\\*
The alarm from mouth to mouth was blown,\\
And the cry rang out all o'er the town:\\
Good Heavens! The {\sc Tay Bridge} is blown down,\\
And a passenger train from Edinburgh,\\
Which filled all the people's hearts with sorrow,\\
And made them for to turn pale,\\
Because none of the passengers were saved to tell the tale\\
How the disaster happened on the last sabbath day of eighteen seventy-nine,\\*
Which will be remembered for a very long time.\\!

It must have been an awful sight,\\*
To witness in the dusky moonlight,\\
While the storm fiend did laugh, and angry did bray,\\
Along the {\sc Railway Bridge} of the silvery {\sc Tay}.\\
O ill-fated {\sc Bridge} of the silvery {\sc Tay},\\
I must now conclude my lay\\
By telling the world fearlessly without the least dismay,\\
That your central girders would not have given way,\\
At least many sensible men do say,\\
Had they been supported on each side with buttresses,\\
At least many sensible men confesses,\\
For the stronger we our houses do build,\\*
The less chance we have of being killed.
\end{verse}

\end{document}