\documentclass[MAIN]{subfiles}
\begin{document}

\settowidth{\versewidth}{For sure our souls were near allied; and thine}
\begin{verse}[\versewidth]
\poemlines{5}
Farewell, too little \& too lately known,\footnotetext{`To the Memory of Mr Oldham', John Dryden, Poet Laureate (1631 -- 1700). \cite{norton}. \P Nisus is a character from the \emph{Aeneid}, who, having slipped and fallen during a race, and seeing that he can't recover his lead, tackles one of the other competitors to ensure his friend's victory. \P The name Marcellus refers to a number of figures from Roman history, although Dryden is probably referring here to Marcus Claudius Marcellus, the nephew and proposed heir of Augustus, whose death at nineteen years of age is a good example of a man who died before his youthful promise could be realised -- just like John Oldham, the subject of this elegy.}\\*
Whom I began to think \& call my own;\\
For sure our souls were near allied; and thine\\
Cast in the same poetic mould with mine.\\
One common note on either lyre did strike,\\
And knaves \& fools we both abhorred alike:\\
To the same goal did both our studies drive,\\
The last set out the soonest did arrive.\\
Thus \emph{Nisus} fell upon the slippery place,\\
While his young friend performed and won the race.\\
O early ripe! to thy abundant store\\
What could advancing age have added more?\\
It might (what nature never gives the young)\\
Have taught the numbers of thy native tongue.\\
But satire needs not those, and wit will shine\\
Through the harsh cadence of a rugged line.\\
A noble error, and but seldom made,\\
When poets are by too much force betrayed.\\
Thy generous fruits, though gathered ere their prime\\
Still showed a quickness; and maturing time\\
But mellows what we write to the dull sweets of rhyme.\\
Once more, hail \& farewell; farewell thou young,\\
But ah too short, \emph{Marcellus} of our tongue;\\
Thy brows with ivy, and with laurels bound;\\*
But fate \& gloomy night encompass thee around.
\end{verse}

\end{document}