\documentclass[MAIN]{subfiles}
\begin{document}

\settowidth{\versewidth}{She sighed not that th\'ey stayed, but that sh\'e went.}
\begin{verse}[\versewidth]
\poemlines{5}
As some fond virgin, whom her mother's care\footnotetext{`Epistle to Miss Blount, On Her Leaving the Town, After the Coronation', Alexander Pope (1688 -- 1744). The Miss Blount in question must have been one of the two Blount sisters, Teresa and Martha, with whom Pope was friendly -- more likely Martha Blount, since, as Robert Carruthers argued in his \emph{Life} of 1857, Pope and she were particularly close, perhaps even lovers, and indeed Pope made her his principal heir. \P The word `whisk' as used here is an archaic name for the card-game whist.}\\*
Drags from the town to wholesome country air,\\
Just when she learns to roll a melting eye,\\
And hear a spark, yet think no danger nigh;\\
From the dear man unwillingly she must sever,\\
Yet takes one kiss before she parts for ever:\\
Thus from the world fair \emph{Zephalinda} flew,\\
Saw others happy, and with sighs withdrew;\\
Not that their pleasures caused her discontent:\\
She sighed not that th\'ey stayed, but that sh\'e went.\\
She went, to plain-work, and to purling brooks,\\
Old-fashioned halls, dull aunts, \& croaking rooks;\\
She went from opera, park, assembly, play,\\
To morning walks, \& prayers three hours a day;\\
To pass her time 'twixt reading \& bohea,\\
To muse, and spill her solitary tea,\\
Or o'er cold coffee trifle with the spoon,\\
Count the slow clock, and dine exact at noon;\\
Divert her eyes with pictures in the fire,\\
Hum \sfrac{$1$}{$2$} a tune, tell stories to the squire;\\
Up to her godly garret after seven;\\
There starve \& pray, for that's the way to heaven.\\
Some squire, perhaps, you take a delight to rack;\\
Whose game is whisk, whose treat a toast in sack,\\
Who visits with a gun, presents you birds,\\
Then gives a smacking buss, \& cries, `No words!'\\
Or with his hound comes hollowing from the stable,\\
Makes love with nods, \& knees beneath a table;\\
Whose laughs are hearty, though his jests are coarse,\\
And loves you best of all things -- but his horse.\\
In some fair evening, on your elbow laid,\\
Your dream of triumphs in the rural shade;\\
In pensive thought recall the fancied scene,\\
See coronations rise on every green;\\
Before you pass th'imaginary sights\\
Of lords \& earls \& dukes \& gartered knights;\\
While the spread fan o'ershades your closing eyes;\\
Then give one flirt, and all the vision flies.\\
Thus vanish scepters, coronets \& balls,\\
And leave you in lone woods, or empty walls.\\
So when your slave, at some dear, idle time,\\
(Not plagued with headaches, or the want of rhyme)\\
Stands in the streets, abstracted from the crew,\\
And while he seems to study, thinks of you:\\
Just when his fancy points your sprightly eyes,\\
Or sees the blush of soft \emph{Parthenia} rise,\\
Gay pats my shoulder, and you vanish quite;\\
Streets, chairs, and coxcombs rush upon my sight;\\
Vexed to be still in town, I knit my brow,\\*
Look sour and hum a tune -- as you may now.
\end{verse}

\end{document}