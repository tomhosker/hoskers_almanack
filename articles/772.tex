\documentclass[MAIN]{subfiles}

\begin{document}

\begin{center}
\emph{Tune: Sweet William's Ghost}
\end{center}

\bigskip

\settowidth{\versewidth}{And, ay, he's turled long at the pin,}
\begin{verse}[\versewidth]
There came a ghost to \emph{Margaret}'s door\footnotetext{Anon. \cite{rusby_underneath}. This song is essentially the same as the seventy-seventh Child Ballad.}\\*
With many a grievious groan,\\
And, ay, he's turled long at the pin,\\
But answer she gave none.\\
`Is it my father \emph{Philip}?\\
Or yet my brother \emph{John}?\\
Or yet my own dear \emph{William}\\*
From Scotland now came home?'\\!

`My faith and troth you'll never get,\\*
Of me you'll never win,\\
Till you take me to yon churchyard\\
And wed me with a ring.'\\
`O I do dwell in a churchyard\\
But far beyond the sea,\\
And it is but my ghost, \emph{Margaret},\\*
That speaks now unto thee.'\\!

So she's put on her robes of green,\\*
With a piece below the knee,\\
And all the live long winter's night\\
The sweet ghost followed she.\\
`O is there room at your head, \emph{Willie},\\
Or room here at your feet,\\
Or room here at your side, \emph{Willie},\\*
Wherein that I may sleep?'\\!

`There's no room at my head, \emph{Margaret}.\\*
There's no room at my feet.\\
There's no room at my side, \emph{Margaret}.\\
My coffin is so neat.'\\
Then up and spoke the red robin,\\
And up and spoke the grey.\\
`Tis' time, tis' time, my dear \emph{Margaret}\\*
That I were gone away.'\\!

No more the ghost to \emph{Margaret} came\\*
With many a grievious groan.\\
He's vanished out into the mist\\
And left her there alone\\
`O stay my own true love, stay!\\
My heart you do divide.'\\
Pale grew her cheeks. She closed her eyes,\\*
Stretched out her limbs and cried.
\end{verse}
\end{document}
