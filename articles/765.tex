\documentclass[MAIN]{subfiles}
\begin{document}

\bigskip

\begin{center}
{\it Tune: We're All Jolly Fellows that Follow the Plough}
\end{center}

\bigskip

\settowidth{\versewidth}{`Come rise up, young fellows, with the best of good will.}
\begin{verse}[\versewidth]
\poemlines{5}
It was early one morning at the break of the day;\footnotetext{Anon. \cite{rusby_hourglass}. This song seems to have originated from a broadside printed around 1820.}\\*
The farmer came to us, and this he did say:\\
`Come rise up, young fellows, with the best of good will.\\*
Your horses need something, their bellies to fill.'\\!

When four o'clock comes, my boys, it's up we do rise,\\*
And off to the stable we merrily flies.\\
With a-rubbing and a-scrubbing, our horses will go,\\*
For we're all jolly fellows that follows the plough.\\!

When six o'clock comes, at breakfast we'll meet,\\*
And with cold beef and pork we'll heartily eat.\\
With a piece in our pocket, to the fields we do go,\\*
For we're all jolly fellows that follows the plough.\\!

Then up spoke the farmer, and this he did say:\\*
`What have you been doing this long summer's day?\\
You've not ploughed your acre. I'll swear and I'll vow:\\*
You are all lazy fellows that follows the plough.\\!

Then up spoke our carter, and this he did cry:\\*
`We've all ploughed our acre. You tell us a lie.\\
We've all ploughed our acre. I'll swear and I'll vow:\\*
We are all jolly fellows that follows the plough.\\!

Then up spoke the farmer, and laughed at the joke:\\*
`O it's gone \sfrac{$1$}{$2$} past two boys. It's time to unyoke.\\
Unharness your horses, and rub them down well,\\*
And I'll give you a jug of the very best ale.\\!

So all you young ploughboys, where'er you may be,\\*
Come take this advice and be ruled by me:\\
Never fear any master, where'er you may go,\\*
Fror we're all jolly fellows that follows the plough.
\end{verse}

\end{document}
