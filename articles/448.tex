\documentclass[MAIN]{subfiles}
\begin{document}

\settowidth{\versewidth}{The tangled bine-stems scored the sky}
\begin{verse}[\versewidth]
\poemlines{5}
I leant upon a coppice gate\footnotetext{`The Darkling Thrush', Thomas Hardy (1840 -- 1928). \cite{norton}. Hardy began writing this poem on the thirty-first day of December (of the New Style) of 1900.}\\*
\vin When frost was spectre-grey,\\
And winter's dregs made desolate\\
\vin The weakening eye of day.\\
The tangled bine-stems scored the sky\\
\vin Like strings of broken lyres,\\
And all mankind that haunted nigh\\*
\vin Had sought their household fires.\\!

The land's sharp features seemed to be\\*
\vin The century's corpse outleant,\\
His crypt the cloudy canopy,\\
\vin The wind his death-lament.\\
The ancient pulse of germ \& birth\\
\vin Was shrunken hard \& dry,\\
And every spirit upon earth\\*
\vin Seemed fervourless as I.\\!

At once a voice arose among\\*
\vin The bleak twigs overhead\\
In a full-hearted evensong\\
\vin Of joy illimited;\\
An ag\`ed thrush, frail, gaunt, \& small,\\
\vin In blast-beruffled plume,\\ 
Had chosen thus to fling his soul\\*
\vin Upon the growing gloom.\\!

So little cause for carolings\\*
\vin Of such ecstatic sound\\
Was written on terrestrial things\\
\vin Afar or nigh around,\\
That I could think there trembled through\\
\vin His happy good-night air\\
Some blessed hope, whereof he knew\\*
\vin And I was unaware.
\end{verse}

\end{document}